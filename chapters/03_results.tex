\chapter{Results} \label{sec:results}
\epigraphhead[30]{\epigraph{%
\textit{"Even if rounding error vanished, numerical analysis would remain. Approximating numbers, the task of floating-point arithmetic, is indeed a rather small topic and may even be a tedious one."}}%
{\textsc{Lloyd N. Trefethen (1955)}}%
}

\section{Planar maneuvers} \label{sec:resplanar}

For the planar maneuvers, we will skip the semimajor-axis only case as explained in the previous chapter since we were not able to find a validation example, and only briefly comment the numerical results of the other two maneuvers. More details will be given for the more complicated combined maneuvers.

\subsection{Eccentricity change} \label{sec:resecc}

For the two cases in consideration (the one described in the previous chapter and the one corresponding to a reverse change in eccentricity) we were able to recover the expected results from \cite{pollard1997simplified} with a relative error of $10^{-4}$. This can be regarded as a nice result, given that the actual values were not present in the original paper and had to be computed or interpreted from the plots.

The same relative and absolute errors of $10^{-4}$ are obtained for the expected eccentricity after integrating the equations of motion, either with a positive or a negative change in eccentricity. This validates our intuition of reversing the direction of the thrust depending on the sign of $\Delta e$.

\subsection{Argument of periapsis adjustment}

Yeah.

\clearpage

\section{Non planar maneuvers} \label{sec:resnonplanar}

\subsection{Combined semimajor axis and inclination change} \label{sec:resedelbaum}

For all the evaluated cases we were able to recover the original results of \cite{kechichian1997reformulation}, with a varying degree of accuracy. For both the expected time of flight $t_f$ and cost $\Delta V$, a relative error of $10^{-5}$ was achieved for the first case, $10^{-3}$ for the second and $10^{-2}$ for the singular case. With the figures given in the original paper, we are unable to assess if the loss of accuracy is due to problems in our algorithm or simply the fact that not all decimal places are included in the text. For the sake of completeness, our results to machine precision are summarized in table~\ref{tab:aincanares}.

Regarding the numerical validation, after integrating the equations of motion for a time $t = t_f$ for the two non singular cases we recover the expected values for final semimajor axis $a$ and inclination $i$ with a relative error of $10^{-5}$ for the former and an absolute error of $10^{-3}$ for the latter. Notice that the relative error makes no sense if we are comparing to zero. The eccentricity does not experiment significant growth and stays equal to zero with an absolute tolerance of $10^{-2}$.

For completeness, we also include the plots of the time history for the semimajor axis, inclination and velocity for the two non singular cases (figures \ref{fig:aincnumres28} and \ref{fig:aincnumres90}), recording the evolution during the integration of the equations of motion. We notice how the semimajor axis significantly increases for the second case, and how most of the inclination change takes place at that moment.

\begin{figure}%[h]
\begin{subfigure}[b]{0.5\textwidth}
\centering
\resizebox{1.0\textwidth}{!}{
%% Creator: Matplotlib, PGF backend
%%
%% To include the figure in your LaTeX document, write
%%   \input{<filename>.pgf}
%%
%% Make sure the required packages are loaded in your preamble
%%   \usepackage{pgf}
%%
%% Figures using additional raster images can only be included by \input if
%% they are in the same directory as the main LaTeX file. For loading figures
%% from other directories you can use the `import` package
%%   \usepackage{import}
%% and then include the figures with
%%   \import{<path to file>}{<filename>.pgf}
%%
%% Matplotlib used the following preamble
%%   \usepackage{fontspec}
%%
\begingroup%
\makeatletter%
\begin{pgfpicture}%
\pgfpathrectangle{\pgfpointorigin}{\pgfqpoint{6.400000in}{4.800000in}}%
\pgfusepath{use as bounding box, clip}%
\begin{pgfscope}%
\pgfsetbuttcap%
\pgfsetmiterjoin%
\definecolor{currentfill}{rgb}{1.000000,1.000000,1.000000}%
\pgfsetfillcolor{currentfill}%
\pgfsetlinewidth{0.000000pt}%
\definecolor{currentstroke}{rgb}{1.000000,1.000000,1.000000}%
\pgfsetstrokecolor{currentstroke}%
\pgfsetdash{}{0pt}%
\pgfpathmoveto{\pgfqpoint{0.000000in}{0.000000in}}%
\pgfpathlineto{\pgfqpoint{6.400000in}{0.000000in}}%
\pgfpathlineto{\pgfqpoint{6.400000in}{4.800000in}}%
\pgfpathlineto{\pgfqpoint{0.000000in}{4.800000in}}%
\pgfpathclose%
\pgfusepath{fill}%
\end{pgfscope}%
\begin{pgfscope}%
\pgfsetbuttcap%
\pgfsetmiterjoin%
\definecolor{currentfill}{rgb}{1.000000,1.000000,1.000000}%
\pgfsetfillcolor{currentfill}%
\pgfsetlinewidth{0.000000pt}%
\definecolor{currentstroke}{rgb}{0.000000,0.000000,0.000000}%
\pgfsetstrokecolor{currentstroke}%
\pgfsetstrokeopacity{0.000000}%
\pgfsetdash{}{0pt}%
\pgfpathmoveto{\pgfqpoint{0.800000in}{0.528000in}}%
\pgfpathlineto{\pgfqpoint{5.760000in}{0.528000in}}%
\pgfpathlineto{\pgfqpoint{5.760000in}{4.224000in}}%
\pgfpathlineto{\pgfqpoint{0.800000in}{4.224000in}}%
\pgfpathclose%
\pgfusepath{fill}%
\end{pgfscope}%
\begin{pgfscope}%
\pgfsetbuttcap%
\pgfsetroundjoin%
\definecolor{currentfill}{rgb}{0.000000,0.000000,0.000000}%
\pgfsetfillcolor{currentfill}%
\pgfsetlinewidth{0.803000pt}%
\definecolor{currentstroke}{rgb}{0.000000,0.000000,0.000000}%
\pgfsetstrokecolor{currentstroke}%
\pgfsetdash{}{0pt}%
\pgfsys@defobject{currentmarker}{\pgfqpoint{0.000000in}{-0.048611in}}{\pgfqpoint{0.000000in}{0.000000in}}{%
\pgfpathmoveto{\pgfqpoint{0.000000in}{0.000000in}}%
\pgfpathlineto{\pgfqpoint{0.000000in}{-0.048611in}}%
\pgfusepath{stroke,fill}%
}%
\begin{pgfscope}%
\pgfsys@transformshift{1.025455in}{0.528000in}%
\pgfsys@useobject{currentmarker}{}%
\end{pgfscope}%
\end{pgfscope}%
\begin{pgfscope}%
\pgftext[x=1.025455in,y=0.430778in,,top]{\sffamily\fontsize{10.000000}{12.000000}\selectfont \(\displaystyle 0\)}%
\end{pgfscope}%
\begin{pgfscope}%
\pgfsetbuttcap%
\pgfsetroundjoin%
\definecolor{currentfill}{rgb}{0.000000,0.000000,0.000000}%
\pgfsetfillcolor{currentfill}%
\pgfsetlinewidth{0.803000pt}%
\definecolor{currentstroke}{rgb}{0.000000,0.000000,0.000000}%
\pgfsetstrokecolor{currentstroke}%
\pgfsetdash{}{0pt}%
\pgfsys@defobject{currentmarker}{\pgfqpoint{0.000000in}{-0.048611in}}{\pgfqpoint{0.000000in}{0.000000in}}{%
\pgfpathmoveto{\pgfqpoint{0.000000in}{0.000000in}}%
\pgfpathlineto{\pgfqpoint{0.000000in}{-0.048611in}}%
\pgfusepath{stroke,fill}%
}%
\begin{pgfscope}%
\pgfsys@transformshift{1.614840in}{0.528000in}%
\pgfsys@useobject{currentmarker}{}%
\end{pgfscope}%
\end{pgfscope}%
\begin{pgfscope}%
\pgftext[x=1.614840in,y=0.430778in,,top]{\sffamily\fontsize{10.000000}{12.000000}\selectfont \(\displaystyle 25\)}%
\end{pgfscope}%
\begin{pgfscope}%
\pgfsetbuttcap%
\pgfsetroundjoin%
\definecolor{currentfill}{rgb}{0.000000,0.000000,0.000000}%
\pgfsetfillcolor{currentfill}%
\pgfsetlinewidth{0.803000pt}%
\definecolor{currentstroke}{rgb}{0.000000,0.000000,0.000000}%
\pgfsetstrokecolor{currentstroke}%
\pgfsetdash{}{0pt}%
\pgfsys@defobject{currentmarker}{\pgfqpoint{0.000000in}{-0.048611in}}{\pgfqpoint{0.000000in}{0.000000in}}{%
\pgfpathmoveto{\pgfqpoint{0.000000in}{0.000000in}}%
\pgfpathlineto{\pgfqpoint{0.000000in}{-0.048611in}}%
\pgfusepath{stroke,fill}%
}%
\begin{pgfscope}%
\pgfsys@transformshift{2.204226in}{0.528000in}%
\pgfsys@useobject{currentmarker}{}%
\end{pgfscope}%
\end{pgfscope}%
\begin{pgfscope}%
\pgftext[x=2.204226in,y=0.430778in,,top]{\sffamily\fontsize{10.000000}{12.000000}\selectfont \(\displaystyle 50\)}%
\end{pgfscope}%
\begin{pgfscope}%
\pgfsetbuttcap%
\pgfsetroundjoin%
\definecolor{currentfill}{rgb}{0.000000,0.000000,0.000000}%
\pgfsetfillcolor{currentfill}%
\pgfsetlinewidth{0.803000pt}%
\definecolor{currentstroke}{rgb}{0.000000,0.000000,0.000000}%
\pgfsetstrokecolor{currentstroke}%
\pgfsetdash{}{0pt}%
\pgfsys@defobject{currentmarker}{\pgfqpoint{0.000000in}{-0.048611in}}{\pgfqpoint{0.000000in}{0.000000in}}{%
\pgfpathmoveto{\pgfqpoint{0.000000in}{0.000000in}}%
\pgfpathlineto{\pgfqpoint{0.000000in}{-0.048611in}}%
\pgfusepath{stroke,fill}%
}%
\begin{pgfscope}%
\pgfsys@transformshift{2.793612in}{0.528000in}%
\pgfsys@useobject{currentmarker}{}%
\end{pgfscope}%
\end{pgfscope}%
\begin{pgfscope}%
\pgftext[x=2.793612in,y=0.430778in,,top]{\sffamily\fontsize{10.000000}{12.000000}\selectfont \(\displaystyle 75\)}%
\end{pgfscope}%
\begin{pgfscope}%
\pgfsetbuttcap%
\pgfsetroundjoin%
\definecolor{currentfill}{rgb}{0.000000,0.000000,0.000000}%
\pgfsetfillcolor{currentfill}%
\pgfsetlinewidth{0.803000pt}%
\definecolor{currentstroke}{rgb}{0.000000,0.000000,0.000000}%
\pgfsetstrokecolor{currentstroke}%
\pgfsetdash{}{0pt}%
\pgfsys@defobject{currentmarker}{\pgfqpoint{0.000000in}{-0.048611in}}{\pgfqpoint{0.000000in}{0.000000in}}{%
\pgfpathmoveto{\pgfqpoint{0.000000in}{0.000000in}}%
\pgfpathlineto{\pgfqpoint{0.000000in}{-0.048611in}}%
\pgfusepath{stroke,fill}%
}%
\begin{pgfscope}%
\pgfsys@transformshift{3.382998in}{0.528000in}%
\pgfsys@useobject{currentmarker}{}%
\end{pgfscope}%
\end{pgfscope}%
\begin{pgfscope}%
\pgftext[x=3.382998in,y=0.430778in,,top]{\sffamily\fontsize{10.000000}{12.000000}\selectfont \(\displaystyle 100\)}%
\end{pgfscope}%
\begin{pgfscope}%
\pgfsetbuttcap%
\pgfsetroundjoin%
\definecolor{currentfill}{rgb}{0.000000,0.000000,0.000000}%
\pgfsetfillcolor{currentfill}%
\pgfsetlinewidth{0.803000pt}%
\definecolor{currentstroke}{rgb}{0.000000,0.000000,0.000000}%
\pgfsetstrokecolor{currentstroke}%
\pgfsetdash{}{0pt}%
\pgfsys@defobject{currentmarker}{\pgfqpoint{0.000000in}{-0.048611in}}{\pgfqpoint{0.000000in}{0.000000in}}{%
\pgfpathmoveto{\pgfqpoint{0.000000in}{0.000000in}}%
\pgfpathlineto{\pgfqpoint{0.000000in}{-0.048611in}}%
\pgfusepath{stroke,fill}%
}%
\begin{pgfscope}%
\pgfsys@transformshift{3.972384in}{0.528000in}%
\pgfsys@useobject{currentmarker}{}%
\end{pgfscope}%
\end{pgfscope}%
\begin{pgfscope}%
\pgftext[x=3.972384in,y=0.430778in,,top]{\sffamily\fontsize{10.000000}{12.000000}\selectfont \(\displaystyle 125\)}%
\end{pgfscope}%
\begin{pgfscope}%
\pgfsetbuttcap%
\pgfsetroundjoin%
\definecolor{currentfill}{rgb}{0.000000,0.000000,0.000000}%
\pgfsetfillcolor{currentfill}%
\pgfsetlinewidth{0.803000pt}%
\definecolor{currentstroke}{rgb}{0.000000,0.000000,0.000000}%
\pgfsetstrokecolor{currentstroke}%
\pgfsetdash{}{0pt}%
\pgfsys@defobject{currentmarker}{\pgfqpoint{0.000000in}{-0.048611in}}{\pgfqpoint{0.000000in}{0.000000in}}{%
\pgfpathmoveto{\pgfqpoint{0.000000in}{0.000000in}}%
\pgfpathlineto{\pgfqpoint{0.000000in}{-0.048611in}}%
\pgfusepath{stroke,fill}%
}%
\begin{pgfscope}%
\pgfsys@transformshift{4.561769in}{0.528000in}%
\pgfsys@useobject{currentmarker}{}%
\end{pgfscope}%
\end{pgfscope}%
\begin{pgfscope}%
\pgftext[x=4.561769in,y=0.430778in,,top]{\sffamily\fontsize{10.000000}{12.000000}\selectfont \(\displaystyle 150\)}%
\end{pgfscope}%
\begin{pgfscope}%
\pgfsetbuttcap%
\pgfsetroundjoin%
\definecolor{currentfill}{rgb}{0.000000,0.000000,0.000000}%
\pgfsetfillcolor{currentfill}%
\pgfsetlinewidth{0.803000pt}%
\definecolor{currentstroke}{rgb}{0.000000,0.000000,0.000000}%
\pgfsetstrokecolor{currentstroke}%
\pgfsetdash{}{0pt}%
\pgfsys@defobject{currentmarker}{\pgfqpoint{0.000000in}{-0.048611in}}{\pgfqpoint{0.000000in}{0.000000in}}{%
\pgfpathmoveto{\pgfqpoint{0.000000in}{0.000000in}}%
\pgfpathlineto{\pgfqpoint{0.000000in}{-0.048611in}}%
\pgfusepath{stroke,fill}%
}%
\begin{pgfscope}%
\pgfsys@transformshift{5.151155in}{0.528000in}%
\pgfsys@useobject{currentmarker}{}%
\end{pgfscope}%
\end{pgfscope}%
\begin{pgfscope}%
\pgftext[x=5.151155in,y=0.430778in,,top]{\sffamily\fontsize{10.000000}{12.000000}\selectfont \(\displaystyle 175\)}%
\end{pgfscope}%
\begin{pgfscope}%
\pgfsetbuttcap%
\pgfsetroundjoin%
\definecolor{currentfill}{rgb}{0.000000,0.000000,0.000000}%
\pgfsetfillcolor{currentfill}%
\pgfsetlinewidth{0.803000pt}%
\definecolor{currentstroke}{rgb}{0.000000,0.000000,0.000000}%
\pgfsetstrokecolor{currentstroke}%
\pgfsetdash{}{0pt}%
\pgfsys@defobject{currentmarker}{\pgfqpoint{0.000000in}{-0.048611in}}{\pgfqpoint{0.000000in}{0.000000in}}{%
\pgfpathmoveto{\pgfqpoint{0.000000in}{0.000000in}}%
\pgfpathlineto{\pgfqpoint{0.000000in}{-0.048611in}}%
\pgfusepath{stroke,fill}%
}%
\begin{pgfscope}%
\pgfsys@transformshift{5.740541in}{0.528000in}%
\pgfsys@useobject{currentmarker}{}%
\end{pgfscope}%
\end{pgfscope}%
\begin{pgfscope}%
\pgftext[x=5.740541in,y=0.430778in,,top]{\sffamily\fontsize{10.000000}{12.000000}\selectfont \(\displaystyle 200\)}%
\end{pgfscope}%
\begin{pgfscope}%
\pgftext[x=3.280000in,y=0.251889in,,top]{\sffamily\fontsize{10.000000}{12.000000}\selectfont Time, days}%
\end{pgfscope}%
\begin{pgfscope}%
\pgfsetbuttcap%
\pgfsetroundjoin%
\definecolor{currentfill}{rgb}{0.000000,0.000000,0.000000}%
\pgfsetfillcolor{currentfill}%
\pgfsetlinewidth{0.803000pt}%
\definecolor{currentstroke}{rgb}{0.000000,0.000000,0.000000}%
\pgfsetstrokecolor{currentstroke}%
\pgfsetdash{}{0pt}%
\pgfsys@defobject{currentmarker}{\pgfqpoint{-0.048611in}{0.000000in}}{\pgfqpoint{0.000000in}{0.000000in}}{%
\pgfpathmoveto{\pgfqpoint{0.000000in}{0.000000in}}%
\pgfpathlineto{\pgfqpoint{-0.048611in}{0.000000in}}%
\pgfusepath{stroke,fill}%
}%
\begin{pgfscope}%
\pgfsys@transformshift{0.800000in}{0.982640in}%
\pgfsys@useobject{currentmarker}{}%
\end{pgfscope}%
\end{pgfscope}%
\begin{pgfscope}%
\pgftext[x=0.563888in,y=0.934446in,left,base]{\sffamily\fontsize{10.000000}{12.000000}\selectfont \(\displaystyle 10\)}%
\end{pgfscope}%
\begin{pgfscope}%
\pgfsetbuttcap%
\pgfsetroundjoin%
\definecolor{currentfill}{rgb}{0.000000,0.000000,0.000000}%
\pgfsetfillcolor{currentfill}%
\pgfsetlinewidth{0.803000pt}%
\definecolor{currentstroke}{rgb}{0.000000,0.000000,0.000000}%
\pgfsetstrokecolor{currentstroke}%
\pgfsetdash{}{0pt}%
\pgfsys@defobject{currentmarker}{\pgfqpoint{-0.048611in}{0.000000in}}{\pgfqpoint{0.000000in}{0.000000in}}{%
\pgfpathmoveto{\pgfqpoint{0.000000in}{0.000000in}}%
\pgfpathlineto{\pgfqpoint{-0.048611in}{0.000000in}}%
\pgfusepath{stroke,fill}%
}%
\begin{pgfscope}%
\pgfsys@transformshift{0.800000in}{1.460374in}%
\pgfsys@useobject{currentmarker}{}%
\end{pgfscope}%
\end{pgfscope}%
\begin{pgfscope}%
\pgftext[x=0.563888in,y=1.412179in,left,base]{\sffamily\fontsize{10.000000}{12.000000}\selectfont \(\displaystyle 15\)}%
\end{pgfscope}%
\begin{pgfscope}%
\pgfsetbuttcap%
\pgfsetroundjoin%
\definecolor{currentfill}{rgb}{0.000000,0.000000,0.000000}%
\pgfsetfillcolor{currentfill}%
\pgfsetlinewidth{0.803000pt}%
\definecolor{currentstroke}{rgb}{0.000000,0.000000,0.000000}%
\pgfsetstrokecolor{currentstroke}%
\pgfsetdash{}{0pt}%
\pgfsys@defobject{currentmarker}{\pgfqpoint{-0.048611in}{0.000000in}}{\pgfqpoint{0.000000in}{0.000000in}}{%
\pgfpathmoveto{\pgfqpoint{0.000000in}{0.000000in}}%
\pgfpathlineto{\pgfqpoint{-0.048611in}{0.000000in}}%
\pgfusepath{stroke,fill}%
}%
\begin{pgfscope}%
\pgfsys@transformshift{0.800000in}{1.938107in}%
\pgfsys@useobject{currentmarker}{}%
\end{pgfscope}%
\end{pgfscope}%
\begin{pgfscope}%
\pgftext[x=0.563888in,y=1.889913in,left,base]{\sffamily\fontsize{10.000000}{12.000000}\selectfont \(\displaystyle 20\)}%
\end{pgfscope}%
\begin{pgfscope}%
\pgfsetbuttcap%
\pgfsetroundjoin%
\definecolor{currentfill}{rgb}{0.000000,0.000000,0.000000}%
\pgfsetfillcolor{currentfill}%
\pgfsetlinewidth{0.803000pt}%
\definecolor{currentstroke}{rgb}{0.000000,0.000000,0.000000}%
\pgfsetstrokecolor{currentstroke}%
\pgfsetdash{}{0pt}%
\pgfsys@defobject{currentmarker}{\pgfqpoint{-0.048611in}{0.000000in}}{\pgfqpoint{0.000000in}{0.000000in}}{%
\pgfpathmoveto{\pgfqpoint{0.000000in}{0.000000in}}%
\pgfpathlineto{\pgfqpoint{-0.048611in}{0.000000in}}%
\pgfusepath{stroke,fill}%
}%
\begin{pgfscope}%
\pgfsys@transformshift{0.800000in}{2.415841in}%
\pgfsys@useobject{currentmarker}{}%
\end{pgfscope}%
\end{pgfscope}%
\begin{pgfscope}%
\pgftext[x=0.563888in,y=2.367647in,left,base]{\sffamily\fontsize{10.000000}{12.000000}\selectfont \(\displaystyle 25\)}%
\end{pgfscope}%
\begin{pgfscope}%
\pgfsetbuttcap%
\pgfsetroundjoin%
\definecolor{currentfill}{rgb}{0.000000,0.000000,0.000000}%
\pgfsetfillcolor{currentfill}%
\pgfsetlinewidth{0.803000pt}%
\definecolor{currentstroke}{rgb}{0.000000,0.000000,0.000000}%
\pgfsetstrokecolor{currentstroke}%
\pgfsetdash{}{0pt}%
\pgfsys@defobject{currentmarker}{\pgfqpoint{-0.048611in}{0.000000in}}{\pgfqpoint{0.000000in}{0.000000in}}{%
\pgfpathmoveto{\pgfqpoint{0.000000in}{0.000000in}}%
\pgfpathlineto{\pgfqpoint{-0.048611in}{0.000000in}}%
\pgfusepath{stroke,fill}%
}%
\begin{pgfscope}%
\pgfsys@transformshift{0.800000in}{2.893575in}%
\pgfsys@useobject{currentmarker}{}%
\end{pgfscope}%
\end{pgfscope}%
\begin{pgfscope}%
\pgftext[x=0.563888in,y=2.845380in,left,base]{\sffamily\fontsize{10.000000}{12.000000}\selectfont \(\displaystyle 30\)}%
\end{pgfscope}%
\begin{pgfscope}%
\pgfsetbuttcap%
\pgfsetroundjoin%
\definecolor{currentfill}{rgb}{0.000000,0.000000,0.000000}%
\pgfsetfillcolor{currentfill}%
\pgfsetlinewidth{0.803000pt}%
\definecolor{currentstroke}{rgb}{0.000000,0.000000,0.000000}%
\pgfsetstrokecolor{currentstroke}%
\pgfsetdash{}{0pt}%
\pgfsys@defobject{currentmarker}{\pgfqpoint{-0.048611in}{0.000000in}}{\pgfqpoint{0.000000in}{0.000000in}}{%
\pgfpathmoveto{\pgfqpoint{0.000000in}{0.000000in}}%
\pgfpathlineto{\pgfqpoint{-0.048611in}{0.000000in}}%
\pgfusepath{stroke,fill}%
}%
\begin{pgfscope}%
\pgfsys@transformshift{0.800000in}{3.371308in}%
\pgfsys@useobject{currentmarker}{}%
\end{pgfscope}%
\end{pgfscope}%
\begin{pgfscope}%
\pgftext[x=0.563888in,y=3.323114in,left,base]{\sffamily\fontsize{10.000000}{12.000000}\selectfont \(\displaystyle 35\)}%
\end{pgfscope}%
\begin{pgfscope}%
\pgfsetbuttcap%
\pgfsetroundjoin%
\definecolor{currentfill}{rgb}{0.000000,0.000000,0.000000}%
\pgfsetfillcolor{currentfill}%
\pgfsetlinewidth{0.803000pt}%
\definecolor{currentstroke}{rgb}{0.000000,0.000000,0.000000}%
\pgfsetstrokecolor{currentstroke}%
\pgfsetdash{}{0pt}%
\pgfsys@defobject{currentmarker}{\pgfqpoint{-0.048611in}{0.000000in}}{\pgfqpoint{0.000000in}{0.000000in}}{%
\pgfpathmoveto{\pgfqpoint{0.000000in}{0.000000in}}%
\pgfpathlineto{\pgfqpoint{-0.048611in}{0.000000in}}%
\pgfusepath{stroke,fill}%
}%
\begin{pgfscope}%
\pgfsys@transformshift{0.800000in}{3.849042in}%
\pgfsys@useobject{currentmarker}{}%
\end{pgfscope}%
\end{pgfscope}%
\begin{pgfscope}%
\pgftext[x=0.563888in,y=3.800847in,left,base]{\sffamily\fontsize{10.000000}{12.000000}\selectfont \(\displaystyle 40\)}%
\end{pgfscope}%
\begin{pgfscope}%
\pgftext[x=0.508333in,y=2.376000in,,bottom,rotate=90.000000]{\sffamily\fontsize{10.000000}{12.000000}\selectfont Semimajor axis, km (thousands)}%
\end{pgfscope}%
\begin{pgfscope}%
\pgfpathrectangle{\pgfqpoint{0.800000in}{0.528000in}}{\pgfqpoint{4.960000in}{3.696000in}} %
\pgfusepath{clip}%
\pgfsetbuttcap%
\pgfsetroundjoin%
\pgfsetlinewidth{1.505625pt}%
\definecolor{currentstroke}{rgb}{0.000000,0.000000,0.000000}%
\pgfsetstrokecolor{currentstroke}%
\pgfsetdash{{5.600000pt}{2.400000pt}}{0.000000pt}%
\pgfpathmoveto{\pgfqpoint{1.025455in}{0.696000in}}%
\pgfpathlineto{\pgfqpoint{1.178409in}{0.729384in}}%
\pgfpathlineto{\pgfqpoint{1.325058in}{0.763648in}}%
\pgfpathlineto{\pgfqpoint{1.466196in}{0.798891in}}%
\pgfpathlineto{\pgfqpoint{1.601945in}{0.835063in}}%
\pgfpathlineto{\pgfqpoint{1.733043in}{0.872287in}}%
\pgfpathlineto{\pgfqpoint{1.859575in}{0.910518in}}%
\pgfpathlineto{\pgfqpoint{1.981652in}{0.949713in}}%
\pgfpathlineto{\pgfqpoint{2.099843in}{0.989985in}}%
\pgfpathlineto{\pgfqpoint{2.214281in}{1.031316in}}%
\pgfpathlineto{\pgfqpoint{2.325449in}{1.073823in}}%
\pgfpathlineto{\pgfqpoint{2.433434in}{1.117489in}}%
\pgfpathlineto{\pgfqpoint{2.538147in}{1.162218in}}%
\pgfpathlineto{\pgfqpoint{2.640273in}{1.208250in}}%
\pgfpathlineto{\pgfqpoint{2.739691in}{1.255488in}}%
\pgfpathlineto{\pgfqpoint{2.836865in}{1.304112in}}%
\pgfpathlineto{\pgfqpoint{2.931847in}{1.354120in}}%
\pgfpathlineto{\pgfqpoint{3.024917in}{1.405635in}}%
\pgfpathlineto{\pgfqpoint{3.115849in}{1.458498in}}%
\pgfpathlineto{\pgfqpoint{3.205091in}{1.512941in}}%
\pgfpathlineto{\pgfqpoint{3.292822in}{1.569060in}}%
\pgfpathlineto{\pgfqpoint{3.379287in}{1.627010in}}%
\pgfpathlineto{\pgfqpoint{3.464554in}{1.686842in}}%
\pgfpathlineto{\pgfqpoint{3.548556in}{1.748509in}}%
\pgfpathlineto{\pgfqpoint{3.631487in}{1.812148in}}%
\pgfpathlineto{\pgfqpoint{3.713588in}{1.877955in}}%
\pgfpathlineto{\pgfqpoint{3.794941in}{1.946012in}}%
\pgfpathlineto{\pgfqpoint{3.875984in}{2.016723in}}%
\pgfpathlineto{\pgfqpoint{3.956424in}{2.089866in}}%
\pgfpathlineto{\pgfqpoint{4.036770in}{2.165942in}}%
\pgfpathlineto{\pgfqpoint{4.117123in}{2.245103in}}%
\pgfpathlineto{\pgfqpoint{4.197577in}{2.327500in}}%
\pgfpathlineto{\pgfqpoint{4.278239in}{2.413292in}}%
\pgfpathlineto{\pgfqpoint{4.360042in}{2.503565in}}%
\pgfpathlineto{\pgfqpoint{4.442622in}{2.598017in}}%
\pgfpathlineto{\pgfqpoint{4.526936in}{2.697843in}}%
\pgfpathlineto{\pgfqpoint{4.613432in}{2.803715in}}%
\pgfpathlineto{\pgfqpoint{4.703471in}{2.917476in}}%
\pgfpathlineto{\pgfqpoint{4.798260in}{3.040875in}}%
\pgfpathlineto{\pgfqpoint{4.900712in}{3.177981in}}%
\pgfpathlineto{\pgfqpoint{5.016473in}{3.336728in}}%
\pgfpathlineto{\pgfqpoint{5.168766in}{3.549647in}}%
\pgfpathlineto{\pgfqpoint{5.417822in}{3.898023in}}%
\pgfpathlineto{\pgfqpoint{5.518213in}{4.034261in}}%
\pgfpathlineto{\pgfqpoint{5.534545in}{4.056000in}}%
\pgfpathlineto{\pgfqpoint{5.534545in}{4.056000in}}%
\pgfusepath{stroke}%
\end{pgfscope}%
\begin{pgfscope}%
\pgfsetrectcap%
\pgfsetmiterjoin%
\pgfsetlinewidth{0.803000pt}%
\definecolor{currentstroke}{rgb}{0.000000,0.000000,0.000000}%
\pgfsetstrokecolor{currentstroke}%
\pgfsetdash{}{0pt}%
\pgfpathmoveto{\pgfqpoint{0.800000in}{0.528000in}}%
\pgfpathlineto{\pgfqpoint{0.800000in}{4.224000in}}%
\pgfusepath{stroke}%
\end{pgfscope}%
\begin{pgfscope}%
\pgfsetrectcap%
\pgfsetmiterjoin%
\pgfsetlinewidth{0.803000pt}%
\definecolor{currentstroke}{rgb}{0.000000,0.000000,0.000000}%
\pgfsetstrokecolor{currentstroke}%
\pgfsetdash{}{0pt}%
\pgfpathmoveto{\pgfqpoint{5.760000in}{0.528000in}}%
\pgfpathlineto{\pgfqpoint{5.760000in}{4.224000in}}%
\pgfusepath{stroke}%
\end{pgfscope}%
\begin{pgfscope}%
\pgfsetrectcap%
\pgfsetmiterjoin%
\pgfsetlinewidth{0.803000pt}%
\definecolor{currentstroke}{rgb}{0.000000,0.000000,0.000000}%
\pgfsetstrokecolor{currentstroke}%
\pgfsetdash{}{0pt}%
\pgfpathmoveto{\pgfqpoint{0.800000in}{0.528000in}}%
\pgfpathlineto{\pgfqpoint{5.760000in}{0.528000in}}%
\pgfusepath{stroke}%
\end{pgfscope}%
\begin{pgfscope}%
\pgfsetrectcap%
\pgfsetmiterjoin%
\pgfsetlinewidth{0.803000pt}%
\definecolor{currentstroke}{rgb}{0.000000,0.000000,0.000000}%
\pgfsetstrokecolor{currentstroke}%
\pgfsetdash{}{0pt}%
\pgfpathmoveto{\pgfqpoint{0.800000in}{4.224000in}}%
\pgfpathlineto{\pgfqpoint{5.760000in}{4.224000in}}%
\pgfusepath{stroke}%
\end{pgfscope}%
\end{pgfpicture}%
\makeatother%
\endgroup%

}
\end{subfigure}
\begin{subfigure}[b]{0.5\textwidth}
\centering
\resizebox{1.0\textwidth}{!}{
%% Creator: Matplotlib, PGF backend
%%
%% To include the figure in your LaTeX document, write
%%   \input{<filename>.pgf}
%%
%% Make sure the required packages are loaded in your preamble
%%   \usepackage{pgf}
%%
%% Figures using additional raster images can only be included by \input if
%% they are in the same directory as the main LaTeX file. For loading figures
%% from other directories you can use the `import` package
%%   \usepackage{import}
%% and then include the figures with
%%   \import{<path to file>}{<filename>.pgf}
%%
%% Matplotlib used the following preamble
%%   \usepackage{fontspec}
%%
\begingroup%
\makeatletter%
\begin{pgfpicture}%
\pgfpathrectangle{\pgfpointorigin}{\pgfqpoint{6.400000in}{4.800000in}}%
\pgfusepath{use as bounding box, clip}%
\begin{pgfscope}%
\pgfsetbuttcap%
\pgfsetmiterjoin%
\definecolor{currentfill}{rgb}{1.000000,1.000000,1.000000}%
\pgfsetfillcolor{currentfill}%
\pgfsetlinewidth{0.000000pt}%
\definecolor{currentstroke}{rgb}{1.000000,1.000000,1.000000}%
\pgfsetstrokecolor{currentstroke}%
\pgfsetdash{}{0pt}%
\pgfpathmoveto{\pgfqpoint{0.000000in}{0.000000in}}%
\pgfpathlineto{\pgfqpoint{6.400000in}{0.000000in}}%
\pgfpathlineto{\pgfqpoint{6.400000in}{4.800000in}}%
\pgfpathlineto{\pgfqpoint{0.000000in}{4.800000in}}%
\pgfpathclose%
\pgfusepath{fill}%
\end{pgfscope}%
\begin{pgfscope}%
\pgfsetbuttcap%
\pgfsetmiterjoin%
\definecolor{currentfill}{rgb}{1.000000,1.000000,1.000000}%
\pgfsetfillcolor{currentfill}%
\pgfsetlinewidth{0.000000pt}%
\definecolor{currentstroke}{rgb}{0.000000,0.000000,0.000000}%
\pgfsetstrokecolor{currentstroke}%
\pgfsetstrokeopacity{0.000000}%
\pgfsetdash{}{0pt}%
\pgfpathmoveto{\pgfqpoint{0.800000in}{0.528000in}}%
\pgfpathlineto{\pgfqpoint{5.760000in}{0.528000in}}%
\pgfpathlineto{\pgfqpoint{5.760000in}{4.224000in}}%
\pgfpathlineto{\pgfqpoint{0.800000in}{4.224000in}}%
\pgfpathclose%
\pgfusepath{fill}%
\end{pgfscope}%
\begin{pgfscope}%
\pgfsetbuttcap%
\pgfsetroundjoin%
\definecolor{currentfill}{rgb}{0.000000,0.000000,0.000000}%
\pgfsetfillcolor{currentfill}%
\pgfsetlinewidth{0.803000pt}%
\definecolor{currentstroke}{rgb}{0.000000,0.000000,0.000000}%
\pgfsetstrokecolor{currentstroke}%
\pgfsetdash{}{0pt}%
\pgfsys@defobject{currentmarker}{\pgfqpoint{0.000000in}{-0.048611in}}{\pgfqpoint{0.000000in}{0.000000in}}{%
\pgfpathmoveto{\pgfqpoint{0.000000in}{0.000000in}}%
\pgfpathlineto{\pgfqpoint{0.000000in}{-0.048611in}}%
\pgfusepath{stroke,fill}%
}%
\begin{pgfscope}%
\pgfsys@transformshift{1.025455in}{0.528000in}%
\pgfsys@useobject{currentmarker}{}%
\end{pgfscope}%
\end{pgfscope}%
\begin{pgfscope}%
\pgftext[x=1.025455in,y=0.430778in,,top]{\sffamily\fontsize{10.000000}{12.000000}\selectfont \(\displaystyle 0\)}%
\end{pgfscope}%
\begin{pgfscope}%
\pgfsetbuttcap%
\pgfsetroundjoin%
\definecolor{currentfill}{rgb}{0.000000,0.000000,0.000000}%
\pgfsetfillcolor{currentfill}%
\pgfsetlinewidth{0.803000pt}%
\definecolor{currentstroke}{rgb}{0.000000,0.000000,0.000000}%
\pgfsetstrokecolor{currentstroke}%
\pgfsetdash{}{0pt}%
\pgfsys@defobject{currentmarker}{\pgfqpoint{0.000000in}{-0.048611in}}{\pgfqpoint{0.000000in}{0.000000in}}{%
\pgfpathmoveto{\pgfqpoint{0.000000in}{0.000000in}}%
\pgfpathlineto{\pgfqpoint{0.000000in}{-0.048611in}}%
\pgfusepath{stroke,fill}%
}%
\begin{pgfscope}%
\pgfsys@transformshift{1.614840in}{0.528000in}%
\pgfsys@useobject{currentmarker}{}%
\end{pgfscope}%
\end{pgfscope}%
\begin{pgfscope}%
\pgftext[x=1.614840in,y=0.430778in,,top]{\sffamily\fontsize{10.000000}{12.000000}\selectfont \(\displaystyle 25\)}%
\end{pgfscope}%
\begin{pgfscope}%
\pgfsetbuttcap%
\pgfsetroundjoin%
\definecolor{currentfill}{rgb}{0.000000,0.000000,0.000000}%
\pgfsetfillcolor{currentfill}%
\pgfsetlinewidth{0.803000pt}%
\definecolor{currentstroke}{rgb}{0.000000,0.000000,0.000000}%
\pgfsetstrokecolor{currentstroke}%
\pgfsetdash{}{0pt}%
\pgfsys@defobject{currentmarker}{\pgfqpoint{0.000000in}{-0.048611in}}{\pgfqpoint{0.000000in}{0.000000in}}{%
\pgfpathmoveto{\pgfqpoint{0.000000in}{0.000000in}}%
\pgfpathlineto{\pgfqpoint{0.000000in}{-0.048611in}}%
\pgfusepath{stroke,fill}%
}%
\begin{pgfscope}%
\pgfsys@transformshift{2.204226in}{0.528000in}%
\pgfsys@useobject{currentmarker}{}%
\end{pgfscope}%
\end{pgfscope}%
\begin{pgfscope}%
\pgftext[x=2.204226in,y=0.430778in,,top]{\sffamily\fontsize{10.000000}{12.000000}\selectfont \(\displaystyle 50\)}%
\end{pgfscope}%
\begin{pgfscope}%
\pgfsetbuttcap%
\pgfsetroundjoin%
\definecolor{currentfill}{rgb}{0.000000,0.000000,0.000000}%
\pgfsetfillcolor{currentfill}%
\pgfsetlinewidth{0.803000pt}%
\definecolor{currentstroke}{rgb}{0.000000,0.000000,0.000000}%
\pgfsetstrokecolor{currentstroke}%
\pgfsetdash{}{0pt}%
\pgfsys@defobject{currentmarker}{\pgfqpoint{0.000000in}{-0.048611in}}{\pgfqpoint{0.000000in}{0.000000in}}{%
\pgfpathmoveto{\pgfqpoint{0.000000in}{0.000000in}}%
\pgfpathlineto{\pgfqpoint{0.000000in}{-0.048611in}}%
\pgfusepath{stroke,fill}%
}%
\begin{pgfscope}%
\pgfsys@transformshift{2.793612in}{0.528000in}%
\pgfsys@useobject{currentmarker}{}%
\end{pgfscope}%
\end{pgfscope}%
\begin{pgfscope}%
\pgftext[x=2.793612in,y=0.430778in,,top]{\sffamily\fontsize{10.000000}{12.000000}\selectfont \(\displaystyle 75\)}%
\end{pgfscope}%
\begin{pgfscope}%
\pgfsetbuttcap%
\pgfsetroundjoin%
\definecolor{currentfill}{rgb}{0.000000,0.000000,0.000000}%
\pgfsetfillcolor{currentfill}%
\pgfsetlinewidth{0.803000pt}%
\definecolor{currentstroke}{rgb}{0.000000,0.000000,0.000000}%
\pgfsetstrokecolor{currentstroke}%
\pgfsetdash{}{0pt}%
\pgfsys@defobject{currentmarker}{\pgfqpoint{0.000000in}{-0.048611in}}{\pgfqpoint{0.000000in}{0.000000in}}{%
\pgfpathmoveto{\pgfqpoint{0.000000in}{0.000000in}}%
\pgfpathlineto{\pgfqpoint{0.000000in}{-0.048611in}}%
\pgfusepath{stroke,fill}%
}%
\begin{pgfscope}%
\pgfsys@transformshift{3.382998in}{0.528000in}%
\pgfsys@useobject{currentmarker}{}%
\end{pgfscope}%
\end{pgfscope}%
\begin{pgfscope}%
\pgftext[x=3.382998in,y=0.430778in,,top]{\sffamily\fontsize{10.000000}{12.000000}\selectfont \(\displaystyle 100\)}%
\end{pgfscope}%
\begin{pgfscope}%
\pgfsetbuttcap%
\pgfsetroundjoin%
\definecolor{currentfill}{rgb}{0.000000,0.000000,0.000000}%
\pgfsetfillcolor{currentfill}%
\pgfsetlinewidth{0.803000pt}%
\definecolor{currentstroke}{rgb}{0.000000,0.000000,0.000000}%
\pgfsetstrokecolor{currentstroke}%
\pgfsetdash{}{0pt}%
\pgfsys@defobject{currentmarker}{\pgfqpoint{0.000000in}{-0.048611in}}{\pgfqpoint{0.000000in}{0.000000in}}{%
\pgfpathmoveto{\pgfqpoint{0.000000in}{0.000000in}}%
\pgfpathlineto{\pgfqpoint{0.000000in}{-0.048611in}}%
\pgfusepath{stroke,fill}%
}%
\begin{pgfscope}%
\pgfsys@transformshift{3.972384in}{0.528000in}%
\pgfsys@useobject{currentmarker}{}%
\end{pgfscope}%
\end{pgfscope}%
\begin{pgfscope}%
\pgftext[x=3.972384in,y=0.430778in,,top]{\sffamily\fontsize{10.000000}{12.000000}\selectfont \(\displaystyle 125\)}%
\end{pgfscope}%
\begin{pgfscope}%
\pgfsetbuttcap%
\pgfsetroundjoin%
\definecolor{currentfill}{rgb}{0.000000,0.000000,0.000000}%
\pgfsetfillcolor{currentfill}%
\pgfsetlinewidth{0.803000pt}%
\definecolor{currentstroke}{rgb}{0.000000,0.000000,0.000000}%
\pgfsetstrokecolor{currentstroke}%
\pgfsetdash{}{0pt}%
\pgfsys@defobject{currentmarker}{\pgfqpoint{0.000000in}{-0.048611in}}{\pgfqpoint{0.000000in}{0.000000in}}{%
\pgfpathmoveto{\pgfqpoint{0.000000in}{0.000000in}}%
\pgfpathlineto{\pgfqpoint{0.000000in}{-0.048611in}}%
\pgfusepath{stroke,fill}%
}%
\begin{pgfscope}%
\pgfsys@transformshift{4.561769in}{0.528000in}%
\pgfsys@useobject{currentmarker}{}%
\end{pgfscope}%
\end{pgfscope}%
\begin{pgfscope}%
\pgftext[x=4.561769in,y=0.430778in,,top]{\sffamily\fontsize{10.000000}{12.000000}\selectfont \(\displaystyle 150\)}%
\end{pgfscope}%
\begin{pgfscope}%
\pgfsetbuttcap%
\pgfsetroundjoin%
\definecolor{currentfill}{rgb}{0.000000,0.000000,0.000000}%
\pgfsetfillcolor{currentfill}%
\pgfsetlinewidth{0.803000pt}%
\definecolor{currentstroke}{rgb}{0.000000,0.000000,0.000000}%
\pgfsetstrokecolor{currentstroke}%
\pgfsetdash{}{0pt}%
\pgfsys@defobject{currentmarker}{\pgfqpoint{0.000000in}{-0.048611in}}{\pgfqpoint{0.000000in}{0.000000in}}{%
\pgfpathmoveto{\pgfqpoint{0.000000in}{0.000000in}}%
\pgfpathlineto{\pgfqpoint{0.000000in}{-0.048611in}}%
\pgfusepath{stroke,fill}%
}%
\begin{pgfscope}%
\pgfsys@transformshift{5.151155in}{0.528000in}%
\pgfsys@useobject{currentmarker}{}%
\end{pgfscope}%
\end{pgfscope}%
\begin{pgfscope}%
\pgftext[x=5.151155in,y=0.430778in,,top]{\sffamily\fontsize{10.000000}{12.000000}\selectfont \(\displaystyle 175\)}%
\end{pgfscope}%
\begin{pgfscope}%
\pgfsetbuttcap%
\pgfsetroundjoin%
\definecolor{currentfill}{rgb}{0.000000,0.000000,0.000000}%
\pgfsetfillcolor{currentfill}%
\pgfsetlinewidth{0.803000pt}%
\definecolor{currentstroke}{rgb}{0.000000,0.000000,0.000000}%
\pgfsetstrokecolor{currentstroke}%
\pgfsetdash{}{0pt}%
\pgfsys@defobject{currentmarker}{\pgfqpoint{0.000000in}{-0.048611in}}{\pgfqpoint{0.000000in}{0.000000in}}{%
\pgfpathmoveto{\pgfqpoint{0.000000in}{0.000000in}}%
\pgfpathlineto{\pgfqpoint{0.000000in}{-0.048611in}}%
\pgfusepath{stroke,fill}%
}%
\begin{pgfscope}%
\pgfsys@transformshift{5.740541in}{0.528000in}%
\pgfsys@useobject{currentmarker}{}%
\end{pgfscope}%
\end{pgfscope}%
\begin{pgfscope}%
\pgftext[x=5.740541in,y=0.430778in,,top]{\sffamily\fontsize{10.000000}{12.000000}\selectfont \(\displaystyle 200\)}%
\end{pgfscope}%
\begin{pgfscope}%
\pgftext[x=3.280000in,y=0.251889in,,top]{\sffamily\fontsize{10.000000}{12.000000}\selectfont Time, days}%
\end{pgfscope}%
\begin{pgfscope}%
\pgfsetbuttcap%
\pgfsetroundjoin%
\definecolor{currentfill}{rgb}{0.000000,0.000000,0.000000}%
\pgfsetfillcolor{currentfill}%
\pgfsetlinewidth{0.803000pt}%
\definecolor{currentstroke}{rgb}{0.000000,0.000000,0.000000}%
\pgfsetstrokecolor{currentstroke}%
\pgfsetdash{}{0pt}%
\pgfsys@defobject{currentmarker}{\pgfqpoint{-0.048611in}{0.000000in}}{\pgfqpoint{0.000000in}{0.000000in}}{%
\pgfpathmoveto{\pgfqpoint{0.000000in}{0.000000in}}%
\pgfpathlineto{\pgfqpoint{-0.048611in}{0.000000in}}%
\pgfusepath{stroke,fill}%
}%
\begin{pgfscope}%
\pgfsys@transformshift{0.800000in}{0.639880in}%
\pgfsys@useobject{currentmarker}{}%
\end{pgfscope}%
\end{pgfscope}%
\begin{pgfscope}%
\pgftext[x=0.633333in,y=0.591685in,left,base]{\sffamily\fontsize{10.000000}{12.000000}\selectfont \(\displaystyle 3\)}%
\end{pgfscope}%
\begin{pgfscope}%
\pgfsetbuttcap%
\pgfsetroundjoin%
\definecolor{currentfill}{rgb}{0.000000,0.000000,0.000000}%
\pgfsetfillcolor{currentfill}%
\pgfsetlinewidth{0.803000pt}%
\definecolor{currentstroke}{rgb}{0.000000,0.000000,0.000000}%
\pgfsetstrokecolor{currentstroke}%
\pgfsetdash{}{0pt}%
\pgfsys@defobject{currentmarker}{\pgfqpoint{-0.048611in}{0.000000in}}{\pgfqpoint{0.000000in}{0.000000in}}{%
\pgfpathmoveto{\pgfqpoint{0.000000in}{0.000000in}}%
\pgfpathlineto{\pgfqpoint{-0.048611in}{0.000000in}}%
\pgfusepath{stroke,fill}%
}%
\begin{pgfscope}%
\pgfsys@transformshift{0.800000in}{1.391294in}%
\pgfsys@useobject{currentmarker}{}%
\end{pgfscope}%
\end{pgfscope}%
\begin{pgfscope}%
\pgftext[x=0.633333in,y=1.343099in,left,base]{\sffamily\fontsize{10.000000}{12.000000}\selectfont \(\displaystyle 4\)}%
\end{pgfscope}%
\begin{pgfscope}%
\pgfsetbuttcap%
\pgfsetroundjoin%
\definecolor{currentfill}{rgb}{0.000000,0.000000,0.000000}%
\pgfsetfillcolor{currentfill}%
\pgfsetlinewidth{0.803000pt}%
\definecolor{currentstroke}{rgb}{0.000000,0.000000,0.000000}%
\pgfsetstrokecolor{currentstroke}%
\pgfsetdash{}{0pt}%
\pgfsys@defobject{currentmarker}{\pgfqpoint{-0.048611in}{0.000000in}}{\pgfqpoint{0.000000in}{0.000000in}}{%
\pgfpathmoveto{\pgfqpoint{0.000000in}{0.000000in}}%
\pgfpathlineto{\pgfqpoint{-0.048611in}{0.000000in}}%
\pgfusepath{stroke,fill}%
}%
\begin{pgfscope}%
\pgfsys@transformshift{0.800000in}{2.142708in}%
\pgfsys@useobject{currentmarker}{}%
\end{pgfscope}%
\end{pgfscope}%
\begin{pgfscope}%
\pgftext[x=0.633333in,y=2.094514in,left,base]{\sffamily\fontsize{10.000000}{12.000000}\selectfont \(\displaystyle 5\)}%
\end{pgfscope}%
\begin{pgfscope}%
\pgfsetbuttcap%
\pgfsetroundjoin%
\definecolor{currentfill}{rgb}{0.000000,0.000000,0.000000}%
\pgfsetfillcolor{currentfill}%
\pgfsetlinewidth{0.803000pt}%
\definecolor{currentstroke}{rgb}{0.000000,0.000000,0.000000}%
\pgfsetstrokecolor{currentstroke}%
\pgfsetdash{}{0pt}%
\pgfsys@defobject{currentmarker}{\pgfqpoint{-0.048611in}{0.000000in}}{\pgfqpoint{0.000000in}{0.000000in}}{%
\pgfpathmoveto{\pgfqpoint{0.000000in}{0.000000in}}%
\pgfpathlineto{\pgfqpoint{-0.048611in}{0.000000in}}%
\pgfusepath{stroke,fill}%
}%
\begin{pgfscope}%
\pgfsys@transformshift{0.800000in}{2.894122in}%
\pgfsys@useobject{currentmarker}{}%
\end{pgfscope}%
\end{pgfscope}%
\begin{pgfscope}%
\pgftext[x=0.633333in,y=2.845928in,left,base]{\sffamily\fontsize{10.000000}{12.000000}\selectfont \(\displaystyle 6\)}%
\end{pgfscope}%
\begin{pgfscope}%
\pgfsetbuttcap%
\pgfsetroundjoin%
\definecolor{currentfill}{rgb}{0.000000,0.000000,0.000000}%
\pgfsetfillcolor{currentfill}%
\pgfsetlinewidth{0.803000pt}%
\definecolor{currentstroke}{rgb}{0.000000,0.000000,0.000000}%
\pgfsetstrokecolor{currentstroke}%
\pgfsetdash{}{0pt}%
\pgfsys@defobject{currentmarker}{\pgfqpoint{-0.048611in}{0.000000in}}{\pgfqpoint{0.000000in}{0.000000in}}{%
\pgfpathmoveto{\pgfqpoint{0.000000in}{0.000000in}}%
\pgfpathlineto{\pgfqpoint{-0.048611in}{0.000000in}}%
\pgfusepath{stroke,fill}%
}%
\begin{pgfscope}%
\pgfsys@transformshift{0.800000in}{3.645536in}%
\pgfsys@useobject{currentmarker}{}%
\end{pgfscope}%
\end{pgfscope}%
\begin{pgfscope}%
\pgftext[x=0.633333in,y=3.597342in,left,base]{\sffamily\fontsize{10.000000}{12.000000}\selectfont \(\displaystyle 7\)}%
\end{pgfscope}%
\begin{pgfscope}%
\pgftext[x=0.577777in,y=2.376000in,,bottom,rotate=90.000000]{\sffamily\fontsize{10.000000}{12.000000}\selectfont Velocity, km/s}%
\end{pgfscope}%
\begin{pgfscope}%
\pgfpathrectangle{\pgfqpoint{0.800000in}{0.528000in}}{\pgfqpoint{4.960000in}{3.696000in}} %
\pgfusepath{clip}%
\pgfsetrectcap%
\pgfsetroundjoin%
\pgfsetlinewidth{1.505625pt}%
\definecolor{currentstroke}{rgb}{0.000000,0.000000,0.000000}%
\pgfsetstrokecolor{currentstroke}%
\pgfsetdash{}{0pt}%
\pgfpathmoveto{\pgfqpoint{1.025455in}{4.055845in}}%
\pgfpathlineto{\pgfqpoint{1.025866in}{4.055929in}}%
\pgfpathlineto{\pgfqpoint{1.026267in}{4.055089in}}%
\pgfpathlineto{\pgfqpoint{1.028250in}{4.052896in}}%
\pgfpathlineto{\pgfqpoint{1.034359in}{4.047654in}}%
\pgfpathlineto{\pgfqpoint{1.036401in}{4.045725in}}%
\pgfpathlineto{\pgfqpoint{1.037274in}{4.045523in}}%
\pgfpathlineto{\pgfqpoint{1.037449in}{4.045069in}}%
\pgfpathlineto{\pgfqpoint{1.039443in}{4.042897in}}%
\pgfpathlineto{\pgfqpoint{1.043816in}{4.039453in}}%
\pgfpathlineto{\pgfqpoint{1.045837in}{4.037184in}}%
\pgfpathlineto{\pgfqpoint{1.053752in}{4.030166in}}%
\pgfpathlineto{\pgfqpoint{1.055760in}{4.028553in}}%
\pgfpathlineto{\pgfqpoint{1.055998in}{4.028740in}}%
\pgfpathlineto{\pgfqpoint{1.056435in}{4.028550in}}%
\pgfpathlineto{\pgfqpoint{1.056668in}{4.028011in}}%
\pgfpathlineto{\pgfqpoint{1.058727in}{4.025673in}}%
\pgfpathlineto{\pgfqpoint{1.063553in}{4.021366in}}%
\pgfpathlineto{\pgfqpoint{1.072287in}{4.014325in}}%
\pgfpathlineto{\pgfqpoint{1.072938in}{4.013368in}}%
\pgfpathlineto{\pgfqpoint{1.074962in}{4.011184in}}%
\pgfpathlineto{\pgfqpoint{1.079436in}{4.007602in}}%
\pgfpathlineto{\pgfqpoint{1.081489in}{4.005360in}}%
\pgfpathlineto{\pgfqpoint{1.084650in}{4.002555in}}%
\pgfpathlineto{\pgfqpoint{1.088489in}{3.999687in}}%
\pgfpathlineto{\pgfqpoint{1.088904in}{3.999605in}}%
\pgfpathlineto{\pgfqpoint{1.089199in}{3.998973in}}%
\pgfpathlineto{\pgfqpoint{1.091291in}{3.996612in}}%
\pgfpathlineto{\pgfqpoint{1.097457in}{3.991529in}}%
\pgfpathlineto{\pgfqpoint{1.099538in}{3.989260in}}%
\pgfpathlineto{\pgfqpoint{1.102589in}{3.986668in}}%
\pgfpathlineto{\pgfqpoint{1.104697in}{3.984838in}}%
\pgfpathlineto{\pgfqpoint{1.104757in}{3.984872in}}%
\pgfpathlineto{\pgfqpoint{1.105346in}{3.984942in}}%
\pgfpathlineto{\pgfqpoint{1.105643in}{3.984326in}}%
\pgfpathlineto{\pgfqpoint{1.107758in}{3.981925in}}%
\pgfpathlineto{\pgfqpoint{1.122515in}{3.968863in}}%
\pgfpathlineto{\pgfqpoint{1.124626in}{3.967094in}}%
\pgfpathlineto{\pgfqpoint{1.124748in}{3.967173in}}%
\pgfpathlineto{\pgfqpoint{1.125249in}{3.967199in}}%
\pgfpathlineto{\pgfqpoint{1.125609in}{3.966448in}}%
\pgfpathlineto{\pgfqpoint{1.127711in}{3.964139in}}%
\pgfpathlineto{\pgfqpoint{1.132503in}{3.959993in}}%
\pgfpathlineto{\pgfqpoint{1.134662in}{3.958161in}}%
\pgfpathlineto{\pgfqpoint{1.135375in}{3.958113in}}%
\pgfpathlineto{\pgfqpoint{1.135617in}{3.957569in}}%
\pgfpathlineto{\pgfqpoint{1.137746in}{3.955197in}}%
\pgfpathlineto{\pgfqpoint{1.145866in}{3.948174in}}%
\pgfpathlineto{\pgfqpoint{1.148001in}{3.946148in}}%
\pgfpathlineto{\pgfqpoint{1.149187in}{3.945291in}}%
\pgfpathlineto{\pgfqpoint{1.151276in}{3.943160in}}%
\pgfpathlineto{\pgfqpoint{1.154072in}{3.941226in}}%
\pgfpathlineto{\pgfqpoint{1.156283in}{3.938685in}}%
\pgfpathlineto{\pgfqpoint{1.164609in}{3.931334in}}%
\pgfpathlineto{\pgfqpoint{1.166795in}{3.929621in}}%
\pgfpathlineto{\pgfqpoint{1.166920in}{3.929707in}}%
\pgfpathlineto{\pgfqpoint{1.167444in}{3.929559in}}%
\pgfpathlineto{\pgfqpoint{1.167689in}{3.929031in}}%
\pgfpathlineto{\pgfqpoint{1.169866in}{3.926592in}}%
\pgfpathlineto{\pgfqpoint{1.186749in}{3.911701in}}%
\pgfpathlineto{\pgfqpoint{1.188940in}{3.909784in}}%
\pgfpathlineto{\pgfqpoint{1.189808in}{3.909525in}}%
\pgfpathlineto{\pgfqpoint{1.189996in}{3.909080in}}%
\pgfpathlineto{\pgfqpoint{1.192147in}{3.906764in}}%
\pgfpathlineto{\pgfqpoint{1.197155in}{3.902370in}}%
\pgfpathlineto{\pgfqpoint{1.199417in}{3.900652in}}%
\pgfpathlineto{\pgfqpoint{1.199481in}{3.900694in}}%
\pgfpathlineto{\pgfqpoint{1.199991in}{3.900617in}}%
\pgfpathlineto{\pgfqpoint{1.200302in}{3.899978in}}%
\pgfpathlineto{\pgfqpoint{1.202493in}{3.897561in}}%
\pgfpathlineto{\pgfqpoint{1.239011in}{3.865162in}}%
\pgfpathlineto{\pgfqpoint{1.241390in}{3.863439in}}%
\pgfpathlineto{\pgfqpoint{1.241950in}{3.863282in}}%
\pgfpathlineto{\pgfqpoint{1.242205in}{3.862755in}}%
\pgfpathlineto{\pgfqpoint{1.244434in}{3.860284in}}%
\pgfpathlineto{\pgfqpoint{1.253983in}{3.852526in}}%
\pgfpathlineto{\pgfqpoint{1.254697in}{3.851506in}}%
\pgfpathlineto{\pgfqpoint{1.256873in}{3.849235in}}%
\pgfpathlineto{\pgfqpoint{1.263947in}{3.842957in}}%
\pgfpathlineto{\pgfqpoint{1.270005in}{3.838282in}}%
\pgfpathlineto{\pgfqpoint{1.270700in}{3.837351in}}%
\pgfpathlineto{\pgfqpoint{1.272926in}{3.834983in}}%
\pgfpathlineto{\pgfqpoint{1.290623in}{3.819472in}}%
\pgfpathlineto{\pgfqpoint{1.292922in}{3.817337in}}%
\pgfpathlineto{\pgfqpoint{1.294157in}{3.816446in}}%
\pgfpathlineto{\pgfqpoint{1.296355in}{3.814195in}}%
\pgfpathlineto{\pgfqpoint{1.303276in}{3.808294in}}%
\pgfpathlineto{\pgfqpoint{1.305547in}{3.806081in}}%
\pgfpathlineto{\pgfqpoint{1.307069in}{3.804765in}}%
\pgfpathlineto{\pgfqpoint{1.309504in}{3.802924in}}%
\pgfpathlineto{\pgfqpoint{1.310185in}{3.802655in}}%
\pgfpathlineto{\pgfqpoint{1.310383in}{3.802239in}}%
\pgfpathlineto{\pgfqpoint{1.312680in}{3.799721in}}%
\pgfpathlineto{\pgfqpoint{1.319006in}{3.794905in}}%
\pgfpathlineto{\pgfqpoint{1.321353in}{3.792521in}}%
\pgfpathlineto{\pgfqpoint{1.323672in}{3.789978in}}%
\pgfpathlineto{\pgfqpoint{1.331714in}{3.783507in}}%
\pgfpathlineto{\pgfqpoint{1.332436in}{3.782633in}}%
\pgfpathlineto{\pgfqpoint{1.334716in}{3.780193in}}%
\pgfpathlineto{\pgfqpoint{1.341065in}{3.775324in}}%
\pgfpathlineto{\pgfqpoint{1.341887in}{3.773990in}}%
\pgfpathlineto{\pgfqpoint{1.344235in}{3.771874in}}%
\pgfpathlineto{\pgfqpoint{1.345416in}{3.771094in}}%
\pgfpathlineto{\pgfqpoint{1.347708in}{3.768681in}}%
\pgfpathlineto{\pgfqpoint{1.367195in}{3.752200in}}%
\pgfpathlineto{\pgfqpoint{1.368232in}{3.750513in}}%
\pgfpathlineto{\pgfqpoint{1.376487in}{3.743888in}}%
\pgfpathlineto{\pgfqpoint{1.377286in}{3.742791in}}%
\pgfpathlineto{\pgfqpoint{1.379571in}{3.740477in}}%
\pgfpathlineto{\pgfqpoint{1.388658in}{3.732696in}}%
\pgfpathlineto{\pgfqpoint{1.390975in}{3.730396in}}%
\pgfpathlineto{\pgfqpoint{1.394424in}{3.727515in}}%
\pgfpathlineto{\pgfqpoint{1.396816in}{3.725303in}}%
\pgfpathlineto{\pgfqpoint{1.398149in}{3.724319in}}%
\pgfpathlineto{\pgfqpoint{1.400447in}{3.722013in}}%
\pgfpathlineto{\pgfqpoint{1.407756in}{3.715755in}}%
\pgfpathlineto{\pgfqpoint{1.410153in}{3.713486in}}%
\pgfpathlineto{\pgfqpoint{1.411614in}{3.712316in}}%
\pgfpathlineto{\pgfqpoint{1.414050in}{3.710086in}}%
\pgfpathlineto{\pgfqpoint{1.415310in}{3.709223in}}%
\pgfpathlineto{\pgfqpoint{1.417668in}{3.706790in}}%
\pgfpathlineto{\pgfqpoint{1.455364in}{3.674220in}}%
\pgfpathlineto{\pgfqpoint{1.456274in}{3.672818in}}%
\pgfpathlineto{\pgfqpoint{1.458804in}{3.670644in}}%
\pgfpathlineto{\pgfqpoint{1.459933in}{3.669897in}}%
\pgfpathlineto{\pgfqpoint{1.462339in}{3.667356in}}%
\pgfpathlineto{\pgfqpoint{1.467137in}{3.663809in}}%
\pgfpathlineto{\pgfqpoint{1.467992in}{3.662564in}}%
\pgfpathlineto{\pgfqpoint{1.470408in}{3.660244in}}%
\pgfpathlineto{\pgfqpoint{1.473616in}{3.657964in}}%
\pgfpathlineto{\pgfqpoint{1.476248in}{3.655071in}}%
\pgfpathlineto{\pgfqpoint{1.489997in}{3.642996in}}%
\pgfpathlineto{\pgfqpoint{1.492818in}{3.641042in}}%
\pgfpathlineto{\pgfqpoint{1.493669in}{3.640077in}}%
\pgfpathlineto{\pgfqpoint{1.496073in}{3.637602in}}%
\pgfpathlineto{\pgfqpoint{1.502769in}{3.632199in}}%
\pgfpathlineto{\pgfqpoint{1.503643in}{3.631337in}}%
\pgfpathlineto{\pgfqpoint{1.506058in}{3.628819in}}%
\pgfpathlineto{\pgfqpoint{1.509103in}{3.626862in}}%
\pgfpathlineto{\pgfqpoint{1.511962in}{3.623711in}}%
\pgfpathlineto{\pgfqpoint{1.514549in}{3.621471in}}%
\pgfpathlineto{\pgfqpoint{1.515727in}{3.620694in}}%
\pgfpathlineto{\pgfqpoint{1.518185in}{3.618125in}}%
\pgfpathlineto{\pgfqpoint{1.522932in}{3.614413in}}%
\pgfpathlineto{\pgfqpoint{1.523823in}{3.613568in}}%
\pgfpathlineto{\pgfqpoint{1.526290in}{3.610988in}}%
\pgfpathlineto{\pgfqpoint{1.531048in}{3.607249in}}%
\pgfpathlineto{\pgfqpoint{1.531945in}{3.606427in}}%
\pgfpathlineto{\pgfqpoint{1.534428in}{3.603825in}}%
\pgfpathlineto{\pgfqpoint{1.539126in}{3.600037in}}%
\pgfpathlineto{\pgfqpoint{1.540083in}{3.599297in}}%
\pgfpathlineto{\pgfqpoint{1.542646in}{3.596578in}}%
\pgfpathlineto{\pgfqpoint{1.547493in}{3.592864in}}%
\pgfpathlineto{\pgfqpoint{1.548356in}{3.591930in}}%
\pgfpathlineto{\pgfqpoint{1.550792in}{3.589434in}}%
\pgfpathlineto{\pgfqpoint{1.553797in}{3.587434in}}%
\pgfpathlineto{\pgfqpoint{1.554711in}{3.586136in}}%
\pgfpathlineto{\pgfqpoint{1.557255in}{3.583750in}}%
\pgfpathlineto{\pgfqpoint{1.559095in}{3.582108in}}%
\pgfpathlineto{\pgfqpoint{1.565973in}{3.576506in}}%
\pgfpathlineto{\pgfqpoint{1.566920in}{3.575614in}}%
\pgfpathlineto{\pgfqpoint{1.569416in}{3.573042in}}%
\pgfpathlineto{\pgfqpoint{1.574227in}{3.569184in}}%
\pgfpathlineto{\pgfqpoint{1.575188in}{3.568395in}}%
\pgfpathlineto{\pgfqpoint{1.577744in}{3.565718in}}%
\pgfpathlineto{\pgfqpoint{1.582594in}{3.561854in}}%
\pgfpathlineto{\pgfqpoint{1.583537in}{3.561054in}}%
\pgfpathlineto{\pgfqpoint{1.586075in}{3.558408in}}%
\pgfpathlineto{\pgfqpoint{1.589177in}{3.556344in}}%
\pgfpathlineto{\pgfqpoint{1.590266in}{3.554726in}}%
\pgfpathlineto{\pgfqpoint{1.593324in}{3.552664in}}%
\pgfpathlineto{\pgfqpoint{1.594246in}{3.551399in}}%
\pgfpathlineto{\pgfqpoint{1.596824in}{3.548948in}}%
\pgfpathlineto{\pgfqpoint{1.600284in}{3.546399in}}%
\pgfpathlineto{\pgfqpoint{1.602949in}{3.543565in}}%
\pgfpathlineto{\pgfqpoint{1.608070in}{3.539690in}}%
\pgfpathlineto{\pgfqpoint{1.609008in}{3.538446in}}%
\pgfpathlineto{\pgfqpoint{1.611617in}{3.535954in}}%
\pgfpathlineto{\pgfqpoint{1.615156in}{3.533284in}}%
\pgfpathlineto{\pgfqpoint{1.617790in}{3.530534in}}%
\pgfpathlineto{\pgfqpoint{1.622903in}{3.526624in}}%
\pgfpathlineto{\pgfqpoint{1.623842in}{3.525480in}}%
\pgfpathlineto{\pgfqpoint{1.626435in}{3.522930in}}%
\pgfpathlineto{\pgfqpoint{1.636903in}{3.513832in}}%
\pgfpathlineto{\pgfqpoint{1.639776in}{3.511535in}}%
\pgfpathlineto{\pgfqpoint{1.640819in}{3.510774in}}%
\pgfpathlineto{\pgfqpoint{1.643466in}{3.508016in}}%
\pgfpathlineto{\pgfqpoint{1.646827in}{3.505778in}}%
\pgfpathlineto{\pgfqpoint{1.671621in}{3.483379in}}%
\pgfpathlineto{\pgfqpoint{1.674588in}{3.481077in}}%
\pgfpathlineto{\pgfqpoint{1.675629in}{3.480214in}}%
\pgfpathlineto{\pgfqpoint{1.678255in}{3.477536in}}%
\pgfpathlineto{\pgfqpoint{1.681397in}{3.475327in}}%
\pgfpathlineto{\pgfqpoint{1.682381in}{3.474130in}}%
\pgfpathlineto{\pgfqpoint{1.685046in}{3.471547in}}%
\pgfpathlineto{\pgfqpoint{1.693408in}{3.464479in}}%
\pgfpathlineto{\pgfqpoint{1.696091in}{3.461876in}}%
\pgfpathlineto{\pgfqpoint{1.704632in}{3.454539in}}%
\pgfpathlineto{\pgfqpoint{1.707494in}{3.452026in}}%
\pgfpathlineto{\pgfqpoint{1.708885in}{3.451008in}}%
\pgfpathlineto{\pgfqpoint{1.711531in}{3.448382in}}%
\pgfpathlineto{\pgfqpoint{1.716804in}{3.444235in}}%
\pgfpathlineto{\pgfqpoint{1.717841in}{3.443159in}}%
\pgfpathlineto{\pgfqpoint{1.720502in}{3.440530in}}%
\pgfpathlineto{\pgfqpoint{1.726039in}{3.436346in}}%
\pgfpathlineto{\pgfqpoint{1.728851in}{3.433751in}}%
\pgfpathlineto{\pgfqpoint{1.731870in}{3.430575in}}%
\pgfpathlineto{\pgfqpoint{1.742862in}{3.421094in}}%
\pgfpathlineto{\pgfqpoint{1.745796in}{3.418555in}}%
\pgfpathlineto{\pgfqpoint{1.747180in}{3.417526in}}%
\pgfpathlineto{\pgfqpoint{1.749890in}{3.414844in}}%
\pgfpathlineto{\pgfqpoint{1.755337in}{3.410609in}}%
\pgfpathlineto{\pgfqpoint{1.756330in}{3.409506in}}%
\pgfpathlineto{\pgfqpoint{1.759051in}{3.406834in}}%
\pgfpathlineto{\pgfqpoint{1.766842in}{3.400595in}}%
\pgfpathlineto{\pgfqpoint{1.767922in}{3.399268in}}%
\pgfpathlineto{\pgfqpoint{1.770770in}{3.396620in}}%
\pgfpathlineto{\pgfqpoint{1.774428in}{3.393958in}}%
\pgfpathlineto{\pgfqpoint{1.777634in}{3.390610in}}%
\pgfpathlineto{\pgfqpoint{1.781835in}{3.387075in}}%
\pgfpathlineto{\pgfqpoint{1.784790in}{3.384462in}}%
\pgfpathlineto{\pgfqpoint{1.786386in}{3.383176in}}%
\pgfpathlineto{\pgfqpoint{1.789218in}{3.380500in}}%
\pgfpathlineto{\pgfqpoint{1.793620in}{3.376693in}}%
\pgfpathlineto{\pgfqpoint{1.796982in}{3.374300in}}%
\pgfpathlineto{\pgfqpoint{1.798024in}{3.373034in}}%
\pgfpathlineto{\pgfqpoint{1.800901in}{3.370314in}}%
\pgfpathlineto{\pgfqpoint{1.805122in}{3.366783in}}%
\pgfpathlineto{\pgfqpoint{1.808090in}{3.364110in}}%
\pgfpathlineto{\pgfqpoint{1.809826in}{3.362676in}}%
\pgfpathlineto{\pgfqpoint{1.812794in}{3.360009in}}%
\pgfpathlineto{\pgfqpoint{1.814525in}{3.358588in}}%
\pgfpathlineto{\pgfqpoint{1.817452in}{3.355920in}}%
\pgfpathlineto{\pgfqpoint{1.819531in}{3.354075in}}%
\pgfpathlineto{\pgfqpoint{1.827584in}{3.347618in}}%
\pgfpathlineto{\pgfqpoint{1.828656in}{3.346329in}}%
\pgfpathlineto{\pgfqpoint{1.831546in}{3.343603in}}%
\pgfpathlineto{\pgfqpoint{1.837944in}{3.338449in}}%
\pgfpathlineto{\pgfqpoint{1.840853in}{3.335525in}}%
\pgfpathlineto{\pgfqpoint{1.844505in}{3.332992in}}%
\pgfpathlineto{\pgfqpoint{1.850332in}{3.327325in}}%
\pgfpathlineto{\pgfqpoint{1.853579in}{3.324735in}}%
\pgfpathlineto{\pgfqpoint{1.854785in}{3.323735in}}%
\pgfpathlineto{\pgfqpoint{1.857663in}{3.320894in}}%
\pgfpathlineto{\pgfqpoint{1.863138in}{3.316339in}}%
\pgfpathlineto{\pgfqpoint{1.864432in}{3.315348in}}%
\pgfpathlineto{\pgfqpoint{1.867328in}{3.312488in}}%
\pgfpathlineto{\pgfqpoint{1.871005in}{3.309923in}}%
\pgfpathlineto{\pgfqpoint{1.874567in}{3.306214in}}%
\pgfpathlineto{\pgfqpoint{1.878035in}{3.303663in}}%
\pgfpathlineto{\pgfqpoint{1.879177in}{3.302381in}}%
\pgfpathlineto{\pgfqpoint{1.882150in}{3.299574in}}%
\pgfpathlineto{\pgfqpoint{1.889036in}{3.293751in}}%
\pgfpathlineto{\pgfqpoint{1.892129in}{3.290959in}}%
\pgfpathlineto{\pgfqpoint{1.894058in}{3.289314in}}%
\pgfpathlineto{\pgfqpoint{1.897318in}{3.286637in}}%
\pgfpathlineto{\pgfqpoint{1.898642in}{3.285594in}}%
\pgfpathlineto{\pgfqpoint{1.901618in}{3.282673in}}%
\pgfpathlineto{\pgfqpoint{1.907548in}{3.278039in}}%
\pgfpathlineto{\pgfqpoint{1.908683in}{3.276741in}}%
\pgfpathlineto{\pgfqpoint{1.911668in}{3.273931in}}%
\pgfpathlineto{\pgfqpoint{1.921291in}{3.265672in}}%
\pgfpathlineto{\pgfqpoint{1.924571in}{3.262948in}}%
\pgfpathlineto{\pgfqpoint{1.926004in}{3.261791in}}%
\pgfpathlineto{\pgfqpoint{1.928990in}{3.258909in}}%
\pgfpathlineto{\pgfqpoint{1.937428in}{3.252054in}}%
\pgfpathlineto{\pgfqpoint{1.938643in}{3.250751in}}%
\pgfpathlineto{\pgfqpoint{1.941723in}{3.247859in}}%
\pgfpathlineto{\pgfqpoint{1.948733in}{3.242008in}}%
\pgfpathlineto{\pgfqpoint{1.951796in}{3.239124in}}%
\pgfpathlineto{\pgfqpoint{1.961952in}{3.230324in}}%
\pgfpathlineto{\pgfqpoint{1.987237in}{3.208558in}}%
\pgfpathlineto{\pgfqpoint{1.990553in}{3.205676in}}%
\pgfpathlineto{\pgfqpoint{1.992301in}{3.204253in}}%
\pgfpathlineto{\pgfqpoint{1.995447in}{3.201332in}}%
\pgfpathlineto{\pgfqpoint{2.002701in}{3.195252in}}%
\pgfpathlineto{\pgfqpoint{2.005934in}{3.192277in}}%
\pgfpathlineto{\pgfqpoint{2.010358in}{3.188766in}}%
\pgfpathlineto{\pgfqpoint{2.013494in}{3.185743in}}%
\pgfpathlineto{\pgfqpoint{2.019674in}{3.180802in}}%
\pgfpathlineto{\pgfqpoint{2.020941in}{3.179551in}}%
\pgfpathlineto{\pgfqpoint{2.024147in}{3.176517in}}%
\pgfpathlineto{\pgfqpoint{2.089758in}{3.120529in}}%
\pgfpathlineto{\pgfqpoint{2.096099in}{3.114738in}}%
\pgfpathlineto{\pgfqpoint{2.099337in}{3.111683in}}%
\pgfpathlineto{\pgfqpoint{2.105944in}{3.106470in}}%
\pgfpathlineto{\pgfqpoint{2.107334in}{3.104908in}}%
\pgfpathlineto{\pgfqpoint{2.110875in}{3.101837in}}%
\pgfpathlineto{\pgfqpoint{2.112812in}{3.100221in}}%
\pgfpathlineto{\pgfqpoint{2.116282in}{3.097126in}}%
\pgfpathlineto{\pgfqpoint{2.120334in}{3.094205in}}%
\pgfpathlineto{\pgfqpoint{2.127707in}{3.087462in}}%
\pgfpathlineto{\pgfqpoint{2.129318in}{3.086133in}}%
\pgfpathlineto{\pgfqpoint{2.132710in}{3.082953in}}%
\pgfpathlineto{\pgfqpoint{2.162031in}{3.058303in}}%
\pgfpathlineto{\pgfqpoint{2.163832in}{3.056218in}}%
\pgfpathlineto{\pgfqpoint{2.172195in}{3.049051in}}%
\pgfpathlineto{\pgfqpoint{2.181787in}{3.041209in}}%
\pgfpathlineto{\pgfqpoint{2.183224in}{3.039787in}}%
\pgfpathlineto{\pgfqpoint{2.186758in}{3.036551in}}%
\pgfpathlineto{\pgfqpoint{2.194766in}{3.029888in}}%
\pgfpathlineto{\pgfqpoint{2.198291in}{3.026661in}}%
\pgfpathlineto{\pgfqpoint{2.209229in}{3.017537in}}%
\pgfpathlineto{\pgfqpoint{2.212807in}{3.014224in}}%
\pgfpathlineto{\pgfqpoint{2.244883in}{2.986857in}}%
\pgfpathlineto{\pgfqpoint{2.248930in}{2.983589in}}%
\pgfpathlineto{\pgfqpoint{2.250532in}{2.982198in}}%
\pgfpathlineto{\pgfqpoint{2.254204in}{2.978807in}}%
\pgfpathlineto{\pgfqpoint{2.293380in}{2.945371in}}%
\pgfpathlineto{\pgfqpoint{2.303857in}{2.936920in}}%
\pgfpathlineto{\pgfqpoint{2.305796in}{2.934773in}}%
\pgfpathlineto{\pgfqpoint{2.349064in}{2.898006in}}%
\pgfpathlineto{\pgfqpoint{2.353460in}{2.894530in}}%
\pgfpathlineto{\pgfqpoint{2.355232in}{2.892829in}}%
\pgfpathlineto{\pgfqpoint{2.359321in}{2.889299in}}%
\pgfpathlineto{\pgfqpoint{2.361836in}{2.887113in}}%
\pgfpathlineto{\pgfqpoint{2.366794in}{2.883430in}}%
\pgfpathlineto{\pgfqpoint{2.380885in}{2.870973in}}%
\pgfpathlineto{\pgfqpoint{2.385243in}{2.867452in}}%
\pgfpathlineto{\pgfqpoint{2.387096in}{2.865802in}}%
\pgfpathlineto{\pgfqpoint{2.391198in}{2.862179in}}%
\pgfpathlineto{\pgfqpoint{2.396306in}{2.858244in}}%
\pgfpathlineto{\pgfqpoint{2.400872in}{2.853961in}}%
\pgfpathlineto{\pgfqpoint{2.406302in}{2.849640in}}%
\pgfpathlineto{\pgfqpoint{2.410297in}{2.845955in}}%
\pgfpathlineto{\pgfqpoint{2.424527in}{2.834250in}}%
\pgfpathlineto{\pgfqpoint{2.426335in}{2.832480in}}%
\pgfpathlineto{\pgfqpoint{2.430611in}{2.828841in}}%
\pgfpathlineto{\pgfqpoint{2.433065in}{2.826732in}}%
\pgfpathlineto{\pgfqpoint{2.437635in}{2.823091in}}%
\pgfpathlineto{\pgfqpoint{2.439451in}{2.821432in}}%
\pgfpathlineto{\pgfqpoint{2.443697in}{2.817703in}}%
\pgfpathlineto{\pgfqpoint{2.448914in}{2.813687in}}%
\pgfpathlineto{\pgfqpoint{2.453735in}{2.809218in}}%
\pgfpathlineto{\pgfqpoint{2.458769in}{2.805422in}}%
\pgfpathlineto{\pgfqpoint{2.464374in}{2.800471in}}%
\pgfpathlineto{\pgfqpoint{2.466352in}{2.798628in}}%
\pgfpathlineto{\pgfqpoint{2.470726in}{2.794915in}}%
\pgfpathlineto{\pgfqpoint{2.473347in}{2.792638in}}%
\pgfpathlineto{\pgfqpoint{2.478701in}{2.788625in}}%
\pgfpathlineto{\pgfqpoint{2.549489in}{2.728479in}}%
\pgfpathlineto{\pgfqpoint{2.554926in}{2.724380in}}%
\pgfpathlineto{\pgfqpoint{2.563438in}{2.716843in}}%
\pgfpathlineto{\pgfqpoint{2.568056in}{2.712916in}}%
\pgfpathlineto{\pgfqpoint{2.571189in}{2.710205in}}%
\pgfpathlineto{\pgfqpoint{2.625356in}{2.664872in}}%
\pgfpathlineto{\pgfqpoint{2.630507in}{2.660779in}}%
\pgfpathlineto{\pgfqpoint{2.632807in}{2.658627in}}%
\pgfpathlineto{\pgfqpoint{2.638078in}{2.654502in}}%
\pgfpathlineto{\pgfqpoint{2.640408in}{2.652236in}}%
\pgfpathlineto{\pgfqpoint{2.649272in}{2.645135in}}%
\pgfpathlineto{\pgfqpoint{2.651585in}{2.642907in}}%
\pgfpathlineto{\pgfqpoint{2.660006in}{2.635943in}}%
\pgfpathlineto{\pgfqpoint{2.663039in}{2.633317in}}%
\pgfpathlineto{\pgfqpoint{2.676174in}{2.622814in}}%
\pgfpathlineto{\pgfqpoint{2.681610in}{2.617935in}}%
\pgfpathlineto{\pgfqpoint{2.686666in}{2.613709in}}%
\pgfpathlineto{\pgfqpoint{2.689732in}{2.611053in}}%
\pgfpathlineto{\pgfqpoint{2.706389in}{2.597448in}}%
\pgfpathlineto{\pgfqpoint{2.708748in}{2.595280in}}%
\pgfpathlineto{\pgfqpoint{2.714290in}{2.590920in}}%
\pgfpathlineto{\pgfqpoint{2.716686in}{2.588639in}}%
\pgfpathlineto{\pgfqpoint{2.725425in}{2.581433in}}%
\pgfpathlineto{\pgfqpoint{2.728782in}{2.578545in}}%
\pgfpathlineto{\pgfqpoint{3.008070in}{2.349144in}}%
\pgfpathlineto{\pgfqpoint{3.014340in}{2.343935in}}%
\pgfpathlineto{\pgfqpoint{3.020734in}{2.338496in}}%
\pgfpathlineto{\pgfqpoint{3.028887in}{2.331976in}}%
\pgfpathlineto{\pgfqpoint{3.035299in}{2.326748in}}%
\pgfpathlineto{\pgfqpoint{3.039783in}{2.323049in}}%
\pgfpathlineto{\pgfqpoint{3.047113in}{2.317385in}}%
\pgfpathlineto{\pgfqpoint{3.054204in}{2.311373in}}%
\pgfpathlineto{\pgfqpoint{3.060928in}{2.306272in}}%
\pgfpathlineto{\pgfqpoint{3.078597in}{2.291678in}}%
\pgfpathlineto{\pgfqpoint{3.082765in}{2.288179in}}%
\pgfpathlineto{\pgfqpoint{3.096240in}{2.277429in}}%
\pgfpathlineto{\pgfqpoint{3.102910in}{2.271981in}}%
\pgfpathlineto{\pgfqpoint{3.109633in}{2.266890in}}%
\pgfpathlineto{\pgfqpoint{3.143661in}{2.239326in}}%
\pgfpathlineto{\pgfqpoint{3.149567in}{2.234716in}}%
\pgfpathlineto{\pgfqpoint{3.182518in}{2.207917in}}%
\pgfpathlineto{\pgfqpoint{3.190283in}{2.201977in}}%
\pgfpathlineto{\pgfqpoint{3.197496in}{2.196069in}}%
\pgfpathlineto{\pgfqpoint{3.204711in}{2.190143in}}%
\pgfpathlineto{\pgfqpoint{3.212886in}{2.183854in}}%
\pgfpathlineto{\pgfqpoint{3.220794in}{2.177329in}}%
\pgfpathlineto{\pgfqpoint{3.228173in}{2.171740in}}%
\pgfpathlineto{\pgfqpoint{3.247573in}{2.155963in}}%
\pgfpathlineto{\pgfqpoint{3.255281in}{2.150118in}}%
\pgfpathlineto{\pgfqpoint{3.269346in}{2.138637in}}%
\pgfpathlineto{\pgfqpoint{3.277178in}{2.132706in}}%
\pgfpathlineto{\pgfqpoint{3.291394in}{2.121124in}}%
\pgfpathlineto{\pgfqpoint{3.299605in}{2.114884in}}%
\pgfpathlineto{\pgfqpoint{3.309178in}{2.107240in}}%
\pgfpathlineto{\pgfqpoint{3.316407in}{2.101588in}}%
\pgfpathlineto{\pgfqpoint{3.326402in}{2.093661in}}%
\pgfpathlineto{\pgfqpoint{3.334347in}{2.087243in}}%
\pgfpathlineto{\pgfqpoint{3.342318in}{2.080852in}}%
\pgfpathlineto{\pgfqpoint{3.350750in}{2.074458in}}%
\pgfpathlineto{\pgfqpoint{3.360784in}{2.066518in}}%
\pgfpathlineto{\pgfqpoint{3.368598in}{2.060333in}}%
\pgfpathlineto{\pgfqpoint{3.377222in}{2.053375in}}%
\pgfpathlineto{\pgfqpoint{3.385597in}{2.047064in}}%
\pgfpathlineto{\pgfqpoint{3.400650in}{2.034947in}}%
\pgfpathlineto{\pgfqpoint{3.410416in}{2.027436in}}%
\pgfpathlineto{\pgfqpoint{3.418749in}{2.020786in}}%
\pgfpathlineto{\pgfqpoint{3.428166in}{2.013617in}}%
\pgfpathlineto{\pgfqpoint{3.437299in}{2.006341in}}%
\pgfpathlineto{\pgfqpoint{3.445578in}{2.000128in}}%
\pgfpathlineto{\pgfqpoint{3.468831in}{1.981889in}}%
\pgfpathlineto{\pgfqpoint{3.476880in}{1.975731in}}%
\pgfpathlineto{\pgfqpoint{3.492842in}{1.963048in}}%
\pgfpathlineto{\pgfqpoint{3.503247in}{1.955104in}}%
\pgfpathlineto{\pgfqpoint{3.512164in}{1.948103in}}%
\pgfpathlineto{\pgfqpoint{3.521751in}{1.940901in}}%
\pgfpathlineto{\pgfqpoint{3.532308in}{1.932658in}}%
\pgfpathlineto{\pgfqpoint{3.540763in}{1.926266in}}%
\pgfpathlineto{\pgfqpoint{3.558365in}{1.912543in}}%
\pgfpathlineto{\pgfqpoint{3.567173in}{1.905922in}}%
\pgfpathlineto{\pgfqpoint{3.584611in}{1.892308in}}%
\pgfpathlineto{\pgfqpoint{3.594244in}{1.885118in}}%
\pgfpathlineto{\pgfqpoint{3.606417in}{1.875812in}}%
\pgfpathlineto{\pgfqpoint{3.622636in}{1.863250in}}%
\pgfpathlineto{\pgfqpoint{3.632483in}{1.855705in}}%
\pgfpathlineto{\pgfqpoint{3.642360in}{1.848361in}}%
\pgfpathlineto{\pgfqpoint{3.655149in}{1.838660in}}%
\pgfpathlineto{\pgfqpoint{3.685764in}{1.815288in}}%
\pgfpathlineto{\pgfqpoint{3.696270in}{1.807279in}}%
\pgfpathlineto{\pgfqpoint{3.706519in}{1.799716in}}%
\pgfpathlineto{\pgfqpoint{3.720394in}{1.789289in}}%
\pgfpathlineto{\pgfqpoint{3.794660in}{1.733571in}}%
\pgfpathlineto{\pgfqpoint{3.806777in}{1.724311in}}%
\pgfpathlineto{\pgfqpoint{3.832771in}{1.705128in}}%
\pgfpathlineto{\pgfqpoint{3.844243in}{1.696730in}}%
\pgfpathlineto{\pgfqpoint{3.857341in}{1.687026in}}%
\pgfpathlineto{\pgfqpoint{3.868878in}{1.678488in}}%
\pgfpathlineto{\pgfqpoint{3.881175in}{1.669404in}}%
\pgfpathlineto{\pgfqpoint{3.893072in}{1.660737in}}%
\pgfpathlineto{\pgfqpoint{3.906160in}{1.651092in}}%
\pgfpathlineto{\pgfqpoint{3.918213in}{1.642303in}}%
\pgfpathlineto{\pgfqpoint{3.931377in}{1.632664in}}%
\pgfpathlineto{\pgfqpoint{3.943516in}{1.623857in}}%
\pgfpathlineto{\pgfqpoint{3.957058in}{1.613980in}}%
\pgfpathlineto{\pgfqpoint{3.969514in}{1.604959in}}%
\pgfpathlineto{\pgfqpoint{3.983045in}{1.595141in}}%
\pgfpathlineto{\pgfqpoint{3.995724in}{1.586020in}}%
\pgfpathlineto{\pgfqpoint{4.009794in}{1.575869in}}%
\pgfpathlineto{\pgfqpoint{4.023071in}{1.566285in}}%
\pgfpathlineto{\pgfqpoint{4.036770in}{1.556501in}}%
\pgfpathlineto{\pgfqpoint{4.050329in}{1.546771in}}%
\pgfpathlineto{\pgfqpoint{4.064555in}{1.536681in}}%
\pgfpathlineto{\pgfqpoint{4.078598in}{1.526592in}}%
\pgfpathlineto{\pgfqpoint{4.093615in}{1.516107in}}%
\pgfpathlineto{\pgfqpoint{4.125288in}{1.493646in}}%
\pgfpathlineto{\pgfqpoint{4.140614in}{1.482895in}}%
\pgfpathlineto{\pgfqpoint{4.155810in}{1.472121in}}%
\pgfpathlineto{\pgfqpoint{4.178025in}{1.456580in}}%
\pgfpathlineto{\pgfqpoint{4.205458in}{1.437597in}}%
\pgfpathlineto{\pgfqpoint{4.221935in}{1.426359in}}%
\pgfpathlineto{\pgfqpoint{4.257748in}{1.401563in}}%
\pgfpathlineto{\pgfqpoint{4.281217in}{1.385483in}}%
\pgfpathlineto{\pgfqpoint{4.343935in}{1.343191in}}%
\pgfpathlineto{\pgfqpoint{4.362134in}{1.331098in}}%
\pgfpathlineto{\pgfqpoint{4.390730in}{1.311890in}}%
\pgfpathlineto{\pgfqpoint{4.418686in}{1.293500in}}%
\pgfpathlineto{\pgfqpoint{4.437925in}{1.280761in}}%
\pgfpathlineto{\pgfqpoint{4.468069in}{1.261256in}}%
\pgfpathlineto{\pgfqpoint{4.508204in}{1.235297in}}%
\pgfpathlineto{\pgfqpoint{4.539352in}{1.215182in}}%
\pgfpathlineto{\pgfqpoint{4.586543in}{1.185432in}}%
\pgfpathlineto{\pgfqpoint{4.609509in}{1.171129in}}%
\pgfpathlineto{\pgfqpoint{4.805179in}{1.052719in}}%
\pgfpathlineto{\pgfqpoint{4.861335in}{1.020392in}}%
\pgfpathlineto{\pgfqpoint{4.890558in}{1.003881in}}%
\pgfpathlineto{\pgfqpoint{4.951659in}{0.969762in}}%
\pgfpathlineto{\pgfqpoint{4.993587in}{0.946943in}}%
\pgfpathlineto{\pgfqpoint{5.025740in}{0.929798in}}%
\pgfpathlineto{\pgfqpoint{5.085706in}{0.898245in}}%
\pgfpathlineto{\pgfqpoint{5.144456in}{0.868401in}}%
\pgfpathlineto{\pgfqpoint{5.301966in}{0.793153in}}%
\pgfpathlineto{\pgfqpoint{5.393390in}{0.752951in}}%
\pgfpathlineto{\pgfqpoint{5.534545in}{0.696000in}}%
\pgfpathlineto{\pgfqpoint{5.534545in}{0.696000in}}%
\pgfusepath{stroke}%
\end{pgfscope}%
\begin{pgfscope}%
\pgfsetrectcap%
\pgfsetmiterjoin%
\pgfsetlinewidth{0.803000pt}%
\definecolor{currentstroke}{rgb}{0.000000,0.000000,0.000000}%
\pgfsetstrokecolor{currentstroke}%
\pgfsetdash{}{0pt}%
\pgfpathmoveto{\pgfqpoint{0.800000in}{0.528000in}}%
\pgfpathlineto{\pgfqpoint{0.800000in}{4.224000in}}%
\pgfusepath{stroke}%
\end{pgfscope}%
\begin{pgfscope}%
\pgfsetrectcap%
\pgfsetmiterjoin%
\pgfsetlinewidth{0.803000pt}%
\definecolor{currentstroke}{rgb}{0.000000,0.000000,0.000000}%
\pgfsetstrokecolor{currentstroke}%
\pgfsetdash{}{0pt}%
\pgfpathmoveto{\pgfqpoint{5.760000in}{0.528000in}}%
\pgfpathlineto{\pgfqpoint{5.760000in}{4.224000in}}%
\pgfusepath{stroke}%
\end{pgfscope}%
\begin{pgfscope}%
\pgfsetrectcap%
\pgfsetmiterjoin%
\pgfsetlinewidth{0.803000pt}%
\definecolor{currentstroke}{rgb}{0.000000,0.000000,0.000000}%
\pgfsetstrokecolor{currentstroke}%
\pgfsetdash{}{0pt}%
\pgfpathmoveto{\pgfqpoint{0.800000in}{0.528000in}}%
\pgfpathlineto{\pgfqpoint{5.760000in}{0.528000in}}%
\pgfusepath{stroke}%
\end{pgfscope}%
\begin{pgfscope}%
\pgfsetrectcap%
\pgfsetmiterjoin%
\pgfsetlinewidth{0.803000pt}%
\definecolor{currentstroke}{rgb}{0.000000,0.000000,0.000000}%
\pgfsetstrokecolor{currentstroke}%
\pgfsetdash{}{0pt}%
\pgfpathmoveto{\pgfqpoint{0.800000in}{4.224000in}}%
\pgfpathlineto{\pgfqpoint{5.760000in}{4.224000in}}%
\pgfusepath{stroke}%
\end{pgfscope}%
\begin{pgfscope}%
\pgfsetbuttcap%
\pgfsetroundjoin%
\definecolor{currentfill}{rgb}{0.000000,0.000000,0.000000}%
\pgfsetfillcolor{currentfill}%
\pgfsetlinewidth{0.803000pt}%
\definecolor{currentstroke}{rgb}{0.000000,0.000000,0.000000}%
\pgfsetstrokecolor{currentstroke}%
\pgfsetdash{}{0pt}%
\pgfsys@defobject{currentmarker}{\pgfqpoint{0.000000in}{0.000000in}}{\pgfqpoint{0.048611in}{0.000000in}}{%
\pgfpathmoveto{\pgfqpoint{0.000000in}{0.000000in}}%
\pgfpathlineto{\pgfqpoint{0.048611in}{0.000000in}}%
\pgfusepath{stroke,fill}%
}%
\begin{pgfscope}%
\pgfsys@transformshift{5.760000in}{0.693448in}%
\pgfsys@useobject{currentmarker}{}%
\end{pgfscope}%
\end{pgfscope}%
\begin{pgfscope}%
\pgftext[x=5.857222in,y=0.645253in,left,base]{\sffamily\fontsize{10.000000}{12.000000}\selectfont \(\displaystyle 0\)}%
\end{pgfscope}%
\begin{pgfscope}%
\pgfsetbuttcap%
\pgfsetroundjoin%
\definecolor{currentfill}{rgb}{0.000000,0.000000,0.000000}%
\pgfsetfillcolor{currentfill}%
\pgfsetlinewidth{0.803000pt}%
\definecolor{currentstroke}{rgb}{0.000000,0.000000,0.000000}%
\pgfsetstrokecolor{currentstroke}%
\pgfsetdash{}{0pt}%
\pgfsys@defobject{currentmarker}{\pgfqpoint{0.000000in}{0.000000in}}{\pgfqpoint{0.048611in}{0.000000in}}{%
\pgfpathmoveto{\pgfqpoint{0.000000in}{0.000000in}}%
\pgfpathlineto{\pgfqpoint{0.048611in}{0.000000in}}%
\pgfusepath{stroke,fill}%
}%
\begin{pgfscope}%
\pgfsys@transformshift{5.760000in}{1.283369in}%
\pgfsys@useobject{currentmarker}{}%
\end{pgfscope}%
\end{pgfscope}%
\begin{pgfscope}%
\pgftext[x=5.857222in,y=1.235175in,left,base]{\sffamily\fontsize{10.000000}{12.000000}\selectfont \(\displaystyle 5\)}%
\end{pgfscope}%
\begin{pgfscope}%
\pgfsetbuttcap%
\pgfsetroundjoin%
\definecolor{currentfill}{rgb}{0.000000,0.000000,0.000000}%
\pgfsetfillcolor{currentfill}%
\pgfsetlinewidth{0.803000pt}%
\definecolor{currentstroke}{rgb}{0.000000,0.000000,0.000000}%
\pgfsetstrokecolor{currentstroke}%
\pgfsetdash{}{0pt}%
\pgfsys@defobject{currentmarker}{\pgfqpoint{0.000000in}{0.000000in}}{\pgfqpoint{0.048611in}{0.000000in}}{%
\pgfpathmoveto{\pgfqpoint{0.000000in}{0.000000in}}%
\pgfpathlineto{\pgfqpoint{0.048611in}{0.000000in}}%
\pgfusepath{stroke,fill}%
}%
\begin{pgfscope}%
\pgfsys@transformshift{5.760000in}{1.873291in}%
\pgfsys@useobject{currentmarker}{}%
\end{pgfscope}%
\end{pgfscope}%
\begin{pgfscope}%
\pgftext[x=5.857222in,y=1.825096in,left,base]{\sffamily\fontsize{10.000000}{12.000000}\selectfont \(\displaystyle 10\)}%
\end{pgfscope}%
\begin{pgfscope}%
\pgfsetbuttcap%
\pgfsetroundjoin%
\definecolor{currentfill}{rgb}{0.000000,0.000000,0.000000}%
\pgfsetfillcolor{currentfill}%
\pgfsetlinewidth{0.803000pt}%
\definecolor{currentstroke}{rgb}{0.000000,0.000000,0.000000}%
\pgfsetstrokecolor{currentstroke}%
\pgfsetdash{}{0pt}%
\pgfsys@defobject{currentmarker}{\pgfqpoint{0.000000in}{0.000000in}}{\pgfqpoint{0.048611in}{0.000000in}}{%
\pgfpathmoveto{\pgfqpoint{0.000000in}{0.000000in}}%
\pgfpathlineto{\pgfqpoint{0.048611in}{0.000000in}}%
\pgfusepath{stroke,fill}%
}%
\begin{pgfscope}%
\pgfsys@transformshift{5.760000in}{2.463212in}%
\pgfsys@useobject{currentmarker}{}%
\end{pgfscope}%
\end{pgfscope}%
\begin{pgfscope}%
\pgftext[x=5.857222in,y=2.415018in,left,base]{\sffamily\fontsize{10.000000}{12.000000}\selectfont \(\displaystyle 15\)}%
\end{pgfscope}%
\begin{pgfscope}%
\pgfsetbuttcap%
\pgfsetroundjoin%
\definecolor{currentfill}{rgb}{0.000000,0.000000,0.000000}%
\pgfsetfillcolor{currentfill}%
\pgfsetlinewidth{0.803000pt}%
\definecolor{currentstroke}{rgb}{0.000000,0.000000,0.000000}%
\pgfsetstrokecolor{currentstroke}%
\pgfsetdash{}{0pt}%
\pgfsys@defobject{currentmarker}{\pgfqpoint{0.000000in}{0.000000in}}{\pgfqpoint{0.048611in}{0.000000in}}{%
\pgfpathmoveto{\pgfqpoint{0.000000in}{0.000000in}}%
\pgfpathlineto{\pgfqpoint{0.048611in}{0.000000in}}%
\pgfusepath{stroke,fill}%
}%
\begin{pgfscope}%
\pgfsys@transformshift{5.760000in}{3.053134in}%
\pgfsys@useobject{currentmarker}{}%
\end{pgfscope}%
\end{pgfscope}%
\begin{pgfscope}%
\pgftext[x=5.857222in,y=3.004939in,left,base]{\sffamily\fontsize{10.000000}{12.000000}\selectfont \(\displaystyle 20\)}%
\end{pgfscope}%
\begin{pgfscope}%
\pgfsetbuttcap%
\pgfsetroundjoin%
\definecolor{currentfill}{rgb}{0.000000,0.000000,0.000000}%
\pgfsetfillcolor{currentfill}%
\pgfsetlinewidth{0.803000pt}%
\definecolor{currentstroke}{rgb}{0.000000,0.000000,0.000000}%
\pgfsetstrokecolor{currentstroke}%
\pgfsetdash{}{0pt}%
\pgfsys@defobject{currentmarker}{\pgfqpoint{0.000000in}{0.000000in}}{\pgfqpoint{0.048611in}{0.000000in}}{%
\pgfpathmoveto{\pgfqpoint{0.000000in}{0.000000in}}%
\pgfpathlineto{\pgfqpoint{0.048611in}{0.000000in}}%
\pgfusepath{stroke,fill}%
}%
\begin{pgfscope}%
\pgfsys@transformshift{5.760000in}{3.643055in}%
\pgfsys@useobject{currentmarker}{}%
\end{pgfscope}%
\end{pgfscope}%
\begin{pgfscope}%
\pgftext[x=5.857222in,y=3.594861in,left,base]{\sffamily\fontsize{10.000000}{12.000000}\selectfont \(\displaystyle 25\)}%
\end{pgfscope}%
\begin{pgfscope}%
\pgftext[x=6.051667in,y=2.376000in,,top,rotate=90.000000]{\sffamily\fontsize{10.000000}{12.000000}\selectfont Inclination, degrees}%
\end{pgfscope}%
\begin{pgfscope}%
\pgfpathrectangle{\pgfqpoint{0.800000in}{0.528000in}}{\pgfqpoint{4.960000in}{3.696000in}} %
\pgfusepath{clip}%
\pgfsetrectcap%
\pgfsetroundjoin%
\pgfsetlinewidth{1.505625pt}%
\definecolor{currentstroke}{rgb}{0.000000,0.000000,0.000000}%
\pgfsetstrokecolor{currentstroke}%
\pgfsetdash{}{0pt}%
\pgfpathmoveto{\pgfqpoint{1.025455in}{4.056000in}}%
\pgfpathlineto{\pgfqpoint{1.033048in}{4.053924in}}%
\pgfpathlineto{\pgfqpoint{1.039004in}{4.052284in}}%
\pgfpathlineto{\pgfqpoint{1.046692in}{4.050178in}}%
\pgfpathlineto{\pgfqpoint{1.055069in}{4.047854in}}%
\pgfpathlineto{\pgfqpoint{1.062772in}{4.045723in}}%
\pgfpathlineto{\pgfqpoint{1.068041in}{4.044256in}}%
\pgfpathlineto{\pgfqpoint{1.075862in}{4.042106in}}%
\pgfpathlineto{\pgfqpoint{1.097396in}{4.036080in}}%
\pgfpathlineto{\pgfqpoint{1.105181in}{4.033870in}}%
\pgfpathlineto{\pgfqpoint{1.111494in}{4.032111in}}%
\pgfpathlineto{\pgfqpoint{1.119369in}{4.029897in}}%
\pgfpathlineto{\pgfqpoint{1.123895in}{4.028624in}}%
\pgfpathlineto{\pgfqpoint{1.131720in}{4.026379in}}%
\pgfpathlineto{\pgfqpoint{1.144916in}{4.022652in}}%
\pgfpathlineto{\pgfqpoint{1.152964in}{4.020382in}}%
\pgfpathlineto{\pgfqpoint{1.161989in}{4.017763in}}%
\pgfpathlineto{\pgfqpoint{1.341820in}{3.964807in}}%
\pgfpathlineto{\pgfqpoint{1.353398in}{3.961332in}}%
\pgfpathlineto{\pgfqpoint{1.364089in}{3.958043in}}%
\pgfpathlineto{\pgfqpoint{1.371923in}{3.955632in}}%
\pgfpathlineto{\pgfqpoint{1.377355in}{3.953966in}}%
\pgfpathlineto{\pgfqpoint{1.391802in}{3.949530in}}%
\pgfpathlineto{\pgfqpoint{1.396953in}{3.947981in}}%
\pgfpathlineto{\pgfqpoint{1.405411in}{3.945373in}}%
\pgfpathlineto{\pgfqpoint{1.444815in}{3.933067in}}%
\pgfpathlineto{\pgfqpoint{1.449919in}{3.931490in}}%
\pgfpathlineto{\pgfqpoint{1.467280in}{3.925969in}}%
\pgfpathlineto{\pgfqpoint{1.472021in}{3.924454in}}%
\pgfpathlineto{\pgfqpoint{1.720258in}{3.842462in}}%
\pgfpathlineto{\pgfqpoint{1.739793in}{3.835751in}}%
\pgfpathlineto{\pgfqpoint{1.743872in}{3.834306in}}%
\pgfpathlineto{\pgfqpoint{1.752359in}{3.831399in}}%
\pgfpathlineto{\pgfqpoint{1.758548in}{3.829223in}}%
\pgfpathlineto{\pgfqpoint{1.764912in}{3.827007in}}%
\pgfpathlineto{\pgfqpoint{1.769752in}{3.825342in}}%
\pgfpathlineto{\pgfqpoint{1.786944in}{3.819289in}}%
\pgfpathlineto{\pgfqpoint{1.796722in}{3.815801in}}%
\pgfpathlineto{\pgfqpoint{1.803166in}{3.813526in}}%
\pgfpathlineto{\pgfqpoint{1.807219in}{3.812089in}}%
\pgfpathlineto{\pgfqpoint{1.813653in}{3.809773in}}%
\pgfpathlineto{\pgfqpoint{1.817792in}{3.808332in}}%
\pgfpathlineto{\pgfqpoint{1.824250in}{3.805986in}}%
\pgfpathlineto{\pgfqpoint{1.828478in}{3.804511in}}%
\pgfpathlineto{\pgfqpoint{1.834914in}{3.802164in}}%
\pgfpathlineto{\pgfqpoint{1.840154in}{3.800362in}}%
\pgfpathlineto{\pgfqpoint{1.851773in}{3.796126in}}%
\pgfpathlineto{\pgfqpoint{1.854965in}{3.794955in}}%
\pgfpathlineto{\pgfqpoint{1.863911in}{3.791765in}}%
\pgfpathlineto{\pgfqpoint{1.868946in}{3.789970in}}%
\pgfpathlineto{\pgfqpoint{2.170434in}{3.674343in}}%
\pgfpathlineto{\pgfqpoint{2.182390in}{3.669556in}}%
\pgfpathlineto{\pgfqpoint{2.185987in}{3.668076in}}%
\pgfpathlineto{\pgfqpoint{2.192180in}{3.665499in}}%
\pgfpathlineto{\pgfqpoint{2.202013in}{3.661510in}}%
\pgfpathlineto{\pgfqpoint{2.209551in}{3.658390in}}%
\pgfpathlineto{\pgfqpoint{2.231235in}{3.649500in}}%
\pgfpathlineto{\pgfqpoint{2.237590in}{3.646825in}}%
\pgfpathlineto{\pgfqpoint{2.244335in}{3.644121in}}%
\pgfpathlineto{\pgfqpoint{2.250754in}{3.641354in}}%
\pgfpathlineto{\pgfqpoint{2.267773in}{3.634321in}}%
\pgfpathlineto{\pgfqpoint{2.271775in}{3.632558in}}%
\pgfpathlineto{\pgfqpoint{2.287591in}{3.626009in}}%
\pgfpathlineto{\pgfqpoint{2.297292in}{3.621859in}}%
\pgfpathlineto{\pgfqpoint{2.302740in}{3.619547in}}%
\pgfpathlineto{\pgfqpoint{2.306389in}{3.618023in}}%
\pgfpathlineto{\pgfqpoint{2.312124in}{3.615593in}}%
\pgfpathlineto{\pgfqpoint{2.320331in}{3.612069in}}%
\pgfpathlineto{\pgfqpoint{2.325958in}{3.609643in}}%
\pgfpathlineto{\pgfqpoint{2.329644in}{3.608084in}}%
\pgfpathlineto{\pgfqpoint{2.335355in}{3.605619in}}%
\pgfpathlineto{\pgfqpoint{2.339378in}{3.603818in}}%
\pgfpathlineto{\pgfqpoint{2.346436in}{3.600867in}}%
\pgfpathlineto{\pgfqpoint{2.353578in}{3.597679in}}%
\pgfpathlineto{\pgfqpoint{2.361836in}{3.594075in}}%
\pgfpathlineto{\pgfqpoint{2.369092in}{3.591027in}}%
\pgfpathlineto{\pgfqpoint{2.374920in}{3.588470in}}%
\pgfpathlineto{\pgfqpoint{2.378828in}{3.586746in}}%
\pgfpathlineto{\pgfqpoint{2.384321in}{3.584278in}}%
\pgfpathlineto{\pgfqpoint{2.388347in}{3.582607in}}%
\pgfpathlineto{\pgfqpoint{2.394180in}{3.580008in}}%
\pgfpathlineto{\pgfqpoint{2.396783in}{3.578816in}}%
\pgfpathlineto{\pgfqpoint{2.403920in}{3.575717in}}%
\pgfpathlineto{\pgfqpoint{2.406546in}{3.574507in}}%
\pgfpathlineto{\pgfqpoint{2.413765in}{3.571384in}}%
\pgfpathlineto{\pgfqpoint{2.417638in}{3.569680in}}%
\pgfpathlineto{\pgfqpoint{2.423408in}{3.567028in}}%
\pgfpathlineto{\pgfqpoint{2.426985in}{3.565535in}}%
\pgfpathlineto{\pgfqpoint{2.432699in}{3.562919in}}%
\pgfpathlineto{\pgfqpoint{2.438435in}{3.560380in}}%
\pgfpathlineto{\pgfqpoint{2.441092in}{3.559143in}}%
\pgfpathlineto{\pgfqpoint{2.448245in}{3.555923in}}%
\pgfpathlineto{\pgfqpoint{2.452161in}{3.554298in}}%
\pgfpathlineto{\pgfqpoint{2.466862in}{3.547586in}}%
\pgfpathlineto{\pgfqpoint{2.469366in}{3.546468in}}%
\pgfpathlineto{\pgfqpoint{2.474934in}{3.543855in}}%
\pgfpathlineto{\pgfqpoint{2.502116in}{3.531472in}}%
\pgfpathlineto{\pgfqpoint{2.513301in}{3.526481in}}%
\pgfpathlineto{\pgfqpoint{2.533110in}{3.517241in}}%
\pgfpathlineto{\pgfqpoint{2.538794in}{3.514526in}}%
\pgfpathlineto{\pgfqpoint{2.581262in}{3.494655in}}%
\pgfpathlineto{\pgfqpoint{2.587543in}{3.491793in}}%
\pgfpathlineto{\pgfqpoint{2.593474in}{3.488984in}}%
\pgfpathlineto{\pgfqpoint{2.599909in}{3.485795in}}%
\pgfpathlineto{\pgfqpoint{2.603278in}{3.484167in}}%
\pgfpathlineto{\pgfqpoint{2.708606in}{3.432916in}}%
\pgfpathlineto{\pgfqpoint{2.719029in}{3.427836in}}%
\pgfpathlineto{\pgfqpoint{2.721947in}{3.426346in}}%
\pgfpathlineto{\pgfqpoint{2.726865in}{3.423921in}}%
\pgfpathlineto{\pgfqpoint{2.729851in}{3.422377in}}%
\pgfpathlineto{\pgfqpoint{2.734861in}{3.419958in}}%
\pgfpathlineto{\pgfqpoint{2.739841in}{3.417330in}}%
\pgfpathlineto{\pgfqpoint{2.746370in}{3.414023in}}%
\pgfpathlineto{\pgfqpoint{2.749684in}{3.412390in}}%
\pgfpathlineto{\pgfqpoint{2.754735in}{3.409950in}}%
\pgfpathlineto{\pgfqpoint{2.760026in}{3.407090in}}%
\pgfpathlineto{\pgfqpoint{2.867580in}{3.351356in}}%
\pgfpathlineto{\pgfqpoint{3.028887in}{3.263131in}}%
\pgfpathlineto{\pgfqpoint{3.034623in}{3.260039in}}%
\pgfpathlineto{\pgfqpoint{3.037929in}{3.258138in}}%
\pgfpathlineto{\pgfqpoint{3.042173in}{3.255831in}}%
\pgfpathlineto{\pgfqpoint{3.055968in}{3.247843in}}%
\pgfpathlineto{\pgfqpoint{3.058652in}{3.246413in}}%
\pgfpathlineto{\pgfqpoint{3.066822in}{3.241587in}}%
\pgfpathlineto{\pgfqpoint{3.071130in}{3.239259in}}%
\pgfpathlineto{\pgfqpoint{3.077727in}{3.235362in}}%
\pgfpathlineto{\pgfqpoint{3.080613in}{3.233812in}}%
\pgfpathlineto{\pgfqpoint{3.107149in}{3.218277in}}%
\pgfpathlineto{\pgfqpoint{3.109985in}{3.216739in}}%
\pgfpathlineto{\pgfqpoint{3.128540in}{3.205627in}}%
\pgfpathlineto{\pgfqpoint{3.134350in}{3.202175in}}%
\pgfpathlineto{\pgfqpoint{3.137445in}{3.200587in}}%
\pgfpathlineto{\pgfqpoint{3.146262in}{3.195128in}}%
\pgfpathlineto{\pgfqpoint{3.152296in}{3.191634in}}%
\pgfpathlineto{\pgfqpoint{3.153840in}{3.190637in}}%
\pgfpathlineto{\pgfqpoint{3.159708in}{3.187107in}}%
\pgfpathlineto{\pgfqpoint{3.163128in}{3.185341in}}%
\pgfpathlineto{\pgfqpoint{3.177951in}{3.176275in}}%
\pgfpathlineto{\pgfqpoint{3.179526in}{3.175262in}}%
\pgfpathlineto{\pgfqpoint{3.185302in}{3.171704in}}%
\pgfpathlineto{\pgfqpoint{3.192193in}{3.167731in}}%
\pgfpathlineto{\pgfqpoint{3.196365in}{3.165257in}}%
\pgfpathlineto{\pgfqpoint{3.199853in}{3.163151in}}%
\pgfpathlineto{\pgfqpoint{3.204335in}{3.160454in}}%
\pgfpathlineto{\pgfqpoint{3.208479in}{3.157677in}}%
\pgfpathlineto{\pgfqpoint{3.217186in}{3.152442in}}%
\pgfpathlineto{\pgfqpoint{3.219079in}{3.151200in}}%
\pgfpathlineto{\pgfqpoint{3.227901in}{3.145910in}}%
\pgfpathlineto{\pgfqpoint{3.229533in}{3.144842in}}%
\pgfpathlineto{\pgfqpoint{3.235270in}{3.141230in}}%
\pgfpathlineto{\pgfqpoint{3.383860in}{3.046821in}}%
\pgfpathlineto{\pgfqpoint{3.388164in}{3.044123in}}%
\pgfpathlineto{\pgfqpoint{3.390100in}{3.042624in}}%
\pgfpathlineto{\pgfqpoint{3.397452in}{3.038143in}}%
\pgfpathlineto{\pgfqpoint{3.410865in}{3.028921in}}%
\pgfpathlineto{\pgfqpoint{3.417593in}{3.024428in}}%
\pgfpathlineto{\pgfqpoint{3.420297in}{3.022583in}}%
\pgfpathlineto{\pgfqpoint{3.434887in}{3.012941in}}%
\pgfpathlineto{\pgfqpoint{3.439241in}{3.010135in}}%
\pgfpathlineto{\pgfqpoint{3.440959in}{3.008875in}}%
\pgfpathlineto{\pgfqpoint{3.445376in}{3.006010in}}%
\pgfpathlineto{\pgfqpoint{3.447140in}{3.004695in}}%
\pgfpathlineto{\pgfqpoint{3.451607in}{3.001852in}}%
\pgfpathlineto{\pgfqpoint{3.453624in}{3.000223in}}%
\pgfpathlineto{\pgfqpoint{3.463867in}{2.993473in}}%
\pgfpathlineto{\pgfqpoint{3.465456in}{2.992378in}}%
\pgfpathlineto{\pgfqpoint{3.469964in}{2.989246in}}%
\pgfpathlineto{\pgfqpoint{3.471778in}{2.988040in}}%
\pgfpathlineto{\pgfqpoint{3.476228in}{2.984980in}}%
\pgfpathlineto{\pgfqpoint{3.478016in}{2.983789in}}%
\pgfpathlineto{\pgfqpoint{3.482488in}{2.980689in}}%
\pgfpathlineto{\pgfqpoint{3.484254in}{2.979537in}}%
\pgfpathlineto{\pgfqpoint{3.488644in}{2.976398in}}%
\pgfpathlineto{\pgfqpoint{3.490811in}{2.974945in}}%
\pgfpathlineto{\pgfqpoint{3.495314in}{2.972002in}}%
\pgfpathlineto{\pgfqpoint{3.497155in}{2.970574in}}%
\pgfpathlineto{\pgfqpoint{3.501652in}{2.967620in}}%
\pgfpathlineto{\pgfqpoint{3.503489in}{2.966213in}}%
\pgfpathlineto{\pgfqpoint{3.508025in}{2.963207in}}%
\pgfpathlineto{\pgfqpoint{3.509583in}{2.962107in}}%
\pgfpathlineto{\pgfqpoint{3.514044in}{2.958831in}}%
\pgfpathlineto{\pgfqpoint{3.516363in}{2.957265in}}%
\pgfpathlineto{\pgfqpoint{3.520871in}{2.954289in}}%
\pgfpathlineto{\pgfqpoint{3.522455in}{2.953157in}}%
\pgfpathlineto{\pgfqpoint{3.527015in}{2.949825in}}%
\pgfpathlineto{\pgfqpoint{3.529177in}{2.948368in}}%
\pgfpathlineto{\pgfqpoint{3.533786in}{2.945245in}}%
\pgfpathlineto{\pgfqpoint{3.535369in}{2.944152in}}%
\pgfpathlineto{\pgfqpoint{3.539937in}{2.940745in}}%
\pgfpathlineto{\pgfqpoint{3.542433in}{2.938999in}}%
\pgfpathlineto{\pgfqpoint{3.547368in}{2.935931in}}%
\pgfpathlineto{\pgfqpoint{3.559101in}{2.927186in}}%
\pgfpathlineto{\pgfqpoint{3.576907in}{2.914909in}}%
\pgfpathlineto{\pgfqpoint{3.579100in}{2.912954in}}%
\pgfpathlineto{\pgfqpoint{3.611004in}{2.890489in}}%
\pgfpathlineto{\pgfqpoint{3.622636in}{2.881737in}}%
\pgfpathlineto{\pgfqpoint{3.627106in}{2.878295in}}%
\pgfpathlineto{\pgfqpoint{3.632483in}{2.874764in}}%
\pgfpathlineto{\pgfqpoint{3.637653in}{2.870615in}}%
\pgfpathlineto{\pgfqpoint{3.640332in}{2.868672in}}%
\pgfpathlineto{\pgfqpoint{3.645317in}{2.865364in}}%
\pgfpathlineto{\pgfqpoint{3.647355in}{2.863499in}}%
\pgfpathlineto{\pgfqpoint{3.652335in}{2.860199in}}%
\pgfpathlineto{\pgfqpoint{3.654383in}{2.858316in}}%
\pgfpathlineto{\pgfqpoint{3.659445in}{2.854975in}}%
\pgfpathlineto{\pgfqpoint{3.663805in}{2.851779in}}%
\pgfpathlineto{\pgfqpoint{3.669481in}{2.847185in}}%
\pgfpathlineto{\pgfqpoint{3.671378in}{2.845990in}}%
\pgfpathlineto{\pgfqpoint{3.676132in}{2.842113in}}%
\pgfpathlineto{\pgfqpoint{3.685764in}{2.835223in}}%
\pgfpathlineto{\pgfqpoint{3.690332in}{2.831493in}}%
\pgfpathlineto{\pgfqpoint{3.733612in}{2.798859in}}%
\pgfpathlineto{\pgfqpoint{3.738381in}{2.795098in}}%
\pgfpathlineto{\pgfqpoint{3.741182in}{2.793012in}}%
\pgfpathlineto{\pgfqpoint{3.746285in}{2.789258in}}%
\pgfpathlineto{\pgfqpoint{3.747826in}{2.788250in}}%
\pgfpathlineto{\pgfqpoint{3.752843in}{2.783936in}}%
\pgfpathlineto{\pgfqpoint{3.869786in}{2.691377in}}%
\pgfpathlineto{\pgfqpoint{3.875403in}{2.687271in}}%
\pgfpathlineto{\pgfqpoint{3.877165in}{2.685676in}}%
\pgfpathlineto{\pgfqpoint{3.880285in}{2.683461in}}%
\pgfpathlineto{\pgfqpoint{3.887950in}{2.677144in}}%
\pgfpathlineto{\pgfqpoint{3.893372in}{2.672519in}}%
\pgfpathlineto{\pgfqpoint{3.896760in}{2.669969in}}%
\pgfpathlineto{\pgfqpoint{3.904358in}{2.663609in}}%
\pgfpathlineto{\pgfqpoint{3.906477in}{2.661424in}}%
\pgfpathlineto{\pgfqpoint{3.911610in}{2.657024in}}%
\pgfpathlineto{\pgfqpoint{3.915608in}{2.653693in}}%
\pgfpathlineto{\pgfqpoint{3.927067in}{2.644507in}}%
\pgfpathlineto{\pgfqpoint{3.930146in}{2.642332in}}%
\pgfpathlineto{\pgfqpoint{3.941889in}{2.632153in}}%
\pgfpathlineto{\pgfqpoint{3.943516in}{2.630990in}}%
\pgfpathlineto{\pgfqpoint{3.947244in}{2.628007in}}%
\pgfpathlineto{\pgfqpoint{3.955188in}{2.621234in}}%
\pgfpathlineto{\pgfqpoint{3.957382in}{2.618859in}}%
\pgfpathlineto{\pgfqpoint{3.962794in}{2.614189in}}%
\pgfpathlineto{\pgfqpoint{3.965635in}{2.612045in}}%
\pgfpathlineto{\pgfqpoint{3.968575in}{2.609996in}}%
\pgfpathlineto{\pgfqpoint{3.994140in}{2.587920in}}%
\pgfpathlineto{\pgfqpoint{3.996046in}{2.586191in}}%
\pgfpathlineto{\pgfqpoint{3.998933in}{2.583985in}}%
\pgfpathlineto{\pgfqpoint{4.001522in}{2.581053in}}%
\pgfpathlineto{\pgfqpoint{4.011796in}{2.572604in}}%
\pgfpathlineto{\pgfqpoint{4.013397in}{2.571418in}}%
\pgfpathlineto{\pgfqpoint{4.017129in}{2.568331in}}%
\pgfpathlineto{\pgfqpoint{4.043572in}{2.544998in}}%
\pgfpathlineto{\pgfqpoint{4.045544in}{2.543148in}}%
\pgfpathlineto{\pgfqpoint{4.048343in}{2.540919in}}%
\pgfpathlineto{\pgfqpoint{4.050329in}{2.538835in}}%
\pgfpathlineto{\pgfqpoint{4.053295in}{2.536678in}}%
\pgfpathlineto{\pgfqpoint{4.064896in}{2.525647in}}%
\pgfpathlineto{\pgfqpoint{4.071077in}{2.520569in}}%
\pgfpathlineto{\pgfqpoint{4.072688in}{2.519392in}}%
\pgfpathlineto{\pgfqpoint{4.076548in}{2.516076in}}%
\pgfpathlineto{\pgfqpoint{4.140974in}{2.456966in}}%
\pgfpathlineto{\pgfqpoint{4.147307in}{2.451427in}}%
\pgfpathlineto{\pgfqpoint{4.148948in}{2.450438in}}%
\pgfpathlineto{\pgfqpoint{4.156168in}{2.442882in}}%
\pgfpathlineto{\pgfqpoint{4.169762in}{2.430704in}}%
\pgfpathlineto{\pgfqpoint{4.172727in}{2.428383in}}%
\pgfpathlineto{\pgfqpoint{4.174896in}{2.425845in}}%
\pgfpathlineto{\pgfqpoint{4.177668in}{2.423755in}}%
\pgfpathlineto{\pgfqpoint{4.179841in}{2.421259in}}%
\pgfpathlineto{\pgfqpoint{4.182610in}{2.419104in}}%
\pgfpathlineto{\pgfqpoint{4.184786in}{2.416673in}}%
\pgfpathlineto{\pgfqpoint{4.187404in}{2.414469in}}%
\pgfpathlineto{\pgfqpoint{4.189210in}{2.412842in}}%
\pgfpathlineto{\pgfqpoint{4.192504in}{2.409721in}}%
\pgfpathlineto{\pgfqpoint{4.194317in}{2.408020in}}%
\pgfpathlineto{\pgfqpoint{4.197577in}{2.404957in}}%
\pgfpathlineto{\pgfqpoint{4.199397in}{2.403233in}}%
\pgfpathlineto{\pgfqpoint{4.202863in}{2.400094in}}%
\pgfpathlineto{\pgfqpoint{4.205071in}{2.397570in}}%
\pgfpathlineto{\pgfqpoint{4.208008in}{2.395255in}}%
\pgfpathlineto{\pgfqpoint{4.210226in}{2.392666in}}%
\pgfpathlineto{\pgfqpoint{4.212815in}{2.390518in}}%
\pgfpathlineto{\pgfqpoint{4.214652in}{2.388835in}}%
\pgfpathlineto{\pgfqpoint{4.218185in}{2.385569in}}%
\pgfpathlineto{\pgfqpoint{4.220414in}{2.383033in}}%
\pgfpathlineto{\pgfqpoint{4.223384in}{2.380646in}}%
\pgfpathlineto{\pgfqpoint{4.225624in}{2.378043in}}%
\pgfpathlineto{\pgfqpoint{4.228519in}{2.375731in}}%
\pgfpathlineto{\pgfqpoint{4.230764in}{2.373154in}}%
\pgfpathlineto{\pgfqpoint{4.233376in}{2.370865in}}%
\pgfpathlineto{\pgfqpoint{4.235238in}{2.369248in}}%
\pgfpathlineto{\pgfqpoint{4.238924in}{2.365776in}}%
\pgfpathlineto{\pgfqpoint{4.241186in}{2.363159in}}%
\pgfpathlineto{\pgfqpoint{4.243899in}{2.360841in}}%
\pgfpathlineto{\pgfqpoint{4.245776in}{2.359087in}}%
\pgfpathlineto{\pgfqpoint{4.249160in}{2.355788in}}%
\pgfpathlineto{\pgfqpoint{4.251044in}{2.354003in}}%
\pgfpathlineto{\pgfqpoint{4.254295in}{2.350732in}}%
\pgfpathlineto{\pgfqpoint{4.256184in}{2.349100in}}%
\pgfpathlineto{\pgfqpoint{4.259637in}{2.345613in}}%
\pgfpathlineto{\pgfqpoint{4.261533in}{2.343893in}}%
\pgfpathlineto{\pgfqpoint{4.265178in}{2.340404in}}%
\pgfpathlineto{\pgfqpoint{4.267086in}{2.338380in}}%
\pgfpathlineto{\pgfqpoint{4.270222in}{2.335307in}}%
\pgfpathlineto{\pgfqpoint{4.272132in}{2.333611in}}%
\pgfpathlineto{\pgfqpoint{4.275537in}{2.330108in}}%
\pgfpathlineto{\pgfqpoint{4.277454in}{2.328433in}}%
\pgfpathlineto{\pgfqpoint{4.281217in}{2.324787in}}%
\pgfpathlineto{\pgfqpoint{4.283147in}{2.322702in}}%
\pgfpathlineto{\pgfqpoint{4.286220in}{2.319613in}}%
\pgfpathlineto{\pgfqpoint{4.288150in}{2.317988in}}%
\pgfpathlineto{\pgfqpoint{4.292022in}{2.314209in}}%
\pgfpathlineto{\pgfqpoint{4.293967in}{2.312068in}}%
\pgfpathlineto{\pgfqpoint{4.297252in}{2.308944in}}%
\pgfpathlineto{\pgfqpoint{4.299199in}{2.307017in}}%
\pgfpathlineto{\pgfqpoint{4.302632in}{2.303595in}}%
\pgfpathlineto{\pgfqpoint{4.304585in}{2.301730in}}%
\pgfpathlineto{\pgfqpoint{4.308023in}{2.298210in}}%
\pgfpathlineto{\pgfqpoint{4.309983in}{2.296421in}}%
\pgfpathlineto{\pgfqpoint{4.313559in}{2.292768in}}%
\pgfpathlineto{\pgfqpoint{4.315527in}{2.290893in}}%
\pgfpathlineto{\pgfqpoint{4.319063in}{2.287302in}}%
\pgfpathlineto{\pgfqpoint{4.321038in}{2.285409in}}%
\pgfpathlineto{\pgfqpoint{4.324445in}{2.281824in}}%
\pgfpathlineto{\pgfqpoint{4.326027in}{2.280652in}}%
\pgfpathlineto{\pgfqpoint{4.330881in}{2.275858in}}%
\pgfpathlineto{\pgfqpoint{4.355739in}{2.249911in}}%
\pgfpathlineto{\pgfqpoint{4.358419in}{2.247973in}}%
\pgfpathlineto{\pgfqpoint{4.360454in}{2.245653in}}%
\pgfpathlineto{\pgfqpoint{4.363820in}{2.242316in}}%
\pgfpathlineto{\pgfqpoint{4.365856in}{2.240317in}}%
\pgfpathlineto{\pgfqpoint{4.369460in}{2.236540in}}%
\pgfpathlineto{\pgfqpoint{4.371091in}{2.235191in}}%
\pgfpathlineto{\pgfqpoint{4.375480in}{2.230619in}}%
\pgfpathlineto{\pgfqpoint{4.377538in}{2.228271in}}%
\pgfpathlineto{\pgfqpoint{4.380972in}{2.224841in}}%
\pgfpathlineto{\pgfqpoint{4.383032in}{2.222786in}}%
\pgfpathlineto{\pgfqpoint{4.386537in}{2.218971in}}%
\pgfpathlineto{\pgfqpoint{4.388181in}{2.217759in}}%
\pgfpathlineto{\pgfqpoint{4.392871in}{2.212859in}}%
\pgfpathlineto{\pgfqpoint{4.394958in}{2.210332in}}%
\pgfpathlineto{\pgfqpoint{4.398083in}{2.207057in}}%
\pgfpathlineto{\pgfqpoint{4.399560in}{2.206096in}}%
\pgfpathlineto{\pgfqpoint{4.409267in}{2.195068in}}%
\pgfpathlineto{\pgfqpoint{4.411638in}{2.193499in}}%
\pgfpathlineto{\pgfqpoint{4.416134in}{2.188753in}}%
\pgfpathlineto{\pgfqpoint{4.418251in}{2.186269in}}%
\pgfpathlineto{\pgfqpoint{4.421639in}{2.182736in}}%
\pgfpathlineto{\pgfqpoint{4.423329in}{2.181371in}}%
\pgfpathlineto{\pgfqpoint{4.428054in}{2.176381in}}%
\pgfpathlineto{\pgfqpoint{4.430192in}{2.173745in}}%
\pgfpathlineto{\pgfqpoint{4.433382in}{2.170346in}}%
\pgfpathlineto{\pgfqpoint{4.434865in}{2.169388in}}%
\pgfpathlineto{\pgfqpoint{4.444814in}{2.157878in}}%
\pgfpathlineto{\pgfqpoint{4.447207in}{2.156325in}}%
\pgfpathlineto{\pgfqpoint{4.451868in}{2.151296in}}%
\pgfpathlineto{\pgfqpoint{4.454037in}{2.148722in}}%
\pgfpathlineto{\pgfqpoint{4.457391in}{2.145036in}}%
\pgfpathlineto{\pgfqpoint{4.458869in}{2.144093in}}%
\pgfpathlineto{\pgfqpoint{4.469468in}{2.132120in}}%
\pgfpathlineto{\pgfqpoint{4.470837in}{2.131389in}}%
\pgfpathlineto{\pgfqpoint{4.488195in}{2.112377in}}%
\pgfpathlineto{\pgfqpoint{4.489963in}{2.110789in}}%
\pgfpathlineto{\pgfqpoint{4.494596in}{2.105649in}}%
\pgfpathlineto{\pgfqpoint{4.496826in}{2.103077in}}%
\pgfpathlineto{\pgfqpoint{4.500493in}{2.099024in}}%
\pgfpathlineto{\pgfqpoint{4.502276in}{2.097559in}}%
\pgfpathlineto{\pgfqpoint{4.506864in}{2.092244in}}%
\pgfpathlineto{\pgfqpoint{4.508655in}{2.090611in}}%
\pgfpathlineto{\pgfqpoint{4.513152in}{2.085435in}}%
\pgfpathlineto{\pgfqpoint{4.514951in}{2.083771in}}%
\pgfpathlineto{\pgfqpoint{4.519446in}{2.078583in}}%
\pgfpathlineto{\pgfqpoint{4.521253in}{2.076913in}}%
\pgfpathlineto{\pgfqpoint{4.525577in}{2.071712in}}%
\pgfpathlineto{\pgfqpoint{4.527392in}{2.070280in}}%
\pgfpathlineto{\pgfqpoint{4.532393in}{2.064633in}}%
\pgfpathlineto{\pgfqpoint{4.534682in}{2.061853in}}%
\pgfpathlineto{\pgfqpoint{4.538255in}{2.057762in}}%
\pgfpathlineto{\pgfqpoint{4.539811in}{2.056737in}}%
\pgfpathlineto{\pgfqpoint{4.550838in}{2.043626in}}%
\pgfpathlineto{\pgfqpoint{4.552907in}{2.042257in}}%
\pgfpathlineto{\pgfqpoint{4.558184in}{2.036267in}}%
\pgfpathlineto{\pgfqpoint{4.560517in}{2.033284in}}%
\pgfpathlineto{\pgfqpoint{4.564037in}{2.029243in}}%
\pgfpathlineto{\pgfqpoint{4.565540in}{2.028304in}}%
\pgfpathlineto{\pgfqpoint{4.577169in}{2.014665in}}%
\pgfpathlineto{\pgfqpoint{4.579001in}{2.013226in}}%
\pgfpathlineto{\pgfqpoint{4.584173in}{2.007189in}}%
\pgfpathlineto{\pgfqpoint{4.586063in}{2.005130in}}%
\pgfpathlineto{\pgfqpoint{4.590490in}{1.999856in}}%
\pgfpathlineto{\pgfqpoint{4.592385in}{1.998151in}}%
\pgfpathlineto{\pgfqpoint{4.597145in}{1.992374in}}%
\pgfpathlineto{\pgfqpoint{4.599048in}{1.990634in}}%
\pgfpathlineto{\pgfqpoint{4.603673in}{1.984859in}}%
\pgfpathlineto{\pgfqpoint{4.605587in}{1.983290in}}%
\pgfpathlineto{\pgfqpoint{4.610535in}{1.977243in}}%
\pgfpathlineto{\pgfqpoint{4.612456in}{1.975444in}}%
\pgfpathlineto{\pgfqpoint{4.617096in}{1.969620in}}%
\pgfpathlineto{\pgfqpoint{4.619027in}{1.968030in}}%
\pgfpathlineto{\pgfqpoint{4.624008in}{1.961891in}}%
\pgfpathlineto{\pgfqpoint{4.625946in}{1.960084in}}%
\pgfpathlineto{\pgfqpoint{4.630648in}{1.954148in}}%
\pgfpathlineto{\pgfqpoint{4.632597in}{1.952526in}}%
\pgfpathlineto{\pgfqpoint{4.637526in}{1.946317in}}%
\pgfpathlineto{\pgfqpoint{4.639482in}{1.944606in}}%
\pgfpathlineto{\pgfqpoint{4.644345in}{1.938437in}}%
\pgfpathlineto{\pgfqpoint{4.646310in}{1.936755in}}%
\pgfpathlineto{\pgfqpoint{4.651603in}{1.930392in}}%
\pgfpathlineto{\pgfqpoint{4.653580in}{1.928205in}}%
\pgfpathlineto{\pgfqpoint{4.658145in}{1.922486in}}%
\pgfpathlineto{\pgfqpoint{4.660129in}{1.920774in}}%
\pgfpathlineto{\pgfqpoint{4.665054in}{1.914421in}}%
\pgfpathlineto{\pgfqpoint{4.667042in}{1.912756in}}%
\pgfpathlineto{\pgfqpoint{4.672116in}{1.906280in}}%
\pgfpathlineto{\pgfqpoint{4.674117in}{1.904481in}}%
\pgfpathlineto{\pgfqpoint{4.679045in}{1.898095in}}%
\pgfpathlineto{\pgfqpoint{4.681054in}{1.896400in}}%
\pgfpathlineto{\pgfqpoint{4.686110in}{1.889834in}}%
\pgfpathlineto{\pgfqpoint{4.688132in}{1.888092in}}%
\pgfpathlineto{\pgfqpoint{4.693335in}{1.881486in}}%
\pgfpathlineto{\pgfqpoint{4.695365in}{1.879529in}}%
\pgfpathlineto{\pgfqpoint{4.700396in}{1.873106in}}%
\pgfpathlineto{\pgfqpoint{4.702436in}{1.871205in}}%
\pgfpathlineto{\pgfqpoint{4.707554in}{1.864647in}}%
\pgfpathlineto{\pgfqpoint{4.709603in}{1.862714in}}%
\pgfpathlineto{\pgfqpoint{4.714604in}{1.856141in}}%
\pgfpathlineto{\pgfqpoint{4.716659in}{1.854381in}}%
\pgfpathlineto{\pgfqpoint{4.721912in}{1.847537in}}%
\pgfpathlineto{\pgfqpoint{4.723982in}{1.845623in}}%
\pgfpathlineto{\pgfqpoint{4.729247in}{1.838858in}}%
\pgfpathlineto{\pgfqpoint{4.731326in}{1.836810in}}%
\pgfpathlineto{\pgfqpoint{4.736361in}{1.830155in}}%
\pgfpathlineto{\pgfqpoint{4.737926in}{1.829061in}}%
\pgfpathlineto{\pgfqpoint{4.751097in}{1.812471in}}%
\pgfpathlineto{\pgfqpoint{4.752676in}{1.811291in}}%
\pgfpathlineto{\pgfqpoint{4.765681in}{1.794521in}}%
\pgfpathlineto{\pgfqpoint{4.767105in}{1.793782in}}%
\pgfpathlineto{\pgfqpoint{4.770374in}{1.788275in}}%
\pgfpathlineto{\pgfqpoint{4.773276in}{1.785423in}}%
\pgfpathlineto{\pgfqpoint{4.774726in}{1.784516in}}%
\pgfpathlineto{\pgfqpoint{4.796627in}{1.757424in}}%
\pgfpathlineto{\pgfqpoint{4.798809in}{1.754526in}}%
\pgfpathlineto{\pgfqpoint{4.803364in}{1.748293in}}%
\pgfpathlineto{\pgfqpoint{4.804633in}{1.747716in}}%
\pgfpathlineto{\pgfqpoint{4.807398in}{1.743261in}}%
\pgfpathlineto{\pgfqpoint{4.810977in}{1.738817in}}%
\pgfpathlineto{\pgfqpoint{4.812402in}{1.738124in}}%
\pgfpathlineto{\pgfqpoint{4.815767in}{1.732409in}}%
\pgfpathlineto{\pgfqpoint{4.818789in}{1.729255in}}%
\pgfpathlineto{\pgfqpoint{4.820562in}{1.728020in}}%
\pgfpathlineto{\pgfqpoint{4.834273in}{1.709896in}}%
\pgfpathlineto{\pgfqpoint{4.835804in}{1.708977in}}%
\pgfpathlineto{\pgfqpoint{4.874454in}{1.659806in}}%
\pgfpathlineto{\pgfqpoint{4.876748in}{1.656787in}}%
\pgfpathlineto{\pgfqpoint{4.881441in}{1.649826in}}%
\pgfpathlineto{\pgfqpoint{4.883242in}{1.648900in}}%
\pgfpathlineto{\pgfqpoint{4.886772in}{1.642607in}}%
\pgfpathlineto{\pgfqpoint{4.889801in}{1.639476in}}%
\pgfpathlineto{\pgfqpoint{4.891133in}{1.638753in}}%
\pgfpathlineto{\pgfqpoint{4.894062in}{1.633760in}}%
\pgfpathlineto{\pgfqpoint{4.897618in}{1.629086in}}%
\pgfpathlineto{\pgfqpoint{4.899544in}{1.627986in}}%
\pgfpathlineto{\pgfqpoint{4.922581in}{1.597262in}}%
\pgfpathlineto{\pgfqpoint{4.924349in}{1.595857in}}%
\pgfpathlineto{\pgfqpoint{4.938745in}{1.575632in}}%
\pgfpathlineto{\pgfqpoint{4.940392in}{1.574848in}}%
\pgfpathlineto{\pgfqpoint{4.943414in}{1.569584in}}%
\pgfpathlineto{\pgfqpoint{4.947085in}{1.564657in}}%
\pgfpathlineto{\pgfqpoint{4.949225in}{1.563298in}}%
\pgfpathlineto{\pgfqpoint{4.964333in}{1.542374in}}%
\pgfpathlineto{\pgfqpoint{4.966150in}{1.540943in}}%
\pgfpathlineto{\pgfqpoint{4.981083in}{1.519736in}}%
\pgfpathlineto{\pgfqpoint{4.982658in}{1.518933in}}%
\pgfpathlineto{\pgfqpoint{4.985763in}{1.513430in}}%
\pgfpathlineto{\pgfqpoint{4.988914in}{1.508537in}}%
\pgfpathlineto{\pgfqpoint{4.992340in}{1.505922in}}%
\pgfpathlineto{\pgfqpoint{4.998683in}{1.496636in}}%
\pgfpathlineto{\pgfqpoint{5.000537in}{1.495023in}}%
\pgfpathlineto{\pgfqpoint{5.007931in}{1.484704in}}%
\pgfpathlineto{\pgfqpoint{5.010429in}{1.481208in}}%
\pgfpathlineto{\pgfqpoint{5.015540in}{1.473202in}}%
\pgfpathlineto{\pgfqpoint{5.017721in}{1.471843in}}%
\pgfpathlineto{\pgfqpoint{5.033729in}{1.449209in}}%
\pgfpathlineto{\pgfqpoint{5.035625in}{1.447482in}}%
\pgfpathlineto{\pgfqpoint{5.043170in}{1.436822in}}%
\pgfpathlineto{\pgfqpoint{5.045724in}{1.433157in}}%
\pgfpathlineto{\pgfqpoint{5.050304in}{1.425181in}}%
\pgfpathlineto{\pgfqpoint{5.053381in}{1.423170in}}%
\pgfpathlineto{\pgfqpoint{5.060983in}{1.412275in}}%
\pgfpathlineto{\pgfqpoint{5.062914in}{1.409896in}}%
\pgfpathlineto{\pgfqpoint{5.069285in}{1.400049in}}%
\pgfpathlineto{\pgfqpoint{5.070982in}{1.398883in}}%
\pgfpathlineto{\pgfqpoint{5.074958in}{1.391167in}}%
\pgfpathlineto{\pgfqpoint{5.077593in}{1.387631in}}%
\pgfpathlineto{\pgfqpoint{5.080370in}{1.385867in}}%
\pgfpathlineto{\pgfqpoint{5.096317in}{1.361954in}}%
\pgfpathlineto{\pgfqpoint{5.097887in}{1.361280in}}%
\pgfpathlineto{\pgfqpoint{5.100537in}{1.356857in}}%
\pgfpathlineto{\pgfqpoint{5.105270in}{1.349083in}}%
\pgfpathlineto{\pgfqpoint{5.107800in}{1.347522in}}%
\pgfpathlineto{\pgfqpoint{5.124782in}{1.322664in}}%
\pgfpathlineto{\pgfqpoint{5.126787in}{1.320696in}}%
\pgfpathlineto{\pgfqpoint{5.134624in}{1.309133in}}%
\pgfpathlineto{\pgfqpoint{5.136644in}{1.306569in}}%
\pgfpathlineto{\pgfqpoint{5.143318in}{1.295951in}}%
\pgfpathlineto{\pgfqpoint{5.145132in}{1.294630in}}%
\pgfpathlineto{\pgfqpoint{5.148580in}{1.287792in}}%
\pgfpathlineto{\pgfqpoint{5.152255in}{1.282452in}}%
\pgfpathlineto{\pgfqpoint{5.154687in}{1.280971in}}%
\pgfpathlineto{\pgfqpoint{5.158873in}{1.272497in}}%
\pgfpathlineto{\pgfqpoint{5.162058in}{1.268687in}}%
\pgfpathlineto{\pgfqpoint{5.163859in}{1.267734in}}%
\pgfpathlineto{\pgfqpoint{5.166622in}{1.262735in}}%
\pgfpathlineto{\pgfqpoint{5.170855in}{1.255196in}}%
\pgfpathlineto{\pgfqpoint{5.174193in}{1.252905in}}%
\pgfpathlineto{\pgfqpoint{5.182248in}{1.240660in}}%
\pgfpathlineto{\pgfqpoint{5.184325in}{1.238138in}}%
\pgfpathlineto{\pgfqpoint{5.191307in}{1.226756in}}%
\pgfpathlineto{\pgfqpoint{5.193061in}{1.225543in}}%
\pgfpathlineto{\pgfqpoint{5.196601in}{1.218519in}}%
\pgfpathlineto{\pgfqpoint{5.200180in}{1.212750in}}%
\pgfpathlineto{\pgfqpoint{5.203114in}{1.210917in}}%
\pgfpathlineto{\pgfqpoint{5.207424in}{1.201969in}}%
\pgfpathlineto{\pgfqpoint{5.210488in}{1.198171in}}%
\pgfpathlineto{\pgfqpoint{5.212470in}{1.197147in}}%
\pgfpathlineto{\pgfqpoint{5.215314in}{1.191905in}}%
\pgfpathlineto{\pgfqpoint{5.219671in}{1.183991in}}%
\pgfpathlineto{\pgfqpoint{5.223457in}{1.180999in}}%
\pgfpathlineto{\pgfqpoint{5.230908in}{1.168922in}}%
\pgfpathlineto{\pgfqpoint{5.233044in}{1.166932in}}%
\pgfpathlineto{\pgfqpoint{5.241575in}{1.153820in}}%
\pgfpathlineto{\pgfqpoint{5.243727in}{1.150927in}}%
\pgfpathlineto{\pgfqpoint{5.250699in}{1.139196in}}%
\pgfpathlineto{\pgfqpoint{5.252228in}{1.138346in}}%
\pgfpathlineto{\pgfqpoint{5.255134in}{1.133098in}}%
\pgfpathlineto{\pgfqpoint{5.259593in}{1.124622in}}%
\pgfpathlineto{\pgfqpoint{5.263420in}{1.121751in}}%
\pgfpathlineto{\pgfqpoint{5.271509in}{1.108749in}}%
\pgfpathlineto{\pgfqpoint{5.273695in}{1.106339in}}%
\pgfpathlineto{\pgfqpoint{5.281386in}{1.093448in}}%
\pgfpathlineto{\pgfqpoint{5.283589in}{1.091511in}}%
\pgfpathlineto{\pgfqpoint{5.293208in}{1.076891in}}%
\pgfpathlineto{\pgfqpoint{5.296185in}{1.071286in}}%
\pgfpathlineto{\pgfqpoint{5.300750in}{1.062654in}}%
\pgfpathlineto{\pgfqpoint{5.304572in}{1.059725in}}%
\pgfpathlineto{\pgfqpoint{5.312528in}{1.046339in}}%
\pgfpathlineto{\pgfqpoint{5.314766in}{1.044246in}}%
\pgfpathlineto{\pgfqpoint{5.323732in}{1.030009in}}%
\pgfpathlineto{\pgfqpoint{5.325986in}{1.026910in}}%
\pgfpathlineto{\pgfqpoint{5.333210in}{1.014188in}}%
\pgfpathlineto{\pgfqpoint{5.335091in}{1.013058in}}%
\pgfpathlineto{\pgfqpoint{5.338138in}{1.007175in}}%
\pgfpathlineto{\pgfqpoint{5.342811in}{0.998272in}}%
\pgfpathlineto{\pgfqpoint{5.346712in}{0.995201in}}%
\pgfpathlineto{\pgfqpoint{5.354907in}{0.981326in}}%
\pgfpathlineto{\pgfqpoint{5.357197in}{0.979026in}}%
\pgfpathlineto{\pgfqpoint{5.366324in}{0.964335in}}%
\pgfpathlineto{\pgfqpoint{5.368632in}{0.961075in}}%
\pgfpathlineto{\pgfqpoint{5.375956in}{0.947910in}}%
\pgfpathlineto{\pgfqpoint{5.377881in}{0.946776in}}%
\pgfpathlineto{\pgfqpoint{5.380998in}{0.940723in}}%
\pgfpathlineto{\pgfqpoint{5.385781in}{0.931384in}}%
\pgfpathlineto{\pgfqpoint{5.389419in}{0.928836in}}%
\pgfpathlineto{\pgfqpoint{5.394204in}{0.918127in}}%
\pgfpathlineto{\pgfqpoint{5.397372in}{0.913850in}}%
\pgfpathlineto{\pgfqpoint{5.400030in}{0.912155in}}%
\pgfpathlineto{\pgfqpoint{5.404001in}{0.903571in}}%
\pgfpathlineto{\pgfqpoint{5.408029in}{0.896677in}}%
\pgfpathlineto{\pgfqpoint{5.410586in}{0.895392in}}%
\pgfpathlineto{\pgfqpoint{5.413754in}{0.889288in}}%
\pgfpathlineto{\pgfqpoint{5.418621in}{0.879515in}}%
\pgfpathlineto{\pgfqpoint{5.422374in}{0.876881in}}%
\pgfpathlineto{\pgfqpoint{5.426406in}{0.867836in}}%
\pgfpathlineto{\pgfqpoint{5.429663in}{0.861916in}}%
\pgfpathlineto{\pgfqpoint{5.434217in}{0.857934in}}%
\pgfpathlineto{\pgfqpoint{5.442280in}{0.843488in}}%
\pgfpathlineto{\pgfqpoint{5.444081in}{0.842221in}}%
\pgfpathlineto{\pgfqpoint{5.447303in}{0.835807in}}%
\pgfpathlineto{\pgfqpoint{5.452254in}{0.825921in}}%
\pgfpathlineto{\pgfqpoint{5.456150in}{0.822988in}}%
\pgfpathlineto{\pgfqpoint{5.465288in}{0.807180in}}%
\pgfpathlineto{\pgfqpoint{5.467710in}{0.804366in}}%
\pgfpathlineto{\pgfqpoint{5.476421in}{0.788921in}}%
\pgfpathlineto{\pgfqpoint{5.478857in}{0.786407in}}%
\pgfpathlineto{\pgfqpoint{5.488743in}{0.769877in}}%
\pgfpathlineto{\pgfqpoint{5.491192in}{0.766177in}}%
\pgfpathlineto{\pgfqpoint{5.498742in}{0.751770in}}%
\pgfpathlineto{\pgfqpoint{5.500177in}{0.751236in}}%
\pgfpathlineto{\pgfqpoint{5.502634in}{0.747601in}}%
\pgfpathlineto{\pgfqpoint{5.510493in}{0.732978in}}%
\pgfpathlineto{\pgfqpoint{5.511541in}{0.732560in}}%
\pgfpathlineto{\pgfqpoint{5.514000in}{0.729125in}}%
\pgfpathlineto{\pgfqpoint{5.522196in}{0.714267in}}%
\pgfpathlineto{\pgfqpoint{5.522272in}{0.714314in}}%
\pgfpathlineto{\pgfqpoint{5.522541in}{0.714271in}}%
\pgfpathlineto{\pgfqpoint{5.525770in}{0.709825in}}%
\pgfpathlineto{\pgfqpoint{5.531836in}{0.696000in}}%
\pgfpathlineto{\pgfqpoint{5.532831in}{0.696687in}}%
\pgfpathlineto{\pgfqpoint{5.534105in}{0.699700in}}%
\pgfpathlineto{\pgfqpoint{5.534545in}{0.698563in}}%
\pgfusepath{stroke}%
\end{pgfscope}%
\begin{pgfscope}%
\pgfsetrectcap%
\pgfsetmiterjoin%
\pgfsetlinewidth{0.803000pt}%
\definecolor{currentstroke}{rgb}{0.000000,0.000000,0.000000}%
\pgfsetstrokecolor{currentstroke}%
\pgfsetdash{}{0pt}%
\pgfpathmoveto{\pgfqpoint{0.800000in}{0.528000in}}%
\pgfpathlineto{\pgfqpoint{0.800000in}{4.224000in}}%
\pgfusepath{stroke}%
\end{pgfscope}%
\begin{pgfscope}%
\pgfsetrectcap%
\pgfsetmiterjoin%
\pgfsetlinewidth{0.803000pt}%
\definecolor{currentstroke}{rgb}{0.000000,0.000000,0.000000}%
\pgfsetstrokecolor{currentstroke}%
\pgfsetdash{}{0pt}%
\pgfpathmoveto{\pgfqpoint{5.760000in}{0.528000in}}%
\pgfpathlineto{\pgfqpoint{5.760000in}{4.224000in}}%
\pgfusepath{stroke}%
\end{pgfscope}%
\begin{pgfscope}%
\pgfsetrectcap%
\pgfsetmiterjoin%
\pgfsetlinewidth{0.803000pt}%
\definecolor{currentstroke}{rgb}{0.000000,0.000000,0.000000}%
\pgfsetstrokecolor{currentstroke}%
\pgfsetdash{}{0pt}%
\pgfpathmoveto{\pgfqpoint{0.800000in}{0.528000in}}%
\pgfpathlineto{\pgfqpoint{5.760000in}{0.528000in}}%
\pgfusepath{stroke}%
\end{pgfscope}%
\begin{pgfscope}%
\pgfsetrectcap%
\pgfsetmiterjoin%
\pgfsetlinewidth{0.803000pt}%
\definecolor{currentstroke}{rgb}{0.000000,0.000000,0.000000}%
\pgfsetstrokecolor{currentstroke}%
\pgfsetdash{}{0pt}%
\pgfpathmoveto{\pgfqpoint{0.800000in}{4.224000in}}%
\pgfpathlineto{\pgfqpoint{5.760000in}{4.224000in}}%
\pgfusepath{stroke}%
\end{pgfscope}%
\end{pgfpicture}%
\makeatother%
\endgroup%

}
\end{subfigure}
\caption{Evolution of the combined $a$ and $i$ transfer orbit for the case $i_0 = 28.5~\text{deg}$}
\label{fig:aincnumres28}
\end{figure}

\begin{figure}%[h]
\begin{subfigure}[b]{0.5\textwidth}
\centering
\resizebox{1.0\textwidth}{!}{
%% Creator: Matplotlib, PGF backend
%%
%% To include the figure in your LaTeX document, write
%%   \input{<filename>.pgf}
%%
%% Make sure the required packages are loaded in your preamble
%%   \usepackage{pgf}
%%
%% Figures using additional raster images can only be included by \input if
%% they are in the same directory as the main LaTeX file. For loading figures
%% from other directories you can use the `import` package
%%   \usepackage{import}
%% and then include the figures with
%%   \import{<path to file>}{<filename>.pgf}
%%
%% Matplotlib used the following preamble
%%   \usepackage{fontspec}
%%
\begingroup%
\makeatletter%
\begin{pgfpicture}%
\pgfpathrectangle{\pgfpointorigin}{\pgfqpoint{6.400000in}{4.800000in}}%
\pgfusepath{use as bounding box, clip}%
\begin{pgfscope}%
\pgfsetbuttcap%
\pgfsetmiterjoin%
\definecolor{currentfill}{rgb}{1.000000,1.000000,1.000000}%
\pgfsetfillcolor{currentfill}%
\pgfsetlinewidth{0.000000pt}%
\definecolor{currentstroke}{rgb}{1.000000,1.000000,1.000000}%
\pgfsetstrokecolor{currentstroke}%
\pgfsetdash{}{0pt}%
\pgfpathmoveto{\pgfqpoint{0.000000in}{0.000000in}}%
\pgfpathlineto{\pgfqpoint{6.400000in}{0.000000in}}%
\pgfpathlineto{\pgfqpoint{6.400000in}{4.800000in}}%
\pgfpathlineto{\pgfqpoint{0.000000in}{4.800000in}}%
\pgfpathclose%
\pgfusepath{fill}%
\end{pgfscope}%
\begin{pgfscope}%
\pgfsetbuttcap%
\pgfsetmiterjoin%
\definecolor{currentfill}{rgb}{1.000000,1.000000,1.000000}%
\pgfsetfillcolor{currentfill}%
\pgfsetlinewidth{0.000000pt}%
\definecolor{currentstroke}{rgb}{0.000000,0.000000,0.000000}%
\pgfsetstrokecolor{currentstroke}%
\pgfsetstrokeopacity{0.000000}%
\pgfsetdash{}{0pt}%
\pgfpathmoveto{\pgfqpoint{0.800000in}{0.528000in}}%
\pgfpathlineto{\pgfqpoint{5.760000in}{0.528000in}}%
\pgfpathlineto{\pgfqpoint{5.760000in}{4.224000in}}%
\pgfpathlineto{\pgfqpoint{0.800000in}{4.224000in}}%
\pgfpathclose%
\pgfusepath{fill}%
\end{pgfscope}%
\begin{pgfscope}%
\pgfsetbuttcap%
\pgfsetroundjoin%
\definecolor{currentfill}{rgb}{0.000000,0.000000,0.000000}%
\pgfsetfillcolor{currentfill}%
\pgfsetlinewidth{0.803000pt}%
\definecolor{currentstroke}{rgb}{0.000000,0.000000,0.000000}%
\pgfsetstrokecolor{currentstroke}%
\pgfsetdash{}{0pt}%
\pgfsys@defobject{currentmarker}{\pgfqpoint{0.000000in}{-0.048611in}}{\pgfqpoint{0.000000in}{0.000000in}}{%
\pgfpathmoveto{\pgfqpoint{0.000000in}{0.000000in}}%
\pgfpathlineto{\pgfqpoint{0.000000in}{-0.048611in}}%
\pgfusepath{stroke,fill}%
}%
\begin{pgfscope}%
\pgfsys@transformshift{1.025455in}{0.528000in}%
\pgfsys@useobject{currentmarker}{}%
\end{pgfscope}%
\end{pgfscope}%
\begin{pgfscope}%
\pgftext[x=1.025455in,y=0.430778in,,top]{\sffamily\fontsize{10.000000}{12.000000}\selectfont \(\displaystyle 0\)}%
\end{pgfscope}%
\begin{pgfscope}%
\pgfsetbuttcap%
\pgfsetroundjoin%
\definecolor{currentfill}{rgb}{0.000000,0.000000,0.000000}%
\pgfsetfillcolor{currentfill}%
\pgfsetlinewidth{0.803000pt}%
\definecolor{currentstroke}{rgb}{0.000000,0.000000,0.000000}%
\pgfsetstrokecolor{currentstroke}%
\pgfsetdash{}{0pt}%
\pgfsys@defobject{currentmarker}{\pgfqpoint{0.000000in}{-0.048611in}}{\pgfqpoint{0.000000in}{0.000000in}}{%
\pgfpathmoveto{\pgfqpoint{0.000000in}{0.000000in}}%
\pgfpathlineto{\pgfqpoint{0.000000in}{-0.048611in}}%
\pgfusepath{stroke,fill}%
}%
\begin{pgfscope}%
\pgfsys@transformshift{1.698385in}{0.528000in}%
\pgfsys@useobject{currentmarker}{}%
\end{pgfscope}%
\end{pgfscope}%
\begin{pgfscope}%
\pgftext[x=1.698385in,y=0.430778in,,top]{\sffamily\fontsize{10.000000}{12.000000}\selectfont \(\displaystyle 50\)}%
\end{pgfscope}%
\begin{pgfscope}%
\pgfsetbuttcap%
\pgfsetroundjoin%
\definecolor{currentfill}{rgb}{0.000000,0.000000,0.000000}%
\pgfsetfillcolor{currentfill}%
\pgfsetlinewidth{0.803000pt}%
\definecolor{currentstroke}{rgb}{0.000000,0.000000,0.000000}%
\pgfsetstrokecolor{currentstroke}%
\pgfsetdash{}{0pt}%
\pgfsys@defobject{currentmarker}{\pgfqpoint{0.000000in}{-0.048611in}}{\pgfqpoint{0.000000in}{0.000000in}}{%
\pgfpathmoveto{\pgfqpoint{0.000000in}{0.000000in}}%
\pgfpathlineto{\pgfqpoint{0.000000in}{-0.048611in}}%
\pgfusepath{stroke,fill}%
}%
\begin{pgfscope}%
\pgfsys@transformshift{2.371316in}{0.528000in}%
\pgfsys@useobject{currentmarker}{}%
\end{pgfscope}%
\end{pgfscope}%
\begin{pgfscope}%
\pgftext[x=2.371316in,y=0.430778in,,top]{\sffamily\fontsize{10.000000}{12.000000}\selectfont \(\displaystyle 100\)}%
\end{pgfscope}%
\begin{pgfscope}%
\pgfsetbuttcap%
\pgfsetroundjoin%
\definecolor{currentfill}{rgb}{0.000000,0.000000,0.000000}%
\pgfsetfillcolor{currentfill}%
\pgfsetlinewidth{0.803000pt}%
\definecolor{currentstroke}{rgb}{0.000000,0.000000,0.000000}%
\pgfsetstrokecolor{currentstroke}%
\pgfsetdash{}{0pt}%
\pgfsys@defobject{currentmarker}{\pgfqpoint{0.000000in}{-0.048611in}}{\pgfqpoint{0.000000in}{0.000000in}}{%
\pgfpathmoveto{\pgfqpoint{0.000000in}{0.000000in}}%
\pgfpathlineto{\pgfqpoint{0.000000in}{-0.048611in}}%
\pgfusepath{stroke,fill}%
}%
\begin{pgfscope}%
\pgfsys@transformshift{3.044246in}{0.528000in}%
\pgfsys@useobject{currentmarker}{}%
\end{pgfscope}%
\end{pgfscope}%
\begin{pgfscope}%
\pgftext[x=3.044246in,y=0.430778in,,top]{\sffamily\fontsize{10.000000}{12.000000}\selectfont \(\displaystyle 150\)}%
\end{pgfscope}%
\begin{pgfscope}%
\pgfsetbuttcap%
\pgfsetroundjoin%
\definecolor{currentfill}{rgb}{0.000000,0.000000,0.000000}%
\pgfsetfillcolor{currentfill}%
\pgfsetlinewidth{0.803000pt}%
\definecolor{currentstroke}{rgb}{0.000000,0.000000,0.000000}%
\pgfsetstrokecolor{currentstroke}%
\pgfsetdash{}{0pt}%
\pgfsys@defobject{currentmarker}{\pgfqpoint{0.000000in}{-0.048611in}}{\pgfqpoint{0.000000in}{0.000000in}}{%
\pgfpathmoveto{\pgfqpoint{0.000000in}{0.000000in}}%
\pgfpathlineto{\pgfqpoint{0.000000in}{-0.048611in}}%
\pgfusepath{stroke,fill}%
}%
\begin{pgfscope}%
\pgfsys@transformshift{3.717176in}{0.528000in}%
\pgfsys@useobject{currentmarker}{}%
\end{pgfscope}%
\end{pgfscope}%
\begin{pgfscope}%
\pgftext[x=3.717176in,y=0.430778in,,top]{\sffamily\fontsize{10.000000}{12.000000}\selectfont \(\displaystyle 200\)}%
\end{pgfscope}%
\begin{pgfscope}%
\pgfsetbuttcap%
\pgfsetroundjoin%
\definecolor{currentfill}{rgb}{0.000000,0.000000,0.000000}%
\pgfsetfillcolor{currentfill}%
\pgfsetlinewidth{0.803000pt}%
\definecolor{currentstroke}{rgb}{0.000000,0.000000,0.000000}%
\pgfsetstrokecolor{currentstroke}%
\pgfsetdash{}{0pt}%
\pgfsys@defobject{currentmarker}{\pgfqpoint{0.000000in}{-0.048611in}}{\pgfqpoint{0.000000in}{0.000000in}}{%
\pgfpathmoveto{\pgfqpoint{0.000000in}{0.000000in}}%
\pgfpathlineto{\pgfqpoint{0.000000in}{-0.048611in}}%
\pgfusepath{stroke,fill}%
}%
\begin{pgfscope}%
\pgfsys@transformshift{4.390107in}{0.528000in}%
\pgfsys@useobject{currentmarker}{}%
\end{pgfscope}%
\end{pgfscope}%
\begin{pgfscope}%
\pgftext[x=4.390107in,y=0.430778in,,top]{\sffamily\fontsize{10.000000}{12.000000}\selectfont \(\displaystyle 250\)}%
\end{pgfscope}%
\begin{pgfscope}%
\pgfsetbuttcap%
\pgfsetroundjoin%
\definecolor{currentfill}{rgb}{0.000000,0.000000,0.000000}%
\pgfsetfillcolor{currentfill}%
\pgfsetlinewidth{0.803000pt}%
\definecolor{currentstroke}{rgb}{0.000000,0.000000,0.000000}%
\pgfsetstrokecolor{currentstroke}%
\pgfsetdash{}{0pt}%
\pgfsys@defobject{currentmarker}{\pgfqpoint{0.000000in}{-0.048611in}}{\pgfqpoint{0.000000in}{0.000000in}}{%
\pgfpathmoveto{\pgfqpoint{0.000000in}{0.000000in}}%
\pgfpathlineto{\pgfqpoint{0.000000in}{-0.048611in}}%
\pgfusepath{stroke,fill}%
}%
\begin{pgfscope}%
\pgfsys@transformshift{5.063037in}{0.528000in}%
\pgfsys@useobject{currentmarker}{}%
\end{pgfscope}%
\end{pgfscope}%
\begin{pgfscope}%
\pgftext[x=5.063037in,y=0.430778in,,top]{\sffamily\fontsize{10.000000}{12.000000}\selectfont \(\displaystyle 300\)}%
\end{pgfscope}%
\begin{pgfscope}%
\pgfsetbuttcap%
\pgfsetroundjoin%
\definecolor{currentfill}{rgb}{0.000000,0.000000,0.000000}%
\pgfsetfillcolor{currentfill}%
\pgfsetlinewidth{0.803000pt}%
\definecolor{currentstroke}{rgb}{0.000000,0.000000,0.000000}%
\pgfsetstrokecolor{currentstroke}%
\pgfsetdash{}{0pt}%
\pgfsys@defobject{currentmarker}{\pgfqpoint{0.000000in}{-0.048611in}}{\pgfqpoint{0.000000in}{0.000000in}}{%
\pgfpathmoveto{\pgfqpoint{0.000000in}{0.000000in}}%
\pgfpathlineto{\pgfqpoint{0.000000in}{-0.048611in}}%
\pgfusepath{stroke,fill}%
}%
\begin{pgfscope}%
\pgfsys@transformshift{5.735968in}{0.528000in}%
\pgfsys@useobject{currentmarker}{}%
\end{pgfscope}%
\end{pgfscope}%
\begin{pgfscope}%
\pgftext[x=5.735968in,y=0.430778in,,top]{\sffamily\fontsize{10.000000}{12.000000}\selectfont \(\displaystyle 350\)}%
\end{pgfscope}%
\begin{pgfscope}%
\pgftext[x=3.280000in,y=0.251889in,,top]{\sffamily\fontsize{10.000000}{12.000000}\selectfont Time, days}%
\end{pgfscope}%
\begin{pgfscope}%
\pgfsetbuttcap%
\pgfsetroundjoin%
\definecolor{currentfill}{rgb}{0.000000,0.000000,0.000000}%
\pgfsetfillcolor{currentfill}%
\pgfsetlinewidth{0.803000pt}%
\definecolor{currentstroke}{rgb}{0.000000,0.000000,0.000000}%
\pgfsetstrokecolor{currentstroke}%
\pgfsetdash{}{0pt}%
\pgfsys@defobject{currentmarker}{\pgfqpoint{-0.048611in}{0.000000in}}{\pgfqpoint{0.000000in}{0.000000in}}{%
\pgfpathmoveto{\pgfqpoint{0.000000in}{0.000000in}}%
\pgfpathlineto{\pgfqpoint{-0.048611in}{0.000000in}}%
\pgfusepath{stroke,fill}%
}%
\begin{pgfscope}%
\pgfsys@transformshift{0.800000in}{0.570930in}%
\pgfsys@useobject{currentmarker}{}%
\end{pgfscope}%
\end{pgfscope}%
\begin{pgfscope}%
\pgftext[x=0.633333in,y=0.522735in,left,base]{\sffamily\fontsize{10.000000}{12.000000}\selectfont \(\displaystyle 0\)}%
\end{pgfscope}%
\begin{pgfscope}%
\pgfsetbuttcap%
\pgfsetroundjoin%
\definecolor{currentfill}{rgb}{0.000000,0.000000,0.000000}%
\pgfsetfillcolor{currentfill}%
\pgfsetlinewidth{0.803000pt}%
\definecolor{currentstroke}{rgb}{0.000000,0.000000,0.000000}%
\pgfsetstrokecolor{currentstroke}%
\pgfsetdash{}{0pt}%
\pgfsys@defobject{currentmarker}{\pgfqpoint{-0.048611in}{0.000000in}}{\pgfqpoint{0.000000in}{0.000000in}}{%
\pgfpathmoveto{\pgfqpoint{0.000000in}{0.000000in}}%
\pgfpathlineto{\pgfqpoint{-0.048611in}{0.000000in}}%
\pgfusepath{stroke,fill}%
}%
\begin{pgfscope}%
\pgfsys@transformshift{0.800000in}{1.017609in}%
\pgfsys@useobject{currentmarker}{}%
\end{pgfscope}%
\end{pgfscope}%
\begin{pgfscope}%
\pgftext[x=0.563888in,y=0.969414in,left,base]{\sffamily\fontsize{10.000000}{12.000000}\selectfont \(\displaystyle 25\)}%
\end{pgfscope}%
\begin{pgfscope}%
\pgfsetbuttcap%
\pgfsetroundjoin%
\definecolor{currentfill}{rgb}{0.000000,0.000000,0.000000}%
\pgfsetfillcolor{currentfill}%
\pgfsetlinewidth{0.803000pt}%
\definecolor{currentstroke}{rgb}{0.000000,0.000000,0.000000}%
\pgfsetstrokecolor{currentstroke}%
\pgfsetdash{}{0pt}%
\pgfsys@defobject{currentmarker}{\pgfqpoint{-0.048611in}{0.000000in}}{\pgfqpoint{0.000000in}{0.000000in}}{%
\pgfpathmoveto{\pgfqpoint{0.000000in}{0.000000in}}%
\pgfpathlineto{\pgfqpoint{-0.048611in}{0.000000in}}%
\pgfusepath{stroke,fill}%
}%
\begin{pgfscope}%
\pgfsys@transformshift{0.800000in}{1.464288in}%
\pgfsys@useobject{currentmarker}{}%
\end{pgfscope}%
\end{pgfscope}%
\begin{pgfscope}%
\pgftext[x=0.563888in,y=1.416093in,left,base]{\sffamily\fontsize{10.000000}{12.000000}\selectfont \(\displaystyle 50\)}%
\end{pgfscope}%
\begin{pgfscope}%
\pgfsetbuttcap%
\pgfsetroundjoin%
\definecolor{currentfill}{rgb}{0.000000,0.000000,0.000000}%
\pgfsetfillcolor{currentfill}%
\pgfsetlinewidth{0.803000pt}%
\definecolor{currentstroke}{rgb}{0.000000,0.000000,0.000000}%
\pgfsetstrokecolor{currentstroke}%
\pgfsetdash{}{0pt}%
\pgfsys@defobject{currentmarker}{\pgfqpoint{-0.048611in}{0.000000in}}{\pgfqpoint{0.000000in}{0.000000in}}{%
\pgfpathmoveto{\pgfqpoint{0.000000in}{0.000000in}}%
\pgfpathlineto{\pgfqpoint{-0.048611in}{0.000000in}}%
\pgfusepath{stroke,fill}%
}%
\begin{pgfscope}%
\pgfsys@transformshift{0.800000in}{1.910967in}%
\pgfsys@useobject{currentmarker}{}%
\end{pgfscope}%
\end{pgfscope}%
\begin{pgfscope}%
\pgftext[x=0.563888in,y=1.862772in,left,base]{\sffamily\fontsize{10.000000}{12.000000}\selectfont \(\displaystyle 75\)}%
\end{pgfscope}%
\begin{pgfscope}%
\pgfsetbuttcap%
\pgfsetroundjoin%
\definecolor{currentfill}{rgb}{0.000000,0.000000,0.000000}%
\pgfsetfillcolor{currentfill}%
\pgfsetlinewidth{0.803000pt}%
\definecolor{currentstroke}{rgb}{0.000000,0.000000,0.000000}%
\pgfsetstrokecolor{currentstroke}%
\pgfsetdash{}{0pt}%
\pgfsys@defobject{currentmarker}{\pgfqpoint{-0.048611in}{0.000000in}}{\pgfqpoint{0.000000in}{0.000000in}}{%
\pgfpathmoveto{\pgfqpoint{0.000000in}{0.000000in}}%
\pgfpathlineto{\pgfqpoint{-0.048611in}{0.000000in}}%
\pgfusepath{stroke,fill}%
}%
\begin{pgfscope}%
\pgfsys@transformshift{0.800000in}{2.357645in}%
\pgfsys@useobject{currentmarker}{}%
\end{pgfscope}%
\end{pgfscope}%
\begin{pgfscope}%
\pgftext[x=0.494444in,y=2.309451in,left,base]{\sffamily\fontsize{10.000000}{12.000000}\selectfont \(\displaystyle 100\)}%
\end{pgfscope}%
\begin{pgfscope}%
\pgfsetbuttcap%
\pgfsetroundjoin%
\definecolor{currentfill}{rgb}{0.000000,0.000000,0.000000}%
\pgfsetfillcolor{currentfill}%
\pgfsetlinewidth{0.803000pt}%
\definecolor{currentstroke}{rgb}{0.000000,0.000000,0.000000}%
\pgfsetstrokecolor{currentstroke}%
\pgfsetdash{}{0pt}%
\pgfsys@defobject{currentmarker}{\pgfqpoint{-0.048611in}{0.000000in}}{\pgfqpoint{0.000000in}{0.000000in}}{%
\pgfpathmoveto{\pgfqpoint{0.000000in}{0.000000in}}%
\pgfpathlineto{\pgfqpoint{-0.048611in}{0.000000in}}%
\pgfusepath{stroke,fill}%
}%
\begin{pgfscope}%
\pgfsys@transformshift{0.800000in}{2.804324in}%
\pgfsys@useobject{currentmarker}{}%
\end{pgfscope}%
\end{pgfscope}%
\begin{pgfscope}%
\pgftext[x=0.494444in,y=2.756130in,left,base]{\sffamily\fontsize{10.000000}{12.000000}\selectfont \(\displaystyle 125\)}%
\end{pgfscope}%
\begin{pgfscope}%
\pgfsetbuttcap%
\pgfsetroundjoin%
\definecolor{currentfill}{rgb}{0.000000,0.000000,0.000000}%
\pgfsetfillcolor{currentfill}%
\pgfsetlinewidth{0.803000pt}%
\definecolor{currentstroke}{rgb}{0.000000,0.000000,0.000000}%
\pgfsetstrokecolor{currentstroke}%
\pgfsetdash{}{0pt}%
\pgfsys@defobject{currentmarker}{\pgfqpoint{-0.048611in}{0.000000in}}{\pgfqpoint{0.000000in}{0.000000in}}{%
\pgfpathmoveto{\pgfqpoint{0.000000in}{0.000000in}}%
\pgfpathlineto{\pgfqpoint{-0.048611in}{0.000000in}}%
\pgfusepath{stroke,fill}%
}%
\begin{pgfscope}%
\pgfsys@transformshift{0.800000in}{3.251003in}%
\pgfsys@useobject{currentmarker}{}%
\end{pgfscope}%
\end{pgfscope}%
\begin{pgfscope}%
\pgftext[x=0.494444in,y=3.202809in,left,base]{\sffamily\fontsize{10.000000}{12.000000}\selectfont \(\displaystyle 150\)}%
\end{pgfscope}%
\begin{pgfscope}%
\pgfsetbuttcap%
\pgfsetroundjoin%
\definecolor{currentfill}{rgb}{0.000000,0.000000,0.000000}%
\pgfsetfillcolor{currentfill}%
\pgfsetlinewidth{0.803000pt}%
\definecolor{currentstroke}{rgb}{0.000000,0.000000,0.000000}%
\pgfsetstrokecolor{currentstroke}%
\pgfsetdash{}{0pt}%
\pgfsys@defobject{currentmarker}{\pgfqpoint{-0.048611in}{0.000000in}}{\pgfqpoint{0.000000in}{0.000000in}}{%
\pgfpathmoveto{\pgfqpoint{0.000000in}{0.000000in}}%
\pgfpathlineto{\pgfqpoint{-0.048611in}{0.000000in}}%
\pgfusepath{stroke,fill}%
}%
\begin{pgfscope}%
\pgfsys@transformshift{0.800000in}{3.697682in}%
\pgfsys@useobject{currentmarker}{}%
\end{pgfscope}%
\end{pgfscope}%
\begin{pgfscope}%
\pgftext[x=0.494444in,y=3.649488in,left,base]{\sffamily\fontsize{10.000000}{12.000000}\selectfont \(\displaystyle 175\)}%
\end{pgfscope}%
\begin{pgfscope}%
\pgfsetbuttcap%
\pgfsetroundjoin%
\definecolor{currentfill}{rgb}{0.000000,0.000000,0.000000}%
\pgfsetfillcolor{currentfill}%
\pgfsetlinewidth{0.803000pt}%
\definecolor{currentstroke}{rgb}{0.000000,0.000000,0.000000}%
\pgfsetstrokecolor{currentstroke}%
\pgfsetdash{}{0pt}%
\pgfsys@defobject{currentmarker}{\pgfqpoint{-0.048611in}{0.000000in}}{\pgfqpoint{0.000000in}{0.000000in}}{%
\pgfpathmoveto{\pgfqpoint{0.000000in}{0.000000in}}%
\pgfpathlineto{\pgfqpoint{-0.048611in}{0.000000in}}%
\pgfusepath{stroke,fill}%
}%
\begin{pgfscope}%
\pgfsys@transformshift{0.800000in}{4.144361in}%
\pgfsys@useobject{currentmarker}{}%
\end{pgfscope}%
\end{pgfscope}%
\begin{pgfscope}%
\pgftext[x=0.494444in,y=4.096167in,left,base]{\sffamily\fontsize{10.000000}{12.000000}\selectfont \(\displaystyle 200\)}%
\end{pgfscope}%
\begin{pgfscope}%
\pgftext[x=0.438888in,y=2.376000in,,bottom,rotate=90.000000]{\sffamily\fontsize{10.000000}{12.000000}\selectfont Semimajor axis, km (thousands)}%
\end{pgfscope}%
\begin{pgfscope}%
\pgfpathrectangle{\pgfqpoint{0.800000in}{0.528000in}}{\pgfqpoint{4.960000in}{3.696000in}} %
\pgfusepath{clip}%
\pgfsetbuttcap%
\pgfsetroundjoin%
\pgfsetlinewidth{1.505625pt}%
\definecolor{currentstroke}{rgb}{0.000000,0.000000,0.000000}%
\pgfsetstrokecolor{currentstroke}%
\pgfsetdash{{5.600000pt}{2.400000pt}}{0.000000pt}%
\pgfpathmoveto{\pgfqpoint{1.025455in}{0.696000in}}%
\pgfpathlineto{\pgfqpoint{1.206446in}{0.710350in}}%
\pgfpathlineto{\pgfqpoint{1.369737in}{0.725491in}}%
\pgfpathlineto{\pgfqpoint{1.517719in}{0.741408in}}%
\pgfpathlineto{\pgfqpoint{1.652785in}{0.758140in}}%
\pgfpathlineto{\pgfqpoint{1.776489in}{0.775675in}}%
\pgfpathlineto{\pgfqpoint{1.890307in}{0.794027in}}%
\pgfpathlineto{\pgfqpoint{1.995426in}{0.813200in}}%
\pgfpathlineto{\pgfqpoint{2.092939in}{0.833220in}}%
\pgfpathlineto{\pgfqpoint{2.183594in}{0.854074in}}%
\pgfpathlineto{\pgfqpoint{2.268345in}{0.875828in}}%
\pgfpathlineto{\pgfqpoint{2.347771in}{0.898489in}}%
\pgfpathlineto{\pgfqpoint{2.422351in}{0.922058in}}%
\pgfpathlineto{\pgfqpoint{2.492654in}{0.946585in}}%
\pgfpathlineto{\pgfqpoint{2.559163in}{0.972124in}}%
\pgfpathlineto{\pgfqpoint{2.622189in}{0.998690in}}%
\pgfpathlineto{\pgfqpoint{2.682110in}{1.026344in}}%
\pgfpathlineto{\pgfqpoint{2.739153in}{1.055097in}}%
\pgfpathlineto{\pgfqpoint{2.793681in}{1.085052in}}%
\pgfpathlineto{\pgfqpoint{2.845878in}{1.116242in}}%
\pgfpathlineto{\pgfqpoint{2.896109in}{1.148830in}}%
\pgfpathlineto{\pgfqpoint{2.944670in}{1.182988in}}%
\pgfpathlineto{\pgfqpoint{2.991603in}{1.218736in}}%
\pgfpathlineto{\pgfqpoint{3.036918in}{1.256055in}}%
\pgfpathlineto{\pgfqpoint{3.080692in}{1.294978in}}%
\pgfpathlineto{\pgfqpoint{3.123510in}{1.336039in}}%
\pgfpathlineto{\pgfqpoint{3.164892in}{1.378790in}}%
\pgfpathlineto{\pgfqpoint{3.205797in}{1.424277in}}%
\pgfpathlineto{\pgfqpoint{3.245405in}{1.471647in}}%
\pgfpathlineto{\pgfqpoint{3.284361in}{1.521698in}}%
\pgfpathlineto{\pgfqpoint{3.322871in}{1.574827in}}%
\pgfpathlineto{\pgfqpoint{3.360358in}{1.630302in}}%
\pgfpathlineto{\pgfqpoint{3.398145in}{1.690263in}}%
\pgfpathlineto{\pgfqpoint{3.434949in}{1.752844in}}%
\pgfpathlineto{\pgfqpoint{3.472035in}{1.820368in}}%
\pgfpathlineto{\pgfqpoint{3.508725in}{1.891859in}}%
\pgfpathlineto{\pgfqpoint{3.544958in}{1.967313in}}%
\pgfpathlineto{\pgfqpoint{3.581233in}{2.047931in}}%
\pgfpathlineto{\pgfqpoint{3.617512in}{2.133873in}}%
\pgfpathlineto{\pgfqpoint{3.655370in}{2.229416in}}%
\pgfpathlineto{\pgfqpoint{3.692729in}{2.329705in}}%
\pgfpathlineto{\pgfqpoint{3.732205in}{2.442197in}}%
\pgfpathlineto{\pgfqpoint{3.772149in}{2.562724in}}%
\pgfpathlineto{\pgfqpoint{3.814330in}{2.696927in}}%
\pgfpathlineto{\pgfqpoint{3.862778in}{2.858844in}}%
\pgfpathlineto{\pgfqpoint{3.920899in}{3.061357in}}%
\pgfpathlineto{\pgfqpoint{4.049709in}{3.513363in}}%
\pgfpathlineto{\pgfqpoint{4.088494in}{3.639398in}}%
\pgfpathlineto{\pgfqpoint{4.116577in}{3.724299in}}%
\pgfpathlineto{\pgfqpoint{4.142299in}{3.796042in}}%
\pgfpathlineto{\pgfqpoint{4.166255in}{3.856724in}}%
\pgfpathlineto{\pgfqpoint{4.188886in}{3.907839in}}%
\pgfpathlineto{\pgfqpoint{4.209906in}{3.949323in}}%
\pgfpathlineto{\pgfqpoint{4.227002in}{3.978463in}}%
\pgfpathlineto{\pgfqpoint{4.245761in}{4.005395in}}%
\pgfpathlineto{\pgfqpoint{4.260502in}{4.022690in}}%
\pgfpathlineto{\pgfqpoint{4.276930in}{4.037809in}}%
\pgfpathlineto{\pgfqpoint{4.287887in}{4.045402in}}%
\pgfpathlineto{\pgfqpoint{4.302284in}{4.052298in}}%
\pgfpathlineto{\pgfqpoint{4.312051in}{4.054964in}}%
\pgfpathlineto{\pgfqpoint{4.322086in}{4.056000in}}%
\pgfpathlineto{\pgfqpoint{4.331806in}{4.055355in}}%
\pgfpathlineto{\pgfqpoint{4.344453in}{4.052089in}}%
\pgfpathlineto{\pgfqpoint{4.358068in}{4.045525in}}%
\pgfpathlineto{\pgfqpoint{4.372480in}{4.035170in}}%
\pgfpathlineto{\pgfqpoint{4.387287in}{4.020949in}}%
\pgfpathlineto{\pgfqpoint{4.401582in}{4.003857in}}%
\pgfpathlineto{\pgfqpoint{4.415769in}{3.983728in}}%
\pgfpathlineto{\pgfqpoint{4.434125in}{3.953183in}}%
\pgfpathlineto{\pgfqpoint{4.453045in}{3.916667in}}%
\pgfpathlineto{\pgfqpoint{4.474194in}{3.870219in}}%
\pgfpathlineto{\pgfqpoint{4.495134in}{3.818881in}}%
\pgfpathlineto{\pgfqpoint{4.521371in}{3.747883in}}%
\pgfpathlineto{\pgfqpoint{4.549782in}{3.663849in}}%
\pgfpathlineto{\pgfqpoint{4.581921in}{3.561585in}}%
\pgfpathlineto{\pgfqpoint{4.622419in}{3.424856in}}%
\pgfpathlineto{\pgfqpoint{4.684616in}{3.205765in}}%
\pgfpathlineto{\pgfqpoint{4.778762in}{2.874659in}}%
\pgfpathlineto{\pgfqpoint{4.830245in}{2.702102in}}%
\pgfpathlineto{\pgfqpoint{4.875438in}{2.558308in}}%
\pgfpathlineto{\pgfqpoint{4.918184in}{2.429827in}}%
\pgfpathlineto{\pgfqpoint{4.956815in}{2.320373in}}%
\pgfpathlineto{\pgfqpoint{4.995553in}{2.217060in}}%
\pgfpathlineto{\pgfqpoint{5.033274in}{2.122611in}}%
\pgfpathlineto{\pgfqpoint{5.070953in}{2.034178in}}%
\pgfpathlineto{\pgfqpoint{5.108694in}{1.951302in}}%
\pgfpathlineto{\pgfqpoint{5.145464in}{1.875801in}}%
\pgfpathlineto{\pgfqpoint{5.183170in}{1.803470in}}%
\pgfpathlineto{\pgfqpoint{5.220291in}{1.737006in}}%
\pgfpathlineto{\pgfqpoint{5.258046in}{1.673933in}}%
\pgfpathlineto{\pgfqpoint{5.295733in}{1.615228in}}%
\pgfpathlineto{\pgfqpoint{5.334494in}{1.558978in}}%
\pgfpathlineto{\pgfqpoint{5.373294in}{1.506556in}}%
\pgfpathlineto{\pgfqpoint{5.413196in}{1.456399in}}%
\pgfpathlineto{\pgfqpoint{5.453657in}{1.409125in}}%
\pgfpathlineto{\pgfqpoint{5.494666in}{1.364602in}}%
\pgfpathlineto{\pgfqpoint{5.534545in}{1.324320in}}%
\pgfpathlineto{\pgfqpoint{5.534545in}{1.324320in}}%
\pgfusepath{stroke}%
\end{pgfscope}%
\begin{pgfscope}%
\pgfsetrectcap%
\pgfsetmiterjoin%
\pgfsetlinewidth{0.803000pt}%
\definecolor{currentstroke}{rgb}{0.000000,0.000000,0.000000}%
\pgfsetstrokecolor{currentstroke}%
\pgfsetdash{}{0pt}%
\pgfpathmoveto{\pgfqpoint{0.800000in}{0.528000in}}%
\pgfpathlineto{\pgfqpoint{0.800000in}{4.224000in}}%
\pgfusepath{stroke}%
\end{pgfscope}%
\begin{pgfscope}%
\pgfsetrectcap%
\pgfsetmiterjoin%
\pgfsetlinewidth{0.803000pt}%
\definecolor{currentstroke}{rgb}{0.000000,0.000000,0.000000}%
\pgfsetstrokecolor{currentstroke}%
\pgfsetdash{}{0pt}%
\pgfpathmoveto{\pgfqpoint{5.760000in}{0.528000in}}%
\pgfpathlineto{\pgfqpoint{5.760000in}{4.224000in}}%
\pgfusepath{stroke}%
\end{pgfscope}%
\begin{pgfscope}%
\pgfsetrectcap%
\pgfsetmiterjoin%
\pgfsetlinewidth{0.803000pt}%
\definecolor{currentstroke}{rgb}{0.000000,0.000000,0.000000}%
\pgfsetstrokecolor{currentstroke}%
\pgfsetdash{}{0pt}%
\pgfpathmoveto{\pgfqpoint{0.800000in}{0.528000in}}%
\pgfpathlineto{\pgfqpoint{5.760000in}{0.528000in}}%
\pgfusepath{stroke}%
\end{pgfscope}%
\begin{pgfscope}%
\pgfsetrectcap%
\pgfsetmiterjoin%
\pgfsetlinewidth{0.803000pt}%
\definecolor{currentstroke}{rgb}{0.000000,0.000000,0.000000}%
\pgfsetstrokecolor{currentstroke}%
\pgfsetdash{}{0pt}%
\pgfpathmoveto{\pgfqpoint{0.800000in}{4.224000in}}%
\pgfpathlineto{\pgfqpoint{5.760000in}{4.224000in}}%
\pgfusepath{stroke}%
\end{pgfscope}%
\end{pgfpicture}%
\makeatother%
\endgroup%

}
\end{subfigure}
\begin{subfigure}[b]{0.5\textwidth}
\centering
\resizebox{1.0\textwidth}{!}{
%% Creator: Matplotlib, PGF backend
%%
%% To include the figure in your LaTeX document, write
%%   \input{<filename>.pgf}
%%
%% Make sure the required packages are loaded in your preamble
%%   \usepackage{pgf}
%%
%% Figures using additional raster images can only be included by \input if
%% they are in the same directory as the main LaTeX file. For loading figures
%% from other directories you can use the `import` package
%%   \usepackage{import}
%% and then include the figures with
%%   \import{<path to file>}{<filename>.pgf}
%%
%% Matplotlib used the following preamble
%%   \usepackage{fontspec}
%%
\begingroup%
\makeatletter%
\begin{pgfpicture}%
\pgfpathrectangle{\pgfpointorigin}{\pgfqpoint{6.400000in}{4.800000in}}%
\pgfusepath{use as bounding box, clip}%
\begin{pgfscope}%
\pgfsetbuttcap%
\pgfsetmiterjoin%
\definecolor{currentfill}{rgb}{1.000000,1.000000,1.000000}%
\pgfsetfillcolor{currentfill}%
\pgfsetlinewidth{0.000000pt}%
\definecolor{currentstroke}{rgb}{1.000000,1.000000,1.000000}%
\pgfsetstrokecolor{currentstroke}%
\pgfsetdash{}{0pt}%
\pgfpathmoveto{\pgfqpoint{0.000000in}{0.000000in}}%
\pgfpathlineto{\pgfqpoint{6.400000in}{0.000000in}}%
\pgfpathlineto{\pgfqpoint{6.400000in}{4.800000in}}%
\pgfpathlineto{\pgfqpoint{0.000000in}{4.800000in}}%
\pgfpathclose%
\pgfusepath{fill}%
\end{pgfscope}%
\begin{pgfscope}%
\pgfsetbuttcap%
\pgfsetmiterjoin%
\definecolor{currentfill}{rgb}{1.000000,1.000000,1.000000}%
\pgfsetfillcolor{currentfill}%
\pgfsetlinewidth{0.000000pt}%
\definecolor{currentstroke}{rgb}{0.000000,0.000000,0.000000}%
\pgfsetstrokecolor{currentstroke}%
\pgfsetstrokeopacity{0.000000}%
\pgfsetdash{}{0pt}%
\pgfpathmoveto{\pgfqpoint{0.800000in}{0.528000in}}%
\pgfpathlineto{\pgfqpoint{5.760000in}{0.528000in}}%
\pgfpathlineto{\pgfqpoint{5.760000in}{4.224000in}}%
\pgfpathlineto{\pgfqpoint{0.800000in}{4.224000in}}%
\pgfpathclose%
\pgfusepath{fill}%
\end{pgfscope}%
\begin{pgfscope}%
\pgfsetbuttcap%
\pgfsetroundjoin%
\definecolor{currentfill}{rgb}{0.000000,0.000000,0.000000}%
\pgfsetfillcolor{currentfill}%
\pgfsetlinewidth{0.803000pt}%
\definecolor{currentstroke}{rgb}{0.000000,0.000000,0.000000}%
\pgfsetstrokecolor{currentstroke}%
\pgfsetdash{}{0pt}%
\pgfsys@defobject{currentmarker}{\pgfqpoint{0.000000in}{-0.048611in}}{\pgfqpoint{0.000000in}{0.000000in}}{%
\pgfpathmoveto{\pgfqpoint{0.000000in}{0.000000in}}%
\pgfpathlineto{\pgfqpoint{0.000000in}{-0.048611in}}%
\pgfusepath{stroke,fill}%
}%
\begin{pgfscope}%
\pgfsys@transformshift{1.025455in}{0.528000in}%
\pgfsys@useobject{currentmarker}{}%
\end{pgfscope}%
\end{pgfscope}%
\begin{pgfscope}%
\pgftext[x=1.025455in,y=0.430778in,,top]{\sffamily\fontsize{10.000000}{12.000000}\selectfont \(\displaystyle 0\)}%
\end{pgfscope}%
\begin{pgfscope}%
\pgfsetbuttcap%
\pgfsetroundjoin%
\definecolor{currentfill}{rgb}{0.000000,0.000000,0.000000}%
\pgfsetfillcolor{currentfill}%
\pgfsetlinewidth{0.803000pt}%
\definecolor{currentstroke}{rgb}{0.000000,0.000000,0.000000}%
\pgfsetstrokecolor{currentstroke}%
\pgfsetdash{}{0pt}%
\pgfsys@defobject{currentmarker}{\pgfqpoint{0.000000in}{-0.048611in}}{\pgfqpoint{0.000000in}{0.000000in}}{%
\pgfpathmoveto{\pgfqpoint{0.000000in}{0.000000in}}%
\pgfpathlineto{\pgfqpoint{0.000000in}{-0.048611in}}%
\pgfusepath{stroke,fill}%
}%
\begin{pgfscope}%
\pgfsys@transformshift{1.698385in}{0.528000in}%
\pgfsys@useobject{currentmarker}{}%
\end{pgfscope}%
\end{pgfscope}%
\begin{pgfscope}%
\pgftext[x=1.698385in,y=0.430778in,,top]{\sffamily\fontsize{10.000000}{12.000000}\selectfont \(\displaystyle 50\)}%
\end{pgfscope}%
\begin{pgfscope}%
\pgfsetbuttcap%
\pgfsetroundjoin%
\definecolor{currentfill}{rgb}{0.000000,0.000000,0.000000}%
\pgfsetfillcolor{currentfill}%
\pgfsetlinewidth{0.803000pt}%
\definecolor{currentstroke}{rgb}{0.000000,0.000000,0.000000}%
\pgfsetstrokecolor{currentstroke}%
\pgfsetdash{}{0pt}%
\pgfsys@defobject{currentmarker}{\pgfqpoint{0.000000in}{-0.048611in}}{\pgfqpoint{0.000000in}{0.000000in}}{%
\pgfpathmoveto{\pgfqpoint{0.000000in}{0.000000in}}%
\pgfpathlineto{\pgfqpoint{0.000000in}{-0.048611in}}%
\pgfusepath{stroke,fill}%
}%
\begin{pgfscope}%
\pgfsys@transformshift{2.371316in}{0.528000in}%
\pgfsys@useobject{currentmarker}{}%
\end{pgfscope}%
\end{pgfscope}%
\begin{pgfscope}%
\pgftext[x=2.371316in,y=0.430778in,,top]{\sffamily\fontsize{10.000000}{12.000000}\selectfont \(\displaystyle 100\)}%
\end{pgfscope}%
\begin{pgfscope}%
\pgfsetbuttcap%
\pgfsetroundjoin%
\definecolor{currentfill}{rgb}{0.000000,0.000000,0.000000}%
\pgfsetfillcolor{currentfill}%
\pgfsetlinewidth{0.803000pt}%
\definecolor{currentstroke}{rgb}{0.000000,0.000000,0.000000}%
\pgfsetstrokecolor{currentstroke}%
\pgfsetdash{}{0pt}%
\pgfsys@defobject{currentmarker}{\pgfqpoint{0.000000in}{-0.048611in}}{\pgfqpoint{0.000000in}{0.000000in}}{%
\pgfpathmoveto{\pgfqpoint{0.000000in}{0.000000in}}%
\pgfpathlineto{\pgfqpoint{0.000000in}{-0.048611in}}%
\pgfusepath{stroke,fill}%
}%
\begin{pgfscope}%
\pgfsys@transformshift{3.044246in}{0.528000in}%
\pgfsys@useobject{currentmarker}{}%
\end{pgfscope}%
\end{pgfscope}%
\begin{pgfscope}%
\pgftext[x=3.044246in,y=0.430778in,,top]{\sffamily\fontsize{10.000000}{12.000000}\selectfont \(\displaystyle 150\)}%
\end{pgfscope}%
\begin{pgfscope}%
\pgfsetbuttcap%
\pgfsetroundjoin%
\definecolor{currentfill}{rgb}{0.000000,0.000000,0.000000}%
\pgfsetfillcolor{currentfill}%
\pgfsetlinewidth{0.803000pt}%
\definecolor{currentstroke}{rgb}{0.000000,0.000000,0.000000}%
\pgfsetstrokecolor{currentstroke}%
\pgfsetdash{}{0pt}%
\pgfsys@defobject{currentmarker}{\pgfqpoint{0.000000in}{-0.048611in}}{\pgfqpoint{0.000000in}{0.000000in}}{%
\pgfpathmoveto{\pgfqpoint{0.000000in}{0.000000in}}%
\pgfpathlineto{\pgfqpoint{0.000000in}{-0.048611in}}%
\pgfusepath{stroke,fill}%
}%
\begin{pgfscope}%
\pgfsys@transformshift{3.717176in}{0.528000in}%
\pgfsys@useobject{currentmarker}{}%
\end{pgfscope}%
\end{pgfscope}%
\begin{pgfscope}%
\pgftext[x=3.717176in,y=0.430778in,,top]{\sffamily\fontsize{10.000000}{12.000000}\selectfont \(\displaystyle 200\)}%
\end{pgfscope}%
\begin{pgfscope}%
\pgfsetbuttcap%
\pgfsetroundjoin%
\definecolor{currentfill}{rgb}{0.000000,0.000000,0.000000}%
\pgfsetfillcolor{currentfill}%
\pgfsetlinewidth{0.803000pt}%
\definecolor{currentstroke}{rgb}{0.000000,0.000000,0.000000}%
\pgfsetstrokecolor{currentstroke}%
\pgfsetdash{}{0pt}%
\pgfsys@defobject{currentmarker}{\pgfqpoint{0.000000in}{-0.048611in}}{\pgfqpoint{0.000000in}{0.000000in}}{%
\pgfpathmoveto{\pgfqpoint{0.000000in}{0.000000in}}%
\pgfpathlineto{\pgfqpoint{0.000000in}{-0.048611in}}%
\pgfusepath{stroke,fill}%
}%
\begin{pgfscope}%
\pgfsys@transformshift{4.390107in}{0.528000in}%
\pgfsys@useobject{currentmarker}{}%
\end{pgfscope}%
\end{pgfscope}%
\begin{pgfscope}%
\pgftext[x=4.390107in,y=0.430778in,,top]{\sffamily\fontsize{10.000000}{12.000000}\selectfont \(\displaystyle 250\)}%
\end{pgfscope}%
\begin{pgfscope}%
\pgfsetbuttcap%
\pgfsetroundjoin%
\definecolor{currentfill}{rgb}{0.000000,0.000000,0.000000}%
\pgfsetfillcolor{currentfill}%
\pgfsetlinewidth{0.803000pt}%
\definecolor{currentstroke}{rgb}{0.000000,0.000000,0.000000}%
\pgfsetstrokecolor{currentstroke}%
\pgfsetdash{}{0pt}%
\pgfsys@defobject{currentmarker}{\pgfqpoint{0.000000in}{-0.048611in}}{\pgfqpoint{0.000000in}{0.000000in}}{%
\pgfpathmoveto{\pgfqpoint{0.000000in}{0.000000in}}%
\pgfpathlineto{\pgfqpoint{0.000000in}{-0.048611in}}%
\pgfusepath{stroke,fill}%
}%
\begin{pgfscope}%
\pgfsys@transformshift{5.063037in}{0.528000in}%
\pgfsys@useobject{currentmarker}{}%
\end{pgfscope}%
\end{pgfscope}%
\begin{pgfscope}%
\pgftext[x=5.063037in,y=0.430778in,,top]{\sffamily\fontsize{10.000000}{12.000000}\selectfont \(\displaystyle 300\)}%
\end{pgfscope}%
\begin{pgfscope}%
\pgfsetbuttcap%
\pgfsetroundjoin%
\definecolor{currentfill}{rgb}{0.000000,0.000000,0.000000}%
\pgfsetfillcolor{currentfill}%
\pgfsetlinewidth{0.803000pt}%
\definecolor{currentstroke}{rgb}{0.000000,0.000000,0.000000}%
\pgfsetstrokecolor{currentstroke}%
\pgfsetdash{}{0pt}%
\pgfsys@defobject{currentmarker}{\pgfqpoint{0.000000in}{-0.048611in}}{\pgfqpoint{0.000000in}{0.000000in}}{%
\pgfpathmoveto{\pgfqpoint{0.000000in}{0.000000in}}%
\pgfpathlineto{\pgfqpoint{0.000000in}{-0.048611in}}%
\pgfusepath{stroke,fill}%
}%
\begin{pgfscope}%
\pgfsys@transformshift{5.735968in}{0.528000in}%
\pgfsys@useobject{currentmarker}{}%
\end{pgfscope}%
\end{pgfscope}%
\begin{pgfscope}%
\pgftext[x=5.735968in,y=0.430778in,,top]{\sffamily\fontsize{10.000000}{12.000000}\selectfont \(\displaystyle 350\)}%
\end{pgfscope}%
\begin{pgfscope}%
\pgftext[x=3.280000in,y=0.251889in,,top]{\sffamily\fontsize{10.000000}{12.000000}\selectfont Time, days}%
\end{pgfscope}%
\begin{pgfscope}%
\pgfsetbuttcap%
\pgfsetroundjoin%
\definecolor{currentfill}{rgb}{0.000000,0.000000,0.000000}%
\pgfsetfillcolor{currentfill}%
\pgfsetlinewidth{0.803000pt}%
\definecolor{currentstroke}{rgb}{0.000000,0.000000,0.000000}%
\pgfsetstrokecolor{currentstroke}%
\pgfsetdash{}{0pt}%
\pgfsys@defobject{currentmarker}{\pgfqpoint{-0.048611in}{0.000000in}}{\pgfqpoint{0.000000in}{0.000000in}}{%
\pgfpathmoveto{\pgfqpoint{0.000000in}{0.000000in}}%
\pgfpathlineto{\pgfqpoint{-0.048611in}{0.000000in}}%
\pgfusepath{stroke,fill}%
}%
\begin{pgfscope}%
\pgfsys@transformshift{0.800000in}{1.010994in}%
\pgfsys@useobject{currentmarker}{}%
\end{pgfscope}%
\end{pgfscope}%
\begin{pgfscope}%
\pgftext[x=0.633333in,y=0.962799in,left,base]{\sffamily\fontsize{10.000000}{12.000000}\selectfont \(\displaystyle 2\)}%
\end{pgfscope}%
\begin{pgfscope}%
\pgfsetbuttcap%
\pgfsetroundjoin%
\definecolor{currentfill}{rgb}{0.000000,0.000000,0.000000}%
\pgfsetfillcolor{currentfill}%
\pgfsetlinewidth{0.803000pt}%
\definecolor{currentstroke}{rgb}{0.000000,0.000000,0.000000}%
\pgfsetstrokecolor{currentstroke}%
\pgfsetdash{}{0pt}%
\pgfsys@defobject{currentmarker}{\pgfqpoint{-0.048611in}{0.000000in}}{\pgfqpoint{0.000000in}{0.000000in}}{%
\pgfpathmoveto{\pgfqpoint{0.000000in}{0.000000in}}%
\pgfpathlineto{\pgfqpoint{-0.048611in}{0.000000in}}%
\pgfusepath{stroke,fill}%
}%
\begin{pgfscope}%
\pgfsys@transformshift{0.800000in}{1.560013in}%
\pgfsys@useobject{currentmarker}{}%
\end{pgfscope}%
\end{pgfscope}%
\begin{pgfscope}%
\pgftext[x=0.633333in,y=1.511818in,left,base]{\sffamily\fontsize{10.000000}{12.000000}\selectfont \(\displaystyle 3\)}%
\end{pgfscope}%
\begin{pgfscope}%
\pgfsetbuttcap%
\pgfsetroundjoin%
\definecolor{currentfill}{rgb}{0.000000,0.000000,0.000000}%
\pgfsetfillcolor{currentfill}%
\pgfsetlinewidth{0.803000pt}%
\definecolor{currentstroke}{rgb}{0.000000,0.000000,0.000000}%
\pgfsetstrokecolor{currentstroke}%
\pgfsetdash{}{0pt}%
\pgfsys@defobject{currentmarker}{\pgfqpoint{-0.048611in}{0.000000in}}{\pgfqpoint{0.000000in}{0.000000in}}{%
\pgfpathmoveto{\pgfqpoint{0.000000in}{0.000000in}}%
\pgfpathlineto{\pgfqpoint{-0.048611in}{0.000000in}}%
\pgfusepath{stroke,fill}%
}%
\begin{pgfscope}%
\pgfsys@transformshift{0.800000in}{2.109032in}%
\pgfsys@useobject{currentmarker}{}%
\end{pgfscope}%
\end{pgfscope}%
\begin{pgfscope}%
\pgftext[x=0.633333in,y=2.060837in,left,base]{\sffamily\fontsize{10.000000}{12.000000}\selectfont \(\displaystyle 4\)}%
\end{pgfscope}%
\begin{pgfscope}%
\pgfsetbuttcap%
\pgfsetroundjoin%
\definecolor{currentfill}{rgb}{0.000000,0.000000,0.000000}%
\pgfsetfillcolor{currentfill}%
\pgfsetlinewidth{0.803000pt}%
\definecolor{currentstroke}{rgb}{0.000000,0.000000,0.000000}%
\pgfsetstrokecolor{currentstroke}%
\pgfsetdash{}{0pt}%
\pgfsys@defobject{currentmarker}{\pgfqpoint{-0.048611in}{0.000000in}}{\pgfqpoint{0.000000in}{0.000000in}}{%
\pgfpathmoveto{\pgfqpoint{0.000000in}{0.000000in}}%
\pgfpathlineto{\pgfqpoint{-0.048611in}{0.000000in}}%
\pgfusepath{stroke,fill}%
}%
\begin{pgfscope}%
\pgfsys@transformshift{0.800000in}{2.658051in}%
\pgfsys@useobject{currentmarker}{}%
\end{pgfscope}%
\end{pgfscope}%
\begin{pgfscope}%
\pgftext[x=0.633333in,y=2.609856in,left,base]{\sffamily\fontsize{10.000000}{12.000000}\selectfont \(\displaystyle 5\)}%
\end{pgfscope}%
\begin{pgfscope}%
\pgfsetbuttcap%
\pgfsetroundjoin%
\definecolor{currentfill}{rgb}{0.000000,0.000000,0.000000}%
\pgfsetfillcolor{currentfill}%
\pgfsetlinewidth{0.803000pt}%
\definecolor{currentstroke}{rgb}{0.000000,0.000000,0.000000}%
\pgfsetstrokecolor{currentstroke}%
\pgfsetdash{}{0pt}%
\pgfsys@defobject{currentmarker}{\pgfqpoint{-0.048611in}{0.000000in}}{\pgfqpoint{0.000000in}{0.000000in}}{%
\pgfpathmoveto{\pgfqpoint{0.000000in}{0.000000in}}%
\pgfpathlineto{\pgfqpoint{-0.048611in}{0.000000in}}%
\pgfusepath{stroke,fill}%
}%
\begin{pgfscope}%
\pgfsys@transformshift{0.800000in}{3.207070in}%
\pgfsys@useobject{currentmarker}{}%
\end{pgfscope}%
\end{pgfscope}%
\begin{pgfscope}%
\pgftext[x=0.633333in,y=3.158875in,left,base]{\sffamily\fontsize{10.000000}{12.000000}\selectfont \(\displaystyle 6\)}%
\end{pgfscope}%
\begin{pgfscope}%
\pgfsetbuttcap%
\pgfsetroundjoin%
\definecolor{currentfill}{rgb}{0.000000,0.000000,0.000000}%
\pgfsetfillcolor{currentfill}%
\pgfsetlinewidth{0.803000pt}%
\definecolor{currentstroke}{rgb}{0.000000,0.000000,0.000000}%
\pgfsetstrokecolor{currentstroke}%
\pgfsetdash{}{0pt}%
\pgfsys@defobject{currentmarker}{\pgfqpoint{-0.048611in}{0.000000in}}{\pgfqpoint{0.000000in}{0.000000in}}{%
\pgfpathmoveto{\pgfqpoint{0.000000in}{0.000000in}}%
\pgfpathlineto{\pgfqpoint{-0.048611in}{0.000000in}}%
\pgfusepath{stroke,fill}%
}%
\begin{pgfscope}%
\pgfsys@transformshift{0.800000in}{3.756089in}%
\pgfsys@useobject{currentmarker}{}%
\end{pgfscope}%
\end{pgfscope}%
\begin{pgfscope}%
\pgftext[x=0.633333in,y=3.707894in,left,base]{\sffamily\fontsize{10.000000}{12.000000}\selectfont \(\displaystyle 7\)}%
\end{pgfscope}%
\begin{pgfscope}%
\pgftext[x=0.577777in,y=2.376000in,,bottom,rotate=90.000000]{\sffamily\fontsize{10.000000}{12.000000}\selectfont Velocity, km/s}%
\end{pgfscope}%
\begin{pgfscope}%
\pgfpathrectangle{\pgfqpoint{0.800000in}{0.528000in}}{\pgfqpoint{4.960000in}{3.696000in}} %
\pgfusepath{clip}%
\pgfsetrectcap%
\pgfsetroundjoin%
\pgfsetlinewidth{1.505625pt}%
\definecolor{currentstroke}{rgb}{0.000000,0.000000,0.000000}%
\pgfsetstrokecolor{currentstroke}%
\pgfsetdash{}{0pt}%
\pgfpathmoveto{\pgfqpoint{1.025455in}{4.055880in}}%
\pgfpathlineto{\pgfqpoint{1.025674in}{4.055964in}}%
\pgfpathlineto{\pgfqpoint{1.025830in}{4.055608in}}%
\pgfpathlineto{\pgfqpoint{1.025830in}{4.055608in}}%
\pgfpathlineto{\pgfqpoint{1.027548in}{4.053675in}}%
\pgfpathlineto{\pgfqpoint{1.029887in}{4.050261in}}%
\pgfpathlineto{\pgfqpoint{1.030151in}{4.050489in}}%
\pgfpathlineto{\pgfqpoint{1.030620in}{4.049312in}}%
\pgfpathlineto{\pgfqpoint{1.032617in}{4.046943in}}%
\pgfpathlineto{\pgfqpoint{1.032882in}{4.047171in}}%
\pgfpathlineto{\pgfqpoint{1.033346in}{4.046027in}}%
\pgfpathlineto{\pgfqpoint{1.035367in}{4.043621in}}%
\pgfpathlineto{\pgfqpoint{1.035533in}{4.043785in}}%
\pgfpathlineto{\pgfqpoint{1.035707in}{4.043810in}}%
\pgfpathlineto{\pgfqpoint{1.035854in}{4.043486in}}%
\pgfpathlineto{\pgfqpoint{1.035854in}{4.043486in}}%
\pgfpathlineto{\pgfqpoint{1.037922in}{4.040489in}}%
\pgfpathlineto{\pgfqpoint{1.039966in}{4.038070in}}%
\pgfpathlineto{\pgfqpoint{1.040100in}{4.038202in}}%
\pgfpathlineto{\pgfqpoint{1.040298in}{4.038249in}}%
\pgfpathlineto{\pgfqpoint{1.040443in}{4.037929in}}%
\pgfpathlineto{\pgfqpoint{1.040443in}{4.037929in}}%
\pgfpathlineto{\pgfqpoint{1.042523in}{4.034915in}}%
\pgfpathlineto{\pgfqpoint{1.044561in}{4.032486in}}%
\pgfpathlineto{\pgfqpoint{1.044729in}{4.032644in}}%
\pgfpathlineto{\pgfqpoint{1.044904in}{4.032670in}}%
\pgfpathlineto{\pgfqpoint{1.045057in}{4.032333in}}%
\pgfpathlineto{\pgfqpoint{1.045057in}{4.032333in}}%
\pgfpathlineto{\pgfqpoint{1.046844in}{4.030270in}}%
\pgfpathlineto{\pgfqpoint{1.049183in}{4.026883in}}%
\pgfpathlineto{\pgfqpoint{1.049250in}{4.026928in}}%
\pgfpathlineto{\pgfqpoint{1.049528in}{4.027069in}}%
\pgfpathlineto{\pgfqpoint{1.049753in}{4.026498in}}%
\pgfpathlineto{\pgfqpoint{1.049753in}{4.026498in}}%
\pgfpathlineto{\pgfqpoint{1.051044in}{4.024638in}}%
\pgfpathlineto{\pgfqpoint{1.051379in}{4.024826in}}%
\pgfpathlineto{\pgfqpoint{1.056396in}{4.018173in}}%
\pgfpathlineto{\pgfqpoint{1.058497in}{4.015615in}}%
\pgfpathlineto{\pgfqpoint{1.058902in}{4.015699in}}%
\pgfpathlineto{\pgfqpoint{1.059206in}{4.014760in}}%
\pgfpathlineto{\pgfqpoint{1.061307in}{4.012216in}}%
\pgfpathlineto{\pgfqpoint{1.061341in}{4.012235in}}%
\pgfpathlineto{\pgfqpoint{1.061660in}{4.012378in}}%
\pgfpathlineto{\pgfqpoint{1.061962in}{4.011520in}}%
\pgfpathlineto{\pgfqpoint{1.061962in}{4.011520in}}%
\pgfpathlineto{\pgfqpoint{1.063645in}{4.009889in}}%
\pgfpathlineto{\pgfqpoint{1.065967in}{4.006527in}}%
\pgfpathlineto{\pgfqpoint{1.066772in}{4.005529in}}%
\pgfpathlineto{\pgfqpoint{1.069804in}{4.001985in}}%
\pgfpathlineto{\pgfqpoint{1.069907in}{4.002082in}}%
\pgfpathlineto{\pgfqpoint{1.070112in}{4.002141in}}%
\pgfpathlineto{\pgfqpoint{1.070298in}{4.001726in}}%
\pgfpathlineto{\pgfqpoint{1.070298in}{4.001726in}}%
\pgfpathlineto{\pgfqpoint{1.071645in}{3.999676in}}%
\pgfpathlineto{\pgfqpoint{1.072022in}{3.999831in}}%
\pgfpathlineto{\pgfqpoint{1.088162in}{3.980268in}}%
\pgfpathlineto{\pgfqpoint{1.089508in}{3.978138in}}%
\pgfpathlineto{\pgfqpoint{1.091593in}{3.975464in}}%
\pgfpathlineto{\pgfqpoint{1.095356in}{3.970958in}}%
\pgfpathlineto{\pgfqpoint{1.097515in}{3.968430in}}%
\pgfpathlineto{\pgfqpoint{1.097692in}{3.968579in}}%
\pgfpathlineto{\pgfqpoint{1.097906in}{3.968478in}}%
\pgfpathlineto{\pgfqpoint{1.098115in}{3.967872in}}%
\pgfpathlineto{\pgfqpoint{1.098115in}{3.967872in}}%
\pgfpathlineto{\pgfqpoint{1.099424in}{3.966074in}}%
\pgfpathlineto{\pgfqpoint{1.099740in}{3.966253in}}%
\pgfpathlineto{\pgfqpoint{1.101160in}{3.963942in}}%
\pgfpathlineto{\pgfqpoint{1.103340in}{3.961386in}}%
\pgfpathlineto{\pgfqpoint{1.103482in}{3.961506in}}%
\pgfpathlineto{\pgfqpoint{1.103690in}{3.961492in}}%
\pgfpathlineto{\pgfqpoint{1.103864in}{3.961075in}}%
\pgfpathlineto{\pgfqpoint{1.103864in}{3.961075in}}%
\pgfpathlineto{\pgfqpoint{1.105259in}{3.959018in}}%
\pgfpathlineto{\pgfqpoint{1.105577in}{3.959191in}}%
\pgfpathlineto{\pgfqpoint{1.107834in}{3.956124in}}%
\pgfpathlineto{\pgfqpoint{1.109164in}{3.954293in}}%
\pgfpathlineto{\pgfqpoint{1.109484in}{3.954463in}}%
\pgfpathlineto{\pgfqpoint{1.111730in}{3.951448in}}%
\pgfpathlineto{\pgfqpoint{1.113082in}{3.949552in}}%
\pgfpathlineto{\pgfqpoint{1.113403in}{3.949720in}}%
\pgfpathlineto{\pgfqpoint{1.115642in}{3.946747in}}%
\pgfpathlineto{\pgfqpoint{1.117033in}{3.944800in}}%
\pgfpathlineto{\pgfqpoint{1.117319in}{3.944966in}}%
\pgfpathlineto{\pgfqpoint{1.118629in}{3.943063in}}%
\pgfpathlineto{\pgfqpoint{1.119967in}{3.941214in}}%
\pgfpathlineto{\pgfqpoint{1.120290in}{3.941381in}}%
\pgfpathlineto{\pgfqpoint{1.121700in}{3.939143in}}%
\pgfpathlineto{\pgfqpoint{1.123928in}{3.936427in}}%
\pgfpathlineto{\pgfqpoint{1.124450in}{3.936243in}}%
\pgfpathlineto{\pgfqpoint{1.124667in}{3.935561in}}%
\pgfpathlineto{\pgfqpoint{1.126873in}{3.932825in}}%
\pgfpathlineto{\pgfqpoint{1.127715in}{3.931788in}}%
\pgfpathlineto{\pgfqpoint{1.130924in}{3.928022in}}%
\pgfpathlineto{\pgfqpoint{1.130996in}{3.928073in}}%
\pgfpathlineto{\pgfqpoint{1.131264in}{3.928123in}}%
\pgfpathlineto{\pgfqpoint{1.131521in}{3.927477in}}%
\pgfpathlineto{\pgfqpoint{1.131521in}{3.927477in}}%
\pgfpathlineto{\pgfqpoint{1.132881in}{3.925594in}}%
\pgfpathlineto{\pgfqpoint{1.133208in}{3.925751in}}%
\pgfpathlineto{\pgfqpoint{1.134693in}{3.923352in}}%
\pgfpathlineto{\pgfqpoint{1.137231in}{3.920894in}}%
\pgfpathlineto{\pgfqpoint{1.140697in}{3.916106in}}%
\pgfpathlineto{\pgfqpoint{1.142942in}{3.913461in}}%
\pgfpathlineto{\pgfqpoint{1.143127in}{3.913591in}}%
\pgfpathlineto{\pgfqpoint{1.143370in}{3.913450in}}%
\pgfpathlineto{\pgfqpoint{1.143626in}{3.912703in}}%
\pgfpathlineto{\pgfqpoint{1.145347in}{3.911081in}}%
\pgfpathlineto{\pgfqpoint{1.150039in}{3.904905in}}%
\pgfpathlineto{\pgfqpoint{1.150150in}{3.904983in}}%
\pgfpathlineto{\pgfqpoint{1.150401in}{3.904970in}}%
\pgfpathlineto{\pgfqpoint{1.150619in}{3.904427in}}%
\pgfpathlineto{\pgfqpoint{1.150619in}{3.904427in}}%
\pgfpathlineto{\pgfqpoint{1.152065in}{3.902446in}}%
\pgfpathlineto{\pgfqpoint{1.152325in}{3.902590in}}%
\pgfpathlineto{\pgfqpoint{1.152937in}{3.901255in}}%
\pgfpathlineto{\pgfqpoint{1.157822in}{3.895572in}}%
\pgfpathlineto{\pgfqpoint{1.159208in}{3.893814in}}%
\pgfpathlineto{\pgfqpoint{1.159507in}{3.893939in}}%
\pgfpathlineto{\pgfqpoint{1.161880in}{3.890737in}}%
\pgfpathlineto{\pgfqpoint{1.163303in}{3.888853in}}%
\pgfpathlineto{\pgfqpoint{1.163603in}{3.888981in}}%
\pgfpathlineto{\pgfqpoint{1.165116in}{3.886575in}}%
\pgfpathlineto{\pgfqpoint{1.167424in}{3.883877in}}%
\pgfpathlineto{\pgfqpoint{1.167613in}{3.883999in}}%
\pgfpathlineto{\pgfqpoint{1.167968in}{3.883535in}}%
\pgfpathlineto{\pgfqpoint{1.168120in}{3.883064in}}%
\pgfpathlineto{\pgfqpoint{1.169890in}{3.881385in}}%
\pgfpathlineto{\pgfqpoint{1.173664in}{3.876396in}}%
\pgfpathlineto{\pgfqpoint{1.174000in}{3.876427in}}%
\pgfpathlineto{\pgfqpoint{1.174338in}{3.875528in}}%
\pgfpathlineto{\pgfqpoint{1.176568in}{3.872683in}}%
\pgfpathlineto{\pgfqpoint{1.190648in}{3.856277in}}%
\pgfpathlineto{\pgfqpoint{1.193051in}{3.853128in}}%
\pgfpathlineto{\pgfqpoint{1.194566in}{3.851070in}}%
\pgfpathlineto{\pgfqpoint{1.194836in}{3.851187in}}%
\pgfpathlineto{\pgfqpoint{1.196149in}{3.849509in}}%
\pgfpathlineto{\pgfqpoint{1.198104in}{3.847281in}}%
\pgfpathlineto{\pgfqpoint{1.545630in}{3.428163in}}%
\pgfpathlineto{\pgfqpoint{1.552050in}{3.420194in}}%
\pgfpathlineto{\pgfqpoint{1.554455in}{3.417473in}}%
\pgfpathlineto{\pgfqpoint{1.556234in}{3.415414in}}%
\pgfpathlineto{\pgfqpoint{1.561833in}{3.408459in}}%
\pgfpathlineto{\pgfqpoint{1.562756in}{3.407162in}}%
\pgfpathlineto{\pgfqpoint{1.566192in}{3.403042in}}%
\pgfpathlineto{\pgfqpoint{1.567438in}{3.401466in}}%
\pgfpathlineto{\pgfqpoint{1.577158in}{3.390127in}}%
\pgfpathlineto{\pgfqpoint{1.578082in}{3.388697in}}%
\pgfpathlineto{\pgfqpoint{1.583105in}{3.382771in}}%
\pgfpathlineto{\pgfqpoint{1.584189in}{3.381394in}}%
\pgfpathlineto{\pgfqpoint{1.587783in}{3.377179in}}%
\pgfpathlineto{\pgfqpoint{1.588803in}{3.375879in}}%
\pgfpathlineto{\pgfqpoint{1.592265in}{3.371657in}}%
\pgfpathlineto{\pgfqpoint{1.594889in}{3.368705in}}%
\pgfpathlineto{\pgfqpoint{1.597415in}{3.365852in}}%
\pgfpathlineto{\pgfqpoint{1.599424in}{3.363412in}}%
\pgfpathlineto{\pgfqpoint{1.602250in}{3.360104in}}%
\pgfpathlineto{\pgfqpoint{1.607665in}{3.353204in}}%
\pgfpathlineto{\pgfqpoint{1.611285in}{3.348883in}}%
\pgfpathlineto{\pgfqpoint{1.612544in}{3.347287in}}%
\pgfpathlineto{\pgfqpoint{1.619490in}{3.339222in}}%
\pgfpathlineto{\pgfqpoint{1.620460in}{3.337824in}}%
\pgfpathlineto{\pgfqpoint{1.624266in}{3.333439in}}%
\pgfpathlineto{\pgfqpoint{1.625270in}{3.332063in}}%
\pgfpathlineto{\pgfqpoint{1.628997in}{3.327651in}}%
\pgfpathlineto{\pgfqpoint{1.630181in}{3.326137in}}%
\pgfpathlineto{\pgfqpoint{1.634262in}{3.321640in}}%
\pgfpathlineto{\pgfqpoint{1.637913in}{3.317192in}}%
\pgfpathlineto{\pgfqpoint{1.640911in}{3.313688in}}%
\pgfpathlineto{\pgfqpoint{1.656355in}{3.294834in}}%
\pgfpathlineto{\pgfqpoint{1.659078in}{3.291869in}}%
\pgfpathlineto{\pgfqpoint{1.664642in}{3.284974in}}%
\pgfpathlineto{\pgfqpoint{1.667398in}{3.281847in}}%
\pgfpathlineto{\pgfqpoint{1.669707in}{3.278898in}}%
\pgfpathlineto{\pgfqpoint{1.672465in}{3.275760in}}%
\pgfpathlineto{\pgfqpoint{1.674757in}{3.272876in}}%
\pgfpathlineto{\pgfqpoint{1.677559in}{3.269642in}}%
\pgfpathlineto{\pgfqpoint{1.679812in}{3.266866in}}%
\pgfpathlineto{\pgfqpoint{1.682711in}{3.263471in}}%
\pgfpathlineto{\pgfqpoint{1.685063in}{3.260485in}}%
\pgfpathlineto{\pgfqpoint{1.687877in}{3.257277in}}%
\pgfpathlineto{\pgfqpoint{1.690269in}{3.254219in}}%
\pgfpathlineto{\pgfqpoint{1.693052in}{3.251061in}}%
\pgfpathlineto{\pgfqpoint{1.695349in}{3.248230in}}%
\pgfpathlineto{\pgfqpoint{1.698322in}{3.244767in}}%
\pgfpathlineto{\pgfqpoint{1.700885in}{3.241364in}}%
\pgfpathlineto{\pgfqpoint{1.704862in}{3.236592in}}%
\pgfpathlineto{\pgfqpoint{1.707343in}{3.233990in}}%
\pgfpathlineto{\pgfqpoint{1.714298in}{3.225656in}}%
\pgfpathlineto{\pgfqpoint{1.725599in}{3.211833in}}%
\pgfpathlineto{\pgfqpoint{1.728539in}{3.208569in}}%
\pgfpathlineto{\pgfqpoint{1.732958in}{3.202926in}}%
\pgfpathlineto{\pgfqpoint{1.737165in}{3.197951in}}%
\pgfpathlineto{\pgfqpoint{1.738656in}{3.196031in}}%
\pgfpathlineto{\pgfqpoint{1.764402in}{3.165232in}}%
\pgfpathlineto{\pgfqpoint{1.770702in}{3.157902in}}%
\pgfpathlineto{\pgfqpoint{1.772112in}{3.155978in}}%
\pgfpathlineto{\pgfqpoint{1.785077in}{3.140592in}}%
\pgfpathlineto{\pgfqpoint{1.788382in}{3.136930in}}%
\pgfpathlineto{\pgfqpoint{1.797710in}{3.125684in}}%
\pgfpathlineto{\pgfqpoint{1.800244in}{3.122657in}}%
\pgfpathlineto{\pgfqpoint{1.804004in}{3.118224in}}%
\pgfpathlineto{\pgfqpoint{1.809781in}{3.111333in}}%
\pgfpathlineto{\pgfqpoint{1.831318in}{3.085424in}}%
\pgfpathlineto{\pgfqpoint{1.833742in}{3.082684in}}%
\pgfpathlineto{\pgfqpoint{1.867523in}{3.042026in}}%
\pgfpathlineto{\pgfqpoint{1.872178in}{3.036369in}}%
\pgfpathlineto{\pgfqpoint{1.882124in}{3.024749in}}%
\pgfpathlineto{\pgfqpoint{1.885869in}{3.020406in}}%
\pgfpathlineto{\pgfqpoint{1.895462in}{3.008630in}}%
\pgfpathlineto{\pgfqpoint{1.900725in}{3.002540in}}%
\pgfpathlineto{\pgfqpoint{1.903483in}{2.999387in}}%
\pgfpathlineto{\pgfqpoint{1.912084in}{2.989123in}}%
\pgfpathlineto{\pgfqpoint{1.922016in}{2.976944in}}%
\pgfpathlineto{\pgfqpoint{1.927994in}{2.970149in}}%
\pgfpathlineto{\pgfqpoint{1.936731in}{2.959704in}}%
\pgfpathlineto{\pgfqpoint{1.942024in}{2.953294in}}%
\pgfpathlineto{\pgfqpoint{1.946034in}{2.948645in}}%
\pgfpathlineto{\pgfqpoint{1.958696in}{2.933234in}}%
\pgfpathlineto{\pgfqpoint{1.966200in}{2.924282in}}%
\pgfpathlineto{\pgfqpoint{1.972213in}{2.917418in}}%
\pgfpathlineto{\pgfqpoint{1.978363in}{2.909894in}}%
\pgfpathlineto{\pgfqpoint{1.981699in}{2.906148in}}%
\pgfpathlineto{\pgfqpoint{1.988770in}{2.897753in}}%
\pgfpathlineto{\pgfqpoint{2.030792in}{2.847717in}}%
\pgfpathlineto{\pgfqpoint{2.036510in}{2.840928in}}%
\pgfpathlineto{\pgfqpoint{2.043355in}{2.832671in}}%
\pgfpathlineto{\pgfqpoint{2.046913in}{2.828552in}}%
\pgfpathlineto{\pgfqpoint{2.053508in}{2.820497in}}%
\pgfpathlineto{\pgfqpoint{2.057691in}{2.815643in}}%
\pgfpathlineto{\pgfqpoint{2.062392in}{2.810218in}}%
\pgfpathlineto{\pgfqpoint{2.085999in}{2.782055in}}%
\pgfpathlineto{\pgfqpoint{2.089862in}{2.777507in}}%
\pgfpathlineto{\pgfqpoint{2.096020in}{2.769992in}}%
\pgfpathlineto{\pgfqpoint{2.108042in}{2.755862in}}%
\pgfpathlineto{\pgfqpoint{2.112077in}{2.751183in}}%
\pgfpathlineto{\pgfqpoint{2.118846in}{2.742919in}}%
\pgfpathlineto{\pgfqpoint{2.129805in}{2.729975in}}%
\pgfpathlineto{\pgfqpoint{2.137185in}{2.721484in}}%
\pgfpathlineto{\pgfqpoint{2.147197in}{2.709412in}}%
\pgfpathlineto{\pgfqpoint{2.154207in}{2.701136in}}%
\pgfpathlineto{\pgfqpoint{2.158780in}{2.695804in}}%
\pgfpathlineto{\pgfqpoint{2.165269in}{2.687984in}}%
\pgfpathlineto{\pgfqpoint{2.175010in}{2.676442in}}%
\pgfpathlineto{\pgfqpoint{2.183368in}{2.666639in}}%
\pgfpathlineto{\pgfqpoint{2.190569in}{2.658067in}}%
\pgfpathlineto{\pgfqpoint{2.198034in}{2.649508in}}%
\pgfpathlineto{\pgfqpoint{2.273804in}{2.560063in}}%
\pgfpathlineto{\pgfqpoint{2.285361in}{2.546388in}}%
\pgfpathlineto{\pgfqpoint{2.293863in}{2.536206in}}%
\pgfpathlineto{\pgfqpoint{2.306437in}{2.521657in}}%
\pgfpathlineto{\pgfqpoint{2.320902in}{2.504657in}}%
\pgfpathlineto{\pgfqpoint{2.366599in}{2.450979in}}%
\pgfpathlineto{\pgfqpoint{2.404324in}{2.406636in}}%
\pgfpathlineto{\pgfqpoint{2.414378in}{2.394799in}}%
\pgfpathlineto{\pgfqpoint{2.424880in}{2.382674in}}%
\pgfpathlineto{\pgfqpoint{2.439293in}{2.365616in}}%
\pgfpathlineto{\pgfqpoint{2.481679in}{2.316322in}}%
\pgfpathlineto{\pgfqpoint{2.493698in}{2.302147in}}%
\pgfpathlineto{\pgfqpoint{2.518898in}{2.272832in}}%
\pgfpathlineto{\pgfqpoint{2.532248in}{2.257329in}}%
\pgfpathlineto{\pgfqpoint{2.548287in}{2.238638in}}%
\pgfpathlineto{\pgfqpoint{2.681356in}{2.085026in}}%
\pgfpathlineto{\pgfqpoint{3.029107in}{1.691742in}}%
\pgfpathlineto{\pgfqpoint{3.093772in}{1.620546in}}%
\pgfpathlineto{\pgfqpoint{3.156294in}{1.552363in}}%
\pgfpathlineto{\pgfqpoint{3.379387in}{1.317039in}}%
\pgfpathlineto{\pgfqpoint{3.491272in}{1.204935in}}%
\pgfpathlineto{\pgfqpoint{3.563419in}{1.135429in}}%
\pgfpathlineto{\pgfqpoint{3.629831in}{1.073888in}}%
\pgfpathlineto{\pgfqpoint{3.695957in}{1.015160in}}%
\pgfpathlineto{\pgfqpoint{3.761023in}{0.960434in}}%
\pgfpathlineto{\pgfqpoint{3.821442in}{0.912657in}}%
\pgfpathlineto{\pgfqpoint{3.877939in}{0.871024in}}%
\pgfpathlineto{\pgfqpoint{3.929982in}{0.835655in}}%
\pgfpathlineto{\pgfqpoint{3.973908in}{0.808269in}}%
\pgfpathlineto{\pgfqpoint{4.017415in}{0.783548in}}%
\pgfpathlineto{\pgfqpoint{4.057197in}{0.763248in}}%
\pgfpathlineto{\pgfqpoint{4.096076in}{0.745712in}}%
\pgfpathlineto{\pgfqpoint{4.133788in}{0.730965in}}%
\pgfpathlineto{\pgfqpoint{4.170656in}{0.718904in}}%
\pgfpathlineto{\pgfqpoint{4.208553in}{0.709058in}}%
\pgfpathlineto{\pgfqpoint{4.241018in}{0.702735in}}%
\pgfpathlineto{\pgfqpoint{4.276930in}{0.698121in}}%
\pgfpathlineto{\pgfqpoint{4.307134in}{0.696247in}}%
\pgfpathlineto{\pgfqpoint{4.342920in}{0.696401in}}%
\pgfpathlineto{\pgfqpoint{4.372480in}{0.698448in}}%
\pgfpathlineto{\pgfqpoint{4.407767in}{0.703156in}}%
\pgfpathlineto{\pgfqpoint{4.438823in}{0.709318in}}%
\pgfpathlineto{\pgfqpoint{4.474194in}{0.718539in}}%
\pgfpathlineto{\pgfqpoint{4.508659in}{0.729650in}}%
\pgfpathlineto{\pgfqpoint{4.545768in}{0.743885in}}%
\pgfpathlineto{\pgfqpoint{4.585826in}{0.761766in}}%
\pgfpathlineto{\pgfqpoint{4.625973in}{0.782054in}}%
\pgfpathlineto{\pgfqpoint{4.669487in}{0.806624in}}%
\pgfpathlineto{\pgfqpoint{4.717367in}{0.836423in}}%
\pgfpathlineto{\pgfqpoint{4.765953in}{0.869451in}}%
\pgfpathlineto{\pgfqpoint{4.818456in}{0.907884in}}%
\pgfpathlineto{\pgfqpoint{4.877569in}{0.954266in}}%
\pgfpathlineto{\pgfqpoint{4.941717in}{1.007885in}}%
\pgfpathlineto{\pgfqpoint{5.009614in}{1.067754in}}%
\pgfpathlineto{\pgfqpoint{5.078636in}{1.131576in}}%
\pgfpathlineto{\pgfqpoint{5.147748in}{1.197914in}}%
\pgfpathlineto{\pgfqpoint{5.219354in}{1.268943in}}%
\pgfpathlineto{\pgfqpoint{5.332298in}{1.384730in}}%
\pgfpathlineto{\pgfqpoint{5.496873in}{1.559932in}}%
\pgfpathlineto{\pgfqpoint{5.534545in}{1.600997in}}%
\pgfpathlineto{\pgfqpoint{5.534545in}{1.600997in}}%
\pgfusepath{stroke}%
\end{pgfscope}%
\begin{pgfscope}%
\pgfsetrectcap%
\pgfsetmiterjoin%
\pgfsetlinewidth{0.803000pt}%
\definecolor{currentstroke}{rgb}{0.000000,0.000000,0.000000}%
\pgfsetstrokecolor{currentstroke}%
\pgfsetdash{}{0pt}%
\pgfpathmoveto{\pgfqpoint{0.800000in}{0.528000in}}%
\pgfpathlineto{\pgfqpoint{0.800000in}{4.224000in}}%
\pgfusepath{stroke}%
\end{pgfscope}%
\begin{pgfscope}%
\pgfsetrectcap%
\pgfsetmiterjoin%
\pgfsetlinewidth{0.803000pt}%
\definecolor{currentstroke}{rgb}{0.000000,0.000000,0.000000}%
\pgfsetstrokecolor{currentstroke}%
\pgfsetdash{}{0pt}%
\pgfpathmoveto{\pgfqpoint{5.760000in}{0.528000in}}%
\pgfpathlineto{\pgfqpoint{5.760000in}{4.224000in}}%
\pgfusepath{stroke}%
\end{pgfscope}%
\begin{pgfscope}%
\pgfsetrectcap%
\pgfsetmiterjoin%
\pgfsetlinewidth{0.803000pt}%
\definecolor{currentstroke}{rgb}{0.000000,0.000000,0.000000}%
\pgfsetstrokecolor{currentstroke}%
\pgfsetdash{}{0pt}%
\pgfpathmoveto{\pgfqpoint{0.800000in}{0.528000in}}%
\pgfpathlineto{\pgfqpoint{5.760000in}{0.528000in}}%
\pgfusepath{stroke}%
\end{pgfscope}%
\begin{pgfscope}%
\pgfsetrectcap%
\pgfsetmiterjoin%
\pgfsetlinewidth{0.803000pt}%
\definecolor{currentstroke}{rgb}{0.000000,0.000000,0.000000}%
\pgfsetstrokecolor{currentstroke}%
\pgfsetdash{}{0pt}%
\pgfpathmoveto{\pgfqpoint{0.800000in}{4.224000in}}%
\pgfpathlineto{\pgfqpoint{5.760000in}{4.224000in}}%
\pgfusepath{stroke}%
\end{pgfscope}%
\begin{pgfscope}%
\pgfsetbuttcap%
\pgfsetroundjoin%
\definecolor{currentfill}{rgb}{0.000000,0.000000,0.000000}%
\pgfsetfillcolor{currentfill}%
\pgfsetlinewidth{0.803000pt}%
\definecolor{currentstroke}{rgb}{0.000000,0.000000,0.000000}%
\pgfsetstrokecolor{currentstroke}%
\pgfsetdash{}{0pt}%
\pgfsys@defobject{currentmarker}{\pgfqpoint{0.000000in}{0.000000in}}{\pgfqpoint{0.048611in}{0.000000in}}{%
\pgfpathmoveto{\pgfqpoint{0.000000in}{0.000000in}}%
\pgfpathlineto{\pgfqpoint{0.048611in}{0.000000in}}%
\pgfusepath{stroke,fill}%
}%
\begin{pgfscope}%
\pgfsys@transformshift{5.760000in}{0.695973in}%
\pgfsys@useobject{currentmarker}{}%
\end{pgfscope}%
\end{pgfscope}%
\begin{pgfscope}%
\pgftext[x=5.857222in,y=0.647778in,left,base]{\sffamily\fontsize{10.000000}{12.000000}\selectfont \(\displaystyle 0\)}%
\end{pgfscope}%
\begin{pgfscope}%
\pgfsetbuttcap%
\pgfsetroundjoin%
\definecolor{currentfill}{rgb}{0.000000,0.000000,0.000000}%
\pgfsetfillcolor{currentfill}%
\pgfsetlinewidth{0.803000pt}%
\definecolor{currentstroke}{rgb}{0.000000,0.000000,0.000000}%
\pgfsetstrokecolor{currentstroke}%
\pgfsetdash{}{0pt}%
\pgfsys@defobject{currentmarker}{\pgfqpoint{0.000000in}{0.000000in}}{\pgfqpoint{0.048611in}{0.000000in}}{%
\pgfpathmoveto{\pgfqpoint{0.000000in}{0.000000in}}%
\pgfpathlineto{\pgfqpoint{0.048611in}{0.000000in}}%
\pgfusepath{stroke,fill}%
}%
\begin{pgfscope}%
\pgfsys@transformshift{5.760000in}{1.442645in}%
\pgfsys@useobject{currentmarker}{}%
\end{pgfscope}%
\end{pgfscope}%
\begin{pgfscope}%
\pgftext[x=5.857222in,y=1.394451in,left,base]{\sffamily\fontsize{10.000000}{12.000000}\selectfont \(\displaystyle 20\)}%
\end{pgfscope}%
\begin{pgfscope}%
\pgfsetbuttcap%
\pgfsetroundjoin%
\definecolor{currentfill}{rgb}{0.000000,0.000000,0.000000}%
\pgfsetfillcolor{currentfill}%
\pgfsetlinewidth{0.803000pt}%
\definecolor{currentstroke}{rgb}{0.000000,0.000000,0.000000}%
\pgfsetstrokecolor{currentstroke}%
\pgfsetdash{}{0pt}%
\pgfsys@defobject{currentmarker}{\pgfqpoint{0.000000in}{0.000000in}}{\pgfqpoint{0.048611in}{0.000000in}}{%
\pgfpathmoveto{\pgfqpoint{0.000000in}{0.000000in}}%
\pgfpathlineto{\pgfqpoint{0.048611in}{0.000000in}}%
\pgfusepath{stroke,fill}%
}%
\begin{pgfscope}%
\pgfsys@transformshift{5.760000in}{2.189318in}%
\pgfsys@useobject{currentmarker}{}%
\end{pgfscope}%
\end{pgfscope}%
\begin{pgfscope}%
\pgftext[x=5.857222in,y=2.141124in,left,base]{\sffamily\fontsize{10.000000}{12.000000}\selectfont \(\displaystyle 40\)}%
\end{pgfscope}%
\begin{pgfscope}%
\pgfsetbuttcap%
\pgfsetroundjoin%
\definecolor{currentfill}{rgb}{0.000000,0.000000,0.000000}%
\pgfsetfillcolor{currentfill}%
\pgfsetlinewidth{0.803000pt}%
\definecolor{currentstroke}{rgb}{0.000000,0.000000,0.000000}%
\pgfsetstrokecolor{currentstroke}%
\pgfsetdash{}{0pt}%
\pgfsys@defobject{currentmarker}{\pgfqpoint{0.000000in}{0.000000in}}{\pgfqpoint{0.048611in}{0.000000in}}{%
\pgfpathmoveto{\pgfqpoint{0.000000in}{0.000000in}}%
\pgfpathlineto{\pgfqpoint{0.048611in}{0.000000in}}%
\pgfusepath{stroke,fill}%
}%
\begin{pgfscope}%
\pgfsys@transformshift{5.760000in}{2.935991in}%
\pgfsys@useobject{currentmarker}{}%
\end{pgfscope}%
\end{pgfscope}%
\begin{pgfscope}%
\pgftext[x=5.857222in,y=2.887796in,left,base]{\sffamily\fontsize{10.000000}{12.000000}\selectfont \(\displaystyle 60\)}%
\end{pgfscope}%
\begin{pgfscope}%
\pgfsetbuttcap%
\pgfsetroundjoin%
\definecolor{currentfill}{rgb}{0.000000,0.000000,0.000000}%
\pgfsetfillcolor{currentfill}%
\pgfsetlinewidth{0.803000pt}%
\definecolor{currentstroke}{rgb}{0.000000,0.000000,0.000000}%
\pgfsetstrokecolor{currentstroke}%
\pgfsetdash{}{0pt}%
\pgfsys@defobject{currentmarker}{\pgfqpoint{0.000000in}{0.000000in}}{\pgfqpoint{0.048611in}{0.000000in}}{%
\pgfpathmoveto{\pgfqpoint{0.000000in}{0.000000in}}%
\pgfpathlineto{\pgfqpoint{0.048611in}{0.000000in}}%
\pgfusepath{stroke,fill}%
}%
\begin{pgfscope}%
\pgfsys@transformshift{5.760000in}{3.682664in}%
\pgfsys@useobject{currentmarker}{}%
\end{pgfscope}%
\end{pgfscope}%
\begin{pgfscope}%
\pgftext[x=5.857222in,y=3.634469in,left,base]{\sffamily\fontsize{10.000000}{12.000000}\selectfont \(\displaystyle 80\)}%
\end{pgfscope}%
\begin{pgfscope}%
\pgftext[x=6.051667in,y=2.376000in,,top,rotate=90.000000]{\sffamily\fontsize{10.000000}{12.000000}\selectfont Inclination, degrees}%
\end{pgfscope}%
\begin{pgfscope}%
\pgfpathrectangle{\pgfqpoint{0.800000in}{0.528000in}}{\pgfqpoint{4.960000in}{3.696000in}} %
\pgfusepath{clip}%
\pgfsetrectcap%
\pgfsetroundjoin%
\pgfsetlinewidth{1.505625pt}%
\definecolor{currentstroke}{rgb}{0.000000,0.000000,0.000000}%
\pgfsetstrokecolor{currentstroke}%
\pgfsetdash{}{0pt}%
\pgfpathmoveto{\pgfqpoint{1.025455in}{4.056000in}}%
\pgfpathlineto{\pgfqpoint{1.051346in}{4.053993in}}%
\pgfpathlineto{\pgfqpoint{1.073288in}{4.052270in}}%
\pgfpathlineto{\pgfqpoint{1.145840in}{4.046416in}}%
\pgfpathlineto{\pgfqpoint{1.164452in}{4.044870in}}%
\pgfpathlineto{\pgfqpoint{1.190228in}{4.042706in}}%
\pgfpathlineto{\pgfqpoint{1.222479in}{4.039942in}}%
\pgfpathlineto{\pgfqpoint{1.245735in}{4.037911in}}%
\pgfpathlineto{\pgfqpoint{1.266220in}{4.036106in}}%
\pgfpathlineto{\pgfqpoint{1.289271in}{4.034038in}}%
\pgfpathlineto{\pgfqpoint{1.312965in}{4.031892in}}%
\pgfpathlineto{\pgfqpoint{1.334871in}{4.029877in}}%
\pgfpathlineto{\pgfqpoint{1.424527in}{4.021301in}}%
\pgfpathlineto{\pgfqpoint{1.448223in}{4.018960in}}%
\pgfpathlineto{\pgfqpoint{1.534998in}{4.010024in}}%
\pgfpathlineto{\pgfqpoint{1.554758in}{4.007924in}}%
\pgfpathlineto{\pgfqpoint{1.595654in}{4.003463in}}%
\pgfpathlineto{\pgfqpoint{1.614193in}{4.001398in}}%
\pgfpathlineto{\pgfqpoint{1.626554in}{4.000020in}}%
\pgfpathlineto{\pgfqpoint{1.649318in}{3.997432in}}%
\pgfpathlineto{\pgfqpoint{1.716575in}{3.989520in}}%
\pgfpathlineto{\pgfqpoint{1.733818in}{3.987422in}}%
\pgfpathlineto{\pgfqpoint{1.786460in}{3.980898in}}%
\pgfpathlineto{\pgfqpoint{1.803492in}{3.978711in}}%
\pgfpathlineto{\pgfqpoint{1.826562in}{3.975722in}}%
\pgfpathlineto{\pgfqpoint{1.843685in}{3.973471in}}%
\pgfpathlineto{\pgfqpoint{2.112493in}{3.933910in}}%
\pgfpathlineto{\pgfqpoint{2.131372in}{3.930785in}}%
\pgfpathlineto{\pgfqpoint{2.150031in}{3.927670in}}%
\pgfpathlineto{\pgfqpoint{2.164173in}{3.925285in}}%
\pgfpathlineto{\pgfqpoint{2.174025in}{3.923580in}}%
\pgfpathlineto{\pgfqpoint{2.186862in}{3.921333in}}%
\pgfpathlineto{\pgfqpoint{2.199115in}{3.919184in}}%
\pgfpathlineto{\pgfqpoint{2.236835in}{3.912459in}}%
\pgfpathlineto{\pgfqpoint{2.245519in}{3.910896in}}%
\pgfpathlineto{\pgfqpoint{2.284980in}{3.903528in}}%
\pgfpathlineto{\pgfqpoint{2.344785in}{3.891781in}}%
\pgfpathlineto{\pgfqpoint{2.372950in}{3.886123in}}%
\pgfpathlineto{\pgfqpoint{2.387886in}{3.882989in}}%
\pgfpathlineto{\pgfqpoint{2.398642in}{3.880692in}}%
\pgfpathlineto{\pgfqpoint{2.442846in}{3.871212in}}%
\pgfpathlineto{\pgfqpoint{2.452750in}{3.869026in}}%
\pgfpathlineto{\pgfqpoint{2.457984in}{3.867907in}}%
\pgfpathlineto{\pgfqpoint{2.470771in}{3.864980in}}%
\pgfpathlineto{\pgfqpoint{2.476304in}{3.863776in}}%
\pgfpathlineto{\pgfqpoint{2.486972in}{3.861289in}}%
\pgfpathlineto{\pgfqpoint{2.492654in}{3.860034in}}%
\pgfpathlineto{\pgfqpoint{2.503381in}{3.857464in}}%
\pgfpathlineto{\pgfqpoint{2.511655in}{3.855626in}}%
\pgfpathlineto{\pgfqpoint{2.525402in}{3.852279in}}%
\pgfpathlineto{\pgfqpoint{2.531873in}{3.850772in}}%
\pgfpathlineto{\pgfqpoint{2.540733in}{3.848653in}}%
\pgfpathlineto{\pgfqpoint{2.546917in}{3.847157in}}%
\pgfpathlineto{\pgfqpoint{2.555897in}{3.844908in}}%
\pgfpathlineto{\pgfqpoint{2.562342in}{3.843390in}}%
\pgfpathlineto{\pgfqpoint{2.571581in}{3.841013in}}%
\pgfpathlineto{\pgfqpoint{2.578268in}{3.839421in}}%
\pgfpathlineto{\pgfqpoint{2.587520in}{3.836983in}}%
\pgfpathlineto{\pgfqpoint{2.600176in}{3.833848in}}%
\pgfpathlineto{\pgfqpoint{2.608371in}{3.831759in}}%
\pgfpathlineto{\pgfqpoint{2.618348in}{3.829051in}}%
\pgfpathlineto{\pgfqpoint{2.625429in}{3.827307in}}%
\pgfpathlineto{\pgfqpoint{2.635696in}{3.824467in}}%
\pgfpathlineto{\pgfqpoint{2.642940in}{3.822656in}}%
\pgfpathlineto{\pgfqpoint{2.653532in}{3.819667in}}%
\pgfpathlineto{\pgfqpoint{2.660786in}{3.817840in}}%
\pgfpathlineto{\pgfqpoint{2.675396in}{3.813744in}}%
\pgfpathlineto{\pgfqpoint{2.680205in}{3.812368in}}%
\pgfpathlineto{\pgfqpoint{2.688001in}{3.810219in}}%
\pgfpathlineto{\pgfqpoint{2.692957in}{3.808776in}}%
\pgfpathlineto{\pgfqpoint{2.700877in}{3.806573in}}%
\pgfpathlineto{\pgfqpoint{2.705896in}{3.805093in}}%
\pgfpathlineto{\pgfqpoint{2.714016in}{3.802798in}}%
\pgfpathlineto{\pgfqpoint{2.718829in}{3.801407in}}%
\pgfpathlineto{\pgfqpoint{2.726870in}{3.798942in}}%
\pgfpathlineto{\pgfqpoint{2.735909in}{3.796373in}}%
\pgfpathlineto{\pgfqpoint{2.743761in}{3.793926in}}%
\pgfpathlineto{\pgfqpoint{2.782382in}{3.782240in}}%
\pgfpathlineto{\pgfqpoint{2.788833in}{3.780350in}}%
\pgfpathlineto{\pgfqpoint{2.811328in}{3.773177in}}%
\pgfpathlineto{\pgfqpoint{2.821277in}{3.769750in}}%
\pgfpathlineto{\pgfqpoint{2.831210in}{3.766716in}}%
\pgfpathlineto{\pgfqpoint{2.844959in}{3.762027in}}%
\pgfpathlineto{\pgfqpoint{2.851198in}{3.759951in}}%
\pgfpathlineto{\pgfqpoint{2.854444in}{3.758692in}}%
\pgfpathlineto{\pgfqpoint{2.874742in}{3.751859in}}%
\pgfpathlineto{\pgfqpoint{2.881154in}{3.749682in}}%
\pgfpathlineto{\pgfqpoint{2.884252in}{3.748434in}}%
\pgfpathlineto{\pgfqpoint{2.912931in}{3.738445in}}%
\pgfpathlineto{\pgfqpoint{2.918347in}{3.736587in}}%
\pgfpathlineto{\pgfqpoint{3.070858in}{3.675207in}}%
\pgfpathlineto{\pgfqpoint{3.078452in}{3.671755in}}%
\pgfpathlineto{\pgfqpoint{3.088037in}{3.667983in}}%
\pgfpathlineto{\pgfqpoint{3.097107in}{3.663355in}}%
\pgfpathlineto{\pgfqpoint{3.157896in}{3.634870in}}%
\pgfpathlineto{\pgfqpoint{3.161683in}{3.633217in}}%
\pgfpathlineto{\pgfqpoint{3.164892in}{3.631527in}}%
\pgfpathlineto{\pgfqpoint{3.169041in}{3.629527in}}%
\pgfpathlineto{\pgfqpoint{3.172300in}{3.627863in}}%
\pgfpathlineto{\pgfqpoint{3.176719in}{3.625723in}}%
\pgfpathlineto{\pgfqpoint{3.180027in}{3.623943in}}%
\pgfpathlineto{\pgfqpoint{3.184338in}{3.621839in}}%
\pgfpathlineto{\pgfqpoint{3.187133in}{3.620504in}}%
\pgfpathlineto{\pgfqpoint{3.192049in}{3.617853in}}%
\pgfpathlineto{\pgfqpoint{3.194893in}{3.616546in}}%
\pgfpathlineto{\pgfqpoint{3.199927in}{3.613756in}}%
\pgfpathlineto{\pgfqpoint{3.202823in}{3.612446in}}%
\pgfpathlineto{\pgfqpoint{3.208059in}{3.609541in}}%
\pgfpathlineto{\pgfqpoint{3.210996in}{3.608148in}}%
\pgfpathlineto{\pgfqpoint{3.216328in}{3.605205in}}%
\pgfpathlineto{\pgfqpoint{3.219311in}{3.603723in}}%
\pgfpathlineto{\pgfqpoint{3.224620in}{3.600757in}}%
\pgfpathlineto{\pgfqpoint{3.227657in}{3.599261in}}%
\pgfpathlineto{\pgfqpoint{3.232926in}{3.596184in}}%
\pgfpathlineto{\pgfqpoint{3.235811in}{3.594933in}}%
\pgfpathlineto{\pgfqpoint{3.241611in}{3.591465in}}%
\pgfpathlineto{\pgfqpoint{3.244770in}{3.589956in}}%
\pgfpathlineto{\pgfqpoint{3.250370in}{3.586605in}}%
\pgfpathlineto{\pgfqpoint{3.253532in}{3.585114in}}%
\pgfpathlineto{\pgfqpoint{3.259439in}{3.581587in}}%
\pgfpathlineto{\pgfqpoint{3.262707in}{3.579915in}}%
\pgfpathlineto{\pgfqpoint{3.268400in}{3.576424in}}%
\pgfpathlineto{\pgfqpoint{3.270864in}{3.575484in}}%
\pgfpathlineto{\pgfqpoint{3.278840in}{3.570854in}}%
\pgfpathlineto{\pgfqpoint{3.282947in}{3.567765in}}%
\pgfpathlineto{\pgfqpoint{3.287100in}{3.565571in}}%
\pgfpathlineto{\pgfqpoint{3.289348in}{3.564821in}}%
\pgfpathlineto{\pgfqpoint{3.306817in}{3.553956in}}%
\pgfpathlineto{\pgfqpoint{3.309715in}{3.552628in}}%
\pgfpathlineto{\pgfqpoint{3.316885in}{3.547842in}}%
\pgfpathlineto{\pgfqpoint{3.319850in}{3.546473in}}%
\pgfpathlineto{\pgfqpoint{3.327071in}{3.541506in}}%
\pgfpathlineto{\pgfqpoint{3.329783in}{3.540418in}}%
\pgfpathlineto{\pgfqpoint{3.338539in}{3.534774in}}%
\pgfpathlineto{\pgfqpoint{3.341618in}{3.532749in}}%
\pgfpathlineto{\pgfqpoint{3.347943in}{3.528108in}}%
\pgfpathlineto{\pgfqpoint{3.351316in}{3.526843in}}%
\pgfpathlineto{\pgfqpoint{3.359512in}{3.520975in}}%
\pgfpathlineto{\pgfqpoint{3.362775in}{3.519394in}}%
\pgfpathlineto{\pgfqpoint{3.370674in}{3.513578in}}%
\pgfpathlineto{\pgfqpoint{3.373443in}{3.512498in}}%
\pgfpathlineto{\pgfqpoint{3.383330in}{3.505685in}}%
\pgfpathlineto{\pgfqpoint{3.386726in}{3.503309in}}%
\pgfpathlineto{\pgfqpoint{3.392845in}{3.498025in}}%
\pgfpathlineto{\pgfqpoint{3.398145in}{3.495715in}}%
\pgfpathlineto{\pgfqpoint{3.405311in}{3.489527in}}%
\pgfpathlineto{\pgfqpoint{3.409605in}{3.487964in}}%
\pgfpathlineto{\pgfqpoint{3.419105in}{3.480573in}}%
\pgfpathlineto{\pgfqpoint{3.421908in}{3.479355in}}%
\pgfpathlineto{\pgfqpoint{3.433056in}{3.471119in}}%
\pgfpathlineto{\pgfqpoint{3.436850in}{3.468225in}}%
\pgfpathlineto{\pgfqpoint{3.443696in}{3.461897in}}%
\pgfpathlineto{\pgfqpoint{3.448247in}{3.460331in}}%
\pgfpathlineto{\pgfqpoint{3.459711in}{3.451333in}}%
\pgfpathlineto{\pgfqpoint{3.462734in}{3.449466in}}%
\pgfpathlineto{\pgfqpoint{3.472975in}{3.440717in}}%
\pgfpathlineto{\pgfqpoint{3.475601in}{3.439761in}}%
\pgfpathlineto{\pgfqpoint{3.479784in}{3.435500in}}%
\pgfpathlineto{\pgfqpoint{3.485200in}{3.430037in}}%
\pgfpathlineto{\pgfqpoint{3.493431in}{3.425307in}}%
\pgfpathlineto{\pgfqpoint{3.500128in}{3.418135in}}%
\pgfpathlineto{\pgfqpoint{3.508725in}{3.413088in}}%
\pgfpathlineto{\pgfqpoint{3.515662in}{3.405474in}}%
\pgfpathlineto{\pgfqpoint{3.524596in}{3.400095in}}%
\pgfpathlineto{\pgfqpoint{3.531794in}{3.392004in}}%
\pgfpathlineto{\pgfqpoint{3.541203in}{3.386102in}}%
\pgfpathlineto{\pgfqpoint{3.548679in}{3.377573in}}%
\pgfpathlineto{\pgfqpoint{3.556947in}{3.373083in}}%
\pgfpathlineto{\pgfqpoint{3.567262in}{3.361656in}}%
\pgfpathlineto{\pgfqpoint{3.571904in}{3.360214in}}%
\pgfpathlineto{\pgfqpoint{3.575811in}{3.356044in}}%
\pgfpathlineto{\pgfqpoint{3.583915in}{3.346057in}}%
\pgfpathlineto{\pgfqpoint{3.587651in}{3.344838in}}%
\pgfpathlineto{\pgfqpoint{3.590433in}{3.343549in}}%
\pgfpathlineto{\pgfqpoint{3.594513in}{3.338995in}}%
\pgfpathlineto{\pgfqpoint{3.602996in}{3.328315in}}%
\pgfpathlineto{\pgfqpoint{3.607470in}{3.326943in}}%
\pgfpathlineto{\pgfqpoint{3.610328in}{3.325153in}}%
\pgfpathlineto{\pgfqpoint{3.616041in}{3.317713in}}%
\pgfpathlineto{\pgfqpoint{3.622001in}{3.309957in}}%
\pgfpathlineto{\pgfqpoint{3.626254in}{3.308008in}}%
\pgfpathlineto{\pgfqpoint{3.628338in}{3.307450in}}%
\pgfpathlineto{\pgfqpoint{3.632813in}{3.303225in}}%
\pgfpathlineto{\pgfqpoint{3.645161in}{3.287713in}}%
\pgfpathlineto{\pgfqpoint{3.650665in}{3.285852in}}%
\pgfpathlineto{\pgfqpoint{3.655370in}{3.280295in}}%
\pgfpathlineto{\pgfqpoint{3.665131in}{3.266520in}}%
\pgfpathlineto{\pgfqpoint{3.669484in}{3.264710in}}%
\pgfpathlineto{\pgfqpoint{3.672701in}{3.263332in}}%
\pgfpathlineto{\pgfqpoint{3.677661in}{3.257419in}}%
\pgfpathlineto{\pgfqpoint{3.687943in}{3.242245in}}%
\pgfpathlineto{\pgfqpoint{3.692486in}{3.240068in}}%
\pgfpathlineto{\pgfqpoint{3.694186in}{3.239726in}}%
\pgfpathlineto{\pgfqpoint{3.697705in}{3.237088in}}%
\pgfpathlineto{\pgfqpoint{3.702956in}{3.229520in}}%
\pgfpathlineto{\pgfqpoint{3.712078in}{3.215592in}}%
\pgfpathlineto{\pgfqpoint{3.716740in}{3.212996in}}%
\pgfpathlineto{\pgfqpoint{3.719096in}{3.212527in}}%
\pgfpathlineto{\pgfqpoint{3.722825in}{3.209483in}}%
\pgfpathlineto{\pgfqpoint{3.728394in}{3.201016in}}%
\pgfpathlineto{\pgfqpoint{3.738083in}{3.185756in}}%
\pgfpathlineto{\pgfqpoint{3.742465in}{3.183123in}}%
\pgfpathlineto{\pgfqpoint{3.744908in}{3.182746in}}%
\pgfpathlineto{\pgfqpoint{3.748903in}{3.179676in}}%
\pgfpathlineto{\pgfqpoint{3.754829in}{3.170559in}}%
\pgfpathlineto{\pgfqpoint{3.765162in}{3.153275in}}%
\pgfpathlineto{\pgfqpoint{3.769838in}{3.149968in}}%
\pgfpathlineto{\pgfqpoint{3.774372in}{3.148649in}}%
\pgfpathlineto{\pgfqpoint{3.778599in}{3.144111in}}%
\pgfpathlineto{\pgfqpoint{3.784939in}{3.132993in}}%
\pgfpathlineto{\pgfqpoint{3.793930in}{3.117236in}}%
\pgfpathlineto{\pgfqpoint{3.798292in}{3.113269in}}%
\pgfpathlineto{\pgfqpoint{3.805270in}{3.110606in}}%
\pgfpathlineto{\pgfqpoint{3.809800in}{3.104855in}}%
\pgfpathlineto{\pgfqpoint{3.818941in}{3.087130in}}%
\pgfpathlineto{\pgfqpoint{3.826203in}{3.074896in}}%
\pgfpathlineto{\pgfqpoint{3.831580in}{3.070908in}}%
\pgfpathlineto{\pgfqpoint{3.835178in}{3.070179in}}%
\pgfpathlineto{\pgfqpoint{3.840116in}{3.065544in}}%
\pgfpathlineto{\pgfqpoint{3.847433in}{3.052345in}}%
\pgfpathlineto{\pgfqpoint{3.857694in}{3.031578in}}%
\pgfpathlineto{\pgfqpoint{3.862778in}{3.025162in}}%
\pgfpathlineto{\pgfqpoint{3.867690in}{3.023256in}}%
\pgfpathlineto{\pgfqpoint{3.869956in}{3.022544in}}%
\pgfpathlineto{\pgfqpoint{3.875302in}{3.017162in}}%
\pgfpathlineto{\pgfqpoint{3.880577in}{3.007728in}}%
\pgfpathlineto{\pgfqpoint{3.897107in}{2.974096in}}%
\pgfpathlineto{\pgfqpoint{3.902584in}{2.969362in}}%
\pgfpathlineto{\pgfqpoint{3.909404in}{2.966799in}}%
\pgfpathlineto{\pgfqpoint{3.915163in}{2.958975in}}%
\pgfpathlineto{\pgfqpoint{3.923776in}{2.940247in}}%
\pgfpathlineto{\pgfqpoint{3.933073in}{2.919147in}}%
\pgfpathlineto{\pgfqpoint{3.939094in}{2.909886in}}%
\pgfpathlineto{\pgfqpoint{3.942073in}{2.907410in}}%
\pgfpathlineto{\pgfqpoint{3.948279in}{2.905647in}}%
\pgfpathlineto{\pgfqpoint{3.951502in}{2.902927in}}%
\pgfpathlineto{\pgfqpoint{3.957775in}{2.892975in}}%
\pgfpathlineto{\pgfqpoint{3.967178in}{2.870688in}}%
\pgfpathlineto{\pgfqpoint{3.977258in}{2.846948in}}%
\pgfpathlineto{\pgfqpoint{3.983821in}{2.837090in}}%
\pgfpathlineto{\pgfqpoint{3.989651in}{2.834193in}}%
\pgfpathlineto{\pgfqpoint{3.993208in}{2.833001in}}%
\pgfpathlineto{\pgfqpoint{3.996722in}{2.829700in}}%
\pgfpathlineto{\pgfqpoint{4.003584in}{2.817914in}}%
\pgfpathlineto{\pgfqpoint{4.013869in}{2.791892in}}%
\pgfpathlineto{\pgfqpoint{4.024899in}{2.764456in}}%
\pgfpathlineto{\pgfqpoint{4.032078in}{2.753245in}}%
\pgfpathlineto{\pgfqpoint{4.038227in}{2.750144in}}%
\pgfpathlineto{\pgfqpoint{4.038858in}{2.750091in}}%
\pgfpathlineto{\pgfqpoint{4.042101in}{2.748754in}}%
\pgfpathlineto{\pgfqpoint{4.045946in}{2.744922in}}%
\pgfpathlineto{\pgfqpoint{4.053454in}{2.731251in}}%
\pgfpathlineto{\pgfqpoint{4.064708in}{2.701043in}}%
\pgfpathlineto{\pgfqpoint{4.076763in}{2.669190in}}%
\pgfpathlineto{\pgfqpoint{4.084609in}{2.656168in}}%
\pgfpathlineto{\pgfqpoint{4.090813in}{2.652585in}}%
\pgfpathlineto{\pgfqpoint{4.092124in}{2.652500in}}%
\pgfpathlineto{\pgfqpoint{4.096076in}{2.650565in}}%
\pgfpathlineto{\pgfqpoint{4.100235in}{2.645854in}}%
\pgfpathlineto{\pgfqpoint{4.108409in}{2.629627in}}%
\pgfpathlineto{\pgfqpoint{4.120686in}{2.594485in}}%
\pgfpathlineto{\pgfqpoint{4.133788in}{2.558229in}}%
\pgfpathlineto{\pgfqpoint{4.138058in}{2.549764in}}%
\pgfpathlineto{\pgfqpoint{4.142299in}{2.543850in}}%
\pgfpathlineto{\pgfqpoint{4.147778in}{2.540393in}}%
\pgfpathlineto{\pgfqpoint{4.152904in}{2.538989in}}%
\pgfpathlineto{\pgfqpoint{4.157436in}{2.534478in}}%
\pgfpathlineto{\pgfqpoint{4.161855in}{2.527210in}}%
\pgfpathlineto{\pgfqpoint{4.170656in}{2.505802in}}%
\pgfpathlineto{\pgfqpoint{4.198122in}{2.426596in}}%
\pgfpathlineto{\pgfqpoint{4.202671in}{2.419125in}}%
\pgfpathlineto{\pgfqpoint{4.208553in}{2.414091in}}%
\pgfpathlineto{\pgfqpoint{4.212662in}{2.413415in}}%
\pgfpathlineto{\pgfqpoint{4.217671in}{2.410088in}}%
\pgfpathlineto{\pgfqpoint{4.222346in}{2.403643in}}%
\pgfpathlineto{\pgfqpoint{4.231659in}{2.382555in}}%
\pgfpathlineto{\pgfqpoint{4.245761in}{2.338148in}}%
\pgfpathlineto{\pgfqpoint{4.255690in}{2.307187in}}%
\pgfpathlineto{\pgfqpoint{4.265276in}{2.285143in}}%
\pgfpathlineto{\pgfqpoint{4.270031in}{2.278690in}}%
\pgfpathlineto{\pgfqpoint{4.276930in}{2.275580in}}%
\pgfpathlineto{\pgfqpoint{4.278162in}{2.275322in}}%
\pgfpathlineto{\pgfqpoint{4.283111in}{2.271982in}}%
\pgfpathlineto{\pgfqpoint{4.287887in}{2.265395in}}%
\pgfpathlineto{\pgfqpoint{4.292672in}{2.255787in}}%
\pgfpathlineto{\pgfqpoint{4.302284in}{2.229298in}}%
\pgfpathlineto{\pgfqpoint{4.326960in}{2.152367in}}%
\pgfpathlineto{\pgfqpoint{4.331806in}{2.142297in}}%
\pgfpathlineto{\pgfqpoint{4.336640in}{2.135500in}}%
\pgfpathlineto{\pgfqpoint{4.343758in}{2.132138in}}%
\pgfpathlineto{\pgfqpoint{4.344453in}{2.132024in}}%
\pgfpathlineto{\pgfqpoint{4.348621in}{2.129819in}}%
\pgfpathlineto{\pgfqpoint{4.353321in}{2.124294in}}%
\pgfpathlineto{\pgfqpoint{4.358068in}{2.115684in}}%
\pgfpathlineto{\pgfqpoint{4.367642in}{2.090752in}}%
\pgfpathlineto{\pgfqpoint{4.392067in}{2.014232in}}%
\pgfpathlineto{\pgfqpoint{4.396828in}{2.003559in}}%
\pgfpathlineto{\pgfqpoint{4.401582in}{1.995929in}}%
\pgfpathlineto{\pgfqpoint{4.407767in}{1.991235in}}%
\pgfpathlineto{\pgfqpoint{4.411368in}{1.990755in}}%
\pgfpathlineto{\pgfqpoint{4.415769in}{1.987879in}}%
\pgfpathlineto{\pgfqpoint{4.420284in}{1.982017in}}%
\pgfpathlineto{\pgfqpoint{4.424861in}{1.973321in}}%
\pgfpathlineto{\pgfqpoint{4.434125in}{1.948937in}}%
\pgfpathlineto{\pgfqpoint{4.457584in}{1.878108in}}%
\pgfpathlineto{\pgfqpoint{4.462118in}{1.868871in}}%
\pgfpathlineto{\pgfqpoint{4.466653in}{1.862560in}}%
\pgfpathlineto{\pgfqpoint{4.473466in}{1.859330in}}%
\pgfpathlineto{\pgfqpoint{4.474194in}{1.859215in}}%
\pgfpathlineto{\pgfqpoint{4.477940in}{1.857341in}}%
\pgfpathlineto{\pgfqpoint{4.482148in}{1.852743in}}%
\pgfpathlineto{\pgfqpoint{4.486432in}{1.845546in}}%
\pgfpathlineto{\pgfqpoint{4.495134in}{1.824521in}}%
\pgfpathlineto{\pgfqpoint{4.517144in}{1.760326in}}%
\pgfpathlineto{\pgfqpoint{4.525601in}{1.745181in}}%
\pgfpathlineto{\pgfqpoint{4.531116in}{1.741288in}}%
\pgfpathlineto{\pgfqpoint{4.534025in}{1.740976in}}%
\pgfpathlineto{\pgfqpoint{4.537895in}{1.738813in}}%
\pgfpathlineto{\pgfqpoint{4.541800in}{1.734336in}}%
\pgfpathlineto{\pgfqpoint{4.549782in}{1.718893in}}%
\pgfpathlineto{\pgfqpoint{4.562115in}{1.684678in}}%
\pgfpathlineto{\pgfqpoint{4.574136in}{1.652826in}}%
\pgfpathlineto{\pgfqpoint{4.581921in}{1.640516in}}%
\pgfpathlineto{\pgfqpoint{4.588158in}{1.637797in}}%
\pgfpathlineto{\pgfqpoint{4.589274in}{1.637558in}}%
\pgfpathlineto{\pgfqpoint{4.592813in}{1.635476in}}%
\pgfpathlineto{\pgfqpoint{4.596395in}{1.631395in}}%
\pgfpathlineto{\pgfqpoint{4.603720in}{1.617677in}}%
\pgfpathlineto{\pgfqpoint{4.615052in}{1.587819in}}%
\pgfpathlineto{\pgfqpoint{4.625973in}{1.560848in}}%
\pgfpathlineto{\pgfqpoint{4.633096in}{1.550622in}}%
\pgfpathlineto{\pgfqpoint{4.645674in}{1.543471in}}%
\pgfpathlineto{\pgfqpoint{4.652360in}{1.531929in}}%
\pgfpathlineto{\pgfqpoint{4.662699in}{1.506419in}}%
\pgfpathlineto{\pgfqpoint{4.672733in}{1.482883in}}%
\pgfpathlineto{\pgfqpoint{4.679241in}{1.473818in}}%
\pgfpathlineto{\pgfqpoint{4.684616in}{1.471812in}}%
\pgfpathlineto{\pgfqpoint{4.686721in}{1.471126in}}%
\pgfpathlineto{\pgfqpoint{4.689681in}{1.468894in}}%
\pgfpathlineto{\pgfqpoint{4.695759in}{1.460106in}}%
\pgfpathlineto{\pgfqpoint{4.705146in}{1.439016in}}%
\pgfpathlineto{\pgfqpoint{4.714392in}{1.417737in}}%
\pgfpathlineto{\pgfqpoint{4.720346in}{1.408661in}}%
\pgfpathlineto{\pgfqpoint{4.725126in}{1.405749in}}%
\pgfpathlineto{\pgfqpoint{4.729588in}{1.404487in}}%
\pgfpathlineto{\pgfqpoint{4.735112in}{1.398374in}}%
\pgfpathlineto{\pgfqpoint{4.740774in}{1.387837in}}%
\pgfpathlineto{\pgfqpoint{4.755038in}{1.356885in}}%
\pgfpathlineto{\pgfqpoint{4.760509in}{1.349966in}}%
\pgfpathlineto{\pgfqpoint{4.770968in}{1.344612in}}%
\pgfpathlineto{\pgfqpoint{4.776138in}{1.336802in}}%
\pgfpathlineto{\pgfqpoint{4.794350in}{1.301676in}}%
\pgfpathlineto{\pgfqpoint{4.799406in}{1.298395in}}%
\pgfpathlineto{\pgfqpoint{4.803923in}{1.296979in}}%
\pgfpathlineto{\pgfqpoint{4.808648in}{1.291735in}}%
\pgfpathlineto{\pgfqpoint{4.815939in}{1.278251in}}%
\pgfpathlineto{\pgfqpoint{4.825568in}{1.259987in}}%
\pgfpathlineto{\pgfqpoint{4.830245in}{1.255203in}}%
\pgfpathlineto{\pgfqpoint{4.837127in}{1.252776in}}%
\pgfpathlineto{\pgfqpoint{4.841533in}{1.247718in}}%
\pgfpathlineto{\pgfqpoint{4.848324in}{1.235439in}}%
\pgfpathlineto{\pgfqpoint{4.857238in}{1.219712in}}%
\pgfpathlineto{\pgfqpoint{4.861591in}{1.215916in}}%
\pgfpathlineto{\pgfqpoint{4.867146in}{1.214253in}}%
\pgfpathlineto{\pgfqpoint{4.871248in}{1.209969in}}%
\pgfpathlineto{\pgfqpoint{4.877569in}{1.199217in}}%
\pgfpathlineto{\pgfqpoint{4.885910in}{1.184945in}}%
\pgfpathlineto{\pgfqpoint{4.889977in}{1.181306in}}%
\pgfpathlineto{\pgfqpoint{4.896183in}{1.179174in}}%
\pgfpathlineto{\pgfqpoint{4.900044in}{1.174864in}}%
\pgfpathlineto{\pgfqpoint{4.917610in}{1.149815in}}%
\pgfpathlineto{\pgfqpoint{4.923348in}{1.147666in}}%
\pgfpathlineto{\pgfqpoint{4.928839in}{1.140550in}}%
\pgfpathlineto{\pgfqpoint{4.939921in}{1.123497in}}%
\pgfpathlineto{\pgfqpoint{4.944428in}{1.121453in}}%
\pgfpathlineto{\pgfqpoint{4.946414in}{1.120938in}}%
\pgfpathlineto{\pgfqpoint{4.949804in}{1.118232in}}%
\pgfpathlineto{\pgfqpoint{4.955026in}{1.110729in}}%
\pgfpathlineto{\pgfqpoint{4.963746in}{1.097924in}}%
\pgfpathlineto{\pgfqpoint{4.968179in}{1.095892in}}%
\pgfpathlineto{\pgfqpoint{4.971511in}{1.094561in}}%
\pgfpathlineto{\pgfqpoint{4.976403in}{1.088924in}}%
\pgfpathlineto{\pgfqpoint{4.986324in}{1.074554in}}%
\pgfpathlineto{\pgfqpoint{4.990702in}{1.072550in}}%
\pgfpathlineto{\pgfqpoint{4.994018in}{1.071207in}}%
\pgfpathlineto{\pgfqpoint{4.998666in}{1.065911in}}%
\pgfpathlineto{\pgfqpoint{5.008087in}{1.052824in}}%
\pgfpathlineto{\pgfqpoint{5.012324in}{1.051116in}}%
\pgfpathlineto{\pgfqpoint{5.015167in}{1.049943in}}%
\pgfpathlineto{\pgfqpoint{5.019589in}{1.045110in}}%
\pgfpathlineto{\pgfqpoint{5.028561in}{1.033012in}}%
\pgfpathlineto{\pgfqpoint{5.032516in}{1.031394in}}%
\pgfpathlineto{\pgfqpoint{5.036011in}{1.029817in}}%
\pgfpathlineto{\pgfqpoint{5.040246in}{1.024929in}}%
\pgfpathlineto{\pgfqpoint{5.048762in}{1.014175in}}%
\pgfpathlineto{\pgfqpoint{5.058782in}{1.007865in}}%
\pgfpathlineto{\pgfqpoint{5.068302in}{0.996663in}}%
\pgfpathlineto{\pgfqpoint{5.074730in}{0.993931in}}%
\pgfpathlineto{\pgfqpoint{5.081304in}{0.985511in}}%
\pgfpathlineto{\pgfqpoint{5.086418in}{0.980731in}}%
\pgfpathlineto{\pgfqpoint{5.091714in}{0.978850in}}%
\pgfpathlineto{\pgfqpoint{5.096716in}{0.973022in}}%
\pgfpathlineto{\pgfqpoint{5.102948in}{0.966250in}}%
\pgfpathlineto{\pgfqpoint{5.111078in}{0.961782in}}%
\pgfpathlineto{\pgfqpoint{5.119502in}{0.952356in}}%
\pgfpathlineto{\pgfqpoint{5.126527in}{0.949033in}}%
\pgfpathlineto{\pgfqpoint{5.136943in}{0.938782in}}%
\pgfpathlineto{\pgfqpoint{5.140986in}{0.937295in}}%
\pgfpathlineto{\pgfqpoint{5.146635in}{0.930886in}}%
\pgfpathlineto{\pgfqpoint{5.151044in}{0.926891in}}%
\pgfpathlineto{\pgfqpoint{5.157358in}{0.924051in}}%
\pgfpathlineto{\pgfqpoint{5.167485in}{0.914885in}}%
\pgfpathlineto{\pgfqpoint{5.169975in}{0.914187in}}%
\pgfpathlineto{\pgfqpoint{5.174125in}{0.910264in}}%
\pgfpathlineto{\pgfqpoint{5.180406in}{0.904229in}}%
\pgfpathlineto{\pgfqpoint{5.186144in}{0.901772in}}%
\pgfpathlineto{\pgfqpoint{5.195547in}{0.893521in}}%
\pgfpathlineto{\pgfqpoint{5.197929in}{0.892928in}}%
\pgfpathlineto{\pgfqpoint{5.201823in}{0.889441in}}%
\pgfpathlineto{\pgfqpoint{5.207727in}{0.883943in}}%
\pgfpathlineto{\pgfqpoint{5.213613in}{0.881307in}}%
\pgfpathlineto{\pgfqpoint{5.221227in}{0.874378in}}%
\pgfpathlineto{\pgfqpoint{5.225280in}{0.873076in}}%
\pgfpathlineto{\pgfqpoint{5.246177in}{0.856866in}}%
\pgfpathlineto{\pgfqpoint{5.251021in}{0.854945in}}%
\pgfpathlineto{\pgfqpoint{5.259453in}{0.848490in}}%
\pgfpathlineto{\pgfqpoint{5.262273in}{0.847354in}}%
\pgfpathlineto{\pgfqpoint{5.272090in}{0.840491in}}%
\pgfpathlineto{\pgfqpoint{5.275366in}{0.838317in}}%
\pgfpathlineto{\pgfqpoint{5.282032in}{0.833149in}}%
\pgfpathlineto{\pgfqpoint{5.285093in}{0.832183in}}%
\pgfpathlineto{\pgfqpoint{5.303645in}{0.819059in}}%
\pgfpathlineto{\pgfqpoint{5.307816in}{0.817475in}}%
\pgfpathlineto{\pgfqpoint{5.315268in}{0.812280in}}%
\pgfpathlineto{\pgfqpoint{5.318178in}{0.811000in}}%
\pgfpathlineto{\pgfqpoint{5.325674in}{0.805847in}}%
\pgfpathlineto{\pgfqpoint{5.328551in}{0.804575in}}%
\pgfpathlineto{\pgfqpoint{5.335882in}{0.799643in}}%
\pgfpathlineto{\pgfqpoint{5.338703in}{0.798378in}}%
\pgfpathlineto{\pgfqpoint{5.345871in}{0.793655in}}%
\pgfpathlineto{\pgfqpoint{5.348637in}{0.792399in}}%
\pgfpathlineto{\pgfqpoint{5.355571in}{0.787879in}}%
\pgfpathlineto{\pgfqpoint{5.358279in}{0.786692in}}%
\pgfpathlineto{\pgfqpoint{5.365167in}{0.782287in}}%
\pgfpathlineto{\pgfqpoint{5.368504in}{0.780559in}}%
\pgfpathlineto{\pgfqpoint{5.374321in}{0.776893in}}%
\pgfpathlineto{\pgfqpoint{5.377137in}{0.775727in}}%
\pgfpathlineto{\pgfqpoint{5.383622in}{0.771653in}}%
\pgfpathlineto{\pgfqpoint{5.386826in}{0.770093in}}%
\pgfpathlineto{\pgfqpoint{5.392702in}{0.766575in}}%
\pgfpathlineto{\pgfqpoint{5.395854in}{0.765003in}}%
\pgfpathlineto{\pgfqpoint{5.401421in}{0.761666in}}%
\pgfpathlineto{\pgfqpoint{5.404453in}{0.760276in}}%
\pgfpathlineto{\pgfqpoint{5.410160in}{0.756891in}}%
\pgfpathlineto{\pgfqpoint{5.413196in}{0.755457in}}%
\pgfpathlineto{\pgfqpoint{5.418823in}{0.752246in}}%
\pgfpathlineto{\pgfqpoint{5.421816in}{0.750752in}}%
\pgfpathlineto{\pgfqpoint{5.427259in}{0.747739in}}%
\pgfpathlineto{\pgfqpoint{5.430204in}{0.746247in}}%
\pgfpathlineto{\pgfqpoint{5.435405in}{0.743374in}}%
\pgfpathlineto{\pgfqpoint{5.438296in}{0.741989in}}%
\pgfpathlineto{\pgfqpoint{5.443403in}{0.739127in}}%
\pgfpathlineto{\pgfqpoint{5.446238in}{0.737857in}}%
\pgfpathlineto{\pgfqpoint{5.451430in}{0.734982in}}%
\pgfpathlineto{\pgfqpoint{5.454224in}{0.733707in}}%
\pgfpathlineto{\pgfqpoint{5.459635in}{0.730900in}}%
\pgfpathlineto{\pgfqpoint{5.462968in}{0.728978in}}%
\pgfpathlineto{\pgfqpoint{5.467078in}{0.727014in}}%
\pgfpathlineto{\pgfqpoint{5.470343in}{0.725331in}}%
\pgfpathlineto{\pgfqpoint{5.474641in}{0.723190in}}%
\pgfpathlineto{\pgfqpoint{5.477852in}{0.721582in}}%
\pgfpathlineto{\pgfqpoint{5.482084in}{0.719462in}}%
\pgfpathlineto{\pgfqpoint{5.485241in}{0.717931in}}%
\pgfpathlineto{\pgfqpoint{5.489424in}{0.715824in}}%
\pgfpathlineto{\pgfqpoint{5.492524in}{0.714367in}}%
\pgfpathlineto{\pgfqpoint{5.496873in}{0.712254in}}%
\pgfpathlineto{\pgfqpoint{5.499943in}{0.710709in}}%
\pgfpathlineto{\pgfqpoint{5.504189in}{0.708763in}}%
\pgfpathlineto{\pgfqpoint{5.507735in}{0.706778in}}%
\pgfpathlineto{\pgfqpoint{5.511653in}{0.705282in}}%
\pgfpathlineto{\pgfqpoint{5.516187in}{0.702497in}}%
\pgfpathlineto{\pgfqpoint{5.532296in}{0.696596in}}%
\pgfpathlineto{\pgfqpoint{5.534545in}{0.697696in}}%
\pgfpathlineto{\pgfqpoint{5.534545in}{0.697696in}}%
\pgfusepath{stroke}%
\end{pgfscope}%
\begin{pgfscope}%
\pgfsetrectcap%
\pgfsetmiterjoin%
\pgfsetlinewidth{0.803000pt}%
\definecolor{currentstroke}{rgb}{0.000000,0.000000,0.000000}%
\pgfsetstrokecolor{currentstroke}%
\pgfsetdash{}{0pt}%
\pgfpathmoveto{\pgfqpoint{0.800000in}{0.528000in}}%
\pgfpathlineto{\pgfqpoint{0.800000in}{4.224000in}}%
\pgfusepath{stroke}%
\end{pgfscope}%
\begin{pgfscope}%
\pgfsetrectcap%
\pgfsetmiterjoin%
\pgfsetlinewidth{0.803000pt}%
\definecolor{currentstroke}{rgb}{0.000000,0.000000,0.000000}%
\pgfsetstrokecolor{currentstroke}%
\pgfsetdash{}{0pt}%
\pgfpathmoveto{\pgfqpoint{5.760000in}{0.528000in}}%
\pgfpathlineto{\pgfqpoint{5.760000in}{4.224000in}}%
\pgfusepath{stroke}%
\end{pgfscope}%
\begin{pgfscope}%
\pgfsetrectcap%
\pgfsetmiterjoin%
\pgfsetlinewidth{0.803000pt}%
\definecolor{currentstroke}{rgb}{0.000000,0.000000,0.000000}%
\pgfsetstrokecolor{currentstroke}%
\pgfsetdash{}{0pt}%
\pgfpathmoveto{\pgfqpoint{0.800000in}{0.528000in}}%
\pgfpathlineto{\pgfqpoint{5.760000in}{0.528000in}}%
\pgfusepath{stroke}%
\end{pgfscope}%
\begin{pgfscope}%
\pgfsetrectcap%
\pgfsetmiterjoin%
\pgfsetlinewidth{0.803000pt}%
\definecolor{currentstroke}{rgb}{0.000000,0.000000,0.000000}%
\pgfsetstrokecolor{currentstroke}%
\pgfsetdash{}{0pt}%
\pgfpathmoveto{\pgfqpoint{0.800000in}{4.224000in}}%
\pgfpathlineto{\pgfqpoint{5.760000in}{4.224000in}}%
\pgfusepath{stroke}%
\end{pgfscope}%
\end{pgfpicture}%
\makeatother%
\endgroup%

}
\end{subfigure}
\caption{Evolution of the combined $a$ and $i$ transfer orbit for the case $i_0 = 90~\text{deg}$}
\label{fig:aincnumres90}
\end{figure}

To observe the differences between the numerical and the analytical solution, in figure~\ref{fig:aincdiffinc90} we have plotted the difference between the analytical evolution of the inclination and the numerical solution. While this difference is small compared to the absolute value of the inclination, we notice and oscillatory behavior that is due to the differences within each revolution. In fact, we notice that the frequency of the oscillations decreases as the orbit gets larger, and then rises again.

\begin{figure}%[h]
\centering
\resizebox{1.0\textwidth}{!}{
%% Creator: Matplotlib, PGF backend
%%
%% To include the figure in your LaTeX document, write
%%   \input{<filename>.pgf}
%%
%% Make sure the required packages are loaded in your preamble
%%   \usepackage{pgf}
%%
%% Figures using additional raster images can only be included by \input if
%% they are in the same directory as the main LaTeX file. For loading figures
%% from other directories you can use the `import` package
%%   \usepackage{import}
%% and then include the figures with
%%   \import{<path to file>}{<filename>.pgf}
%%
%% Matplotlib used the following preamble
%%   \usepackage{fontspec}
%%
\begingroup%
\makeatletter%
\begin{pgfpicture}%
\pgfpathrectangle{\pgfpointorigin}{\pgfqpoint{6.400000in}{4.800000in}}%
\pgfusepath{use as bounding box, clip}%
\begin{pgfscope}%
\pgfsetbuttcap%
\pgfsetmiterjoin%
\definecolor{currentfill}{rgb}{1.000000,1.000000,1.000000}%
\pgfsetfillcolor{currentfill}%
\pgfsetlinewidth{0.000000pt}%
\definecolor{currentstroke}{rgb}{1.000000,1.000000,1.000000}%
\pgfsetstrokecolor{currentstroke}%
\pgfsetdash{}{0pt}%
\pgfpathmoveto{\pgfqpoint{0.000000in}{0.000000in}}%
\pgfpathlineto{\pgfqpoint{6.400000in}{0.000000in}}%
\pgfpathlineto{\pgfqpoint{6.400000in}{4.800000in}}%
\pgfpathlineto{\pgfqpoint{0.000000in}{4.800000in}}%
\pgfpathclose%
\pgfusepath{fill}%
\end{pgfscope}%
\begin{pgfscope}%
\pgfsetbuttcap%
\pgfsetmiterjoin%
\definecolor{currentfill}{rgb}{1.000000,1.000000,1.000000}%
\pgfsetfillcolor{currentfill}%
\pgfsetlinewidth{0.000000pt}%
\definecolor{currentstroke}{rgb}{0.000000,0.000000,0.000000}%
\pgfsetstrokecolor{currentstroke}%
\pgfsetstrokeopacity{0.000000}%
\pgfsetdash{}{0pt}%
\pgfpathmoveto{\pgfqpoint{0.800000in}{2.544000in}}%
\pgfpathlineto{\pgfqpoint{5.760000in}{2.544000in}}%
\pgfpathlineto{\pgfqpoint{5.760000in}{4.224000in}}%
\pgfpathlineto{\pgfqpoint{0.800000in}{4.224000in}}%
\pgfpathclose%
\pgfusepath{fill}%
\end{pgfscope}%
\begin{pgfscope}%
\pgfsetbuttcap%
\pgfsetroundjoin%
\definecolor{currentfill}{rgb}{0.000000,0.000000,0.000000}%
\pgfsetfillcolor{currentfill}%
\pgfsetlinewidth{0.803000pt}%
\definecolor{currentstroke}{rgb}{0.000000,0.000000,0.000000}%
\pgfsetstrokecolor{currentstroke}%
\pgfsetdash{}{0pt}%
\pgfsys@defobject{currentmarker}{\pgfqpoint{0.000000in}{-0.048611in}}{\pgfqpoint{0.000000in}{0.000000in}}{%
\pgfpathmoveto{\pgfqpoint{0.000000in}{0.000000in}}%
\pgfpathlineto{\pgfqpoint{0.000000in}{-0.048611in}}%
\pgfusepath{stroke,fill}%
}%
\begin{pgfscope}%
\pgfsys@transformshift{1.025455in}{2.544000in}%
\pgfsys@useobject{currentmarker}{}%
\end{pgfscope}%
\end{pgfscope}%
\begin{pgfscope}%
\pgfsetbuttcap%
\pgfsetroundjoin%
\definecolor{currentfill}{rgb}{0.000000,0.000000,0.000000}%
\pgfsetfillcolor{currentfill}%
\pgfsetlinewidth{0.803000pt}%
\definecolor{currentstroke}{rgb}{0.000000,0.000000,0.000000}%
\pgfsetstrokecolor{currentstroke}%
\pgfsetdash{}{0pt}%
\pgfsys@defobject{currentmarker}{\pgfqpoint{0.000000in}{-0.048611in}}{\pgfqpoint{0.000000in}{0.000000in}}{%
\pgfpathmoveto{\pgfqpoint{0.000000in}{0.000000in}}%
\pgfpathlineto{\pgfqpoint{0.000000in}{-0.048611in}}%
\pgfusepath{stroke,fill}%
}%
\begin{pgfscope}%
\pgfsys@transformshift{1.804309in}{2.544000in}%
\pgfsys@useobject{currentmarker}{}%
\end{pgfscope}%
\end{pgfscope}%
\begin{pgfscope}%
\pgfsetbuttcap%
\pgfsetroundjoin%
\definecolor{currentfill}{rgb}{0.000000,0.000000,0.000000}%
\pgfsetfillcolor{currentfill}%
\pgfsetlinewidth{0.803000pt}%
\definecolor{currentstroke}{rgb}{0.000000,0.000000,0.000000}%
\pgfsetstrokecolor{currentstroke}%
\pgfsetdash{}{0pt}%
\pgfsys@defobject{currentmarker}{\pgfqpoint{0.000000in}{-0.048611in}}{\pgfqpoint{0.000000in}{0.000000in}}{%
\pgfpathmoveto{\pgfqpoint{0.000000in}{0.000000in}}%
\pgfpathlineto{\pgfqpoint{0.000000in}{-0.048611in}}%
\pgfusepath{stroke,fill}%
}%
\begin{pgfscope}%
\pgfsys@transformshift{2.583164in}{2.544000in}%
\pgfsys@useobject{currentmarker}{}%
\end{pgfscope}%
\end{pgfscope}%
\begin{pgfscope}%
\pgfsetbuttcap%
\pgfsetroundjoin%
\definecolor{currentfill}{rgb}{0.000000,0.000000,0.000000}%
\pgfsetfillcolor{currentfill}%
\pgfsetlinewidth{0.803000pt}%
\definecolor{currentstroke}{rgb}{0.000000,0.000000,0.000000}%
\pgfsetstrokecolor{currentstroke}%
\pgfsetdash{}{0pt}%
\pgfsys@defobject{currentmarker}{\pgfqpoint{0.000000in}{-0.048611in}}{\pgfqpoint{0.000000in}{0.000000in}}{%
\pgfpathmoveto{\pgfqpoint{0.000000in}{0.000000in}}%
\pgfpathlineto{\pgfqpoint{0.000000in}{-0.048611in}}%
\pgfusepath{stroke,fill}%
}%
\begin{pgfscope}%
\pgfsys@transformshift{3.362019in}{2.544000in}%
\pgfsys@useobject{currentmarker}{}%
\end{pgfscope}%
\end{pgfscope}%
\begin{pgfscope}%
\pgfsetbuttcap%
\pgfsetroundjoin%
\definecolor{currentfill}{rgb}{0.000000,0.000000,0.000000}%
\pgfsetfillcolor{currentfill}%
\pgfsetlinewidth{0.803000pt}%
\definecolor{currentstroke}{rgb}{0.000000,0.000000,0.000000}%
\pgfsetstrokecolor{currentstroke}%
\pgfsetdash{}{0pt}%
\pgfsys@defobject{currentmarker}{\pgfqpoint{0.000000in}{-0.048611in}}{\pgfqpoint{0.000000in}{0.000000in}}{%
\pgfpathmoveto{\pgfqpoint{0.000000in}{0.000000in}}%
\pgfpathlineto{\pgfqpoint{0.000000in}{-0.048611in}}%
\pgfusepath{stroke,fill}%
}%
\begin{pgfscope}%
\pgfsys@transformshift{4.140873in}{2.544000in}%
\pgfsys@useobject{currentmarker}{}%
\end{pgfscope}%
\end{pgfscope}%
\begin{pgfscope}%
\pgfsetbuttcap%
\pgfsetroundjoin%
\definecolor{currentfill}{rgb}{0.000000,0.000000,0.000000}%
\pgfsetfillcolor{currentfill}%
\pgfsetlinewidth{0.803000pt}%
\definecolor{currentstroke}{rgb}{0.000000,0.000000,0.000000}%
\pgfsetstrokecolor{currentstroke}%
\pgfsetdash{}{0pt}%
\pgfsys@defobject{currentmarker}{\pgfqpoint{0.000000in}{-0.048611in}}{\pgfqpoint{0.000000in}{0.000000in}}{%
\pgfpathmoveto{\pgfqpoint{0.000000in}{0.000000in}}%
\pgfpathlineto{\pgfqpoint{0.000000in}{-0.048611in}}%
\pgfusepath{stroke,fill}%
}%
\begin{pgfscope}%
\pgfsys@transformshift{4.919728in}{2.544000in}%
\pgfsys@useobject{currentmarker}{}%
\end{pgfscope}%
\end{pgfscope}%
\begin{pgfscope}%
\pgfsetbuttcap%
\pgfsetroundjoin%
\definecolor{currentfill}{rgb}{0.000000,0.000000,0.000000}%
\pgfsetfillcolor{currentfill}%
\pgfsetlinewidth{0.803000pt}%
\definecolor{currentstroke}{rgb}{0.000000,0.000000,0.000000}%
\pgfsetstrokecolor{currentstroke}%
\pgfsetdash{}{0pt}%
\pgfsys@defobject{currentmarker}{\pgfqpoint{0.000000in}{-0.048611in}}{\pgfqpoint{0.000000in}{0.000000in}}{%
\pgfpathmoveto{\pgfqpoint{0.000000in}{0.000000in}}%
\pgfpathlineto{\pgfqpoint{0.000000in}{-0.048611in}}%
\pgfusepath{stroke,fill}%
}%
\begin{pgfscope}%
\pgfsys@transformshift{5.698583in}{2.544000in}%
\pgfsys@useobject{currentmarker}{}%
\end{pgfscope}%
\end{pgfscope}%
\begin{pgfscope}%
\pgfsetbuttcap%
\pgfsetroundjoin%
\definecolor{currentfill}{rgb}{0.000000,0.000000,0.000000}%
\pgfsetfillcolor{currentfill}%
\pgfsetlinewidth{0.803000pt}%
\definecolor{currentstroke}{rgb}{0.000000,0.000000,0.000000}%
\pgfsetstrokecolor{currentstroke}%
\pgfsetdash{}{0pt}%
\pgfsys@defobject{currentmarker}{\pgfqpoint{-0.048611in}{0.000000in}}{\pgfqpoint{0.000000in}{0.000000in}}{%
\pgfpathmoveto{\pgfqpoint{0.000000in}{0.000000in}}%
\pgfpathlineto{\pgfqpoint{-0.048611in}{0.000000in}}%
\pgfusepath{stroke,fill}%
}%
\begin{pgfscope}%
\pgfsys@transformshift{0.800000in}{2.620351in}%
\pgfsys@useobject{currentmarker}{}%
\end{pgfscope}%
\end{pgfscope}%
\begin{pgfscope}%
\pgftext[x=0.525308in,y=2.572157in,left,base]{\sffamily\fontsize{10.000000}{12.000000}\selectfont \(\displaystyle 0.0\)}%
\end{pgfscope}%
\begin{pgfscope}%
\pgfsetbuttcap%
\pgfsetroundjoin%
\definecolor{currentfill}{rgb}{0.000000,0.000000,0.000000}%
\pgfsetfillcolor{currentfill}%
\pgfsetlinewidth{0.803000pt}%
\definecolor{currentstroke}{rgb}{0.000000,0.000000,0.000000}%
\pgfsetstrokecolor{currentstroke}%
\pgfsetdash{}{0pt}%
\pgfsys@defobject{currentmarker}{\pgfqpoint{-0.048611in}{0.000000in}}{\pgfqpoint{0.000000in}{0.000000in}}{%
\pgfpathmoveto{\pgfqpoint{0.000000in}{0.000000in}}%
\pgfpathlineto{\pgfqpoint{-0.048611in}{0.000000in}}%
\pgfusepath{stroke,fill}%
}%
\begin{pgfscope}%
\pgfsys@transformshift{0.800000in}{3.106501in}%
\pgfsys@useobject{currentmarker}{}%
\end{pgfscope}%
\end{pgfscope}%
\begin{pgfscope}%
\pgftext[x=0.525308in,y=3.058307in,left,base]{\sffamily\fontsize{10.000000}{12.000000}\selectfont \(\displaystyle 0.5\)}%
\end{pgfscope}%
\begin{pgfscope}%
\pgfsetbuttcap%
\pgfsetroundjoin%
\definecolor{currentfill}{rgb}{0.000000,0.000000,0.000000}%
\pgfsetfillcolor{currentfill}%
\pgfsetlinewidth{0.803000pt}%
\definecolor{currentstroke}{rgb}{0.000000,0.000000,0.000000}%
\pgfsetstrokecolor{currentstroke}%
\pgfsetdash{}{0pt}%
\pgfsys@defobject{currentmarker}{\pgfqpoint{-0.048611in}{0.000000in}}{\pgfqpoint{0.000000in}{0.000000in}}{%
\pgfpathmoveto{\pgfqpoint{0.000000in}{0.000000in}}%
\pgfpathlineto{\pgfqpoint{-0.048611in}{0.000000in}}%
\pgfusepath{stroke,fill}%
}%
\begin{pgfscope}%
\pgfsys@transformshift{0.800000in}{3.592651in}%
\pgfsys@useobject{currentmarker}{}%
\end{pgfscope}%
\end{pgfscope}%
\begin{pgfscope}%
\pgftext[x=0.525308in,y=3.544457in,left,base]{\sffamily\fontsize{10.000000}{12.000000}\selectfont \(\displaystyle 1.0\)}%
\end{pgfscope}%
\begin{pgfscope}%
\pgfsetbuttcap%
\pgfsetroundjoin%
\definecolor{currentfill}{rgb}{0.000000,0.000000,0.000000}%
\pgfsetfillcolor{currentfill}%
\pgfsetlinewidth{0.803000pt}%
\definecolor{currentstroke}{rgb}{0.000000,0.000000,0.000000}%
\pgfsetstrokecolor{currentstroke}%
\pgfsetdash{}{0pt}%
\pgfsys@defobject{currentmarker}{\pgfqpoint{-0.048611in}{0.000000in}}{\pgfqpoint{0.000000in}{0.000000in}}{%
\pgfpathmoveto{\pgfqpoint{0.000000in}{0.000000in}}%
\pgfpathlineto{\pgfqpoint{-0.048611in}{0.000000in}}%
\pgfusepath{stroke,fill}%
}%
\begin{pgfscope}%
\pgfsys@transformshift{0.800000in}{4.078801in}%
\pgfsys@useobject{currentmarker}{}%
\end{pgfscope}%
\end{pgfscope}%
\begin{pgfscope}%
\pgftext[x=0.525308in,y=4.030607in,left,base]{\sffamily\fontsize{10.000000}{12.000000}\selectfont \(\displaystyle 1.5\)}%
\end{pgfscope}%
\begin{pgfscope}%
\pgfpathrectangle{\pgfqpoint{0.800000in}{2.544000in}}{\pgfqpoint{4.960000in}{1.680000in}} %
\pgfusepath{clip}%
\pgfsetrectcap%
\pgfsetroundjoin%
\pgfsetlinewidth{1.505625pt}%
\definecolor{currentstroke}{rgb}{0.121569,0.466667,0.705882}%
\pgfsetstrokecolor{currentstroke}%
\pgfsetdash{}{0pt}%
\pgfpathmoveto{\pgfqpoint{1.025455in}{4.147636in}}%
\pgfpathlineto{\pgfqpoint{1.082974in}{4.145594in}}%
\pgfpathlineto{\pgfqpoint{1.132685in}{4.143769in}}%
\pgfpathlineto{\pgfqpoint{1.202751in}{4.141105in}}%
\pgfpathlineto{\pgfqpoint{1.480703in}{4.129304in}}%
\pgfpathlineto{\pgfqpoint{1.540392in}{4.126475in}}%
\pgfpathlineto{\pgfqpoint{1.595713in}{4.123751in}}%
\pgfpathlineto{\pgfqpoint{1.643491in}{4.121317in}}%
\pgfpathlineto{\pgfqpoint{1.698945in}{4.118376in}}%
\pgfpathlineto{\pgfqpoint{1.737632in}{4.116253in}}%
\pgfpathlineto{\pgfqpoint{1.769739in}{4.114452in}}%
\pgfpathlineto{\pgfqpoint{1.806670in}{4.112318in}}%
\pgfpathlineto{\pgfqpoint{1.846391in}{4.109971in}}%
\pgfpathlineto{\pgfqpoint{1.886201in}{4.107539in}}%
\pgfpathlineto{\pgfqpoint{2.165048in}{4.088141in}}%
\pgfpathlineto{\pgfqpoint{2.202905in}{4.085164in}}%
\pgfpathlineto{\pgfqpoint{2.235752in}{4.082486in}}%
\pgfpathlineto{\pgfqpoint{2.263776in}{4.080129in}}%
\pgfpathlineto{\pgfqpoint{2.276916in}{4.079003in}}%
\pgfpathlineto{\pgfqpoint{2.305872in}{4.076478in}}%
\pgfpathlineto{\pgfqpoint{2.340647in}{4.073382in}}%
\pgfpathlineto{\pgfqpoint{2.364198in}{4.071205in}}%
\pgfpathlineto{\pgfqpoint{2.428735in}{4.065053in}}%
\pgfpathlineto{\pgfqpoint{2.445330in}{4.063379in}}%
\pgfpathlineto{\pgfqpoint{2.482404in}{4.059617in}}%
\pgfpathlineto{\pgfqpoint{2.495032in}{4.058311in}}%
\pgfpathlineto{\pgfqpoint{2.515986in}{4.056088in}}%
\pgfpathlineto{\pgfqpoint{2.539398in}{4.053529in}}%
\pgfpathlineto{\pgfqpoint{2.558379in}{4.051400in}}%
\pgfpathlineto{\pgfqpoint{2.573990in}{4.049632in}}%
\pgfpathlineto{\pgfqpoint{2.608782in}{4.045635in}}%
\pgfpathlineto{\pgfqpoint{2.626911in}{4.043451in}}%
\pgfpathlineto{\pgfqpoint{2.643594in}{4.041449in}}%
\pgfpathlineto{\pgfqpoint{2.662859in}{4.039072in}}%
\pgfpathlineto{\pgfqpoint{2.673433in}{4.037786in}}%
\pgfpathlineto{\pgfqpoint{2.691180in}{4.035509in}}%
\pgfpathlineto{\pgfqpoint{2.701808in}{4.034172in}}%
\pgfpathlineto{\pgfqpoint{2.727114in}{4.030774in}}%
\pgfpathlineto{\pgfqpoint{2.738158in}{4.029365in}}%
\pgfpathlineto{\pgfqpoint{2.786112in}{4.022670in}}%
\pgfpathlineto{\pgfqpoint{2.799834in}{4.020755in}}%
\pgfpathlineto{\pgfqpoint{2.836592in}{4.015293in}}%
\pgfpathlineto{\pgfqpoint{2.850503in}{4.013107in}}%
\pgfpathlineto{\pgfqpoint{2.869250in}{4.010336in}}%
\pgfpathlineto{\pgfqpoint{2.895460in}{4.006183in}}%
\pgfpathlineto{\pgfqpoint{2.956747in}{3.995936in}}%
\pgfpathlineto{\pgfqpoint{2.979023in}{3.991962in}}%
\pgfpathlineto{\pgfqpoint{2.991603in}{3.989596in}}%
\pgfpathlineto{\pgfqpoint{3.005936in}{3.987078in}}%
\pgfpathlineto{\pgfqpoint{3.018205in}{3.984700in}}%
\pgfpathlineto{\pgfqpoint{3.055420in}{3.977767in}}%
\pgfpathlineto{\pgfqpoint{3.069480in}{3.974972in}}%
\pgfpathlineto{\pgfqpoint{3.079788in}{3.972890in}}%
\pgfpathlineto{\pgfqpoint{3.088501in}{3.971127in}}%
\pgfpathlineto{\pgfqpoint{3.099035in}{3.968952in}}%
\pgfpathlineto{\pgfqpoint{3.107414in}{3.967285in}}%
\pgfpathlineto{\pgfqpoint{3.125002in}{3.963332in}}%
\pgfpathlineto{\pgfqpoint{3.133650in}{3.961722in}}%
\pgfpathlineto{\pgfqpoint{3.159535in}{3.955705in}}%
\pgfpathlineto{\pgfqpoint{3.221163in}{3.941364in}}%
\pgfpathlineto{\pgfqpoint{3.232628in}{3.938633in}}%
\pgfpathlineto{\pgfqpoint{3.237060in}{3.937593in}}%
\pgfpathlineto{\pgfqpoint{3.244139in}{3.936017in}}%
\pgfpathlineto{\pgfqpoint{3.258969in}{3.932002in}}%
\pgfpathlineto{\pgfqpoint{3.263367in}{3.930975in}}%
\pgfpathlineto{\pgfqpoint{3.270200in}{3.929406in}}%
\pgfpathlineto{\pgfqpoint{3.400804in}{3.891699in}}%
\pgfpathlineto{\pgfqpoint{3.406406in}{3.890086in}}%
\pgfpathlineto{\pgfqpoint{3.410502in}{3.889075in}}%
\pgfpathlineto{\pgfqpoint{3.419105in}{3.886079in}}%
\pgfpathlineto{\pgfqpoint{3.423755in}{3.884751in}}%
\pgfpathlineto{\pgfqpoint{3.432071in}{3.881879in}}%
\pgfpathlineto{\pgfqpoint{3.436850in}{3.880466in}}%
\pgfpathlineto{\pgfqpoint{3.445078in}{3.877479in}}%
\pgfpathlineto{\pgfqpoint{3.449225in}{3.876488in}}%
\pgfpathlineto{\pgfqpoint{3.459711in}{3.872788in}}%
\pgfpathlineto{\pgfqpoint{3.464761in}{3.870958in}}%
\pgfpathlineto{\pgfqpoint{3.472648in}{3.867966in}}%
\pgfpathlineto{\pgfqpoint{3.476641in}{3.867155in}}%
\pgfpathlineto{\pgfqpoint{3.489121in}{3.862648in}}%
\pgfpathlineto{\pgfqpoint{3.494520in}{3.860355in}}%
\pgfpathlineto{\pgfqpoint{3.501224in}{3.857462in}}%
\pgfpathlineto{\pgfqpoint{3.507605in}{3.855991in}}%
\pgfpathlineto{\pgfqpoint{3.518298in}{3.851577in}}%
\pgfpathlineto{\pgfqpoint{3.522278in}{3.850663in}}%
\pgfpathlineto{\pgfqpoint{3.536396in}{3.845213in}}%
\pgfpathlineto{\pgfqpoint{3.542417in}{3.842411in}}%
\pgfpathlineto{\pgfqpoint{3.549903in}{3.838993in}}%
\pgfpathlineto{\pgfqpoint{3.555697in}{3.837837in}}%
\pgfpathlineto{\pgfqpoint{3.569222in}{3.831885in}}%
\pgfpathlineto{\pgfqpoint{3.574506in}{3.830222in}}%
\pgfpathlineto{\pgfqpoint{3.586570in}{3.824413in}}%
\pgfpathlineto{\pgfqpoint{3.591790in}{3.823238in}}%
\pgfpathlineto{\pgfqpoint{3.607470in}{3.816247in}}%
\pgfpathlineto{\pgfqpoint{3.613173in}{3.813942in}}%
\pgfpathlineto{\pgfqpoint{3.624912in}{3.807721in}}%
\pgfpathlineto{\pgfqpoint{3.631321in}{3.806284in}}%
\pgfpathlineto{\pgfqpoint{3.647447in}{3.798211in}}%
\pgfpathlineto{\pgfqpoint{3.652232in}{3.796904in}}%
\pgfpathlineto{\pgfqpoint{3.671036in}{3.787785in}}%
\pgfpathlineto{\pgfqpoint{3.676006in}{3.785724in}}%
\pgfpathlineto{\pgfqpoint{3.691334in}{3.776819in}}%
\pgfpathlineto{\pgfqpoint{3.697705in}{3.775404in}}%
\pgfpathlineto{\pgfqpoint{3.704722in}{3.770587in}}%
\pgfpathlineto{\pgfqpoint{3.712078in}{3.765633in}}%
\pgfpathlineto{\pgfqpoint{3.717392in}{3.764442in}}%
\pgfpathlineto{\pgfqpoint{3.722825in}{3.762856in}}%
\pgfpathlineto{\pgfqpoint{3.730268in}{3.757484in}}%
\pgfpathlineto{\pgfqpoint{3.738083in}{3.752071in}}%
\pgfpathlineto{\pgfqpoint{3.743468in}{3.750846in}}%
\pgfpathlineto{\pgfqpoint{3.748903in}{3.749307in}}%
\pgfpathlineto{\pgfqpoint{3.756818in}{3.743485in}}%
\pgfpathlineto{\pgfqpoint{3.765162in}{3.737307in}}%
\pgfpathlineto{\pgfqpoint{3.771204in}{3.735738in}}%
\pgfpathlineto{\pgfqpoint{3.776491in}{3.734346in}}%
\pgfpathlineto{\pgfqpoint{3.782819in}{3.729936in}}%
\pgfpathlineto{\pgfqpoint{3.793930in}{3.720926in}}%
\pgfpathlineto{\pgfqpoint{3.800457in}{3.718845in}}%
\pgfpathlineto{\pgfqpoint{3.805270in}{3.717912in}}%
\pgfpathlineto{\pgfqpoint{3.812062in}{3.713513in}}%
\pgfpathlineto{\pgfqpoint{3.826203in}{3.701680in}}%
\pgfpathlineto{\pgfqpoint{3.832821in}{3.699808in}}%
\pgfpathlineto{\pgfqpoint{3.837671in}{3.698719in}}%
\pgfpathlineto{\pgfqpoint{3.844990in}{3.693696in}}%
\pgfpathlineto{\pgfqpoint{3.860245in}{3.680295in}}%
\pgfpathlineto{\pgfqpoint{3.867086in}{3.678219in}}%
\pgfpathlineto{\pgfqpoint{3.869956in}{3.677884in}}%
\pgfpathlineto{\pgfqpoint{3.875302in}{3.675437in}}%
\pgfpathlineto{\pgfqpoint{3.883221in}{3.668556in}}%
\pgfpathlineto{\pgfqpoint{3.894341in}{3.657785in}}%
\pgfpathlineto{\pgfqpoint{3.899853in}{3.654484in}}%
\pgfpathlineto{\pgfqpoint{3.912293in}{3.651051in}}%
\pgfpathlineto{\pgfqpoint{3.920899in}{3.643573in}}%
\pgfpathlineto{\pgfqpoint{3.936098in}{3.628479in}}%
\pgfpathlineto{\pgfqpoint{3.942073in}{3.625550in}}%
\pgfpathlineto{\pgfqpoint{3.951502in}{3.623512in}}%
\pgfpathlineto{\pgfqpoint{3.957775in}{3.618989in}}%
\pgfpathlineto{\pgfqpoint{3.970399in}{3.605139in}}%
\pgfpathlineto{\pgfqpoint{3.980553in}{3.595463in}}%
\pgfpathlineto{\pgfqpoint{3.988039in}{3.592360in}}%
\pgfpathlineto{\pgfqpoint{3.993208in}{3.591728in}}%
\pgfpathlineto{\pgfqpoint{4.000161in}{3.587905in}}%
\pgfpathlineto{\pgfqpoint{4.007005in}{3.581265in}}%
\pgfpathlineto{\pgfqpoint{4.028503in}{3.557596in}}%
\pgfpathlineto{\pgfqpoint{4.035633in}{3.554299in}}%
\pgfpathlineto{\pgfqpoint{4.045946in}{3.551692in}}%
\pgfpathlineto{\pgfqpoint{4.053454in}{3.545478in}}%
\pgfpathlineto{\pgfqpoint{4.064708in}{3.531747in}}%
\pgfpathlineto{\pgfqpoint{4.076763in}{3.517268in}}%
\pgfpathlineto{\pgfqpoint{4.084609in}{3.511349in}}%
\pgfpathlineto{\pgfqpoint{4.092124in}{3.509682in}}%
\pgfpathlineto{\pgfqpoint{4.096076in}{3.508802in}}%
\pgfpathlineto{\pgfqpoint{4.104329in}{3.503439in}}%
\pgfpathlineto{\pgfqpoint{4.112489in}{3.494391in}}%
\pgfpathlineto{\pgfqpoint{4.133788in}{3.466831in}}%
\pgfpathlineto{\pgfqpoint{4.142299in}{3.460296in}}%
\pgfpathlineto{\pgfqpoint{4.149760in}{3.458656in}}%
\pgfpathlineto{\pgfqpoint{4.152904in}{3.458086in}}%
\pgfpathlineto{\pgfqpoint{4.157436in}{3.456036in}}%
\pgfpathlineto{\pgfqpoint{4.166255in}{3.448320in}}%
\pgfpathlineto{\pgfqpoint{4.179501in}{3.430640in}}%
\pgfpathlineto{\pgfqpoint{4.193541in}{3.411650in}}%
\pgfpathlineto{\pgfqpoint{4.202671in}{3.403602in}}%
\pgfpathlineto{\pgfqpoint{4.209906in}{3.401146in}}%
\pgfpathlineto{\pgfqpoint{4.212662in}{3.401007in}}%
\pgfpathlineto{\pgfqpoint{4.217671in}{3.399494in}}%
\pgfpathlineto{\pgfqpoint{4.222346in}{3.396565in}}%
\pgfpathlineto{\pgfqpoint{4.231659in}{3.386979in}}%
\pgfpathlineto{\pgfqpoint{4.245761in}{3.366795in}}%
\pgfpathlineto{\pgfqpoint{4.255690in}{3.352721in}}%
\pgfpathlineto{\pgfqpoint{4.265276in}{3.342701in}}%
\pgfpathlineto{\pgfqpoint{4.270031in}{3.339768in}}%
\pgfpathlineto{\pgfqpoint{4.287887in}{3.333725in}}%
\pgfpathlineto{\pgfqpoint{4.297468in}{3.323808in}}%
\pgfpathlineto{\pgfqpoint{4.312051in}{3.302664in}}%
\pgfpathlineto{\pgfqpoint{4.322086in}{3.288187in}}%
\pgfpathlineto{\pgfqpoint{4.331806in}{3.277771in}}%
\pgfpathlineto{\pgfqpoint{4.336640in}{3.274682in}}%
\pgfpathlineto{\pgfqpoint{4.353321in}{3.269588in}}%
\pgfpathlineto{\pgfqpoint{4.362841in}{3.260519in}}%
\pgfpathlineto{\pgfqpoint{4.377385in}{3.240011in}}%
\pgfpathlineto{\pgfqpoint{4.392067in}{3.219560in}}%
\pgfpathlineto{\pgfqpoint{4.401582in}{3.211241in}}%
\pgfpathlineto{\pgfqpoint{4.410123in}{3.209019in}}%
\pgfpathlineto{\pgfqpoint{4.415769in}{3.207582in}}%
\pgfpathlineto{\pgfqpoint{4.424861in}{3.200964in}}%
\pgfpathlineto{\pgfqpoint{4.434125in}{3.189880in}}%
\pgfpathlineto{\pgfqpoint{4.457584in}{3.157685in}}%
\pgfpathlineto{\pgfqpoint{4.466653in}{3.150618in}}%
\pgfpathlineto{\pgfqpoint{4.482148in}{3.146156in}}%
\pgfpathlineto{\pgfqpoint{4.490764in}{3.138549in}}%
\pgfpathlineto{\pgfqpoint{4.504040in}{3.121135in}}%
\pgfpathlineto{\pgfqpoint{4.517144in}{3.104148in}}%
\pgfpathlineto{\pgfqpoint{4.525601in}{3.097264in}}%
\pgfpathlineto{\pgfqpoint{4.533167in}{3.095427in}}%
\pgfpathlineto{\pgfqpoint{4.537895in}{3.094370in}}%
\pgfpathlineto{\pgfqpoint{4.545768in}{3.089280in}}%
\pgfpathlineto{\pgfqpoint{4.553833in}{3.080605in}}%
\pgfpathlineto{\pgfqpoint{4.574136in}{3.055285in}}%
\pgfpathlineto{\pgfqpoint{4.581921in}{3.049689in}}%
\pgfpathlineto{\pgfqpoint{4.596395in}{3.045543in}}%
\pgfpathlineto{\pgfqpoint{4.603720in}{3.039308in}}%
\pgfpathlineto{\pgfqpoint{4.629532in}{3.010705in}}%
\pgfpathlineto{\pgfqpoint{4.636671in}{3.007958in}}%
\pgfpathlineto{\pgfqpoint{4.642405in}{3.007103in}}%
\pgfpathlineto{\pgfqpoint{4.648997in}{3.003293in}}%
\pgfpathlineto{\pgfqpoint{4.659191in}{2.992898in}}%
\pgfpathlineto{\pgfqpoint{4.672733in}{2.978038in}}%
\pgfpathlineto{\pgfqpoint{4.679241in}{2.973917in}}%
\pgfpathlineto{\pgfqpoint{4.692699in}{2.969992in}}%
\pgfpathlineto{\pgfqpoint{4.701974in}{2.961608in}}%
\pgfpathlineto{\pgfqpoint{4.717367in}{2.946062in}}%
\pgfpathlineto{\pgfqpoint{4.724228in}{2.943075in}}%
\pgfpathlineto{\pgfqpoint{4.732327in}{2.941299in}}%
\pgfpathlineto{\pgfqpoint{4.737929in}{2.937439in}}%
\pgfpathlineto{\pgfqpoint{4.760509in}{2.917621in}}%
\pgfpathlineto{\pgfqpoint{4.773539in}{2.913627in}}%
\pgfpathlineto{\pgfqpoint{4.781419in}{2.906724in}}%
\pgfpathlineto{\pgfqpoint{4.791827in}{2.897203in}}%
\pgfpathlineto{\pgfqpoint{4.796875in}{2.894643in}}%
\pgfpathlineto{\pgfqpoint{4.806267in}{2.892558in}}%
\pgfpathlineto{\pgfqpoint{4.813481in}{2.887301in}}%
\pgfpathlineto{\pgfqpoint{4.827905in}{2.875409in}}%
\pgfpathlineto{\pgfqpoint{4.833560in}{2.874166in}}%
\pgfpathlineto{\pgfqpoint{4.839316in}{2.872478in}}%
\pgfpathlineto{\pgfqpoint{4.846032in}{2.867610in}}%
\pgfpathlineto{\pgfqpoint{4.857238in}{2.858415in}}%
\pgfpathlineto{\pgfqpoint{4.861591in}{2.856689in}}%
\pgfpathlineto{\pgfqpoint{4.869183in}{2.855127in}}%
\pgfpathlineto{\pgfqpoint{4.875438in}{2.850905in}}%
\pgfpathlineto{\pgfqpoint{4.887942in}{2.841605in}}%
\pgfpathlineto{\pgfqpoint{4.902003in}{2.836668in}}%
\pgfpathlineto{\pgfqpoint{4.915701in}{2.827094in}}%
\pgfpathlineto{\pgfqpoint{4.926992in}{2.823723in}}%
\pgfpathlineto{\pgfqpoint{4.941717in}{2.814071in}}%
\pgfpathlineto{\pgfqpoint{4.949804in}{2.812287in}}%
\pgfpathlineto{\pgfqpoint{4.958643in}{2.806132in}}%
\pgfpathlineto{\pgfqpoint{4.965442in}{2.802472in}}%
\pgfpathlineto{\pgfqpoint{4.974759in}{2.800004in}}%
\pgfpathlineto{\pgfqpoint{4.990024in}{2.791536in}}%
\pgfpathlineto{\pgfqpoint{4.995553in}{2.790287in}}%
\pgfpathlineto{\pgfqpoint{5.013728in}{2.781617in}}%
\pgfpathlineto{\pgfqpoint{5.019589in}{2.779050in}}%
\pgfpathlineto{\pgfqpoint{5.030015in}{2.773092in}}%
\pgfpathlineto{\pgfqpoint{5.038823in}{2.770744in}}%
\pgfpathlineto{\pgfqpoint{5.050152in}{2.764649in}}%
\pgfpathlineto{\pgfqpoint{5.057426in}{2.762887in}}%
\pgfpathlineto{\pgfqpoint{5.070390in}{2.756809in}}%
\pgfpathlineto{\pgfqpoint{5.074730in}{2.755787in}}%
\pgfpathlineto{\pgfqpoint{5.090491in}{2.749316in}}%
\pgfpathlineto{\pgfqpoint{5.096716in}{2.746283in}}%
\pgfpathlineto{\pgfqpoint{5.104173in}{2.742973in}}%
\pgfpathlineto{\pgfqpoint{5.109882in}{2.741769in}}%
\pgfpathlineto{\pgfqpoint{5.122032in}{2.736609in}}%
\pgfpathlineto{\pgfqpoint{5.126527in}{2.735379in}}%
\pgfpathlineto{\pgfqpoint{5.137807in}{2.730695in}}%
\pgfpathlineto{\pgfqpoint{5.142091in}{2.729590in}}%
\pgfpathlineto{\pgfqpoint{5.153169in}{2.725116in}}%
\pgfpathlineto{\pgfqpoint{5.158434in}{2.723503in}}%
\pgfpathlineto{\pgfqpoint{5.167581in}{2.719856in}}%
\pgfpathlineto{\pgfqpoint{5.172034in}{2.718801in}}%
\pgfpathlineto{\pgfqpoint{5.182219in}{2.714866in}}%
\pgfpathlineto{\pgfqpoint{5.187151in}{2.713431in}}%
\pgfpathlineto{\pgfqpoint{5.195896in}{2.710142in}}%
\pgfpathlineto{\pgfqpoint{5.199862in}{2.709224in}}%
\pgfpathlineto{\pgfqpoint{5.210801in}{2.705504in}}%
\pgfpathlineto{\pgfqpoint{5.216506in}{2.703163in}}%
\pgfpathlineto{\pgfqpoint{5.222601in}{2.701367in}}%
\pgfpathlineto{\pgfqpoint{5.227116in}{2.700109in}}%
\pgfpathlineto{\pgfqpoint{5.235468in}{2.697281in}}%
\pgfpathlineto{\pgfqpoint{5.239876in}{2.695992in}}%
\pgfpathlineto{\pgfqpoint{5.247616in}{2.693396in}}%
\pgfpathlineto{\pgfqpoint{5.252772in}{2.691832in}}%
\pgfpathlineto{\pgfqpoint{5.259453in}{2.689677in}}%
\pgfpathlineto{\pgfqpoint{5.263969in}{2.688486in}}%
\pgfpathlineto{\pgfqpoint{5.271305in}{2.686108in}}%
\pgfpathlineto{\pgfqpoint{5.276201in}{2.684677in}}%
\pgfpathlineto{\pgfqpoint{5.282738in}{2.682688in}}%
\pgfpathlineto{\pgfqpoint{5.287519in}{2.681301in}}%
\pgfpathlineto{\pgfqpoint{5.294196in}{2.679386in}}%
\pgfpathlineto{\pgfqpoint{5.298891in}{2.677912in}}%
\pgfpathlineto{\pgfqpoint{5.304798in}{2.676248in}}%
\pgfpathlineto{\pgfqpoint{5.309365in}{2.674938in}}%
\pgfpathlineto{\pgfqpoint{5.315953in}{2.673171in}}%
\pgfpathlineto{\pgfqpoint{5.321247in}{2.671353in}}%
\pgfpathlineto{\pgfqpoint{5.327092in}{2.670126in}}%
\pgfpathlineto{\pgfqpoint{5.351517in}{2.663018in}}%
\pgfpathlineto{\pgfqpoint{5.362486in}{2.659970in}}%
\pgfpathlineto{\pgfqpoint{5.427259in}{2.643881in}}%
\pgfpathlineto{\pgfqpoint{5.432634in}{2.642323in}}%
\pgfpathlineto{\pgfqpoint{5.451430in}{2.638083in}}%
\pgfpathlineto{\pgfqpoint{5.456540in}{2.636686in}}%
\pgfpathlineto{\pgfqpoint{5.473672in}{2.632763in}}%
\pgfpathlineto{\pgfqpoint{5.480057in}{2.631270in}}%
\pgfpathlineto{\pgfqpoint{5.532847in}{2.620635in}}%
\pgfpathlineto{\pgfqpoint{5.534545in}{2.621134in}}%
\pgfpathlineto{\pgfqpoint{5.534545in}{2.621134in}}%
\pgfusepath{stroke}%
\end{pgfscope}%
\begin{pgfscope}%
\pgfpathrectangle{\pgfqpoint{0.800000in}{2.544000in}}{\pgfqpoint{4.960000in}{1.680000in}} %
\pgfusepath{clip}%
\pgfsetrectcap%
\pgfsetroundjoin%
\pgfsetlinewidth{1.505625pt}%
\definecolor{currentstroke}{rgb}{1.000000,0.498039,0.054902}%
\pgfsetstrokecolor{currentstroke}%
\pgfsetdash{}{0pt}%
\pgfpathmoveto{\pgfqpoint{1.025455in}{4.147636in}}%
\pgfpathlineto{\pgfqpoint{1.339880in}{4.135527in}}%
\pgfpathlineto{\pgfqpoint{1.613298in}{4.122821in}}%
\pgfpathlineto{\pgfqpoint{1.852728in}{4.109517in}}%
\pgfpathlineto{\pgfqpoint{2.063901in}{4.095602in}}%
\pgfpathlineto{\pgfqpoint{2.251196in}{4.081076in}}%
\pgfpathlineto{\pgfqpoint{2.418454in}{4.065915in}}%
\pgfpathlineto{\pgfqpoint{2.568462in}{4.050125in}}%
\pgfpathlineto{\pgfqpoint{2.703952in}{4.033664in}}%
\pgfpathlineto{\pgfqpoint{2.826700in}{4.016546in}}%
\pgfpathlineto{\pgfqpoint{2.938472in}{3.998752in}}%
\pgfpathlineto{\pgfqpoint{3.041008in}{3.980209in}}%
\pgfpathlineto{\pgfqpoint{3.135162in}{3.960955in}}%
\pgfpathlineto{\pgfqpoint{3.222388in}{3.940876in}}%
\pgfpathlineto{\pgfqpoint{3.303483in}{3.919946in}}%
\pgfpathlineto{\pgfqpoint{3.378538in}{3.898323in}}%
\pgfpathlineto{\pgfqpoint{3.449225in}{3.875683in}}%
\pgfpathlineto{\pgfqpoint{3.515662in}{3.852112in}}%
\pgfpathlineto{\pgfqpoint{3.578451in}{3.827529in}}%
\pgfpathlineto{\pgfqpoint{3.638976in}{3.801456in}}%
\pgfpathlineto{\pgfqpoint{3.695957in}{3.774550in}}%
\pgfpathlineto{\pgfqpoint{3.750876in}{3.746241in}}%
\pgfpathlineto{\pgfqpoint{3.805270in}{3.715709in}}%
\pgfpathlineto{\pgfqpoint{3.857694in}{3.683771in}}%
\pgfpathlineto{\pgfqpoint{3.909404in}{3.649723in}}%
\pgfpathlineto{\pgfqpoint{3.960903in}{3.613216in}}%
\pgfpathlineto{\pgfqpoint{4.013869in}{3.572940in}}%
\pgfpathlineto{\pgfqpoint{4.068586in}{3.528490in}}%
\pgfpathlineto{\pgfqpoint{4.124974in}{3.479864in}}%
\pgfpathlineto{\pgfqpoint{4.188886in}{3.421769in}}%
\pgfpathlineto{\pgfqpoint{4.270031in}{3.344716in}}%
\pgfpathlineto{\pgfqpoint{4.473570in}{3.149806in}}%
\pgfpathlineto{\pgfqpoint{4.537895in}{3.092120in}}%
\pgfpathlineto{\pgfqpoint{4.596395in}{3.042598in}}%
\pgfpathlineto{\pgfqpoint{4.652360in}{2.998193in}}%
\pgfpathlineto{\pgfqpoint{4.705146in}{2.959115in}}%
\pgfpathlineto{\pgfqpoint{4.757772in}{2.922891in}}%
\pgfpathlineto{\pgfqpoint{4.811053in}{2.888942in}}%
\pgfpathlineto{\pgfqpoint{4.865156in}{2.857158in}}%
\pgfpathlineto{\pgfqpoint{4.919819in}{2.827637in}}%
\pgfpathlineto{\pgfqpoint{4.976403in}{2.799625in}}%
\pgfpathlineto{\pgfqpoint{5.034627in}{2.773280in}}%
\pgfpathlineto{\pgfqpoint{5.095446in}{2.748199in}}%
\pgfpathlineto{\pgfqpoint{5.159519in}{2.724196in}}%
\pgfpathlineto{\pgfqpoint{5.227116in}{2.701272in}}%
\pgfpathlineto{\pgfqpoint{5.298891in}{2.679310in}}%
\pgfpathlineto{\pgfqpoint{5.375167in}{2.658323in}}%
\pgfpathlineto{\pgfqpoint{5.456540in}{2.638255in}}%
\pgfpathlineto{\pgfqpoint{5.534545in}{2.620970in}}%
\pgfpathlineto{\pgfqpoint{5.534545in}{2.620970in}}%
\pgfusepath{stroke}%
\end{pgfscope}%
\begin{pgfscope}%
\pgfsetrectcap%
\pgfsetmiterjoin%
\pgfsetlinewidth{0.803000pt}%
\definecolor{currentstroke}{rgb}{0.000000,0.000000,0.000000}%
\pgfsetstrokecolor{currentstroke}%
\pgfsetdash{}{0pt}%
\pgfpathmoveto{\pgfqpoint{0.800000in}{2.544000in}}%
\pgfpathlineto{\pgfqpoint{0.800000in}{4.224000in}}%
\pgfusepath{stroke}%
\end{pgfscope}%
\begin{pgfscope}%
\pgfsetrectcap%
\pgfsetmiterjoin%
\pgfsetlinewidth{0.803000pt}%
\definecolor{currentstroke}{rgb}{0.000000,0.000000,0.000000}%
\pgfsetstrokecolor{currentstroke}%
\pgfsetdash{}{0pt}%
\pgfpathmoveto{\pgfqpoint{5.760000in}{2.544000in}}%
\pgfpathlineto{\pgfqpoint{5.760000in}{4.224000in}}%
\pgfusepath{stroke}%
\end{pgfscope}%
\begin{pgfscope}%
\pgfsetrectcap%
\pgfsetmiterjoin%
\pgfsetlinewidth{0.803000pt}%
\definecolor{currentstroke}{rgb}{0.000000,0.000000,0.000000}%
\pgfsetstrokecolor{currentstroke}%
\pgfsetdash{}{0pt}%
\pgfpathmoveto{\pgfqpoint{0.800000in}{2.544000in}}%
\pgfpathlineto{\pgfqpoint{5.760000in}{2.544000in}}%
\pgfusepath{stroke}%
\end{pgfscope}%
\begin{pgfscope}%
\pgfsetrectcap%
\pgfsetmiterjoin%
\pgfsetlinewidth{0.803000pt}%
\definecolor{currentstroke}{rgb}{0.000000,0.000000,0.000000}%
\pgfsetstrokecolor{currentstroke}%
\pgfsetdash{}{0pt}%
\pgfpathmoveto{\pgfqpoint{0.800000in}{4.224000in}}%
\pgfpathlineto{\pgfqpoint{5.760000in}{4.224000in}}%
\pgfusepath{stroke}%
\end{pgfscope}%
\begin{pgfscope}%
\pgfsetbuttcap%
\pgfsetmiterjoin%
\definecolor{currentfill}{rgb}{1.000000,1.000000,1.000000}%
\pgfsetfillcolor{currentfill}%
\pgfsetlinewidth{0.000000pt}%
\definecolor{currentstroke}{rgb}{0.000000,0.000000,0.000000}%
\pgfsetstrokecolor{currentstroke}%
\pgfsetstrokeopacity{0.000000}%
\pgfsetdash{}{0pt}%
\pgfpathmoveto{\pgfqpoint{0.800000in}{0.528000in}}%
\pgfpathlineto{\pgfqpoint{5.760000in}{0.528000in}}%
\pgfpathlineto{\pgfqpoint{5.760000in}{2.208000in}}%
\pgfpathlineto{\pgfqpoint{0.800000in}{2.208000in}}%
\pgfpathclose%
\pgfusepath{fill}%
\end{pgfscope}%
\begin{pgfscope}%
\pgfsetbuttcap%
\pgfsetroundjoin%
\definecolor{currentfill}{rgb}{0.000000,0.000000,0.000000}%
\pgfsetfillcolor{currentfill}%
\pgfsetlinewidth{0.803000pt}%
\definecolor{currentstroke}{rgb}{0.000000,0.000000,0.000000}%
\pgfsetstrokecolor{currentstroke}%
\pgfsetdash{}{0pt}%
\pgfsys@defobject{currentmarker}{\pgfqpoint{0.000000in}{-0.048611in}}{\pgfqpoint{0.000000in}{0.000000in}}{%
\pgfpathmoveto{\pgfqpoint{0.000000in}{0.000000in}}%
\pgfpathlineto{\pgfqpoint{0.000000in}{-0.048611in}}%
\pgfusepath{stroke,fill}%
}%
\begin{pgfscope}%
\pgfsys@transformshift{1.025455in}{0.528000in}%
\pgfsys@useobject{currentmarker}{}%
\end{pgfscope}%
\end{pgfscope}%
\begin{pgfscope}%
\pgftext[x=1.025455in,y=0.430778in,,top]{\sffamily\fontsize{10.000000}{12.000000}\selectfont \(\displaystyle 0.0\)}%
\end{pgfscope}%
\begin{pgfscope}%
\pgfsetbuttcap%
\pgfsetroundjoin%
\definecolor{currentfill}{rgb}{0.000000,0.000000,0.000000}%
\pgfsetfillcolor{currentfill}%
\pgfsetlinewidth{0.803000pt}%
\definecolor{currentstroke}{rgb}{0.000000,0.000000,0.000000}%
\pgfsetstrokecolor{currentstroke}%
\pgfsetdash{}{0pt}%
\pgfsys@defobject{currentmarker}{\pgfqpoint{0.000000in}{-0.048611in}}{\pgfqpoint{0.000000in}{0.000000in}}{%
\pgfpathmoveto{\pgfqpoint{0.000000in}{0.000000in}}%
\pgfpathlineto{\pgfqpoint{0.000000in}{-0.048611in}}%
\pgfusepath{stroke,fill}%
}%
\begin{pgfscope}%
\pgfsys@transformshift{1.804309in}{0.528000in}%
\pgfsys@useobject{currentmarker}{}%
\end{pgfscope}%
\end{pgfscope}%
\begin{pgfscope}%
\pgftext[x=1.804309in,y=0.430778in,,top]{\sffamily\fontsize{10.000000}{12.000000}\selectfont \(\displaystyle 0.5\)}%
\end{pgfscope}%
\begin{pgfscope}%
\pgfsetbuttcap%
\pgfsetroundjoin%
\definecolor{currentfill}{rgb}{0.000000,0.000000,0.000000}%
\pgfsetfillcolor{currentfill}%
\pgfsetlinewidth{0.803000pt}%
\definecolor{currentstroke}{rgb}{0.000000,0.000000,0.000000}%
\pgfsetstrokecolor{currentstroke}%
\pgfsetdash{}{0pt}%
\pgfsys@defobject{currentmarker}{\pgfqpoint{0.000000in}{-0.048611in}}{\pgfqpoint{0.000000in}{0.000000in}}{%
\pgfpathmoveto{\pgfqpoint{0.000000in}{0.000000in}}%
\pgfpathlineto{\pgfqpoint{0.000000in}{-0.048611in}}%
\pgfusepath{stroke,fill}%
}%
\begin{pgfscope}%
\pgfsys@transformshift{2.583164in}{0.528000in}%
\pgfsys@useobject{currentmarker}{}%
\end{pgfscope}%
\end{pgfscope}%
\begin{pgfscope}%
\pgftext[x=2.583164in,y=0.430778in,,top]{\sffamily\fontsize{10.000000}{12.000000}\selectfont \(\displaystyle 1.0\)}%
\end{pgfscope}%
\begin{pgfscope}%
\pgfsetbuttcap%
\pgfsetroundjoin%
\definecolor{currentfill}{rgb}{0.000000,0.000000,0.000000}%
\pgfsetfillcolor{currentfill}%
\pgfsetlinewidth{0.803000pt}%
\definecolor{currentstroke}{rgb}{0.000000,0.000000,0.000000}%
\pgfsetstrokecolor{currentstroke}%
\pgfsetdash{}{0pt}%
\pgfsys@defobject{currentmarker}{\pgfqpoint{0.000000in}{-0.048611in}}{\pgfqpoint{0.000000in}{0.000000in}}{%
\pgfpathmoveto{\pgfqpoint{0.000000in}{0.000000in}}%
\pgfpathlineto{\pgfqpoint{0.000000in}{-0.048611in}}%
\pgfusepath{stroke,fill}%
}%
\begin{pgfscope}%
\pgfsys@transformshift{3.362019in}{0.528000in}%
\pgfsys@useobject{currentmarker}{}%
\end{pgfscope}%
\end{pgfscope}%
\begin{pgfscope}%
\pgftext[x=3.362019in,y=0.430778in,,top]{\sffamily\fontsize{10.000000}{12.000000}\selectfont \(\displaystyle 1.5\)}%
\end{pgfscope}%
\begin{pgfscope}%
\pgfsetbuttcap%
\pgfsetroundjoin%
\definecolor{currentfill}{rgb}{0.000000,0.000000,0.000000}%
\pgfsetfillcolor{currentfill}%
\pgfsetlinewidth{0.803000pt}%
\definecolor{currentstroke}{rgb}{0.000000,0.000000,0.000000}%
\pgfsetstrokecolor{currentstroke}%
\pgfsetdash{}{0pt}%
\pgfsys@defobject{currentmarker}{\pgfqpoint{0.000000in}{-0.048611in}}{\pgfqpoint{0.000000in}{0.000000in}}{%
\pgfpathmoveto{\pgfqpoint{0.000000in}{0.000000in}}%
\pgfpathlineto{\pgfqpoint{0.000000in}{-0.048611in}}%
\pgfusepath{stroke,fill}%
}%
\begin{pgfscope}%
\pgfsys@transformshift{4.140873in}{0.528000in}%
\pgfsys@useobject{currentmarker}{}%
\end{pgfscope}%
\end{pgfscope}%
\begin{pgfscope}%
\pgftext[x=4.140873in,y=0.430778in,,top]{\sffamily\fontsize{10.000000}{12.000000}\selectfont \(\displaystyle 2.0\)}%
\end{pgfscope}%
\begin{pgfscope}%
\pgfsetbuttcap%
\pgfsetroundjoin%
\definecolor{currentfill}{rgb}{0.000000,0.000000,0.000000}%
\pgfsetfillcolor{currentfill}%
\pgfsetlinewidth{0.803000pt}%
\definecolor{currentstroke}{rgb}{0.000000,0.000000,0.000000}%
\pgfsetstrokecolor{currentstroke}%
\pgfsetdash{}{0pt}%
\pgfsys@defobject{currentmarker}{\pgfqpoint{0.000000in}{-0.048611in}}{\pgfqpoint{0.000000in}{0.000000in}}{%
\pgfpathmoveto{\pgfqpoint{0.000000in}{0.000000in}}%
\pgfpathlineto{\pgfqpoint{0.000000in}{-0.048611in}}%
\pgfusepath{stroke,fill}%
}%
\begin{pgfscope}%
\pgfsys@transformshift{4.919728in}{0.528000in}%
\pgfsys@useobject{currentmarker}{}%
\end{pgfscope}%
\end{pgfscope}%
\begin{pgfscope}%
\pgftext[x=4.919728in,y=0.430778in,,top]{\sffamily\fontsize{10.000000}{12.000000}\selectfont \(\displaystyle 2.5\)}%
\end{pgfscope}%
\begin{pgfscope}%
\pgfsetbuttcap%
\pgfsetroundjoin%
\definecolor{currentfill}{rgb}{0.000000,0.000000,0.000000}%
\pgfsetfillcolor{currentfill}%
\pgfsetlinewidth{0.803000pt}%
\definecolor{currentstroke}{rgb}{0.000000,0.000000,0.000000}%
\pgfsetstrokecolor{currentstroke}%
\pgfsetdash{}{0pt}%
\pgfsys@defobject{currentmarker}{\pgfqpoint{0.000000in}{-0.048611in}}{\pgfqpoint{0.000000in}{0.000000in}}{%
\pgfpathmoveto{\pgfqpoint{0.000000in}{0.000000in}}%
\pgfpathlineto{\pgfqpoint{0.000000in}{-0.048611in}}%
\pgfusepath{stroke,fill}%
}%
\begin{pgfscope}%
\pgfsys@transformshift{5.698583in}{0.528000in}%
\pgfsys@useobject{currentmarker}{}%
\end{pgfscope}%
\end{pgfscope}%
\begin{pgfscope}%
\pgftext[x=5.698583in,y=0.430778in,,top]{\sffamily\fontsize{10.000000}{12.000000}\selectfont \(\displaystyle 3.0\)}%
\end{pgfscope}%
\begin{pgfscope}%
\pgftext[x=5.760000in,y=0.265778in,right,top]{\sffamily\fontsize{10.000000}{12.000000}\selectfont \(\displaystyle \times10^{7}\)}%
\end{pgfscope}%
\begin{pgfscope}%
\pgfsetbuttcap%
\pgfsetroundjoin%
\definecolor{currentfill}{rgb}{0.000000,0.000000,0.000000}%
\pgfsetfillcolor{currentfill}%
\pgfsetlinewidth{0.803000pt}%
\definecolor{currentstroke}{rgb}{0.000000,0.000000,0.000000}%
\pgfsetstrokecolor{currentstroke}%
\pgfsetdash{}{0pt}%
\pgfsys@defobject{currentmarker}{\pgfqpoint{-0.048611in}{0.000000in}}{\pgfqpoint{0.000000in}{0.000000in}}{%
\pgfpathmoveto{\pgfqpoint{0.000000in}{0.000000in}}%
\pgfpathlineto{\pgfqpoint{-0.048611in}{0.000000in}}%
\pgfusepath{stroke,fill}%
}%
\begin{pgfscope}%
\pgfsys@transformshift{0.800000in}{0.788956in}%
\pgfsys@useobject{currentmarker}{}%
\end{pgfscope}%
\end{pgfscope}%
\begin{pgfscope}%
\pgftext[x=0.278394in,y=0.740762in,left,base]{\sffamily\fontsize{10.000000}{12.000000}\selectfont \(\displaystyle -0.005\)}%
\end{pgfscope}%
\begin{pgfscope}%
\pgfsetbuttcap%
\pgfsetroundjoin%
\definecolor{currentfill}{rgb}{0.000000,0.000000,0.000000}%
\pgfsetfillcolor{currentfill}%
\pgfsetlinewidth{0.803000pt}%
\definecolor{currentstroke}{rgb}{0.000000,0.000000,0.000000}%
\pgfsetstrokecolor{currentstroke}%
\pgfsetdash{}{0pt}%
\pgfsys@defobject{currentmarker}{\pgfqpoint{-0.048611in}{0.000000in}}{\pgfqpoint{0.000000in}{0.000000in}}{%
\pgfpathmoveto{\pgfqpoint{0.000000in}{0.000000in}}%
\pgfpathlineto{\pgfqpoint{-0.048611in}{0.000000in}}%
\pgfusepath{stroke,fill}%
}%
\begin{pgfscope}%
\pgfsys@transformshift{0.800000in}{1.325446in}%
\pgfsys@useobject{currentmarker}{}%
\end{pgfscope}%
\end{pgfscope}%
\begin{pgfscope}%
\pgftext[x=0.386419in,y=1.277252in,left,base]{\sffamily\fontsize{10.000000}{12.000000}\selectfont \(\displaystyle 0.000\)}%
\end{pgfscope}%
\begin{pgfscope}%
\pgfsetbuttcap%
\pgfsetroundjoin%
\definecolor{currentfill}{rgb}{0.000000,0.000000,0.000000}%
\pgfsetfillcolor{currentfill}%
\pgfsetlinewidth{0.803000pt}%
\definecolor{currentstroke}{rgb}{0.000000,0.000000,0.000000}%
\pgfsetstrokecolor{currentstroke}%
\pgfsetdash{}{0pt}%
\pgfsys@defobject{currentmarker}{\pgfqpoint{-0.048611in}{0.000000in}}{\pgfqpoint{0.000000in}{0.000000in}}{%
\pgfpathmoveto{\pgfqpoint{0.000000in}{0.000000in}}%
\pgfpathlineto{\pgfqpoint{-0.048611in}{0.000000in}}%
\pgfusepath{stroke,fill}%
}%
\begin{pgfscope}%
\pgfsys@transformshift{0.800000in}{1.861936in}%
\pgfsys@useobject{currentmarker}{}%
\end{pgfscope}%
\end{pgfscope}%
\begin{pgfscope}%
\pgftext[x=0.386419in,y=1.813742in,left,base]{\sffamily\fontsize{10.000000}{12.000000}\selectfont \(\displaystyle 0.005\)}%
\end{pgfscope}%
\begin{pgfscope}%
\pgfpathrectangle{\pgfqpoint{0.800000in}{0.528000in}}{\pgfqpoint{4.960000in}{1.680000in}} %
\pgfusepath{clip}%
\pgfsetrectcap%
\pgfsetroundjoin%
\pgfsetlinewidth{1.505625pt}%
\definecolor{currentstroke}{rgb}{0.121569,0.466667,0.705882}%
\pgfsetstrokecolor{currentstroke}%
\pgfsetdash{}{0pt}%
\pgfpathmoveto{\pgfqpoint{1.025455in}{1.325446in}}%
\pgfpathlineto{\pgfqpoint{1.025609in}{1.325617in}}%
\pgfpathlineto{\pgfqpoint{1.025695in}{1.325397in}}%
\pgfpathlineto{\pgfqpoint{1.025695in}{1.325397in}}%
\pgfpathlineto{\pgfqpoint{1.025796in}{1.325263in}}%
\pgfpathlineto{\pgfqpoint{1.025898in}{1.325420in}}%
\pgfpathlineto{\pgfqpoint{1.025898in}{1.325420in}}%
\pgfpathlineto{\pgfqpoint{1.026065in}{1.325612in}}%
\pgfpathlineto{\pgfqpoint{1.026177in}{1.325318in}}%
\pgfpathlineto{\pgfqpoint{1.026177in}{1.325318in}}%
\pgfpathlineto{\pgfqpoint{1.026242in}{1.325257in}}%
\pgfpathlineto{\pgfqpoint{1.026342in}{1.325394in}}%
\pgfpathlineto{\pgfqpoint{1.026342in}{1.325394in}}%
\pgfpathlineto{\pgfqpoint{1.027415in}{1.325613in}}%
\pgfpathlineto{\pgfqpoint{1.027515in}{1.325375in}}%
\pgfpathlineto{\pgfqpoint{1.027613in}{1.325250in}}%
\pgfpathlineto{\pgfqpoint{1.027712in}{1.325398in}}%
\pgfpathlineto{\pgfqpoint{1.027712in}{1.325398in}}%
\pgfpathlineto{\pgfqpoint{1.028779in}{1.325605in}}%
\pgfpathlineto{\pgfqpoint{1.028879in}{1.325369in}}%
\pgfpathlineto{\pgfqpoint{1.028977in}{1.325239in}}%
\pgfpathlineto{\pgfqpoint{1.029075in}{1.325385in}}%
\pgfpathlineto{\pgfqpoint{1.029075in}{1.325385in}}%
\pgfpathlineto{\pgfqpoint{1.030151in}{1.325593in}}%
\pgfpathlineto{\pgfqpoint{1.030258in}{1.325324in}}%
\pgfpathlineto{\pgfqpoint{1.030356in}{1.325236in}}%
\pgfpathlineto{\pgfqpoint{1.030454in}{1.325402in}}%
\pgfpathlineto{\pgfqpoint{1.030454in}{1.325402in}}%
\pgfpathlineto{\pgfqpoint{1.030587in}{1.325597in}}%
\pgfpathlineto{\pgfqpoint{1.030728in}{1.325286in}}%
\pgfpathlineto{\pgfqpoint{1.030728in}{1.325286in}}%
\pgfpathlineto{\pgfqpoint{1.030793in}{1.325225in}}%
\pgfpathlineto{\pgfqpoint{1.030891in}{1.325358in}}%
\pgfpathlineto{\pgfqpoint{1.030891in}{1.325358in}}%
\pgfpathlineto{\pgfqpoint{1.031973in}{1.325582in}}%
\pgfpathlineto{\pgfqpoint{1.032071in}{1.325348in}}%
\pgfpathlineto{\pgfqpoint{1.032168in}{1.325217in}}%
\pgfpathlineto{\pgfqpoint{1.032300in}{1.325435in}}%
\pgfpathlineto{\pgfqpoint{1.032300in}{1.325435in}}%
\pgfpathlineto{\pgfqpoint{1.032432in}{1.325578in}}%
\pgfpathlineto{\pgfqpoint{1.032520in}{1.325371in}}%
\pgfpathlineto{\pgfqpoint{1.032520in}{1.325371in}}%
\pgfpathlineto{\pgfqpoint{1.032617in}{1.325211in}}%
\pgfpathlineto{\pgfqpoint{1.032749in}{1.325414in}}%
\pgfpathlineto{\pgfqpoint{1.032749in}{1.325414in}}%
\pgfpathlineto{\pgfqpoint{1.032882in}{1.325579in}}%
\pgfpathlineto{\pgfqpoint{1.032984in}{1.325344in}}%
\pgfpathlineto{\pgfqpoint{1.032984in}{1.325344in}}%
\pgfpathlineto{\pgfqpoint{1.033081in}{1.325210in}}%
\pgfpathlineto{\pgfqpoint{1.033213in}{1.325426in}}%
\pgfpathlineto{\pgfqpoint{1.033213in}{1.325426in}}%
\pgfpathlineto{\pgfqpoint{1.033346in}{1.325572in}}%
\pgfpathlineto{\pgfqpoint{1.033442in}{1.325338in}}%
\pgfpathlineto{\pgfqpoint{1.033442in}{1.325338in}}%
\pgfpathlineto{\pgfqpoint{1.033539in}{1.325207in}}%
\pgfpathlineto{\pgfqpoint{1.033671in}{1.325426in}}%
\pgfpathlineto{\pgfqpoint{1.033671in}{1.325426in}}%
\pgfpathlineto{\pgfqpoint{1.033804in}{1.325568in}}%
\pgfpathlineto{\pgfqpoint{1.033893in}{1.325356in}}%
\pgfpathlineto{\pgfqpoint{1.033893in}{1.325356in}}%
\pgfpathlineto{\pgfqpoint{1.033990in}{1.325202in}}%
\pgfpathlineto{\pgfqpoint{1.034122in}{1.325408in}}%
\pgfpathlineto{\pgfqpoint{1.034122in}{1.325408in}}%
\pgfpathlineto{\pgfqpoint{1.034255in}{1.325569in}}%
\pgfpathlineto{\pgfqpoint{1.034346in}{1.325367in}}%
\pgfpathlineto{\pgfqpoint{1.034346in}{1.325367in}}%
\pgfpathlineto{\pgfqpoint{1.034469in}{1.325209in}}%
\pgfpathlineto{\pgfqpoint{1.034568in}{1.325378in}}%
\pgfpathlineto{\pgfqpoint{1.034568in}{1.325378in}}%
\pgfpathlineto{\pgfqpoint{1.034701in}{1.325569in}}%
\pgfpathlineto{\pgfqpoint{1.034840in}{1.325256in}}%
\pgfpathlineto{\pgfqpoint{1.034840in}{1.325256in}}%
\pgfpathlineto{\pgfqpoint{1.034906in}{1.325195in}}%
\pgfpathlineto{\pgfqpoint{1.035004in}{1.325329in}}%
\pgfpathlineto{\pgfqpoint{1.035004in}{1.325329in}}%
\pgfpathlineto{\pgfqpoint{1.036087in}{1.325557in}}%
\pgfpathlineto{\pgfqpoint{1.036192in}{1.325306in}}%
\pgfpathlineto{\pgfqpoint{1.036290in}{1.325189in}}%
\pgfpathlineto{\pgfqpoint{1.036389in}{1.325342in}}%
\pgfpathlineto{\pgfqpoint{1.036389in}{1.325342in}}%
\pgfpathlineto{\pgfqpoint{1.036555in}{1.325547in}}%
\pgfpathlineto{\pgfqpoint{1.036679in}{1.325232in}}%
\pgfpathlineto{\pgfqpoint{1.036679in}{1.325232in}}%
\pgfpathlineto{\pgfqpoint{1.036744in}{1.325184in}}%
\pgfpathlineto{\pgfqpoint{1.036843in}{1.325329in}}%
\pgfpathlineto{\pgfqpoint{1.036843in}{1.325329in}}%
\pgfpathlineto{\pgfqpoint{1.037922in}{1.325544in}}%
\pgfpathlineto{\pgfqpoint{1.038015in}{1.325331in}}%
\pgfpathlineto{\pgfqpoint{1.038113in}{1.325172in}}%
\pgfpathlineto{\pgfqpoint{1.038245in}{1.325377in}}%
\pgfpathlineto{\pgfqpoint{1.038245in}{1.325377in}}%
\pgfpathlineto{\pgfqpoint{1.038379in}{1.325542in}}%
\pgfpathlineto{\pgfqpoint{1.038474in}{1.325330in}}%
\pgfpathlineto{\pgfqpoint{1.038474in}{1.325330in}}%
\pgfpathlineto{\pgfqpoint{1.038572in}{1.325169in}}%
\pgfpathlineto{\pgfqpoint{1.038704in}{1.325373in}}%
\pgfpathlineto{\pgfqpoint{1.038704in}{1.325373in}}%
\pgfpathlineto{\pgfqpoint{1.038838in}{1.325539in}}%
\pgfpathlineto{\pgfqpoint{1.038934in}{1.325322in}}%
\pgfpathlineto{\pgfqpoint{1.038934in}{1.325322in}}%
\pgfpathlineto{\pgfqpoint{1.039032in}{1.325166in}}%
\pgfpathlineto{\pgfqpoint{1.039164in}{1.325372in}}%
\pgfpathlineto{\pgfqpoint{1.039164in}{1.325372in}}%
\pgfpathlineto{\pgfqpoint{1.039298in}{1.325536in}}%
\pgfpathlineto{\pgfqpoint{1.039397in}{1.325306in}}%
\pgfpathlineto{\pgfqpoint{1.039397in}{1.325306in}}%
\pgfpathlineto{\pgfqpoint{1.039495in}{1.325163in}}%
\pgfpathlineto{\pgfqpoint{1.039628in}{1.325378in}}%
\pgfpathlineto{\pgfqpoint{1.039628in}{1.325378in}}%
\pgfpathlineto{\pgfqpoint{1.039762in}{1.325531in}}%
\pgfpathlineto{\pgfqpoint{1.039868in}{1.325267in}}%
\pgfpathlineto{\pgfqpoint{1.039868in}{1.325267in}}%
\pgfpathlineto{\pgfqpoint{1.039966in}{1.325165in}}%
\pgfpathlineto{\pgfqpoint{1.040066in}{1.325326in}}%
\pgfpathlineto{\pgfqpoint{1.040066in}{1.325326in}}%
\pgfpathlineto{\pgfqpoint{1.040232in}{1.325519in}}%
\pgfpathlineto{\pgfqpoint{1.040345in}{1.325220in}}%
\pgfpathlineto{\pgfqpoint{1.040345in}{1.325220in}}%
\pgfpathlineto{\pgfqpoint{1.040443in}{1.325176in}}%
\pgfpathlineto{\pgfqpoint{1.040510in}{1.325288in}}%
\pgfpathlineto{\pgfqpoint{1.040510in}{1.325288in}}%
\pgfpathlineto{\pgfqpoint{1.041599in}{1.325520in}}%
\pgfpathlineto{\pgfqpoint{1.041693in}{1.325305in}}%
\pgfpathlineto{\pgfqpoint{1.041791in}{1.325146in}}%
\pgfpathlineto{\pgfqpoint{1.041924in}{1.325352in}}%
\pgfpathlineto{\pgfqpoint{1.041924in}{1.325352in}}%
\pgfpathlineto{\pgfqpoint{1.042058in}{1.325518in}}%
\pgfpathlineto{\pgfqpoint{1.042158in}{1.325286in}}%
\pgfpathlineto{\pgfqpoint{1.042158in}{1.325286in}}%
\pgfpathlineto{\pgfqpoint{1.042256in}{1.325143in}}%
\pgfpathlineto{\pgfqpoint{1.042389in}{1.325359in}}%
\pgfpathlineto{\pgfqpoint{1.042389in}{1.325359in}}%
\pgfpathlineto{\pgfqpoint{1.042523in}{1.325512in}}%
\pgfpathlineto{\pgfqpoint{1.042621in}{1.325277in}}%
\pgfpathlineto{\pgfqpoint{1.042621in}{1.325277in}}%
\pgfpathlineto{\pgfqpoint{1.042719in}{1.325141in}}%
\pgfpathlineto{\pgfqpoint{1.042852in}{1.325360in}}%
\pgfpathlineto{\pgfqpoint{1.042852in}{1.325360in}}%
\pgfpathlineto{\pgfqpoint{1.042986in}{1.325508in}}%
\pgfpathlineto{\pgfqpoint{1.043074in}{1.325300in}}%
\pgfpathlineto{\pgfqpoint{1.043074in}{1.325300in}}%
\pgfpathlineto{\pgfqpoint{1.043172in}{1.325136in}}%
\pgfpathlineto{\pgfqpoint{1.043305in}{1.325340in}}%
\pgfpathlineto{\pgfqpoint{1.043305in}{1.325340in}}%
\pgfpathlineto{\pgfqpoint{1.043440in}{1.325509in}}%
\pgfpathlineto{\pgfqpoint{1.043539in}{1.325283in}}%
\pgfpathlineto{\pgfqpoint{1.043539in}{1.325283in}}%
\pgfpathlineto{\pgfqpoint{1.043638in}{1.325133in}}%
\pgfpathlineto{\pgfqpoint{1.043771in}{1.325346in}}%
\pgfpathlineto{\pgfqpoint{1.043771in}{1.325346in}}%
\pgfpathlineto{\pgfqpoint{1.043905in}{1.325504in}}%
\pgfpathlineto{\pgfqpoint{1.044003in}{1.325270in}}%
\pgfpathlineto{\pgfqpoint{1.044003in}{1.325270in}}%
\pgfpathlineto{\pgfqpoint{1.044102in}{1.325130in}}%
\pgfpathlineto{\pgfqpoint{1.044235in}{1.325349in}}%
\pgfpathlineto{\pgfqpoint{1.044235in}{1.325349in}}%
\pgfpathlineto{\pgfqpoint{1.044369in}{1.325499in}}%
\pgfpathlineto{\pgfqpoint{1.044462in}{1.325275in}}%
\pgfpathlineto{\pgfqpoint{1.044462in}{1.325275in}}%
\pgfpathlineto{\pgfqpoint{1.044561in}{1.325126in}}%
\pgfpathlineto{\pgfqpoint{1.044694in}{1.325340in}}%
\pgfpathlineto{\pgfqpoint{1.044694in}{1.325340in}}%
\pgfpathlineto{\pgfqpoint{1.044828in}{1.325498in}}%
\pgfpathlineto{\pgfqpoint{1.044925in}{1.325268in}}%
\pgfpathlineto{\pgfqpoint{1.044925in}{1.325268in}}%
\pgfpathlineto{\pgfqpoint{1.045024in}{1.325123in}}%
\pgfpathlineto{\pgfqpoint{1.045157in}{1.325340in}}%
\pgfpathlineto{\pgfqpoint{1.045157in}{1.325340in}}%
\pgfpathlineto{\pgfqpoint{1.045291in}{1.325494in}}%
\pgfpathlineto{\pgfqpoint{1.045390in}{1.325256in}}%
\pgfpathlineto{\pgfqpoint{1.045390in}{1.325256in}}%
\pgfpathlineto{\pgfqpoint{1.045488in}{1.325121in}}%
\pgfpathlineto{\pgfqpoint{1.045622in}{1.325343in}}%
\pgfpathlineto{\pgfqpoint{1.045622in}{1.325343in}}%
\pgfpathlineto{\pgfqpoint{1.045756in}{1.325489in}}%
\pgfpathlineto{\pgfqpoint{1.045851in}{1.325255in}}%
\pgfpathlineto{\pgfqpoint{1.045851in}{1.325255in}}%
\pgfpathlineto{\pgfqpoint{1.045950in}{1.325117in}}%
\pgfpathlineto{\pgfqpoint{1.046083in}{1.325337in}}%
\pgfpathlineto{\pgfqpoint{1.046083in}{1.325337in}}%
\pgfpathlineto{\pgfqpoint{1.046217in}{1.325487in}}%
\pgfpathlineto{\pgfqpoint{1.046310in}{1.325263in}}%
\pgfpathlineto{\pgfqpoint{1.046310in}{1.325263in}}%
\pgfpathlineto{\pgfqpoint{1.046409in}{1.325113in}}%
\pgfpathlineto{\pgfqpoint{1.046542in}{1.325327in}}%
\pgfpathlineto{\pgfqpoint{1.046542in}{1.325327in}}%
\pgfpathlineto{\pgfqpoint{1.046677in}{1.325485in}}%
\pgfpathlineto{\pgfqpoint{1.046778in}{1.325242in}}%
\pgfpathlineto{\pgfqpoint{1.046778in}{1.325242in}}%
\pgfpathlineto{\pgfqpoint{1.046877in}{1.325111in}}%
\pgfpathlineto{\pgfqpoint{1.046976in}{1.325261in}}%
\pgfpathlineto{\pgfqpoint{1.046976in}{1.325261in}}%
\pgfpathlineto{\pgfqpoint{1.048060in}{1.325479in}}%
\pgfpathlineto{\pgfqpoint{1.048163in}{1.325245in}}%
\pgfpathlineto{\pgfqpoint{1.048261in}{1.325100in}}%
\pgfpathlineto{\pgfqpoint{1.048395in}{1.325318in}}%
\pgfpathlineto{\pgfqpoint{1.048395in}{1.325318in}}%
\pgfpathlineto{\pgfqpoint{1.048530in}{1.325472in}}%
\pgfpathlineto{\pgfqpoint{1.048625in}{1.325244in}}%
\pgfpathlineto{\pgfqpoint{1.048625in}{1.325244in}}%
\pgfpathlineto{\pgfqpoint{1.048724in}{1.325097in}}%
\pgfpathlineto{\pgfqpoint{1.048858in}{1.325313in}}%
\pgfpathlineto{\pgfqpoint{1.048858in}{1.325313in}}%
\pgfpathlineto{\pgfqpoint{1.048993in}{1.325470in}}%
\pgfpathlineto{\pgfqpoint{1.049085in}{1.325254in}}%
\pgfpathlineto{\pgfqpoint{1.049085in}{1.325254in}}%
\pgfpathlineto{\pgfqpoint{1.049183in}{1.325092in}}%
\pgfpathlineto{\pgfqpoint{1.049317in}{1.325301in}}%
\pgfpathlineto{\pgfqpoint{1.049317in}{1.325301in}}%
\pgfpathlineto{\pgfqpoint{1.049452in}{1.325469in}}%
\pgfpathlineto{\pgfqpoint{1.049554in}{1.325231in}}%
\pgfpathlineto{\pgfqpoint{1.049554in}{1.325231in}}%
\pgfpathlineto{\pgfqpoint{1.049653in}{1.325091in}}%
\pgfpathlineto{\pgfqpoint{1.049787in}{1.325311in}}%
\pgfpathlineto{\pgfqpoint{1.049787in}{1.325311in}}%
\pgfpathlineto{\pgfqpoint{1.049922in}{1.325462in}}%
\pgfpathlineto{\pgfqpoint{1.050013in}{1.325243in}}%
\pgfpathlineto{\pgfqpoint{1.050013in}{1.325243in}}%
\pgfpathlineto{\pgfqpoint{1.050112in}{1.325086in}}%
\pgfpathlineto{\pgfqpoint{1.050246in}{1.325297in}}%
\pgfpathlineto{\pgfqpoint{1.050246in}{1.325297in}}%
\pgfpathlineto{\pgfqpoint{1.050381in}{1.325462in}}%
\pgfpathlineto{\pgfqpoint{1.050481in}{1.325225in}}%
\pgfpathlineto{\pgfqpoint{1.050481in}{1.325225in}}%
\pgfpathlineto{\pgfqpoint{1.050580in}{1.325084in}}%
\pgfpathlineto{\pgfqpoint{1.050714in}{1.325304in}}%
\pgfpathlineto{\pgfqpoint{1.050714in}{1.325304in}}%
\pgfpathlineto{\pgfqpoint{1.050849in}{1.325456in}}%
\pgfpathlineto{\pgfqpoint{1.050945in}{1.325225in}}%
\pgfpathlineto{\pgfqpoint{1.050945in}{1.325225in}}%
\pgfpathlineto{\pgfqpoint{1.051044in}{1.325080in}}%
\pgfpathlineto{\pgfqpoint{1.051178in}{1.325298in}}%
\pgfpathlineto{\pgfqpoint{1.051178in}{1.325298in}}%
\pgfpathlineto{\pgfqpoint{1.051313in}{1.325454in}}%
\pgfpathlineto{\pgfqpoint{1.051413in}{1.325208in}}%
\pgfpathlineto{\pgfqpoint{1.051413in}{1.325208in}}%
\pgfpathlineto{\pgfqpoint{1.051512in}{1.325078in}}%
\pgfpathlineto{\pgfqpoint{1.051612in}{1.325229in}}%
\pgfpathlineto{\pgfqpoint{1.051612in}{1.325229in}}%
\pgfpathlineto{\pgfqpoint{1.052708in}{1.325444in}}%
\pgfpathlineto{\pgfqpoint{1.052795in}{1.325242in}}%
\pgfpathlineto{\pgfqpoint{1.052915in}{1.325073in}}%
\pgfpathlineto{\pgfqpoint{1.053050in}{1.325314in}}%
\pgfpathlineto{\pgfqpoint{1.053050in}{1.325314in}}%
\pgfpathlineto{\pgfqpoint{1.053185in}{1.325431in}}%
\pgfpathlineto{\pgfqpoint{1.053259in}{1.325244in}}%
\pgfpathlineto{\pgfqpoint{1.053259in}{1.325244in}}%
\pgfpathlineto{\pgfqpoint{1.053371in}{1.325064in}}%
\pgfpathlineto{\pgfqpoint{1.053506in}{1.325290in}}%
\pgfpathlineto{\pgfqpoint{1.053506in}{1.325290in}}%
\pgfpathlineto{\pgfqpoint{1.053640in}{1.325436in}}%
\pgfpathlineto{\pgfqpoint{1.053731in}{1.325217in}}%
\pgfpathlineto{\pgfqpoint{1.053731in}{1.325217in}}%
\pgfpathlineto{\pgfqpoint{1.053830in}{1.325059in}}%
\pgfpathlineto{\pgfqpoint{1.053964in}{1.325272in}}%
\pgfpathlineto{\pgfqpoint{1.053964in}{1.325272in}}%
\pgfpathlineto{\pgfqpoint{1.054100in}{1.325437in}}%
\pgfpathlineto{\pgfqpoint{1.054200in}{1.325199in}}%
\pgfpathlineto{\pgfqpoint{1.054200in}{1.325199in}}%
\pgfpathlineto{\pgfqpoint{1.054299in}{1.325057in}}%
\pgfpathlineto{\pgfqpoint{1.054434in}{1.325278in}}%
\pgfpathlineto{\pgfqpoint{1.054434in}{1.325278in}}%
\pgfpathlineto{\pgfqpoint{1.054569in}{1.325431in}}%
\pgfpathlineto{\pgfqpoint{1.054661in}{1.325212in}}%
\pgfpathlineto{\pgfqpoint{1.054661in}{1.325212in}}%
\pgfpathlineto{\pgfqpoint{1.054761in}{1.325052in}}%
\pgfpathlineto{\pgfqpoint{1.054895in}{1.325264in}}%
\pgfpathlineto{\pgfqpoint{1.054895in}{1.325264in}}%
\pgfpathlineto{\pgfqpoint{1.055031in}{1.325431in}}%
\pgfpathlineto{\pgfqpoint{1.055127in}{1.325208in}}%
\pgfpathlineto{\pgfqpoint{1.055127in}{1.325208in}}%
\pgfpathlineto{\pgfqpoint{1.055227in}{1.325049in}}%
\pgfpathlineto{\pgfqpoint{1.055361in}{1.325261in}}%
\pgfpathlineto{\pgfqpoint{1.055361in}{1.325261in}}%
\pgfpathlineto{\pgfqpoint{1.055497in}{1.325428in}}%
\pgfpathlineto{\pgfqpoint{1.055601in}{1.325181in}}%
\pgfpathlineto{\pgfqpoint{1.055601in}{1.325181in}}%
\pgfpathlineto{\pgfqpoint{1.055700in}{1.325048in}}%
\pgfpathlineto{\pgfqpoint{1.055800in}{1.325199in}}%
\pgfpathlineto{\pgfqpoint{1.055800in}{1.325199in}}%
\pgfpathlineto{\pgfqpoint{1.056895in}{1.325419in}}%
\pgfpathlineto{\pgfqpoint{1.056991in}{1.325203in}}%
\pgfpathlineto{\pgfqpoint{1.057090in}{1.325035in}}%
\pgfpathlineto{\pgfqpoint{1.057225in}{1.325244in}}%
\pgfpathlineto{\pgfqpoint{1.057225in}{1.325244in}}%
\pgfpathlineto{\pgfqpoint{1.057361in}{1.325416in}}%
\pgfpathlineto{\pgfqpoint{1.057461in}{1.325188in}}%
\pgfpathlineto{\pgfqpoint{1.057461in}{1.325188in}}%
\pgfpathlineto{\pgfqpoint{1.057560in}{1.325032in}}%
\pgfpathlineto{\pgfqpoint{1.057695in}{1.325248in}}%
\pgfpathlineto{\pgfqpoint{1.057695in}{1.325248in}}%
\pgfpathlineto{\pgfqpoint{1.057831in}{1.325412in}}%
\pgfpathlineto{\pgfqpoint{1.057926in}{1.325192in}}%
\pgfpathlineto{\pgfqpoint{1.057926in}{1.325192in}}%
\pgfpathlineto{\pgfqpoint{1.058025in}{1.325029in}}%
\pgfpathlineto{\pgfqpoint{1.058160in}{1.325240in}}%
\pgfpathlineto{\pgfqpoint{1.058160in}{1.325240in}}%
\pgfpathlineto{\pgfqpoint{1.058296in}{1.325409in}}%
\pgfpathlineto{\pgfqpoint{1.058398in}{1.325172in}}%
\pgfpathlineto{\pgfqpoint{1.058398in}{1.325172in}}%
\pgfpathlineto{\pgfqpoint{1.058497in}{1.325026in}}%
\pgfpathlineto{\pgfqpoint{1.058632in}{1.325247in}}%
\pgfpathlineto{\pgfqpoint{1.058632in}{1.325247in}}%
\pgfpathlineto{\pgfqpoint{1.058768in}{1.325404in}}%
\pgfpathlineto{\pgfqpoint{1.058869in}{1.325156in}}%
\pgfpathlineto{\pgfqpoint{1.058869in}{1.325156in}}%
\pgfpathlineto{\pgfqpoint{1.058968in}{1.325024in}}%
\pgfpathlineto{\pgfqpoint{1.059069in}{1.325177in}}%
\pgfpathlineto{\pgfqpoint{1.059069in}{1.325177in}}%
\pgfpathlineto{\pgfqpoint{1.060174in}{1.325393in}}%
\pgfpathlineto{\pgfqpoint{1.060272in}{1.325147in}}%
\pgfpathlineto{\pgfqpoint{1.060372in}{1.325014in}}%
\pgfpathlineto{\pgfqpoint{1.060473in}{1.325166in}}%
\pgfpathlineto{\pgfqpoint{1.060473in}{1.325166in}}%
\pgfpathlineto{\pgfqpoint{1.060643in}{1.325389in}}%
\pgfpathlineto{\pgfqpoint{1.060774in}{1.325057in}}%
\pgfpathlineto{\pgfqpoint{1.060774in}{1.325057in}}%
\pgfpathlineto{\pgfqpoint{1.060840in}{1.325011in}}%
\pgfpathlineto{\pgfqpoint{1.060941in}{1.325164in}}%
\pgfpathlineto{\pgfqpoint{1.060941in}{1.325164in}}%
\pgfpathlineto{\pgfqpoint{1.061112in}{1.325386in}}%
\pgfpathlineto{\pgfqpoint{1.061240in}{1.325057in}}%
\pgfpathlineto{\pgfqpoint{1.061240in}{1.325057in}}%
\pgfpathlineto{\pgfqpoint{1.061307in}{1.325007in}}%
\pgfpathlineto{\pgfqpoint{1.061408in}{1.325156in}}%
\pgfpathlineto{\pgfqpoint{1.061408in}{1.325156in}}%
\pgfpathlineto{\pgfqpoint{1.062510in}{1.325381in}}%
\pgfpathlineto{\pgfqpoint{1.062610in}{1.325146in}}%
\pgfpathlineto{\pgfqpoint{1.062710in}{1.324995in}}%
\pgfpathlineto{\pgfqpoint{1.062846in}{1.325216in}}%
\pgfpathlineto{\pgfqpoint{1.062846in}{1.325216in}}%
\pgfpathlineto{\pgfqpoint{1.062983in}{1.325376in}}%
\pgfpathlineto{\pgfqpoint{1.063081in}{1.325136in}}%
\pgfpathlineto{\pgfqpoint{1.063081in}{1.325136in}}%
\pgfpathlineto{\pgfqpoint{1.063181in}{1.324992in}}%
\pgfpathlineto{\pgfqpoint{1.063317in}{1.325217in}}%
\pgfpathlineto{\pgfqpoint{1.063317in}{1.325217in}}%
\pgfpathlineto{\pgfqpoint{1.063454in}{1.325372in}}%
\pgfpathlineto{\pgfqpoint{1.063545in}{1.325153in}}%
\pgfpathlineto{\pgfqpoint{1.063545in}{1.325153in}}%
\pgfpathlineto{\pgfqpoint{1.063645in}{1.324988in}}%
\pgfpathlineto{\pgfqpoint{1.063781in}{1.325200in}}%
\pgfpathlineto{\pgfqpoint{1.063781in}{1.325200in}}%
\pgfpathlineto{\pgfqpoint{1.063918in}{1.325372in}}%
\pgfpathlineto{\pgfqpoint{1.064020in}{1.325130in}}%
\pgfpathlineto{\pgfqpoint{1.064020in}{1.325130in}}%
\pgfpathlineto{\pgfqpoint{1.064120in}{1.324986in}}%
\pgfpathlineto{\pgfqpoint{1.064256in}{1.325210in}}%
\pgfpathlineto{\pgfqpoint{1.064256in}{1.325210in}}%
\pgfpathlineto{\pgfqpoint{1.064393in}{1.325365in}}%
\pgfpathlineto{\pgfqpoint{1.064486in}{1.325140in}}%
\pgfpathlineto{\pgfqpoint{1.064486in}{1.325140in}}%
\pgfpathlineto{\pgfqpoint{1.064586in}{1.324981in}}%
\pgfpathlineto{\pgfqpoint{1.064722in}{1.325198in}}%
\pgfpathlineto{\pgfqpoint{1.064722in}{1.325198in}}%
\pgfpathlineto{\pgfqpoint{1.064859in}{1.325365in}}%
\pgfpathlineto{\pgfqpoint{1.064952in}{1.325152in}}%
\pgfpathlineto{\pgfqpoint{1.064952in}{1.325152in}}%
\pgfpathlineto{\pgfqpoint{1.065046in}{1.324977in}}%
\pgfpathlineto{\pgfqpoint{1.065181in}{1.325171in}}%
\pgfpathlineto{\pgfqpoint{1.065181in}{1.325171in}}%
\pgfpathlineto{\pgfqpoint{1.065318in}{1.325365in}}%
\pgfpathlineto{\pgfqpoint{1.065433in}{1.325111in}}%
\pgfpathlineto{\pgfqpoint{1.065433in}{1.325111in}}%
\pgfpathlineto{\pgfqpoint{1.065533in}{1.324976in}}%
\pgfpathlineto{\pgfqpoint{1.065635in}{1.325130in}}%
\pgfpathlineto{\pgfqpoint{1.065635in}{1.325130in}}%
\pgfpathlineto{\pgfqpoint{1.065806in}{1.325354in}}%
\pgfpathlineto{\pgfqpoint{1.065933in}{1.325027in}}%
\pgfpathlineto{\pgfqpoint{1.065933in}{1.325027in}}%
\pgfpathlineto{\pgfqpoint{1.066000in}{1.324972in}}%
\pgfpathlineto{\pgfqpoint{1.066102in}{1.325119in}}%
\pgfpathlineto{\pgfqpoint{1.066102in}{1.325119in}}%
\pgfpathlineto{\pgfqpoint{1.067193in}{1.325353in}}%
\pgfpathlineto{\pgfqpoint{1.067311in}{1.325113in}}%
\pgfpathlineto{\pgfqpoint{1.067411in}{1.324961in}}%
\pgfpathlineto{\pgfqpoint{1.067547in}{1.325183in}}%
\pgfpathlineto{\pgfqpoint{1.067547in}{1.325183in}}%
\pgfpathlineto{\pgfqpoint{1.067685in}{1.325344in}}%
\pgfpathlineto{\pgfqpoint{1.067778in}{1.325122in}}%
\pgfpathlineto{\pgfqpoint{1.067778in}{1.325122in}}%
\pgfpathlineto{\pgfqpoint{1.067879in}{1.324957in}}%
\pgfpathlineto{\pgfqpoint{1.068015in}{1.325172in}}%
\pgfpathlineto{\pgfqpoint{1.068015in}{1.325172in}}%
\pgfpathlineto{\pgfqpoint{1.068153in}{1.325343in}}%
\pgfpathlineto{\pgfqpoint{1.068247in}{1.325130in}}%
\pgfpathlineto{\pgfqpoint{1.068247in}{1.325130in}}%
\pgfpathlineto{\pgfqpoint{1.068340in}{1.324954in}}%
\pgfpathlineto{\pgfqpoint{1.068475in}{1.325144in}}%
\pgfpathlineto{\pgfqpoint{1.068475in}{1.325144in}}%
\pgfpathlineto{\pgfqpoint{1.068613in}{1.325343in}}%
\pgfpathlineto{\pgfqpoint{1.068724in}{1.325107in}}%
\pgfpathlineto{\pgfqpoint{1.068724in}{1.325107in}}%
\pgfpathlineto{\pgfqpoint{1.068824in}{1.324950in}}%
\pgfpathlineto{\pgfqpoint{1.068960in}{1.325171in}}%
\pgfpathlineto{\pgfqpoint{1.068960in}{1.325171in}}%
\pgfpathlineto{\pgfqpoint{1.069098in}{1.325336in}}%
\pgfpathlineto{\pgfqpoint{1.069199in}{1.325093in}}%
\pgfpathlineto{\pgfqpoint{1.069199in}{1.325093in}}%
\pgfpathlineto{\pgfqpoint{1.069299in}{1.324948in}}%
\pgfpathlineto{\pgfqpoint{1.069436in}{1.325175in}}%
\pgfpathlineto{\pgfqpoint{1.069436in}{1.325175in}}%
\pgfpathlineto{\pgfqpoint{1.069573in}{1.325330in}}%
\pgfpathlineto{\pgfqpoint{1.069670in}{1.325093in}}%
\pgfpathlineto{\pgfqpoint{1.069670in}{1.325093in}}%
\pgfpathlineto{\pgfqpoint{1.069770in}{1.324944in}}%
\pgfpathlineto{\pgfqpoint{1.069907in}{1.325169in}}%
\pgfpathlineto{\pgfqpoint{1.069907in}{1.325169in}}%
\pgfpathlineto{\pgfqpoint{1.070044in}{1.325328in}}%
\pgfpathlineto{\pgfqpoint{1.070135in}{1.325115in}}%
\pgfpathlineto{\pgfqpoint{1.070135in}{1.325115in}}%
\pgfpathlineto{\pgfqpoint{1.070230in}{1.324939in}}%
\pgfpathlineto{\pgfqpoint{1.070366in}{1.325137in}}%
\pgfpathlineto{\pgfqpoint{1.070366in}{1.325137in}}%
\pgfpathlineto{\pgfqpoint{1.070504in}{1.325330in}}%
\pgfpathlineto{\pgfqpoint{1.070611in}{1.325097in}}%
\pgfpathlineto{\pgfqpoint{1.070611in}{1.325097in}}%
\pgfpathlineto{\pgfqpoint{1.070711in}{1.324936in}}%
\pgfpathlineto{\pgfqpoint{1.070848in}{1.325155in}}%
\pgfpathlineto{\pgfqpoint{1.070848in}{1.325155in}}%
\pgfpathlineto{\pgfqpoint{1.070986in}{1.325323in}}%
\pgfpathlineto{\pgfqpoint{1.071083in}{1.325094in}}%
\pgfpathlineto{\pgfqpoint{1.071083in}{1.325094in}}%
\pgfpathlineto{\pgfqpoint{1.071184in}{1.324933in}}%
\pgfpathlineto{\pgfqpoint{1.071321in}{1.325152in}}%
\pgfpathlineto{\pgfqpoint{1.071321in}{1.325152in}}%
\pgfpathlineto{\pgfqpoint{1.071459in}{1.325320in}}%
\pgfpathlineto{\pgfqpoint{1.071552in}{1.325106in}}%
\pgfpathlineto{\pgfqpoint{1.071552in}{1.325106in}}%
\pgfpathlineto{\pgfqpoint{1.071645in}{1.324929in}}%
\pgfpathlineto{\pgfqpoint{1.071781in}{1.325121in}}%
\pgfpathlineto{\pgfqpoint{1.071781in}{1.325121in}}%
\pgfpathlineto{\pgfqpoint{1.071919in}{1.325321in}}%
\pgfpathlineto{\pgfqpoint{1.072033in}{1.325075in}}%
\pgfpathlineto{\pgfqpoint{1.072033in}{1.325075in}}%
\pgfpathlineto{\pgfqpoint{1.072133in}{1.324927in}}%
\pgfpathlineto{\pgfqpoint{1.072270in}{1.325154in}}%
\pgfpathlineto{\pgfqpoint{1.072270in}{1.325154in}}%
\pgfpathlineto{\pgfqpoint{1.072408in}{1.325312in}}%
\pgfpathlineto{\pgfqpoint{1.072507in}{1.325067in}}%
\pgfpathlineto{\pgfqpoint{1.072507in}{1.325067in}}%
\pgfpathlineto{\pgfqpoint{1.072608in}{1.324924in}}%
\pgfpathlineto{\pgfqpoint{1.072745in}{1.325153in}}%
\pgfpathlineto{\pgfqpoint{1.072745in}{1.325153in}}%
\pgfpathlineto{\pgfqpoint{1.072882in}{1.325308in}}%
\pgfpathlineto{\pgfqpoint{1.072981in}{1.325062in}}%
\pgfpathlineto{\pgfqpoint{1.072981in}{1.325062in}}%
\pgfpathlineto{\pgfqpoint{1.073082in}{1.324921in}}%
\pgfpathlineto{\pgfqpoint{1.073184in}{1.325073in}}%
\pgfpathlineto{\pgfqpoint{1.073184in}{1.325073in}}%
\pgfpathlineto{\pgfqpoint{1.074305in}{1.325297in}}%
\pgfpathlineto{\pgfqpoint{1.074389in}{1.325097in}}%
\pgfpathlineto{\pgfqpoint{1.074503in}{1.324910in}}%
\pgfpathlineto{\pgfqpoint{1.074640in}{1.325141in}}%
\pgfpathlineto{\pgfqpoint{1.074640in}{1.325141in}}%
\pgfpathlineto{\pgfqpoint{1.074778in}{1.325295in}}%
\pgfpathlineto{\pgfqpoint{1.074866in}{1.325084in}}%
\pgfpathlineto{\pgfqpoint{1.074866in}{1.325084in}}%
\pgfpathlineto{\pgfqpoint{1.074992in}{1.324915in}}%
\pgfpathlineto{\pgfqpoint{1.075095in}{1.325094in}}%
\pgfpathlineto{\pgfqpoint{1.075095in}{1.325094in}}%
\pgfpathlineto{\pgfqpoint{1.075233in}{1.325299in}}%
\pgfpathlineto{\pgfqpoint{1.075352in}{1.325041in}}%
\pgfpathlineto{\pgfqpoint{1.075352in}{1.325041in}}%
\pgfpathlineto{\pgfqpoint{1.075453in}{1.324904in}}%
\pgfpathlineto{\pgfqpoint{1.075555in}{1.325060in}}%
\pgfpathlineto{\pgfqpoint{1.075555in}{1.325060in}}%
\pgfpathlineto{\pgfqpoint{1.075728in}{1.325287in}}%
\pgfpathlineto{\pgfqpoint{1.075855in}{1.324960in}}%
\pgfpathlineto{\pgfqpoint{1.075855in}{1.324960in}}%
\pgfpathlineto{\pgfqpoint{1.075922in}{1.324899in}}%
\pgfpathlineto{\pgfqpoint{1.076024in}{1.325045in}}%
\pgfpathlineto{\pgfqpoint{1.076024in}{1.325045in}}%
\pgfpathlineto{\pgfqpoint{1.077148in}{1.325281in}}%
\pgfpathlineto{\pgfqpoint{1.077237in}{1.325077in}}%
\pgfpathlineto{\pgfqpoint{1.077351in}{1.324889in}}%
\pgfpathlineto{\pgfqpoint{1.077489in}{1.325122in}}%
\pgfpathlineto{\pgfqpoint{1.077489in}{1.325122in}}%
\pgfpathlineto{\pgfqpoint{1.077627in}{1.325275in}}%
\pgfpathlineto{\pgfqpoint{1.077726in}{1.325024in}}%
\pgfpathlineto{\pgfqpoint{1.077726in}{1.325024in}}%
\pgfpathlineto{\pgfqpoint{1.077827in}{1.324886in}}%
\pgfpathlineto{\pgfqpoint{1.077930in}{1.325042in}}%
\pgfpathlineto{\pgfqpoint{1.077930in}{1.325042in}}%
\pgfpathlineto{\pgfqpoint{1.078103in}{1.325271in}}%
\pgfpathlineto{\pgfqpoint{1.078231in}{1.324941in}}%
\pgfpathlineto{\pgfqpoint{1.078231in}{1.324941in}}%
\pgfpathlineto{\pgfqpoint{1.078298in}{1.324881in}}%
\pgfpathlineto{\pgfqpoint{1.078401in}{1.325029in}}%
\pgfpathlineto{\pgfqpoint{1.078401in}{1.325029in}}%
\pgfpathlineto{\pgfqpoint{1.079533in}{1.325260in}}%
\pgfpathlineto{\pgfqpoint{1.079615in}{1.325061in}}%
\pgfpathlineto{\pgfqpoint{1.079728in}{1.324871in}}%
\pgfpathlineto{\pgfqpoint{1.079866in}{1.325102in}}%
\pgfpathlineto{\pgfqpoint{1.079866in}{1.325102in}}%
\pgfpathlineto{\pgfqpoint{1.080005in}{1.325260in}}%
\pgfpathlineto{\pgfqpoint{1.080096in}{1.325042in}}%
\pgfpathlineto{\pgfqpoint{1.080096in}{1.325042in}}%
\pgfpathlineto{\pgfqpoint{1.080195in}{1.324866in}}%
\pgfpathlineto{\pgfqpoint{1.080332in}{1.325076in}}%
\pgfpathlineto{\pgfqpoint{1.080332in}{1.325076in}}%
\pgfpathlineto{\pgfqpoint{1.080472in}{1.325262in}}%
\pgfpathlineto{\pgfqpoint{1.080581in}{1.325007in}}%
\pgfpathlineto{\pgfqpoint{1.080581in}{1.325007in}}%
\pgfpathlineto{\pgfqpoint{1.080683in}{1.324865in}}%
\pgfpathlineto{\pgfqpoint{1.080786in}{1.325020in}}%
\pgfpathlineto{\pgfqpoint{1.080786in}{1.325020in}}%
\pgfpathlineto{\pgfqpoint{1.081914in}{1.325246in}}%
\pgfpathlineto{\pgfqpoint{1.081999in}{1.325045in}}%
\pgfpathlineto{\pgfqpoint{1.082111in}{1.324854in}}%
\pgfpathlineto{\pgfqpoint{1.082249in}{1.325083in}}%
\pgfpathlineto{\pgfqpoint{1.082249in}{1.325083in}}%
\pgfpathlineto{\pgfqpoint{1.082388in}{1.325244in}}%
\pgfpathlineto{\pgfqpoint{1.082485in}{1.325008in}}%
\pgfpathlineto{\pgfqpoint{1.082485in}{1.325008in}}%
\pgfpathlineto{\pgfqpoint{1.082587in}{1.324850in}}%
\pgfpathlineto{\pgfqpoint{1.082725in}{1.325076in}}%
\pgfpathlineto{\pgfqpoint{1.082725in}{1.325076in}}%
\pgfpathlineto{\pgfqpoint{1.082864in}{1.325242in}}%
\pgfpathlineto{\pgfqpoint{1.082974in}{1.324968in}}%
\pgfpathlineto{\pgfqpoint{1.082974in}{1.324968in}}%
\pgfpathlineto{\pgfqpoint{1.083075in}{1.324851in}}%
\pgfpathlineto{\pgfqpoint{1.083179in}{1.325018in}}%
\pgfpathlineto{\pgfqpoint{1.083179in}{1.325018in}}%
\pgfpathlineto{\pgfqpoint{1.083353in}{1.325230in}}%
\pgfpathlineto{\pgfqpoint{1.083480in}{1.324893in}}%
\pgfpathlineto{\pgfqpoint{1.083480in}{1.324893in}}%
\pgfpathlineto{\pgfqpoint{1.083548in}{1.324845in}}%
\pgfpathlineto{\pgfqpoint{1.083651in}{1.325003in}}%
\pgfpathlineto{\pgfqpoint{1.083651in}{1.325003in}}%
\pgfpathlineto{\pgfqpoint{1.083826in}{1.325232in}}%
\pgfpathlineto{\pgfqpoint{1.083958in}{1.324890in}}%
\pgfpathlineto{\pgfqpoint{1.083958in}{1.324890in}}%
\pgfpathlineto{\pgfqpoint{1.084026in}{1.324841in}}%
\pgfpathlineto{\pgfqpoint{1.084129in}{1.324999in}}%
\pgfpathlineto{\pgfqpoint{1.084129in}{1.324999in}}%
\pgfpathlineto{\pgfqpoint{1.084303in}{1.325229in}}%
\pgfpathlineto{\pgfqpoint{1.084435in}{1.324887in}}%
\pgfpathlineto{\pgfqpoint{1.084435in}{1.324887in}}%
\pgfpathlineto{\pgfqpoint{1.084504in}{1.324837in}}%
\pgfpathlineto{\pgfqpoint{1.084607in}{1.324995in}}%
\pgfpathlineto{\pgfqpoint{1.084607in}{1.324995in}}%
\pgfpathlineto{\pgfqpoint{1.084781in}{1.325226in}}%
\pgfpathlineto{\pgfqpoint{1.084914in}{1.324883in}}%
\pgfpathlineto{\pgfqpoint{1.084914in}{1.324883in}}%
\pgfpathlineto{\pgfqpoint{1.084982in}{1.324834in}}%
\pgfpathlineto{\pgfqpoint{1.085085in}{1.324992in}}%
\pgfpathlineto{\pgfqpoint{1.085085in}{1.324992in}}%
\pgfpathlineto{\pgfqpoint{1.085260in}{1.325223in}}%
\pgfpathlineto{\pgfqpoint{1.085387in}{1.324892in}}%
\pgfpathlineto{\pgfqpoint{1.085387in}{1.324892in}}%
\pgfpathlineto{\pgfqpoint{1.085455in}{1.324828in}}%
\pgfpathlineto{\pgfqpoint{1.085558in}{1.324975in}}%
\pgfpathlineto{\pgfqpoint{1.085558in}{1.324975in}}%
\pgfpathlineto{\pgfqpoint{1.086676in}{1.325221in}}%
\pgfpathlineto{\pgfqpoint{1.086781in}{1.325011in}}%
\pgfpathlineto{\pgfqpoint{1.086893in}{1.324818in}}%
\pgfpathlineto{\pgfqpoint{1.087032in}{1.325047in}}%
\pgfpathlineto{\pgfqpoint{1.087032in}{1.325047in}}%
\pgfpathlineto{\pgfqpoint{1.087172in}{1.325212in}}%
\pgfpathlineto{\pgfqpoint{1.087275in}{1.324957in}}%
\pgfpathlineto{\pgfqpoint{1.087275in}{1.324957in}}%
\pgfpathlineto{\pgfqpoint{1.087377in}{1.324816in}}%
\pgfpathlineto{\pgfqpoint{1.087480in}{1.324974in}}%
\pgfpathlineto{\pgfqpoint{1.087480in}{1.324974in}}%
\pgfpathlineto{\pgfqpoint{1.087655in}{1.325207in}}%
\pgfpathlineto{\pgfqpoint{1.087780in}{1.324881in}}%
\pgfpathlineto{\pgfqpoint{1.087780in}{1.324881in}}%
\pgfpathlineto{\pgfqpoint{1.087848in}{1.324810in}}%
\pgfpathlineto{\pgfqpoint{1.087952in}{1.324952in}}%
\pgfpathlineto{\pgfqpoint{1.087952in}{1.324952in}}%
\pgfpathlineto{\pgfqpoint{1.089074in}{1.325205in}}%
\pgfpathlineto{\pgfqpoint{1.089181in}{1.324991in}}%
\pgfpathlineto{\pgfqpoint{1.089298in}{1.324802in}}%
\pgfpathlineto{\pgfqpoint{1.089437in}{1.325043in}}%
\pgfpathlineto{\pgfqpoint{1.089437in}{1.325043in}}%
\pgfpathlineto{\pgfqpoint{1.089577in}{1.325193in}}%
\pgfpathlineto{\pgfqpoint{1.089669in}{1.324957in}}%
\pgfpathlineto{\pgfqpoint{1.089669in}{1.324957in}}%
\pgfpathlineto{\pgfqpoint{1.089772in}{1.324797in}}%
\pgfpathlineto{\pgfqpoint{1.089911in}{1.325025in}}%
\pgfpathlineto{\pgfqpoint{1.089911in}{1.325025in}}%
\pgfpathlineto{\pgfqpoint{1.090051in}{1.325193in}}%
\pgfpathlineto{\pgfqpoint{1.090151in}{1.324951in}}%
\pgfpathlineto{\pgfqpoint{1.090151in}{1.324951in}}%
\pgfpathlineto{\pgfqpoint{1.090253in}{1.324793in}}%
\pgfpathlineto{\pgfqpoint{1.090392in}{1.325023in}}%
\pgfpathlineto{\pgfqpoint{1.090392in}{1.325023in}}%
\pgfpathlineto{\pgfqpoint{1.090533in}{1.325190in}}%
\pgfpathlineto{\pgfqpoint{1.090634in}{1.324941in}}%
\pgfpathlineto{\pgfqpoint{1.090634in}{1.324941in}}%
\pgfpathlineto{\pgfqpoint{1.090736in}{1.324790in}}%
\pgfpathlineto{\pgfqpoint{1.090875in}{1.325025in}}%
\pgfpathlineto{\pgfqpoint{1.090875in}{1.325025in}}%
\pgfpathlineto{\pgfqpoint{1.091016in}{1.325185in}}%
\pgfpathlineto{\pgfqpoint{1.091112in}{1.324946in}}%
\pgfpathlineto{\pgfqpoint{1.091112in}{1.324946in}}%
\pgfpathlineto{\pgfqpoint{1.091215in}{1.324786in}}%
\pgfpathlineto{\pgfqpoint{1.091354in}{1.325015in}}%
\pgfpathlineto{\pgfqpoint{1.091354in}{1.325015in}}%
\pgfpathlineto{\pgfqpoint{1.091494in}{1.325184in}}%
\pgfpathlineto{\pgfqpoint{1.091593in}{1.324945in}}%
\pgfpathlineto{\pgfqpoint{1.091593in}{1.324945in}}%
\pgfpathlineto{\pgfqpoint{1.091695in}{1.324782in}}%
\pgfpathlineto{\pgfqpoint{1.091834in}{1.325010in}}%
\pgfpathlineto{\pgfqpoint{1.091834in}{1.325010in}}%
\pgfpathlineto{\pgfqpoint{1.091975in}{1.325181in}}%
\pgfpathlineto{\pgfqpoint{1.092078in}{1.324929in}}%
\pgfpathlineto{\pgfqpoint{1.092078in}{1.324929in}}%
\pgfpathlineto{\pgfqpoint{1.092180in}{1.324780in}}%
\pgfpathlineto{\pgfqpoint{1.092320in}{1.325015in}}%
\pgfpathlineto{\pgfqpoint{1.092320in}{1.325015in}}%
\pgfpathlineto{\pgfqpoint{1.092460in}{1.325175in}}%
\pgfpathlineto{\pgfqpoint{1.092563in}{1.324913in}}%
\pgfpathlineto{\pgfqpoint{1.092563in}{1.324913in}}%
\pgfpathlineto{\pgfqpoint{1.092666in}{1.324778in}}%
\pgfpathlineto{\pgfqpoint{1.092770in}{1.324940in}}%
\pgfpathlineto{\pgfqpoint{1.092770in}{1.324940in}}%
\pgfpathlineto{\pgfqpoint{1.092946in}{1.325169in}}%
\pgfpathlineto{\pgfqpoint{1.093076in}{1.324828in}}%
\pgfpathlineto{\pgfqpoint{1.093076in}{1.324828in}}%
\pgfpathlineto{\pgfqpoint{1.093144in}{1.324773in}}%
\pgfpathlineto{\pgfqpoint{1.093248in}{1.324929in}}%
\pgfpathlineto{\pgfqpoint{1.093248in}{1.324929in}}%
\pgfpathlineto{\pgfqpoint{1.093424in}{1.325168in}}%
\pgfpathlineto{\pgfqpoint{1.093575in}{1.324795in}}%
\pgfpathlineto{\pgfqpoint{1.093575in}{1.324795in}}%
\pgfpathlineto{\pgfqpoint{1.093678in}{1.324822in}}%
\pgfpathlineto{\pgfqpoint{1.093784in}{1.325046in}}%
\pgfpathlineto{\pgfqpoint{1.093784in}{1.325046in}}%
\pgfpathlineto{\pgfqpoint{1.093889in}{1.325172in}}%
\pgfpathlineto{\pgfqpoint{1.093998in}{1.324943in}}%
\pgfpathlineto{\pgfqpoint{1.093998in}{1.324943in}}%
\pgfpathlineto{\pgfqpoint{1.094099in}{1.324764in}}%
\pgfpathlineto{\pgfqpoint{1.094239in}{1.324980in}}%
\pgfpathlineto{\pgfqpoint{1.094239in}{1.324980in}}%
\pgfpathlineto{\pgfqpoint{1.094380in}{1.325167in}}%
\pgfpathlineto{\pgfqpoint{1.094481in}{1.324935in}}%
\pgfpathlineto{\pgfqpoint{1.094481in}{1.324935in}}%
\pgfpathlineto{\pgfqpoint{1.094584in}{1.324760in}}%
\pgfpathlineto{\pgfqpoint{1.094724in}{1.324983in}}%
\pgfpathlineto{\pgfqpoint{1.094724in}{1.324983in}}%
\pgfpathlineto{\pgfqpoint{1.094865in}{1.325162in}}%
\pgfpathlineto{\pgfqpoint{1.094972in}{1.324903in}}%
\pgfpathlineto{\pgfqpoint{1.094972in}{1.324903in}}%
\pgfpathlineto{\pgfqpoint{1.095075in}{1.324759in}}%
\pgfpathlineto{\pgfqpoint{1.095180in}{1.324918in}}%
\pgfpathlineto{\pgfqpoint{1.095180in}{1.324918in}}%
\pgfpathlineto{\pgfqpoint{1.095356in}{1.325154in}}%
\pgfpathlineto{\pgfqpoint{1.095489in}{1.324809in}}%
\pgfpathlineto{\pgfqpoint{1.095489in}{1.324809in}}%
\pgfpathlineto{\pgfqpoint{1.095558in}{1.324755in}}%
\pgfpathlineto{\pgfqpoint{1.095662in}{1.324913in}}%
\pgfpathlineto{\pgfqpoint{1.095662in}{1.324913in}}%
\pgfpathlineto{\pgfqpoint{1.095838in}{1.325151in}}%
\pgfpathlineto{\pgfqpoint{1.095964in}{1.324822in}}%
\pgfpathlineto{\pgfqpoint{1.095964in}{1.324822in}}%
\pgfpathlineto{\pgfqpoint{1.096033in}{1.324750in}}%
\pgfpathlineto{\pgfqpoint{1.096137in}{1.324892in}}%
\pgfpathlineto{\pgfqpoint{1.096137in}{1.324892in}}%
\pgfpathlineto{\pgfqpoint{1.097292in}{1.325139in}}%
\pgfpathlineto{\pgfqpoint{1.097373in}{1.324943in}}%
\pgfpathlineto{\pgfqpoint{1.097481in}{1.324739in}}%
\pgfpathlineto{\pgfqpoint{1.097620in}{1.324954in}}%
\pgfpathlineto{\pgfqpoint{1.097620in}{1.324954in}}%
\pgfpathlineto{\pgfqpoint{1.097762in}{1.325144in}}%
\pgfpathlineto{\pgfqpoint{1.097871in}{1.324886in}}%
\pgfpathlineto{\pgfqpoint{1.097871in}{1.324886in}}%
\pgfpathlineto{\pgfqpoint{1.097975in}{1.324737in}}%
\pgfpathlineto{\pgfqpoint{1.098115in}{1.324975in}}%
\pgfpathlineto{\pgfqpoint{1.098115in}{1.324975in}}%
\pgfpathlineto{\pgfqpoint{1.098256in}{1.325135in}}%
\pgfpathlineto{\pgfqpoint{1.098350in}{1.324903in}}%
\pgfpathlineto{\pgfqpoint{1.098350in}{1.324903in}}%
\pgfpathlineto{\pgfqpoint{1.098453in}{1.324732in}}%
\pgfpathlineto{\pgfqpoint{1.098593in}{1.324958in}}%
\pgfpathlineto{\pgfqpoint{1.098593in}{1.324958in}}%
\pgfpathlineto{\pgfqpoint{1.098735in}{1.325135in}}%
\pgfpathlineto{\pgfqpoint{1.098841in}{1.324877in}}%
\pgfpathlineto{\pgfqpoint{1.098841in}{1.324877in}}%
\pgfpathlineto{\pgfqpoint{1.098944in}{1.324730in}}%
\pgfpathlineto{\pgfqpoint{1.099049in}{1.324889in}}%
\pgfpathlineto{\pgfqpoint{1.099049in}{1.324889in}}%
\pgfpathlineto{\pgfqpoint{1.099226in}{1.325128in}}%
\pgfpathlineto{\pgfqpoint{1.099355in}{1.324791in}}%
\pgfpathlineto{\pgfqpoint{1.099355in}{1.324791in}}%
\pgfpathlineto{\pgfqpoint{1.099424in}{1.324725in}}%
\pgfpathlineto{\pgfqpoint{1.099528in}{1.324874in}}%
\pgfpathlineto{\pgfqpoint{1.099528in}{1.324874in}}%
\pgfpathlineto{\pgfqpoint{1.100678in}{1.325120in}}%
\pgfpathlineto{\pgfqpoint{1.100762in}{1.324933in}}%
\pgfpathlineto{\pgfqpoint{1.100878in}{1.324714in}}%
\pgfpathlineto{\pgfqpoint{1.101018in}{1.324942in}}%
\pgfpathlineto{\pgfqpoint{1.101018in}{1.324942in}}%
\pgfpathlineto{\pgfqpoint{1.101160in}{1.325119in}}%
\pgfpathlineto{\pgfqpoint{1.101261in}{1.324877in}}%
\pgfpathlineto{\pgfqpoint{1.101261in}{1.324877in}}%
\pgfpathlineto{\pgfqpoint{1.101364in}{1.324710in}}%
\pgfpathlineto{\pgfqpoint{1.101504in}{1.324941in}}%
\pgfpathlineto{\pgfqpoint{1.101504in}{1.324941in}}%
\pgfpathlineto{\pgfqpoint{1.101647in}{1.325115in}}%
\pgfpathlineto{\pgfqpoint{1.101741in}{1.324891in}}%
\pgfpathlineto{\pgfqpoint{1.101741in}{1.324891in}}%
\pgfpathlineto{\pgfqpoint{1.101838in}{1.324706in}}%
\pgfpathlineto{\pgfqpoint{1.101978in}{1.324909in}}%
\pgfpathlineto{\pgfqpoint{1.101978in}{1.324909in}}%
\pgfpathlineto{\pgfqpoint{1.102120in}{1.325116in}}%
\pgfpathlineto{\pgfqpoint{1.102230in}{1.324876in}}%
\pgfpathlineto{\pgfqpoint{1.102230in}{1.324876in}}%
\pgfpathlineto{\pgfqpoint{1.102334in}{1.324703in}}%
\pgfpathlineto{\pgfqpoint{1.102475in}{1.324930in}}%
\pgfpathlineto{\pgfqpoint{1.102475in}{1.324930in}}%
\pgfpathlineto{\pgfqpoint{1.102617in}{1.325109in}}%
\pgfpathlineto{\pgfqpoint{1.102721in}{1.324856in}}%
\pgfpathlineto{\pgfqpoint{1.102721in}{1.324856in}}%
\pgfpathlineto{\pgfqpoint{1.102825in}{1.324700in}}%
\pgfpathlineto{\pgfqpoint{1.102966in}{1.324938in}}%
\pgfpathlineto{\pgfqpoint{1.102966in}{1.324938in}}%
\pgfpathlineto{\pgfqpoint{1.103108in}{1.325103in}}%
\pgfpathlineto{\pgfqpoint{1.103201in}{1.324874in}}%
\pgfpathlineto{\pgfqpoint{1.103201in}{1.324874in}}%
\pgfpathlineto{\pgfqpoint{1.103305in}{1.324695in}}%
\pgfpathlineto{\pgfqpoint{1.103446in}{1.324920in}}%
\pgfpathlineto{\pgfqpoint{1.103446in}{1.324920in}}%
\pgfpathlineto{\pgfqpoint{1.103588in}{1.325103in}}%
\pgfpathlineto{\pgfqpoint{1.103690in}{1.324862in}}%
\pgfpathlineto{\pgfqpoint{1.103690in}{1.324862in}}%
\pgfpathlineto{\pgfqpoint{1.103794in}{1.324692in}}%
\pgfpathlineto{\pgfqpoint{1.103935in}{1.324922in}}%
\pgfpathlineto{\pgfqpoint{1.103935in}{1.324922in}}%
\pgfpathlineto{\pgfqpoint{1.104077in}{1.325098in}}%
\pgfpathlineto{\pgfqpoint{1.104174in}{1.324868in}}%
\pgfpathlineto{\pgfqpoint{1.104174in}{1.324868in}}%
\pgfpathlineto{\pgfqpoint{1.104278in}{1.324688in}}%
\pgfpathlineto{\pgfqpoint{1.104419in}{1.324913in}}%
\pgfpathlineto{\pgfqpoint{1.104419in}{1.324913in}}%
\pgfpathlineto{\pgfqpoint{1.104561in}{1.325096in}}%
\pgfpathlineto{\pgfqpoint{1.104666in}{1.324848in}}%
\pgfpathlineto{\pgfqpoint{1.104666in}{1.324848in}}%
\pgfpathlineto{\pgfqpoint{1.104769in}{1.324685in}}%
\pgfpathlineto{\pgfqpoint{1.104910in}{1.324919in}}%
\pgfpathlineto{\pgfqpoint{1.104910in}{1.324919in}}%
\pgfpathlineto{\pgfqpoint{1.105053in}{1.325091in}}%
\pgfpathlineto{\pgfqpoint{1.105155in}{1.324836in}}%
\pgfpathlineto{\pgfqpoint{1.105155in}{1.324836in}}%
\pgfpathlineto{\pgfqpoint{1.105259in}{1.324682in}}%
\pgfpathlineto{\pgfqpoint{1.105400in}{1.324922in}}%
\pgfpathlineto{\pgfqpoint{1.105400in}{1.324922in}}%
\pgfpathlineto{\pgfqpoint{1.105542in}{1.325086in}}%
\pgfpathlineto{\pgfqpoint{1.105643in}{1.324832in}}%
\pgfpathlineto{\pgfqpoint{1.105643in}{1.324832in}}%
\pgfpathlineto{\pgfqpoint{1.105747in}{1.324679in}}%
\pgfpathlineto{\pgfqpoint{1.105888in}{1.324919in}}%
\pgfpathlineto{\pgfqpoint{1.105888in}{1.324919in}}%
\pgfpathlineto{\pgfqpoint{1.106030in}{1.325082in}}%
\pgfpathlineto{\pgfqpoint{1.106120in}{1.324864in}}%
\pgfpathlineto{\pgfqpoint{1.106120in}{1.324864in}}%
\pgfpathlineto{\pgfqpoint{1.106246in}{1.324681in}}%
\pgfpathlineto{\pgfqpoint{1.106352in}{1.324861in}}%
\pgfpathlineto{\pgfqpoint{1.106352in}{1.324861in}}%
\pgfpathlineto{\pgfqpoint{1.106495in}{1.325086in}}%
\pgfpathlineto{\pgfqpoint{1.106618in}{1.324823in}}%
\pgfpathlineto{\pgfqpoint{1.106618in}{1.324823in}}%
\pgfpathlineto{\pgfqpoint{1.106722in}{1.324672in}}%
\pgfpathlineto{\pgfqpoint{1.106828in}{1.324831in}}%
\pgfpathlineto{\pgfqpoint{1.106828in}{1.324831in}}%
\pgfpathlineto{\pgfqpoint{1.107006in}{1.325075in}}%
\pgfpathlineto{\pgfqpoint{1.107137in}{1.324734in}}%
\pgfpathlineto{\pgfqpoint{1.107137in}{1.324734in}}%
\pgfpathlineto{\pgfqpoint{1.107206in}{1.324667in}}%
\pgfpathlineto{\pgfqpoint{1.107311in}{1.324817in}}%
\pgfpathlineto{\pgfqpoint{1.107311in}{1.324817in}}%
\pgfpathlineto{\pgfqpoint{1.108457in}{1.325072in}}%
\pgfpathlineto{\pgfqpoint{1.108552in}{1.324885in}}%
\pgfpathlineto{\pgfqpoint{1.108671in}{1.324656in}}%
\pgfpathlineto{\pgfqpoint{1.108813in}{1.324889in}}%
\pgfpathlineto{\pgfqpoint{1.108813in}{1.324889in}}%
\pgfpathlineto{\pgfqpoint{1.108956in}{1.325064in}}%
\pgfpathlineto{\pgfqpoint{1.109060in}{1.324808in}}%
\pgfpathlineto{\pgfqpoint{1.109060in}{1.324808in}}%
\pgfpathlineto{\pgfqpoint{1.109164in}{1.324653in}}%
\pgfpathlineto{\pgfqpoint{1.109306in}{1.324894in}}%
\pgfpathlineto{\pgfqpoint{1.109306in}{1.324894in}}%
\pgfpathlineto{\pgfqpoint{1.109448in}{1.325059in}}%
\pgfpathlineto{\pgfqpoint{1.109552in}{1.324794in}}%
\pgfpathlineto{\pgfqpoint{1.109552in}{1.324794in}}%
\pgfpathlineto{\pgfqpoint{1.109656in}{1.324651in}}%
\pgfpathlineto{\pgfqpoint{1.109762in}{1.324815in}}%
\pgfpathlineto{\pgfqpoint{1.109762in}{1.324815in}}%
\pgfpathlineto{\pgfqpoint{1.109940in}{1.325053in}}%
\pgfpathlineto{\pgfqpoint{1.110065in}{1.324721in}}%
\pgfpathlineto{\pgfqpoint{1.110065in}{1.324721in}}%
\pgfpathlineto{\pgfqpoint{1.110135in}{1.324644in}}%
\pgfpathlineto{\pgfqpoint{1.110240in}{1.324788in}}%
\pgfpathlineto{\pgfqpoint{1.110240in}{1.324788in}}%
\pgfpathlineto{\pgfqpoint{1.111404in}{1.325047in}}%
\pgfpathlineto{\pgfqpoint{1.111485in}{1.324874in}}%
\pgfpathlineto{\pgfqpoint{1.111623in}{1.324641in}}%
\pgfpathlineto{\pgfqpoint{1.111766in}{1.324905in}}%
\pgfpathlineto{\pgfqpoint{1.111766in}{1.324905in}}%
\pgfpathlineto{\pgfqpoint{1.111873in}{1.325049in}}%
\pgfpathlineto{\pgfqpoint{1.111985in}{1.324830in}}%
\pgfpathlineto{\pgfqpoint{1.111985in}{1.324830in}}%
\pgfpathlineto{\pgfqpoint{1.112102in}{1.324631in}}%
\pgfpathlineto{\pgfqpoint{1.112244in}{1.324873in}}%
\pgfpathlineto{\pgfqpoint{1.112244in}{1.324873in}}%
\pgfpathlineto{\pgfqpoint{1.112387in}{1.325039in}}%
\pgfpathlineto{\pgfqpoint{1.112486in}{1.324790in}}%
\pgfpathlineto{\pgfqpoint{1.112486in}{1.324790in}}%
\pgfpathlineto{\pgfqpoint{1.112590in}{1.324627in}}%
\pgfpathlineto{\pgfqpoint{1.112732in}{1.324865in}}%
\pgfpathlineto{\pgfqpoint{1.112732in}{1.324865in}}%
\pgfpathlineto{\pgfqpoint{1.112876in}{1.325036in}}%
\pgfpathlineto{\pgfqpoint{1.112977in}{1.324782in}}%
\pgfpathlineto{\pgfqpoint{1.112977in}{1.324782in}}%
\pgfpathlineto{\pgfqpoint{1.113082in}{1.324624in}}%
\pgfpathlineto{\pgfqpoint{1.113224in}{1.324864in}}%
\pgfpathlineto{\pgfqpoint{1.113224in}{1.324864in}}%
\pgfpathlineto{\pgfqpoint{1.113367in}{1.325032in}}%
\pgfpathlineto{\pgfqpoint{1.113467in}{1.324781in}}%
\pgfpathlineto{\pgfqpoint{1.113467in}{1.324781in}}%
\pgfpathlineto{\pgfqpoint{1.113572in}{1.324620in}}%
\pgfpathlineto{\pgfqpoint{1.113714in}{1.324859in}}%
\pgfpathlineto{\pgfqpoint{1.113714in}{1.324859in}}%
\pgfpathlineto{\pgfqpoint{1.113858in}{1.325029in}}%
\pgfpathlineto{\pgfqpoint{1.113956in}{1.324785in}}%
\pgfpathlineto{\pgfqpoint{1.113956in}{1.324785in}}%
\pgfpathlineto{\pgfqpoint{1.114061in}{1.324615in}}%
\pgfpathlineto{\pgfqpoint{1.114203in}{1.324850in}}%
\pgfpathlineto{\pgfqpoint{1.114203in}{1.324850in}}%
\pgfpathlineto{\pgfqpoint{1.114347in}{1.325027in}}%
\pgfpathlineto{\pgfqpoint{1.114443in}{1.324798in}}%
\pgfpathlineto{\pgfqpoint{1.114443in}{1.324798in}}%
\pgfpathlineto{\pgfqpoint{1.114542in}{1.324611in}}%
\pgfpathlineto{\pgfqpoint{1.114684in}{1.324823in}}%
\pgfpathlineto{\pgfqpoint{1.114684in}{1.324823in}}%
\pgfpathlineto{\pgfqpoint{1.114828in}{1.325028in}}%
\pgfpathlineto{\pgfqpoint{1.114942in}{1.324766in}}%
\pgfpathlineto{\pgfqpoint{1.114942in}{1.324766in}}%
\pgfpathlineto{\pgfqpoint{1.115047in}{1.324609in}}%
\pgfpathlineto{\pgfqpoint{1.115189in}{1.324851in}}%
\pgfpathlineto{\pgfqpoint{1.115189in}{1.324851in}}%
\pgfpathlineto{\pgfqpoint{1.115333in}{1.325018in}}%
\pgfpathlineto{\pgfqpoint{1.115431in}{1.324773in}}%
\pgfpathlineto{\pgfqpoint{1.115431in}{1.324773in}}%
\pgfpathlineto{\pgfqpoint{1.115535in}{1.324604in}}%
\pgfpathlineto{\pgfqpoint{1.115678in}{1.324841in}}%
\pgfpathlineto{\pgfqpoint{1.115678in}{1.324841in}}%
\pgfpathlineto{\pgfqpoint{1.115822in}{1.325017in}}%
\pgfpathlineto{\pgfqpoint{1.115918in}{1.324788in}}%
\pgfpathlineto{\pgfqpoint{1.115918in}{1.324788in}}%
\pgfpathlineto{\pgfqpoint{1.116017in}{1.324600in}}%
\pgfpathlineto{\pgfqpoint{1.116159in}{1.324813in}}%
\pgfpathlineto{\pgfqpoint{1.116159in}{1.324813in}}%
\pgfpathlineto{\pgfqpoint{1.116303in}{1.325018in}}%
\pgfpathlineto{\pgfqpoint{1.116414in}{1.324767in}}%
\pgfpathlineto{\pgfqpoint{1.116414in}{1.324767in}}%
\pgfpathlineto{\pgfqpoint{1.116519in}{1.324597in}}%
\pgfpathlineto{\pgfqpoint{1.116662in}{1.324833in}}%
\pgfpathlineto{\pgfqpoint{1.116662in}{1.324833in}}%
\pgfpathlineto{\pgfqpoint{1.116806in}{1.325010in}}%
\pgfpathlineto{\pgfqpoint{1.116901in}{1.324783in}}%
\pgfpathlineto{\pgfqpoint{1.116901in}{1.324783in}}%
\pgfpathlineto{\pgfqpoint{1.116998in}{1.324593in}}%
\pgfpathlineto{\pgfqpoint{1.117140in}{1.324796in}}%
\pgfpathlineto{\pgfqpoint{1.117140in}{1.324796in}}%
\pgfpathlineto{\pgfqpoint{1.117284in}{1.325012in}}%
\pgfpathlineto{\pgfqpoint{1.117396in}{1.324771in}}%
\pgfpathlineto{\pgfqpoint{1.117396in}{1.324771in}}%
\pgfpathlineto{\pgfqpoint{1.117501in}{1.324589in}}%
\pgfpathlineto{\pgfqpoint{1.117644in}{1.324819in}}%
\pgfpathlineto{\pgfqpoint{1.117644in}{1.324819in}}%
\pgfpathlineto{\pgfqpoint{1.117788in}{1.325005in}}%
\pgfpathlineto{\pgfqpoint{1.117896in}{1.324740in}}%
\pgfpathlineto{\pgfqpoint{1.117896in}{1.324740in}}%
\pgfpathlineto{\pgfqpoint{1.118001in}{1.324587in}}%
\pgfpathlineto{\pgfqpoint{1.118108in}{1.324750in}}%
\pgfpathlineto{\pgfqpoint{1.118108in}{1.324750in}}%
\pgfpathlineto{\pgfqpoint{1.118288in}{1.324997in}}%
\pgfpathlineto{\pgfqpoint{1.118417in}{1.324656in}}%
\pgfpathlineto{\pgfqpoint{1.118417in}{1.324656in}}%
\pgfpathlineto{\pgfqpoint{1.118487in}{1.324582in}}%
\pgfpathlineto{\pgfqpoint{1.118593in}{1.324729in}}%
\pgfpathlineto{\pgfqpoint{1.118593in}{1.324729in}}%
\pgfpathlineto{\pgfqpoint{1.119762in}{1.324991in}}%
\pgfpathlineto{\pgfqpoint{1.119844in}{1.324824in}}%
\pgfpathlineto{\pgfqpoint{1.119967in}{1.324570in}}%
\pgfpathlineto{\pgfqpoint{1.120147in}{1.324883in}}%
\pgfpathlineto{\pgfqpoint{1.120147in}{1.324883in}}%
\pgfpathlineto{\pgfqpoint{1.120254in}{1.324988in}}%
\pgfpathlineto{\pgfqpoint{1.120350in}{1.324770in}}%
\pgfpathlineto{\pgfqpoint{1.120350in}{1.324770in}}%
\pgfpathlineto{\pgfqpoint{1.120467in}{1.324568in}}%
\pgfpathlineto{\pgfqpoint{1.120610in}{1.324812in}}%
\pgfpathlineto{\pgfqpoint{1.120610in}{1.324812in}}%
\pgfpathlineto{\pgfqpoint{1.120754in}{1.324981in}}%
\pgfpathlineto{\pgfqpoint{1.120855in}{1.324728in}}%
\pgfpathlineto{\pgfqpoint{1.120855in}{1.324728in}}%
\pgfpathlineto{\pgfqpoint{1.120960in}{1.324564in}}%
\pgfpathlineto{\pgfqpoint{1.121103in}{1.324806in}}%
\pgfpathlineto{\pgfqpoint{1.121103in}{1.324806in}}%
\pgfpathlineto{\pgfqpoint{1.121248in}{1.324978in}}%
\pgfpathlineto{\pgfqpoint{1.121343in}{1.324745in}}%
\pgfpathlineto{\pgfqpoint{1.121343in}{1.324745in}}%
\pgfpathlineto{\pgfqpoint{1.121448in}{1.324559in}}%
\pgfpathlineto{\pgfqpoint{1.121591in}{1.324787in}}%
\pgfpathlineto{\pgfqpoint{1.121591in}{1.324787in}}%
\pgfpathlineto{\pgfqpoint{1.121736in}{1.324979in}}%
\pgfpathlineto{\pgfqpoint{1.121842in}{1.324724in}}%
\pgfpathlineto{\pgfqpoint{1.121842in}{1.324724in}}%
\pgfpathlineto{\pgfqpoint{1.121948in}{1.324556in}}%
\pgfpathlineto{\pgfqpoint{1.122091in}{1.324797in}}%
\pgfpathlineto{\pgfqpoint{1.122091in}{1.324797in}}%
\pgfpathlineto{\pgfqpoint{1.122236in}{1.324972in}}%
\pgfpathlineto{\pgfqpoint{1.122332in}{1.324739in}}%
\pgfpathlineto{\pgfqpoint{1.122332in}{1.324739in}}%
\pgfpathlineto{\pgfqpoint{1.122435in}{1.324552in}}%
\pgfpathlineto{\pgfqpoint{1.122578in}{1.324774in}}%
\pgfpathlineto{\pgfqpoint{1.122578in}{1.324774in}}%
\pgfpathlineto{\pgfqpoint{1.122723in}{1.324973in}}%
\pgfpathlineto{\pgfqpoint{1.122852in}{1.324654in}}%
\pgfpathlineto{\pgfqpoint{1.122852in}{1.324654in}}%
\pgfpathlineto{\pgfqpoint{1.122957in}{1.324559in}}%
\pgfpathlineto{\pgfqpoint{1.123065in}{1.324752in}}%
\pgfpathlineto{\pgfqpoint{1.123065in}{1.324752in}}%
\pgfpathlineto{\pgfqpoint{1.123210in}{1.324971in}}%
\pgfpathlineto{\pgfqpoint{1.123331in}{1.324699in}}%
\pgfpathlineto{\pgfqpoint{1.123331in}{1.324699in}}%
\pgfpathlineto{\pgfqpoint{1.123437in}{1.324546in}}%
\pgfpathlineto{\pgfqpoint{1.123544in}{1.324711in}}%
\pgfpathlineto{\pgfqpoint{1.123544in}{1.324711in}}%
\pgfpathlineto{\pgfqpoint{1.123725in}{1.324959in}}%
\pgfpathlineto{\pgfqpoint{1.123857in}{1.324608in}}%
\pgfpathlineto{\pgfqpoint{1.123857in}{1.324608in}}%
\pgfpathlineto{\pgfqpoint{1.123928in}{1.324541in}}%
\pgfpathlineto{\pgfqpoint{1.124035in}{1.324698in}}%
\pgfpathlineto{\pgfqpoint{1.124035in}{1.324698in}}%
\pgfpathlineto{\pgfqpoint{1.125189in}{1.324958in}}%
\pgfpathlineto{\pgfqpoint{1.125260in}{1.324870in}}%
\pgfpathlineto{\pgfqpoint{1.126402in}{1.324522in}}%
\pgfpathlineto{\pgfqpoint{1.126474in}{1.324594in}}%
\pgfpathlineto{\pgfqpoint{1.127679in}{1.324940in}}%
\pgfpathlineto{\pgfqpoint{1.127780in}{1.324731in}}%
\pgfpathlineto{\pgfqpoint{1.127888in}{1.324510in}}%
\pgfpathlineto{\pgfqpoint{1.128031in}{1.324726in}}%
\pgfpathlineto{\pgfqpoint{1.128031in}{1.324726in}}%
\pgfpathlineto{\pgfqpoint{1.128177in}{1.324936in}}%
\pgfpathlineto{\pgfqpoint{1.128285in}{1.324697in}}%
\pgfpathlineto{\pgfqpoint{1.128285in}{1.324697in}}%
\pgfpathlineto{\pgfqpoint{1.128387in}{1.324507in}}%
\pgfpathlineto{\pgfqpoint{1.128530in}{1.324726in}}%
\pgfpathlineto{\pgfqpoint{1.128530in}{1.324726in}}%
\pgfpathlineto{\pgfqpoint{1.128676in}{1.324932in}}%
\pgfpathlineto{\pgfqpoint{1.128793in}{1.324656in}}%
\pgfpathlineto{\pgfqpoint{1.128793in}{1.324656in}}%
\pgfpathlineto{\pgfqpoint{1.128899in}{1.324505in}}%
\pgfpathlineto{\pgfqpoint{1.129007in}{1.324673in}}%
\pgfpathlineto{\pgfqpoint{1.129007in}{1.324673in}}%
\pgfpathlineto{\pgfqpoint{1.129189in}{1.324921in}}%
\pgfpathlineto{\pgfqpoint{1.129322in}{1.324565in}}%
\pgfpathlineto{\pgfqpoint{1.129322in}{1.324565in}}%
\pgfpathlineto{\pgfqpoint{1.129393in}{1.324500in}}%
\pgfpathlineto{\pgfqpoint{1.129500in}{1.324660in}}%
\pgfpathlineto{\pgfqpoint{1.129500in}{1.324660in}}%
\pgfpathlineto{\pgfqpoint{1.129683in}{1.324920in}}%
\pgfpathlineto{\pgfqpoint{1.129843in}{1.324521in}}%
\pgfpathlineto{\pgfqpoint{1.129843in}{1.324521in}}%
\pgfpathlineto{\pgfqpoint{1.131448in}{1.324551in}}%
\pgfpathlineto{\pgfqpoint{1.131484in}{1.324623in}}%
\pgfpathlineto{\pgfqpoint{1.132685in}{1.324891in}}%
\pgfpathlineto{\pgfqpoint{1.132721in}{1.324831in}}%
\pgfpathlineto{\pgfqpoint{1.133884in}{1.324466in}}%
\pgfpathlineto{\pgfqpoint{1.133956in}{1.324551in}}%
\pgfpathlineto{\pgfqpoint{1.135179in}{1.324879in}}%
\pgfpathlineto{\pgfqpoint{1.135251in}{1.324726in}}%
\pgfpathlineto{\pgfqpoint{1.135388in}{1.324456in}}%
\pgfpathlineto{\pgfqpoint{1.135570in}{1.324790in}}%
\pgfpathlineto{\pgfqpoint{1.135570in}{1.324790in}}%
\pgfpathlineto{\pgfqpoint{1.135679in}{1.324876in}}%
\pgfpathlineto{\pgfqpoint{1.135762in}{1.324685in}}%
\pgfpathlineto{\pgfqpoint{1.135762in}{1.324685in}}%
\pgfpathlineto{\pgfqpoint{1.135896in}{1.324456in}}%
\pgfpathlineto{\pgfqpoint{1.136042in}{1.324724in}}%
\pgfpathlineto{\pgfqpoint{1.136042in}{1.324724in}}%
\pgfpathlineto{\pgfqpoint{1.136188in}{1.324866in}}%
\pgfpathlineto{\pgfqpoint{1.136273in}{1.324640in}}%
\pgfpathlineto{\pgfqpoint{1.136273in}{1.324640in}}%
\pgfpathlineto{\pgfqpoint{1.136374in}{1.324446in}}%
\pgfpathlineto{\pgfqpoint{1.136519in}{1.324664in}}%
\pgfpathlineto{\pgfqpoint{1.136519in}{1.324664in}}%
\pgfpathlineto{\pgfqpoint{1.136666in}{1.324877in}}%
\pgfpathlineto{\pgfqpoint{1.136796in}{1.324560in}}%
\pgfpathlineto{\pgfqpoint{1.136796in}{1.324560in}}%
\pgfpathlineto{\pgfqpoint{1.136903in}{1.324451in}}%
\pgfpathlineto{\pgfqpoint{1.137012in}{1.324642in}}%
\pgfpathlineto{\pgfqpoint{1.137012in}{1.324642in}}%
\pgfpathlineto{\pgfqpoint{1.137159in}{1.324875in}}%
\pgfpathlineto{\pgfqpoint{1.137312in}{1.324519in}}%
\pgfpathlineto{\pgfqpoint{1.137312in}{1.324519in}}%
\pgfpathlineto{\pgfqpoint{1.137383in}{1.324438in}}%
\pgfpathlineto{\pgfqpoint{1.137491in}{1.324587in}}%
\pgfpathlineto{\pgfqpoint{1.137491in}{1.324587in}}%
\pgfpathlineto{\pgfqpoint{1.138685in}{1.324858in}}%
\pgfpathlineto{\pgfqpoint{1.138721in}{1.324810in}}%
\pgfpathlineto{\pgfqpoint{1.139919in}{1.324431in}}%
\pgfpathlineto{\pgfqpoint{1.139991in}{1.324544in}}%
\pgfpathlineto{\pgfqpoint{1.141188in}{1.324845in}}%
\pgfpathlineto{\pgfqpoint{1.141271in}{1.324696in}}%
\pgfpathlineto{\pgfqpoint{1.141405in}{1.324408in}}%
\pgfpathlineto{\pgfqpoint{1.141588in}{1.324732in}}%
\pgfpathlineto{\pgfqpoint{1.141588in}{1.324732in}}%
\pgfpathlineto{\pgfqpoint{1.141698in}{1.324839in}}%
\pgfpathlineto{\pgfqpoint{1.141794in}{1.324619in}}%
\pgfpathlineto{\pgfqpoint{1.141794in}{1.324619in}}%
\pgfpathlineto{\pgfqpoint{1.141902in}{1.324404in}}%
\pgfpathlineto{\pgfqpoint{1.142047in}{1.324629in}}%
\pgfpathlineto{\pgfqpoint{1.142047in}{1.324629in}}%
\pgfpathlineto{\pgfqpoint{1.142195in}{1.324838in}}%
\pgfpathlineto{\pgfqpoint{1.142307in}{1.324579in}}%
\pgfpathlineto{\pgfqpoint{1.142307in}{1.324579in}}%
\pgfpathlineto{\pgfqpoint{1.142414in}{1.324401in}}%
\pgfpathlineto{\pgfqpoint{1.142560in}{1.324645in}}%
\pgfpathlineto{\pgfqpoint{1.142560in}{1.324645in}}%
\pgfpathlineto{\pgfqpoint{1.142708in}{1.324831in}}%
\pgfpathlineto{\pgfqpoint{1.142806in}{1.324593in}}%
\pgfpathlineto{\pgfqpoint{1.142806in}{1.324593in}}%
\pgfpathlineto{\pgfqpoint{1.142906in}{1.324396in}}%
\pgfpathlineto{\pgfqpoint{1.143052in}{1.324613in}}%
\pgfpathlineto{\pgfqpoint{1.143052in}{1.324613in}}%
\pgfpathlineto{\pgfqpoint{1.143200in}{1.324832in}}%
\pgfpathlineto{\pgfqpoint{1.143334in}{1.324507in}}%
\pgfpathlineto{\pgfqpoint{1.143334in}{1.324507in}}%
\pgfpathlineto{\pgfqpoint{1.143442in}{1.324403in}}%
\pgfpathlineto{\pgfqpoint{1.143551in}{1.324598in}}%
\pgfpathlineto{\pgfqpoint{1.143551in}{1.324598in}}%
\pgfpathlineto{\pgfqpoint{1.143700in}{1.324829in}}%
\pgfpathlineto{\pgfqpoint{1.143826in}{1.324545in}}%
\pgfpathlineto{\pgfqpoint{1.143826in}{1.324545in}}%
\pgfpathlineto{\pgfqpoint{1.143933in}{1.324391in}}%
\pgfpathlineto{\pgfqpoint{1.144042in}{1.324563in}}%
\pgfpathlineto{\pgfqpoint{1.144042in}{1.324563in}}%
\pgfpathlineto{\pgfqpoint{1.144227in}{1.324816in}}%
\pgfpathlineto{\pgfqpoint{1.144362in}{1.324451in}}%
\pgfpathlineto{\pgfqpoint{1.144362in}{1.324451in}}%
\pgfpathlineto{\pgfqpoint{1.144434in}{1.324386in}}%
\pgfpathlineto{\pgfqpoint{1.144543in}{1.324550in}}%
\pgfpathlineto{\pgfqpoint{1.144543in}{1.324550in}}%
\pgfpathlineto{\pgfqpoint{1.144728in}{1.324815in}}%
\pgfpathlineto{\pgfqpoint{1.144872in}{1.324436in}}%
\pgfpathlineto{\pgfqpoint{1.144872in}{1.324436in}}%
\pgfpathlineto{\pgfqpoint{1.144944in}{1.324384in}}%
\pgfpathlineto{\pgfqpoint{1.145053in}{1.324558in}}%
\pgfpathlineto{\pgfqpoint{1.145053in}{1.324558in}}%
\pgfpathlineto{\pgfqpoint{1.145238in}{1.324808in}}%
\pgfpathlineto{\pgfqpoint{1.145383in}{1.324421in}}%
\pgfpathlineto{\pgfqpoint{1.145383in}{1.324421in}}%
\pgfpathlineto{\pgfqpoint{1.145455in}{1.324383in}}%
\pgfpathlineto{\pgfqpoint{1.145565in}{1.324569in}}%
\pgfpathlineto{\pgfqpoint{1.145565in}{1.324569in}}%
\pgfpathlineto{\pgfqpoint{1.145713in}{1.324815in}}%
\pgfpathlineto{\pgfqpoint{1.145876in}{1.324443in}}%
\pgfpathlineto{\pgfqpoint{1.145876in}{1.324443in}}%
\pgfpathlineto{\pgfqpoint{1.145948in}{1.324374in}}%
\pgfpathlineto{\pgfqpoint{1.146057in}{1.324535in}}%
\pgfpathlineto{\pgfqpoint{1.146057in}{1.324535in}}%
\pgfpathlineto{\pgfqpoint{1.147263in}{1.324792in}}%
\pgfpathlineto{\pgfqpoint{1.147299in}{1.324734in}}%
\pgfpathlineto{\pgfqpoint{1.148497in}{1.324364in}}%
\pgfpathlineto{\pgfqpoint{1.148533in}{1.324408in}}%
\pgfpathlineto{\pgfqpoint{1.149775in}{1.324786in}}%
\pgfpathlineto{\pgfqpoint{1.149877in}{1.324577in}}%
\pgfpathlineto{\pgfqpoint{1.150003in}{1.324344in}}%
\pgfpathlineto{\pgfqpoint{1.150150in}{1.324601in}}%
\pgfpathlineto{\pgfqpoint{1.150150in}{1.324601in}}%
\pgfpathlineto{\pgfqpoint{1.150298in}{1.324775in}}%
\pgfpathlineto{\pgfqpoint{1.150401in}{1.324507in}}%
\pgfpathlineto{\pgfqpoint{1.150401in}{1.324507in}}%
\pgfpathlineto{\pgfqpoint{1.150509in}{1.324340in}}%
\pgfpathlineto{\pgfqpoint{1.150656in}{1.324595in}}%
\pgfpathlineto{\pgfqpoint{1.150656in}{1.324595in}}%
\pgfpathlineto{\pgfqpoint{1.150805in}{1.324772in}}%
\pgfpathlineto{\pgfqpoint{1.150910in}{1.324498in}}%
\pgfpathlineto{\pgfqpoint{1.150910in}{1.324498in}}%
\pgfpathlineto{\pgfqpoint{1.151018in}{1.324336in}}%
\pgfpathlineto{\pgfqpoint{1.151128in}{1.324507in}}%
\pgfpathlineto{\pgfqpoint{1.151128in}{1.324507in}}%
\pgfpathlineto{\pgfqpoint{1.151314in}{1.324767in}}%
\pgfpathlineto{\pgfqpoint{1.151454in}{1.324393in}}%
\pgfpathlineto{\pgfqpoint{1.151454in}{1.324393in}}%
\pgfpathlineto{\pgfqpoint{1.151526in}{1.324332in}}%
\pgfpathlineto{\pgfqpoint{1.151636in}{1.324502in}}%
\pgfpathlineto{\pgfqpoint{1.151636in}{1.324502in}}%
\pgfpathlineto{\pgfqpoint{1.151822in}{1.324764in}}%
\pgfpathlineto{\pgfqpoint{1.151956in}{1.324402in}}%
\pgfpathlineto{\pgfqpoint{1.151956in}{1.324402in}}%
\pgfpathlineto{\pgfqpoint{1.152028in}{1.324327in}}%
\pgfpathlineto{\pgfqpoint{1.152138in}{1.324485in}}%
\pgfpathlineto{\pgfqpoint{1.152138in}{1.324485in}}%
\pgfpathlineto{\pgfqpoint{1.153342in}{1.324757in}}%
\pgfpathlineto{\pgfqpoint{1.153379in}{1.324712in}}%
\pgfpathlineto{\pgfqpoint{1.154573in}{1.324307in}}%
\pgfpathlineto{\pgfqpoint{1.154646in}{1.324389in}}%
\pgfpathlineto{\pgfqpoint{1.155893in}{1.324736in}}%
\pgfpathlineto{\pgfqpoint{1.155967in}{1.324586in}}%
\pgfpathlineto{\pgfqpoint{1.156108in}{1.324297in}}%
\pgfpathlineto{\pgfqpoint{1.156294in}{1.324642in}}%
\pgfpathlineto{\pgfqpoint{1.156294in}{1.324642in}}%
\pgfpathlineto{\pgfqpoint{1.156405in}{1.324731in}}%
\pgfpathlineto{\pgfqpoint{1.156490in}{1.324534in}}%
\pgfpathlineto{\pgfqpoint{1.156490in}{1.324534in}}%
\pgfpathlineto{\pgfqpoint{1.156609in}{1.324291in}}%
\pgfpathlineto{\pgfqpoint{1.156757in}{1.324531in}}%
\pgfpathlineto{\pgfqpoint{1.156757in}{1.324531in}}%
\pgfpathlineto{\pgfqpoint{1.156907in}{1.324733in}}%
\pgfpathlineto{\pgfqpoint{1.157012in}{1.324482in}}%
\pgfpathlineto{\pgfqpoint{1.157012in}{1.324482in}}%
\pgfpathlineto{\pgfqpoint{1.157121in}{1.324287in}}%
\pgfpathlineto{\pgfqpoint{1.157269in}{1.324531in}}%
\pgfpathlineto{\pgfqpoint{1.157269in}{1.324531in}}%
\pgfpathlineto{\pgfqpoint{1.157419in}{1.324729in}}%
\pgfpathlineto{\pgfqpoint{1.157526in}{1.324466in}}%
\pgfpathlineto{\pgfqpoint{1.157526in}{1.324466in}}%
\pgfpathlineto{\pgfqpoint{1.157635in}{1.324284in}}%
\pgfpathlineto{\pgfqpoint{1.157783in}{1.324535in}}%
\pgfpathlineto{\pgfqpoint{1.157783in}{1.324535in}}%
\pgfpathlineto{\pgfqpoint{1.157933in}{1.324724in}}%
\pgfpathlineto{\pgfqpoint{1.158033in}{1.324478in}}%
\pgfpathlineto{\pgfqpoint{1.158033in}{1.324478in}}%
\pgfpathlineto{\pgfqpoint{1.158137in}{1.324279in}}%
\pgfpathlineto{\pgfqpoint{1.158285in}{1.324509in}}%
\pgfpathlineto{\pgfqpoint{1.158285in}{1.324509in}}%
\pgfpathlineto{\pgfqpoint{1.158435in}{1.324724in}}%
\pgfpathlineto{\pgfqpoint{1.158547in}{1.324463in}}%
\pgfpathlineto{\pgfqpoint{1.158547in}{1.324463in}}%
\pgfpathlineto{\pgfqpoint{1.158656in}{1.324276in}}%
\pgfpathlineto{\pgfqpoint{1.158804in}{1.324524in}}%
\pgfpathlineto{\pgfqpoint{1.158804in}{1.324524in}}%
\pgfpathlineto{\pgfqpoint{1.158954in}{1.324717in}}%
\pgfpathlineto{\pgfqpoint{1.159063in}{1.324444in}}%
\pgfpathlineto{\pgfqpoint{1.159063in}{1.324444in}}%
\pgfpathlineto{\pgfqpoint{1.159172in}{1.324273in}}%
\pgfpathlineto{\pgfqpoint{1.159320in}{1.324531in}}%
\pgfpathlineto{\pgfqpoint{1.159320in}{1.324531in}}%
\pgfpathlineto{\pgfqpoint{1.159470in}{1.324711in}}%
\pgfpathlineto{\pgfqpoint{1.159573in}{1.324446in}}%
\pgfpathlineto{\pgfqpoint{1.159573in}{1.324446in}}%
\pgfpathlineto{\pgfqpoint{1.159682in}{1.324268in}}%
\pgfpathlineto{\pgfqpoint{1.159830in}{1.324523in}}%
\pgfpathlineto{\pgfqpoint{1.159830in}{1.324523in}}%
\pgfpathlineto{\pgfqpoint{1.159980in}{1.324708in}}%
\pgfpathlineto{\pgfqpoint{1.160080in}{1.324458in}}%
\pgfpathlineto{\pgfqpoint{1.160080in}{1.324458in}}%
\pgfpathlineto{\pgfqpoint{1.160190in}{1.324263in}}%
\pgfpathlineto{\pgfqpoint{1.160338in}{1.324509in}}%
\pgfpathlineto{\pgfqpoint{1.160338in}{1.324509in}}%
\pgfpathlineto{\pgfqpoint{1.160488in}{1.324707in}}%
\pgfpathlineto{\pgfqpoint{1.160595in}{1.324444in}}%
\pgfpathlineto{\pgfqpoint{1.160595in}{1.324444in}}%
\pgfpathlineto{\pgfqpoint{1.160704in}{1.324260in}}%
\pgfpathlineto{\pgfqpoint{1.160853in}{1.324511in}}%
\pgfpathlineto{\pgfqpoint{1.160853in}{1.324511in}}%
\pgfpathlineto{\pgfqpoint{1.161003in}{1.324702in}}%
\pgfpathlineto{\pgfqpoint{1.161102in}{1.324463in}}%
\pgfpathlineto{\pgfqpoint{1.161102in}{1.324463in}}%
\pgfpathlineto{\pgfqpoint{1.161233in}{1.324263in}}%
\pgfpathlineto{\pgfqpoint{1.161344in}{1.324458in}}%
\pgfpathlineto{\pgfqpoint{1.161344in}{1.324458in}}%
\pgfpathlineto{\pgfqpoint{1.161495in}{1.324704in}}%
\pgfpathlineto{\pgfqpoint{1.161622in}{1.324431in}}%
\pgfpathlineto{\pgfqpoint{1.161622in}{1.324431in}}%
\pgfpathlineto{\pgfqpoint{1.161731in}{1.324252in}}%
\pgfpathlineto{\pgfqpoint{1.161880in}{1.324508in}}%
\pgfpathlineto{\pgfqpoint{1.161880in}{1.324508in}}%
\pgfpathlineto{\pgfqpoint{1.162030in}{1.324694in}}%
\pgfpathlineto{\pgfqpoint{1.162132in}{1.324436in}}%
\pgfpathlineto{\pgfqpoint{1.162132in}{1.324436in}}%
\pgfpathlineto{\pgfqpoint{1.162241in}{1.324248in}}%
\pgfpathlineto{\pgfqpoint{1.162390in}{1.324498in}}%
\pgfpathlineto{\pgfqpoint{1.162390in}{1.324498in}}%
\pgfpathlineto{\pgfqpoint{1.162541in}{1.324692in}}%
\pgfpathlineto{\pgfqpoint{1.162644in}{1.324436in}}%
\pgfpathlineto{\pgfqpoint{1.162644in}{1.324436in}}%
\pgfpathlineto{\pgfqpoint{1.162754in}{1.324244in}}%
\pgfpathlineto{\pgfqpoint{1.162902in}{1.324492in}}%
\pgfpathlineto{\pgfqpoint{1.162902in}{1.324492in}}%
\pgfpathlineto{\pgfqpoint{1.163053in}{1.324689in}}%
\pgfpathlineto{\pgfqpoint{1.163157in}{1.324434in}}%
\pgfpathlineto{\pgfqpoint{1.163157in}{1.324434in}}%
\pgfpathlineto{\pgfqpoint{1.163267in}{1.324240in}}%
\pgfpathlineto{\pgfqpoint{1.163415in}{1.324487in}}%
\pgfpathlineto{\pgfqpoint{1.163415in}{1.324487in}}%
\pgfpathlineto{\pgfqpoint{1.163566in}{1.324685in}}%
\pgfpathlineto{\pgfqpoint{1.163671in}{1.324428in}}%
\pgfpathlineto{\pgfqpoint{1.163671in}{1.324428in}}%
\pgfpathlineto{\pgfqpoint{1.163781in}{1.324236in}}%
\pgfpathlineto{\pgfqpoint{1.163929in}{1.324484in}}%
\pgfpathlineto{\pgfqpoint{1.163929in}{1.324484in}}%
\pgfpathlineto{\pgfqpoint{1.164080in}{1.324681in}}%
\pgfpathlineto{\pgfqpoint{1.164193in}{1.324393in}}%
\pgfpathlineto{\pgfqpoint{1.164193in}{1.324393in}}%
\pgfpathlineto{\pgfqpoint{1.164303in}{1.324234in}}%
\pgfpathlineto{\pgfqpoint{1.164414in}{1.324411in}}%
\pgfpathlineto{\pgfqpoint{1.164414in}{1.324411in}}%
\pgfpathlineto{\pgfqpoint{1.164602in}{1.324672in}}%
\pgfpathlineto{\pgfqpoint{1.164743in}{1.324290in}}%
\pgfpathlineto{\pgfqpoint{1.164743in}{1.324290in}}%
\pgfpathlineto{\pgfqpoint{1.164816in}{1.324230in}}%
\pgfpathlineto{\pgfqpoint{1.164927in}{1.324405in}}%
\pgfpathlineto{\pgfqpoint{1.164927in}{1.324405in}}%
\pgfpathlineto{\pgfqpoint{1.165116in}{1.324669in}}%
\pgfpathlineto{\pgfqpoint{1.165255in}{1.324290in}}%
\pgfpathlineto{\pgfqpoint{1.165255in}{1.324290in}}%
\pgfpathlineto{\pgfqpoint{1.165329in}{1.324225in}}%
\pgfpathlineto{\pgfqpoint{1.165440in}{1.324397in}}%
\pgfpathlineto{\pgfqpoint{1.165440in}{1.324397in}}%
\pgfpathlineto{\pgfqpoint{1.165629in}{1.324667in}}%
\pgfpathlineto{\pgfqpoint{1.165771in}{1.324285in}}%
\pgfpathlineto{\pgfqpoint{1.165771in}{1.324285in}}%
\pgfpathlineto{\pgfqpoint{1.165844in}{1.324221in}}%
\pgfpathlineto{\pgfqpoint{1.165955in}{1.324395in}}%
\pgfpathlineto{\pgfqpoint{1.165955in}{1.324395in}}%
\pgfpathlineto{\pgfqpoint{1.166144in}{1.324663in}}%
\pgfpathlineto{\pgfqpoint{1.166284in}{1.324284in}}%
\pgfpathlineto{\pgfqpoint{1.166284in}{1.324284in}}%
\pgfpathlineto{\pgfqpoint{1.166357in}{1.324217in}}%
\pgfpathlineto{\pgfqpoint{1.166469in}{1.324387in}}%
\pgfpathlineto{\pgfqpoint{1.166469in}{1.324387in}}%
\pgfpathlineto{\pgfqpoint{1.166657in}{1.324660in}}%
\pgfpathlineto{\pgfqpoint{1.166800in}{1.324278in}}%
\pgfpathlineto{\pgfqpoint{1.166800in}{1.324278in}}%
\pgfpathlineto{\pgfqpoint{1.166873in}{1.324213in}}%
\pgfpathlineto{\pgfqpoint{1.166985in}{1.324385in}}%
\pgfpathlineto{\pgfqpoint{1.166985in}{1.324385in}}%
\pgfpathlineto{\pgfqpoint{1.167174in}{1.324656in}}%
\pgfpathlineto{\pgfqpoint{1.167313in}{1.324278in}}%
\pgfpathlineto{\pgfqpoint{1.167313in}{1.324278in}}%
\pgfpathlineto{\pgfqpoint{1.167387in}{1.324209in}}%
\pgfpathlineto{\pgfqpoint{1.167498in}{1.324377in}}%
\pgfpathlineto{\pgfqpoint{1.167498in}{1.324377in}}%
\pgfpathlineto{\pgfqpoint{1.167687in}{1.324654in}}%
\pgfpathlineto{\pgfqpoint{1.167820in}{1.324295in}}%
\pgfpathlineto{\pgfqpoint{1.167820in}{1.324295in}}%
\pgfpathlineto{\pgfqpoint{1.167894in}{1.324203in}}%
\pgfpathlineto{\pgfqpoint{1.168005in}{1.324353in}}%
\pgfpathlineto{\pgfqpoint{1.168005in}{1.324353in}}%
\pgfpathlineto{\pgfqpoint{1.169238in}{1.324641in}}%
\pgfpathlineto{\pgfqpoint{1.169275in}{1.324588in}}%
\pgfpathlineto{\pgfqpoint{1.170488in}{1.324186in}}%
\pgfpathlineto{\pgfqpoint{1.170525in}{1.324218in}}%
\pgfpathlineto{\pgfqpoint{1.171817in}{1.324626in}}%
\pgfpathlineto{\pgfqpoint{1.171915in}{1.324394in}}%
\pgfpathlineto{\pgfqpoint{1.172033in}{1.324172in}}%
\pgfpathlineto{\pgfqpoint{1.172183in}{1.324427in}}%
\pgfpathlineto{\pgfqpoint{1.172183in}{1.324427in}}%
\pgfpathlineto{\pgfqpoint{1.172335in}{1.324622in}}%
\pgfpathlineto{\pgfqpoint{1.172444in}{1.324346in}}%
\pgfpathlineto{\pgfqpoint{1.172444in}{1.324346in}}%
\pgfpathlineto{\pgfqpoint{1.172554in}{1.324169in}}%
\pgfpathlineto{\pgfqpoint{1.172704in}{1.324431in}}%
\pgfpathlineto{\pgfqpoint{1.172704in}{1.324431in}}%
\pgfpathlineto{\pgfqpoint{1.172856in}{1.324616in}}%
\pgfpathlineto{\pgfqpoint{1.172961in}{1.324343in}}%
\pgfpathlineto{\pgfqpoint{1.172961in}{1.324343in}}%
\pgfpathlineto{\pgfqpoint{1.173072in}{1.324165in}}%
\pgfpathlineto{\pgfqpoint{1.173222in}{1.324426in}}%
\pgfpathlineto{\pgfqpoint{1.173222in}{1.324426in}}%
\pgfpathlineto{\pgfqpoint{1.173374in}{1.324613in}}%
\pgfpathlineto{\pgfqpoint{1.173479in}{1.324343in}}%
\pgfpathlineto{\pgfqpoint{1.173479in}{1.324343in}}%
\pgfpathlineto{\pgfqpoint{1.173589in}{1.324160in}}%
\pgfpathlineto{\pgfqpoint{1.173739in}{1.324420in}}%
\pgfpathlineto{\pgfqpoint{1.173739in}{1.324420in}}%
\pgfpathlineto{\pgfqpoint{1.173891in}{1.324610in}}%
\pgfpathlineto{\pgfqpoint{1.174000in}{1.324329in}}%
\pgfpathlineto{\pgfqpoint{1.174000in}{1.324329in}}%
\pgfpathlineto{\pgfqpoint{1.174110in}{1.324157in}}%
\pgfpathlineto{\pgfqpoint{1.174222in}{1.324331in}}%
\pgfpathlineto{\pgfqpoint{1.174222in}{1.324331in}}%
\pgfpathlineto{\pgfqpoint{1.174413in}{1.324604in}}%
\pgfpathlineto{\pgfqpoint{1.174556in}{1.324216in}}%
\pgfpathlineto{\pgfqpoint{1.174556in}{1.324216in}}%
\pgfpathlineto{\pgfqpoint{1.174630in}{1.324153in}}%
\pgfpathlineto{\pgfqpoint{1.174742in}{1.324330in}}%
\pgfpathlineto{\pgfqpoint{1.174742in}{1.324330in}}%
\pgfpathlineto{\pgfqpoint{1.174933in}{1.324600in}}%
\pgfpathlineto{\pgfqpoint{1.175076in}{1.324211in}}%
\pgfpathlineto{\pgfqpoint{1.175076in}{1.324211in}}%
\pgfpathlineto{\pgfqpoint{1.175150in}{1.324149in}}%
\pgfpathlineto{\pgfqpoint{1.175262in}{1.324327in}}%
\pgfpathlineto{\pgfqpoint{1.175262in}{1.324327in}}%
\pgfpathlineto{\pgfqpoint{1.175452in}{1.324596in}}%
\pgfpathlineto{\pgfqpoint{1.175590in}{1.324218in}}%
\pgfpathlineto{\pgfqpoint{1.175590in}{1.324218in}}%
\pgfpathlineto{\pgfqpoint{1.175664in}{1.324144in}}%
\pgfpathlineto{\pgfqpoint{1.175776in}{1.324311in}}%
\pgfpathlineto{\pgfqpoint{1.175776in}{1.324311in}}%
\pgfpathlineto{\pgfqpoint{1.175966in}{1.324596in}}%
\pgfpathlineto{\pgfqpoint{1.176132in}{1.324171in}}%
\pgfpathlineto{\pgfqpoint{1.176132in}{1.324171in}}%
\pgfpathlineto{\pgfqpoint{1.176206in}{1.324152in}}%
\pgfpathlineto{\pgfqpoint{1.176319in}{1.324362in}}%
\pgfpathlineto{\pgfqpoint{1.176319in}{1.324362in}}%
\pgfpathlineto{\pgfqpoint{1.176471in}{1.324598in}}%
\pgfpathlineto{\pgfqpoint{1.176615in}{1.324245in}}%
\pgfpathlineto{\pgfqpoint{1.176615in}{1.324245in}}%
\pgfpathlineto{\pgfqpoint{1.176726in}{1.324149in}}%
\pgfpathlineto{\pgfqpoint{1.176801in}{1.324272in}}%
\pgfpathlineto{\pgfqpoint{1.176801in}{1.324272in}}%
\pgfpathlineto{\pgfqpoint{1.178043in}{1.324583in}}%
\pgfpathlineto{\pgfqpoint{1.178081in}{1.324539in}}%
\pgfpathlineto{\pgfqpoint{1.179309in}{1.324116in}}%
\pgfpathlineto{\pgfqpoint{1.179384in}{1.324209in}}%
\pgfpathlineto{\pgfqpoint{1.180657in}{1.324559in}}%
\pgfpathlineto{\pgfqpoint{1.180695in}{1.324504in}}%
\pgfpathlineto{\pgfqpoint{1.181920in}{1.324097in}}%
\pgfpathlineto{\pgfqpoint{1.181957in}{1.324130in}}%
\pgfpathlineto{\pgfqpoint{1.183264in}{1.324544in}}%
\pgfpathlineto{\pgfqpoint{1.183357in}{1.324327in}}%
\pgfpathlineto{\pgfqpoint{1.183488in}{1.324085in}}%
\pgfpathlineto{\pgfqpoint{1.183640in}{1.324359in}}%
\pgfpathlineto{\pgfqpoint{1.183640in}{1.324359in}}%
\pgfpathlineto{\pgfqpoint{1.183793in}{1.324536in}}%
\pgfpathlineto{\pgfqpoint{1.183893in}{1.324271in}}%
\pgfpathlineto{\pgfqpoint{1.183893in}{1.324271in}}%
\pgfpathlineto{\pgfqpoint{1.184004in}{1.324078in}}%
\pgfpathlineto{\pgfqpoint{1.184156in}{1.324338in}}%
\pgfpathlineto{\pgfqpoint{1.184156in}{1.324338in}}%
\pgfpathlineto{\pgfqpoint{1.184310in}{1.324537in}}%
\pgfpathlineto{\pgfqpoint{1.184417in}{1.324263in}}%
\pgfpathlineto{\pgfqpoint{1.184417in}{1.324263in}}%
\pgfpathlineto{\pgfqpoint{1.184529in}{1.324074in}}%
\pgfpathlineto{\pgfqpoint{1.184681in}{1.324337in}}%
\pgfpathlineto{\pgfqpoint{1.184681in}{1.324337in}}%
\pgfpathlineto{\pgfqpoint{1.184834in}{1.324532in}}%
\pgfpathlineto{\pgfqpoint{1.184934in}{1.324288in}}%
\pgfpathlineto{\pgfqpoint{1.184934in}{1.324288in}}%
\pgfpathlineto{\pgfqpoint{1.185065in}{1.324075in}}%
\pgfpathlineto{\pgfqpoint{1.185218in}{1.324364in}}%
\pgfpathlineto{\pgfqpoint{1.185218in}{1.324364in}}%
\pgfpathlineto{\pgfqpoint{1.185371in}{1.324518in}}%
\pgfpathlineto{\pgfqpoint{1.185456in}{1.324289in}}%
\pgfpathlineto{\pgfqpoint{1.185456in}{1.324289in}}%
\pgfpathlineto{\pgfqpoint{1.185581in}{1.324068in}}%
\pgfpathlineto{\pgfqpoint{1.185733in}{1.324340in}}%
\pgfpathlineto{\pgfqpoint{1.185733in}{1.324340in}}%
\pgfpathlineto{\pgfqpoint{1.185887in}{1.324522in}}%
\pgfpathlineto{\pgfqpoint{1.185982in}{1.324280in}}%
\pgfpathlineto{\pgfqpoint{1.185982in}{1.324280in}}%
\pgfpathlineto{\pgfqpoint{1.186113in}{1.324067in}}%
\pgfpathlineto{\pgfqpoint{1.186266in}{1.324355in}}%
\pgfpathlineto{\pgfqpoint{1.186266in}{1.324355in}}%
\pgfpathlineto{\pgfqpoint{1.186419in}{1.324511in}}%
\pgfpathlineto{\pgfqpoint{1.186511in}{1.324259in}}%
\pgfpathlineto{\pgfqpoint{1.186511in}{1.324259in}}%
\pgfpathlineto{\pgfqpoint{1.186622in}{1.324057in}}%
\pgfpathlineto{\pgfqpoint{1.186774in}{1.324313in}}%
\pgfpathlineto{\pgfqpoint{1.186774in}{1.324313in}}%
\pgfpathlineto{\pgfqpoint{1.186928in}{1.324519in}}%
\pgfpathlineto{\pgfqpoint{1.187043in}{1.324225in}}%
\pgfpathlineto{\pgfqpoint{1.187043in}{1.324225in}}%
\pgfpathlineto{\pgfqpoint{1.187155in}{1.324055in}}%
\pgfpathlineto{\pgfqpoint{1.187269in}{1.324236in}}%
\pgfpathlineto{\pgfqpoint{1.187269in}{1.324236in}}%
\pgfpathlineto{\pgfqpoint{1.187461in}{1.324510in}}%
\pgfpathlineto{\pgfqpoint{1.187606in}{1.324113in}}%
\pgfpathlineto{\pgfqpoint{1.187606in}{1.324113in}}%
\pgfpathlineto{\pgfqpoint{1.187680in}{1.324051in}}%
\pgfpathlineto{\pgfqpoint{1.187794in}{1.324233in}}%
\pgfpathlineto{\pgfqpoint{1.187794in}{1.324233in}}%
\pgfpathlineto{\pgfqpoint{1.187986in}{1.324506in}}%
\pgfpathlineto{\pgfqpoint{1.188113in}{1.324155in}}%
\pgfpathlineto{\pgfqpoint{1.188113in}{1.324155in}}%
\pgfpathlineto{\pgfqpoint{1.188225in}{1.324059in}}%
\pgfpathlineto{\pgfqpoint{1.188301in}{1.324186in}}%
\pgfpathlineto{\pgfqpoint{1.188301in}{1.324186in}}%
\pgfpathlineto{\pgfqpoint{1.189550in}{1.324503in}}%
\pgfpathlineto{\pgfqpoint{1.189588in}{1.324464in}}%
\pgfpathlineto{\pgfqpoint{1.190831in}{1.324025in}}%
\pgfpathlineto{\pgfqpoint{1.190907in}{1.324115in}}%
\pgfpathlineto{\pgfqpoint{1.192194in}{1.324478in}}%
\pgfpathlineto{\pgfqpoint{1.192232in}{1.324425in}}%
\pgfpathlineto{\pgfqpoint{1.193469in}{1.324005in}}%
\pgfpathlineto{\pgfqpoint{1.193507in}{1.324035in}}%
\pgfpathlineto{\pgfqpoint{1.194836in}{1.324457in}}%
\pgfpathlineto{\pgfqpoint{1.194924in}{1.324247in}}%
\pgfpathlineto{\pgfqpoint{1.195052in}{1.323992in}}%
\pgfpathlineto{\pgfqpoint{1.195205in}{1.324258in}}%
\pgfpathlineto{\pgfqpoint{1.195205in}{1.324258in}}%
\pgfpathlineto{\pgfqpoint{1.195360in}{1.324457in}}%
\pgfpathlineto{\pgfqpoint{1.195473in}{1.324159in}}%
\pgfpathlineto{\pgfqpoint{1.195473in}{1.324159in}}%
\pgfpathlineto{\pgfqpoint{1.195586in}{1.323989in}}%
\pgfpathlineto{\pgfqpoint{1.195701in}{1.324173in}}%
\pgfpathlineto{\pgfqpoint{1.195701in}{1.324173in}}%
\pgfpathlineto{\pgfqpoint{1.195894in}{1.324449in}}%
\pgfpathlineto{\pgfqpoint{1.196035in}{1.324057in}}%
\pgfpathlineto{\pgfqpoint{1.196035in}{1.324057in}}%
\pgfpathlineto{\pgfqpoint{1.196111in}{1.323983in}}%
\pgfpathlineto{\pgfqpoint{1.196225in}{1.324159in}}%
\pgfpathlineto{\pgfqpoint{1.196225in}{1.324159in}}%
\pgfpathlineto{\pgfqpoint{1.196420in}{1.324449in}}%
\pgfpathlineto{\pgfqpoint{1.196564in}{1.324055in}}%
\pgfpathlineto{\pgfqpoint{1.196564in}{1.324055in}}%
\pgfpathlineto{\pgfqpoint{1.196639in}{1.323979in}}%
\pgfpathlineto{\pgfqpoint{1.196754in}{1.324152in}}%
\pgfpathlineto{\pgfqpoint{1.196754in}{1.324152in}}%
\pgfpathlineto{\pgfqpoint{1.196948in}{1.324445in}}%
\pgfpathlineto{\pgfqpoint{1.197097in}{1.324042in}}%
\pgfpathlineto{\pgfqpoint{1.197097in}{1.324042in}}%
\pgfpathlineto{\pgfqpoint{1.197173in}{1.323976in}}%
\pgfpathlineto{\pgfqpoint{1.197287in}{1.324158in}}%
\pgfpathlineto{\pgfqpoint{1.197287in}{1.324158in}}%
\pgfpathlineto{\pgfqpoint{1.197482in}{1.324439in}}%
\pgfpathlineto{\pgfqpoint{1.197622in}{1.324049in}}%
\pgfpathlineto{\pgfqpoint{1.197622in}{1.324049in}}%
\pgfpathlineto{\pgfqpoint{1.197698in}{1.323970in}}%
\pgfpathlineto{\pgfqpoint{1.197812in}{1.324142in}}%
\pgfpathlineto{\pgfqpoint{1.197812in}{1.324142in}}%
\pgfpathlineto{\pgfqpoint{1.198007in}{1.324438in}}%
\pgfpathlineto{\pgfqpoint{1.198155in}{1.324038in}}%
\pgfpathlineto{\pgfqpoint{1.198155in}{1.324038in}}%
\pgfpathlineto{\pgfqpoint{1.198231in}{1.323967in}}%
\pgfpathlineto{\pgfqpoint{1.198345in}{1.324145in}}%
\pgfpathlineto{\pgfqpoint{1.198345in}{1.324145in}}%
\pgfpathlineto{\pgfqpoint{1.198540in}{1.324433in}}%
\pgfpathlineto{\pgfqpoint{1.198687in}{1.324030in}}%
\pgfpathlineto{\pgfqpoint{1.198687in}{1.324030in}}%
\pgfpathlineto{\pgfqpoint{1.198763in}{1.323963in}}%
\pgfpathlineto{\pgfqpoint{1.198878in}{1.324145in}}%
\pgfpathlineto{\pgfqpoint{1.198878in}{1.324145in}}%
\pgfpathlineto{\pgfqpoint{1.199072in}{1.324428in}}%
\pgfpathlineto{\pgfqpoint{1.199228in}{1.324006in}}%
\pgfpathlineto{\pgfqpoint{1.199228in}{1.324006in}}%
\pgfpathlineto{\pgfqpoint{1.199303in}{1.323964in}}%
\pgfpathlineto{\pgfqpoint{1.199418in}{1.324165in}}%
\pgfpathlineto{\pgfqpoint{1.199418in}{1.324165in}}%
\pgfpathlineto{\pgfqpoint{1.199575in}{1.324432in}}%
\pgfpathlineto{\pgfqpoint{1.199744in}{1.324033in}}%
\pgfpathlineto{\pgfqpoint{1.199744in}{1.324033in}}%
\pgfpathlineto{\pgfqpoint{1.199819in}{1.323954in}}%
\pgfpathlineto{\pgfqpoint{1.199934in}{1.324125in}}%
\pgfpathlineto{\pgfqpoint{1.199934in}{1.324125in}}%
\pgfpathlineto{\pgfqpoint{1.200129in}{1.324423in}}%
\pgfpathlineto{\pgfqpoint{1.200278in}{1.324022in}}%
\pgfpathlineto{\pgfqpoint{1.200278in}{1.324022in}}%
\pgfpathlineto{\pgfqpoint{1.200354in}{1.323950in}}%
\pgfpathlineto{\pgfqpoint{1.200468in}{1.324128in}}%
\pgfpathlineto{\pgfqpoint{1.200468in}{1.324128in}}%
\pgfpathlineto{\pgfqpoint{1.200664in}{1.324418in}}%
\pgfpathlineto{\pgfqpoint{1.200809in}{1.324018in}}%
\pgfpathlineto{\pgfqpoint{1.200809in}{1.324018in}}%
\pgfpathlineto{\pgfqpoint{1.200885in}{1.323946in}}%
\pgfpathlineto{\pgfqpoint{1.201000in}{1.324125in}}%
\pgfpathlineto{\pgfqpoint{1.201000in}{1.324125in}}%
\pgfpathlineto{\pgfqpoint{1.201195in}{1.324414in}}%
\pgfpathlineto{\pgfqpoint{1.201341in}{1.324015in}}%
\pgfpathlineto{\pgfqpoint{1.201341in}{1.324015in}}%
\pgfpathlineto{\pgfqpoint{1.201416in}{1.323942in}}%
\pgfpathlineto{\pgfqpoint{1.201531in}{1.324119in}}%
\pgfpathlineto{\pgfqpoint{1.201531in}{1.324119in}}%
\pgfpathlineto{\pgfqpoint{1.201727in}{1.324410in}}%
\pgfpathlineto{\pgfqpoint{1.201875in}{1.324006in}}%
\pgfpathlineto{\pgfqpoint{1.201875in}{1.324006in}}%
\pgfpathlineto{\pgfqpoint{1.201950in}{1.323938in}}%
\pgfpathlineto{\pgfqpoint{1.202065in}{1.324120in}}%
\pgfpathlineto{\pgfqpoint{1.202065in}{1.324120in}}%
\pgfpathlineto{\pgfqpoint{1.202261in}{1.324405in}}%
\pgfpathlineto{\pgfqpoint{1.202403in}{1.324009in}}%
\pgfpathlineto{\pgfqpoint{1.202403in}{1.324009in}}%
\pgfpathlineto{\pgfqpoint{1.202479in}{1.323933in}}%
\pgfpathlineto{\pgfqpoint{1.202594in}{1.324108in}}%
\pgfpathlineto{\pgfqpoint{1.202594in}{1.324108in}}%
\pgfpathlineto{\pgfqpoint{1.202790in}{1.324403in}}%
\pgfpathlineto{\pgfqpoint{1.202932in}{1.324014in}}%
\pgfpathlineto{\pgfqpoint{1.202932in}{1.324014in}}%
\pgfpathlineto{\pgfqpoint{1.203008in}{1.323928in}}%
\pgfpathlineto{\pgfqpoint{1.203123in}{1.324095in}}%
\pgfpathlineto{\pgfqpoint{1.203123in}{1.324095in}}%
\pgfpathlineto{\pgfqpoint{1.204388in}{1.324392in}}%
\pgfpathlineto{\pgfqpoint{1.204426in}{1.324342in}}%
\pgfpathlineto{\pgfqpoint{1.205679in}{1.323908in}}%
\pgfpathlineto{\pgfqpoint{1.205717in}{1.323938in}}%
\pgfpathlineto{\pgfqpoint{1.207058in}{1.324372in}}%
\pgfpathlineto{\pgfqpoint{1.207150in}{1.324159in}}%
\pgfpathlineto{\pgfqpoint{1.207282in}{1.323895in}}%
\pgfpathlineto{\pgfqpoint{1.207437in}{1.324169in}}%
\pgfpathlineto{\pgfqpoint{1.207437in}{1.324169in}}%
\pgfpathlineto{\pgfqpoint{1.207594in}{1.324368in}}%
\pgfpathlineto{\pgfqpoint{1.207704in}{1.324077in}}%
\pgfpathlineto{\pgfqpoint{1.207704in}{1.324077in}}%
\pgfpathlineto{\pgfqpoint{1.207818in}{1.323891in}}%
\pgfpathlineto{\pgfqpoint{1.207933in}{1.324072in}}%
\pgfpathlineto{\pgfqpoint{1.207933in}{1.324072in}}%
\pgfpathlineto{\pgfqpoint{1.208130in}{1.324363in}}%
\pgfpathlineto{\pgfqpoint{1.208278in}{1.323955in}}%
\pgfpathlineto{\pgfqpoint{1.208278in}{1.323955in}}%
\pgfpathlineto{\pgfqpoint{1.208354in}{1.323887in}}%
\pgfpathlineto{\pgfqpoint{1.208470in}{1.324071in}}%
\pgfpathlineto{\pgfqpoint{1.208470in}{1.324071in}}%
\pgfpathlineto{\pgfqpoint{1.208666in}{1.324358in}}%
\pgfpathlineto{\pgfqpoint{1.208809in}{1.323961in}}%
\pgfpathlineto{\pgfqpoint{1.208809in}{1.323961in}}%
\pgfpathlineto{\pgfqpoint{1.208885in}{1.323882in}}%
\pgfpathlineto{\pgfqpoint{1.209001in}{1.324057in}}%
\pgfpathlineto{\pgfqpoint{1.209001in}{1.324057in}}%
\pgfpathlineto{\pgfqpoint{1.209198in}{1.324357in}}%
\pgfpathlineto{\pgfqpoint{1.209346in}{1.323953in}}%
\pgfpathlineto{\pgfqpoint{1.209346in}{1.323953in}}%
\pgfpathlineto{\pgfqpoint{1.209423in}{1.323878in}}%
\pgfpathlineto{\pgfqpoint{1.209539in}{1.324057in}}%
\pgfpathlineto{\pgfqpoint{1.209539in}{1.324057in}}%
\pgfpathlineto{\pgfqpoint{1.209735in}{1.324352in}}%
\pgfpathlineto{\pgfqpoint{1.209878in}{1.323959in}}%
\pgfpathlineto{\pgfqpoint{1.209878in}{1.323959in}}%
\pgfpathlineto{\pgfqpoint{1.209954in}{1.323873in}}%
\pgfpathlineto{\pgfqpoint{1.210070in}{1.324043in}}%
\pgfpathlineto{\pgfqpoint{1.210070in}{1.324043in}}%
\pgfpathlineto{\pgfqpoint{1.210267in}{1.324351in}}%
\pgfpathlineto{\pgfqpoint{1.210434in}{1.323912in}}%
\pgfpathlineto{\pgfqpoint{1.210434in}{1.323912in}}%
\pgfpathlineto{\pgfqpoint{1.210511in}{1.323877in}}%
\pgfpathlineto{\pgfqpoint{1.210627in}{1.324089in}}%
\pgfpathlineto{\pgfqpoint{1.210627in}{1.324089in}}%
\pgfpathlineto{\pgfqpoint{1.210785in}{1.324352in}}%
\pgfpathlineto{\pgfqpoint{1.210915in}{1.324064in}}%
\pgfpathlineto{\pgfqpoint{1.210915in}{1.324064in}}%
\pgfpathlineto{\pgfqpoint{1.211029in}{1.323864in}}%
\pgfpathlineto{\pgfqpoint{1.211185in}{1.324136in}}%
\pgfpathlineto{\pgfqpoint{1.211185in}{1.324136in}}%
\pgfpathlineto{\pgfqpoint{1.211342in}{1.324342in}}%
\pgfpathlineto{\pgfqpoint{1.211450in}{1.324067in}}%
\pgfpathlineto{\pgfqpoint{1.211450in}{1.324067in}}%
\pgfpathlineto{\pgfqpoint{1.211564in}{1.323860in}}%
\pgfpathlineto{\pgfqpoint{1.211720in}{1.324127in}}%
\pgfpathlineto{\pgfqpoint{1.211720in}{1.324127in}}%
\pgfpathlineto{\pgfqpoint{1.211877in}{1.324339in}}%
\pgfpathlineto{\pgfqpoint{1.211993in}{1.324037in}}%
\pgfpathlineto{\pgfqpoint{1.211993in}{1.324037in}}%
\pgfpathlineto{\pgfqpoint{1.212108in}{1.323857in}}%
\pgfpathlineto{\pgfqpoint{1.212224in}{1.324044in}}%
\pgfpathlineto{\pgfqpoint{1.212224in}{1.324044in}}%
\pgfpathlineto{\pgfqpoint{1.212421in}{1.324331in}}%
\pgfpathlineto{\pgfqpoint{1.212567in}{1.323924in}}%
\pgfpathlineto{\pgfqpoint{1.212567in}{1.323924in}}%
\pgfpathlineto{\pgfqpoint{1.212644in}{1.323852in}}%
\pgfpathlineto{\pgfqpoint{1.212760in}{1.324036in}}%
\pgfpathlineto{\pgfqpoint{1.212760in}{1.324036in}}%
\pgfpathlineto{\pgfqpoint{1.212957in}{1.324328in}}%
\pgfpathlineto{\pgfqpoint{1.213088in}{1.323961in}}%
\pgfpathlineto{\pgfqpoint{1.213088in}{1.323961in}}%
\pgfpathlineto{\pgfqpoint{1.213203in}{1.323862in}}%
\pgfpathlineto{\pgfqpoint{1.213281in}{1.323993in}}%
\pgfpathlineto{\pgfqpoint{1.213281in}{1.323993in}}%
\pgfpathlineto{\pgfqpoint{1.214568in}{1.324318in}}%
\pgfpathlineto{\pgfqpoint{1.215885in}{1.323833in}}%
\pgfpathlineto{\pgfqpoint{1.216042in}{1.324140in}}%
\pgfpathlineto{\pgfqpoint{1.216160in}{1.324313in}}%
\pgfpathlineto{\pgfqpoint{1.216285in}{1.324058in}}%
\pgfpathlineto{\pgfqpoint{1.216285in}{1.324058in}}%
\pgfpathlineto{\pgfqpoint{1.216410in}{1.323822in}}%
\pgfpathlineto{\pgfqpoint{1.216567in}{1.324101in}}%
\pgfpathlineto{\pgfqpoint{1.216567in}{1.324101in}}%
\pgfpathlineto{\pgfqpoint{1.216725in}{1.324301in}}%
\pgfpathlineto{\pgfqpoint{1.216839in}{1.323992in}}%
\pgfpathlineto{\pgfqpoint{1.216839in}{1.323992in}}%
\pgfpathlineto{\pgfqpoint{1.216955in}{1.323819in}}%
\pgfpathlineto{\pgfqpoint{1.217071in}{1.324012in}}%
\pgfpathlineto{\pgfqpoint{1.217071in}{1.324012in}}%
\pgfpathlineto{\pgfqpoint{1.217269in}{1.324294in}}%
\pgfpathlineto{\pgfqpoint{1.217394in}{1.323939in}}%
\pgfpathlineto{\pgfqpoint{1.217394in}{1.323939in}}%
\pgfpathlineto{\pgfqpoint{1.217509in}{1.323824in}}%
\pgfpathlineto{\pgfqpoint{1.217587in}{1.323951in}}%
\pgfpathlineto{\pgfqpoint{1.217587in}{1.323951in}}%
\pgfpathlineto{\pgfqpoint{1.218881in}{1.324288in}}%
\pgfpathlineto{\pgfqpoint{1.220209in}{1.323801in}}%
\pgfpathlineto{\pgfqpoint{1.220367in}{1.324116in}}%
\pgfpathlineto{\pgfqpoint{1.220486in}{1.324282in}}%
\pgfpathlineto{\pgfqpoint{1.220607in}{1.324024in}}%
\pgfpathlineto{\pgfqpoint{1.220607in}{1.324024in}}%
\pgfpathlineto{\pgfqpoint{1.220734in}{1.323787in}}%
\pgfpathlineto{\pgfqpoint{1.220891in}{1.324070in}}%
\pgfpathlineto{\pgfqpoint{1.220891in}{1.324070in}}%
\pgfpathlineto{\pgfqpoint{1.221050in}{1.324269in}}%
\pgfpathlineto{\pgfqpoint{1.221151in}{1.324009in}}%
\pgfpathlineto{\pgfqpoint{1.221151in}{1.324009in}}%
\pgfpathlineto{\pgfqpoint{1.221290in}{1.323791in}}%
\pgfpathlineto{\pgfqpoint{1.221408in}{1.324005in}}%
\pgfpathlineto{\pgfqpoint{1.221408in}{1.324005in}}%
\pgfpathlineto{\pgfqpoint{1.221567in}{1.324274in}}%
\pgfpathlineto{\pgfqpoint{1.221699in}{1.323979in}}%
\pgfpathlineto{\pgfqpoint{1.221699in}{1.323979in}}%
\pgfpathlineto{\pgfqpoint{1.221814in}{1.323778in}}%
\pgfpathlineto{\pgfqpoint{1.221972in}{1.324055in}}%
\pgfpathlineto{\pgfqpoint{1.221972in}{1.324055in}}%
\pgfpathlineto{\pgfqpoint{1.222131in}{1.324263in}}%
\pgfpathlineto{\pgfqpoint{1.222246in}{1.323956in}}%
\pgfpathlineto{\pgfqpoint{1.222246in}{1.323956in}}%
\pgfpathlineto{\pgfqpoint{1.222362in}{1.323775in}}%
\pgfpathlineto{\pgfqpoint{1.222479in}{1.323966in}}%
\pgfpathlineto{\pgfqpoint{1.222479in}{1.323966in}}%
\pgfpathlineto{\pgfqpoint{1.222678in}{1.324256in}}%
\pgfpathlineto{\pgfqpoint{1.222829in}{1.323833in}}%
\pgfpathlineto{\pgfqpoint{1.222829in}{1.323833in}}%
\pgfpathlineto{\pgfqpoint{1.222906in}{1.323772in}}%
\pgfpathlineto{\pgfqpoint{1.223024in}{1.323967in}}%
\pgfpathlineto{\pgfqpoint{1.223024in}{1.323967in}}%
\pgfpathlineto{\pgfqpoint{1.223222in}{1.324250in}}%
\pgfpathlineto{\pgfqpoint{1.223368in}{1.323836in}}%
\pgfpathlineto{\pgfqpoint{1.223368in}{1.323836in}}%
\pgfpathlineto{\pgfqpoint{1.223446in}{1.323766in}}%
\pgfpathlineto{\pgfqpoint{1.223563in}{1.323955in}}%
\pgfpathlineto{\pgfqpoint{1.223563in}{1.323955in}}%
\pgfpathlineto{\pgfqpoint{1.223762in}{1.324249in}}%
\pgfpathlineto{\pgfqpoint{1.223886in}{1.323899in}}%
\pgfpathlineto{\pgfqpoint{1.223886in}{1.323899in}}%
\pgfpathlineto{\pgfqpoint{1.224002in}{1.323769in}}%
\pgfpathlineto{\pgfqpoint{1.224120in}{1.323985in}}%
\pgfpathlineto{\pgfqpoint{1.224120in}{1.323985in}}%
\pgfpathlineto{\pgfqpoint{1.224280in}{1.324254in}}%
\pgfpathlineto{\pgfqpoint{1.224442in}{1.323858in}}%
\pgfpathlineto{\pgfqpoint{1.224442in}{1.323858in}}%
\pgfpathlineto{\pgfqpoint{1.224519in}{1.323755in}}%
\pgfpathlineto{\pgfqpoint{1.224637in}{1.323917in}}%
\pgfpathlineto{\pgfqpoint{1.224637in}{1.323917in}}%
\pgfpathlineto{\pgfqpoint{1.225937in}{1.324232in}}%
\pgfpathlineto{\pgfqpoint{1.227250in}{1.323735in}}%
\pgfpathlineto{\pgfqpoint{1.227408in}{1.324020in}}%
\pgfpathlineto{\pgfqpoint{1.227568in}{1.324222in}}%
\pgfpathlineto{\pgfqpoint{1.227677in}{1.323932in}}%
\pgfpathlineto{\pgfqpoint{1.227677in}{1.323932in}}%
\pgfpathlineto{\pgfqpoint{1.227793in}{1.323730in}}%
\pgfpathlineto{\pgfqpoint{1.227951in}{1.324011in}}%
\pgfpathlineto{\pgfqpoint{1.227951in}{1.324011in}}%
\pgfpathlineto{\pgfqpoint{1.228111in}{1.324219in}}%
\pgfpathlineto{\pgfqpoint{1.228217in}{1.323945in}}%
\pgfpathlineto{\pgfqpoint{1.228217in}{1.323945in}}%
\pgfpathlineto{\pgfqpoint{1.228331in}{1.323725in}}%
\pgfpathlineto{\pgfqpoint{1.228489in}{1.323990in}}%
\pgfpathlineto{\pgfqpoint{1.228489in}{1.323990in}}%
\pgfpathlineto{\pgfqpoint{1.228649in}{1.324219in}}%
\pgfpathlineto{\pgfqpoint{1.228769in}{1.323914in}}%
\pgfpathlineto{\pgfqpoint{1.228769in}{1.323914in}}%
\pgfpathlineto{\pgfqpoint{1.228886in}{1.323722in}}%
\pgfpathlineto{\pgfqpoint{1.229004in}{1.323909in}}%
\pgfpathlineto{\pgfqpoint{1.229004in}{1.323909in}}%
\pgfpathlineto{\pgfqpoint{1.229204in}{1.324210in}}%
\pgfpathlineto{\pgfqpoint{1.229337in}{1.323836in}}%
\pgfpathlineto{\pgfqpoint{1.229337in}{1.323836in}}%
\pgfpathlineto{\pgfqpoint{1.229453in}{1.323731in}}%
\pgfpathlineto{\pgfqpoint{1.229532in}{1.323864in}}%
\pgfpathlineto{\pgfqpoint{1.229532in}{1.323864in}}%
\pgfpathlineto{\pgfqpoint{1.230841in}{1.324199in}}%
\pgfpathlineto{\pgfqpoint{1.232176in}{1.323701in}}%
\pgfpathlineto{\pgfqpoint{1.232335in}{1.324014in}}%
\pgfpathlineto{\pgfqpoint{1.232456in}{1.324194in}}%
\pgfpathlineto{\pgfqpoint{1.232589in}{1.323905in}}%
\pgfpathlineto{\pgfqpoint{1.232589in}{1.323905in}}%
\pgfpathlineto{\pgfqpoint{1.232706in}{1.323690in}}%
\pgfpathlineto{\pgfqpoint{1.232864in}{1.323965in}}%
\pgfpathlineto{\pgfqpoint{1.232864in}{1.323965in}}%
\pgfpathlineto{\pgfqpoint{1.233025in}{1.324185in}}%
\pgfpathlineto{\pgfqpoint{1.233137in}{1.323898in}}%
\pgfpathlineto{\pgfqpoint{1.233137in}{1.323898in}}%
\pgfpathlineto{\pgfqpoint{1.233254in}{1.323685in}}%
\pgfpathlineto{\pgfqpoint{1.233413in}{1.323963in}}%
\pgfpathlineto{\pgfqpoint{1.233413in}{1.323963in}}%
\pgfpathlineto{\pgfqpoint{1.233574in}{1.324181in}}%
\pgfpathlineto{\pgfqpoint{1.233680in}{1.323912in}}%
\pgfpathlineto{\pgfqpoint{1.233680in}{1.323912in}}%
\pgfpathlineto{\pgfqpoint{1.233822in}{1.323690in}}%
\pgfpathlineto{\pgfqpoint{1.233941in}{1.323910in}}%
\pgfpathlineto{\pgfqpoint{1.233941in}{1.323910in}}%
\pgfpathlineto{\pgfqpoint{1.234102in}{1.324182in}}%
\pgfpathlineto{\pgfqpoint{1.234240in}{1.323863in}}%
\pgfpathlineto{\pgfqpoint{1.234240in}{1.323863in}}%
\pgfpathlineto{\pgfqpoint{1.234357in}{1.323678in}}%
\pgfpathlineto{\pgfqpoint{1.234476in}{1.323873in}}%
\pgfpathlineto{\pgfqpoint{1.234476in}{1.323873in}}%
\pgfpathlineto{\pgfqpoint{1.234677in}{1.324168in}}%
\pgfpathlineto{\pgfqpoint{1.234828in}{1.323741in}}%
\pgfpathlineto{\pgfqpoint{1.234828in}{1.323741in}}%
\pgfpathlineto{\pgfqpoint{1.234906in}{1.323674in}}%
\pgfpathlineto{\pgfqpoint{1.235025in}{1.323870in}}%
\pgfpathlineto{\pgfqpoint{1.235025in}{1.323870in}}%
\pgfpathlineto{\pgfqpoint{1.235226in}{1.324163in}}%
\pgfpathlineto{\pgfqpoint{1.235374in}{1.323743in}}%
\pgfpathlineto{\pgfqpoint{1.235374in}{1.323743in}}%
\pgfpathlineto{\pgfqpoint{1.235452in}{1.323669in}}%
\pgfpathlineto{\pgfqpoint{1.235571in}{1.323858in}}%
\pgfpathlineto{\pgfqpoint{1.235571in}{1.323858in}}%
\pgfpathlineto{\pgfqpoint{1.235772in}{1.324161in}}%
\pgfpathlineto{\pgfqpoint{1.235924in}{1.323736in}}%
\pgfpathlineto{\pgfqpoint{1.235924in}{1.323736in}}%
\pgfpathlineto{\pgfqpoint{1.236002in}{1.323665in}}%
\pgfpathlineto{\pgfqpoint{1.236121in}{1.323857in}}%
\pgfpathlineto{\pgfqpoint{1.236121in}{1.323857in}}%
\pgfpathlineto{\pgfqpoint{1.236322in}{1.324157in}}%
\pgfpathlineto{\pgfqpoint{1.236479in}{1.323718in}}%
\pgfpathlineto{\pgfqpoint{1.236479in}{1.323718in}}%
\pgfpathlineto{\pgfqpoint{1.236558in}{1.323663in}}%
\pgfpathlineto{\pgfqpoint{1.236637in}{1.323776in}}%
\pgfpathlineto{\pgfqpoint{1.236637in}{1.323776in}}%
\pgfpathlineto{\pgfqpoint{1.237965in}{1.324149in}}%
\pgfpathlineto{\pgfqpoint{1.239336in}{1.323665in}}%
\pgfpathlineto{\pgfqpoint{1.239456in}{1.323917in}}%
\pgfpathlineto{\pgfqpoint{1.239618in}{1.324136in}}%
\pgfpathlineto{\pgfqpoint{1.239729in}{1.323850in}}%
\pgfpathlineto{\pgfqpoint{1.239729in}{1.323850in}}%
\pgfpathlineto{\pgfqpoint{1.239846in}{1.323632in}}%
\pgfpathlineto{\pgfqpoint{1.240006in}{1.323910in}}%
\pgfpathlineto{\pgfqpoint{1.240006in}{1.323910in}}%
\pgfpathlineto{\pgfqpoint{1.240168in}{1.324133in}}%
\pgfpathlineto{\pgfqpoint{1.240287in}{1.323817in}}%
\pgfpathlineto{\pgfqpoint{1.240287in}{1.323817in}}%
\pgfpathlineto{\pgfqpoint{1.240405in}{1.323629in}}%
\pgfpathlineto{\pgfqpoint{1.240524in}{1.323824in}}%
\pgfpathlineto{\pgfqpoint{1.240524in}{1.323824in}}%
\pgfpathlineto{\pgfqpoint{1.240727in}{1.324124in}}%
\pgfpathlineto{\pgfqpoint{1.240874in}{1.323703in}}%
\pgfpathlineto{\pgfqpoint{1.240874in}{1.323703in}}%
\pgfpathlineto{\pgfqpoint{1.240953in}{1.323623in}}%
\pgfpathlineto{\pgfqpoint{1.241072in}{1.323811in}}%
\pgfpathlineto{\pgfqpoint{1.241072in}{1.323811in}}%
\pgfpathlineto{\pgfqpoint{1.241275in}{1.324122in}}%
\pgfpathlineto{\pgfqpoint{1.241423in}{1.323707in}}%
\pgfpathlineto{\pgfqpoint{1.241423in}{1.323707in}}%
\pgfpathlineto{\pgfqpoint{1.241501in}{1.323618in}}%
\pgfpathlineto{\pgfqpoint{1.241621in}{1.323798in}}%
\pgfpathlineto{\pgfqpoint{1.241621in}{1.323798in}}%
\pgfpathlineto{\pgfqpoint{1.241824in}{1.324120in}}%
\pgfpathlineto{\pgfqpoint{1.241987in}{1.323675in}}%
\pgfpathlineto{\pgfqpoint{1.241987in}{1.323675in}}%
\pgfpathlineto{\pgfqpoint{1.242065in}{1.323618in}}%
\pgfpathlineto{\pgfqpoint{1.242145in}{1.323730in}}%
\pgfpathlineto{\pgfqpoint{1.242145in}{1.323730in}}%
\pgfpathlineto{\pgfqpoint{1.243464in}{1.324113in}}%
\pgfpathlineto{\pgfqpoint{1.243504in}{1.324086in}}%
\pgfpathlineto{\pgfqpoint{1.244834in}{1.323597in}}%
\pgfpathlineto{\pgfqpoint{1.244913in}{1.323716in}}%
\pgfpathlineto{\pgfqpoint{1.246255in}{1.324084in}}%
\pgfpathlineto{\pgfqpoint{1.247595in}{1.323570in}}%
\pgfpathlineto{\pgfqpoint{1.247756in}{1.323863in}}%
\pgfpathlineto{\pgfqpoint{1.247919in}{1.324072in}}%
\pgfpathlineto{\pgfqpoint{1.248023in}{1.323799in}}%
\pgfpathlineto{\pgfqpoint{1.248023in}{1.323799in}}%
\pgfpathlineto{\pgfqpoint{1.248168in}{1.323575in}}%
\pgfpathlineto{\pgfqpoint{1.248289in}{1.323804in}}%
\pgfpathlineto{\pgfqpoint{1.248289in}{1.323804in}}%
\pgfpathlineto{\pgfqpoint{1.248453in}{1.324077in}}%
\pgfpathlineto{\pgfqpoint{1.248586in}{1.323767in}}%
\pgfpathlineto{\pgfqpoint{1.248586in}{1.323767in}}%
\pgfpathlineto{\pgfqpoint{1.248704in}{1.323560in}}%
\pgfpathlineto{\pgfqpoint{1.248865in}{1.323851in}}%
\pgfpathlineto{\pgfqpoint{1.248865in}{1.323851in}}%
\pgfpathlineto{\pgfqpoint{1.249029in}{1.324065in}}%
\pgfpathlineto{\pgfqpoint{1.249136in}{1.323784in}}%
\pgfpathlineto{\pgfqpoint{1.249136in}{1.323784in}}%
\pgfpathlineto{\pgfqpoint{1.249249in}{1.323555in}}%
\pgfpathlineto{\pgfqpoint{1.249410in}{1.323818in}}%
\pgfpathlineto{\pgfqpoint{1.249410in}{1.323818in}}%
\pgfpathlineto{\pgfqpoint{1.249573in}{1.324067in}}%
\pgfpathlineto{\pgfqpoint{1.249697in}{1.323761in}}%
\pgfpathlineto{\pgfqpoint{1.249697in}{1.323761in}}%
\pgfpathlineto{\pgfqpoint{1.249815in}{1.323551in}}%
\pgfpathlineto{\pgfqpoint{1.249977in}{1.323840in}}%
\pgfpathlineto{\pgfqpoint{1.249977in}{1.323840in}}%
\pgfpathlineto{\pgfqpoint{1.250140in}{1.324057in}}%
\pgfpathlineto{\pgfqpoint{1.250254in}{1.323752in}}%
\pgfpathlineto{\pgfqpoint{1.250254in}{1.323752in}}%
\pgfpathlineto{\pgfqpoint{1.250373in}{1.323547in}}%
\pgfpathlineto{\pgfqpoint{1.250534in}{1.323839in}}%
\pgfpathlineto{\pgfqpoint{1.250534in}{1.323839in}}%
\pgfpathlineto{\pgfqpoint{1.250698in}{1.324052in}}%
\pgfpathlineto{\pgfqpoint{1.250808in}{1.323760in}}%
\pgfpathlineto{\pgfqpoint{1.250808in}{1.323760in}}%
\pgfpathlineto{\pgfqpoint{1.250926in}{1.323541in}}%
\pgfpathlineto{\pgfqpoint{1.251088in}{1.323827in}}%
\pgfpathlineto{\pgfqpoint{1.251088in}{1.323827in}}%
\pgfpathlineto{\pgfqpoint{1.251252in}{1.324050in}}%
\pgfpathlineto{\pgfqpoint{1.251364in}{1.323757in}}%
\pgfpathlineto{\pgfqpoint{1.251364in}{1.323757in}}%
\pgfpathlineto{\pgfqpoint{1.251483in}{1.323537in}}%
\pgfpathlineto{\pgfqpoint{1.251645in}{1.323821in}}%
\pgfpathlineto{\pgfqpoint{1.251645in}{1.323821in}}%
\pgfpathlineto{\pgfqpoint{1.251808in}{1.324047in}}%
\pgfpathlineto{\pgfqpoint{1.251918in}{1.323766in}}%
\pgfpathlineto{\pgfqpoint{1.251918in}{1.323766in}}%
\pgfpathlineto{\pgfqpoint{1.252028in}{1.323532in}}%
\pgfpathlineto{\pgfqpoint{1.252189in}{1.323785in}}%
\pgfpathlineto{\pgfqpoint{1.252189in}{1.323785in}}%
\pgfpathlineto{\pgfqpoint{1.252353in}{1.324047in}}%
\pgfpathlineto{\pgfqpoint{1.252501in}{1.323668in}}%
\pgfpathlineto{\pgfqpoint{1.252501in}{1.323668in}}%
\pgfpathlineto{\pgfqpoint{1.252620in}{1.323538in}}%
\pgfpathlineto{\pgfqpoint{1.252701in}{1.323669in}}%
\pgfpathlineto{\pgfqpoint{1.252701in}{1.323669in}}%
\pgfpathlineto{\pgfqpoint{1.254048in}{1.324024in}}%
\pgfpathlineto{\pgfqpoint{1.255394in}{1.323505in}}%
\pgfpathlineto{\pgfqpoint{1.255515in}{1.323698in}}%
\pgfpathlineto{\pgfqpoint{1.255721in}{1.324015in}}%
\pgfpathlineto{\pgfqpoint{1.255874in}{1.323581in}}%
\pgfpathlineto{\pgfqpoint{1.255874in}{1.323581in}}%
\pgfpathlineto{\pgfqpoint{1.255954in}{1.323501in}}%
\pgfpathlineto{\pgfqpoint{1.256075in}{1.323694in}}%
\pgfpathlineto{\pgfqpoint{1.256075in}{1.323694in}}%
\pgfpathlineto{\pgfqpoint{1.256280in}{1.324011in}}%
\pgfpathlineto{\pgfqpoint{1.256431in}{1.323584in}}%
\pgfpathlineto{\pgfqpoint{1.256431in}{1.323584in}}%
\pgfpathlineto{\pgfqpoint{1.256510in}{1.323496in}}%
\pgfpathlineto{\pgfqpoint{1.256631in}{1.323682in}}%
\pgfpathlineto{\pgfqpoint{1.256631in}{1.323682in}}%
\pgfpathlineto{\pgfqpoint{1.256837in}{1.324009in}}%
\pgfpathlineto{\pgfqpoint{1.256995in}{1.323568in}}%
\pgfpathlineto{\pgfqpoint{1.256995in}{1.323568in}}%
\pgfpathlineto{\pgfqpoint{1.257075in}{1.323492in}}%
\pgfpathlineto{\pgfqpoint{1.257196in}{1.323690in}}%
\pgfpathlineto{\pgfqpoint{1.257196in}{1.323690in}}%
\pgfpathlineto{\pgfqpoint{1.257402in}{1.324001in}}%
\pgfpathlineto{\pgfqpoint{1.257553in}{1.323570in}}%
\pgfpathlineto{\pgfqpoint{1.257553in}{1.323570in}}%
\pgfpathlineto{\pgfqpoint{1.257632in}{1.323487in}}%
\pgfpathlineto{\pgfqpoint{1.257754in}{1.323678in}}%
\pgfpathlineto{\pgfqpoint{1.257754in}{1.323678in}}%
\pgfpathlineto{\pgfqpoint{1.257960in}{1.323999in}}%
\pgfpathlineto{\pgfqpoint{1.258115in}{1.323561in}}%
\pgfpathlineto{\pgfqpoint{1.258115in}{1.323561in}}%
\pgfpathlineto{\pgfqpoint{1.258195in}{1.323483in}}%
\pgfpathlineto{\pgfqpoint{1.258316in}{1.323679in}}%
\pgfpathlineto{\pgfqpoint{1.258316in}{1.323679in}}%
\pgfpathlineto{\pgfqpoint{1.258522in}{1.323993in}}%
\pgfpathlineto{\pgfqpoint{1.258683in}{1.323541in}}%
\pgfpathlineto{\pgfqpoint{1.258683in}{1.323541in}}%
\pgfpathlineto{\pgfqpoint{1.258763in}{1.323481in}}%
\pgfpathlineto{\pgfqpoint{1.258843in}{1.323596in}}%
\pgfpathlineto{\pgfqpoint{1.258843in}{1.323596in}}%
\pgfpathlineto{\pgfqpoint{1.260192in}{1.323988in}}%
\pgfpathlineto{\pgfqpoint{1.261576in}{1.323461in}}%
\pgfpathlineto{\pgfqpoint{1.261823in}{1.323949in}}%
\pgfpathlineto{\pgfqpoint{1.261905in}{1.323957in}}%
\pgfpathlineto{\pgfqpoint{1.261994in}{1.323707in}}%
\pgfpathlineto{\pgfqpoint{1.261994in}{1.323707in}}%
\pgfpathlineto{\pgfqpoint{1.262118in}{1.323449in}}%
\pgfpathlineto{\pgfqpoint{1.262281in}{1.323730in}}%
\pgfpathlineto{\pgfqpoint{1.262281in}{1.323730in}}%
\pgfpathlineto{\pgfqpoint{1.262447in}{1.323969in}}%
\pgfpathlineto{\pgfqpoint{1.262560in}{1.323688in}}%
\pgfpathlineto{\pgfqpoint{1.262560in}{1.323688in}}%
\pgfpathlineto{\pgfqpoint{1.262704in}{1.323453in}}%
\pgfpathlineto{\pgfqpoint{1.262826in}{1.323679in}}%
\pgfpathlineto{\pgfqpoint{1.262826in}{1.323679in}}%
\pgfpathlineto{\pgfqpoint{1.262992in}{1.323969in}}%
\pgfpathlineto{\pgfqpoint{1.263134in}{1.323640in}}%
\pgfpathlineto{\pgfqpoint{1.263134in}{1.323640in}}%
\pgfpathlineto{\pgfqpoint{1.263254in}{1.323441in}}%
\pgfpathlineto{\pgfqpoint{1.263376in}{1.323641in}}%
\pgfpathlineto{\pgfqpoint{1.263376in}{1.323641in}}%
\pgfpathlineto{\pgfqpoint{1.263583in}{1.323955in}}%
\pgfpathlineto{\pgfqpoint{1.263732in}{1.323524in}}%
\pgfpathlineto{\pgfqpoint{1.263732in}{1.323524in}}%
\pgfpathlineto{\pgfqpoint{1.263813in}{1.323435in}}%
\pgfpathlineto{\pgfqpoint{1.263935in}{1.323625in}}%
\pgfpathlineto{\pgfqpoint{1.263935in}{1.323625in}}%
\pgfpathlineto{\pgfqpoint{1.264142in}{1.323954in}}%
\pgfpathlineto{\pgfqpoint{1.264295in}{1.323522in}}%
\pgfpathlineto{\pgfqpoint{1.264295in}{1.323522in}}%
\pgfpathlineto{\pgfqpoint{1.264375in}{1.323431in}}%
\pgfpathlineto{\pgfqpoint{1.264497in}{1.323618in}}%
\pgfpathlineto{\pgfqpoint{1.264497in}{1.323618in}}%
\pgfpathlineto{\pgfqpoint{1.264705in}{1.323951in}}%
\pgfpathlineto{\pgfqpoint{1.264860in}{1.323515in}}%
\pgfpathlineto{\pgfqpoint{1.264860in}{1.323515in}}%
\pgfpathlineto{\pgfqpoint{1.264941in}{1.323426in}}%
\pgfpathlineto{\pgfqpoint{1.265063in}{1.323616in}}%
\pgfpathlineto{\pgfqpoint{1.265063in}{1.323616in}}%
\pgfpathlineto{\pgfqpoint{1.265270in}{1.323946in}}%
\pgfpathlineto{\pgfqpoint{1.265420in}{1.323522in}}%
\pgfpathlineto{\pgfqpoint{1.265420in}{1.323522in}}%
\pgfpathlineto{\pgfqpoint{1.265500in}{1.323421in}}%
\pgfpathlineto{\pgfqpoint{1.265622in}{1.323600in}}%
\pgfpathlineto{\pgfqpoint{1.265622in}{1.323600in}}%
\pgfpathlineto{\pgfqpoint{1.265830in}{1.323944in}}%
\pgfpathlineto{\pgfqpoint{1.266017in}{1.323450in}}%
\pgfpathlineto{\pgfqpoint{1.266017in}{1.323450in}}%
\pgfpathlineto{\pgfqpoint{1.266097in}{1.323433in}}%
\pgfpathlineto{\pgfqpoint{1.266220in}{1.323678in}}%
\pgfpathlineto{\pgfqpoint{1.266220in}{1.323678in}}%
\pgfpathlineto{\pgfqpoint{1.266386in}{1.323943in}}%
\pgfpathlineto{\pgfqpoint{1.266529in}{1.323577in}}%
\pgfpathlineto{\pgfqpoint{1.266529in}{1.323577in}}%
\pgfpathlineto{\pgfqpoint{1.266649in}{1.323418in}}%
\pgfpathlineto{\pgfqpoint{1.266772in}{1.323640in}}%
\pgfpathlineto{\pgfqpoint{1.266772in}{1.323640in}}%
\pgfpathlineto{\pgfqpoint{1.266938in}{1.323939in}}%
\pgfpathlineto{\pgfqpoint{1.267083in}{1.323614in}}%
\pgfpathlineto{\pgfqpoint{1.267083in}{1.323614in}}%
\pgfpathlineto{\pgfqpoint{1.267203in}{1.323408in}}%
\pgfpathlineto{\pgfqpoint{1.267326in}{1.323606in}}%
\pgfpathlineto{\pgfqpoint{1.267326in}{1.323606in}}%
\pgfpathlineto{\pgfqpoint{1.267533in}{1.323926in}}%
\pgfpathlineto{\pgfqpoint{1.267692in}{1.323475in}}%
\pgfpathlineto{\pgfqpoint{1.267692in}{1.323475in}}%
\pgfpathlineto{\pgfqpoint{1.267772in}{1.323405in}}%
\pgfpathlineto{\pgfqpoint{1.267854in}{1.323515in}}%
\pgfpathlineto{\pgfqpoint{1.267854in}{1.323515in}}%
\pgfpathlineto{\pgfqpoint{1.269231in}{1.323914in}}%
\pgfpathlineto{\pgfqpoint{1.270621in}{1.323391in}}%
\pgfpathlineto{\pgfqpoint{1.270744in}{1.323630in}}%
\pgfpathlineto{\pgfqpoint{1.270911in}{1.323910in}}%
\pgfpathlineto{\pgfqpoint{1.271057in}{1.323548in}}%
\pgfpathlineto{\pgfqpoint{1.271057in}{1.323548in}}%
\pgfpathlineto{\pgfqpoint{1.271178in}{1.323379in}}%
\pgfpathlineto{\pgfqpoint{1.271301in}{1.323598in}}%
\pgfpathlineto{\pgfqpoint{1.271301in}{1.323598in}}%
\pgfpathlineto{\pgfqpoint{1.271509in}{1.323889in}}%
\pgfpathlineto{\pgfqpoint{1.271614in}{1.323582in}}%
\pgfpathlineto{\pgfqpoint{1.271614in}{1.323582in}}%
\pgfpathlineto{\pgfqpoint{1.271735in}{1.323370in}}%
\pgfpathlineto{\pgfqpoint{1.271858in}{1.323568in}}%
\pgfpathlineto{\pgfqpoint{1.271858in}{1.323568in}}%
\pgfpathlineto{\pgfqpoint{1.272067in}{1.323893in}}%
\pgfpathlineto{\pgfqpoint{1.272220in}{1.323454in}}%
\pgfpathlineto{\pgfqpoint{1.272220in}{1.323454in}}%
\pgfpathlineto{\pgfqpoint{1.272301in}{1.323365in}}%
\pgfpathlineto{\pgfqpoint{1.272424in}{1.323558in}}%
\pgfpathlineto{\pgfqpoint{1.272424in}{1.323558in}}%
\pgfpathlineto{\pgfqpoint{1.272633in}{1.323890in}}%
\pgfpathlineto{\pgfqpoint{1.272801in}{1.323421in}}%
\pgfpathlineto{\pgfqpoint{1.272801in}{1.323421in}}%
\pgfpathlineto{\pgfqpoint{1.272882in}{1.323365in}}%
\pgfpathlineto{\pgfqpoint{1.272964in}{1.323486in}}%
\pgfpathlineto{\pgfqpoint{1.272964in}{1.323486in}}%
\pgfpathlineto{\pgfqpoint{1.274323in}{1.323884in}}%
\pgfpathlineto{\pgfqpoint{1.276325in}{1.323361in}}%
\pgfpathlineto{\pgfqpoint{1.276407in}{1.323521in}}%
\pgfpathlineto{\pgfqpoint{1.276617in}{1.323862in}}%
\pgfpathlineto{\pgfqpoint{1.276776in}{1.323413in}}%
\pgfpathlineto{\pgfqpoint{1.276776in}{1.323413in}}%
\pgfpathlineto{\pgfqpoint{1.276857in}{1.323327in}}%
\pgfpathlineto{\pgfqpoint{1.276981in}{1.323525in}}%
\pgfpathlineto{\pgfqpoint{1.276981in}{1.323525in}}%
\pgfpathlineto{\pgfqpoint{1.277191in}{1.323855in}}%
\pgfpathlineto{\pgfqpoint{1.277344in}{1.323416in}}%
\pgfpathlineto{\pgfqpoint{1.277344in}{1.323416in}}%
\pgfpathlineto{\pgfqpoint{1.277425in}{1.323322in}}%
\pgfpathlineto{\pgfqpoint{1.277549in}{1.323513in}}%
\pgfpathlineto{\pgfqpoint{1.277549in}{1.323513in}}%
\pgfpathlineto{\pgfqpoint{1.277759in}{1.323853in}}%
\pgfpathlineto{\pgfqpoint{1.277918in}{1.323404in}}%
\pgfpathlineto{\pgfqpoint{1.277918in}{1.323404in}}%
\pgfpathlineto{\pgfqpoint{1.277999in}{1.323318in}}%
\pgfpathlineto{\pgfqpoint{1.278123in}{1.323516in}}%
\pgfpathlineto{\pgfqpoint{1.278123in}{1.323516in}}%
\pgfpathlineto{\pgfqpoint{1.278333in}{1.323847in}}%
\pgfpathlineto{\pgfqpoint{1.278493in}{1.323390in}}%
\pgfpathlineto{\pgfqpoint{1.278493in}{1.323390in}}%
\pgfpathlineto{\pgfqpoint{1.278575in}{1.323314in}}%
\pgfpathlineto{\pgfqpoint{1.278657in}{1.323424in}}%
\pgfpathlineto{\pgfqpoint{1.278657in}{1.323424in}}%
\pgfpathlineto{\pgfqpoint{1.280012in}{1.323839in}}%
\pgfpathlineto{\pgfqpoint{1.285662in}{1.323670in}}%
\pgfpathlineto{\pgfqpoint{1.285788in}{1.323790in}}%
\pgfpathlineto{\pgfqpoint{1.285883in}{1.323561in}}%
\pgfpathlineto{\pgfqpoint{1.285883in}{1.323561in}}%
\pgfpathlineto{\pgfqpoint{1.286041in}{1.323257in}}%
\pgfpathlineto{\pgfqpoint{1.286208in}{1.323596in}}%
\pgfpathlineto{\pgfqpoint{1.286208in}{1.323596in}}%
\pgfpathlineto{\pgfqpoint{1.286335in}{1.323794in}}%
\pgfpathlineto{\pgfqpoint{1.286469in}{1.323511in}}%
\pgfpathlineto{\pgfqpoint{1.286469in}{1.323511in}}%
\pgfpathlineto{\pgfqpoint{1.286602in}{1.323246in}}%
\pgfpathlineto{\pgfqpoint{1.286769in}{1.323553in}}%
\pgfpathlineto{\pgfqpoint{1.286769in}{1.323553in}}%
\pgfpathlineto{\pgfqpoint{1.286938in}{1.323782in}}%
\pgfpathlineto{\pgfqpoint{1.287058in}{1.323451in}}%
\pgfpathlineto{\pgfqpoint{1.287058in}{1.323451in}}%
\pgfpathlineto{\pgfqpoint{1.287181in}{1.323242in}}%
\pgfpathlineto{\pgfqpoint{1.287306in}{1.323448in}}%
\pgfpathlineto{\pgfqpoint{1.287306in}{1.323448in}}%
\pgfpathlineto{\pgfqpoint{1.287517in}{1.323776in}}%
\pgfpathlineto{\pgfqpoint{1.287672in}{1.323324in}}%
\pgfpathlineto{\pgfqpoint{1.287672in}{1.323324in}}%
\pgfpathlineto{\pgfqpoint{1.287755in}{1.323236in}}%
\pgfpathlineto{\pgfqpoint{1.287879in}{1.323437in}}%
\pgfpathlineto{\pgfqpoint{1.287879in}{1.323437in}}%
\pgfpathlineto{\pgfqpoint{1.288091in}{1.323773in}}%
\pgfpathlineto{\pgfqpoint{1.288251in}{1.323313in}}%
\pgfpathlineto{\pgfqpoint{1.288251in}{1.323313in}}%
\pgfpathlineto{\pgfqpoint{1.288334in}{1.323232in}}%
\pgfpathlineto{\pgfqpoint{1.288417in}{1.323340in}}%
\pgfpathlineto{\pgfqpoint{1.288417in}{1.323340in}}%
\pgfpathlineto{\pgfqpoint{1.289784in}{1.323764in}}%
\pgfpathlineto{\pgfqpoint{1.297271in}{1.323655in}}%
\pgfpathlineto{\pgfqpoint{1.297356in}{1.323705in}}%
\pgfpathlineto{\pgfqpoint{1.297439in}{1.323534in}}%
\pgfpathlineto{\pgfqpoint{1.297439in}{1.323534in}}%
\pgfpathlineto{\pgfqpoint{1.297582in}{1.323154in}}%
\pgfpathlineto{\pgfqpoint{1.297837in}{1.323623in}}%
\pgfpathlineto{\pgfqpoint{1.297837in}{1.323623in}}%
\pgfpathlineto{\pgfqpoint{1.297922in}{1.323706in}}%
\pgfpathlineto{\pgfqpoint{1.298018in}{1.323539in}}%
\pgfpathlineto{\pgfqpoint{1.298018in}{1.323539in}}%
\pgfpathlineto{\pgfqpoint{1.298195in}{1.323153in}}%
\pgfpathlineto{\pgfqpoint{1.298408in}{1.323595in}}%
\pgfpathlineto{\pgfqpoint{1.298408in}{1.323595in}}%
\pgfpathlineto{\pgfqpoint{1.298534in}{1.323684in}}%
\pgfpathlineto{\pgfqpoint{1.298576in}{1.323608in}}%
\pgfpathlineto{\pgfqpoint{1.298576in}{1.323608in}}%
\pgfpathlineto{\pgfqpoint{1.299349in}{1.323139in}}%
\pgfpathlineto{\pgfqpoint{1.299690in}{1.323685in}}%
\pgfpathlineto{\pgfqpoint{1.301094in}{1.323123in}}%
\pgfpathlineto{\pgfqpoint{1.301221in}{1.323318in}}%
\pgfpathlineto{\pgfqpoint{1.301435in}{1.323675in}}%
\pgfpathlineto{\pgfqpoint{1.301619in}{1.323165in}}%
\pgfpathlineto{\pgfqpoint{1.301619in}{1.323165in}}%
\pgfpathlineto{\pgfqpoint{1.301702in}{1.323129in}}%
\pgfpathlineto{\pgfqpoint{1.301787in}{1.323270in}}%
\pgfpathlineto{\pgfqpoint{1.301787in}{1.323270in}}%
\pgfpathlineto{\pgfqpoint{1.303107in}{1.323614in}}%
\pgfpathlineto{\pgfqpoint{1.303192in}{1.323660in}}%
\pgfpathlineto{\pgfqpoint{1.303276in}{1.323479in}}%
\pgfpathlineto{\pgfqpoint{1.303276in}{1.323479in}}%
\pgfpathlineto{\pgfqpoint{1.303445in}{1.323107in}}%
\pgfpathlineto{\pgfqpoint{1.303659in}{1.323548in}}%
\pgfpathlineto{\pgfqpoint{1.303659in}{1.323548in}}%
\pgfpathlineto{\pgfqpoint{1.303787in}{1.323647in}}%
\pgfpathlineto{\pgfqpoint{1.303871in}{1.323437in}}%
\pgfpathlineto{\pgfqpoint{1.303871in}{1.323437in}}%
\pgfpathlineto{\pgfqpoint{1.304023in}{1.323099in}}%
\pgfpathlineto{\pgfqpoint{1.304193in}{1.323419in}}%
\pgfpathlineto{\pgfqpoint{1.304193in}{1.323419in}}%
\pgfpathlineto{\pgfqpoint{1.304365in}{1.323649in}}%
\pgfpathlineto{\pgfqpoint{1.304477in}{1.323341in}}%
\pgfpathlineto{\pgfqpoint{1.304477in}{1.323341in}}%
\pgfpathlineto{\pgfqpoint{1.304599in}{1.323093in}}%
\pgfpathlineto{\pgfqpoint{1.304769in}{1.323389in}}%
\pgfpathlineto{\pgfqpoint{1.304769in}{1.323389in}}%
\pgfpathlineto{\pgfqpoint{1.304941in}{1.323650in}}%
\pgfpathlineto{\pgfqpoint{1.305061in}{1.323347in}}%
\pgfpathlineto{\pgfqpoint{1.305061in}{1.323347in}}%
\pgfpathlineto{\pgfqpoint{1.305213in}{1.323099in}}%
\pgfpathlineto{\pgfqpoint{1.305340in}{1.323346in}}%
\pgfpathlineto{\pgfqpoint{1.305340in}{1.323346in}}%
\pgfpathlineto{\pgfqpoint{1.305512in}{1.323649in}}%
\pgfpathlineto{\pgfqpoint{1.305652in}{1.323322in}}%
\pgfpathlineto{\pgfqpoint{1.305652in}{1.323322in}}%
\pgfpathlineto{\pgfqpoint{1.305777in}{1.323083in}}%
\pgfpathlineto{\pgfqpoint{1.305947in}{1.323395in}}%
\pgfpathlineto{\pgfqpoint{1.305947in}{1.323395in}}%
\pgfpathlineto{\pgfqpoint{1.306119in}{1.323638in}}%
\pgfpathlineto{\pgfqpoint{1.306243in}{1.323299in}}%
\pgfpathlineto{\pgfqpoint{1.306243in}{1.323299in}}%
\pgfpathlineto{\pgfqpoint{1.306368in}{1.323080in}}%
\pgfpathlineto{\pgfqpoint{1.306495in}{1.323290in}}%
\pgfpathlineto{\pgfqpoint{1.306495in}{1.323290in}}%
\pgfpathlineto{\pgfqpoint{1.306711in}{1.323631in}}%
\pgfpathlineto{\pgfqpoint{1.306873in}{1.323154in}}%
\pgfpathlineto{\pgfqpoint{1.306873in}{1.323154in}}%
\pgfpathlineto{\pgfqpoint{1.306957in}{1.323075in}}%
\pgfpathlineto{\pgfqpoint{1.307042in}{1.323190in}}%
\pgfpathlineto{\pgfqpoint{1.307042in}{1.323190in}}%
\pgfpathlineto{\pgfqpoint{1.308407in}{1.323603in}}%
\pgfpathlineto{\pgfqpoint{1.308492in}{1.323597in}}%
\pgfpathlineto{\pgfqpoint{1.308584in}{1.323315in}}%
\pgfpathlineto{\pgfqpoint{1.308584in}{1.323315in}}%
\pgfpathlineto{\pgfqpoint{1.308698in}{1.323059in}}%
\pgfpathlineto{\pgfqpoint{1.308868in}{1.323332in}}%
\pgfpathlineto{\pgfqpoint{1.308868in}{1.323332in}}%
\pgfpathlineto{\pgfqpoint{1.309042in}{1.323622in}}%
\pgfpathlineto{\pgfqpoint{1.309179in}{1.323281in}}%
\pgfpathlineto{\pgfqpoint{1.309179in}{1.323281in}}%
\pgfpathlineto{\pgfqpoint{1.309304in}{1.323054in}}%
\pgfpathlineto{\pgfqpoint{1.309432in}{1.323262in}}%
\pgfpathlineto{\pgfqpoint{1.309432in}{1.323262in}}%
\pgfpathlineto{\pgfqpoint{1.309648in}{1.323609in}}%
\pgfpathlineto{\pgfqpoint{1.309785in}{1.323211in}}%
\pgfpathlineto{\pgfqpoint{1.309785in}{1.323211in}}%
\pgfpathlineto{\pgfqpoint{1.309911in}{1.323058in}}%
\pgfpathlineto{\pgfqpoint{1.310038in}{1.323304in}}%
\pgfpathlineto{\pgfqpoint{1.310038in}{1.323304in}}%
\pgfpathlineto{\pgfqpoint{1.310212in}{1.323613in}}%
\pgfpathlineto{\pgfqpoint{1.310355in}{1.323277in}}%
\pgfpathlineto{\pgfqpoint{1.310355in}{1.323277in}}%
\pgfpathlineto{\pgfqpoint{1.310480in}{1.323044in}}%
\pgfpathlineto{\pgfqpoint{1.310651in}{1.323361in}}%
\pgfpathlineto{\pgfqpoint{1.310651in}{1.323361in}}%
\pgfpathlineto{\pgfqpoint{1.310825in}{1.323601in}}%
\pgfpathlineto{\pgfqpoint{1.310950in}{1.323248in}}%
\pgfpathlineto{\pgfqpoint{1.310950in}{1.323248in}}%
\pgfpathlineto{\pgfqpoint{1.311076in}{1.323041in}}%
\pgfpathlineto{\pgfqpoint{1.311203in}{1.323260in}}%
\pgfpathlineto{\pgfqpoint{1.311203in}{1.323260in}}%
\pgfpathlineto{\pgfqpoint{1.311420in}{1.323592in}}%
\pgfpathlineto{\pgfqpoint{1.311575in}{1.323129in}}%
\pgfpathlineto{\pgfqpoint{1.311575in}{1.323129in}}%
\pgfpathlineto{\pgfqpoint{1.311659in}{1.323033in}}%
\pgfpathlineto{\pgfqpoint{1.311786in}{1.323237in}}%
\pgfpathlineto{\pgfqpoint{1.311786in}{1.323237in}}%
\pgfpathlineto{\pgfqpoint{1.312003in}{1.323593in}}%
\pgfpathlineto{\pgfqpoint{1.312181in}{1.323087in}}%
\pgfpathlineto{\pgfqpoint{1.312181in}{1.323087in}}%
\pgfpathlineto{\pgfqpoint{1.312265in}{1.323036in}}%
\pgfpathlineto{\pgfqpoint{1.312350in}{1.323170in}}%
\pgfpathlineto{\pgfqpoint{1.312350in}{1.323170in}}%
\pgfpathlineto{\pgfqpoint{1.313690in}{1.323533in}}%
\pgfpathlineto{\pgfqpoint{1.313775in}{1.323579in}}%
\pgfpathlineto{\pgfqpoint{1.313861in}{1.323394in}}%
\pgfpathlineto{\pgfqpoint{1.313861in}{1.323394in}}%
\pgfpathlineto{\pgfqpoint{1.314019in}{1.323013in}}%
\pgfpathlineto{\pgfqpoint{1.314236in}{1.323440in}}%
\pgfpathlineto{\pgfqpoint{1.314236in}{1.323440in}}%
\pgfpathlineto{\pgfqpoint{1.314365in}{1.323575in}}%
\pgfpathlineto{\pgfqpoint{1.314466in}{1.323330in}}%
\pgfpathlineto{\pgfqpoint{1.314466in}{1.323330in}}%
\pgfpathlineto{\pgfqpoint{1.314609in}{1.323008in}}%
\pgfpathlineto{\pgfqpoint{1.314780in}{1.323319in}}%
\pgfpathlineto{\pgfqpoint{1.314780in}{1.323319in}}%
\pgfpathlineto{\pgfqpoint{1.314954in}{1.323572in}}%
\pgfpathlineto{\pgfqpoint{1.315079in}{1.323233in}}%
\pgfpathlineto{\pgfqpoint{1.315079in}{1.323233in}}%
\pgfpathlineto{\pgfqpoint{1.315205in}{1.323004in}}%
\pgfpathlineto{\pgfqpoint{1.315333in}{1.323214in}}%
\pgfpathlineto{\pgfqpoint{1.315333in}{1.323214in}}%
\pgfpathlineto{\pgfqpoint{1.315551in}{1.323564in}}%
\pgfpathlineto{\pgfqpoint{1.315714in}{1.323087in}}%
\pgfpathlineto{\pgfqpoint{1.315714in}{1.323087in}}%
\pgfpathlineto{\pgfqpoint{1.315798in}{1.322999in}}%
\pgfpathlineto{\pgfqpoint{1.315883in}{1.323110in}}%
\pgfpathlineto{\pgfqpoint{1.315883in}{1.323110in}}%
\pgfpathlineto{\pgfqpoint{1.317245in}{1.323512in}}%
\pgfpathlineto{\pgfqpoint{1.317331in}{1.323549in}}%
\pgfpathlineto{\pgfqpoint{1.317416in}{1.323352in}}%
\pgfpathlineto{\pgfqpoint{1.317416in}{1.323352in}}%
\pgfpathlineto{\pgfqpoint{1.317577in}{1.322984in}}%
\pgfpathlineto{\pgfqpoint{1.317794in}{1.323423in}}%
\pgfpathlineto{\pgfqpoint{1.317794in}{1.323423in}}%
\pgfpathlineto{\pgfqpoint{1.317923in}{1.323544in}}%
\pgfpathlineto{\pgfqpoint{1.318009in}{1.323349in}}%
\pgfpathlineto{\pgfqpoint{1.318009in}{1.323349in}}%
\pgfpathlineto{\pgfqpoint{1.318160in}{1.322977in}}%
\pgfpathlineto{\pgfqpoint{1.318377in}{1.323395in}}%
\pgfpathlineto{\pgfqpoint{1.318377in}{1.323395in}}%
\pgfpathlineto{\pgfqpoint{1.318507in}{1.323546in}}%
\pgfpathlineto{\pgfqpoint{1.318618in}{1.323280in}}%
\pgfpathlineto{\pgfqpoint{1.318618in}{1.323280in}}%
\pgfpathlineto{\pgfqpoint{1.318752in}{1.322972in}}%
\pgfpathlineto{\pgfqpoint{1.318924in}{1.323273in}}%
\pgfpathlineto{\pgfqpoint{1.318924in}{1.323273in}}%
\pgfpathlineto{\pgfqpoint{1.319099in}{1.323542in}}%
\pgfpathlineto{\pgfqpoint{1.319227in}{1.323205in}}%
\pgfpathlineto{\pgfqpoint{1.319227in}{1.323205in}}%
\pgfpathlineto{\pgfqpoint{1.319354in}{1.322968in}}%
\pgfpathlineto{\pgfqpoint{1.319526in}{1.323290in}}%
\pgfpathlineto{\pgfqpoint{1.319526in}{1.323290in}}%
\pgfpathlineto{\pgfqpoint{1.319701in}{1.323533in}}%
\pgfpathlineto{\pgfqpoint{1.319818in}{1.323215in}}%
\pgfpathlineto{\pgfqpoint{1.319818in}{1.323215in}}%
\pgfpathlineto{\pgfqpoint{1.319943in}{1.322962in}}%
\pgfpathlineto{\pgfqpoint{1.320115in}{1.323270in}}%
\pgfpathlineto{\pgfqpoint{1.320115in}{1.323270in}}%
\pgfpathlineto{\pgfqpoint{1.320290in}{1.323532in}}%
\pgfpathlineto{\pgfqpoint{1.320417in}{1.323191in}}%
\pgfpathlineto{\pgfqpoint{1.320417in}{1.323191in}}%
\pgfpathlineto{\pgfqpoint{1.320544in}{1.322958in}}%
\pgfpathlineto{\pgfqpoint{1.320673in}{1.323169in}}%
\pgfpathlineto{\pgfqpoint{1.320673in}{1.323169in}}%
\pgfpathlineto{\pgfqpoint{1.320892in}{1.323524in}}%
\pgfpathlineto{\pgfqpoint{1.321054in}{1.323046in}}%
\pgfpathlineto{\pgfqpoint{1.321054in}{1.323046in}}%
\pgfpathlineto{\pgfqpoint{1.321139in}{1.322953in}}%
\pgfpathlineto{\pgfqpoint{1.321268in}{1.323163in}}%
\pgfpathlineto{\pgfqpoint{1.321268in}{1.323163in}}%
\pgfpathlineto{\pgfqpoint{1.321486in}{1.323519in}}%
\pgfpathlineto{\pgfqpoint{1.321650in}{1.323039in}}%
\pgfpathlineto{\pgfqpoint{1.321650in}{1.323039in}}%
\pgfpathlineto{\pgfqpoint{1.321735in}{1.322948in}}%
\pgfpathlineto{\pgfqpoint{1.321820in}{1.323058in}}%
\pgfpathlineto{\pgfqpoint{1.321820in}{1.323058in}}%
\pgfpathlineto{\pgfqpoint{1.323177in}{1.323444in}}%
\pgfpathlineto{\pgfqpoint{1.323264in}{1.323511in}}%
\pgfpathlineto{\pgfqpoint{1.323363in}{1.323305in}}%
\pgfpathlineto{\pgfqpoint{1.323363in}{1.323305in}}%
\pgfpathlineto{\pgfqpoint{1.323523in}{1.322932in}}%
\pgfpathlineto{\pgfqpoint{1.323742in}{1.323373in}}%
\pgfpathlineto{\pgfqpoint{1.323742in}{1.323373in}}%
\pgfpathlineto{\pgfqpoint{1.323872in}{1.323500in}}%
\pgfpathlineto{\pgfqpoint{1.323958in}{1.323307in}}%
\pgfpathlineto{\pgfqpoint{1.323958in}{1.323307in}}%
\pgfpathlineto{\pgfqpoint{1.324120in}{1.322927in}}%
\pgfpathlineto{\pgfqpoint{1.324339in}{1.323369in}}%
\pgfpathlineto{\pgfqpoint{1.324339in}{1.323369in}}%
\pgfpathlineto{\pgfqpoint{1.324469in}{1.323495in}}%
\pgfpathlineto{\pgfqpoint{1.324555in}{1.323301in}}%
\pgfpathlineto{\pgfqpoint{1.324555in}{1.323301in}}%
\pgfpathlineto{\pgfqpoint{1.324712in}{1.322921in}}%
\pgfpathlineto{\pgfqpoint{1.324931in}{1.323353in}}%
\pgfpathlineto{\pgfqpoint{1.324931in}{1.323353in}}%
\pgfpathlineto{\pgfqpoint{1.325061in}{1.323494in}}%
\pgfpathlineto{\pgfqpoint{1.325160in}{1.323264in}}%
\pgfpathlineto{\pgfqpoint{1.325160in}{1.323264in}}%
\pgfpathlineto{\pgfqpoint{1.325329in}{1.322924in}}%
\pgfpathlineto{\pgfqpoint{1.325503in}{1.323287in}}%
\pgfpathlineto{\pgfqpoint{1.325503in}{1.323287in}}%
\pgfpathlineto{\pgfqpoint{1.325635in}{1.323495in}}%
\pgfpathlineto{\pgfqpoint{1.325777in}{1.323174in}}%
\pgfpathlineto{\pgfqpoint{1.325777in}{1.323174in}}%
\pgfpathlineto{\pgfqpoint{1.325894in}{1.322911in}}%
\pgfpathlineto{\pgfqpoint{1.326067in}{1.323194in}}%
\pgfpathlineto{\pgfqpoint{1.326067in}{1.323194in}}%
\pgfpathlineto{\pgfqpoint{1.326243in}{1.323490in}}%
\pgfpathlineto{\pgfqpoint{1.326378in}{1.323155in}}%
\pgfpathlineto{\pgfqpoint{1.326378in}{1.323155in}}%
\pgfpathlineto{\pgfqpoint{1.326506in}{1.322906in}}%
\pgfpathlineto{\pgfqpoint{1.326679in}{1.323226in}}%
\pgfpathlineto{\pgfqpoint{1.326679in}{1.323226in}}%
\pgfpathlineto{\pgfqpoint{1.326855in}{1.323480in}}%
\pgfpathlineto{\pgfqpoint{1.326971in}{1.323172in}}%
\pgfpathlineto{\pgfqpoint{1.326971in}{1.323172in}}%
\pgfpathlineto{\pgfqpoint{1.327122in}{1.322908in}}%
\pgfpathlineto{\pgfqpoint{1.327252in}{1.323154in}}%
\pgfpathlineto{\pgfqpoint{1.327252in}{1.323154in}}%
\pgfpathlineto{\pgfqpoint{1.327428in}{1.323481in}}%
\pgfpathlineto{\pgfqpoint{1.327579in}{1.323127in}}%
\pgfpathlineto{\pgfqpoint{1.327579in}{1.323127in}}%
\pgfpathlineto{\pgfqpoint{1.327707in}{1.322896in}}%
\pgfpathlineto{\pgfqpoint{1.327837in}{1.323112in}}%
\pgfpathlineto{\pgfqpoint{1.327837in}{1.323112in}}%
\pgfpathlineto{\pgfqpoint{1.328057in}{1.323468in}}%
\pgfpathlineto{\pgfqpoint{1.328208in}{1.323017in}}%
\pgfpathlineto{\pgfqpoint{1.328208in}{1.323017in}}%
\pgfpathlineto{\pgfqpoint{1.328293in}{1.322890in}}%
\pgfpathlineto{\pgfqpoint{1.328422in}{1.323074in}}%
\pgfpathlineto{\pgfqpoint{1.328422in}{1.323074in}}%
\pgfpathlineto{\pgfqpoint{1.328643in}{1.323470in}}%
\pgfpathlineto{\pgfqpoint{1.328835in}{1.322943in}}%
\pgfpathlineto{\pgfqpoint{1.328835in}{1.322943in}}%
\pgfpathlineto{\pgfqpoint{1.328921in}{1.322894in}}%
\pgfpathlineto{\pgfqpoint{1.329007in}{1.323034in}}%
\pgfpathlineto{\pgfqpoint{1.329007in}{1.323034in}}%
\pgfpathlineto{\pgfqpoint{1.329828in}{1.323463in}}%
\pgfpathlineto{\pgfqpoint{1.330136in}{1.322897in}}%
\pgfpathlineto{\pgfqpoint{1.331653in}{1.323443in}}%
\pgfpathlineto{\pgfqpoint{1.331696in}{1.323386in}}%
\pgfpathlineto{\pgfqpoint{1.332528in}{1.322865in}}%
\pgfpathlineto{\pgfqpoint{1.332835in}{1.323440in}}%
\pgfpathlineto{\pgfqpoint{1.334957in}{1.322865in}}%
\pgfpathlineto{\pgfqpoint{1.335089in}{1.323158in}}%
\pgfpathlineto{\pgfqpoint{1.335266in}{1.323415in}}%
\pgfpathlineto{\pgfqpoint{1.335386in}{1.323090in}}%
\pgfpathlineto{\pgfqpoint{1.335386in}{1.323090in}}%
\pgfpathlineto{\pgfqpoint{1.335509in}{1.322827in}}%
\pgfpathlineto{\pgfqpoint{1.335684in}{1.323132in}}%
\pgfpathlineto{\pgfqpoint{1.335684in}{1.323132in}}%
\pgfpathlineto{\pgfqpoint{1.335862in}{1.323414in}}%
\pgfpathlineto{\pgfqpoint{1.335990in}{1.323081in}}%
\pgfpathlineto{\pgfqpoint{1.335990in}{1.323081in}}%
\pgfpathlineto{\pgfqpoint{1.336118in}{1.322822in}}%
\pgfpathlineto{\pgfqpoint{1.336293in}{1.323142in}}%
\pgfpathlineto{\pgfqpoint{1.336293in}{1.323142in}}%
\pgfpathlineto{\pgfqpoint{1.336471in}{1.323407in}}%
\pgfpathlineto{\pgfqpoint{1.336590in}{1.323094in}}%
\pgfpathlineto{\pgfqpoint{1.336590in}{1.323094in}}%
\pgfpathlineto{\pgfqpoint{1.336741in}{1.322824in}}%
\pgfpathlineto{\pgfqpoint{1.336872in}{1.323071in}}%
\pgfpathlineto{\pgfqpoint{1.336872in}{1.323071in}}%
\pgfpathlineto{\pgfqpoint{1.337050in}{1.323406in}}%
\pgfpathlineto{\pgfqpoint{1.337218in}{1.322995in}}%
\pgfpathlineto{\pgfqpoint{1.337218in}{1.322995in}}%
\pgfpathlineto{\pgfqpoint{1.337347in}{1.322820in}}%
\pgfpathlineto{\pgfqpoint{1.337478in}{1.323069in}}%
\pgfpathlineto{\pgfqpoint{1.337478in}{1.323069in}}%
\pgfpathlineto{\pgfqpoint{1.337656in}{1.323402in}}%
\pgfpathlineto{\pgfqpoint{1.337837in}{1.322940in}}%
\pgfpathlineto{\pgfqpoint{1.337837in}{1.322940in}}%
\pgfpathlineto{\pgfqpoint{1.337923in}{1.322807in}}%
\pgfpathlineto{\pgfqpoint{1.338054in}{1.322990in}}%
\pgfpathlineto{\pgfqpoint{1.338054in}{1.322990in}}%
\pgfpathlineto{\pgfqpoint{1.338276in}{1.323396in}}%
\pgfpathlineto{\pgfqpoint{1.338461in}{1.322883in}}%
\pgfpathlineto{\pgfqpoint{1.338461in}{1.322883in}}%
\pgfpathlineto{\pgfqpoint{1.338548in}{1.322804in}}%
\pgfpathlineto{\pgfqpoint{1.338635in}{1.322928in}}%
\pgfpathlineto{\pgfqpoint{1.338635in}{1.322928in}}%
\pgfpathlineto{\pgfqpoint{1.339505in}{1.323377in}}%
\pgfpathlineto{\pgfqpoint{1.339792in}{1.322823in}}%
\pgfpathlineto{\pgfqpoint{1.341320in}{1.323365in}}%
\pgfpathlineto{\pgfqpoint{1.342786in}{1.322765in}}%
\pgfpathlineto{\pgfqpoint{1.342874in}{1.322880in}}%
\pgfpathlineto{\pgfqpoint{1.343751in}{1.323345in}}%
\pgfpathlineto{\pgfqpoint{1.344007in}{1.322756in}}%
\pgfpathlineto{\pgfqpoint{1.345579in}{1.323330in}}%
\pgfpathlineto{\pgfqpoint{1.345695in}{1.322994in}}%
\pgfpathlineto{\pgfqpoint{1.345825in}{1.322738in}}%
\pgfpathlineto{\pgfqpoint{1.346002in}{1.323069in}}%
\pgfpathlineto{\pgfqpoint{1.346002in}{1.323069in}}%
\pgfpathlineto{\pgfqpoint{1.346181in}{1.323330in}}%
\pgfpathlineto{\pgfqpoint{1.346307in}{1.322979in}}%
\pgfpathlineto{\pgfqpoint{1.346307in}{1.322979in}}%
\pgfpathlineto{\pgfqpoint{1.346437in}{1.322733in}}%
\pgfpathlineto{\pgfqpoint{1.346569in}{1.322952in}}%
\pgfpathlineto{\pgfqpoint{1.346569in}{1.322952in}}%
\pgfpathlineto{\pgfqpoint{1.346794in}{1.323324in}}%
\pgfpathlineto{\pgfqpoint{1.346962in}{1.322823in}}%
\pgfpathlineto{\pgfqpoint{1.346962in}{1.322823in}}%
\pgfpathlineto{\pgfqpoint{1.347049in}{1.322728in}}%
\pgfpathlineto{\pgfqpoint{1.347136in}{1.322844in}}%
\pgfpathlineto{\pgfqpoint{1.347136in}{1.322844in}}%
\pgfpathlineto{\pgfqpoint{1.348025in}{1.323304in}}%
\pgfpathlineto{\pgfqpoint{1.348269in}{1.322718in}}%
\pgfpathlineto{\pgfqpoint{1.349844in}{1.323302in}}%
\pgfpathlineto{\pgfqpoint{1.350015in}{1.322801in}}%
\pgfpathlineto{\pgfqpoint{1.350102in}{1.322701in}}%
\pgfpathlineto{\pgfqpoint{1.350190in}{1.322814in}}%
\pgfpathlineto{\pgfqpoint{1.350190in}{1.322814in}}%
\pgfpathlineto{\pgfqpoint{1.351069in}{1.323292in}}%
\pgfpathlineto{\pgfqpoint{1.351319in}{1.322690in}}%
\pgfpathlineto{\pgfqpoint{1.353573in}{1.323199in}}%
\pgfpathlineto{\pgfqpoint{1.353651in}{1.322909in}}%
\pgfpathlineto{\pgfqpoint{1.353782in}{1.322670in}}%
\pgfpathlineto{\pgfqpoint{1.353915in}{1.322895in}}%
\pgfpathlineto{\pgfqpoint{1.353915in}{1.322895in}}%
\pgfpathlineto{\pgfqpoint{1.354141in}{1.323265in}}%
\pgfpathlineto{\pgfqpoint{1.354307in}{1.322762in}}%
\pgfpathlineto{\pgfqpoint{1.354307in}{1.322762in}}%
\pgfpathlineto{\pgfqpoint{1.354395in}{1.322664in}}%
\pgfpathlineto{\pgfqpoint{1.354483in}{1.322779in}}%
\pgfpathlineto{\pgfqpoint{1.354483in}{1.322779in}}%
\pgfpathlineto{\pgfqpoint{1.355362in}{1.323261in}}%
\pgfpathlineto{\pgfqpoint{1.355606in}{1.322653in}}%
\pgfpathlineto{\pgfqpoint{1.359110in}{1.323163in}}%
\pgfpathlineto{\pgfqpoint{1.359194in}{1.322853in}}%
\pgfpathlineto{\pgfqpoint{1.359326in}{1.322622in}}%
\pgfpathlineto{\pgfqpoint{1.359460in}{1.322855in}}%
\pgfpathlineto{\pgfqpoint{1.359460in}{1.322855in}}%
\pgfpathlineto{\pgfqpoint{1.359687in}{1.323220in}}%
\pgfpathlineto{\pgfqpoint{1.359814in}{1.322839in}}%
\pgfpathlineto{\pgfqpoint{1.359814in}{1.322839in}}%
\pgfpathlineto{\pgfqpoint{1.359946in}{1.322617in}}%
\pgfpathlineto{\pgfqpoint{1.360080in}{1.322856in}}%
\pgfpathlineto{\pgfqpoint{1.360080in}{1.322856in}}%
\pgfpathlineto{\pgfqpoint{1.360307in}{1.323213in}}%
\pgfpathlineto{\pgfqpoint{1.360449in}{1.322769in}}%
\pgfpathlineto{\pgfqpoint{1.360449in}{1.322769in}}%
\pgfpathlineto{\pgfqpoint{1.360581in}{1.322623in}}%
\pgfpathlineto{\pgfqpoint{1.360671in}{1.322782in}}%
\pgfpathlineto{\pgfqpoint{1.360671in}{1.322782in}}%
\pgfpathlineto{\pgfqpoint{1.361512in}{1.323217in}}%
\pgfpathlineto{\pgfqpoint{1.361779in}{1.322598in}}%
\pgfpathlineto{\pgfqpoint{1.364681in}{1.323115in}}%
\pgfpathlineto{\pgfqpoint{1.364804in}{1.322673in}}%
\pgfpathlineto{\pgfqpoint{1.364893in}{1.322571in}}%
\pgfpathlineto{\pgfqpoint{1.364982in}{1.322686in}}%
\pgfpathlineto{\pgfqpoint{1.364982in}{1.322686in}}%
\pgfpathlineto{\pgfqpoint{1.365878in}{1.323174in}}%
\pgfpathlineto{\pgfqpoint{1.366092in}{1.322577in}}%
\pgfpathlineto{\pgfqpoint{1.367652in}{1.323110in}}%
\pgfpathlineto{\pgfqpoint{1.367742in}{1.323160in}}%
\pgfpathlineto{\pgfqpoint{1.367832in}{1.322959in}}%
\pgfpathlineto{\pgfqpoint{1.367832in}{1.322959in}}%
\pgfpathlineto{\pgfqpoint{1.368011in}{1.322547in}}%
\pgfpathlineto{\pgfqpoint{1.368238in}{1.323034in}}%
\pgfpathlineto{\pgfqpoint{1.368238in}{1.323034in}}%
\pgfpathlineto{\pgfqpoint{1.368374in}{1.323147in}}%
\pgfpathlineto{\pgfqpoint{1.368464in}{1.322917in}}%
\pgfpathlineto{\pgfqpoint{1.368464in}{1.322917in}}%
\pgfpathlineto{\pgfqpoint{1.368627in}{1.322539in}}%
\pgfpathlineto{\pgfqpoint{1.368808in}{1.322895in}}%
\pgfpathlineto{\pgfqpoint{1.368808in}{1.322895in}}%
\pgfpathlineto{\pgfqpoint{1.368991in}{1.323148in}}%
\pgfpathlineto{\pgfqpoint{1.369115in}{1.322782in}}%
\pgfpathlineto{\pgfqpoint{1.369115in}{1.322782in}}%
\pgfpathlineto{\pgfqpoint{1.369248in}{1.322533in}}%
\pgfpathlineto{\pgfqpoint{1.369383in}{1.322762in}}%
\pgfpathlineto{\pgfqpoint{1.369383in}{1.322762in}}%
\pgfpathlineto{\pgfqpoint{1.369613in}{1.323144in}}%
\pgfpathlineto{\pgfqpoint{1.369781in}{1.322630in}}%
\pgfpathlineto{\pgfqpoint{1.369781in}{1.322630in}}%
\pgfpathlineto{\pgfqpoint{1.369870in}{1.322527in}}%
\pgfpathlineto{\pgfqpoint{1.369960in}{1.322643in}}%
\pgfpathlineto{\pgfqpoint{1.369960in}{1.322643in}}%
\pgfpathlineto{\pgfqpoint{1.370857in}{1.323136in}}%
\pgfpathlineto{\pgfqpoint{1.371091in}{1.322520in}}%
\pgfpathlineto{\pgfqpoint{1.384853in}{1.322516in}}%
\pgfpathlineto{\pgfqpoint{1.384943in}{1.322391in}}%
\pgfpathlineto{\pgfqpoint{1.385080in}{1.322606in}}%
\pgfpathlineto{\pgfqpoint{1.385080in}{1.322606in}}%
\pgfpathlineto{\pgfqpoint{1.385313in}{1.323025in}}%
\pgfpathlineto{\pgfqpoint{1.385493in}{1.322492in}}%
\pgfpathlineto{\pgfqpoint{1.385493in}{1.322492in}}%
\pgfpathlineto{\pgfqpoint{1.385583in}{1.322387in}}%
\pgfpathlineto{\pgfqpoint{1.385674in}{1.322506in}}%
\pgfpathlineto{\pgfqpoint{1.385674in}{1.322506in}}%
\pgfpathlineto{\pgfqpoint{1.386562in}{1.323019in}}%
\pgfpathlineto{\pgfqpoint{1.386781in}{1.322431in}}%
\pgfpathlineto{\pgfqpoint{1.386872in}{1.322386in}}%
\pgfpathlineto{\pgfqpoint{1.386963in}{1.322546in}}%
\pgfpathlineto{\pgfqpoint{1.386963in}{1.322546in}}%
\pgfpathlineto{\pgfqpoint{1.387829in}{1.323009in}}%
\pgfpathlineto{\pgfqpoint{1.388052in}{1.322417in}}%
\pgfpathlineto{\pgfqpoint{1.388142in}{1.322376in}}%
\pgfpathlineto{\pgfqpoint{1.388234in}{1.322538in}}%
\pgfpathlineto{\pgfqpoint{1.388234in}{1.322538in}}%
\pgfpathlineto{\pgfqpoint{1.389125in}{1.322991in}}%
\pgfpathlineto{\pgfqpoint{1.389323in}{1.322404in}}%
\pgfpathlineto{\pgfqpoint{1.389414in}{1.322366in}}%
\pgfpathlineto{\pgfqpoint{1.389506in}{1.322530in}}%
\pgfpathlineto{\pgfqpoint{1.389506in}{1.322530in}}%
\pgfpathlineto{\pgfqpoint{1.390392in}{1.322984in}}%
\pgfpathlineto{\pgfqpoint{1.390625in}{1.322353in}}%
\pgfpathlineto{\pgfqpoint{1.393478in}{1.322878in}}%
\pgfpathlineto{\pgfqpoint{1.393571in}{1.322962in}}%
\pgfpathlineto{\pgfqpoint{1.393676in}{1.322744in}}%
\pgfpathlineto{\pgfqpoint{1.393676in}{1.322744in}}%
\pgfpathlineto{\pgfqpoint{1.393848in}{1.322312in}}%
\pgfpathlineto{\pgfqpoint{1.394082in}{1.322798in}}%
\pgfpathlineto{\pgfqpoint{1.394082in}{1.322798in}}%
\pgfpathlineto{\pgfqpoint{1.394221in}{1.322951in}}%
\pgfpathlineto{\pgfqpoint{1.394327in}{1.322687in}}%
\pgfpathlineto{\pgfqpoint{1.394327in}{1.322687in}}%
\pgfpathlineto{\pgfqpoint{1.394501in}{1.322311in}}%
\pgfpathlineto{\pgfqpoint{1.394686in}{1.322705in}}%
\pgfpathlineto{\pgfqpoint{1.394686in}{1.322705in}}%
\pgfpathlineto{\pgfqpoint{1.394874in}{1.322934in}}%
\pgfpathlineto{\pgfqpoint{1.394981in}{1.322613in}}%
\pgfpathlineto{\pgfqpoint{1.394981in}{1.322613in}}%
\pgfpathlineto{\pgfqpoint{1.395132in}{1.322302in}}%
\pgfpathlineto{\pgfqpoint{1.395317in}{1.322677in}}%
\pgfpathlineto{\pgfqpoint{1.395317in}{1.322677in}}%
\pgfpathlineto{\pgfqpoint{1.395505in}{1.322936in}}%
\pgfpathlineto{\pgfqpoint{1.395629in}{1.322566in}}%
\pgfpathlineto{\pgfqpoint{1.395629in}{1.322566in}}%
\pgfpathlineto{\pgfqpoint{1.395766in}{1.322295in}}%
\pgfpathlineto{\pgfqpoint{1.395904in}{1.322528in}}%
\pgfpathlineto{\pgfqpoint{1.395904in}{1.322528in}}%
\pgfpathlineto{\pgfqpoint{1.396140in}{1.322935in}}%
\pgfpathlineto{\pgfqpoint{1.396323in}{1.322378in}}%
\pgfpathlineto{\pgfqpoint{1.396323in}{1.322378in}}%
\pgfpathlineto{\pgfqpoint{1.396414in}{1.322291in}}%
\pgfpathlineto{\pgfqpoint{1.396506in}{1.322427in}}%
\pgfpathlineto{\pgfqpoint{1.396506in}{1.322427in}}%
\pgfpathlineto{\pgfqpoint{1.397409in}{1.322931in}}%
\pgfpathlineto{\pgfqpoint{1.397643in}{1.322296in}}%
\pgfpathlineto{\pgfqpoint{1.399095in}{1.322467in}}%
\pgfpathlineto{\pgfqpoint{1.399238in}{1.322831in}}%
\pgfpathlineto{\pgfqpoint{1.399331in}{1.322916in}}%
\pgfpathlineto{\pgfqpoint{1.399438in}{1.322697in}}%
\pgfpathlineto{\pgfqpoint{1.399438in}{1.322697in}}%
\pgfpathlineto{\pgfqpoint{1.399628in}{1.322267in}}%
\pgfpathlineto{\pgfqpoint{1.399863in}{1.322790in}}%
\pgfpathlineto{\pgfqpoint{1.399863in}{1.322790in}}%
\pgfpathlineto{\pgfqpoint{1.400003in}{1.322889in}}%
\pgfpathlineto{\pgfqpoint{1.400049in}{1.322796in}}%
\pgfpathlineto{\pgfqpoint{1.400049in}{1.322796in}}%
\pgfpathlineto{\pgfqpoint{1.400894in}{1.322248in}}%
\pgfpathlineto{\pgfqpoint{1.401177in}{1.322837in}}%
\pgfpathlineto{\pgfqpoint{1.401270in}{1.322895in}}%
\pgfpathlineto{\pgfqpoint{1.401363in}{1.322691in}}%
\pgfpathlineto{\pgfqpoint{1.401363in}{1.322691in}}%
\pgfpathlineto{\pgfqpoint{1.401539in}{1.322242in}}%
\pgfpathlineto{\pgfqpoint{1.401775in}{1.322736in}}%
\pgfpathlineto{\pgfqpoint{1.401775in}{1.322736in}}%
\pgfpathlineto{\pgfqpoint{1.401915in}{1.322889in}}%
\pgfpathlineto{\pgfqpoint{1.402022in}{1.322620in}}%
\pgfpathlineto{\pgfqpoint{1.402022in}{1.322620in}}%
\pgfpathlineto{\pgfqpoint{1.402183in}{1.322237in}}%
\pgfpathlineto{\pgfqpoint{1.402370in}{1.322604in}}%
\pgfpathlineto{\pgfqpoint{1.402370in}{1.322604in}}%
\pgfpathlineto{\pgfqpoint{1.402559in}{1.322883in}}%
\pgfpathlineto{\pgfqpoint{1.402683in}{1.322533in}}%
\pgfpathlineto{\pgfqpoint{1.402683in}{1.322533in}}%
\pgfpathlineto{\pgfqpoint{1.402850in}{1.322243in}}%
\pgfpathlineto{\pgfqpoint{1.402990in}{1.322533in}}%
\pgfpathlineto{\pgfqpoint{1.402990in}{1.322533in}}%
\pgfpathlineto{\pgfqpoint{1.403180in}{1.322887in}}%
\pgfpathlineto{\pgfqpoint{1.403334in}{1.322495in}}%
\pgfpathlineto{\pgfqpoint{1.403334in}{1.322495in}}%
\pgfpathlineto{\pgfqpoint{1.403472in}{1.322225in}}%
\pgfpathlineto{\pgfqpoint{1.403611in}{1.322464in}}%
\pgfpathlineto{\pgfqpoint{1.403611in}{1.322464in}}%
\pgfpathlineto{\pgfqpoint{1.403849in}{1.322873in}}%
\pgfpathlineto{\pgfqpoint{1.404026in}{1.322325in}}%
\pgfpathlineto{\pgfqpoint{1.404026in}{1.322325in}}%
\pgfpathlineto{\pgfqpoint{1.404117in}{1.322219in}}%
\pgfpathlineto{\pgfqpoint{1.404210in}{1.322345in}}%
\pgfpathlineto{\pgfqpoint{1.404210in}{1.322345in}}%
\pgfpathlineto{\pgfqpoint{1.405127in}{1.322869in}}%
\pgfpathlineto{\pgfqpoint{1.405360in}{1.322229in}}%
\pgfpathlineto{\pgfqpoint{1.406792in}{1.322318in}}%
\pgfpathlineto{\pgfqpoint{1.406935in}{1.322694in}}%
\pgfpathlineto{\pgfqpoint{1.407077in}{1.322847in}}%
\pgfpathlineto{\pgfqpoint{1.407184in}{1.322575in}}%
\pgfpathlineto{\pgfqpoint{1.407184in}{1.322575in}}%
\pgfpathlineto{\pgfqpoint{1.407345in}{1.322189in}}%
\pgfpathlineto{\pgfqpoint{1.407532in}{1.322557in}}%
\pgfpathlineto{\pgfqpoint{1.407532in}{1.322557in}}%
\pgfpathlineto{\pgfqpoint{1.407723in}{1.322843in}}%
\pgfpathlineto{\pgfqpoint{1.407847in}{1.322495in}}%
\pgfpathlineto{\pgfqpoint{1.407847in}{1.322495in}}%
\pgfpathlineto{\pgfqpoint{1.408007in}{1.322190in}}%
\pgfpathlineto{\pgfqpoint{1.408148in}{1.322463in}}%
\pgfpathlineto{\pgfqpoint{1.408148in}{1.322463in}}%
\pgfpathlineto{\pgfqpoint{1.408338in}{1.322844in}}%
\pgfpathlineto{\pgfqpoint{1.408530in}{1.322345in}}%
\pgfpathlineto{\pgfqpoint{1.408530in}{1.322345in}}%
\pgfpathlineto{\pgfqpoint{1.408668in}{1.322194in}}%
\pgfpathlineto{\pgfqpoint{1.408762in}{1.322368in}}%
\pgfpathlineto{\pgfqpoint{1.408762in}{1.322368in}}%
\pgfpathlineto{\pgfqpoint{1.409000in}{1.322840in}}%
\pgfpathlineto{\pgfqpoint{1.409209in}{1.322250in}}%
\pgfpathlineto{\pgfqpoint{1.409209in}{1.322250in}}%
\pgfpathlineto{\pgfqpoint{1.409302in}{1.322177in}}%
\pgfpathlineto{\pgfqpoint{1.409395in}{1.322326in}}%
\pgfpathlineto{\pgfqpoint{1.409395in}{1.322326in}}%
\pgfpathlineto{\pgfqpoint{1.410315in}{1.322822in}}%
\pgfpathlineto{\pgfqpoint{1.410500in}{1.322253in}}%
\pgfpathlineto{\pgfqpoint{1.410593in}{1.322162in}}%
\pgfpathlineto{\pgfqpoint{1.410686in}{1.322300in}}%
\pgfpathlineto{\pgfqpoint{1.410686in}{1.322300in}}%
\pgfpathlineto{\pgfqpoint{1.411604in}{1.322816in}}%
\pgfpathlineto{\pgfqpoint{1.411796in}{1.322249in}}%
\pgfpathlineto{\pgfqpoint{1.411888in}{1.322149in}}%
\pgfpathlineto{\pgfqpoint{1.411982in}{1.322281in}}%
\pgfpathlineto{\pgfqpoint{1.411982in}{1.322281in}}%
\pgfpathlineto{\pgfqpoint{1.412918in}{1.322797in}}%
\pgfpathlineto{\pgfqpoint{1.413094in}{1.322243in}}%
\pgfpathlineto{\pgfqpoint{1.413187in}{1.322137in}}%
\pgfpathlineto{\pgfqpoint{1.413280in}{1.322264in}}%
\pgfpathlineto{\pgfqpoint{1.413280in}{1.322264in}}%
\pgfpathlineto{\pgfqpoint{1.414218in}{1.322788in}}%
\pgfpathlineto{\pgfqpoint{1.414421in}{1.322175in}}%
\pgfpathlineto{\pgfqpoint{1.414514in}{1.322140in}}%
\pgfpathlineto{\pgfqpoint{1.414608in}{1.322314in}}%
\pgfpathlineto{\pgfqpoint{1.414608in}{1.322314in}}%
\pgfpathlineto{\pgfqpoint{1.414848in}{1.322793in}}%
\pgfpathlineto{\pgfqpoint{1.415081in}{1.322155in}}%
\pgfpathlineto{\pgfqpoint{1.415081in}{1.322155in}}%
\pgfpathlineto{\pgfqpoint{1.415174in}{1.322143in}}%
\pgfpathlineto{\pgfqpoint{1.415269in}{1.322330in}}%
\pgfpathlineto{\pgfqpoint{1.415269in}{1.322330in}}%
\pgfpathlineto{\pgfqpoint{1.415508in}{1.322785in}}%
\pgfpathlineto{\pgfqpoint{1.415699in}{1.322222in}}%
\pgfpathlineto{\pgfqpoint{1.415699in}{1.322222in}}%
\pgfpathlineto{\pgfqpoint{1.415792in}{1.322113in}}%
\pgfpathlineto{\pgfqpoint{1.415886in}{1.322239in}}%
\pgfpathlineto{\pgfqpoint{1.415886in}{1.322239in}}%
\pgfpathlineto{\pgfqpoint{1.416827in}{1.322767in}}%
\pgfpathlineto{\pgfqpoint{1.417004in}{1.322211in}}%
\pgfpathlineto{\pgfqpoint{1.417097in}{1.322101in}}%
\pgfpathlineto{\pgfqpoint{1.417191in}{1.322227in}}%
\pgfpathlineto{\pgfqpoint{1.417191in}{1.322227in}}%
\pgfpathlineto{\pgfqpoint{1.418117in}{1.322765in}}%
\pgfpathlineto{\pgfqpoint{1.418310in}{1.322203in}}%
\pgfpathlineto{\pgfqpoint{1.418403in}{1.322088in}}%
\pgfpathlineto{\pgfqpoint{1.418497in}{1.322212in}}%
\pgfpathlineto{\pgfqpoint{1.418497in}{1.322212in}}%
\pgfpathlineto{\pgfqpoint{1.419441in}{1.322747in}}%
\pgfpathlineto{\pgfqpoint{1.419645in}{1.322130in}}%
\pgfpathlineto{\pgfqpoint{1.419739in}{1.322091in}}%
\pgfpathlineto{\pgfqpoint{1.419833in}{1.322264in}}%
\pgfpathlineto{\pgfqpoint{1.419833in}{1.322264in}}%
\pgfpathlineto{\pgfqpoint{1.420754in}{1.322735in}}%
\pgfpathlineto{\pgfqpoint{1.420927in}{1.322187in}}%
\pgfpathlineto{\pgfqpoint{1.421020in}{1.322064in}}%
\pgfpathlineto{\pgfqpoint{1.421115in}{1.322183in}}%
\pgfpathlineto{\pgfqpoint{1.421115in}{1.322183in}}%
\pgfpathlineto{\pgfqpoint{1.422032in}{1.322734in}}%
\pgfpathlineto{\pgfqpoint{1.422263in}{1.322115in}}%
\pgfpathlineto{\pgfqpoint{1.422357in}{1.322063in}}%
\pgfpathlineto{\pgfqpoint{1.422452in}{1.322229in}}%
\pgfpathlineto{\pgfqpoint{1.422452in}{1.322229in}}%
\pgfpathlineto{\pgfqpoint{1.423376in}{1.322716in}}%
\pgfpathlineto{\pgfqpoint{1.423572in}{1.322114in}}%
\pgfpathlineto{\pgfqpoint{1.423666in}{1.322047in}}%
\pgfpathlineto{\pgfqpoint{1.423761in}{1.322204in}}%
\pgfpathlineto{\pgfqpoint{1.423761in}{1.322204in}}%
\pgfpathlineto{\pgfqpoint{1.424672in}{1.322713in}}%
\pgfpathlineto{\pgfqpoint{1.424870in}{1.322146in}}%
\pgfpathlineto{\pgfqpoint{1.424964in}{1.322028in}}%
\pgfpathlineto{\pgfqpoint{1.425059in}{1.322151in}}%
\pgfpathlineto{\pgfqpoint{1.425059in}{1.322151in}}%
\pgfpathlineto{\pgfqpoint{1.426009in}{1.322694in}}%
\pgfpathlineto{\pgfqpoint{1.426209in}{1.322082in}}%
\pgfpathlineto{\pgfqpoint{1.426303in}{1.322026in}}%
\pgfpathlineto{\pgfqpoint{1.426398in}{1.322191in}}%
\pgfpathlineto{\pgfqpoint{1.426398in}{1.322191in}}%
\pgfpathlineto{\pgfqpoint{1.427328in}{1.322683in}}%
\pgfpathlineto{\pgfqpoint{1.427522in}{1.322082in}}%
\pgfpathlineto{\pgfqpoint{1.427616in}{1.322010in}}%
\pgfpathlineto{\pgfqpoint{1.427712in}{1.322165in}}%
\pgfpathlineto{\pgfqpoint{1.427712in}{1.322165in}}%
\pgfpathlineto{\pgfqpoint{1.428646in}{1.322674in}}%
\pgfpathlineto{\pgfqpoint{1.428844in}{1.322067in}}%
\pgfpathlineto{\pgfqpoint{1.428939in}{1.321998in}}%
\pgfpathlineto{\pgfqpoint{1.429034in}{1.322156in}}%
\pgfpathlineto{\pgfqpoint{1.429034in}{1.322156in}}%
\pgfpathlineto{\pgfqpoint{1.429968in}{1.322664in}}%
\pgfpathlineto{\pgfqpoint{1.430149in}{1.322101in}}%
\pgfpathlineto{\pgfqpoint{1.430243in}{1.321979in}}%
\pgfpathlineto{\pgfqpoint{1.430338in}{1.322102in}}%
\pgfpathlineto{\pgfqpoint{1.430338in}{1.322102in}}%
\pgfpathlineto{\pgfqpoint{1.431297in}{1.322649in}}%
\pgfpathlineto{\pgfqpoint{1.431470in}{1.322098in}}%
\pgfpathlineto{\pgfqpoint{1.431564in}{1.321966in}}%
\pgfpathlineto{\pgfqpoint{1.431708in}{1.322200in}}%
\pgfpathlineto{\pgfqpoint{1.431708in}{1.322200in}}%
\pgfpathlineto{\pgfqpoint{1.431952in}{1.322649in}}%
\pgfpathlineto{\pgfqpoint{1.432145in}{1.322059in}}%
\pgfpathlineto{\pgfqpoint{1.432145in}{1.322059in}}%
\pgfpathlineto{\pgfqpoint{1.432239in}{1.321963in}}%
\pgfpathlineto{\pgfqpoint{1.432335in}{1.322104in}}%
\pgfpathlineto{\pgfqpoint{1.432335in}{1.322104in}}%
\pgfpathlineto{\pgfqpoint{1.433281in}{1.322637in}}%
\pgfpathlineto{\pgfqpoint{1.433463in}{1.322071in}}%
\pgfpathlineto{\pgfqpoint{1.433557in}{1.321948in}}%
\pgfpathlineto{\pgfqpoint{1.433653in}{1.322071in}}%
\pgfpathlineto{\pgfqpoint{1.433653in}{1.322071in}}%
\pgfpathlineto{\pgfqpoint{1.434604in}{1.322629in}}%
\pgfpathlineto{\pgfqpoint{1.434815in}{1.322000in}}%
\pgfpathlineto{\pgfqpoint{1.434910in}{1.321947in}}%
\pgfpathlineto{\pgfqpoint{1.435006in}{1.322117in}}%
\pgfpathlineto{\pgfqpoint{1.435006in}{1.322117in}}%
\pgfpathlineto{\pgfqpoint{1.435941in}{1.322614in}}%
\pgfpathlineto{\pgfqpoint{1.436122in}{1.322045in}}%
\pgfpathlineto{\pgfqpoint{1.436217in}{1.321923in}}%
\pgfpathlineto{\pgfqpoint{1.436313in}{1.322048in}}%
\pgfpathlineto{\pgfqpoint{1.436313in}{1.322048in}}%
\pgfpathlineto{\pgfqpoint{1.437274in}{1.322602in}}%
\pgfpathlineto{\pgfqpoint{1.437457in}{1.322026in}}%
\pgfpathlineto{\pgfqpoint{1.437552in}{1.321912in}}%
\pgfpathlineto{\pgfqpoint{1.437648in}{1.322042in}}%
\pgfpathlineto{\pgfqpoint{1.437648in}{1.322042in}}%
\pgfpathlineto{\pgfqpoint{1.438607in}{1.322592in}}%
\pgfpathlineto{\pgfqpoint{1.438790in}{1.322016in}}%
\pgfpathlineto{\pgfqpoint{1.438885in}{1.321899in}}%
\pgfpathlineto{\pgfqpoint{1.438981in}{1.322028in}}%
\pgfpathlineto{\pgfqpoint{1.438981in}{1.322028in}}%
\pgfpathlineto{\pgfqpoint{1.439929in}{1.322588in}}%
\pgfpathlineto{\pgfqpoint{1.440119in}{1.322023in}}%
\pgfpathlineto{\pgfqpoint{1.440214in}{1.321886in}}%
\pgfpathlineto{\pgfqpoint{1.440358in}{1.322120in}}%
\pgfpathlineto{\pgfqpoint{1.440358in}{1.322120in}}%
\pgfpathlineto{\pgfqpoint{1.440604in}{1.322579in}}%
\pgfpathlineto{\pgfqpoint{1.440791in}{1.322006in}}%
\pgfpathlineto{\pgfqpoint{1.440791in}{1.322006in}}%
\pgfpathlineto{\pgfqpoint{1.440886in}{1.321880in}}%
\pgfpathlineto{\pgfqpoint{1.440983in}{1.322004in}}%
\pgfpathlineto{\pgfqpoint{1.440983in}{1.322004in}}%
\pgfpathlineto{\pgfqpoint{1.441948in}{1.322565in}}%
\pgfpathlineto{\pgfqpoint{1.442125in}{1.322006in}}%
\pgfpathlineto{\pgfqpoint{1.442221in}{1.321867in}}%
\pgfpathlineto{\pgfqpoint{1.442365in}{1.322101in}}%
\pgfpathlineto{\pgfqpoint{1.442365in}{1.322101in}}%
\pgfpathlineto{\pgfqpoint{1.442612in}{1.322563in}}%
\pgfpathlineto{\pgfqpoint{1.442800in}{1.321984in}}%
\pgfpathlineto{\pgfqpoint{1.442800in}{1.321984in}}%
\pgfpathlineto{\pgfqpoint{1.442896in}{1.321861in}}%
\pgfpathlineto{\pgfqpoint{1.442992in}{1.321988in}}%
\pgfpathlineto{\pgfqpoint{1.442992in}{1.321988in}}%
\pgfpathlineto{\pgfqpoint{1.443961in}{1.322547in}}%
\pgfpathlineto{\pgfqpoint{1.444153in}{1.321939in}}%
\pgfpathlineto{\pgfqpoint{1.444249in}{1.321853in}}%
\pgfpathlineto{\pgfqpoint{1.444346in}{1.322004in}}%
\pgfpathlineto{\pgfqpoint{1.444346in}{1.322004in}}%
\pgfpathlineto{\pgfqpoint{1.445309in}{1.322531in}}%
\pgfpathlineto{\pgfqpoint{1.445485in}{1.321956in}}%
\pgfpathlineto{\pgfqpoint{1.445581in}{1.321836in}}%
\pgfpathlineto{\pgfqpoint{1.445677in}{1.321966in}}%
\pgfpathlineto{\pgfqpoint{1.445677in}{1.321966in}}%
\pgfpathlineto{\pgfqpoint{1.446645in}{1.322527in}}%
\pgfpathlineto{\pgfqpoint{1.446843in}{1.321909in}}%
\pgfpathlineto{\pgfqpoint{1.446939in}{1.321829in}}%
\pgfpathlineto{\pgfqpoint{1.447036in}{1.321985in}}%
\pgfpathlineto{\pgfqpoint{1.447036in}{1.321985in}}%
\pgfpathlineto{\pgfqpoint{1.447974in}{1.322523in}}%
\pgfpathlineto{\pgfqpoint{1.448176in}{1.321932in}}%
\pgfpathlineto{\pgfqpoint{1.448272in}{1.321811in}}%
\pgfpathlineto{\pgfqpoint{1.448368in}{1.321940in}}%
\pgfpathlineto{\pgfqpoint{1.448368in}{1.321940in}}%
\pgfpathlineto{\pgfqpoint{1.449311in}{1.322512in}}%
\pgfpathlineto{\pgfqpoint{1.449513in}{1.321954in}}%
\pgfpathlineto{\pgfqpoint{1.449609in}{1.321798in}}%
\pgfpathlineto{\pgfqpoint{1.449754in}{1.322021in}}%
\pgfpathlineto{\pgfqpoint{1.449754in}{1.322021in}}%
\pgfpathlineto{\pgfqpoint{1.450003in}{1.322505in}}%
\pgfpathlineto{\pgfqpoint{1.450202in}{1.321905in}}%
\pgfpathlineto{\pgfqpoint{1.450202in}{1.321905in}}%
\pgfpathlineto{\pgfqpoint{1.450298in}{1.321793in}}%
\pgfpathlineto{\pgfqpoint{1.450395in}{1.321929in}}%
\pgfpathlineto{\pgfqpoint{1.450395in}{1.321929in}}%
\pgfpathlineto{\pgfqpoint{1.451366in}{1.322487in}}%
\pgfpathlineto{\pgfqpoint{1.451542in}{1.321923in}}%
\pgfpathlineto{\pgfqpoint{1.451638in}{1.321778in}}%
\pgfpathlineto{\pgfqpoint{1.451785in}{1.322014in}}%
\pgfpathlineto{\pgfqpoint{1.451785in}{1.322014in}}%
\pgfpathlineto{\pgfqpoint{1.452033in}{1.322486in}}%
\pgfpathlineto{\pgfqpoint{1.452224in}{1.321899in}}%
\pgfpathlineto{\pgfqpoint{1.452224in}{1.321899in}}%
\pgfpathlineto{\pgfqpoint{1.452320in}{1.321773in}}%
\pgfpathlineto{\pgfqpoint{1.452418in}{1.321900in}}%
\pgfpathlineto{\pgfqpoint{1.452418in}{1.321900in}}%
\pgfpathlineto{\pgfqpoint{1.453394in}{1.322470in}}%
\pgfpathlineto{\pgfqpoint{1.453553in}{1.321971in}}%
\pgfpathlineto{\pgfqpoint{1.453697in}{1.321771in}}%
\pgfpathlineto{\pgfqpoint{1.453795in}{1.321944in}}%
\pgfpathlineto{\pgfqpoint{1.453795in}{1.321944in}}%
\pgfpathlineto{\pgfqpoint{1.454740in}{1.322465in}}%
\pgfpathlineto{\pgfqpoint{1.454911in}{1.321946in}}%
\pgfpathlineto{\pgfqpoint{1.455056in}{1.321761in}}%
\pgfpathlineto{\pgfqpoint{1.455154in}{1.321940in}}%
\pgfpathlineto{\pgfqpoint{1.455154in}{1.321940in}}%
\pgfpathlineto{\pgfqpoint{1.456110in}{1.322446in}}%
\pgfpathlineto{\pgfqpoint{1.456292in}{1.321856in}}%
\pgfpathlineto{\pgfqpoint{1.456389in}{1.321735in}}%
\pgfpathlineto{\pgfqpoint{1.456487in}{1.321867in}}%
\pgfpathlineto{\pgfqpoint{1.456487in}{1.321867in}}%
\pgfpathlineto{\pgfqpoint{1.457464in}{1.322438in}}%
\pgfpathlineto{\pgfqpoint{1.457650in}{1.321849in}}%
\pgfpathlineto{\pgfqpoint{1.457746in}{1.321722in}}%
\pgfpathlineto{\pgfqpoint{1.457844in}{1.321851in}}%
\pgfpathlineto{\pgfqpoint{1.457844in}{1.321851in}}%
\pgfpathlineto{\pgfqpoint{1.458826in}{1.322426in}}%
\pgfpathlineto{\pgfqpoint{1.459012in}{1.321831in}}%
\pgfpathlineto{\pgfqpoint{1.459109in}{1.321709in}}%
\pgfpathlineto{\pgfqpoint{1.459207in}{1.321842in}}%
\pgfpathlineto{\pgfqpoint{1.459207in}{1.321842in}}%
\pgfpathlineto{\pgfqpoint{1.460172in}{1.322422in}}%
\pgfpathlineto{\pgfqpoint{1.460374in}{1.321820in}}%
\pgfpathlineto{\pgfqpoint{1.460472in}{1.321696in}}%
\pgfpathlineto{\pgfqpoint{1.460570in}{1.321828in}}%
\pgfpathlineto{\pgfqpoint{1.460570in}{1.321828in}}%
\pgfpathlineto{\pgfqpoint{1.461521in}{1.322411in}}%
\pgfpathlineto{\pgfqpoint{1.461743in}{1.321795in}}%
\pgfpathlineto{\pgfqpoint{1.461841in}{1.321684in}}%
\pgfpathlineto{\pgfqpoint{1.461939in}{1.321826in}}%
\pgfpathlineto{\pgfqpoint{1.461939in}{1.321826in}}%
\pgfpathlineto{\pgfqpoint{1.462924in}{1.322388in}}%
\pgfpathlineto{\pgfqpoint{1.463088in}{1.321850in}}%
\pgfpathlineto{\pgfqpoint{1.463185in}{1.321671in}}%
\pgfpathlineto{\pgfqpoint{1.463333in}{1.321882in}}%
\pgfpathlineto{\pgfqpoint{1.463333in}{1.321882in}}%
\pgfpathlineto{\pgfqpoint{1.463585in}{1.322394in}}%
\pgfpathlineto{\pgfqpoint{1.463790in}{1.321786in}}%
\pgfpathlineto{\pgfqpoint{1.463790in}{1.321786in}}%
\pgfpathlineto{\pgfqpoint{1.463888in}{1.321664in}}%
\pgfpathlineto{\pgfqpoint{1.463986in}{1.321798in}}%
\pgfpathlineto{\pgfqpoint{1.463986in}{1.321798in}}%
\pgfpathlineto{\pgfqpoint{1.464975in}{1.322373in}}%
\pgfpathlineto{\pgfqpoint{1.465150in}{1.321802in}}%
\pgfpathlineto{\pgfqpoint{1.465248in}{1.321650in}}%
\pgfpathlineto{\pgfqpoint{1.465396in}{1.321887in}}%
\pgfpathlineto{\pgfqpoint{1.465396in}{1.321887in}}%
\pgfpathlineto{\pgfqpoint{1.465648in}{1.322375in}}%
\pgfpathlineto{\pgfqpoint{1.465845in}{1.321767in}}%
\pgfpathlineto{\pgfqpoint{1.465845in}{1.321767in}}%
\pgfpathlineto{\pgfqpoint{1.465943in}{1.321644in}}%
\pgfpathlineto{\pgfqpoint{1.466041in}{1.321779in}}%
\pgfpathlineto{\pgfqpoint{1.466041in}{1.321779in}}%
\pgfpathlineto{\pgfqpoint{1.467032in}{1.322356in}}%
\pgfpathlineto{\pgfqpoint{1.467215in}{1.321760in}}%
\pgfpathlineto{\pgfqpoint{1.467313in}{1.321631in}}%
\pgfpathlineto{\pgfqpoint{1.467411in}{1.321762in}}%
\pgfpathlineto{\pgfqpoint{1.467411in}{1.321762in}}%
\pgfpathlineto{\pgfqpoint{1.468403in}{1.322347in}}%
\pgfpathlineto{\pgfqpoint{1.468591in}{1.321743in}}%
\pgfpathlineto{\pgfqpoint{1.468688in}{1.321618in}}%
\pgfpathlineto{\pgfqpoint{1.468787in}{1.321753in}}%
\pgfpathlineto{\pgfqpoint{1.468787in}{1.321753in}}%
\pgfpathlineto{\pgfqpoint{1.469771in}{1.322339in}}%
\pgfpathlineto{\pgfqpoint{1.469967in}{1.321730in}}%
\pgfpathlineto{\pgfqpoint{1.470065in}{1.321605in}}%
\pgfpathlineto{\pgfqpoint{1.470164in}{1.321740in}}%
\pgfpathlineto{\pgfqpoint{1.470164in}{1.321740in}}%
\pgfpathlineto{\pgfqpoint{1.471121in}{1.322331in}}%
\pgfpathlineto{\pgfqpoint{1.471342in}{1.321723in}}%
\pgfpathlineto{\pgfqpoint{1.471440in}{1.321591in}}%
\pgfpathlineto{\pgfqpoint{1.471540in}{1.321723in}}%
\pgfpathlineto{\pgfqpoint{1.471540in}{1.321723in}}%
\pgfpathlineto{\pgfqpoint{1.472534in}{1.322313in}}%
\pgfpathlineto{\pgfqpoint{1.472724in}{1.321703in}}%
\pgfpathlineto{\pgfqpoint{1.472823in}{1.321579in}}%
\pgfpathlineto{\pgfqpoint{1.472922in}{1.321715in}}%
\pgfpathlineto{\pgfqpoint{1.472922in}{1.321715in}}%
\pgfpathlineto{\pgfqpoint{1.473886in}{1.322309in}}%
\pgfpathlineto{\pgfqpoint{1.474080in}{1.321779in}}%
\pgfpathlineto{\pgfqpoint{1.474227in}{1.321578in}}%
\pgfpathlineto{\pgfqpoint{1.474327in}{1.321759in}}%
\pgfpathlineto{\pgfqpoint{1.474327in}{1.321759in}}%
\pgfpathlineto{\pgfqpoint{1.475301in}{1.322288in}}%
\pgfpathlineto{\pgfqpoint{1.475463in}{1.321767in}}%
\pgfpathlineto{\pgfqpoint{1.475611in}{1.321564in}}%
\pgfpathlineto{\pgfqpoint{1.475710in}{1.321745in}}%
\pgfpathlineto{\pgfqpoint{1.475710in}{1.321745in}}%
\pgfpathlineto{\pgfqpoint{1.476683in}{1.322279in}}%
\pgfpathlineto{\pgfqpoint{1.476825in}{1.321851in}}%
\pgfpathlineto{\pgfqpoint{1.476973in}{1.321539in}}%
\pgfpathlineto{\pgfqpoint{1.477123in}{1.321806in}}%
\pgfpathlineto{\pgfqpoint{1.477123in}{1.321806in}}%
\pgfpathlineto{\pgfqpoint{1.477379in}{1.322272in}}%
\pgfpathlineto{\pgfqpoint{1.477548in}{1.321724in}}%
\pgfpathlineto{\pgfqpoint{1.477548in}{1.321724in}}%
\pgfpathlineto{\pgfqpoint{1.477696in}{1.321550in}}%
\pgfpathlineto{\pgfqpoint{1.477797in}{1.321743in}}%
\pgfpathlineto{\pgfqpoint{1.477797in}{1.321743in}}%
\pgfpathlineto{\pgfqpoint{1.478052in}{1.322275in}}%
\pgfpathlineto{\pgfqpoint{1.478260in}{1.321657in}}%
\pgfpathlineto{\pgfqpoint{1.478260in}{1.321657in}}%
\pgfpathlineto{\pgfqpoint{1.478359in}{1.321525in}}%
\pgfpathlineto{\pgfqpoint{1.478459in}{1.321658in}}%
\pgfpathlineto{\pgfqpoint{1.478459in}{1.321658in}}%
\pgfpathlineto{\pgfqpoint{1.479454in}{1.322259in}}%
\pgfpathlineto{\pgfqpoint{1.479611in}{1.321785in}}%
\pgfpathlineto{\pgfqpoint{1.479759in}{1.321515in}}%
\pgfpathlineto{\pgfqpoint{1.479910in}{1.321807in}}%
\pgfpathlineto{\pgfqpoint{1.479910in}{1.321807in}}%
\pgfpathlineto{\pgfqpoint{1.480166in}{1.322241in}}%
\pgfpathlineto{\pgfqpoint{1.480300in}{1.321808in}}%
\pgfpathlineto{\pgfqpoint{1.480300in}{1.321808in}}%
\pgfpathlineto{\pgfqpoint{1.480448in}{1.321506in}}%
\pgfpathlineto{\pgfqpoint{1.480599in}{1.321781in}}%
\pgfpathlineto{\pgfqpoint{1.480599in}{1.321781in}}%
\pgfpathlineto{\pgfqpoint{1.480855in}{1.322241in}}%
\pgfpathlineto{\pgfqpoint{1.481001in}{1.321777in}}%
\pgfpathlineto{\pgfqpoint{1.481001in}{1.321777in}}%
\pgfpathlineto{\pgfqpoint{1.481150in}{1.321502in}}%
\pgfpathlineto{\pgfqpoint{1.481301in}{1.321791in}}%
\pgfpathlineto{\pgfqpoint{1.481301in}{1.321791in}}%
\pgfpathlineto{\pgfqpoint{1.481557in}{1.322230in}}%
\pgfpathlineto{\pgfqpoint{1.481688in}{1.321814in}}%
\pgfpathlineto{\pgfqpoint{1.481688in}{1.321814in}}%
\pgfpathlineto{\pgfqpoint{1.481837in}{1.321492in}}%
\pgfpathlineto{\pgfqpoint{1.482039in}{1.321905in}}%
\pgfpathlineto{\pgfqpoint{1.482039in}{1.321905in}}%
\pgfpathlineto{\pgfqpoint{1.482244in}{1.322232in}}%
\pgfpathlineto{\pgfqpoint{1.482386in}{1.321801in}}%
\pgfpathlineto{\pgfqpoint{1.482386in}{1.321801in}}%
\pgfpathlineto{\pgfqpoint{1.482535in}{1.321486in}}%
\pgfpathlineto{\pgfqpoint{1.482686in}{1.321755in}}%
\pgfpathlineto{\pgfqpoint{1.482686in}{1.321755in}}%
\pgfpathlineto{\pgfqpoint{1.482943in}{1.322225in}}%
\pgfpathlineto{\pgfqpoint{1.483134in}{1.321603in}}%
\pgfpathlineto{\pgfqpoint{1.483134in}{1.321603in}}%
\pgfpathlineto{\pgfqpoint{1.483234in}{1.321479in}}%
\pgfpathlineto{\pgfqpoint{1.483334in}{1.321620in}}%
\pgfpathlineto{\pgfqpoint{1.483334in}{1.321620in}}%
\pgfpathlineto{\pgfqpoint{1.484303in}{1.322221in}}%
\pgfpathlineto{\pgfqpoint{1.484480in}{1.321779in}}%
\pgfpathlineto{\pgfqpoint{1.484629in}{1.321466in}}%
\pgfpathlineto{\pgfqpoint{1.484781in}{1.321737in}}%
\pgfpathlineto{\pgfqpoint{1.484781in}{1.321737in}}%
\pgfpathlineto{\pgfqpoint{1.485038in}{1.322207in}}%
\pgfpathlineto{\pgfqpoint{1.485195in}{1.321701in}}%
\pgfpathlineto{\pgfqpoint{1.485195in}{1.321701in}}%
\pgfpathlineto{\pgfqpoint{1.485345in}{1.321466in}}%
\pgfpathlineto{\pgfqpoint{1.485496in}{1.321779in}}%
\pgfpathlineto{\pgfqpoint{1.485496in}{1.321779in}}%
\pgfpathlineto{\pgfqpoint{1.485703in}{1.322210in}}%
\pgfpathlineto{\pgfqpoint{1.485882in}{1.321751in}}%
\pgfpathlineto{\pgfqpoint{1.485882in}{1.321751in}}%
\pgfpathlineto{\pgfqpoint{1.486031in}{1.321453in}}%
\pgfpathlineto{\pgfqpoint{1.486182in}{1.321734in}}%
\pgfpathlineto{\pgfqpoint{1.486182in}{1.321734in}}%
\pgfpathlineto{\pgfqpoint{1.486440in}{1.322193in}}%
\pgfpathlineto{\pgfqpoint{1.486579in}{1.321756in}}%
\pgfpathlineto{\pgfqpoint{1.486579in}{1.321756in}}%
\pgfpathlineto{\pgfqpoint{1.486728in}{1.321446in}}%
\pgfpathlineto{\pgfqpoint{1.486880in}{1.321720in}}%
\pgfpathlineto{\pgfqpoint{1.486880in}{1.321720in}}%
\pgfpathlineto{\pgfqpoint{1.487138in}{1.322189in}}%
\pgfpathlineto{\pgfqpoint{1.487325in}{1.321574in}}%
\pgfpathlineto{\pgfqpoint{1.487325in}{1.321574in}}%
\pgfpathlineto{\pgfqpoint{1.487425in}{1.321438in}}%
\pgfpathlineto{\pgfqpoint{1.487526in}{1.321572in}}%
\pgfpathlineto{\pgfqpoint{1.487526in}{1.321572in}}%
\pgfpathlineto{\pgfqpoint{1.488526in}{1.322185in}}%
\pgfpathlineto{\pgfqpoint{1.488677in}{1.321758in}}%
\pgfpathlineto{\pgfqpoint{1.488824in}{1.321424in}}%
\pgfpathlineto{\pgfqpoint{1.489027in}{1.321831in}}%
\pgfpathlineto{\pgfqpoint{1.489027in}{1.321831in}}%
\pgfpathlineto{\pgfqpoint{1.489234in}{1.322176in}}%
\pgfpathlineto{\pgfqpoint{1.489388in}{1.321706in}}%
\pgfpathlineto{\pgfqpoint{1.489388in}{1.321706in}}%
\pgfpathlineto{\pgfqpoint{1.489538in}{1.321420in}}%
\pgfpathlineto{\pgfqpoint{1.489690in}{1.321710in}}%
\pgfpathlineto{\pgfqpoint{1.489690in}{1.321710in}}%
\pgfpathlineto{\pgfqpoint{1.489948in}{1.322161in}}%
\pgfpathlineto{\pgfqpoint{1.490097in}{1.321670in}}%
\pgfpathlineto{\pgfqpoint{1.490097in}{1.321670in}}%
\pgfpathlineto{\pgfqpoint{1.490247in}{1.321417in}}%
\pgfpathlineto{\pgfqpoint{1.490400in}{1.321724in}}%
\pgfpathlineto{\pgfqpoint{1.490400in}{1.321724in}}%
\pgfpathlineto{\pgfqpoint{1.490607in}{1.322168in}}%
\pgfpathlineto{\pgfqpoint{1.490785in}{1.321729in}}%
\pgfpathlineto{\pgfqpoint{1.490785in}{1.321729in}}%
\pgfpathlineto{\pgfqpoint{1.490935in}{1.321404in}}%
\pgfpathlineto{\pgfqpoint{1.491088in}{1.321675in}}%
\pgfpathlineto{\pgfqpoint{1.491088in}{1.321675in}}%
\pgfpathlineto{\pgfqpoint{1.491346in}{1.322155in}}%
\pgfpathlineto{\pgfqpoint{1.491540in}{1.321525in}}%
\pgfpathlineto{\pgfqpoint{1.491540in}{1.321525in}}%
\pgfpathlineto{\pgfqpoint{1.491640in}{1.321398in}}%
\pgfpathlineto{\pgfqpoint{1.491742in}{1.321539in}}%
\pgfpathlineto{\pgfqpoint{1.491742in}{1.321539in}}%
\pgfpathlineto{\pgfqpoint{1.492749in}{1.322147in}}%
\pgfpathlineto{\pgfqpoint{1.492897in}{1.321706in}}%
\pgfpathlineto{\pgfqpoint{1.493048in}{1.321384in}}%
\pgfpathlineto{\pgfqpoint{1.493200in}{1.321658in}}%
\pgfpathlineto{\pgfqpoint{1.493200in}{1.321658in}}%
\pgfpathlineto{\pgfqpoint{1.493460in}{1.322137in}}%
\pgfpathlineto{\pgfqpoint{1.493652in}{1.321509in}}%
\pgfpathlineto{\pgfqpoint{1.493652in}{1.321509in}}%
\pgfpathlineto{\pgfqpoint{1.493752in}{1.321377in}}%
\pgfpathlineto{\pgfqpoint{1.493854in}{1.321516in}}%
\pgfpathlineto{\pgfqpoint{1.493854in}{1.321516in}}%
\pgfpathlineto{\pgfqpoint{1.494864in}{1.322129in}}%
\pgfpathlineto{\pgfqpoint{1.495014in}{1.321683in}}%
\pgfpathlineto{\pgfqpoint{1.495164in}{1.321364in}}%
\pgfpathlineto{\pgfqpoint{1.495317in}{1.321640in}}%
\pgfpathlineto{\pgfqpoint{1.495317in}{1.321640in}}%
\pgfpathlineto{\pgfqpoint{1.495577in}{1.322119in}}%
\pgfpathlineto{\pgfqpoint{1.495721in}{1.321670in}}%
\pgfpathlineto{\pgfqpoint{1.495721in}{1.321670in}}%
\pgfpathlineto{\pgfqpoint{1.495872in}{1.321357in}}%
\pgfpathlineto{\pgfqpoint{1.496025in}{1.321638in}}%
\pgfpathlineto{\pgfqpoint{1.496025in}{1.321638in}}%
\pgfpathlineto{\pgfqpoint{1.496285in}{1.322111in}}%
\pgfpathlineto{\pgfqpoint{1.496438in}{1.321622in}}%
\pgfpathlineto{\pgfqpoint{1.496438in}{1.321622in}}%
\pgfpathlineto{\pgfqpoint{1.496588in}{1.321354in}}%
\pgfpathlineto{\pgfqpoint{1.496742in}{1.321659in}}%
\pgfpathlineto{\pgfqpoint{1.496742in}{1.321659in}}%
\pgfpathlineto{\pgfqpoint{1.497001in}{1.322097in}}%
\pgfpathlineto{\pgfqpoint{1.497131in}{1.321677in}}%
\pgfpathlineto{\pgfqpoint{1.497131in}{1.321677in}}%
\pgfpathlineto{\pgfqpoint{1.497282in}{1.321343in}}%
\pgfpathlineto{\pgfqpoint{1.497487in}{1.321764in}}%
\pgfpathlineto{\pgfqpoint{1.497487in}{1.321764in}}%
\pgfpathlineto{\pgfqpoint{1.497696in}{1.322103in}}%
\pgfpathlineto{\pgfqpoint{1.497833in}{1.321701in}}%
\pgfpathlineto{\pgfqpoint{1.497833in}{1.321701in}}%
\pgfpathlineto{\pgfqpoint{1.498005in}{1.321341in}}%
\pgfpathlineto{\pgfqpoint{1.498158in}{1.321649in}}%
\pgfpathlineto{\pgfqpoint{1.498158in}{1.321649in}}%
\pgfpathlineto{\pgfqpoint{1.498367in}{1.322102in}}%
\pgfpathlineto{\pgfqpoint{1.498578in}{1.321532in}}%
\pgfpathlineto{\pgfqpoint{1.498578in}{1.321532in}}%
\pgfpathlineto{\pgfqpoint{1.498730in}{1.321346in}}%
\pgfpathlineto{\pgfqpoint{1.498832in}{1.321544in}}%
\pgfpathlineto{\pgfqpoint{1.498832in}{1.321544in}}%
\pgfpathlineto{\pgfqpoint{1.499093in}{1.322098in}}%
\pgfpathlineto{\pgfqpoint{1.499306in}{1.321460in}}%
\pgfpathlineto{\pgfqpoint{1.499306in}{1.321460in}}%
\pgfpathlineto{\pgfqpoint{1.499407in}{1.321322in}}%
\pgfpathlineto{\pgfqpoint{1.499509in}{1.321459in}}%
\pgfpathlineto{\pgfqpoint{1.499509in}{1.321459in}}%
\pgfpathlineto{\pgfqpoint{1.500511in}{1.322086in}}%
\pgfpathlineto{\pgfqpoint{1.500675in}{1.321643in}}%
\pgfpathlineto{\pgfqpoint{1.500827in}{1.321308in}}%
\pgfpathlineto{\pgfqpoint{1.501033in}{1.321733in}}%
\pgfpathlineto{\pgfqpoint{1.501033in}{1.321733in}}%
\pgfpathlineto{\pgfqpoint{1.501242in}{1.322073in}}%
\pgfpathlineto{\pgfqpoint{1.501388in}{1.321626in}}%
\pgfpathlineto{\pgfqpoint{1.501388in}{1.321626in}}%
\pgfpathlineto{\pgfqpoint{1.501539in}{1.321302in}}%
\pgfpathlineto{\pgfqpoint{1.501693in}{1.321580in}}%
\pgfpathlineto{\pgfqpoint{1.501693in}{1.321580in}}%
\pgfpathlineto{\pgfqpoint{1.501955in}{1.322065in}}%
\pgfpathlineto{\pgfqpoint{1.502099in}{1.321616in}}%
\pgfpathlineto{\pgfqpoint{1.502099in}{1.321616in}}%
\pgfpathlineto{\pgfqpoint{1.502251in}{1.321295in}}%
\pgfpathlineto{\pgfqpoint{1.502405in}{1.321576in}}%
\pgfpathlineto{\pgfqpoint{1.502405in}{1.321576in}}%
\pgfpathlineto{\pgfqpoint{1.502667in}{1.322058in}}%
\pgfpathlineto{\pgfqpoint{1.502858in}{1.321427in}}%
\pgfpathlineto{\pgfqpoint{1.502858in}{1.321427in}}%
\pgfpathlineto{\pgfqpoint{1.502959in}{1.321287in}}%
\pgfpathlineto{\pgfqpoint{1.503061in}{1.321424in}}%
\pgfpathlineto{\pgfqpoint{1.503061in}{1.321424in}}%
\pgfpathlineto{\pgfqpoint{1.504089in}{1.322048in}}%
\pgfpathlineto{\pgfqpoint{1.504258in}{1.321496in}}%
\pgfpathlineto{\pgfqpoint{1.504410in}{1.321287in}}%
\pgfpathlineto{\pgfqpoint{1.504513in}{1.321479in}}%
\pgfpathlineto{\pgfqpoint{1.504513in}{1.321479in}}%
\pgfpathlineto{\pgfqpoint{1.504776in}{1.322050in}}%
\pgfpathlineto{\pgfqpoint{1.505015in}{1.321352in}}%
\pgfpathlineto{\pgfqpoint{1.505015in}{1.321352in}}%
\pgfpathlineto{\pgfqpoint{1.505117in}{1.321275in}}%
\pgfpathlineto{\pgfqpoint{1.505220in}{1.321455in}}%
\pgfpathlineto{\pgfqpoint{1.505220in}{1.321455in}}%
\pgfpathlineto{\pgfqpoint{1.506227in}{1.322031in}}%
\pgfpathlineto{\pgfqpoint{1.506386in}{1.321527in}}%
\pgfpathlineto{\pgfqpoint{1.506538in}{1.321257in}}%
\pgfpathlineto{\pgfqpoint{1.506693in}{1.321569in}}%
\pgfpathlineto{\pgfqpoint{1.506693in}{1.321569in}}%
\pgfpathlineto{\pgfqpoint{1.506956in}{1.322012in}}%
\pgfpathlineto{\pgfqpoint{1.507120in}{1.321444in}}%
\pgfpathlineto{\pgfqpoint{1.507120in}{1.321444in}}%
\pgfpathlineto{\pgfqpoint{1.507221in}{1.321247in}}%
\pgfpathlineto{\pgfqpoint{1.507376in}{1.321470in}}%
\pgfpathlineto{\pgfqpoint{1.507376in}{1.321470in}}%
\pgfpathlineto{\pgfqpoint{1.507639in}{1.322026in}}%
\pgfpathlineto{\pgfqpoint{1.507854in}{1.321373in}}%
\pgfpathlineto{\pgfqpoint{1.507854in}{1.321373in}}%
\pgfpathlineto{\pgfqpoint{1.507956in}{1.321239in}}%
\pgfpathlineto{\pgfqpoint{1.508059in}{1.321382in}}%
\pgfpathlineto{\pgfqpoint{1.508059in}{1.321382in}}%
\pgfpathlineto{\pgfqpoint{1.509083in}{1.322009in}}%
\pgfpathlineto{\pgfqpoint{1.509235in}{1.321554in}}%
\pgfpathlineto{\pgfqpoint{1.509388in}{1.321225in}}%
\pgfpathlineto{\pgfqpoint{1.509543in}{1.321507in}}%
\pgfpathlineto{\pgfqpoint{1.509543in}{1.321507in}}%
\pgfpathlineto{\pgfqpoint{1.509807in}{1.321999in}}%
\pgfpathlineto{\pgfqpoint{1.510002in}{1.321351in}}%
\pgfpathlineto{\pgfqpoint{1.510002in}{1.321351in}}%
\pgfpathlineto{\pgfqpoint{1.510104in}{1.321218in}}%
\pgfpathlineto{\pgfqpoint{1.510207in}{1.321363in}}%
\pgfpathlineto{\pgfqpoint{1.510207in}{1.321363in}}%
\pgfpathlineto{\pgfqpoint{1.511240in}{1.321987in}}%
\pgfpathlineto{\pgfqpoint{1.511386in}{1.321532in}}%
\pgfpathlineto{\pgfqpoint{1.511539in}{1.321204in}}%
\pgfpathlineto{\pgfqpoint{1.511695in}{1.321488in}}%
\pgfpathlineto{\pgfqpoint{1.511695in}{1.321488in}}%
\pgfpathlineto{\pgfqpoint{1.511959in}{1.321980in}}%
\pgfpathlineto{\pgfqpoint{1.512105in}{1.321524in}}%
\pgfpathlineto{\pgfqpoint{1.512105in}{1.321524in}}%
\pgfpathlineto{\pgfqpoint{1.512258in}{1.321197in}}%
\pgfpathlineto{\pgfqpoint{1.512414in}{1.321482in}}%
\pgfpathlineto{\pgfqpoint{1.512414in}{1.321482in}}%
\pgfpathlineto{\pgfqpoint{1.512678in}{1.321973in}}%
\pgfpathlineto{\pgfqpoint{1.512823in}{1.321520in}}%
\pgfpathlineto{\pgfqpoint{1.512823in}{1.321520in}}%
\pgfpathlineto{\pgfqpoint{1.512976in}{1.321190in}}%
\pgfpathlineto{\pgfqpoint{1.513132in}{1.321474in}}%
\pgfpathlineto{\pgfqpoint{1.513132in}{1.321474in}}%
\pgfpathlineto{\pgfqpoint{1.513396in}{1.321968in}}%
\pgfpathlineto{\pgfqpoint{1.513544in}{1.321502in}}%
\pgfpathlineto{\pgfqpoint{1.513544in}{1.321502in}}%
\pgfpathlineto{\pgfqpoint{1.513698in}{1.321183in}}%
\pgfpathlineto{\pgfqpoint{1.513853in}{1.321474in}}%
\pgfpathlineto{\pgfqpoint{1.513853in}{1.321474in}}%
\pgfpathlineto{\pgfqpoint{1.514118in}{1.321960in}}%
\pgfpathlineto{\pgfqpoint{1.514277in}{1.321437in}}%
\pgfpathlineto{\pgfqpoint{1.514277in}{1.321437in}}%
\pgfpathlineto{\pgfqpoint{1.514431in}{1.321182in}}%
\pgfpathlineto{\pgfqpoint{1.514587in}{1.321507in}}%
\pgfpathlineto{\pgfqpoint{1.514587in}{1.321507in}}%
\pgfpathlineto{\pgfqpoint{1.514799in}{1.321964in}}%
\pgfpathlineto{\pgfqpoint{1.514977in}{1.321524in}}%
\pgfpathlineto{\pgfqpoint{1.514977in}{1.321524in}}%
\pgfpathlineto{\pgfqpoint{1.515121in}{1.321168in}}%
\pgfpathlineto{\pgfqpoint{1.515329in}{1.321566in}}%
\pgfpathlineto{\pgfqpoint{1.515329in}{1.321566in}}%
\pgfpathlineto{\pgfqpoint{1.515542in}{1.321958in}}%
\pgfpathlineto{\pgfqpoint{1.515712in}{1.321448in}}%
\pgfpathlineto{\pgfqpoint{1.515712in}{1.321448in}}%
\pgfpathlineto{\pgfqpoint{1.515866in}{1.321165in}}%
\pgfpathlineto{\pgfqpoint{1.516022in}{1.321476in}}%
\pgfpathlineto{\pgfqpoint{1.516022in}{1.321476in}}%
\pgfpathlineto{\pgfqpoint{1.516287in}{1.321935in}}%
\pgfpathlineto{\pgfqpoint{1.516420in}{1.321503in}}%
\pgfpathlineto{\pgfqpoint{1.516420in}{1.321503in}}%
\pgfpathlineto{\pgfqpoint{1.516572in}{1.321154in}}%
\pgfpathlineto{\pgfqpoint{1.516782in}{1.321582in}}%
\pgfpathlineto{\pgfqpoint{1.516782in}{1.321582in}}%
\pgfpathlineto{\pgfqpoint{1.516994in}{1.321941in}}%
\pgfpathlineto{\pgfqpoint{1.517142in}{1.321494in}}%
\pgfpathlineto{\pgfqpoint{1.517142in}{1.321494in}}%
\pgfpathlineto{\pgfqpoint{1.517296in}{1.321147in}}%
\pgfpathlineto{\pgfqpoint{1.517506in}{1.321582in}}%
\pgfpathlineto{\pgfqpoint{1.517506in}{1.321582in}}%
\pgfpathlineto{\pgfqpoint{1.517719in}{1.321933in}}%
\pgfpathlineto{\pgfqpoint{1.517861in}{1.321504in}}%
\pgfpathlineto{\pgfqpoint{1.517861in}{1.321504in}}%
\pgfpathlineto{\pgfqpoint{1.517998in}{1.321142in}}%
\pgfpathlineto{\pgfqpoint{1.518207in}{1.321512in}}%
\pgfpathlineto{\pgfqpoint{1.518207in}{1.321512in}}%
\pgfpathlineto{\pgfqpoint{1.518420in}{1.321935in}}%
\pgfpathlineto{\pgfqpoint{1.518591in}{1.321461in}}%
\pgfpathlineto{\pgfqpoint{1.518591in}{1.321461in}}%
\pgfpathlineto{\pgfqpoint{1.518746in}{1.321133in}}%
\pgfpathlineto{\pgfqpoint{1.518902in}{1.321423in}}%
\pgfpathlineto{\pgfqpoint{1.518902in}{1.321423in}}%
\pgfpathlineto{\pgfqpoint{1.519169in}{1.321918in}}%
\pgfpathlineto{\pgfqpoint{1.519311in}{1.321471in}}%
\pgfpathlineto{\pgfqpoint{1.519311in}{1.321471in}}%
\pgfpathlineto{\pgfqpoint{1.519465in}{1.321125in}}%
\pgfpathlineto{\pgfqpoint{1.519622in}{1.321406in}}%
\pgfpathlineto{\pgfqpoint{1.519622in}{1.321406in}}%
\pgfpathlineto{\pgfqpoint{1.519889in}{1.321914in}}%
\pgfpathlineto{\pgfqpoint{1.520074in}{1.321302in}}%
\pgfpathlineto{\pgfqpoint{1.520074in}{1.321302in}}%
\pgfpathlineto{\pgfqpoint{1.520177in}{1.321118in}}%
\pgfpathlineto{\pgfqpoint{1.520333in}{1.321362in}}%
\pgfpathlineto{\pgfqpoint{1.520333in}{1.321362in}}%
\pgfpathlineto{\pgfqpoint{1.520600in}{1.321915in}}%
\pgfpathlineto{\pgfqpoint{1.520808in}{1.321262in}}%
\pgfpathlineto{\pgfqpoint{1.520808in}{1.321262in}}%
\pgfpathlineto{\pgfqpoint{1.520911in}{1.321111in}}%
\pgfpathlineto{\pgfqpoint{1.521015in}{1.321247in}}%
\pgfpathlineto{\pgfqpoint{1.521015in}{1.321247in}}%
\pgfpathlineto{\pgfqpoint{1.522074in}{1.321886in}}%
\pgfpathlineto{\pgfqpoint{1.522211in}{1.321439in}}%
\pgfpathlineto{\pgfqpoint{1.522366in}{1.321097in}}%
\pgfpathlineto{\pgfqpoint{1.522523in}{1.321382in}}%
\pgfpathlineto{\pgfqpoint{1.522523in}{1.321382in}}%
\pgfpathlineto{\pgfqpoint{1.522790in}{1.321889in}}%
\pgfpathlineto{\pgfqpoint{1.522987in}{1.321234in}}%
\pgfpathlineto{\pgfqpoint{1.522987in}{1.321234in}}%
\pgfpathlineto{\pgfqpoint{1.523090in}{1.321089in}}%
\pgfpathlineto{\pgfqpoint{1.523195in}{1.321231in}}%
\pgfpathlineto{\pgfqpoint{1.523195in}{1.321231in}}%
\pgfpathlineto{\pgfqpoint{1.524218in}{1.321885in}}%
\pgfpathlineto{\pgfqpoint{1.524388in}{1.321431in}}%
\pgfpathlineto{\pgfqpoint{1.524539in}{1.321075in}}%
\pgfpathlineto{\pgfqpoint{1.524750in}{1.321500in}}%
\pgfpathlineto{\pgfqpoint{1.524750in}{1.321500in}}%
\pgfpathlineto{\pgfqpoint{1.524964in}{1.321874in}}%
\pgfpathlineto{\pgfqpoint{1.525111in}{1.321446in}}%
\pgfpathlineto{\pgfqpoint{1.525111in}{1.321446in}}%
\pgfpathlineto{\pgfqpoint{1.525293in}{1.321077in}}%
\pgfpathlineto{\pgfqpoint{1.525451in}{1.321412in}}%
\pgfpathlineto{\pgfqpoint{1.525451in}{1.321412in}}%
\pgfpathlineto{\pgfqpoint{1.525665in}{1.321872in}}%
\pgfpathlineto{\pgfqpoint{1.525847in}{1.321399in}}%
\pgfpathlineto{\pgfqpoint{1.525847in}{1.321399in}}%
\pgfpathlineto{\pgfqpoint{1.526003in}{1.321061in}}%
\pgfpathlineto{\pgfqpoint{1.526161in}{1.321351in}}%
\pgfpathlineto{\pgfqpoint{1.526161in}{1.321351in}}%
\pgfpathlineto{\pgfqpoint{1.526429in}{1.321857in}}%
\pgfpathlineto{\pgfqpoint{1.526588in}{1.321338in}}%
\pgfpathlineto{\pgfqpoint{1.526588in}{1.321338in}}%
\pgfpathlineto{\pgfqpoint{1.526743in}{1.321058in}}%
\pgfpathlineto{\pgfqpoint{1.526902in}{1.321380in}}%
\pgfpathlineto{\pgfqpoint{1.526902in}{1.321380in}}%
\pgfpathlineto{\pgfqpoint{1.527117in}{1.321858in}}%
\pgfpathlineto{\pgfqpoint{1.527302in}{1.321401in}}%
\pgfpathlineto{\pgfqpoint{1.527302in}{1.321401in}}%
\pgfpathlineto{\pgfqpoint{1.527457in}{1.321046in}}%
\pgfpathlineto{\pgfqpoint{1.527668in}{1.321484in}}%
\pgfpathlineto{\pgfqpoint{1.527668in}{1.321484in}}%
\pgfpathlineto{\pgfqpoint{1.527883in}{1.321847in}}%
\pgfpathlineto{\pgfqpoint{1.528036in}{1.321372in}}%
\pgfpathlineto{\pgfqpoint{1.528036in}{1.321372in}}%
\pgfpathlineto{\pgfqpoint{1.528192in}{1.321040in}}%
\pgfpathlineto{\pgfqpoint{1.528350in}{1.321335in}}%
\pgfpathlineto{\pgfqpoint{1.528350in}{1.321335in}}%
\pgfpathlineto{\pgfqpoint{1.528619in}{1.321837in}}%
\pgfpathlineto{\pgfqpoint{1.528810in}{1.321193in}}%
\pgfpathlineto{\pgfqpoint{1.528810in}{1.321193in}}%
\pgfpathlineto{\pgfqpoint{1.528914in}{1.321031in}}%
\pgfpathlineto{\pgfqpoint{1.529072in}{1.321301in}}%
\pgfpathlineto{\pgfqpoint{1.529072in}{1.321301in}}%
\pgfpathlineto{\pgfqpoint{1.529341in}{1.321837in}}%
\pgfpathlineto{\pgfqpoint{1.529549in}{1.321161in}}%
\pgfpathlineto{\pgfqpoint{1.529549in}{1.321161in}}%
\pgfpathlineto{\pgfqpoint{1.529653in}{1.321025in}}%
\pgfpathlineto{\pgfqpoint{1.529758in}{1.321175in}}%
\pgfpathlineto{\pgfqpoint{1.529758in}{1.321175in}}%
\pgfpathlineto{\pgfqpoint{1.530819in}{1.321813in}}%
\pgfpathlineto{\pgfqpoint{1.530961in}{1.321346in}}%
\pgfpathlineto{\pgfqpoint{1.531117in}{1.321011in}}%
\pgfpathlineto{\pgfqpoint{1.531276in}{1.321306in}}%
\pgfpathlineto{\pgfqpoint{1.531276in}{1.321306in}}%
\pgfpathlineto{\pgfqpoint{1.531545in}{1.321812in}}%
\pgfpathlineto{\pgfqpoint{1.531739in}{1.321156in}}%
\pgfpathlineto{\pgfqpoint{1.531739in}{1.321156in}}%
\pgfpathlineto{\pgfqpoint{1.531843in}{1.321002in}}%
\pgfpathlineto{\pgfqpoint{1.531949in}{1.321141in}}%
\pgfpathlineto{\pgfqpoint{1.531949in}{1.321141in}}%
\pgfpathlineto{\pgfqpoint{1.533023in}{1.321788in}}%
\pgfpathlineto{\pgfqpoint{1.533152in}{1.321366in}}%
\pgfpathlineto{\pgfqpoint{1.533340in}{1.321001in}}%
\pgfpathlineto{\pgfqpoint{1.533499in}{1.321356in}}%
\pgfpathlineto{\pgfqpoint{1.533499in}{1.321356in}}%
\pgfpathlineto{\pgfqpoint{1.533716in}{1.321804in}}%
\pgfpathlineto{\pgfqpoint{1.533889in}{1.321345in}}%
\pgfpathlineto{\pgfqpoint{1.533889in}{1.321345in}}%
\pgfpathlineto{\pgfqpoint{1.534039in}{1.320980in}}%
\pgfpathlineto{\pgfqpoint{1.534252in}{1.321403in}}%
\pgfpathlineto{\pgfqpoint{1.534252in}{1.321403in}}%
\pgfpathlineto{\pgfqpoint{1.534468in}{1.321794in}}%
\pgfpathlineto{\pgfqpoint{1.534627in}{1.321318in}}%
\pgfpathlineto{\pgfqpoint{1.534627in}{1.321318in}}%
\pgfpathlineto{\pgfqpoint{1.534784in}{1.320974in}}%
\pgfpathlineto{\pgfqpoint{1.534943in}{1.321267in}}%
\pgfpathlineto{\pgfqpoint{1.534943in}{1.321267in}}%
\pgfpathlineto{\pgfqpoint{1.535214in}{1.321781in}}%
\pgfpathlineto{\pgfqpoint{1.535372in}{1.321266in}}%
\pgfpathlineto{\pgfqpoint{1.535372in}{1.321266in}}%
\pgfpathlineto{\pgfqpoint{1.535529in}{1.320970in}}%
\pgfpathlineto{\pgfqpoint{1.535689in}{1.321290in}}%
\pgfpathlineto{\pgfqpoint{1.535689in}{1.321290in}}%
\pgfpathlineto{\pgfqpoint{1.535959in}{1.321767in}}%
\pgfpathlineto{\pgfqpoint{1.536097in}{1.321312in}}%
\pgfpathlineto{\pgfqpoint{1.536097in}{1.321312in}}%
\pgfpathlineto{\pgfqpoint{1.536254in}{1.320959in}}%
\pgfpathlineto{\pgfqpoint{1.536413in}{1.321249in}}%
\pgfpathlineto{\pgfqpoint{1.536413in}{1.321249in}}%
\pgfpathlineto{\pgfqpoint{1.536684in}{1.321770in}}%
\pgfpathlineto{\pgfqpoint{1.536846in}{1.321247in}}%
\pgfpathlineto{\pgfqpoint{1.536846in}{1.321247in}}%
\pgfpathlineto{\pgfqpoint{1.537003in}{1.320956in}}%
\pgfpathlineto{\pgfqpoint{1.537163in}{1.321279in}}%
\pgfpathlineto{\pgfqpoint{1.537163in}{1.321279in}}%
\pgfpathlineto{\pgfqpoint{1.537433in}{1.321753in}}%
\pgfpathlineto{\pgfqpoint{1.537577in}{1.321263in}}%
\pgfpathlineto{\pgfqpoint{1.537577in}{1.321263in}}%
\pgfpathlineto{\pgfqpoint{1.537735in}{1.320946in}}%
\pgfpathlineto{\pgfqpoint{1.537895in}{1.321257in}}%
\pgfpathlineto{\pgfqpoint{1.537895in}{1.321257in}}%
\pgfpathlineto{\pgfqpoint{1.538166in}{1.321751in}}%
\pgfpathlineto{\pgfqpoint{1.538309in}{1.321286in}}%
\pgfpathlineto{\pgfqpoint{1.538309in}{1.321286in}}%
\pgfpathlineto{\pgfqpoint{1.538466in}{1.320937in}}%
\pgfpathlineto{\pgfqpoint{1.538626in}{1.321231in}}%
\pgfpathlineto{\pgfqpoint{1.538626in}{1.321231in}}%
\pgfpathlineto{\pgfqpoint{1.538898in}{1.321750in}}%
\pgfpathlineto{\pgfqpoint{1.539050in}{1.321266in}}%
\pgfpathlineto{\pgfqpoint{1.539050in}{1.321266in}}%
\pgfpathlineto{\pgfqpoint{1.539207in}{1.320930in}}%
\pgfpathlineto{\pgfqpoint{1.539367in}{1.321232in}}%
\pgfpathlineto{\pgfqpoint{1.539367in}{1.321232in}}%
\pgfpathlineto{\pgfqpoint{1.539639in}{1.321741in}}%
\pgfpathlineto{\pgfqpoint{1.539802in}{1.321194in}}%
\pgfpathlineto{\pgfqpoint{1.539802in}{1.321194in}}%
\pgfpathlineto{\pgfqpoint{1.539960in}{1.320929in}}%
\pgfpathlineto{\pgfqpoint{1.540121in}{1.321269in}}%
\pgfpathlineto{\pgfqpoint{1.540121in}{1.321269in}}%
\pgfpathlineto{\pgfqpoint{1.540339in}{1.321746in}}%
\pgfpathlineto{\pgfqpoint{1.540521in}{1.321284in}}%
\pgfpathlineto{\pgfqpoint{1.540521in}{1.321284in}}%
\pgfpathlineto{\pgfqpoint{1.540671in}{1.320914in}}%
\pgfpathlineto{\pgfqpoint{1.540885in}{1.321338in}}%
\pgfpathlineto{\pgfqpoint{1.540885in}{1.321338in}}%
\pgfpathlineto{\pgfqpoint{1.541103in}{1.321738in}}%
\pgfpathlineto{\pgfqpoint{1.541257in}{1.321293in}}%
\pgfpathlineto{\pgfqpoint{1.541257in}{1.321293in}}%
\pgfpathlineto{\pgfqpoint{1.541444in}{1.320918in}}%
\pgfpathlineto{\pgfqpoint{1.541605in}{1.321270in}}%
\pgfpathlineto{\pgfqpoint{1.541605in}{1.321270in}}%
\pgfpathlineto{\pgfqpoint{1.541823in}{1.321734in}}%
\pgfpathlineto{\pgfqpoint{1.541999in}{1.321278in}}%
\pgfpathlineto{\pgfqpoint{1.541999in}{1.321278in}}%
\pgfpathlineto{\pgfqpoint{1.542141in}{1.320901in}}%
\pgfpathlineto{\pgfqpoint{1.542355in}{1.321293in}}%
\pgfpathlineto{\pgfqpoint{1.542355in}{1.321293in}}%
\pgfpathlineto{\pgfqpoint{1.542574in}{1.321728in}}%
\pgfpathlineto{\pgfqpoint{1.542741in}{1.321265in}}%
\pgfpathlineto{\pgfqpoint{1.542741in}{1.321265in}}%
\pgfpathlineto{\pgfqpoint{1.542889in}{1.320892in}}%
\pgfpathlineto{\pgfqpoint{1.543104in}{1.321308in}}%
\pgfpathlineto{\pgfqpoint{1.543104in}{1.321308in}}%
\pgfpathlineto{\pgfqpoint{1.543322in}{1.321720in}}%
\pgfpathlineto{\pgfqpoint{1.543479in}{1.321274in}}%
\pgfpathlineto{\pgfqpoint{1.543479in}{1.321274in}}%
\pgfpathlineto{\pgfqpoint{1.543665in}{1.320894in}}%
\pgfpathlineto{\pgfqpoint{1.543826in}{1.321242in}}%
\pgfpathlineto{\pgfqpoint{1.543826in}{1.321242in}}%
\pgfpathlineto{\pgfqpoint{1.544045in}{1.321714in}}%
\pgfpathlineto{\pgfqpoint{1.544224in}{1.321251in}}%
\pgfpathlineto{\pgfqpoint{1.544224in}{1.321251in}}%
\pgfpathlineto{\pgfqpoint{1.544373in}{1.320877in}}%
\pgfpathlineto{\pgfqpoint{1.544588in}{1.321295in}}%
\pgfpathlineto{\pgfqpoint{1.544588in}{1.321295in}}%
\pgfpathlineto{\pgfqpoint{1.544807in}{1.321707in}}%
\pgfpathlineto{\pgfqpoint{1.544984in}{1.321163in}}%
\pgfpathlineto{\pgfqpoint{1.544984in}{1.321163in}}%
\pgfpathlineto{\pgfqpoint{1.545143in}{1.320875in}}%
\pgfpathlineto{\pgfqpoint{1.545304in}{1.321206in}}%
\pgfpathlineto{\pgfqpoint{1.545304in}{1.321206in}}%
\pgfpathlineto{\pgfqpoint{1.545523in}{1.321699in}}%
\pgfpathlineto{\pgfqpoint{1.545715in}{1.321211in}}%
\pgfpathlineto{\pgfqpoint{1.545715in}{1.321211in}}%
\pgfpathlineto{\pgfqpoint{1.545874in}{1.320863in}}%
\pgfpathlineto{\pgfqpoint{1.546035in}{1.321163in}}%
\pgfpathlineto{\pgfqpoint{1.546035in}{1.321163in}}%
\pgfpathlineto{\pgfqpoint{1.546309in}{1.321685in}}%
\pgfpathlineto{\pgfqpoint{1.546457in}{1.321212in}}%
\pgfpathlineto{\pgfqpoint{1.546457in}{1.321212in}}%
\pgfpathlineto{\pgfqpoint{1.546616in}{1.320855in}}%
\pgfpathlineto{\pgfqpoint{1.546777in}{1.321151in}}%
\pgfpathlineto{\pgfqpoint{1.546777in}{1.321151in}}%
\pgfpathlineto{\pgfqpoint{1.547051in}{1.321680in}}%
\pgfpathlineto{\pgfqpoint{1.547253in}{1.320997in}}%
\pgfpathlineto{\pgfqpoint{1.547253in}{1.320997in}}%
\pgfpathlineto{\pgfqpoint{1.547359in}{1.320847in}}%
\pgfpathlineto{\pgfqpoint{1.547466in}{1.320995in}}%
\pgfpathlineto{\pgfqpoint{1.547466in}{1.320995in}}%
\pgfpathlineto{\pgfqpoint{1.548543in}{1.321666in}}%
\pgfpathlineto{\pgfqpoint{1.548691in}{1.321194in}}%
\pgfpathlineto{\pgfqpoint{1.548850in}{1.320832in}}%
\pgfpathlineto{\pgfqpoint{1.549012in}{1.321128in}}%
\pgfpathlineto{\pgfqpoint{1.549012in}{1.321128in}}%
\pgfpathlineto{\pgfqpoint{1.549286in}{1.321661in}}%
\pgfpathlineto{\pgfqpoint{1.549451in}{1.321120in}}%
\pgfpathlineto{\pgfqpoint{1.549451in}{1.321120in}}%
\pgfpathlineto{\pgfqpoint{1.549611in}{1.320830in}}%
\pgfpathlineto{\pgfqpoint{1.549773in}{1.321164in}}%
\pgfpathlineto{\pgfqpoint{1.549773in}{1.321164in}}%
\pgfpathlineto{\pgfqpoint{1.549993in}{1.321661in}}%
\pgfpathlineto{\pgfqpoint{1.550213in}{1.321050in}}%
\pgfpathlineto{\pgfqpoint{1.550213in}{1.321050in}}%
\pgfpathlineto{\pgfqpoint{1.550372in}{1.320833in}}%
\pgfpathlineto{\pgfqpoint{1.550480in}{1.321044in}}%
\pgfpathlineto{\pgfqpoint{1.550480in}{1.321044in}}%
\pgfpathlineto{\pgfqpoint{1.550755in}{1.321657in}}%
\pgfpathlineto{\pgfqpoint{1.550985in}{1.320955in}}%
\pgfpathlineto{\pgfqpoint{1.550985in}{1.320955in}}%
\pgfpathlineto{\pgfqpoint{1.551091in}{1.320810in}}%
\pgfpathlineto{\pgfqpoint{1.551199in}{1.320963in}}%
\pgfpathlineto{\pgfqpoint{1.551199in}{1.320963in}}%
\pgfpathlineto{\pgfqpoint{1.552270in}{1.321638in}}%
\pgfpathlineto{\pgfqpoint{1.552428in}{1.321151in}}%
\pgfpathlineto{\pgfqpoint{1.552588in}{1.320794in}}%
\pgfpathlineto{\pgfqpoint{1.552750in}{1.321097in}}%
\pgfpathlineto{\pgfqpoint{1.552750in}{1.321097in}}%
\pgfpathlineto{\pgfqpoint{1.553026in}{1.321628in}}%
\pgfpathlineto{\pgfqpoint{1.553176in}{1.321148in}}%
\pgfpathlineto{\pgfqpoint{1.553176in}{1.321148in}}%
\pgfpathlineto{\pgfqpoint{1.553336in}{1.320787in}}%
\pgfpathlineto{\pgfqpoint{1.553498in}{1.321087in}}%
\pgfpathlineto{\pgfqpoint{1.553498in}{1.321087in}}%
\pgfpathlineto{\pgfqpoint{1.553774in}{1.321622in}}%
\pgfpathlineto{\pgfqpoint{1.553965in}{1.320974in}}%
\pgfpathlineto{\pgfqpoint{1.553965in}{1.320974in}}%
\pgfpathlineto{\pgfqpoint{1.554071in}{1.320779in}}%
\pgfpathlineto{\pgfqpoint{1.554233in}{1.321038in}}%
\pgfpathlineto{\pgfqpoint{1.554233in}{1.321038in}}%
\pgfpathlineto{\pgfqpoint{1.554509in}{1.321623in}}%
\pgfpathlineto{\pgfqpoint{1.554705in}{1.321004in}}%
\pgfpathlineto{\pgfqpoint{1.554705in}{1.321004in}}%
\pgfpathlineto{\pgfqpoint{1.554865in}{1.320788in}}%
\pgfpathlineto{\pgfqpoint{1.554973in}{1.321002in}}%
\pgfpathlineto{\pgfqpoint{1.554973in}{1.321002in}}%
\pgfpathlineto{\pgfqpoint{1.555249in}{1.321618in}}%
\pgfpathlineto{\pgfqpoint{1.555491in}{1.320877in}}%
\pgfpathlineto{\pgfqpoint{1.555491in}{1.320877in}}%
\pgfpathlineto{\pgfqpoint{1.555598in}{1.320768in}}%
\pgfpathlineto{\pgfqpoint{1.555707in}{1.320948in}}%
\pgfpathlineto{\pgfqpoint{1.555707in}{1.320948in}}%
\pgfpathlineto{\pgfqpoint{1.556780in}{1.321593in}}%
\pgfpathlineto{\pgfqpoint{1.556927in}{1.321120in}}%
\pgfpathlineto{\pgfqpoint{1.557088in}{1.320748in}}%
\pgfpathlineto{\pgfqpoint{1.557250in}{1.321046in}}%
\pgfpathlineto{\pgfqpoint{1.557250in}{1.321046in}}%
\pgfpathlineto{\pgfqpoint{1.557527in}{1.321590in}}%
\pgfpathlineto{\pgfqpoint{1.557702in}{1.321010in}}%
\pgfpathlineto{\pgfqpoint{1.557702in}{1.321010in}}%
\pgfpathlineto{\pgfqpoint{1.557862in}{1.320751in}}%
\pgfpathlineto{\pgfqpoint{1.558026in}{1.321108in}}%
\pgfpathlineto{\pgfqpoint{1.558026in}{1.321108in}}%
\pgfpathlineto{\pgfqpoint{1.558248in}{1.321591in}}%
\pgfpathlineto{\pgfqpoint{1.558432in}{1.321108in}}%
\pgfpathlineto{\pgfqpoint{1.558432in}{1.321108in}}%
\pgfpathlineto{\pgfqpoint{1.558591in}{1.320733in}}%
\pgfpathlineto{\pgfqpoint{1.558810in}{1.321195in}}%
\pgfpathlineto{\pgfqpoint{1.558810in}{1.321195in}}%
\pgfpathlineto{\pgfqpoint{1.559032in}{1.321578in}}%
\pgfpathlineto{\pgfqpoint{1.559180in}{1.321131in}}%
\pgfpathlineto{\pgfqpoint{1.559180in}{1.321131in}}%
\pgfpathlineto{\pgfqpoint{1.559363in}{1.320732in}}%
\pgfpathlineto{\pgfqpoint{1.559527in}{1.321076in}}%
\pgfpathlineto{\pgfqpoint{1.559527in}{1.321076in}}%
\pgfpathlineto{\pgfqpoint{1.559750in}{1.321576in}}%
\pgfpathlineto{\pgfqpoint{1.559940in}{1.321091in}}%
\pgfpathlineto{\pgfqpoint{1.559940in}{1.321091in}}%
\pgfpathlineto{\pgfqpoint{1.560101in}{1.320718in}}%
\pgfpathlineto{\pgfqpoint{1.560264in}{1.321017in}}%
\pgfpathlineto{\pgfqpoint{1.560264in}{1.321017in}}%
\pgfpathlineto{\pgfqpoint{1.560542in}{1.321564in}}%
\pgfpathlineto{\pgfqpoint{1.560718in}{1.320978in}}%
\pgfpathlineto{\pgfqpoint{1.560718in}{1.320978in}}%
\pgfpathlineto{\pgfqpoint{1.560879in}{1.320721in}}%
\pgfpathlineto{\pgfqpoint{1.560988in}{1.320921in}}%
\pgfpathlineto{\pgfqpoint{1.560988in}{1.320921in}}%
\pgfpathlineto{\pgfqpoint{1.561266in}{1.321565in}}%
\pgfpathlineto{\pgfqpoint{1.561559in}{1.320732in}}%
\pgfpathlineto{\pgfqpoint{1.561559in}{1.320732in}}%
\pgfpathlineto{\pgfqpoint{1.561668in}{1.320751in}}%
\pgfpathlineto{\pgfqpoint{1.561833in}{1.321180in}}%
\pgfpathlineto{\pgfqpoint{1.561833in}{1.321180in}}%
\pgfpathlineto{\pgfqpoint{1.562055in}{1.321549in}}%
\pgfpathlineto{\pgfqpoint{1.562207in}{1.321063in}}%
\pgfpathlineto{\pgfqpoint{1.562207in}{1.321063in}}%
\pgfpathlineto{\pgfqpoint{1.562368in}{1.320695in}}%
\pgfpathlineto{\pgfqpoint{1.562532in}{1.320999in}}%
\pgfpathlineto{\pgfqpoint{1.562532in}{1.320999in}}%
\pgfpathlineto{\pgfqpoint{1.562810in}{1.321543in}}%
\pgfpathlineto{\pgfqpoint{1.562994in}{1.320918in}}%
\pgfpathlineto{\pgfqpoint{1.562994in}{1.320918in}}%
\pgfpathlineto{\pgfqpoint{1.563156in}{1.320706in}}%
\pgfpathlineto{\pgfqpoint{1.563265in}{1.320925in}}%
\pgfpathlineto{\pgfqpoint{1.563265in}{1.320925in}}%
\pgfpathlineto{\pgfqpoint{1.563544in}{1.321546in}}%
\pgfpathlineto{\pgfqpoint{1.563773in}{1.320834in}}%
\pgfpathlineto{\pgfqpoint{1.563773in}{1.320834in}}%
\pgfpathlineto{\pgfqpoint{1.563881in}{1.320679in}}%
\pgfpathlineto{\pgfqpoint{1.563990in}{1.320832in}}%
\pgfpathlineto{\pgfqpoint{1.563990in}{1.320832in}}%
\pgfpathlineto{\pgfqpoint{1.565055in}{1.321533in}}%
\pgfpathlineto{\pgfqpoint{1.565232in}{1.321057in}}%
\pgfpathlineto{\pgfqpoint{1.565378in}{1.320665in}}%
\pgfpathlineto{\pgfqpoint{1.565598in}{1.321079in}}%
\pgfpathlineto{\pgfqpoint{1.565598in}{1.321079in}}%
\pgfpathlineto{\pgfqpoint{1.565822in}{1.321526in}}%
\pgfpathlineto{\pgfqpoint{1.565990in}{1.321050in}}%
\pgfpathlineto{\pgfqpoint{1.565990in}{1.321050in}}%
\pgfpathlineto{\pgfqpoint{1.566137in}{1.320658in}}%
\pgfpathlineto{\pgfqpoint{1.566357in}{1.321071in}}%
\pgfpathlineto{\pgfqpoint{1.566357in}{1.321071in}}%
\pgfpathlineto{\pgfqpoint{1.566581in}{1.321519in}}%
\pgfpathlineto{\pgfqpoint{1.566748in}{1.321053in}}%
\pgfpathlineto{\pgfqpoint{1.566748in}{1.321053in}}%
\pgfpathlineto{\pgfqpoint{1.566939in}{1.320659in}}%
\pgfpathlineto{\pgfqpoint{1.567104in}{1.321025in}}%
\pgfpathlineto{\pgfqpoint{1.567104in}{1.321025in}}%
\pgfpathlineto{\pgfqpoint{1.567328in}{1.321512in}}%
\pgfpathlineto{\pgfqpoint{1.567511in}{1.321026in}}%
\pgfpathlineto{\pgfqpoint{1.567511in}{1.321026in}}%
\pgfpathlineto{\pgfqpoint{1.567667in}{1.320640in}}%
\pgfpathlineto{\pgfqpoint{1.567887in}{1.321090in}}%
\pgfpathlineto{\pgfqpoint{1.567887in}{1.321090in}}%
\pgfpathlineto{\pgfqpoint{1.568111in}{1.321502in}}%
\pgfpathlineto{\pgfqpoint{1.568272in}{1.321019in}}%
\pgfpathlineto{\pgfqpoint{1.568272in}{1.321019in}}%
\pgfpathlineto{\pgfqpoint{1.568428in}{1.320632in}}%
\pgfpathlineto{\pgfqpoint{1.568648in}{1.321083in}}%
\pgfpathlineto{\pgfqpoint{1.568648in}{1.321083in}}%
\pgfpathlineto{\pgfqpoint{1.568873in}{1.321496in}}%
\pgfpathlineto{\pgfqpoint{1.569031in}{1.321021in}}%
\pgfpathlineto{\pgfqpoint{1.569031in}{1.321021in}}%
\pgfpathlineto{\pgfqpoint{1.569178in}{1.320627in}}%
\pgfpathlineto{\pgfqpoint{1.569399in}{1.321041in}}%
\pgfpathlineto{\pgfqpoint{1.569399in}{1.321041in}}%
\pgfpathlineto{\pgfqpoint{1.569623in}{1.321492in}}%
\pgfpathlineto{\pgfqpoint{1.569794in}{1.321008in}}%
\pgfpathlineto{\pgfqpoint{1.569794in}{1.321008in}}%
\pgfpathlineto{\pgfqpoint{1.569946in}{1.320617in}}%
\pgfpathlineto{\pgfqpoint{1.570167in}{1.321053in}}%
\pgfpathlineto{\pgfqpoint{1.570167in}{1.321053in}}%
\pgfpathlineto{\pgfqpoint{1.570392in}{1.321484in}}%
\pgfpathlineto{\pgfqpoint{1.570557in}{1.320996in}}%
\pgfpathlineto{\pgfqpoint{1.570557in}{1.320996in}}%
\pgfpathlineto{\pgfqpoint{1.570715in}{1.320609in}}%
\pgfpathlineto{\pgfqpoint{1.570936in}{1.321065in}}%
\pgfpathlineto{\pgfqpoint{1.570936in}{1.321065in}}%
\pgfpathlineto{\pgfqpoint{1.571161in}{1.321475in}}%
\pgfpathlineto{\pgfqpoint{1.571323in}{1.320976in}}%
\pgfpathlineto{\pgfqpoint{1.571323in}{1.320976in}}%
\pgfpathlineto{\pgfqpoint{1.571486in}{1.320602in}}%
\pgfpathlineto{\pgfqpoint{1.571651in}{1.320911in}}%
\pgfpathlineto{\pgfqpoint{1.571651in}{1.320911in}}%
\pgfpathlineto{\pgfqpoint{1.571932in}{1.321463in}}%
\pgfpathlineto{\pgfqpoint{1.572115in}{1.320837in}}%
\pgfpathlineto{\pgfqpoint{1.572115in}{1.320837in}}%
\pgfpathlineto{\pgfqpoint{1.572279in}{1.320610in}}%
\pgfpathlineto{\pgfqpoint{1.572389in}{1.320829in}}%
\pgfpathlineto{\pgfqpoint{1.572389in}{1.320829in}}%
\pgfpathlineto{\pgfqpoint{1.572670in}{1.321466in}}%
\pgfpathlineto{\pgfqpoint{1.572915in}{1.320708in}}%
\pgfpathlineto{\pgfqpoint{1.572915in}{1.320708in}}%
\pgfpathlineto{\pgfqpoint{1.573024in}{1.320590in}}%
\pgfpathlineto{\pgfqpoint{1.573135in}{1.320771in}}%
\pgfpathlineto{\pgfqpoint{1.573135in}{1.320771in}}%
\pgfpathlineto{\pgfqpoint{1.574222in}{1.321446in}}%
\pgfpathlineto{\pgfqpoint{1.574378in}{1.320961in}}%
\pgfpathlineto{\pgfqpoint{1.574534in}{1.320570in}}%
\pgfpathlineto{\pgfqpoint{1.574755in}{1.321021in}}%
\pgfpathlineto{\pgfqpoint{1.574755in}{1.321021in}}%
\pgfpathlineto{\pgfqpoint{1.574981in}{1.321443in}}%
\pgfpathlineto{\pgfqpoint{1.575147in}{1.320937in}}%
\pgfpathlineto{\pgfqpoint{1.575147in}{1.320937in}}%
\pgfpathlineto{\pgfqpoint{1.575311in}{1.320562in}}%
\pgfpathlineto{\pgfqpoint{1.575477in}{1.320875in}}%
\pgfpathlineto{\pgfqpoint{1.575477in}{1.320875in}}%
\pgfpathlineto{\pgfqpoint{1.575759in}{1.321429in}}%
\pgfpathlineto{\pgfqpoint{1.575915in}{1.320925in}}%
\pgfpathlineto{\pgfqpoint{1.575915in}{1.320925in}}%
\pgfpathlineto{\pgfqpoint{1.576078in}{1.320555in}}%
\pgfpathlineto{\pgfqpoint{1.576244in}{1.320870in}}%
\pgfpathlineto{\pgfqpoint{1.576244in}{1.320870in}}%
\pgfpathlineto{\pgfqpoint{1.576527in}{1.321422in}}%
\pgfpathlineto{\pgfqpoint{1.576713in}{1.320780in}}%
\pgfpathlineto{\pgfqpoint{1.576713in}{1.320780in}}%
\pgfpathlineto{\pgfqpoint{1.576877in}{1.320566in}}%
\pgfpathlineto{\pgfqpoint{1.576988in}{1.320792in}}%
\pgfpathlineto{\pgfqpoint{1.576988in}{1.320792in}}%
\pgfpathlineto{\pgfqpoint{1.577270in}{1.321426in}}%
\pgfpathlineto{\pgfqpoint{1.577523in}{1.320637in}}%
\pgfpathlineto{\pgfqpoint{1.577523in}{1.320637in}}%
\pgfpathlineto{\pgfqpoint{1.577633in}{1.320548in}}%
\pgfpathlineto{\pgfqpoint{1.577743in}{1.320751in}}%
\pgfpathlineto{\pgfqpoint{1.577743in}{1.320751in}}%
\pgfpathlineto{\pgfqpoint{1.578026in}{1.321418in}}%
\pgfpathlineto{\pgfqpoint{1.578295in}{1.320621in}}%
\pgfpathlineto{\pgfqpoint{1.578295in}{1.320621in}}%
\pgfpathlineto{\pgfqpoint{1.578404in}{1.320542in}}%
\pgfpathlineto{\pgfqpoint{1.578515in}{1.320752in}}%
\pgfpathlineto{\pgfqpoint{1.578515in}{1.320752in}}%
\pgfpathlineto{\pgfqpoint{1.578798in}{1.321412in}}%
\pgfpathlineto{\pgfqpoint{1.579094in}{1.320556in}}%
\pgfpathlineto{\pgfqpoint{1.579094in}{1.320556in}}%
\pgfpathlineto{\pgfqpoint{1.579204in}{1.320570in}}%
\pgfpathlineto{\pgfqpoint{1.579372in}{1.321008in}}%
\pgfpathlineto{\pgfqpoint{1.579372in}{1.321008in}}%
\pgfpathlineto{\pgfqpoint{1.579599in}{1.321396in}}%
\pgfpathlineto{\pgfqpoint{1.579756in}{1.320889in}}%
\pgfpathlineto{\pgfqpoint{1.579756in}{1.320889in}}%
\pgfpathlineto{\pgfqpoint{1.579921in}{1.320515in}}%
\pgfpathlineto{\pgfqpoint{1.580088in}{1.320832in}}%
\pgfpathlineto{\pgfqpoint{1.580088in}{1.320832in}}%
\pgfpathlineto{\pgfqpoint{1.580371in}{1.321388in}}%
\pgfpathlineto{\pgfqpoint{1.580526in}{1.320886in}}%
\pgfpathlineto{\pgfqpoint{1.580526in}{1.320886in}}%
\pgfpathlineto{\pgfqpoint{1.580690in}{1.320507in}}%
\pgfpathlineto{\pgfqpoint{1.580857in}{1.320821in}}%
\pgfpathlineto{\pgfqpoint{1.580857in}{1.320821in}}%
\pgfpathlineto{\pgfqpoint{1.581141in}{1.321382in}}%
\pgfpathlineto{\pgfqpoint{1.581299in}{1.320865in}}%
\pgfpathlineto{\pgfqpoint{1.581299in}{1.320865in}}%
\pgfpathlineto{\pgfqpoint{1.581464in}{1.320500in}}%
\pgfpathlineto{\pgfqpoint{1.581631in}{1.320822in}}%
\pgfpathlineto{\pgfqpoint{1.581631in}{1.320822in}}%
\pgfpathlineto{\pgfqpoint{1.581915in}{1.321373in}}%
\pgfpathlineto{\pgfqpoint{1.582089in}{1.320775in}}%
\pgfpathlineto{\pgfqpoint{1.582089in}{1.320775in}}%
\pgfpathlineto{\pgfqpoint{1.582253in}{1.320500in}}%
\pgfpathlineto{\pgfqpoint{1.582421in}{1.320871in}}%
\pgfpathlineto{\pgfqpoint{1.582421in}{1.320871in}}%
\pgfpathlineto{\pgfqpoint{1.582649in}{1.321377in}}%
\pgfpathlineto{\pgfqpoint{1.582837in}{1.320881in}}%
\pgfpathlineto{\pgfqpoint{1.582837in}{1.320881in}}%
\pgfpathlineto{\pgfqpoint{1.582994in}{1.320482in}}%
\pgfpathlineto{\pgfqpoint{1.583217in}{1.320938in}}%
\pgfpathlineto{\pgfqpoint{1.583217in}{1.320938in}}%
\pgfpathlineto{\pgfqpoint{1.583445in}{1.321369in}}%
\pgfpathlineto{\pgfqpoint{1.583605in}{1.320898in}}%
\pgfpathlineto{\pgfqpoint{1.583605in}{1.320898in}}%
\pgfpathlineto{\pgfqpoint{1.583793in}{1.320481in}}%
\pgfpathlineto{\pgfqpoint{1.583961in}{1.320839in}}%
\pgfpathlineto{\pgfqpoint{1.583961in}{1.320839in}}%
\pgfpathlineto{\pgfqpoint{1.584189in}{1.321362in}}%
\pgfpathlineto{\pgfqpoint{1.584404in}{1.320764in}}%
\pgfpathlineto{\pgfqpoint{1.584404in}{1.320764in}}%
\pgfpathlineto{\pgfqpoint{1.584570in}{1.320475in}}%
\pgfpathlineto{\pgfqpoint{1.584738in}{1.320840in}}%
\pgfpathlineto{\pgfqpoint{1.584738in}{1.320840in}}%
\pgfpathlineto{\pgfqpoint{1.584966in}{1.321356in}}%
\pgfpathlineto{\pgfqpoint{1.585184in}{1.320733in}}%
\pgfpathlineto{\pgfqpoint{1.585184in}{1.320733in}}%
\pgfpathlineto{\pgfqpoint{1.585349in}{1.320471in}}%
\pgfpathlineto{\pgfqpoint{1.585461in}{1.320682in}}%
\pgfpathlineto{\pgfqpoint{1.585461in}{1.320682in}}%
\pgfpathlineto{\pgfqpoint{1.585746in}{1.321351in}}%
\pgfpathlineto{\pgfqpoint{1.585993in}{1.320598in}}%
\pgfpathlineto{\pgfqpoint{1.585993in}{1.320598in}}%
\pgfpathlineto{\pgfqpoint{1.586104in}{1.320452in}}%
\pgfpathlineto{\pgfqpoint{1.586215in}{1.320620in}}%
\pgfpathlineto{\pgfqpoint{1.586215in}{1.320620in}}%
\pgfpathlineto{\pgfqpoint{1.587326in}{1.321330in}}%
\pgfpathlineto{\pgfqpoint{1.587479in}{1.320857in}}%
\pgfpathlineto{\pgfqpoint{1.587671in}{1.320443in}}%
\pgfpathlineto{\pgfqpoint{1.587839in}{1.320812in}}%
\pgfpathlineto{\pgfqpoint{1.587839in}{1.320812in}}%
\pgfpathlineto{\pgfqpoint{1.588068in}{1.321329in}}%
\pgfpathlineto{\pgfqpoint{1.588257in}{1.320838in}}%
\pgfpathlineto{\pgfqpoint{1.588257in}{1.320838in}}%
\pgfpathlineto{\pgfqpoint{1.588406in}{1.320429in}}%
\pgfpathlineto{\pgfqpoint{1.588630in}{1.320849in}}%
\pgfpathlineto{\pgfqpoint{1.588630in}{1.320849in}}%
\pgfpathlineto{\pgfqpoint{1.588859in}{1.321324in}}%
\pgfpathlineto{\pgfqpoint{1.589037in}{1.320818in}}%
\pgfpathlineto{\pgfqpoint{1.589037in}{1.320818in}}%
\pgfpathlineto{\pgfqpoint{1.589197in}{1.320418in}}%
\pgfpathlineto{\pgfqpoint{1.589423in}{1.320889in}}%
\pgfpathlineto{\pgfqpoint{1.589423in}{1.320889in}}%
\pgfpathlineto{\pgfqpoint{1.589652in}{1.321312in}}%
\pgfpathlineto{\pgfqpoint{1.589816in}{1.320805in}}%
\pgfpathlineto{\pgfqpoint{1.589816in}{1.320805in}}%
\pgfpathlineto{\pgfqpoint{1.589981in}{1.320410in}}%
\pgfpathlineto{\pgfqpoint{1.590150in}{1.320723in}}%
\pgfpathlineto{\pgfqpoint{1.590150in}{1.320723in}}%
\pgfpathlineto{\pgfqpoint{1.590436in}{1.321302in}}%
\pgfpathlineto{\pgfqpoint{1.590599in}{1.320772in}}%
\pgfpathlineto{\pgfqpoint{1.590599in}{1.320772in}}%
\pgfpathlineto{\pgfqpoint{1.590765in}{1.320403in}}%
\pgfpathlineto{\pgfqpoint{1.590934in}{1.320733in}}%
\pgfpathlineto{\pgfqpoint{1.590934in}{1.320733in}}%
\pgfpathlineto{\pgfqpoint{1.591221in}{1.321290in}}%
\pgfpathlineto{\pgfqpoint{1.591374in}{1.320784in}}%
\pgfpathlineto{\pgfqpoint{1.591374in}{1.320784in}}%
\pgfpathlineto{\pgfqpoint{1.591540in}{1.320394in}}%
\pgfpathlineto{\pgfqpoint{1.591709in}{1.320713in}}%
\pgfpathlineto{\pgfqpoint{1.591709in}{1.320713in}}%
\pgfpathlineto{\pgfqpoint{1.591996in}{1.321287in}}%
\pgfpathlineto{\pgfqpoint{1.592154in}{1.320775in}}%
\pgfpathlineto{\pgfqpoint{1.592154in}{1.320775in}}%
\pgfpathlineto{\pgfqpoint{1.592320in}{1.320386in}}%
\pgfpathlineto{\pgfqpoint{1.592489in}{1.320706in}}%
\pgfpathlineto{\pgfqpoint{1.592489in}{1.320706in}}%
\pgfpathlineto{\pgfqpoint{1.592777in}{1.321279in}}%
\pgfpathlineto{\pgfqpoint{1.592934in}{1.320771in}}%
\pgfpathlineto{\pgfqpoint{1.592934in}{1.320771in}}%
\pgfpathlineto{\pgfqpoint{1.593100in}{1.320378in}}%
\pgfpathlineto{\pgfqpoint{1.593269in}{1.320696in}}%
\pgfpathlineto{\pgfqpoint{1.593269in}{1.320696in}}%
\pgfpathlineto{\pgfqpoint{1.593557in}{1.321273in}}%
\pgfpathlineto{\pgfqpoint{1.593743in}{1.320632in}}%
\pgfpathlineto{\pgfqpoint{1.593743in}{1.320632in}}%
\pgfpathlineto{\pgfqpoint{1.593910in}{1.320385in}}%
\pgfpathlineto{\pgfqpoint{1.594023in}{1.320607in}}%
\pgfpathlineto{\pgfqpoint{1.594023in}{1.320607in}}%
\pgfpathlineto{\pgfqpoint{1.594310in}{1.321275in}}%
\pgfpathlineto{\pgfqpoint{1.594551in}{1.320526in}}%
\pgfpathlineto{\pgfqpoint{1.594551in}{1.320526in}}%
\pgfpathlineto{\pgfqpoint{1.594662in}{1.320361in}}%
\pgfpathlineto{\pgfqpoint{1.594775in}{1.320522in}}%
\pgfpathlineto{\pgfqpoint{1.594775in}{1.320522in}}%
\pgfpathlineto{\pgfqpoint{1.595885in}{1.321261in}}%
\pgfpathlineto{\pgfqpoint{1.596050in}{1.320798in}}%
\pgfpathlineto{\pgfqpoint{1.596249in}{1.320354in}}%
\pgfpathlineto{\pgfqpoint{1.596419in}{1.320730in}}%
\pgfpathlineto{\pgfqpoint{1.596419in}{1.320730in}}%
\pgfpathlineto{\pgfqpoint{1.596650in}{1.321253in}}%
\pgfpathlineto{\pgfqpoint{1.596846in}{1.320723in}}%
\pgfpathlineto{\pgfqpoint{1.596846in}{1.320723in}}%
\pgfpathlineto{\pgfqpoint{1.597014in}{1.320337in}}%
\pgfpathlineto{\pgfqpoint{1.597183in}{1.320663in}}%
\pgfpathlineto{\pgfqpoint{1.597183in}{1.320663in}}%
\pgfpathlineto{\pgfqpoint{1.597472in}{1.321237in}}%
\pgfpathlineto{\pgfqpoint{1.597661in}{1.320576in}}%
\pgfpathlineto{\pgfqpoint{1.597661in}{1.320576in}}%
\pgfpathlineto{\pgfqpoint{1.597829in}{1.320348in}}%
\pgfpathlineto{\pgfqpoint{1.597942in}{1.320580in}}%
\pgfpathlineto{\pgfqpoint{1.597942in}{1.320580in}}%
\pgfpathlineto{\pgfqpoint{1.598231in}{1.321241in}}%
\pgfpathlineto{\pgfqpoint{1.598459in}{1.320518in}}%
\pgfpathlineto{\pgfqpoint{1.598459in}{1.320518in}}%
\pgfpathlineto{\pgfqpoint{1.598571in}{1.320320in}}%
\pgfpathlineto{\pgfqpoint{1.598741in}{1.320611in}}%
\pgfpathlineto{\pgfqpoint{1.598741in}{1.320611in}}%
\pgfpathlineto{\pgfqpoint{1.599030in}{1.321230in}}%
\pgfpathlineto{\pgfqpoint{1.599256in}{1.320471in}}%
\pgfpathlineto{\pgfqpoint{1.599256in}{1.320471in}}%
\pgfpathlineto{\pgfqpoint{1.599368in}{1.320313in}}%
\pgfpathlineto{\pgfqpoint{1.599481in}{1.320479in}}%
\pgfpathlineto{\pgfqpoint{1.599481in}{1.320479in}}%
\pgfpathlineto{\pgfqpoint{1.600624in}{1.321200in}}%
\pgfpathlineto{\pgfqpoint{1.600760in}{1.320757in}}%
\pgfpathlineto{\pgfqpoint{1.600957in}{1.320303in}}%
\pgfpathlineto{\pgfqpoint{1.601128in}{1.320675in}}%
\pgfpathlineto{\pgfqpoint{1.601128in}{1.320675in}}%
\pgfpathlineto{\pgfqpoint{1.601360in}{1.321211in}}%
\pgfpathlineto{\pgfqpoint{1.601564in}{1.320658in}}%
\pgfpathlineto{\pgfqpoint{1.601564in}{1.320658in}}%
\pgfpathlineto{\pgfqpoint{1.601732in}{1.320289in}}%
\pgfpathlineto{\pgfqpoint{1.601903in}{1.320628in}}%
\pgfpathlineto{\pgfqpoint{1.601903in}{1.320628in}}%
\pgfpathlineto{\pgfqpoint{1.602193in}{1.321192in}}%
\pgfpathlineto{\pgfqpoint{1.602387in}{1.320499in}}%
\pgfpathlineto{\pgfqpoint{1.602387in}{1.320499in}}%
\pgfpathlineto{\pgfqpoint{1.602499in}{1.320280in}}%
\pgfpathlineto{\pgfqpoint{1.602670in}{1.320555in}}%
\pgfpathlineto{\pgfqpoint{1.602670in}{1.320555in}}%
\pgfpathlineto{\pgfqpoint{1.602960in}{1.321197in}}%
\pgfpathlineto{\pgfqpoint{1.603157in}{1.320567in}}%
\pgfpathlineto{\pgfqpoint{1.603157in}{1.320567in}}%
\pgfpathlineto{\pgfqpoint{1.603325in}{1.320281in}}%
\pgfpathlineto{\pgfqpoint{1.603497in}{1.320664in}}%
\pgfpathlineto{\pgfqpoint{1.603497in}{1.320664in}}%
\pgfpathlineto{\pgfqpoint{1.603730in}{1.321191in}}%
\pgfpathlineto{\pgfqpoint{1.603927in}{1.320648in}}%
\pgfpathlineto{\pgfqpoint{1.603927in}{1.320648in}}%
\pgfpathlineto{\pgfqpoint{1.604096in}{1.320263in}}%
\pgfpathlineto{\pgfqpoint{1.604267in}{1.320596in}}%
\pgfpathlineto{\pgfqpoint{1.604267in}{1.320596in}}%
\pgfpathlineto{\pgfqpoint{1.604558in}{1.321173in}}%
\pgfpathlineto{\pgfqpoint{1.604714in}{1.320658in}}%
\pgfpathlineto{\pgfqpoint{1.604714in}{1.320658in}}%
\pgfpathlineto{\pgfqpoint{1.604883in}{1.320254in}}%
\pgfpathlineto{\pgfqpoint{1.605054in}{1.320578in}}%
\pgfpathlineto{\pgfqpoint{1.605054in}{1.320578in}}%
\pgfpathlineto{\pgfqpoint{1.605345in}{1.321168in}}%
\pgfpathlineto{\pgfqpoint{1.605509in}{1.320627in}}%
\pgfpathlineto{\pgfqpoint{1.605509in}{1.320627in}}%
\pgfpathlineto{\pgfqpoint{1.605678in}{1.320247in}}%
\pgfpathlineto{\pgfqpoint{1.605849in}{1.320584in}}%
\pgfpathlineto{\pgfqpoint{1.605849in}{1.320584in}}%
\pgfpathlineto{\pgfqpoint{1.606141in}{1.321158in}}%
\pgfpathlineto{\pgfqpoint{1.606334in}{1.320469in}}%
\pgfpathlineto{\pgfqpoint{1.606334in}{1.320469in}}%
\pgfpathlineto{\pgfqpoint{1.606447in}{1.320239in}}%
\pgfpathlineto{\pgfqpoint{1.606618in}{1.320507in}}%
\pgfpathlineto{\pgfqpoint{1.606618in}{1.320507in}}%
\pgfpathlineto{\pgfqpoint{1.606909in}{1.321163in}}%
\pgfpathlineto{\pgfqpoint{1.607146in}{1.320389in}}%
\pgfpathlineto{\pgfqpoint{1.607146in}{1.320389in}}%
\pgfpathlineto{\pgfqpoint{1.607259in}{1.320230in}}%
\pgfpathlineto{\pgfqpoint{1.607373in}{1.320400in}}%
\pgfpathlineto{\pgfqpoint{1.607373in}{1.320400in}}%
\pgfpathlineto{\pgfqpoint{1.608507in}{1.321144in}}%
\pgfpathlineto{\pgfqpoint{1.608656in}{1.320715in}}%
\pgfpathlineto{\pgfqpoint{1.608831in}{1.320213in}}%
\pgfpathlineto{\pgfqpoint{1.609060in}{1.320682in}}%
\pgfpathlineto{\pgfqpoint{1.609060in}{1.320682in}}%
\pgfpathlineto{\pgfqpoint{1.609294in}{1.321140in}}%
\pgfpathlineto{\pgfqpoint{1.609472in}{1.320589in}}%
\pgfpathlineto{\pgfqpoint{1.609472in}{1.320589in}}%
\pgfpathlineto{\pgfqpoint{1.609641in}{1.320205in}}%
\pgfpathlineto{\pgfqpoint{1.609813in}{1.320544in}}%
\pgfpathlineto{\pgfqpoint{1.609813in}{1.320544in}}%
\pgfpathlineto{\pgfqpoint{1.610106in}{1.321123in}}%
\pgfpathlineto{\pgfqpoint{1.610262in}{1.320601in}}%
\pgfpathlineto{\pgfqpoint{1.610262in}{1.320601in}}%
\pgfpathlineto{\pgfqpoint{1.610432in}{1.320196in}}%
\pgfpathlineto{\pgfqpoint{1.610604in}{1.320524in}}%
\pgfpathlineto{\pgfqpoint{1.610604in}{1.320524in}}%
\pgfpathlineto{\pgfqpoint{1.610897in}{1.321119in}}%
\pgfpathlineto{\pgfqpoint{1.611059in}{1.320587in}}%
\pgfpathlineto{\pgfqpoint{1.611059in}{1.320587in}}%
\pgfpathlineto{\pgfqpoint{1.611228in}{1.320188in}}%
\pgfpathlineto{\pgfqpoint{1.611401in}{1.320519in}}%
\pgfpathlineto{\pgfqpoint{1.611401in}{1.320519in}}%
\pgfpathlineto{\pgfqpoint{1.611694in}{1.321111in}}%
\pgfpathlineto{\pgfqpoint{1.611852in}{1.320590in}}%
\pgfpathlineto{\pgfqpoint{1.611852in}{1.320590in}}%
\pgfpathlineto{\pgfqpoint{1.612021in}{1.320179in}}%
\pgfpathlineto{\pgfqpoint{1.612194in}{1.320502in}}%
\pgfpathlineto{\pgfqpoint{1.612194in}{1.320502in}}%
\pgfpathlineto{\pgfqpoint{1.612487in}{1.321106in}}%
\pgfpathlineto{\pgfqpoint{1.612662in}{1.320519in}}%
\pgfpathlineto{\pgfqpoint{1.612662in}{1.320519in}}%
\pgfpathlineto{\pgfqpoint{1.612832in}{1.320174in}}%
\pgfpathlineto{\pgfqpoint{1.613004in}{1.320537in}}%
\pgfpathlineto{\pgfqpoint{1.613004in}{1.320537in}}%
\pgfpathlineto{\pgfqpoint{1.613298in}{1.321087in}}%
\pgfpathlineto{\pgfqpoint{1.613481in}{1.320412in}}%
\pgfpathlineto{\pgfqpoint{1.613481in}{1.320412in}}%
\pgfpathlineto{\pgfqpoint{1.613652in}{1.320183in}}%
\pgfpathlineto{\pgfqpoint{1.613767in}{1.320423in}}%
\pgfpathlineto{\pgfqpoint{1.613767in}{1.320423in}}%
\pgfpathlineto{\pgfqpoint{1.614060in}{1.321100in}}%
\pgfpathlineto{\pgfqpoint{1.614324in}{1.320256in}}%
\pgfpathlineto{\pgfqpoint{1.614324in}{1.320256in}}%
\pgfpathlineto{\pgfqpoint{1.614438in}{1.320164in}}%
\pgfpathlineto{\pgfqpoint{1.614554in}{1.320383in}}%
\pgfpathlineto{\pgfqpoint{1.614554in}{1.320383in}}%
\pgfpathlineto{\pgfqpoint{1.614847in}{1.321092in}}%
\pgfpathlineto{\pgfqpoint{1.615123in}{1.320248in}}%
\pgfpathlineto{\pgfqpoint{1.615123in}{1.320248in}}%
\pgfpathlineto{\pgfqpoint{1.615237in}{1.320156in}}%
\pgfpathlineto{\pgfqpoint{1.615352in}{1.320374in}}%
\pgfpathlineto{\pgfqpoint{1.615352in}{1.320374in}}%
\pgfpathlineto{\pgfqpoint{1.615646in}{1.321084in}}%
\pgfpathlineto{\pgfqpoint{1.615925in}{1.320232in}}%
\pgfpathlineto{\pgfqpoint{1.615925in}{1.320232in}}%
\pgfpathlineto{\pgfqpoint{1.616039in}{1.320150in}}%
\pgfpathlineto{\pgfqpoint{1.616155in}{1.320374in}}%
\pgfpathlineto{\pgfqpoint{1.616155in}{1.320374in}}%
\pgfpathlineto{\pgfqpoint{1.616449in}{1.321078in}}%
\pgfpathlineto{\pgfqpoint{1.616721in}{1.320234in}}%
\pgfpathlineto{\pgfqpoint{1.616721in}{1.320234in}}%
\pgfpathlineto{\pgfqpoint{1.616835in}{1.320138in}}%
\pgfpathlineto{\pgfqpoint{1.616950in}{1.320354in}}%
\pgfpathlineto{\pgfqpoint{1.616950in}{1.320354in}}%
\pgfpathlineto{\pgfqpoint{1.617245in}{1.321070in}}%
\pgfpathlineto{\pgfqpoint{1.617522in}{1.320223in}}%
\pgfpathlineto{\pgfqpoint{1.617522in}{1.320223in}}%
\pgfpathlineto{\pgfqpoint{1.617637in}{1.320130in}}%
\pgfpathlineto{\pgfqpoint{1.617752in}{1.320349in}}%
\pgfpathlineto{\pgfqpoint{1.617752in}{1.320349in}}%
\pgfpathlineto{\pgfqpoint{1.618047in}{1.321063in}}%
\pgfpathlineto{\pgfqpoint{1.618302in}{1.320277in}}%
\pgfpathlineto{\pgfqpoint{1.618302in}{1.320277in}}%
\pgfpathlineto{\pgfqpoint{1.618416in}{1.320111in}}%
\pgfpathlineto{\pgfqpoint{1.618532in}{1.320282in}}%
\pgfpathlineto{\pgfqpoint{1.618532in}{1.320282in}}%
\pgfpathlineto{\pgfqpoint{1.619666in}{1.321049in}}%
\pgfpathlineto{\pgfqpoint{1.619816in}{1.320678in}}%
\pgfpathlineto{\pgfqpoint{1.620048in}{1.320109in}}%
\pgfpathlineto{\pgfqpoint{1.620283in}{1.320709in}}%
\pgfpathlineto{\pgfqpoint{1.620283in}{1.320709in}}%
\pgfpathlineto{\pgfqpoint{1.620460in}{1.321042in}}%
\pgfpathlineto{\pgfqpoint{1.620634in}{1.320598in}}%
\pgfpathlineto{\pgfqpoint{1.620634in}{1.320598in}}%
\pgfpathlineto{\pgfqpoint{1.620850in}{1.320099in}}%
\pgfpathlineto{\pgfqpoint{1.621025in}{1.320507in}}%
\pgfpathlineto{\pgfqpoint{1.621025in}{1.320507in}}%
\pgfpathlineto{\pgfqpoint{1.621262in}{1.321035in}}%
\pgfpathlineto{\pgfqpoint{1.621472in}{1.320408in}}%
\pgfpathlineto{\pgfqpoint{1.621472in}{1.320408in}}%
\pgfpathlineto{\pgfqpoint{1.621644in}{1.320083in}}%
\pgfpathlineto{\pgfqpoint{1.621819in}{1.320465in}}%
\pgfpathlineto{\pgfqpoint{1.621819in}{1.320465in}}%
\pgfpathlineto{\pgfqpoint{1.622056in}{1.321025in}}%
\pgfpathlineto{\pgfqpoint{1.622262in}{1.320473in}}%
\pgfpathlineto{\pgfqpoint{1.622262in}{1.320473in}}%
\pgfpathlineto{\pgfqpoint{1.622434in}{1.320069in}}%
\pgfpathlineto{\pgfqpoint{1.622608in}{1.320409in}}%
\pgfpathlineto{\pgfqpoint{1.622608in}{1.320409in}}%
\pgfpathlineto{\pgfqpoint{1.622905in}{1.321009in}}%
\pgfpathlineto{\pgfqpoint{1.623067in}{1.320466in}}%
\pgfpathlineto{\pgfqpoint{1.623067in}{1.320466in}}%
\pgfpathlineto{\pgfqpoint{1.623239in}{1.320060in}}%
\pgfpathlineto{\pgfqpoint{1.623414in}{1.320400in}}%
\pgfpathlineto{\pgfqpoint{1.623414in}{1.320400in}}%
\pgfpathlineto{\pgfqpoint{1.623710in}{1.321002in}}%
\pgfpathlineto{\pgfqpoint{1.623918in}{1.320259in}}%
\pgfpathlineto{\pgfqpoint{1.623918in}{1.320259in}}%
\pgfpathlineto{\pgfqpoint{1.624033in}{1.320051in}}%
\pgfpathlineto{\pgfqpoint{1.624207in}{1.320354in}}%
\pgfpathlineto{\pgfqpoint{1.624207in}{1.320354in}}%
\pgfpathlineto{\pgfqpoint{1.624504in}{1.321003in}}%
\pgfpathlineto{\pgfqpoint{1.624741in}{1.320195in}}%
\pgfpathlineto{\pgfqpoint{1.624741in}{1.320195in}}%
\pgfpathlineto{\pgfqpoint{1.624856in}{1.320044in}}%
\pgfpathlineto{\pgfqpoint{1.624973in}{1.320228in}}%
\pgfpathlineto{\pgfqpoint{1.624973in}{1.320228in}}%
\pgfpathlineto{\pgfqpoint{1.626133in}{1.320980in}}%
\pgfpathlineto{\pgfqpoint{1.626267in}{1.320588in}}%
\pgfpathlineto{\pgfqpoint{1.626496in}{1.320041in}}%
\pgfpathlineto{\pgfqpoint{1.626732in}{1.320650in}}%
\pgfpathlineto{\pgfqpoint{1.626732in}{1.320650in}}%
\pgfpathlineto{\pgfqpoint{1.626910in}{1.320984in}}%
\pgfpathlineto{\pgfqpoint{1.627085in}{1.320530in}}%
\pgfpathlineto{\pgfqpoint{1.627085in}{1.320530in}}%
\pgfpathlineto{\pgfqpoint{1.627294in}{1.320025in}}%
\pgfpathlineto{\pgfqpoint{1.627530in}{1.320609in}}%
\pgfpathlineto{\pgfqpoint{1.627530in}{1.320609in}}%
\pgfpathlineto{\pgfqpoint{1.627709in}{1.320975in}}%
\pgfpathlineto{\pgfqpoint{1.627900in}{1.320490in}}%
\pgfpathlineto{\pgfqpoint{1.627900in}{1.320490in}}%
\pgfpathlineto{\pgfqpoint{1.628086in}{1.320008in}}%
\pgfpathlineto{\pgfqpoint{1.628321in}{1.320543in}}%
\pgfpathlineto{\pgfqpoint{1.628321in}{1.320543in}}%
\pgfpathlineto{\pgfqpoint{1.628560in}{1.320958in}}%
\pgfpathlineto{\pgfqpoint{1.628719in}{1.320432in}}%
\pgfpathlineto{\pgfqpoint{1.628719in}{1.320432in}}%
\pgfpathlineto{\pgfqpoint{1.628881in}{1.319999in}}%
\pgfpathlineto{\pgfqpoint{1.629116in}{1.320483in}}%
\pgfpathlineto{\pgfqpoint{1.629116in}{1.320483in}}%
\pgfpathlineto{\pgfqpoint{1.629355in}{1.320960in}}%
\pgfpathlineto{\pgfqpoint{1.629533in}{1.320407in}}%
\pgfpathlineto{\pgfqpoint{1.629533in}{1.320407in}}%
\pgfpathlineto{\pgfqpoint{1.629706in}{1.319991in}}%
\pgfpathlineto{\pgfqpoint{1.629882in}{1.320332in}}%
\pgfpathlineto{\pgfqpoint{1.629882in}{1.320332in}}%
\pgfpathlineto{\pgfqpoint{1.630181in}{1.320945in}}%
\pgfpathlineto{\pgfqpoint{1.630344in}{1.320400in}}%
\pgfpathlineto{\pgfqpoint{1.630344in}{1.320400in}}%
\pgfpathlineto{\pgfqpoint{1.630517in}{1.319982in}}%
\pgfpathlineto{\pgfqpoint{1.630693in}{1.320322in}}%
\pgfpathlineto{\pgfqpoint{1.630693in}{1.320322in}}%
\pgfpathlineto{\pgfqpoint{1.630992in}{1.320937in}}%
\pgfpathlineto{\pgfqpoint{1.631208in}{1.320166in}}%
\pgfpathlineto{\pgfqpoint{1.631208in}{1.320166in}}%
\pgfpathlineto{\pgfqpoint{1.631323in}{1.319973in}}%
\pgfpathlineto{\pgfqpoint{1.631440in}{1.320130in}}%
\pgfpathlineto{\pgfqpoint{1.631440in}{1.320130in}}%
\pgfpathlineto{\pgfqpoint{1.632580in}{1.320931in}}%
\pgfpathlineto{\pgfqpoint{1.632756in}{1.320523in}}%
\pgfpathlineto{\pgfqpoint{1.632956in}{1.319956in}}%
\pgfpathlineto{\pgfqpoint{1.633193in}{1.320490in}}%
\pgfpathlineto{\pgfqpoint{1.633193in}{1.320490in}}%
\pgfpathlineto{\pgfqpoint{1.633432in}{1.320915in}}%
\pgfpathlineto{\pgfqpoint{1.633589in}{1.320410in}}%
\pgfpathlineto{\pgfqpoint{1.633589in}{1.320410in}}%
\pgfpathlineto{\pgfqpoint{1.633786in}{1.319953in}}%
\pgfpathlineto{\pgfqpoint{1.633963in}{1.320338in}}%
\pgfpathlineto{\pgfqpoint{1.633963in}{1.320338in}}%
\pgfpathlineto{\pgfqpoint{1.634203in}{1.320914in}}%
\pgfpathlineto{\pgfqpoint{1.634410in}{1.320368in}}%
\pgfpathlineto{\pgfqpoint{1.634410in}{1.320368in}}%
\pgfpathlineto{\pgfqpoint{1.634580in}{1.319938in}}%
\pgfpathlineto{\pgfqpoint{1.634817in}{1.320456in}}%
\pgfpathlineto{\pgfqpoint{1.634817in}{1.320456in}}%
\pgfpathlineto{\pgfqpoint{1.635057in}{1.320904in}}%
\pgfpathlineto{\pgfqpoint{1.635228in}{1.320351in}}%
\pgfpathlineto{\pgfqpoint{1.635228in}{1.320351in}}%
\pgfpathlineto{\pgfqpoint{1.635402in}{1.319929in}}%
\pgfpathlineto{\pgfqpoint{1.635579in}{1.320273in}}%
\pgfpathlineto{\pgfqpoint{1.635579in}{1.320273in}}%
\pgfpathlineto{\pgfqpoint{1.635879in}{1.320893in}}%
\pgfpathlineto{\pgfqpoint{1.636073in}{1.320202in}}%
\pgfpathlineto{\pgfqpoint{1.636073in}{1.320202in}}%
\pgfpathlineto{\pgfqpoint{1.636248in}{1.319937in}}%
\pgfpathlineto{\pgfqpoint{1.636366in}{1.320177in}}%
\pgfpathlineto{\pgfqpoint{1.636366in}{1.320177in}}%
\pgfpathlineto{\pgfqpoint{1.636667in}{1.320896in}}%
\pgfpathlineto{\pgfqpoint{1.636916in}{1.320095in}}%
\pgfpathlineto{\pgfqpoint{1.636916in}{1.320095in}}%
\pgfpathlineto{\pgfqpoint{1.637033in}{1.319911in}}%
\pgfpathlineto{\pgfqpoint{1.637150in}{1.320079in}}%
\pgfpathlineto{\pgfqpoint{1.637150in}{1.320079in}}%
\pgfpathlineto{\pgfqpoint{1.638333in}{1.320870in}}%
\pgfpathlineto{\pgfqpoint{1.638468in}{1.320476in}}%
\pgfpathlineto{\pgfqpoint{1.638648in}{1.319897in}}%
\pgfpathlineto{\pgfqpoint{1.638946in}{1.320549in}}%
\pgfpathlineto{\pgfqpoint{1.638946in}{1.320549in}}%
\pgfpathlineto{\pgfqpoint{1.639126in}{1.320874in}}%
\pgfpathlineto{\pgfqpoint{1.639244in}{1.320659in}}%
\pgfpathlineto{\pgfqpoint{1.639244in}{1.320659in}}%
\pgfpathlineto{\pgfqpoint{1.639520in}{1.319901in}}%
\pgfpathlineto{\pgfqpoint{1.639821in}{1.320697in}}%
\pgfpathlineto{\pgfqpoint{1.639821in}{1.320697in}}%
\pgfpathlineto{\pgfqpoint{1.639940in}{1.320866in}}%
\pgfpathlineto{\pgfqpoint{1.640059in}{1.320671in}}%
\pgfpathlineto{\pgfqpoint{1.640059in}{1.320671in}}%
\pgfpathlineto{\pgfqpoint{1.640313in}{1.319877in}}%
\pgfpathlineto{\pgfqpoint{1.640674in}{1.320763in}}%
\pgfpathlineto{\pgfqpoint{1.640674in}{1.320763in}}%
\pgfpathlineto{\pgfqpoint{1.640793in}{1.320847in}}%
\pgfpathlineto{\pgfqpoint{1.640852in}{1.320747in}}%
\pgfpathlineto{\pgfqpoint{1.640852in}{1.320747in}}%
\pgfpathlineto{\pgfqpoint{1.641931in}{1.319861in}}%
\pgfpathlineto{\pgfqpoint{1.642109in}{1.320137in}}%
\pgfpathlineto{\pgfqpoint{1.642411in}{1.320844in}}%
\pgfpathlineto{\pgfqpoint{1.642634in}{1.320117in}}%
\pgfpathlineto{\pgfqpoint{1.642634in}{1.320117in}}%
\pgfpathlineto{\pgfqpoint{1.642810in}{1.319870in}}%
\pgfpathlineto{\pgfqpoint{1.642929in}{1.320121in}}%
\pgfpathlineto{\pgfqpoint{1.642929in}{1.320121in}}%
\pgfpathlineto{\pgfqpoint{1.643232in}{1.320836in}}%
\pgfpathlineto{\pgfqpoint{1.643491in}{1.319988in}}%
\pgfpathlineto{\pgfqpoint{1.643491in}{1.319988in}}%
\pgfpathlineto{\pgfqpoint{1.643608in}{1.319843in}}%
\pgfpathlineto{\pgfqpoint{1.643727in}{1.320042in}}%
\pgfpathlineto{\pgfqpoint{1.643727in}{1.320042in}}%
\pgfpathlineto{\pgfqpoint{1.644905in}{1.320811in}}%
\pgfpathlineto{\pgfqpoint{1.645024in}{1.320500in}}%
\pgfpathlineto{\pgfqpoint{1.645270in}{1.319834in}}%
\pgfpathlineto{\pgfqpoint{1.645511in}{1.320448in}}%
\pgfpathlineto{\pgfqpoint{1.645511in}{1.320448in}}%
\pgfpathlineto{\pgfqpoint{1.645693in}{1.320813in}}%
\pgfpathlineto{\pgfqpoint{1.645877in}{1.320350in}}%
\pgfpathlineto{\pgfqpoint{1.645877in}{1.320350in}}%
\pgfpathlineto{\pgfqpoint{1.646084in}{1.319818in}}%
\pgfpathlineto{\pgfqpoint{1.646324in}{1.320403in}}%
\pgfpathlineto{\pgfqpoint{1.646324in}{1.320403in}}%
\pgfpathlineto{\pgfqpoint{1.646567in}{1.320785in}}%
\pgfpathlineto{\pgfqpoint{1.646714in}{1.320276in}}%
\pgfpathlineto{\pgfqpoint{1.646714in}{1.320276in}}%
\pgfpathlineto{\pgfqpoint{1.646916in}{1.319812in}}%
\pgfpathlineto{\pgfqpoint{1.647095in}{1.320214in}}%
\pgfpathlineto{\pgfqpoint{1.647095in}{1.320214in}}%
\pgfpathlineto{\pgfqpoint{1.647339in}{1.320796in}}%
\pgfpathlineto{\pgfqpoint{1.647557in}{1.320181in}}%
\pgfpathlineto{\pgfqpoint{1.647557in}{1.320181in}}%
\pgfpathlineto{\pgfqpoint{1.647733in}{1.319799in}}%
\pgfpathlineto{\pgfqpoint{1.647912in}{1.320176in}}%
\pgfpathlineto{\pgfqpoint{1.647912in}{1.320176in}}%
\pgfpathlineto{\pgfqpoint{1.648217in}{1.320773in}}%
\pgfpathlineto{\pgfqpoint{1.648372in}{1.320234in}}%
\pgfpathlineto{\pgfqpoint{1.648372in}{1.320234in}}%
\pgfpathlineto{\pgfqpoint{1.648539in}{1.319787in}}%
\pgfpathlineto{\pgfqpoint{1.648779in}{1.320295in}}%
\pgfpathlineto{\pgfqpoint{1.648779in}{1.320295in}}%
\pgfpathlineto{\pgfqpoint{1.649023in}{1.320781in}}%
\pgfpathlineto{\pgfqpoint{1.649201in}{1.320216in}}%
\pgfpathlineto{\pgfqpoint{1.649201in}{1.320216in}}%
\pgfpathlineto{\pgfqpoint{1.649377in}{1.319778in}}%
\pgfpathlineto{\pgfqpoint{1.649556in}{1.320123in}}%
\pgfpathlineto{\pgfqpoint{1.649556in}{1.320123in}}%
\pgfpathlineto{\pgfqpoint{1.649861in}{1.320768in}}%
\pgfpathlineto{\pgfqpoint{1.650033in}{1.320195in}}%
\pgfpathlineto{\pgfqpoint{1.650033in}{1.320195in}}%
\pgfpathlineto{\pgfqpoint{1.650209in}{1.319769in}}%
\pgfpathlineto{\pgfqpoint{1.650389in}{1.320126in}}%
\pgfpathlineto{\pgfqpoint{1.650389in}{1.320126in}}%
\pgfpathlineto{\pgfqpoint{1.650694in}{1.320757in}}%
\pgfpathlineto{\pgfqpoint{1.650902in}{1.320001in}}%
\pgfpathlineto{\pgfqpoint{1.650902in}{1.320001in}}%
\pgfpathlineto{\pgfqpoint{1.651020in}{1.319761in}}%
\pgfpathlineto{\pgfqpoint{1.651199in}{1.320058in}}%
\pgfpathlineto{\pgfqpoint{1.651199in}{1.320058in}}%
\pgfpathlineto{\pgfqpoint{1.651505in}{1.320760in}}%
\pgfpathlineto{\pgfqpoint{1.651712in}{1.320079in}}%
\pgfpathlineto{\pgfqpoint{1.651712in}{1.320079in}}%
\pgfpathlineto{\pgfqpoint{1.651889in}{1.319761in}}%
\pgfpathlineto{\pgfqpoint{1.652070in}{1.320175in}}%
\pgfpathlineto{\pgfqpoint{1.652070in}{1.320175in}}%
\pgfpathlineto{\pgfqpoint{1.652315in}{1.320752in}}%
\pgfpathlineto{\pgfqpoint{1.652548in}{1.320046in}}%
\pgfpathlineto{\pgfqpoint{1.652548in}{1.320046in}}%
\pgfpathlineto{\pgfqpoint{1.652726in}{1.319756in}}%
\pgfpathlineto{\pgfqpoint{1.652845in}{1.319996in}}%
\pgfpathlineto{\pgfqpoint{1.652845in}{1.319996in}}%
\pgfpathlineto{\pgfqpoint{1.653151in}{1.320746in}}%
\pgfpathlineto{\pgfqpoint{1.653405in}{1.319934in}}%
\pgfpathlineto{\pgfqpoint{1.653405in}{1.319934in}}%
\pgfpathlineto{\pgfqpoint{1.653524in}{1.319732in}}%
\pgfpathlineto{\pgfqpoint{1.653643in}{1.319896in}}%
\pgfpathlineto{\pgfqpoint{1.653643in}{1.319896in}}%
\pgfpathlineto{\pgfqpoint{1.654858in}{1.320713in}}%
\pgfpathlineto{\pgfqpoint{1.654978in}{1.320368in}}%
\pgfpathlineto{\pgfqpoint{1.655216in}{1.319723in}}%
\pgfpathlineto{\pgfqpoint{1.655458in}{1.320338in}}%
\pgfpathlineto{\pgfqpoint{1.655458in}{1.320338in}}%
\pgfpathlineto{\pgfqpoint{1.655643in}{1.320721in}}%
\pgfpathlineto{\pgfqpoint{1.655839in}{1.320213in}}%
\pgfpathlineto{\pgfqpoint{1.655839in}{1.320213in}}%
\pgfpathlineto{\pgfqpoint{1.656050in}{1.319714in}}%
\pgfpathlineto{\pgfqpoint{1.656231in}{1.320128in}}%
\pgfpathlineto{\pgfqpoint{1.656231in}{1.320128in}}%
\pgfpathlineto{\pgfqpoint{1.656477in}{1.320714in}}%
\pgfpathlineto{\pgfqpoint{1.656687in}{1.320130in}}%
\pgfpathlineto{\pgfqpoint{1.656687in}{1.320130in}}%
\pgfpathlineto{\pgfqpoint{1.656865in}{1.319696in}}%
\pgfpathlineto{\pgfqpoint{1.657046in}{1.320056in}}%
\pgfpathlineto{\pgfqpoint{1.657046in}{1.320056in}}%
\pgfpathlineto{\pgfqpoint{1.657353in}{1.320696in}}%
\pgfpathlineto{\pgfqpoint{1.657521in}{1.320125in}}%
\pgfpathlineto{\pgfqpoint{1.657521in}{1.320125in}}%
\pgfpathlineto{\pgfqpoint{1.657699in}{1.319687in}}%
\pgfpathlineto{\pgfqpoint{1.657880in}{1.320045in}}%
\pgfpathlineto{\pgfqpoint{1.657880in}{1.320045in}}%
\pgfpathlineto{\pgfqpoint{1.658188in}{1.320689in}}%
\pgfpathlineto{\pgfqpoint{1.658355in}{1.320128in}}%
\pgfpathlineto{\pgfqpoint{1.658355in}{1.320128in}}%
\pgfpathlineto{\pgfqpoint{1.658529in}{1.319677in}}%
\pgfpathlineto{\pgfqpoint{1.658771in}{1.320213in}}%
\pgfpathlineto{\pgfqpoint{1.658771in}{1.320213in}}%
\pgfpathlineto{\pgfqpoint{1.659018in}{1.320687in}}%
\pgfpathlineto{\pgfqpoint{1.659185in}{1.320157in}}%
\pgfpathlineto{\pgfqpoint{1.659185in}{1.320157in}}%
\pgfpathlineto{\pgfqpoint{1.659384in}{1.319673in}}%
\pgfpathlineto{\pgfqpoint{1.659566in}{1.320065in}}%
\pgfpathlineto{\pgfqpoint{1.659566in}{1.320065in}}%
\pgfpathlineto{\pgfqpoint{1.659874in}{1.320663in}}%
\pgfpathlineto{\pgfqpoint{1.660029in}{1.320111in}}%
\pgfpathlineto{\pgfqpoint{1.660029in}{1.320111in}}%
\pgfpathlineto{\pgfqpoint{1.660202in}{1.319659in}}%
\pgfpathlineto{\pgfqpoint{1.660445in}{1.320194in}}%
\pgfpathlineto{\pgfqpoint{1.660445in}{1.320194in}}%
\pgfpathlineto{\pgfqpoint{1.660692in}{1.320672in}}%
\pgfpathlineto{\pgfqpoint{1.660860in}{1.320144in}}%
\pgfpathlineto{\pgfqpoint{1.660860in}{1.320144in}}%
\pgfpathlineto{\pgfqpoint{1.661055in}{1.319652in}}%
\pgfpathlineto{\pgfqpoint{1.661237in}{1.320033in}}%
\pgfpathlineto{\pgfqpoint{1.661237in}{1.320033in}}%
\pgfpathlineto{\pgfqpoint{1.661546in}{1.320652in}}%
\pgfpathlineto{\pgfqpoint{1.661740in}{1.319928in}}%
\pgfpathlineto{\pgfqpoint{1.661740in}{1.319928in}}%
\pgfpathlineto{\pgfqpoint{1.661920in}{1.319660in}}%
\pgfpathlineto{\pgfqpoint{1.662041in}{1.319914in}}%
\pgfpathlineto{\pgfqpoint{1.662041in}{1.319914in}}%
\pgfpathlineto{\pgfqpoint{1.662350in}{1.320662in}}%
\pgfpathlineto{\pgfqpoint{1.662585in}{1.319899in}}%
\pgfpathlineto{\pgfqpoint{1.662585in}{1.319899in}}%
\pgfpathlineto{\pgfqpoint{1.662704in}{1.319634in}}%
\pgfpathlineto{\pgfqpoint{1.662886in}{1.319920in}}%
\pgfpathlineto{\pgfqpoint{1.662886in}{1.319920in}}%
\pgfpathlineto{\pgfqpoint{1.663195in}{1.320654in}}%
\pgfpathlineto{\pgfqpoint{1.663447in}{1.319808in}}%
\pgfpathlineto{\pgfqpoint{1.663447in}{1.319808in}}%
\pgfpathlineto{\pgfqpoint{1.663567in}{1.319622in}}%
\pgfpathlineto{\pgfqpoint{1.663688in}{1.319803in}}%
\pgfpathlineto{\pgfqpoint{1.663688in}{1.319803in}}%
\pgfpathlineto{\pgfqpoint{1.664890in}{1.320635in}}%
\pgfpathlineto{\pgfqpoint{1.665011in}{1.320357in}}%
\pgfpathlineto{\pgfqpoint{1.665253in}{1.319604in}}%
\pgfpathlineto{\pgfqpoint{1.665562in}{1.320378in}}%
\pgfpathlineto{\pgfqpoint{1.665562in}{1.320378in}}%
\pgfpathlineto{\pgfqpoint{1.665746in}{1.320618in}}%
\pgfpathlineto{\pgfqpoint{1.665868in}{1.320288in}}%
\pgfpathlineto{\pgfqpoint{1.665868in}{1.320288in}}%
\pgfpathlineto{\pgfqpoint{1.666106in}{1.319599in}}%
\pgfpathlineto{\pgfqpoint{1.666414in}{1.320396in}}%
\pgfpathlineto{\pgfqpoint{1.666414in}{1.320396in}}%
\pgfpathlineto{\pgfqpoint{1.666599in}{1.320601in}}%
\pgfpathlineto{\pgfqpoint{1.666659in}{1.320478in}}%
\pgfpathlineto{\pgfqpoint{1.666659in}{1.320478in}}%
\pgfpathlineto{\pgfqpoint{1.667792in}{1.319579in}}%
\pgfpathlineto{\pgfqpoint{1.667914in}{1.319789in}}%
\pgfpathlineto{\pgfqpoint{1.669080in}{1.320597in}}%
\pgfpathlineto{\pgfqpoint{1.669202in}{1.320430in}}%
\pgfpathlineto{\pgfqpoint{1.670344in}{1.319560in}}%
\pgfpathlineto{\pgfqpoint{1.670466in}{1.319801in}}%
\pgfpathlineto{\pgfqpoint{1.670778in}{1.320583in}}%
\pgfpathlineto{\pgfqpoint{1.671087in}{1.319612in}}%
\pgfpathlineto{\pgfqpoint{1.671087in}{1.319612in}}%
\pgfpathlineto{\pgfqpoint{1.671208in}{1.319567in}}%
\pgfpathlineto{\pgfqpoint{1.671269in}{1.319675in}}%
\pgfpathlineto{\pgfqpoint{1.671269in}{1.319675in}}%
\pgfpathlineto{\pgfqpoint{1.672526in}{1.320543in}}%
\pgfpathlineto{\pgfqpoint{1.672587in}{1.320412in}}%
\pgfpathlineto{\pgfqpoint{1.673734in}{1.319519in}}%
\pgfpathlineto{\pgfqpoint{1.673857in}{1.319753in}}%
\pgfpathlineto{\pgfqpoint{1.674170in}{1.320551in}}%
\pgfpathlineto{\pgfqpoint{1.674451in}{1.319660in}}%
\pgfpathlineto{\pgfqpoint{1.674451in}{1.319660in}}%
\pgfpathlineto{\pgfqpoint{1.674572in}{1.319503in}}%
\pgfpathlineto{\pgfqpoint{1.674695in}{1.319709in}}%
\pgfpathlineto{\pgfqpoint{1.674695in}{1.319709in}}%
\pgfpathlineto{\pgfqpoint{1.675913in}{1.320527in}}%
\pgfpathlineto{\pgfqpoint{1.675974in}{1.320425in}}%
\pgfpathlineto{\pgfqpoint{1.677121in}{1.319473in}}%
\pgfpathlineto{\pgfqpoint{1.677244in}{1.319662in}}%
\pgfpathlineto{\pgfqpoint{1.678473in}{1.320503in}}%
\pgfpathlineto{\pgfqpoint{1.678534in}{1.320400in}}%
\pgfpathlineto{\pgfqpoint{1.679689in}{1.319445in}}%
\pgfpathlineto{\pgfqpoint{1.679812in}{1.319647in}}%
\pgfpathlineto{\pgfqpoint{1.681041in}{1.320478in}}%
\pgfpathlineto{\pgfqpoint{1.681103in}{1.320372in}}%
\pgfpathlineto{\pgfqpoint{1.682272in}{1.319423in}}%
\pgfpathlineto{\pgfqpoint{1.682396in}{1.319654in}}%
\pgfpathlineto{\pgfqpoint{1.682711in}{1.320471in}}%
\pgfpathlineto{\pgfqpoint{1.683010in}{1.319532in}}%
\pgfpathlineto{\pgfqpoint{1.683010in}{1.319532in}}%
\pgfpathlineto{\pgfqpoint{1.683133in}{1.319415in}}%
\pgfpathlineto{\pgfqpoint{1.683257in}{1.319652in}}%
\pgfpathlineto{\pgfqpoint{1.683257in}{1.319652in}}%
\pgfpathlineto{\pgfqpoint{1.683573in}{1.320464in}}%
\pgfpathlineto{\pgfqpoint{1.683903in}{1.319446in}}%
\pgfpathlineto{\pgfqpoint{1.683903in}{1.319446in}}%
\pgfpathlineto{\pgfqpoint{1.684026in}{1.319442in}}%
\pgfpathlineto{\pgfqpoint{1.684088in}{1.319568in}}%
\pgfpathlineto{\pgfqpoint{1.684088in}{1.319568in}}%
\pgfpathlineto{\pgfqpoint{1.685317in}{1.320448in}}%
\pgfpathlineto{\pgfqpoint{1.685441in}{1.320183in}}%
\pgfpathlineto{\pgfqpoint{1.685689in}{1.319376in}}%
\pgfpathlineto{\pgfqpoint{1.686004in}{1.320162in}}%
\pgfpathlineto{\pgfqpoint{1.686004in}{1.320162in}}%
\pgfpathlineto{\pgfqpoint{1.686192in}{1.320433in}}%
\pgfpathlineto{\pgfqpoint{1.686317in}{1.320112in}}%
\pgfpathlineto{\pgfqpoint{1.686317in}{1.320112in}}%
\pgfpathlineto{\pgfqpoint{1.686553in}{1.319367in}}%
\pgfpathlineto{\pgfqpoint{1.686868in}{1.320162in}}%
\pgfpathlineto{\pgfqpoint{1.686868in}{1.320162in}}%
\pgfpathlineto{\pgfqpoint{1.687057in}{1.320423in}}%
\pgfpathlineto{\pgfqpoint{1.687181in}{1.320092in}}%
\pgfpathlineto{\pgfqpoint{1.687181in}{1.320092in}}%
\pgfpathlineto{\pgfqpoint{1.687435in}{1.319366in}}%
\pgfpathlineto{\pgfqpoint{1.687686in}{1.320013in}}%
\pgfpathlineto{\pgfqpoint{1.687686in}{1.320013in}}%
\pgfpathlineto{\pgfqpoint{1.687877in}{1.320423in}}%
\pgfpathlineto{\pgfqpoint{1.688082in}{1.319884in}}%
\pgfpathlineto{\pgfqpoint{1.688082in}{1.319884in}}%
\pgfpathlineto{\pgfqpoint{1.688275in}{1.319347in}}%
\pgfpathlineto{\pgfqpoint{1.688525in}{1.319923in}}%
\pgfpathlineto{\pgfqpoint{1.688525in}{1.319923in}}%
\pgfpathlineto{\pgfqpoint{1.688780in}{1.320409in}}%
\pgfpathlineto{\pgfqpoint{1.688959in}{1.319800in}}%
\pgfpathlineto{\pgfqpoint{1.688959in}{1.319800in}}%
\pgfpathlineto{\pgfqpoint{1.689143in}{1.319338in}}%
\pgfpathlineto{\pgfqpoint{1.689330in}{1.319717in}}%
\pgfpathlineto{\pgfqpoint{1.689330in}{1.319717in}}%
\pgfpathlineto{\pgfqpoint{1.689649in}{1.320398in}}%
\pgfpathlineto{\pgfqpoint{1.689833in}{1.319741in}}%
\pgfpathlineto{\pgfqpoint{1.689833in}{1.319741in}}%
\pgfpathlineto{\pgfqpoint{1.690017in}{1.319331in}}%
\pgfpathlineto{\pgfqpoint{1.690205in}{1.319739in}}%
\pgfpathlineto{\pgfqpoint{1.690205in}{1.319739in}}%
\pgfpathlineto{\pgfqpoint{1.690523in}{1.320381in}}%
\pgfpathlineto{\pgfqpoint{1.690696in}{1.319742in}}%
\pgfpathlineto{\pgfqpoint{1.690696in}{1.319742in}}%
\pgfpathlineto{\pgfqpoint{1.690881in}{1.319321in}}%
\pgfpathlineto{\pgfqpoint{1.691068in}{1.319724in}}%
\pgfpathlineto{\pgfqpoint{1.691068in}{1.319724in}}%
\pgfpathlineto{\pgfqpoint{1.691387in}{1.320375in}}%
\pgfpathlineto{\pgfqpoint{1.691591in}{1.319598in}}%
\pgfpathlineto{\pgfqpoint{1.691591in}{1.319598in}}%
\pgfpathlineto{\pgfqpoint{1.691714in}{1.319312in}}%
\pgfpathlineto{\pgfqpoint{1.691901in}{1.319607in}}%
\pgfpathlineto{\pgfqpoint{1.691901in}{1.319607in}}%
\pgfpathlineto{\pgfqpoint{1.692220in}{1.320385in}}%
\pgfpathlineto{\pgfqpoint{1.692483in}{1.319489in}}%
\pgfpathlineto{\pgfqpoint{1.692483in}{1.319489in}}%
\pgfpathlineto{\pgfqpoint{1.692607in}{1.319299in}}%
\pgfpathlineto{\pgfqpoint{1.692732in}{1.319493in}}%
\pgfpathlineto{\pgfqpoint{1.692732in}{1.319493in}}%
\pgfpathlineto{\pgfqpoint{1.693942in}{1.320368in}}%
\pgfpathlineto{\pgfqpoint{1.694067in}{1.320184in}}%
\pgfpathlineto{\pgfqpoint{1.694354in}{1.319283in}}%
\pgfpathlineto{\pgfqpoint{1.694737in}{1.320275in}}%
\pgfpathlineto{\pgfqpoint{1.694737in}{1.320275in}}%
\pgfpathlineto{\pgfqpoint{1.694863in}{1.320339in}}%
\pgfpathlineto{\pgfqpoint{1.694925in}{1.320210in}}%
\pgfpathlineto{\pgfqpoint{1.694925in}{1.320210in}}%
\pgfpathlineto{\pgfqpoint{1.696074in}{1.319259in}}%
\pgfpathlineto{\pgfqpoint{1.696199in}{1.319430in}}%
\pgfpathlineto{\pgfqpoint{1.697420in}{1.320334in}}%
\pgfpathlineto{\pgfqpoint{1.697546in}{1.320160in}}%
\pgfpathlineto{\pgfqpoint{1.698692in}{1.319229in}}%
\pgfpathlineto{\pgfqpoint{1.698818in}{1.319406in}}%
\pgfpathlineto{\pgfqpoint{1.700080in}{1.320304in}}%
\pgfpathlineto{\pgfqpoint{1.700143in}{1.320204in}}%
\pgfpathlineto{\pgfqpoint{1.701349in}{1.319214in}}%
\pgfpathlineto{\pgfqpoint{1.701475in}{1.319471in}}%
\pgfpathlineto{\pgfqpoint{1.701798in}{1.320295in}}%
\pgfpathlineto{\pgfqpoint{1.702066in}{1.319414in}}%
\pgfpathlineto{\pgfqpoint{1.702066in}{1.319414in}}%
\pgfpathlineto{\pgfqpoint{1.702191in}{1.319190in}}%
\pgfpathlineto{\pgfqpoint{1.702317in}{1.319366in}}%
\pgfpathlineto{\pgfqpoint{1.702317in}{1.319366in}}%
\pgfpathlineto{\pgfqpoint{1.703557in}{1.320279in}}%
\pgfpathlineto{\pgfqpoint{1.703683in}{1.320061in}}%
\pgfpathlineto{\pgfqpoint{1.703953in}{1.319170in}}%
\pgfpathlineto{\pgfqpoint{1.704340in}{1.320160in}}%
\pgfpathlineto{\pgfqpoint{1.704340in}{1.320160in}}%
\pgfpathlineto{\pgfqpoint{1.704467in}{1.320260in}}%
\pgfpathlineto{\pgfqpoint{1.704530in}{1.320151in}}%
\pgfpathlineto{\pgfqpoint{1.704530in}{1.320151in}}%
\pgfpathlineto{\pgfqpoint{1.705691in}{1.319152in}}%
\pgfpathlineto{\pgfqpoint{1.705818in}{1.319296in}}%
\pgfpathlineto{\pgfqpoint{1.707113in}{1.320229in}}%
\pgfpathlineto{\pgfqpoint{1.707177in}{1.320108in}}%
\pgfpathlineto{\pgfqpoint{1.708356in}{1.319120in}}%
\pgfpathlineto{\pgfqpoint{1.708483in}{1.319321in}}%
\pgfpathlineto{\pgfqpoint{1.709711in}{1.320219in}}%
\pgfpathlineto{\pgfqpoint{1.709839in}{1.320038in}}%
\pgfpathlineto{\pgfqpoint{1.711004in}{1.319090in}}%
\pgfpathlineto{\pgfqpoint{1.711131in}{1.319282in}}%
\pgfpathlineto{\pgfqpoint{1.712394in}{1.320194in}}%
\pgfpathlineto{\pgfqpoint{1.712458in}{1.320113in}}%
\pgfpathlineto{\pgfqpoint{1.713651in}{1.319059in}}%
\pgfpathlineto{\pgfqpoint{1.713779in}{1.319224in}}%
\pgfpathlineto{\pgfqpoint{1.715086in}{1.320146in}}%
\pgfpathlineto{\pgfqpoint{1.715150in}{1.320008in}}%
\pgfpathlineto{\pgfqpoint{1.716317in}{1.319029in}}%
\pgfpathlineto{\pgfqpoint{1.716445in}{1.319193in}}%
\pgfpathlineto{\pgfqpoint{1.717714in}{1.320146in}}%
\pgfpathlineto{\pgfqpoint{1.717842in}{1.319908in}}%
\pgfpathlineto{\pgfqpoint{1.718118in}{1.319010in}}%
\pgfpathlineto{\pgfqpoint{1.718511in}{1.320035in}}%
\pgfpathlineto{\pgfqpoint{1.718511in}{1.320035in}}%
\pgfpathlineto{\pgfqpoint{1.718640in}{1.320121in}}%
\pgfpathlineto{\pgfqpoint{1.718705in}{1.319999in}}%
\pgfpathlineto{\pgfqpoint{1.718705in}{1.319999in}}%
\pgfpathlineto{\pgfqpoint{1.719926in}{1.319003in}}%
\pgfpathlineto{\pgfqpoint{1.719990in}{1.319101in}}%
\pgfpathlineto{\pgfqpoint{1.721314in}{1.320101in}}%
\pgfpathlineto{\pgfqpoint{1.721379in}{1.319992in}}%
\pgfpathlineto{\pgfqpoint{1.722579in}{1.318956in}}%
\pgfpathlineto{\pgfqpoint{1.722708in}{1.319149in}}%
\pgfpathlineto{\pgfqpoint{1.723988in}{1.320085in}}%
\pgfpathlineto{\pgfqpoint{1.724052in}{1.320003in}}%
\pgfpathlineto{\pgfqpoint{1.725270in}{1.318925in}}%
\pgfpathlineto{\pgfqpoint{1.725399in}{1.319115in}}%
\pgfpathlineto{\pgfqpoint{1.726682in}{1.320060in}}%
\pgfpathlineto{\pgfqpoint{1.726747in}{1.319981in}}%
\pgfpathlineto{\pgfqpoint{1.728012in}{1.318917in}}%
\pgfpathlineto{\pgfqpoint{1.728142in}{1.319206in}}%
\pgfpathlineto{\pgfqpoint{1.728474in}{1.320044in}}%
\pgfpathlineto{\pgfqpoint{1.728765in}{1.319036in}}%
\pgfpathlineto{\pgfqpoint{1.728765in}{1.319036in}}%
\pgfpathlineto{\pgfqpoint{1.728894in}{1.318891in}}%
\pgfpathlineto{\pgfqpoint{1.729024in}{1.319136in}}%
\pgfpathlineto{\pgfqpoint{1.729024in}{1.319136in}}%
\pgfpathlineto{\pgfqpoint{1.729356in}{1.320031in}}%
\pgfpathlineto{\pgfqpoint{1.729659in}{1.319055in}}%
\pgfpathlineto{\pgfqpoint{1.729659in}{1.319055in}}%
\pgfpathlineto{\pgfqpoint{1.729788in}{1.318876in}}%
\pgfpathlineto{\pgfqpoint{1.729918in}{1.319100in}}%
\pgfpathlineto{\pgfqpoint{1.729918in}{1.319100in}}%
\pgfpathlineto{\pgfqpoint{1.731209in}{1.320010in}}%
\pgfpathlineto{\pgfqpoint{1.731274in}{1.319903in}}%
\pgfpathlineto{\pgfqpoint{1.732494in}{1.318842in}}%
\pgfpathlineto{\pgfqpoint{1.732624in}{1.319047in}}%
\pgfpathlineto{\pgfqpoint{1.733884in}{1.319989in}}%
\pgfpathlineto{\pgfqpoint{1.734015in}{1.319809in}}%
\pgfpathlineto{\pgfqpoint{1.735250in}{1.318831in}}%
\pgfpathlineto{\pgfqpoint{1.735316in}{1.318941in}}%
\pgfpathlineto{\pgfqpoint{1.736621in}{1.319966in}}%
\pgfpathlineto{\pgfqpoint{1.736753in}{1.319742in}}%
\pgfpathlineto{\pgfqpoint{1.737034in}{1.318789in}}%
\pgfpathlineto{\pgfqpoint{1.737435in}{1.319838in}}%
\pgfpathlineto{\pgfqpoint{1.737435in}{1.319838in}}%
\pgfpathlineto{\pgfqpoint{1.737567in}{1.319946in}}%
\pgfpathlineto{\pgfqpoint{1.737632in}{1.319832in}}%
\pgfpathlineto{\pgfqpoint{1.737632in}{1.319832in}}%
\pgfpathlineto{\pgfqpoint{1.738888in}{1.318785in}}%
\pgfpathlineto{\pgfqpoint{1.738953in}{1.318889in}}%
\pgfpathlineto{\pgfqpoint{1.740283in}{1.319931in}}%
\pgfpathlineto{\pgfqpoint{1.740415in}{1.319648in}}%
\pgfpathlineto{\pgfqpoint{1.740702in}{1.318754in}}%
\pgfpathlineto{\pgfqpoint{1.741037in}{1.319679in}}%
\pgfpathlineto{\pgfqpoint{1.741037in}{1.319679in}}%
\pgfpathlineto{\pgfqpoint{1.741236in}{1.319892in}}%
\pgfpathlineto{\pgfqpoint{1.741302in}{1.319741in}}%
\pgfpathlineto{\pgfqpoint{1.741302in}{1.319741in}}%
\pgfpathlineto{\pgfqpoint{1.742526in}{1.318729in}}%
\pgfpathlineto{\pgfqpoint{1.742592in}{1.318810in}}%
\pgfpathlineto{\pgfqpoint{1.743937in}{1.319897in}}%
\pgfpathlineto{\pgfqpoint{1.744069in}{1.319646in}}%
\pgfpathlineto{\pgfqpoint{1.744317in}{1.318709in}}%
\pgfpathlineto{\pgfqpoint{1.744720in}{1.319692in}}%
\pgfpathlineto{\pgfqpoint{1.744720in}{1.319692in}}%
\pgfpathlineto{\pgfqpoint{1.744853in}{1.319888in}}%
\pgfpathlineto{\pgfqpoint{1.744986in}{1.319641in}}%
\pgfpathlineto{\pgfqpoint{1.744986in}{1.319641in}}%
\pgfpathlineto{\pgfqpoint{1.745271in}{1.318694in}}%
\pgfpathlineto{\pgfqpoint{1.745675in}{1.319771in}}%
\pgfpathlineto{\pgfqpoint{1.745675in}{1.319771in}}%
\pgfpathlineto{\pgfqpoint{1.745808in}{1.319862in}}%
\pgfpathlineto{\pgfqpoint{1.745874in}{1.319734in}}%
\pgfpathlineto{\pgfqpoint{1.745874in}{1.319734in}}%
\pgfpathlineto{\pgfqpoint{1.747099in}{1.318670in}}%
\pgfpathlineto{\pgfqpoint{1.747232in}{1.318869in}}%
\pgfpathlineto{\pgfqpoint{1.748572in}{1.319834in}}%
\pgfpathlineto{\pgfqpoint{1.748638in}{1.319702in}}%
\pgfpathlineto{\pgfqpoint{1.749871in}{1.318638in}}%
\pgfpathlineto{\pgfqpoint{1.749937in}{1.318703in}}%
\pgfpathlineto{\pgfqpoint{1.751335in}{1.319814in}}%
\pgfpathlineto{\pgfqpoint{1.751468in}{1.319441in}}%
\pgfpathlineto{\pgfqpoint{1.751719in}{1.318617in}}%
\pgfpathlineto{\pgfqpoint{1.752058in}{1.319513in}}%
\pgfpathlineto{\pgfqpoint{1.752058in}{1.319513in}}%
\pgfpathlineto{\pgfqpoint{1.752260in}{1.319804in}}%
\pgfpathlineto{\pgfqpoint{1.752394in}{1.319429in}}%
\pgfpathlineto{\pgfqpoint{1.752394in}{1.319429in}}%
\pgfpathlineto{\pgfqpoint{1.752628in}{1.318606in}}%
\pgfpathlineto{\pgfqpoint{1.752966in}{1.319447in}}%
\pgfpathlineto{\pgfqpoint{1.752966in}{1.319447in}}%
\pgfpathlineto{\pgfqpoint{1.753169in}{1.319806in}}%
\pgfpathlineto{\pgfqpoint{1.753323in}{1.319405in}}%
\pgfpathlineto{\pgfqpoint{1.753323in}{1.319405in}}%
\pgfpathlineto{\pgfqpoint{1.753543in}{1.318600in}}%
\pgfpathlineto{\pgfqpoint{1.753880in}{1.319395in}}%
\pgfpathlineto{\pgfqpoint{1.753880in}{1.319395in}}%
\pgfpathlineto{\pgfqpoint{1.754084in}{1.319799in}}%
\pgfpathlineto{\pgfqpoint{1.754218in}{1.319541in}}%
\pgfpathlineto{\pgfqpoint{1.754218in}{1.319541in}}%
\pgfpathlineto{\pgfqpoint{1.754470in}{1.318589in}}%
\pgfpathlineto{\pgfqpoint{1.754877in}{1.319595in}}%
\pgfpathlineto{\pgfqpoint{1.754877in}{1.319595in}}%
\pgfpathlineto{\pgfqpoint{1.755012in}{1.319790in}}%
\pgfpathlineto{\pgfqpoint{1.755146in}{1.319533in}}%
\pgfpathlineto{\pgfqpoint{1.755146in}{1.319533in}}%
\pgfpathlineto{\pgfqpoint{1.755426in}{1.318573in}}%
\pgfpathlineto{\pgfqpoint{1.755835in}{1.319658in}}%
\pgfpathlineto{\pgfqpoint{1.755835in}{1.319658in}}%
\pgfpathlineto{\pgfqpoint{1.755970in}{1.319769in}}%
\pgfpathlineto{\pgfqpoint{1.756037in}{1.319651in}}%
\pgfpathlineto{\pgfqpoint{1.756037in}{1.319651in}}%
\pgfpathlineto{\pgfqpoint{1.757312in}{1.318563in}}%
\pgfpathlineto{\pgfqpoint{1.757379in}{1.318663in}}%
\pgfpathlineto{\pgfqpoint{1.758774in}{1.319732in}}%
\pgfpathlineto{\pgfqpoint{1.758841in}{1.319592in}}%
\pgfpathlineto{\pgfqpoint{1.760069in}{1.318517in}}%
\pgfpathlineto{\pgfqpoint{1.760136in}{1.318559in}}%
\pgfpathlineto{\pgfqpoint{1.761520in}{1.319725in}}%
\pgfpathlineto{\pgfqpoint{1.761655in}{1.319522in}}%
\pgfpathlineto{\pgfqpoint{1.761927in}{1.318499in}}%
\pgfpathlineto{\pgfqpoint{1.762406in}{1.319671in}}%
\pgfpathlineto{\pgfqpoint{1.762406in}{1.319671in}}%
\pgfpathlineto{\pgfqpoint{1.762542in}{1.319649in}}%
\pgfpathlineto{\pgfqpoint{1.762629in}{1.319355in}}%
\pgfpathlineto{\pgfqpoint{1.762629in}{1.319355in}}%
\pgfpathlineto{\pgfqpoint{1.762908in}{1.318492in}}%
\pgfpathlineto{\pgfqpoint{1.763251in}{1.319454in}}%
\pgfpathlineto{\pgfqpoint{1.763251in}{1.319454in}}%
\pgfpathlineto{\pgfqpoint{1.763388in}{1.319705in}}%
\pgfpathlineto{\pgfqpoint{1.763523in}{1.319520in}}%
\pgfpathlineto{\pgfqpoint{1.763523in}{1.319520in}}%
\pgfpathlineto{\pgfqpoint{1.764797in}{1.318481in}}%
\pgfpathlineto{\pgfqpoint{1.764865in}{1.318592in}}%
\pgfpathlineto{\pgfqpoint{1.766237in}{1.319679in}}%
\pgfpathlineto{\pgfqpoint{1.766304in}{1.319590in}}%
\pgfpathlineto{\pgfqpoint{1.767602in}{1.318435in}}%
\pgfpathlineto{\pgfqpoint{1.767670in}{1.318526in}}%
\pgfpathlineto{\pgfqpoint{1.769046in}{1.319655in}}%
\pgfpathlineto{\pgfqpoint{1.769182in}{1.319397in}}%
\pgfpathlineto{\pgfqpoint{1.769466in}{1.318405in}}%
\pgfpathlineto{\pgfqpoint{1.769880in}{1.319510in}}%
\pgfpathlineto{\pgfqpoint{1.769880in}{1.319510in}}%
\pgfpathlineto{\pgfqpoint{1.770017in}{1.319636in}}%
\pgfpathlineto{\pgfqpoint{1.770086in}{1.319522in}}%
\pgfpathlineto{\pgfqpoint{1.770086in}{1.319522in}}%
\pgfpathlineto{\pgfqpoint{1.771349in}{1.318382in}}%
\pgfpathlineto{\pgfqpoint{1.771485in}{1.318579in}}%
\pgfpathlineto{\pgfqpoint{1.772868in}{1.319596in}}%
\pgfpathlineto{\pgfqpoint{1.772936in}{1.319454in}}%
\pgfpathlineto{\pgfqpoint{1.774179in}{1.318350in}}%
\pgfpathlineto{\pgfqpoint{1.774247in}{1.318384in}}%
\pgfpathlineto{\pgfqpoint{1.775703in}{1.319578in}}%
\pgfpathlineto{\pgfqpoint{1.775840in}{1.319189in}}%
\pgfpathlineto{\pgfqpoint{1.776072in}{1.318329in}}%
\pgfpathlineto{\pgfqpoint{1.776418in}{1.319170in}}%
\pgfpathlineto{\pgfqpoint{1.776418in}{1.319170in}}%
\pgfpathlineto{\pgfqpoint{1.776628in}{1.319581in}}%
\pgfpathlineto{\pgfqpoint{1.776806in}{1.319102in}}%
\pgfpathlineto{\pgfqpoint{1.776806in}{1.319102in}}%
\pgfpathlineto{\pgfqpoint{1.777061in}{1.318318in}}%
\pgfpathlineto{\pgfqpoint{1.777338in}{1.319051in}}%
\pgfpathlineto{\pgfqpoint{1.777338in}{1.319051in}}%
\pgfpathlineto{\pgfqpoint{1.777618in}{1.319547in}}%
\pgfpathlineto{\pgfqpoint{1.777785in}{1.318928in}}%
\pgfpathlineto{\pgfqpoint{1.777785in}{1.318928in}}%
\pgfpathlineto{\pgfqpoint{1.777967in}{1.318311in}}%
\pgfpathlineto{\pgfqpoint{1.778242in}{1.318868in}}%
\pgfpathlineto{\pgfqpoint{1.778242in}{1.318868in}}%
\pgfpathlineto{\pgfqpoint{1.778523in}{1.319563in}}%
\pgfpathlineto{\pgfqpoint{1.778743in}{1.318886in}}%
\pgfpathlineto{\pgfqpoint{1.778743in}{1.318886in}}%
\pgfpathlineto{\pgfqpoint{1.778979in}{1.318303in}}%
\pgfpathlineto{\pgfqpoint{1.779186in}{1.318824in}}%
\pgfpathlineto{\pgfqpoint{1.779186in}{1.318824in}}%
\pgfpathlineto{\pgfqpoint{1.779467in}{1.319551in}}%
\pgfpathlineto{\pgfqpoint{1.779729in}{1.318693in}}%
\pgfpathlineto{\pgfqpoint{1.779729in}{1.318693in}}%
\pgfpathlineto{\pgfqpoint{1.779933in}{1.318292in}}%
\pgfpathlineto{\pgfqpoint{1.780071in}{1.318580in}}%
\pgfpathlineto{\pgfqpoint{1.780071in}{1.318580in}}%
\pgfpathlineto{\pgfqpoint{1.780422in}{1.319542in}}%
\pgfpathlineto{\pgfqpoint{1.780722in}{1.318513in}}%
\pgfpathlineto{\pgfqpoint{1.780722in}{1.318513in}}%
\pgfpathlineto{\pgfqpoint{1.780858in}{1.318267in}}%
\pgfpathlineto{\pgfqpoint{1.780996in}{1.318480in}}%
\pgfpathlineto{\pgfqpoint{1.780996in}{1.318480in}}%
\pgfpathlineto{\pgfqpoint{1.782356in}{1.319524in}}%
\pgfpathlineto{\pgfqpoint{1.782425in}{1.319443in}}%
\pgfpathlineto{\pgfqpoint{1.783728in}{1.318232in}}%
\pgfpathlineto{\pgfqpoint{1.783866in}{1.318442in}}%
\pgfpathlineto{\pgfqpoint{1.785217in}{1.319497in}}%
\pgfpathlineto{\pgfqpoint{1.785287in}{1.319443in}}%
\pgfpathlineto{\pgfqpoint{1.786596in}{1.318199in}}%
\pgfpathlineto{\pgfqpoint{1.786735in}{1.318378in}}%
\pgfpathlineto{\pgfqpoint{1.788086in}{1.319463in}}%
\pgfpathlineto{\pgfqpoint{1.788156in}{1.319439in}}%
\pgfpathlineto{\pgfqpoint{1.789514in}{1.318166in}}%
\pgfpathlineto{\pgfqpoint{1.789653in}{1.318422in}}%
\pgfpathlineto{\pgfqpoint{1.790078in}{1.319427in}}%
\pgfpathlineto{\pgfqpoint{1.790368in}{1.318266in}}%
\pgfpathlineto{\pgfqpoint{1.790368in}{1.318266in}}%
\pgfpathlineto{\pgfqpoint{1.790506in}{1.318175in}}%
\pgfpathlineto{\pgfqpoint{1.790576in}{1.318297in}}%
\pgfpathlineto{\pgfqpoint{1.790576in}{1.318297in}}%
\pgfpathlineto{\pgfqpoint{1.791988in}{1.319427in}}%
\pgfpathlineto{\pgfqpoint{1.792057in}{1.319325in}}%
\pgfpathlineto{\pgfqpoint{1.793346in}{1.318119in}}%
\pgfpathlineto{\pgfqpoint{1.793485in}{1.318280in}}%
\pgfpathlineto{\pgfqpoint{1.794899in}{1.319396in}}%
\pgfpathlineto{\pgfqpoint{1.794969in}{1.319284in}}%
\pgfpathlineto{\pgfqpoint{1.796293in}{1.318084in}}%
\pgfpathlineto{\pgfqpoint{1.796363in}{1.318170in}}%
\pgfpathlineto{\pgfqpoint{1.797780in}{1.319373in}}%
\pgfpathlineto{\pgfqpoint{1.797921in}{1.319144in}}%
\pgfpathlineto{\pgfqpoint{1.798224in}{1.318056in}}%
\pgfpathlineto{\pgfqpoint{1.798652in}{1.319222in}}%
\pgfpathlineto{\pgfqpoint{1.798652in}{1.319222in}}%
\pgfpathlineto{\pgfqpoint{1.798793in}{1.319355in}}%
\pgfpathlineto{\pgfqpoint{1.798864in}{1.319234in}}%
\pgfpathlineto{\pgfqpoint{1.798864in}{1.319234in}}%
\pgfpathlineto{\pgfqpoint{1.800174in}{1.318032in}}%
\pgfpathlineto{\pgfqpoint{1.800244in}{1.318095in}}%
\pgfpathlineto{\pgfqpoint{1.801696in}{1.319336in}}%
\pgfpathlineto{\pgfqpoint{1.801838in}{1.319052in}}%
\pgfpathlineto{\pgfqpoint{1.802100in}{1.318015in}}%
\pgfpathlineto{\pgfqpoint{1.802529in}{1.319106in}}%
\pgfpathlineto{\pgfqpoint{1.802529in}{1.319106in}}%
\pgfpathlineto{\pgfqpoint{1.802672in}{1.319326in}}%
\pgfpathlineto{\pgfqpoint{1.802813in}{1.319054in}}%
\pgfpathlineto{\pgfqpoint{1.802813in}{1.319054in}}%
\pgfpathlineto{\pgfqpoint{1.803134in}{1.318006in}}%
\pgfpathlineto{\pgfqpoint{1.803492in}{1.319049in}}%
\pgfpathlineto{\pgfqpoint{1.803492in}{1.319049in}}%
\pgfpathlineto{\pgfqpoint{1.803635in}{1.319313in}}%
\pgfpathlineto{\pgfqpoint{1.803777in}{1.319104in}}%
\pgfpathlineto{\pgfqpoint{1.803777in}{1.319104in}}%
\pgfpathlineto{\pgfqpoint{1.804074in}{1.317984in}}%
\pgfpathlineto{\pgfqpoint{1.804576in}{1.319268in}}%
\pgfpathlineto{\pgfqpoint{1.804576in}{1.319268in}}%
\pgfpathlineto{\pgfqpoint{1.804718in}{1.319211in}}%
\pgfpathlineto{\pgfqpoint{1.804810in}{1.318867in}}%
\pgfpathlineto{\pgfqpoint{1.804810in}{1.318867in}}%
\pgfpathlineto{\pgfqpoint{1.805040in}{1.317979in}}%
\pgfpathlineto{\pgfqpoint{1.805397in}{1.318840in}}%
\pgfpathlineto{\pgfqpoint{1.805397in}{1.318840in}}%
\pgfpathlineto{\pgfqpoint{1.805613in}{1.319297in}}%
\pgfpathlineto{\pgfqpoint{1.805755in}{1.319025in}}%
\pgfpathlineto{\pgfqpoint{1.805755in}{1.319025in}}%
\pgfpathlineto{\pgfqpoint{1.806025in}{1.317965in}}%
\pgfpathlineto{\pgfqpoint{1.806456in}{1.319073in}}%
\pgfpathlineto{\pgfqpoint{1.806456in}{1.319073in}}%
\pgfpathlineto{\pgfqpoint{1.806599in}{1.319287in}}%
\pgfpathlineto{\pgfqpoint{1.806741in}{1.319003in}}%
\pgfpathlineto{\pgfqpoint{1.806741in}{1.319003in}}%
\pgfpathlineto{\pgfqpoint{1.807068in}{1.317963in}}%
\pgfpathlineto{\pgfqpoint{1.807429in}{1.319031in}}%
\pgfpathlineto{\pgfqpoint{1.807429in}{1.319031in}}%
\pgfpathlineto{\pgfqpoint{1.807572in}{1.319276in}}%
\pgfpathlineto{\pgfqpoint{1.807714in}{1.319036in}}%
\pgfpathlineto{\pgfqpoint{1.807714in}{1.319036in}}%
\pgfpathlineto{\pgfqpoint{1.807998in}{1.317938in}}%
\pgfpathlineto{\pgfqpoint{1.808430in}{1.319068in}}%
\pgfpathlineto{\pgfqpoint{1.808430in}{1.319068in}}%
\pgfpathlineto{\pgfqpoint{1.808573in}{1.319267in}}%
\pgfpathlineto{\pgfqpoint{1.808715in}{1.318960in}}%
\pgfpathlineto{\pgfqpoint{1.808715in}{1.318960in}}%
\pgfpathlineto{\pgfqpoint{1.809036in}{1.317938in}}%
\pgfpathlineto{\pgfqpoint{1.809398in}{1.319008in}}%
\pgfpathlineto{\pgfqpoint{1.809398in}{1.319008in}}%
\pgfpathlineto{\pgfqpoint{1.809542in}{1.319257in}}%
\pgfpathlineto{\pgfqpoint{1.809684in}{1.319019in}}%
\pgfpathlineto{\pgfqpoint{1.809684in}{1.319019in}}%
\pgfpathlineto{\pgfqpoint{1.809983in}{1.317910in}}%
\pgfpathlineto{\pgfqpoint{1.810416in}{1.319083in}}%
\pgfpathlineto{\pgfqpoint{1.810416in}{1.319083in}}%
\pgfpathlineto{\pgfqpoint{1.810560in}{1.319243in}}%
\pgfpathlineto{\pgfqpoint{1.810631in}{1.319136in}}%
\pgfpathlineto{\pgfqpoint{1.810631in}{1.319136in}}%
\pgfpathlineto{\pgfqpoint{1.811952in}{1.317887in}}%
\pgfpathlineto{\pgfqpoint{1.812023in}{1.317929in}}%
\pgfpathlineto{\pgfqpoint{1.813529in}{1.319213in}}%
\pgfpathlineto{\pgfqpoint{1.813672in}{1.318840in}}%
\pgfpathlineto{\pgfqpoint{1.813938in}{1.317861in}}%
\pgfpathlineto{\pgfqpoint{1.814301in}{1.318820in}}%
\pgfpathlineto{\pgfqpoint{1.814301in}{1.318820in}}%
\pgfpathlineto{\pgfqpoint{1.814518in}{1.319204in}}%
\pgfpathlineto{\pgfqpoint{1.814683in}{1.318733in}}%
\pgfpathlineto{\pgfqpoint{1.814683in}{1.318733in}}%
\pgfpathlineto{\pgfqpoint{1.814941in}{1.317849in}}%
\pgfpathlineto{\pgfqpoint{1.815229in}{1.318583in}}%
\pgfpathlineto{\pgfqpoint{1.815229in}{1.318583in}}%
\pgfpathlineto{\pgfqpoint{1.815522in}{1.319187in}}%
\pgfpathlineto{\pgfqpoint{1.815720in}{1.318453in}}%
\pgfpathlineto{\pgfqpoint{1.815720in}{1.318453in}}%
\pgfpathlineto{\pgfqpoint{1.815912in}{1.317840in}}%
\pgfpathlineto{\pgfqpoint{1.816199in}{1.318486in}}%
\pgfpathlineto{\pgfqpoint{1.816199in}{1.318486in}}%
\pgfpathlineto{\pgfqpoint{1.816493in}{1.319189in}}%
\pgfpathlineto{\pgfqpoint{1.816716in}{1.318429in}}%
\pgfpathlineto{\pgfqpoint{1.816716in}{1.318429in}}%
\pgfpathlineto{\pgfqpoint{1.816920in}{1.317824in}}%
\pgfpathlineto{\pgfqpoint{1.817135in}{1.318261in}}%
\pgfpathlineto{\pgfqpoint{1.817135in}{1.318261in}}%
\pgfpathlineto{\pgfqpoint{1.817501in}{1.319174in}}%
\pgfpathlineto{\pgfqpoint{1.817720in}{1.318367in}}%
\pgfpathlineto{\pgfqpoint{1.817720in}{1.318367in}}%
\pgfpathlineto{\pgfqpoint{1.817932in}{1.317814in}}%
\pgfpathlineto{\pgfqpoint{1.818148in}{1.318312in}}%
\pgfpathlineto{\pgfqpoint{1.818148in}{1.318312in}}%
\pgfpathlineto{\pgfqpoint{1.818515in}{1.319150in}}%
\pgfpathlineto{\pgfqpoint{1.818748in}{1.318184in}}%
\pgfpathlineto{\pgfqpoint{1.818748in}{1.318184in}}%
\pgfpathlineto{\pgfqpoint{1.818961in}{1.317825in}}%
\pgfpathlineto{\pgfqpoint{1.819105in}{1.318163in}}%
\pgfpathlineto{\pgfqpoint{1.819105in}{1.318163in}}%
\pgfpathlineto{\pgfqpoint{1.819472in}{1.319160in}}%
\pgfpathlineto{\pgfqpoint{1.819729in}{1.318257in}}%
\pgfpathlineto{\pgfqpoint{1.819729in}{1.318257in}}%
\pgfpathlineto{\pgfqpoint{1.819942in}{1.317797in}}%
\pgfpathlineto{\pgfqpoint{1.820158in}{1.318346in}}%
\pgfpathlineto{\pgfqpoint{1.820158in}{1.318346in}}%
\pgfpathlineto{\pgfqpoint{1.820453in}{1.319146in}}%
\pgfpathlineto{\pgfqpoint{1.820712in}{1.318333in}}%
\pgfpathlineto{\pgfqpoint{1.820712in}{1.318333in}}%
\pgfpathlineto{\pgfqpoint{1.820925in}{1.317776in}}%
\pgfpathlineto{\pgfqpoint{1.821141in}{1.318277in}}%
\pgfpathlineto{\pgfqpoint{1.821141in}{1.318277in}}%
\pgfpathlineto{\pgfqpoint{1.821510in}{1.319120in}}%
\pgfpathlineto{\pgfqpoint{1.821717in}{1.318289in}}%
\pgfpathlineto{\pgfqpoint{1.821717in}{1.318289in}}%
\pgfpathlineto{\pgfqpoint{1.821930in}{1.317766in}}%
\pgfpathlineto{\pgfqpoint{1.822147in}{1.318286in}}%
\pgfpathlineto{\pgfqpoint{1.822147in}{1.318286in}}%
\pgfpathlineto{\pgfqpoint{1.822516in}{1.319104in}}%
\pgfpathlineto{\pgfqpoint{1.822705in}{1.318351in}}%
\pgfpathlineto{\pgfqpoint{1.822705in}{1.318351in}}%
\pgfpathlineto{\pgfqpoint{1.822918in}{1.317749in}}%
\pgfpathlineto{\pgfqpoint{1.823135in}{1.318225in}}%
\pgfpathlineto{\pgfqpoint{1.823135in}{1.318225in}}%
\pgfpathlineto{\pgfqpoint{1.823504in}{1.319107in}}%
\pgfpathlineto{\pgfqpoint{1.823705in}{1.318353in}}%
\pgfpathlineto{\pgfqpoint{1.823705in}{1.318353in}}%
\pgfpathlineto{\pgfqpoint{1.823906in}{1.317737in}}%
\pgfpathlineto{\pgfqpoint{1.824197in}{1.318433in}}%
\pgfpathlineto{\pgfqpoint{1.824197in}{1.318433in}}%
\pgfpathlineto{\pgfqpoint{1.824492in}{1.319106in}}%
\pgfpathlineto{\pgfqpoint{1.824700in}{1.318390in}}%
\pgfpathlineto{\pgfqpoint{1.824700in}{1.318390in}}%
\pgfpathlineto{\pgfqpoint{1.824935in}{1.317728in}}%
\pgfpathlineto{\pgfqpoint{1.825153in}{1.318246in}}%
\pgfpathlineto{\pgfqpoint{1.825153in}{1.318246in}}%
\pgfpathlineto{\pgfqpoint{1.825523in}{1.319075in}}%
\pgfpathlineto{\pgfqpoint{1.825710in}{1.318342in}}%
\pgfpathlineto{\pgfqpoint{1.825710in}{1.318342in}}%
\pgfpathlineto{\pgfqpoint{1.825903in}{1.317715in}}%
\pgfpathlineto{\pgfqpoint{1.826193in}{1.318367in}}%
\pgfpathlineto{\pgfqpoint{1.826193in}{1.318367in}}%
\pgfpathlineto{\pgfqpoint{1.826490in}{1.319089in}}%
\pgfpathlineto{\pgfqpoint{1.826718in}{1.318311in}}%
\pgfpathlineto{\pgfqpoint{1.826718in}{1.318311in}}%
\pgfpathlineto{\pgfqpoint{1.826927in}{1.317698in}}%
\pgfpathlineto{\pgfqpoint{1.827145in}{1.318158in}}%
\pgfpathlineto{\pgfqpoint{1.827145in}{1.318158in}}%
\pgfpathlineto{\pgfqpoint{1.827516in}{1.319071in}}%
\pgfpathlineto{\pgfqpoint{1.827737in}{1.318226in}}%
\pgfpathlineto{\pgfqpoint{1.827737in}{1.318226in}}%
\pgfpathlineto{\pgfqpoint{1.827952in}{1.317690in}}%
\pgfpathlineto{\pgfqpoint{1.828170in}{1.318212in}}%
\pgfpathlineto{\pgfqpoint{1.828170in}{1.318212in}}%
\pgfpathlineto{\pgfqpoint{1.828541in}{1.319045in}}%
\pgfpathlineto{\pgfqpoint{1.828756in}{1.318151in}}%
\pgfpathlineto{\pgfqpoint{1.828756in}{1.318151in}}%
\pgfpathlineto{\pgfqpoint{1.828971in}{1.317684in}}%
\pgfpathlineto{\pgfqpoint{1.829190in}{1.318244in}}%
\pgfpathlineto{\pgfqpoint{1.829190in}{1.318244in}}%
\pgfpathlineto{\pgfqpoint{1.829488in}{1.319056in}}%
\pgfpathlineto{\pgfqpoint{1.829736in}{1.318302in}}%
\pgfpathlineto{\pgfqpoint{1.829736in}{1.318302in}}%
\pgfpathlineto{\pgfqpoint{1.829925in}{1.317668in}}%
\pgfpathlineto{\pgfqpoint{1.830217in}{1.318297in}}%
\pgfpathlineto{\pgfqpoint{1.830217in}{1.318297in}}%
\pgfpathlineto{\pgfqpoint{1.830515in}{1.319050in}}%
\pgfpathlineto{\pgfqpoint{1.830749in}{1.318276in}}%
\pgfpathlineto{\pgfqpoint{1.830749in}{1.318276in}}%
\pgfpathlineto{\pgfqpoint{1.830949in}{1.317649in}}%
\pgfpathlineto{\pgfqpoint{1.831242in}{1.318341in}}%
\pgfpathlineto{\pgfqpoint{1.831242in}{1.318341in}}%
\pgfpathlineto{\pgfqpoint{1.831540in}{1.319037in}}%
\pgfpathlineto{\pgfqpoint{1.831761in}{1.318257in}}%
\pgfpathlineto{\pgfqpoint{1.831761in}{1.318257in}}%
\pgfpathlineto{\pgfqpoint{1.831967in}{1.317635in}}%
\pgfpathlineto{\pgfqpoint{1.832186in}{1.318084in}}%
\pgfpathlineto{\pgfqpoint{1.832186in}{1.318084in}}%
\pgfpathlineto{\pgfqpoint{1.832559in}{1.319024in}}%
\pgfpathlineto{\pgfqpoint{1.832787in}{1.318162in}}%
\pgfpathlineto{\pgfqpoint{1.832787in}{1.318162in}}%
\pgfpathlineto{\pgfqpoint{1.833003in}{1.317627in}}%
\pgfpathlineto{\pgfqpoint{1.833223in}{1.318157in}}%
\pgfpathlineto{\pgfqpoint{1.833223in}{1.318157in}}%
\pgfpathlineto{\pgfqpoint{1.833596in}{1.318993in}}%
\pgfpathlineto{\pgfqpoint{1.833786in}{1.318231in}}%
\pgfpathlineto{\pgfqpoint{1.833786in}{1.318231in}}%
\pgfpathlineto{\pgfqpoint{1.833995in}{1.317609in}}%
\pgfpathlineto{\pgfqpoint{1.834215in}{1.318068in}}%
\pgfpathlineto{\pgfqpoint{1.834215in}{1.318068in}}%
\pgfpathlineto{\pgfqpoint{1.834589in}{1.319002in}}%
\pgfpathlineto{\pgfqpoint{1.834856in}{1.317926in}}%
\pgfpathlineto{\pgfqpoint{1.834856in}{1.317926in}}%
\pgfpathlineto{\pgfqpoint{1.835001in}{1.317598in}}%
\pgfpathlineto{\pgfqpoint{1.835220in}{1.318022in}}%
\pgfpathlineto{\pgfqpoint{1.835220in}{1.318022in}}%
\pgfpathlineto{\pgfqpoint{1.835594in}{1.318996in}}%
\pgfpathlineto{\pgfqpoint{1.835858in}{1.317977in}}%
\pgfpathlineto{\pgfqpoint{1.835858in}{1.317977in}}%
\pgfpathlineto{\pgfqpoint{1.836003in}{1.317591in}}%
\pgfpathlineto{\pgfqpoint{1.836223in}{1.317963in}}%
\pgfpathlineto{\pgfqpoint{1.836223in}{1.317963in}}%
\pgfpathlineto{\pgfqpoint{1.836597in}{1.318989in}}%
\pgfpathlineto{\pgfqpoint{1.836853in}{1.318079in}}%
\pgfpathlineto{\pgfqpoint{1.836853in}{1.318079in}}%
\pgfpathlineto{\pgfqpoint{1.837071in}{1.317579in}}%
\pgfpathlineto{\pgfqpoint{1.837291in}{1.318134in}}%
\pgfpathlineto{\pgfqpoint{1.837291in}{1.318134in}}%
\pgfpathlineto{\pgfqpoint{1.837592in}{1.318972in}}%
\pgfpathlineto{\pgfqpoint{1.837847in}{1.318211in}}%
\pgfpathlineto{\pgfqpoint{1.837847in}{1.318211in}}%
\pgfpathlineto{\pgfqpoint{1.838036in}{1.317566in}}%
\pgfpathlineto{\pgfqpoint{1.838331in}{1.318199in}}%
\pgfpathlineto{\pgfqpoint{1.838331in}{1.318199in}}%
\pgfpathlineto{\pgfqpoint{1.838631in}{1.318968in}}%
\pgfpathlineto{\pgfqpoint{1.838870in}{1.318174in}}%
\pgfpathlineto{\pgfqpoint{1.838870in}{1.318174in}}%
\pgfpathlineto{\pgfqpoint{1.839079in}{1.317545in}}%
\pgfpathlineto{\pgfqpoint{1.839300in}{1.318004in}}%
\pgfpathlineto{\pgfqpoint{1.839300in}{1.318004in}}%
\pgfpathlineto{\pgfqpoint{1.839676in}{1.318951in}}%
\pgfpathlineto{\pgfqpoint{1.839899in}{1.318111in}}%
\pgfpathlineto{\pgfqpoint{1.839899in}{1.318111in}}%
\pgfpathlineto{\pgfqpoint{1.840117in}{1.317534in}}%
\pgfpathlineto{\pgfqpoint{1.840338in}{1.318053in}}%
\pgfpathlineto{\pgfqpoint{1.840338in}{1.318053in}}%
\pgfpathlineto{\pgfqpoint{1.840714in}{1.318927in}}%
\pgfpathlineto{\pgfqpoint{1.840916in}{1.318124in}}%
\pgfpathlineto{\pgfqpoint{1.840916in}{1.318124in}}%
\pgfpathlineto{\pgfqpoint{1.841134in}{1.317520in}}%
\pgfpathlineto{\pgfqpoint{1.841356in}{1.318026in}}%
\pgfpathlineto{\pgfqpoint{1.841356in}{1.318026in}}%
\pgfpathlineto{\pgfqpoint{1.841733in}{1.318921in}}%
\pgfpathlineto{\pgfqpoint{1.841950in}{1.318047in}}%
\pgfpathlineto{\pgfqpoint{1.841950in}{1.318047in}}%
\pgfpathlineto{\pgfqpoint{1.842168in}{1.317511in}}%
\pgfpathlineto{\pgfqpoint{1.842390in}{1.318055in}}%
\pgfpathlineto{\pgfqpoint{1.842390in}{1.318055in}}%
\pgfpathlineto{\pgfqpoint{1.842767in}{1.318899in}}%
\pgfpathlineto{\pgfqpoint{1.842957in}{1.318136in}}%
\pgfpathlineto{\pgfqpoint{1.842957in}{1.318136in}}%
\pgfpathlineto{\pgfqpoint{1.843160in}{1.317494in}}%
\pgfpathlineto{\pgfqpoint{1.843457in}{1.318201in}}%
\pgfpathlineto{\pgfqpoint{1.843457in}{1.318201in}}%
\pgfpathlineto{\pgfqpoint{1.843759in}{1.318914in}}%
\pgfpathlineto{\pgfqpoint{1.843978in}{1.318149in}}%
\pgfpathlineto{\pgfqpoint{1.843978in}{1.318149in}}%
\pgfpathlineto{\pgfqpoint{1.844235in}{1.317497in}}%
\pgfpathlineto{\pgfqpoint{1.844458in}{1.318096in}}%
\pgfpathlineto{\pgfqpoint{1.844458in}{1.318096in}}%
\pgfpathlineto{\pgfqpoint{1.844761in}{1.318905in}}%
\pgfpathlineto{\pgfqpoint{1.845005in}{1.318136in}}%
\pgfpathlineto{\pgfqpoint{1.845005in}{1.318136in}}%
\pgfpathlineto{\pgfqpoint{1.845263in}{1.317485in}}%
\pgfpathlineto{\pgfqpoint{1.845486in}{1.318088in}}%
\pgfpathlineto{\pgfqpoint{1.845486in}{1.318088in}}%
\pgfpathlineto{\pgfqpoint{1.845789in}{1.318895in}}%
\pgfpathlineto{\pgfqpoint{1.846036in}{1.318097in}}%
\pgfpathlineto{\pgfqpoint{1.846036in}{1.318097in}}%
\pgfpathlineto{\pgfqpoint{1.846243in}{1.317454in}}%
\pgfpathlineto{\pgfqpoint{1.846541in}{1.318178in}}%
\pgfpathlineto{\pgfqpoint{1.846541in}{1.318178in}}%
\pgfpathlineto{\pgfqpoint{1.846844in}{1.318882in}}%
\pgfpathlineto{\pgfqpoint{1.847058in}{1.318131in}}%
\pgfpathlineto{\pgfqpoint{1.847058in}{1.318131in}}%
\pgfpathlineto{\pgfqpoint{1.847302in}{1.317446in}}%
\pgfpathlineto{\pgfqpoint{1.847525in}{1.317999in}}%
\pgfpathlineto{\pgfqpoint{1.847525in}{1.317999in}}%
\pgfpathlineto{\pgfqpoint{1.847830in}{1.318867in}}%
\pgfpathlineto{\pgfqpoint{1.848092in}{1.318091in}}%
\pgfpathlineto{\pgfqpoint{1.848092in}{1.318091in}}%
\pgfpathlineto{\pgfqpoint{1.848284in}{1.317435in}}%
\pgfpathlineto{\pgfqpoint{1.848582in}{1.318085in}}%
\pgfpathlineto{\pgfqpoint{1.848582in}{1.318085in}}%
\pgfpathlineto{\pgfqpoint{1.848886in}{1.318865in}}%
\pgfpathlineto{\pgfqpoint{1.849123in}{1.318080in}}%
\pgfpathlineto{\pgfqpoint{1.849123in}{1.318080in}}%
\pgfpathlineto{\pgfqpoint{1.849315in}{1.317422in}}%
\pgfpathlineto{\pgfqpoint{1.849613in}{1.318071in}}%
\pgfpathlineto{\pgfqpoint{1.849613in}{1.318071in}}%
\pgfpathlineto{\pgfqpoint{1.849917in}{1.318855in}}%
\pgfpathlineto{\pgfqpoint{1.850160in}{1.318036in}}%
\pgfpathlineto{\pgfqpoint{1.850160in}{1.318036in}}%
\pgfpathlineto{\pgfqpoint{1.850378in}{1.317401in}}%
\pgfpathlineto{\pgfqpoint{1.850602in}{1.317896in}}%
\pgfpathlineto{\pgfqpoint{1.850602in}{1.317896in}}%
\pgfpathlineto{\pgfqpoint{1.850983in}{1.318832in}}%
\pgfpathlineto{\pgfqpoint{1.851221in}{1.317865in}}%
\pgfpathlineto{\pgfqpoint{1.851221in}{1.317865in}}%
\pgfpathlineto{\pgfqpoint{1.851442in}{1.317401in}}%
\pgfpathlineto{\pgfqpoint{1.851591in}{1.317729in}}%
\pgfpathlineto{\pgfqpoint{1.851591in}{1.317729in}}%
\pgfpathlineto{\pgfqpoint{1.851972in}{1.318831in}}%
\pgfpathlineto{\pgfqpoint{1.852281in}{1.317728in}}%
\pgfpathlineto{\pgfqpoint{1.852281in}{1.317728in}}%
\pgfpathlineto{\pgfqpoint{1.852428in}{1.317376in}}%
\pgfpathlineto{\pgfqpoint{1.852652in}{1.317803in}}%
\pgfpathlineto{\pgfqpoint{1.852652in}{1.317803in}}%
\pgfpathlineto{\pgfqpoint{1.853033in}{1.318822in}}%
\pgfpathlineto{\pgfqpoint{1.853319in}{1.317706in}}%
\pgfpathlineto{\pgfqpoint{1.853319in}{1.317706in}}%
\pgfpathlineto{\pgfqpoint{1.853466in}{1.317362in}}%
\pgfpathlineto{\pgfqpoint{1.853691in}{1.317798in}}%
\pgfpathlineto{\pgfqpoint{1.853691in}{1.317798in}}%
\pgfpathlineto{\pgfqpoint{1.854072in}{1.318811in}}%
\pgfpathlineto{\pgfqpoint{1.854364in}{1.317658in}}%
\pgfpathlineto{\pgfqpoint{1.854364in}{1.317658in}}%
\pgfpathlineto{\pgfqpoint{1.854512in}{1.317347in}}%
\pgfpathlineto{\pgfqpoint{1.854661in}{1.317571in}}%
\pgfpathlineto{\pgfqpoint{1.854661in}{1.317571in}}%
\pgfpathlineto{\pgfqpoint{1.856110in}{1.318784in}}%
\pgfpathlineto{\pgfqpoint{1.856186in}{1.318761in}}%
\pgfpathlineto{\pgfqpoint{1.857623in}{1.317309in}}%
\pgfpathlineto{\pgfqpoint{1.857772in}{1.317508in}}%
\pgfpathlineto{\pgfqpoint{1.859272in}{1.318759in}}%
\pgfpathlineto{\pgfqpoint{1.860807in}{1.317287in}}%
\pgfpathlineto{\pgfqpoint{1.861112in}{1.318207in}}%
\pgfpathlineto{\pgfqpoint{1.861343in}{1.318738in}}%
\pgfpathlineto{\pgfqpoint{1.861570in}{1.318073in}}%
\pgfpathlineto{\pgfqpoint{1.861570in}{1.318073in}}%
\pgfpathlineto{\pgfqpoint{1.861835in}{1.317259in}}%
\pgfpathlineto{\pgfqpoint{1.862140in}{1.318123in}}%
\pgfpathlineto{\pgfqpoint{1.862140in}{1.318123in}}%
\pgfpathlineto{\pgfqpoint{1.862372in}{1.318720in}}%
\pgfpathlineto{\pgfqpoint{1.862628in}{1.317988in}}%
\pgfpathlineto{\pgfqpoint{1.862628in}{1.317988in}}%
\pgfpathlineto{\pgfqpoint{1.862843in}{1.317242in}}%
\pgfpathlineto{\pgfqpoint{1.863146in}{1.317955in}}%
\pgfpathlineto{\pgfqpoint{1.863146in}{1.317955in}}%
\pgfpathlineto{\pgfqpoint{1.863456in}{1.318717in}}%
\pgfpathlineto{\pgfqpoint{1.863682in}{1.317937in}}%
\pgfpathlineto{\pgfqpoint{1.863682in}{1.317937in}}%
\pgfpathlineto{\pgfqpoint{1.863932in}{1.317232in}}%
\pgfpathlineto{\pgfqpoint{1.864159in}{1.317807in}}%
\pgfpathlineto{\pgfqpoint{1.864159in}{1.317807in}}%
\pgfpathlineto{\pgfqpoint{1.864470in}{1.318699in}}%
\pgfpathlineto{\pgfqpoint{1.864768in}{1.317704in}}%
\pgfpathlineto{\pgfqpoint{1.864768in}{1.317704in}}%
\pgfpathlineto{\pgfqpoint{1.864993in}{1.317226in}}%
\pgfpathlineto{\pgfqpoint{1.865144in}{1.317561in}}%
\pgfpathlineto{\pgfqpoint{1.865144in}{1.317561in}}%
\pgfpathlineto{\pgfqpoint{1.865531in}{1.318694in}}%
\pgfpathlineto{\pgfqpoint{1.865845in}{1.317559in}}%
\pgfpathlineto{\pgfqpoint{1.865845in}{1.317559in}}%
\pgfpathlineto{\pgfqpoint{1.865995in}{1.317200in}}%
\pgfpathlineto{\pgfqpoint{1.866223in}{1.317641in}}%
\pgfpathlineto{\pgfqpoint{1.866223in}{1.317641in}}%
\pgfpathlineto{\pgfqpoint{1.866610in}{1.318684in}}%
\pgfpathlineto{\pgfqpoint{1.866911in}{1.317489in}}%
\pgfpathlineto{\pgfqpoint{1.866911in}{1.317489in}}%
\pgfpathlineto{\pgfqpoint{1.867060in}{1.317184in}}%
\pgfpathlineto{\pgfqpoint{1.867212in}{1.317424in}}%
\pgfpathlineto{\pgfqpoint{1.867212in}{1.317424in}}%
\pgfpathlineto{\pgfqpoint{1.868697in}{1.318664in}}%
\pgfpathlineto{\pgfqpoint{1.870282in}{1.317180in}}%
\pgfpathlineto{\pgfqpoint{1.870670in}{1.318415in}}%
\pgfpathlineto{\pgfqpoint{1.870824in}{1.318642in}}%
\pgfpathlineto{\pgfqpoint{1.870977in}{1.318300in}}%
\pgfpathlineto{\pgfqpoint{1.870977in}{1.318300in}}%
\pgfpathlineto{\pgfqpoint{1.871261in}{1.317136in}}%
\pgfpathlineto{\pgfqpoint{1.871725in}{1.318394in}}%
\pgfpathlineto{\pgfqpoint{1.871725in}{1.318394in}}%
\pgfpathlineto{\pgfqpoint{1.871879in}{1.318632in}}%
\pgfpathlineto{\pgfqpoint{1.872032in}{1.318303in}}%
\pgfpathlineto{\pgfqpoint{1.872032in}{1.318303in}}%
\pgfpathlineto{\pgfqpoint{1.872329in}{1.317117in}}%
\pgfpathlineto{\pgfqpoint{1.872794in}{1.318411in}}%
\pgfpathlineto{\pgfqpoint{1.872794in}{1.318411in}}%
\pgfpathlineto{\pgfqpoint{1.872948in}{1.318619in}}%
\pgfpathlineto{\pgfqpoint{1.873025in}{1.318517in}}%
\pgfpathlineto{\pgfqpoint{1.873025in}{1.318517in}}%
\pgfpathlineto{\pgfqpoint{1.874489in}{1.317095in}}%
\pgfpathlineto{\pgfqpoint{1.876166in}{1.318563in}}%
\pgfpathlineto{\pgfqpoint{1.876357in}{1.317811in}}%
\pgfpathlineto{\pgfqpoint{1.876575in}{1.317062in}}%
\pgfpathlineto{\pgfqpoint{1.876883in}{1.317801in}}%
\pgfpathlineto{\pgfqpoint{1.876883in}{1.317801in}}%
\pgfpathlineto{\pgfqpoint{1.877197in}{1.318576in}}%
\pgfpathlineto{\pgfqpoint{1.877427in}{1.317763in}}%
\pgfpathlineto{\pgfqpoint{1.877427in}{1.317763in}}%
\pgfpathlineto{\pgfqpoint{1.877692in}{1.317062in}}%
\pgfpathlineto{\pgfqpoint{1.877923in}{1.317691in}}%
\pgfpathlineto{\pgfqpoint{1.877923in}{1.317691in}}%
\pgfpathlineto{\pgfqpoint{1.878238in}{1.318565in}}%
\pgfpathlineto{\pgfqpoint{1.878497in}{1.317728in}}%
\pgfpathlineto{\pgfqpoint{1.878497in}{1.317728in}}%
\pgfpathlineto{\pgfqpoint{1.878703in}{1.317036in}}%
\pgfpathlineto{\pgfqpoint{1.879012in}{1.317761in}}%
\pgfpathlineto{\pgfqpoint{1.879012in}{1.317761in}}%
\pgfpathlineto{\pgfqpoint{1.879326in}{1.318555in}}%
\pgfpathlineto{\pgfqpoint{1.879565in}{1.317716in}}%
\pgfpathlineto{\pgfqpoint{1.879565in}{1.317716in}}%
\pgfpathlineto{\pgfqpoint{1.879770in}{1.317022in}}%
\pgfpathlineto{\pgfqpoint{1.880079in}{1.317746in}}%
\pgfpathlineto{\pgfqpoint{1.880079in}{1.317746in}}%
\pgfpathlineto{\pgfqpoint{1.880394in}{1.318545in}}%
\pgfpathlineto{\pgfqpoint{1.880645in}{1.317633in}}%
\pgfpathlineto{\pgfqpoint{1.880645in}{1.317633in}}%
\pgfpathlineto{\pgfqpoint{1.880873in}{1.317006in}}%
\pgfpathlineto{\pgfqpoint{1.881105in}{1.317565in}}%
\pgfpathlineto{\pgfqpoint{1.881105in}{1.317565in}}%
\pgfpathlineto{\pgfqpoint{1.881499in}{1.318512in}}%
\pgfpathlineto{\pgfqpoint{1.881741in}{1.317470in}}%
\pgfpathlineto{\pgfqpoint{1.881741in}{1.317470in}}%
\pgfpathlineto{\pgfqpoint{1.881970in}{1.317009in}}%
\pgfpathlineto{\pgfqpoint{1.882124in}{1.317368in}}%
\pgfpathlineto{\pgfqpoint{1.882124in}{1.317368in}}%
\pgfpathlineto{\pgfqpoint{1.882519in}{1.318522in}}%
\pgfpathlineto{\pgfqpoint{1.882807in}{1.317486in}}%
\pgfpathlineto{\pgfqpoint{1.882807in}{1.317486in}}%
\pgfpathlineto{\pgfqpoint{1.883036in}{1.316990in}}%
\pgfpathlineto{\pgfqpoint{1.883191in}{1.317337in}}%
\pgfpathlineto{\pgfqpoint{1.883191in}{1.317337in}}%
\pgfpathlineto{\pgfqpoint{1.883586in}{1.318509in}}%
\pgfpathlineto{\pgfqpoint{1.883946in}{1.317175in}}%
\pgfpathlineto{\pgfqpoint{1.883946in}{1.317175in}}%
\pgfpathlineto{\pgfqpoint{1.884099in}{1.316969in}}%
\pgfpathlineto{\pgfqpoint{1.884254in}{1.317291in}}%
\pgfpathlineto{\pgfqpoint{1.884254in}{1.317291in}}%
\pgfpathlineto{\pgfqpoint{1.884650in}{1.318494in}}%
\pgfpathlineto{\pgfqpoint{1.885008in}{1.317205in}}%
\pgfpathlineto{\pgfqpoint{1.885008in}{1.317205in}}%
\pgfpathlineto{\pgfqpoint{1.885162in}{1.316949in}}%
\pgfpathlineto{\pgfqpoint{1.885316in}{1.317238in}}%
\pgfpathlineto{\pgfqpoint{1.885316in}{1.317238in}}%
\pgfpathlineto{\pgfqpoint{1.885791in}{1.318469in}}%
\pgfpathlineto{\pgfqpoint{1.886125in}{1.317060in}}%
\pgfpathlineto{\pgfqpoint{1.886125in}{1.317060in}}%
\pgfpathlineto{\pgfqpoint{1.886279in}{1.316968in}}%
\pgfpathlineto{\pgfqpoint{1.886356in}{1.317120in}}%
\pgfpathlineto{\pgfqpoint{1.886356in}{1.317120in}}%
\pgfpathlineto{\pgfqpoint{1.887884in}{1.318463in}}%
\pgfpathlineto{\pgfqpoint{1.888244in}{1.317144in}}%
\pgfpathlineto{\pgfqpoint{1.888398in}{1.316908in}}%
\pgfpathlineto{\pgfqpoint{1.888554in}{1.317214in}}%
\pgfpathlineto{\pgfqpoint{1.888554in}{1.317214in}}%
\pgfpathlineto{\pgfqpoint{1.889030in}{1.318430in}}%
\pgfpathlineto{\pgfqpoint{1.889369in}{1.316993in}}%
\pgfpathlineto{\pgfqpoint{1.889369in}{1.316993in}}%
\pgfpathlineto{\pgfqpoint{1.889524in}{1.316938in}}%
\pgfpathlineto{\pgfqpoint{1.889602in}{1.317104in}}%
\pgfpathlineto{\pgfqpoint{1.889602in}{1.317104in}}%
\pgfpathlineto{\pgfqpoint{1.891117in}{1.318425in}}%
\pgfpathlineto{\pgfqpoint{1.891438in}{1.317324in}}%
\pgfpathlineto{\pgfqpoint{1.891669in}{1.316886in}}%
\pgfpathlineto{\pgfqpoint{1.891825in}{1.317265in}}%
\pgfpathlineto{\pgfqpoint{1.891825in}{1.317265in}}%
\pgfpathlineto{\pgfqpoint{1.892224in}{1.318423in}}%
\pgfpathlineto{\pgfqpoint{1.892563in}{1.317120in}}%
\pgfpathlineto{\pgfqpoint{1.892563in}{1.317120in}}%
\pgfpathlineto{\pgfqpoint{1.892717in}{1.316848in}}%
\pgfpathlineto{\pgfqpoint{1.892873in}{1.317132in}}%
\pgfpathlineto{\pgfqpoint{1.892873in}{1.317132in}}%
\pgfpathlineto{\pgfqpoint{1.893352in}{1.318394in}}%
\pgfpathlineto{\pgfqpoint{1.893702in}{1.316935in}}%
\pgfpathlineto{\pgfqpoint{1.893702in}{1.316935in}}%
\pgfpathlineto{\pgfqpoint{1.893857in}{1.316881in}}%
\pgfpathlineto{\pgfqpoint{1.893936in}{1.317050in}}%
\pgfpathlineto{\pgfqpoint{1.893936in}{1.317050in}}%
\pgfpathlineto{\pgfqpoint{1.895462in}{1.318383in}}%
\pgfpathlineto{\pgfqpoint{1.895759in}{1.317389in}}%
\pgfpathlineto{\pgfqpoint{1.895991in}{1.316810in}}%
\pgfpathlineto{\pgfqpoint{1.896227in}{1.317421in}}%
\pgfpathlineto{\pgfqpoint{1.896227in}{1.317421in}}%
\pgfpathlineto{\pgfqpoint{1.896549in}{1.318370in}}%
\pgfpathlineto{\pgfqpoint{1.896828in}{1.317499in}}%
\pgfpathlineto{\pgfqpoint{1.896828in}{1.317499in}}%
\pgfpathlineto{\pgfqpoint{1.897046in}{1.316790in}}%
\pgfpathlineto{\pgfqpoint{1.897361in}{1.317582in}}%
\pgfpathlineto{\pgfqpoint{1.897361in}{1.317582in}}%
\pgfpathlineto{\pgfqpoint{1.897683in}{1.318364in}}%
\pgfpathlineto{\pgfqpoint{1.897918in}{1.317485in}}%
\pgfpathlineto{\pgfqpoint{1.897918in}{1.317485in}}%
\pgfpathlineto{\pgfqpoint{1.898137in}{1.316775in}}%
\pgfpathlineto{\pgfqpoint{1.898454in}{1.317574in}}%
\pgfpathlineto{\pgfqpoint{1.898454in}{1.317574in}}%
\pgfpathlineto{\pgfqpoint{1.898775in}{1.318352in}}%
\pgfpathlineto{\pgfqpoint{1.899003in}{1.317517in}}%
\pgfpathlineto{\pgfqpoint{1.899003in}{1.317517in}}%
\pgfpathlineto{\pgfqpoint{1.899266in}{1.316769in}}%
\pgfpathlineto{\pgfqpoint{1.899503in}{1.317393in}}%
\pgfpathlineto{\pgfqpoint{1.899503in}{1.317393in}}%
\pgfpathlineto{\pgfqpoint{1.899826in}{1.318338in}}%
\pgfpathlineto{\pgfqpoint{1.900104in}{1.317456in}}%
\pgfpathlineto{\pgfqpoint{1.900104in}{1.317456in}}%
\pgfpathlineto{\pgfqpoint{1.900325in}{1.316745in}}%
\pgfpathlineto{\pgfqpoint{1.900562in}{1.317246in}}%
\pgfpathlineto{\pgfqpoint{1.900562in}{1.317246in}}%
\pgfpathlineto{\pgfqpoint{1.900965in}{1.318328in}}%
\pgfpathlineto{\pgfqpoint{1.901229in}{1.317256in}}%
\pgfpathlineto{\pgfqpoint{1.901229in}{1.317256in}}%
\pgfpathlineto{\pgfqpoint{1.901463in}{1.316746in}}%
\pgfpathlineto{\pgfqpoint{1.901621in}{1.317106in}}%
\pgfpathlineto{\pgfqpoint{1.901621in}{1.317106in}}%
\pgfpathlineto{\pgfqpoint{1.902025in}{1.318320in}}%
\pgfpathlineto{\pgfqpoint{1.902357in}{1.317082in}}%
\pgfpathlineto{\pgfqpoint{1.902357in}{1.317082in}}%
\pgfpathlineto{\pgfqpoint{1.902513in}{1.316716in}}%
\pgfpathlineto{\pgfqpoint{1.902670in}{1.316944in}}%
\pgfpathlineto{\pgfqpoint{1.902670in}{1.316944in}}%
\pgfpathlineto{\pgfqpoint{1.904203in}{1.318288in}}%
\pgfpathlineto{\pgfqpoint{1.904523in}{1.317205in}}%
\pgfpathlineto{\pgfqpoint{1.904758in}{1.316703in}}%
\pgfpathlineto{\pgfqpoint{1.904916in}{1.317070in}}%
\pgfpathlineto{\pgfqpoint{1.904916in}{1.317070in}}%
\pgfpathlineto{\pgfqpoint{1.905321in}{1.318286in}}%
\pgfpathlineto{\pgfqpoint{1.905642in}{1.317092in}}%
\pgfpathlineto{\pgfqpoint{1.905642in}{1.317092in}}%
\pgfpathlineto{\pgfqpoint{1.905799in}{1.316676in}}%
\pgfpathlineto{\pgfqpoint{1.906037in}{1.317125in}}%
\pgfpathlineto{\pgfqpoint{1.906037in}{1.317125in}}%
\pgfpathlineto{\pgfqpoint{1.906442in}{1.318276in}}%
\pgfpathlineto{\pgfqpoint{1.906703in}{1.317305in}}%
\pgfpathlineto{\pgfqpoint{1.906703in}{1.317305in}}%
\pgfpathlineto{\pgfqpoint{1.906938in}{1.316659in}}%
\pgfpathlineto{\pgfqpoint{1.907177in}{1.317255in}}%
\pgfpathlineto{\pgfqpoint{1.907177in}{1.317255in}}%
\pgfpathlineto{\pgfqpoint{1.907584in}{1.318240in}}%
\pgfpathlineto{\pgfqpoint{1.907821in}{1.317199in}}%
\pgfpathlineto{\pgfqpoint{1.907821in}{1.317199in}}%
\pgfpathlineto{\pgfqpoint{1.908057in}{1.316654in}}%
\pgfpathlineto{\pgfqpoint{1.908297in}{1.317303in}}%
\pgfpathlineto{\pgfqpoint{1.908297in}{1.317303in}}%
\pgfpathlineto{\pgfqpoint{1.908623in}{1.318249in}}%
\pgfpathlineto{\pgfqpoint{1.908917in}{1.317235in}}%
\pgfpathlineto{\pgfqpoint{1.908917in}{1.317235in}}%
\pgfpathlineto{\pgfqpoint{1.909153in}{1.316633in}}%
\pgfpathlineto{\pgfqpoint{1.909393in}{1.317255in}}%
\pgfpathlineto{\pgfqpoint{1.909393in}{1.317255in}}%
\pgfpathlineto{\pgfqpoint{1.909720in}{1.318233in}}%
\pgfpathlineto{\pgfqpoint{1.910003in}{1.317349in}}%
\pgfpathlineto{\pgfqpoint{1.910003in}{1.317349in}}%
\pgfpathlineto{\pgfqpoint{1.910217in}{1.316615in}}%
\pgfpathlineto{\pgfqpoint{1.910537in}{1.317392in}}%
\pgfpathlineto{\pgfqpoint{1.910537in}{1.317392in}}%
\pgfpathlineto{\pgfqpoint{1.910864in}{1.318230in}}%
\pgfpathlineto{\pgfqpoint{1.911118in}{1.317279in}}%
\pgfpathlineto{\pgfqpoint{1.911118in}{1.317279in}}%
\pgfpathlineto{\pgfqpoint{1.911355in}{1.316598in}}%
\pgfpathlineto{\pgfqpoint{1.911595in}{1.317182in}}%
\pgfpathlineto{\pgfqpoint{1.911595in}{1.317182in}}%
\pgfpathlineto{\pgfqpoint{1.912004in}{1.318199in}}%
\pgfpathlineto{\pgfqpoint{1.912255in}{1.317098in}}%
\pgfpathlineto{\pgfqpoint{1.912255in}{1.317098in}}%
\pgfpathlineto{\pgfqpoint{1.912492in}{1.316601in}}%
\pgfpathlineto{\pgfqpoint{1.912652in}{1.316978in}}%
\pgfpathlineto{\pgfqpoint{1.912652in}{1.316978in}}%
\pgfpathlineto{\pgfqpoint{1.913061in}{1.318206in}}%
\pgfpathlineto{\pgfqpoint{1.913428in}{1.316798in}}%
\pgfpathlineto{\pgfqpoint{1.913428in}{1.316798in}}%
\pgfpathlineto{\pgfqpoint{1.913587in}{1.316573in}}%
\pgfpathlineto{\pgfqpoint{1.913747in}{1.316909in}}%
\pgfpathlineto{\pgfqpoint{1.913747in}{1.316909in}}%
\pgfpathlineto{\pgfqpoint{1.914156in}{1.318187in}}%
\pgfpathlineto{\pgfqpoint{1.914494in}{1.316980in}}%
\pgfpathlineto{\pgfqpoint{1.914494in}{1.316980in}}%
\pgfpathlineto{\pgfqpoint{1.914652in}{1.316556in}}%
\pgfpathlineto{\pgfqpoint{1.914893in}{1.317012in}}%
\pgfpathlineto{\pgfqpoint{1.914893in}{1.317012in}}%
\pgfpathlineto{\pgfqpoint{1.915302in}{1.318184in}}%
\pgfpathlineto{\pgfqpoint{1.915563in}{1.317213in}}%
\pgfpathlineto{\pgfqpoint{1.915563in}{1.317213in}}%
\pgfpathlineto{\pgfqpoint{1.915801in}{1.316538in}}%
\pgfpathlineto{\pgfqpoint{1.916043in}{1.317134in}}%
\pgfpathlineto{\pgfqpoint{1.916043in}{1.317134in}}%
\pgfpathlineto{\pgfqpoint{1.916453in}{1.318150in}}%
\pgfpathlineto{\pgfqpoint{1.916681in}{1.317176in}}%
\pgfpathlineto{\pgfqpoint{1.916681in}{1.317176in}}%
\pgfpathlineto{\pgfqpoint{1.916919in}{1.316525in}}%
\pgfpathlineto{\pgfqpoint{1.917161in}{1.317135in}}%
\pgfpathlineto{\pgfqpoint{1.917161in}{1.317135in}}%
\pgfpathlineto{\pgfqpoint{1.917572in}{1.318135in}}%
\pgfpathlineto{\pgfqpoint{1.917826in}{1.316994in}}%
\pgfpathlineto{\pgfqpoint{1.917826in}{1.316994in}}%
\pgfpathlineto{\pgfqpoint{1.918065in}{1.316532in}}%
\pgfpathlineto{\pgfqpoint{1.918226in}{1.316930in}}%
\pgfpathlineto{\pgfqpoint{1.918226in}{1.316930in}}%
\pgfpathlineto{\pgfqpoint{1.918638in}{1.318149in}}%
\pgfpathlineto{\pgfqpoint{1.918986in}{1.316782in}}%
\pgfpathlineto{\pgfqpoint{1.918986in}{1.316782in}}%
\pgfpathlineto{\pgfqpoint{1.919145in}{1.316491in}}%
\pgfpathlineto{\pgfqpoint{1.919306in}{1.316788in}}%
\pgfpathlineto{\pgfqpoint{1.919306in}{1.316788in}}%
\pgfpathlineto{\pgfqpoint{1.919800in}{1.318120in}}%
\pgfpathlineto{\pgfqpoint{1.920126in}{1.316688in}}%
\pgfpathlineto{\pgfqpoint{1.920126in}{1.316688in}}%
\pgfpathlineto{\pgfqpoint{1.920285in}{1.316487in}}%
\pgfpathlineto{\pgfqpoint{1.920366in}{1.316611in}}%
\pgfpathlineto{\pgfqpoint{1.920366in}{1.316611in}}%
\pgfpathlineto{\pgfqpoint{1.922016in}{1.318112in}}%
\pgfpathlineto{\pgfqpoint{1.923643in}{1.316435in}}%
\pgfpathlineto{\pgfqpoint{1.923887in}{1.317070in}}%
\pgfpathlineto{\pgfqpoint{1.924219in}{1.318081in}}%
\pgfpathlineto{\pgfqpoint{1.924523in}{1.317079in}}%
\pgfpathlineto{\pgfqpoint{1.924523in}{1.317079in}}%
\pgfpathlineto{\pgfqpoint{1.924763in}{1.316417in}}%
\pgfpathlineto{\pgfqpoint{1.925007in}{1.317037in}}%
\pgfpathlineto{\pgfqpoint{1.925007in}{1.317037in}}%
\pgfpathlineto{\pgfqpoint{1.925422in}{1.318052in}}%
\pgfpathlineto{\pgfqpoint{1.925654in}{1.317032in}}%
\pgfpathlineto{\pgfqpoint{1.925654in}{1.317032in}}%
\pgfpathlineto{\pgfqpoint{1.925895in}{1.316405in}}%
\pgfpathlineto{\pgfqpoint{1.926139in}{1.317045in}}%
\pgfpathlineto{\pgfqpoint{1.926139in}{1.317045in}}%
\pgfpathlineto{\pgfqpoint{1.926473in}{1.318058in}}%
\pgfpathlineto{\pgfqpoint{1.926785in}{1.316997in}}%
\pgfpathlineto{\pgfqpoint{1.926785in}{1.316997in}}%
\pgfpathlineto{\pgfqpoint{1.927026in}{1.316391in}}%
\pgfpathlineto{\pgfqpoint{1.927271in}{1.317045in}}%
\pgfpathlineto{\pgfqpoint{1.927271in}{1.317045in}}%
\pgfpathlineto{\pgfqpoint{1.927605in}{1.318048in}}%
\pgfpathlineto{\pgfqpoint{1.927914in}{1.316982in}}%
\pgfpathlineto{\pgfqpoint{1.927914in}{1.316982in}}%
\pgfpathlineto{\pgfqpoint{1.928156in}{1.316376in}}%
\pgfpathlineto{\pgfqpoint{1.928401in}{1.317031in}}%
\pgfpathlineto{\pgfqpoint{1.928401in}{1.317031in}}%
\pgfpathlineto{\pgfqpoint{1.928735in}{1.318036in}}%
\pgfpathlineto{\pgfqpoint{1.929042in}{1.316988in}}%
\pgfpathlineto{\pgfqpoint{1.929042in}{1.316988in}}%
\pgfpathlineto{\pgfqpoint{1.929283in}{1.316358in}}%
\pgfpathlineto{\pgfqpoint{1.929529in}{1.317004in}}%
\pgfpathlineto{\pgfqpoint{1.929529in}{1.317004in}}%
\pgfpathlineto{\pgfqpoint{1.929863in}{1.318022in}}%
\pgfpathlineto{\pgfqpoint{1.930156in}{1.317089in}}%
\pgfpathlineto{\pgfqpoint{1.930156in}{1.317089in}}%
\pgfpathlineto{\pgfqpoint{1.930386in}{1.316336in}}%
\pgfpathlineto{\pgfqpoint{1.930631in}{1.316872in}}%
\pgfpathlineto{\pgfqpoint{1.930631in}{1.316872in}}%
\pgfpathlineto{\pgfqpoint{1.931048in}{1.318014in}}%
\pgfpathlineto{\pgfqpoint{1.931300in}{1.317007in}}%
\pgfpathlineto{\pgfqpoint{1.931300in}{1.317007in}}%
\pgfpathlineto{\pgfqpoint{1.931542in}{1.316323in}}%
\pgfpathlineto{\pgfqpoint{1.931788in}{1.316945in}}%
\pgfpathlineto{\pgfqpoint{1.931788in}{1.316945in}}%
\pgfpathlineto{\pgfqpoint{1.932206in}{1.317983in}}%
\pgfpathlineto{\pgfqpoint{1.932423in}{1.317071in}}%
\pgfpathlineto{\pgfqpoint{1.932423in}{1.317071in}}%
\pgfpathlineto{\pgfqpoint{1.932646in}{1.316307in}}%
\pgfpathlineto{\pgfqpoint{1.932975in}{1.317137in}}%
\pgfpathlineto{\pgfqpoint{1.932975in}{1.317137in}}%
\pgfpathlineto{\pgfqpoint{1.933310in}{1.317994in}}%
\pgfpathlineto{\pgfqpoint{1.933552in}{1.317105in}}%
\pgfpathlineto{\pgfqpoint{1.933552in}{1.317105in}}%
\pgfpathlineto{\pgfqpoint{1.933822in}{1.316296in}}%
\pgfpathlineto{\pgfqpoint{1.934069in}{1.316948in}}%
\pgfpathlineto{\pgfqpoint{1.934069in}{1.316948in}}%
\pgfpathlineto{\pgfqpoint{1.934405in}{1.317975in}}%
\pgfpathlineto{\pgfqpoint{1.934738in}{1.316796in}}%
\pgfpathlineto{\pgfqpoint{1.934738in}{1.316796in}}%
\pgfpathlineto{\pgfqpoint{1.934981in}{1.316297in}}%
\pgfpathlineto{\pgfqpoint{1.935146in}{1.316700in}}%
\pgfpathlineto{\pgfqpoint{1.935146in}{1.316700in}}%
\pgfpathlineto{\pgfqpoint{1.935565in}{1.317972in}}%
\pgfpathlineto{\pgfqpoint{1.935902in}{1.316654in}}%
\pgfpathlineto{\pgfqpoint{1.935902in}{1.316654in}}%
\pgfpathlineto{\pgfqpoint{1.936064in}{1.316259in}}%
\pgfpathlineto{\pgfqpoint{1.936228in}{1.316498in}}%
\pgfpathlineto{\pgfqpoint{1.936228in}{1.316498in}}%
\pgfpathlineto{\pgfqpoint{1.937742in}{1.317808in}}%
\pgfpathlineto{\pgfqpoint{1.937908in}{1.317916in}}%
\pgfpathlineto{\pgfqpoint{1.937990in}{1.317718in}}%
\pgfpathlineto{\pgfqpoint{1.937990in}{1.317718in}}%
\pgfpathlineto{\pgfqpoint{1.938405in}{1.316249in}}%
\pgfpathlineto{\pgfqpoint{1.938908in}{1.317843in}}%
\pgfpathlineto{\pgfqpoint{1.938908in}{1.317843in}}%
\pgfpathlineto{\pgfqpoint{1.939074in}{1.317869in}}%
\pgfpathlineto{\pgfqpoint{1.939074in}{1.317869in}}%
\pgfpathlineto{\pgfqpoint{1.939074in}{1.317869in}}%
\pgfpathlineto{\pgfqpoint{1.940634in}{1.316197in}}%
\pgfpathlineto{\pgfqpoint{1.940717in}{1.316243in}}%
\pgfpathlineto{\pgfqpoint{1.942447in}{1.317900in}}%
\pgfpathlineto{\pgfqpoint{1.942530in}{1.317801in}}%
\pgfpathlineto{\pgfqpoint{1.944087in}{1.316147in}}%
\pgfpathlineto{\pgfqpoint{1.945950in}{1.317827in}}%
\pgfpathlineto{\pgfqpoint{1.946250in}{1.316420in}}%
\pgfpathlineto{\pgfqpoint{1.946415in}{1.316115in}}%
\pgfpathlineto{\pgfqpoint{1.946581in}{1.316428in}}%
\pgfpathlineto{\pgfqpoint{1.946581in}{1.316428in}}%
\pgfpathlineto{\pgfqpoint{1.947090in}{1.317832in}}%
\pgfpathlineto{\pgfqpoint{1.947415in}{1.316368in}}%
\pgfpathlineto{\pgfqpoint{1.947415in}{1.316368in}}%
\pgfpathlineto{\pgfqpoint{1.947579in}{1.316102in}}%
\pgfpathlineto{\pgfqpoint{1.947746in}{1.316444in}}%
\pgfpathlineto{\pgfqpoint{1.947746in}{1.316444in}}%
\pgfpathlineto{\pgfqpoint{1.948256in}{1.317809in}}%
\pgfpathlineto{\pgfqpoint{1.948571in}{1.316352in}}%
\pgfpathlineto{\pgfqpoint{1.948571in}{1.316352in}}%
\pgfpathlineto{\pgfqpoint{1.948736in}{1.316086in}}%
\pgfpathlineto{\pgfqpoint{1.948903in}{1.316428in}}%
\pgfpathlineto{\pgfqpoint{1.948903in}{1.316428in}}%
\pgfpathlineto{\pgfqpoint{1.949414in}{1.317797in}}%
\pgfpathlineto{\pgfqpoint{1.949706in}{1.316436in}}%
\pgfpathlineto{\pgfqpoint{1.949706in}{1.316436in}}%
\pgfpathlineto{\pgfqpoint{1.949871in}{1.316065in}}%
\pgfpathlineto{\pgfqpoint{1.950038in}{1.316336in}}%
\pgfpathlineto{\pgfqpoint{1.950038in}{1.316336in}}%
\pgfpathlineto{\pgfqpoint{1.950549in}{1.317807in}}%
\pgfpathlineto{\pgfqpoint{1.950923in}{1.316206in}}%
\pgfpathlineto{\pgfqpoint{1.950923in}{1.316206in}}%
\pgfpathlineto{\pgfqpoint{1.951089in}{1.316082in}}%
\pgfpathlineto{\pgfqpoint{1.951173in}{1.316248in}}%
\pgfpathlineto{\pgfqpoint{1.951173in}{1.316248in}}%
\pgfpathlineto{\pgfqpoint{1.951685in}{1.317803in}}%
\pgfpathlineto{\pgfqpoint{1.952241in}{1.316055in}}%
\pgfpathlineto{\pgfqpoint{1.954015in}{1.317778in}}%
\pgfpathlineto{\pgfqpoint{1.954183in}{1.317394in}}%
\pgfpathlineto{\pgfqpoint{1.954506in}{1.316004in}}%
\pgfpathlineto{\pgfqpoint{1.955018in}{1.317504in}}%
\pgfpathlineto{\pgfqpoint{1.955018in}{1.317504in}}%
\pgfpathlineto{\pgfqpoint{1.955187in}{1.317765in}}%
\pgfpathlineto{\pgfqpoint{1.955271in}{1.317656in}}%
\pgfpathlineto{\pgfqpoint{1.955271in}{1.317656in}}%
\pgfpathlineto{\pgfqpoint{1.955705in}{1.315985in}}%
\pgfpathlineto{\pgfqpoint{1.956388in}{1.317731in}}%
\pgfpathlineto{\pgfqpoint{1.958011in}{1.315954in}}%
\pgfpathlineto{\pgfqpoint{1.958180in}{1.316184in}}%
\pgfpathlineto{\pgfqpoint{1.958696in}{1.317727in}}%
\pgfpathlineto{\pgfqpoint{1.959228in}{1.315943in}}%
\pgfpathlineto{\pgfqpoint{1.961098in}{1.317648in}}%
\pgfpathlineto{\pgfqpoint{1.961183in}{1.317405in}}%
\pgfpathlineto{\pgfqpoint{1.961515in}{1.315914in}}%
\pgfpathlineto{\pgfqpoint{1.962031in}{1.317381in}}%
\pgfpathlineto{\pgfqpoint{1.962031in}{1.317381in}}%
\pgfpathlineto{\pgfqpoint{1.962202in}{1.317691in}}%
\pgfpathlineto{\pgfqpoint{1.962372in}{1.317337in}}%
\pgfpathlineto{\pgfqpoint{1.962372in}{1.317337in}}%
\pgfpathlineto{\pgfqpoint{1.962771in}{1.315909in}}%
\pgfpathlineto{\pgfqpoint{1.963203in}{1.317355in}}%
\pgfpathlineto{\pgfqpoint{1.963203in}{1.317355in}}%
\pgfpathlineto{\pgfqpoint{1.963375in}{1.317678in}}%
\pgfpathlineto{\pgfqpoint{1.963545in}{1.317342in}}%
\pgfpathlineto{\pgfqpoint{1.963545in}{1.317342in}}%
\pgfpathlineto{\pgfqpoint{1.963916in}{1.315869in}}%
\pgfpathlineto{\pgfqpoint{1.964435in}{1.317494in}}%
\pgfpathlineto{\pgfqpoint{1.964435in}{1.317494in}}%
\pgfpathlineto{\pgfqpoint{1.964606in}{1.317643in}}%
\pgfpathlineto{\pgfqpoint{1.964691in}{1.317456in}}%
\pgfpathlineto{\pgfqpoint{1.964691in}{1.317456in}}%
\pgfpathlineto{\pgfqpoint{1.965095in}{1.315852in}}%
\pgfpathlineto{\pgfqpoint{1.965786in}{1.317631in}}%
\pgfpathlineto{\pgfqpoint{1.967475in}{1.315826in}}%
\pgfpathlineto{\pgfqpoint{1.967645in}{1.316204in}}%
\pgfpathlineto{\pgfqpoint{1.968081in}{1.317620in}}%
\pgfpathlineto{\pgfqpoint{1.968479in}{1.316093in}}%
\pgfpathlineto{\pgfqpoint{1.968479in}{1.316093in}}%
\pgfpathlineto{\pgfqpoint{1.968648in}{1.315803in}}%
\pgfpathlineto{\pgfqpoint{1.968819in}{1.316151in}}%
\pgfpathlineto{\pgfqpoint{1.968819in}{1.316151in}}%
\pgfpathlineto{\pgfqpoint{1.969342in}{1.317588in}}%
\pgfpathlineto{\pgfqpoint{1.969705in}{1.315940in}}%
\pgfpathlineto{\pgfqpoint{1.969705in}{1.315940in}}%
\pgfpathlineto{\pgfqpoint{1.969874in}{1.315820in}}%
\pgfpathlineto{\pgfqpoint{1.969960in}{1.315995in}}%
\pgfpathlineto{\pgfqpoint{1.969960in}{1.315995in}}%
\pgfpathlineto{\pgfqpoint{1.970483in}{1.317603in}}%
\pgfpathlineto{\pgfqpoint{1.971031in}{1.315774in}}%
\pgfpathlineto{\pgfqpoint{1.972909in}{1.317549in}}%
\pgfpathlineto{\pgfqpoint{1.973081in}{1.316934in}}%
\pgfpathlineto{\pgfqpoint{1.973433in}{1.315758in}}%
\pgfpathlineto{\pgfqpoint{1.973781in}{1.316905in}}%
\pgfpathlineto{\pgfqpoint{1.973781in}{1.316905in}}%
\pgfpathlineto{\pgfqpoint{1.974044in}{1.317564in}}%
\pgfpathlineto{\pgfqpoint{1.974217in}{1.317219in}}%
\pgfpathlineto{\pgfqpoint{1.974217in}{1.317219in}}%
\pgfpathlineto{\pgfqpoint{1.974599in}{1.315720in}}%
\pgfpathlineto{\pgfqpoint{1.975037in}{1.317129in}}%
\pgfpathlineto{\pgfqpoint{1.975037in}{1.317129in}}%
\pgfpathlineto{\pgfqpoint{1.975298in}{1.317520in}}%
\pgfpathlineto{\pgfqpoint{1.975384in}{1.317311in}}%
\pgfpathlineto{\pgfqpoint{1.975384in}{1.317311in}}%
\pgfpathlineto{\pgfqpoint{1.975746in}{1.315706in}}%
\pgfpathlineto{\pgfqpoint{1.976358in}{1.317462in}}%
\pgfpathlineto{\pgfqpoint{1.976358in}{1.317462in}}%
\pgfpathlineto{\pgfqpoint{1.976445in}{1.317539in}}%
\pgfpathlineto{\pgfqpoint{1.976531in}{1.317441in}}%
\pgfpathlineto{\pgfqpoint{1.976531in}{1.317441in}}%
\pgfpathlineto{\pgfqpoint{1.976999in}{1.315690in}}%
\pgfpathlineto{\pgfqpoint{1.977612in}{1.317518in}}%
\pgfpathlineto{\pgfqpoint{1.977934in}{1.316383in}}%
\pgfpathlineto{\pgfqpoint{1.978190in}{1.315669in}}%
\pgfpathlineto{\pgfqpoint{1.978450in}{1.316369in}}%
\pgfpathlineto{\pgfqpoint{1.978450in}{1.316369in}}%
\pgfpathlineto{\pgfqpoint{1.978805in}{1.317501in}}%
\pgfpathlineto{\pgfqpoint{1.979114in}{1.316494in}}%
\pgfpathlineto{\pgfqpoint{1.979114in}{1.316494in}}%
\pgfpathlineto{\pgfqpoint{1.979344in}{1.315652in}}%
\pgfpathlineto{\pgfqpoint{1.979692in}{1.316527in}}%
\pgfpathlineto{\pgfqpoint{1.979692in}{1.316527in}}%
\pgfpathlineto{\pgfqpoint{1.980046in}{1.317500in}}%
\pgfpathlineto{\pgfqpoint{1.980326in}{1.316410in}}%
\pgfpathlineto{\pgfqpoint{1.980326in}{1.316410in}}%
\pgfpathlineto{\pgfqpoint{1.980582in}{1.315630in}}%
\pgfpathlineto{\pgfqpoint{1.980843in}{1.316301in}}%
\pgfpathlineto{\pgfqpoint{1.980843in}{1.316301in}}%
\pgfpathlineto{\pgfqpoint{1.981286in}{1.317464in}}%
\pgfpathlineto{\pgfqpoint{1.981528in}{1.316396in}}%
\pgfpathlineto{\pgfqpoint{1.981528in}{1.316396in}}%
\pgfpathlineto{\pgfqpoint{1.981785in}{1.315613in}}%
\pgfpathlineto{\pgfqpoint{1.982046in}{1.316284in}}%
\pgfpathlineto{\pgfqpoint{1.982046in}{1.316284in}}%
\pgfpathlineto{\pgfqpoint{1.982490in}{1.317451in}}%
\pgfpathlineto{\pgfqpoint{1.982731in}{1.316392in}}%
\pgfpathlineto{\pgfqpoint{1.982731in}{1.316392in}}%
\pgfpathlineto{\pgfqpoint{1.982988in}{1.315595in}}%
\pgfpathlineto{\pgfqpoint{1.983250in}{1.316261in}}%
\pgfpathlineto{\pgfqpoint{1.983250in}{1.316261in}}%
\pgfpathlineto{\pgfqpoint{1.983694in}{1.317440in}}%
\pgfpathlineto{\pgfqpoint{1.983951in}{1.316288in}}%
\pgfpathlineto{\pgfqpoint{1.983951in}{1.316288in}}%
\pgfpathlineto{\pgfqpoint{1.984208in}{1.315584in}}%
\pgfpathlineto{\pgfqpoint{1.984470in}{1.316301in}}%
\pgfpathlineto{\pgfqpoint{1.984470in}{1.316301in}}%
\pgfpathlineto{\pgfqpoint{1.984828in}{1.317438in}}%
\pgfpathlineto{\pgfqpoint{1.985152in}{1.316318in}}%
\pgfpathlineto{\pgfqpoint{1.985152in}{1.316318in}}%
\pgfpathlineto{\pgfqpoint{1.985410in}{1.315562in}}%
\pgfpathlineto{\pgfqpoint{1.985672in}{1.316256in}}%
\pgfpathlineto{\pgfqpoint{1.985672in}{1.316256in}}%
\pgfpathlineto{\pgfqpoint{1.986118in}{1.317407in}}%
\pgfpathlineto{\pgfqpoint{1.986370in}{1.316245in}}%
\pgfpathlineto{\pgfqpoint{1.986370in}{1.316245in}}%
\pgfpathlineto{\pgfqpoint{1.986629in}{1.315550in}}%
\pgfpathlineto{\pgfqpoint{1.986892in}{1.316277in}}%
\pgfpathlineto{\pgfqpoint{1.986892in}{1.316277in}}%
\pgfpathlineto{\pgfqpoint{1.987250in}{1.317413in}}%
\pgfpathlineto{\pgfqpoint{1.987567in}{1.316326in}}%
\pgfpathlineto{\pgfqpoint{1.987567in}{1.316326in}}%
\pgfpathlineto{\pgfqpoint{1.987826in}{1.315525in}}%
\pgfpathlineto{\pgfqpoint{1.988089in}{1.316200in}}%
\pgfpathlineto{\pgfqpoint{1.988089in}{1.316200in}}%
\pgfpathlineto{\pgfqpoint{1.988536in}{1.317387in}}%
\pgfpathlineto{\pgfqpoint{1.988819in}{1.316068in}}%
\pgfpathlineto{\pgfqpoint{1.988819in}{1.316068in}}%
\pgfpathlineto{\pgfqpoint{1.989079in}{1.315536in}}%
\pgfpathlineto{\pgfqpoint{1.989254in}{1.315993in}}%
\pgfpathlineto{\pgfqpoint{1.989254in}{1.315993in}}%
\pgfpathlineto{\pgfqpoint{1.989702in}{1.317397in}}%
\pgfpathlineto{\pgfqpoint{1.990006in}{1.316223in}}%
\pgfpathlineto{\pgfqpoint{1.990006in}{1.316223in}}%
\pgfpathlineto{\pgfqpoint{1.990266in}{1.315494in}}%
\pgfpathlineto{\pgfqpoint{1.990530in}{1.316213in}}%
\pgfpathlineto{\pgfqpoint{1.990530in}{1.316213in}}%
\pgfpathlineto{\pgfqpoint{1.990890in}{1.317371in}}%
\pgfpathlineto{\pgfqpoint{1.991202in}{1.316352in}}%
\pgfpathlineto{\pgfqpoint{1.991202in}{1.316352in}}%
\pgfpathlineto{\pgfqpoint{1.991426in}{1.315484in}}%
\pgfpathlineto{\pgfqpoint{1.991778in}{1.316327in}}%
\pgfpathlineto{\pgfqpoint{1.991778in}{1.316327in}}%
\pgfpathlineto{\pgfqpoint{1.992138in}{1.317371in}}%
\pgfpathlineto{\pgfqpoint{1.992419in}{1.316345in}}%
\pgfpathlineto{\pgfqpoint{1.992419in}{1.316345in}}%
\pgfpathlineto{\pgfqpoint{1.992725in}{1.315477in}}%
\pgfpathlineto{\pgfqpoint{1.992991in}{1.316280in}}%
\pgfpathlineto{\pgfqpoint{1.992991in}{1.316280in}}%
\pgfpathlineto{\pgfqpoint{1.993351in}{1.317356in}}%
\pgfpathlineto{\pgfqpoint{1.993640in}{1.316326in}}%
\pgfpathlineto{\pgfqpoint{1.993640in}{1.316326in}}%
\pgfpathlineto{\pgfqpoint{1.993949in}{1.315463in}}%
\pgfpathlineto{\pgfqpoint{1.994215in}{1.316278in}}%
\pgfpathlineto{\pgfqpoint{1.994215in}{1.316278in}}%
\pgfpathlineto{\pgfqpoint{1.994576in}{1.317344in}}%
\pgfpathlineto{\pgfqpoint{1.994860in}{1.316324in}}%
\pgfpathlineto{\pgfqpoint{1.994860in}{1.316324in}}%
\pgfpathlineto{\pgfqpoint{1.995160in}{1.315434in}}%
\pgfpathlineto{\pgfqpoint{1.995426in}{1.316212in}}%
\pgfpathlineto{\pgfqpoint{1.995426in}{1.316212in}}%
\pgfpathlineto{\pgfqpoint{1.995787in}{1.317327in}}%
\pgfpathlineto{\pgfqpoint{1.996086in}{1.316296in}}%
\pgfpathlineto{\pgfqpoint{1.996086in}{1.316296in}}%
\pgfpathlineto{\pgfqpoint{1.996395in}{1.315426in}}%
\pgfpathlineto{\pgfqpoint{1.996661in}{1.316241in}}%
\pgfpathlineto{\pgfqpoint{1.996661in}{1.316241in}}%
\pgfpathlineto{\pgfqpoint{1.997023in}{1.317317in}}%
\pgfpathlineto{\pgfqpoint{1.997321in}{1.316213in}}%
\pgfpathlineto{\pgfqpoint{1.997321in}{1.316213in}}%
\pgfpathlineto{\pgfqpoint{1.997583in}{1.315382in}}%
\pgfpathlineto{\pgfqpoint{1.997849in}{1.316062in}}%
\pgfpathlineto{\pgfqpoint{1.997849in}{1.316062in}}%
\pgfpathlineto{\pgfqpoint{1.998301in}{1.317284in}}%
\pgfpathlineto{\pgfqpoint{1.998576in}{1.316017in}}%
\pgfpathlineto{\pgfqpoint{1.998576in}{1.316017in}}%
\pgfpathlineto{\pgfqpoint{1.998839in}{1.315380in}}%
\pgfpathlineto{\pgfqpoint{1.999016in}{1.315808in}}%
\pgfpathlineto{\pgfqpoint{1.999016in}{1.315808in}}%
\pgfpathlineto{\pgfqpoint{1.999469in}{1.317287in}}%
\pgfpathlineto{\pgfqpoint{1.999862in}{1.315705in}}%
\pgfpathlineto{\pgfqpoint{1.999862in}{1.315705in}}%
\pgfpathlineto{\pgfqpoint{2.000037in}{1.315345in}}%
\pgfpathlineto{\pgfqpoint{2.000214in}{1.315680in}}%
\pgfpathlineto{\pgfqpoint{2.000214in}{1.315680in}}%
\pgfpathlineto{\pgfqpoint{2.000757in}{1.317261in}}%
\pgfpathlineto{\pgfqpoint{2.001130in}{1.315542in}}%
\pgfpathlineto{\pgfqpoint{2.001130in}{1.315542in}}%
\pgfpathlineto{\pgfqpoint{2.001306in}{1.315350in}}%
\pgfpathlineto{\pgfqpoint{2.001395in}{1.315512in}}%
\pgfpathlineto{\pgfqpoint{2.001395in}{1.315512in}}%
\pgfpathlineto{\pgfqpoint{2.001938in}{1.317263in}}%
\pgfpathlineto{\pgfqpoint{2.002481in}{1.315311in}}%
\pgfpathlineto{\pgfqpoint{2.003023in}{1.316984in}}%
\pgfpathlineto{\pgfqpoint{2.003203in}{1.317248in}}%
\pgfpathlineto{\pgfqpoint{2.003292in}{1.317113in}}%
\pgfpathlineto{\pgfqpoint{2.003292in}{1.317113in}}%
\pgfpathlineto{\pgfqpoint{2.003745in}{1.315293in}}%
\pgfpathlineto{\pgfqpoint{2.004379in}{1.317218in}}%
\pgfpathlineto{\pgfqpoint{2.004705in}{1.316182in}}%
\pgfpathlineto{\pgfqpoint{2.004930in}{1.315288in}}%
\pgfpathlineto{\pgfqpoint{2.005288in}{1.316140in}}%
\pgfpathlineto{\pgfqpoint{2.005288in}{1.316140in}}%
\pgfpathlineto{\pgfqpoint{2.005654in}{1.317224in}}%
\pgfpathlineto{\pgfqpoint{2.005965in}{1.316013in}}%
\pgfpathlineto{\pgfqpoint{2.005965in}{1.316013in}}%
\pgfpathlineto{\pgfqpoint{2.006230in}{1.315260in}}%
\pgfpathlineto{\pgfqpoint{2.006499in}{1.316002in}}%
\pgfpathlineto{\pgfqpoint{2.006499in}{1.316002in}}%
\pgfpathlineto{\pgfqpoint{2.006866in}{1.317198in}}%
\pgfpathlineto{\pgfqpoint{2.007197in}{1.316051in}}%
\pgfpathlineto{\pgfqpoint{2.007197in}{1.316051in}}%
\pgfpathlineto{\pgfqpoint{2.007462in}{1.315238in}}%
\pgfpathlineto{\pgfqpoint{2.007732in}{1.315950in}}%
\pgfpathlineto{\pgfqpoint{2.007732in}{1.315950in}}%
\pgfpathlineto{\pgfqpoint{2.008189in}{1.317171in}}%
\pgfpathlineto{\pgfqpoint{2.008448in}{1.315980in}}%
\pgfpathlineto{\pgfqpoint{2.008448in}{1.315980in}}%
\pgfpathlineto{\pgfqpoint{2.008713in}{1.315224in}}%
\pgfpathlineto{\pgfqpoint{2.008983in}{1.315969in}}%
\pgfpathlineto{\pgfqpoint{2.008983in}{1.315969in}}%
\pgfpathlineto{\pgfqpoint{2.009351in}{1.317171in}}%
\pgfpathlineto{\pgfqpoint{2.009670in}{1.316114in}}%
\pgfpathlineto{\pgfqpoint{2.009670in}{1.316114in}}%
\pgfpathlineto{\pgfqpoint{2.009899in}{1.315213in}}%
\pgfpathlineto{\pgfqpoint{2.010259in}{1.316086in}}%
\pgfpathlineto{\pgfqpoint{2.010259in}{1.316086in}}%
\pgfpathlineto{\pgfqpoint{2.010627in}{1.317171in}}%
\pgfpathlineto{\pgfqpoint{2.010912in}{1.316129in}}%
\pgfpathlineto{\pgfqpoint{2.010912in}{1.316129in}}%
\pgfpathlineto{\pgfqpoint{2.011209in}{1.315190in}}%
\pgfpathlineto{\pgfqpoint{2.011479in}{1.315956in}}%
\pgfpathlineto{\pgfqpoint{2.011479in}{1.315956in}}%
\pgfpathlineto{\pgfqpoint{2.011848in}{1.317147in}}%
\pgfpathlineto{\pgfqpoint{2.012164in}{1.316082in}}%
\pgfpathlineto{\pgfqpoint{2.012164in}{1.316082in}}%
\pgfpathlineto{\pgfqpoint{2.012394in}{1.315176in}}%
\pgfpathlineto{\pgfqpoint{2.012755in}{1.316054in}}%
\pgfpathlineto{\pgfqpoint{2.012755in}{1.316054in}}%
\pgfpathlineto{\pgfqpoint{2.013124in}{1.317144in}}%
\pgfpathlineto{\pgfqpoint{2.013412in}{1.316076in}}%
\pgfpathlineto{\pgfqpoint{2.013412in}{1.316076in}}%
\pgfpathlineto{\pgfqpoint{2.013725in}{1.315167in}}%
\pgfpathlineto{\pgfqpoint{2.013997in}{1.316000in}}%
\pgfpathlineto{\pgfqpoint{2.013997in}{1.316000in}}%
\pgfpathlineto{\pgfqpoint{2.014366in}{1.317128in}}%
\pgfpathlineto{\pgfqpoint{2.014666in}{1.316040in}}%
\pgfpathlineto{\pgfqpoint{2.014666in}{1.316040in}}%
\pgfpathlineto{\pgfqpoint{2.014903in}{1.315133in}}%
\pgfpathlineto{\pgfqpoint{2.015266in}{1.316055in}}%
\pgfpathlineto{\pgfqpoint{2.015266in}{1.316055in}}%
\pgfpathlineto{\pgfqpoint{2.015636in}{1.317116in}}%
\pgfpathlineto{\pgfqpoint{2.015918in}{1.316040in}}%
\pgfpathlineto{\pgfqpoint{2.015918in}{1.316040in}}%
\pgfpathlineto{\pgfqpoint{2.016234in}{1.315133in}}%
\pgfpathlineto{\pgfqpoint{2.016506in}{1.315979in}}%
\pgfpathlineto{\pgfqpoint{2.016506in}{1.315979in}}%
\pgfpathlineto{\pgfqpoint{2.016877in}{1.317102in}}%
\pgfpathlineto{\pgfqpoint{2.017180in}{1.315971in}}%
\pgfpathlineto{\pgfqpoint{2.017180in}{1.315971in}}%
\pgfpathlineto{\pgfqpoint{2.017444in}{1.315087in}}%
\pgfpathlineto{\pgfqpoint{2.017716in}{1.315765in}}%
\pgfpathlineto{\pgfqpoint{2.017716in}{1.315765in}}%
\pgfpathlineto{\pgfqpoint{2.018179in}{1.317074in}}%
\pgfpathlineto{\pgfqpoint{2.018462in}{1.315791in}}%
\pgfpathlineto{\pgfqpoint{2.018462in}{1.315791in}}%
\pgfpathlineto{\pgfqpoint{2.018731in}{1.315080in}}%
\pgfpathlineto{\pgfqpoint{2.019004in}{1.315873in}}%
\pgfpathlineto{\pgfqpoint{2.019004in}{1.315873in}}%
\pgfpathlineto{\pgfqpoint{2.019375in}{1.317067in}}%
\pgfpathlineto{\pgfqpoint{2.019689in}{1.315993in}}%
\pgfpathlineto{\pgfqpoint{2.019689in}{1.315993in}}%
\pgfpathlineto{\pgfqpoint{2.020005in}{1.315075in}}%
\pgfpathlineto{\pgfqpoint{2.020279in}{1.315921in}}%
\pgfpathlineto{\pgfqpoint{2.020279in}{1.315921in}}%
\pgfpathlineto{\pgfqpoint{2.020651in}{1.317060in}}%
\pgfpathlineto{\pgfqpoint{2.020948in}{1.315991in}}%
\pgfpathlineto{\pgfqpoint{2.020948in}{1.315991in}}%
\pgfpathlineto{\pgfqpoint{2.021255in}{1.315045in}}%
\pgfpathlineto{\pgfqpoint{2.021529in}{1.315851in}}%
\pgfpathlineto{\pgfqpoint{2.021529in}{1.315851in}}%
\pgfpathlineto{\pgfqpoint{2.021902in}{1.317041in}}%
\pgfpathlineto{\pgfqpoint{2.022213in}{1.315963in}}%
\pgfpathlineto{\pgfqpoint{2.022213in}{1.315963in}}%
\pgfpathlineto{\pgfqpoint{2.022527in}{1.315034in}}%
\pgfpathlineto{\pgfqpoint{2.022802in}{1.315874in}}%
\pgfpathlineto{\pgfqpoint{2.022802in}{1.315874in}}%
\pgfpathlineto{\pgfqpoint{2.023175in}{1.317032in}}%
\pgfpathlineto{\pgfqpoint{2.023508in}{1.315737in}}%
\pgfpathlineto{\pgfqpoint{2.023508in}{1.315737in}}%
\pgfpathlineto{\pgfqpoint{2.023779in}{1.315003in}}%
\pgfpathlineto{\pgfqpoint{2.024053in}{1.315795in}}%
\pgfpathlineto{\pgfqpoint{2.024053in}{1.315795in}}%
\pgfpathlineto{\pgfqpoint{2.024428in}{1.317011in}}%
\pgfpathlineto{\pgfqpoint{2.024747in}{1.315915in}}%
\pgfpathlineto{\pgfqpoint{2.024747in}{1.315915in}}%
\pgfpathlineto{\pgfqpoint{2.024981in}{1.314987in}}%
\pgfpathlineto{\pgfqpoint{2.025348in}{1.315894in}}%
\pgfpathlineto{\pgfqpoint{2.025348in}{1.315894in}}%
\pgfpathlineto{\pgfqpoint{2.025723in}{1.317006in}}%
\pgfpathlineto{\pgfqpoint{2.026015in}{1.315906in}}%
\pgfpathlineto{\pgfqpoint{2.026015in}{1.315906in}}%
\pgfpathlineto{\pgfqpoint{2.026336in}{1.314983in}}%
\pgfpathlineto{\pgfqpoint{2.026612in}{1.315850in}}%
\pgfpathlineto{\pgfqpoint{2.026612in}{1.315850in}}%
\pgfpathlineto{\pgfqpoint{2.026987in}{1.316992in}}%
\pgfpathlineto{\pgfqpoint{2.027299in}{1.315792in}}%
\pgfpathlineto{\pgfqpoint{2.027299in}{1.315792in}}%
\pgfpathlineto{\pgfqpoint{2.027571in}{1.314938in}}%
\pgfpathlineto{\pgfqpoint{2.027846in}{1.315675in}}%
\pgfpathlineto{\pgfqpoint{2.027846in}{1.315675in}}%
\pgfpathlineto{\pgfqpoint{2.028316in}{1.316952in}}%
\pgfpathlineto{\pgfqpoint{2.028590in}{1.315654in}}%
\pgfpathlineto{\pgfqpoint{2.028590in}{1.315654in}}%
\pgfpathlineto{\pgfqpoint{2.028862in}{1.314928in}}%
\pgfpathlineto{\pgfqpoint{2.029139in}{1.315737in}}%
\pgfpathlineto{\pgfqpoint{2.029139in}{1.315737in}}%
\pgfpathlineto{\pgfqpoint{2.029515in}{1.316956in}}%
\pgfpathlineto{\pgfqpoint{2.029866in}{1.315630in}}%
\pgfpathlineto{\pgfqpoint{2.029866in}{1.315630in}}%
\pgfpathlineto{\pgfqpoint{2.030138in}{1.314910in}}%
\pgfpathlineto{\pgfqpoint{2.030415in}{1.315724in}}%
\pgfpathlineto{\pgfqpoint{2.030415in}{1.315724in}}%
\pgfpathlineto{\pgfqpoint{2.030792in}{1.316943in}}%
\pgfpathlineto{\pgfqpoint{2.031111in}{1.315834in}}%
\pgfpathlineto{\pgfqpoint{2.031111in}{1.315834in}}%
\pgfpathlineto{\pgfqpoint{2.031346in}{1.314892in}}%
\pgfpathlineto{\pgfqpoint{2.031715in}{1.315802in}}%
\pgfpathlineto{\pgfqpoint{2.031715in}{1.315802in}}%
\pgfpathlineto{\pgfqpoint{2.032093in}{1.316937in}}%
\pgfpathlineto{\pgfqpoint{2.032388in}{1.315824in}}%
\pgfpathlineto{\pgfqpoint{2.032388in}{1.315824in}}%
\pgfpathlineto{\pgfqpoint{2.032710in}{1.314885in}}%
\pgfpathlineto{\pgfqpoint{2.032989in}{1.315759in}}%
\pgfpathlineto{\pgfqpoint{2.032989in}{1.315759in}}%
\pgfpathlineto{\pgfqpoint{2.033367in}{1.316921in}}%
\pgfpathlineto{\pgfqpoint{2.033695in}{1.315623in}}%
\pgfpathlineto{\pgfqpoint{2.033695in}{1.315623in}}%
\pgfpathlineto{\pgfqpoint{2.033969in}{1.314847in}}%
\pgfpathlineto{\pgfqpoint{2.034247in}{1.315642in}}%
\pgfpathlineto{\pgfqpoint{2.034247in}{1.315642in}}%
\pgfpathlineto{\pgfqpoint{2.034627in}{1.316897in}}%
\pgfpathlineto{\pgfqpoint{2.034978in}{1.315604in}}%
\pgfpathlineto{\pgfqpoint{2.034978in}{1.315604in}}%
\pgfpathlineto{\pgfqpoint{2.035252in}{1.314828in}}%
\pgfpathlineto{\pgfqpoint{2.035531in}{1.315625in}}%
\pgfpathlineto{\pgfqpoint{2.035531in}{1.315625in}}%
\pgfpathlineto{\pgfqpoint{2.035911in}{1.316883in}}%
\pgfpathlineto{\pgfqpoint{2.036243in}{1.315722in}}%
\pgfpathlineto{\pgfqpoint{2.036243in}{1.315722in}}%
\pgfpathlineto{\pgfqpoint{2.036510in}{1.314800in}}%
\pgfpathlineto{\pgfqpoint{2.036789in}{1.315490in}}%
\pgfpathlineto{\pgfqpoint{2.036789in}{1.315490in}}%
\pgfpathlineto{\pgfqpoint{2.037263in}{1.316868in}}%
\pgfpathlineto{\pgfqpoint{2.037565in}{1.315473in}}%
\pgfpathlineto{\pgfqpoint{2.037565in}{1.315473in}}%
\pgfpathlineto{\pgfqpoint{2.037840in}{1.314800in}}%
\pgfpathlineto{\pgfqpoint{2.038026in}{1.315270in}}%
\pgfpathlineto{\pgfqpoint{2.038026in}{1.315270in}}%
\pgfpathlineto{\pgfqpoint{2.038500in}{1.316863in}}%
\pgfpathlineto{\pgfqpoint{2.038853in}{1.315459in}}%
\pgfpathlineto{\pgfqpoint{2.038853in}{1.315459in}}%
\pgfpathlineto{\pgfqpoint{2.039129in}{1.314780in}}%
\pgfpathlineto{\pgfqpoint{2.039315in}{1.315250in}}%
\pgfpathlineto{\pgfqpoint{2.039315in}{1.315250in}}%
\pgfpathlineto{\pgfqpoint{2.039790in}{1.316848in}}%
\pgfpathlineto{\pgfqpoint{2.040144in}{1.315438in}}%
\pgfpathlineto{\pgfqpoint{2.040144in}{1.315438in}}%
\pgfpathlineto{\pgfqpoint{2.040420in}{1.314761in}}%
\pgfpathlineto{\pgfqpoint{2.040607in}{1.315233in}}%
\pgfpathlineto{\pgfqpoint{2.040607in}{1.315233in}}%
\pgfpathlineto{\pgfqpoint{2.041083in}{1.316834in}}%
\pgfpathlineto{\pgfqpoint{2.041510in}{1.315043in}}%
\pgfpathlineto{\pgfqpoint{2.041510in}{1.315043in}}%
\pgfpathlineto{\pgfqpoint{2.041694in}{1.314727in}}%
\pgfpathlineto{\pgfqpoint{2.041881in}{1.315143in}}%
\pgfpathlineto{\pgfqpoint{2.041881in}{1.315143in}}%
\pgfpathlineto{\pgfqpoint{2.042452in}{1.316786in}}%
\pgfpathlineto{\pgfqpoint{2.042793in}{1.315073in}}%
\pgfpathlineto{\pgfqpoint{2.042793in}{1.315073in}}%
\pgfpathlineto{\pgfqpoint{2.042978in}{1.314703in}}%
\pgfpathlineto{\pgfqpoint{2.043165in}{1.315083in}}%
\pgfpathlineto{\pgfqpoint{2.043165in}{1.315083in}}%
\pgfpathlineto{\pgfqpoint{2.043737in}{1.316786in}}%
\pgfpathlineto{\pgfqpoint{2.044122in}{1.314926in}}%
\pgfpathlineto{\pgfqpoint{2.044122in}{1.314926in}}%
\pgfpathlineto{\pgfqpoint{2.044308in}{1.314703in}}%
\pgfpathlineto{\pgfqpoint{2.044401in}{1.314875in}}%
\pgfpathlineto{\pgfqpoint{2.044401in}{1.314875in}}%
\pgfpathlineto{\pgfqpoint{2.044973in}{1.316792in}}%
\pgfpathlineto{\pgfqpoint{2.045470in}{1.314763in}}%
\pgfpathlineto{\pgfqpoint{2.045563in}{1.314661in}}%
\pgfpathlineto{\pgfqpoint{2.045750in}{1.315005in}}%
\pgfpathlineto{\pgfqpoint{2.045750in}{1.315005in}}%
\pgfpathlineto{\pgfqpoint{2.046324in}{1.316768in}}%
\pgfpathlineto{\pgfqpoint{2.046727in}{1.314873in}}%
\pgfpathlineto{\pgfqpoint{2.046913in}{1.314667in}}%
\pgfpathlineto{\pgfqpoint{2.047007in}{1.314847in}}%
\pgfpathlineto{\pgfqpoint{2.047007in}{1.314847in}}%
\pgfpathlineto{\pgfqpoint{2.047581in}{1.316765in}}%
\pgfpathlineto{\pgfqpoint{2.048030in}{1.314855in}}%
\pgfpathlineto{\pgfqpoint{2.048216in}{1.314647in}}%
\pgfpathlineto{\pgfqpoint{2.048310in}{1.314827in}}%
\pgfpathlineto{\pgfqpoint{2.048310in}{1.314827in}}%
\pgfpathlineto{\pgfqpoint{2.048885in}{1.316750in}}%
\pgfpathlineto{\pgfqpoint{2.049394in}{1.314675in}}%
\pgfpathlineto{\pgfqpoint{2.049582in}{1.314724in}}%
\pgfpathlineto{\pgfqpoint{2.049676in}{1.314996in}}%
\pgfpathlineto{\pgfqpoint{2.049676in}{1.314996in}}%
\pgfpathlineto{\pgfqpoint{2.050253in}{1.316711in}}%
\pgfpathlineto{\pgfqpoint{2.050626in}{1.314878in}}%
\pgfpathlineto{\pgfqpoint{2.050626in}{1.314878in}}%
\pgfpathlineto{\pgfqpoint{2.050813in}{1.314592in}}%
\pgfpathlineto{\pgfqpoint{2.050907in}{1.314744in}}%
\pgfpathlineto{\pgfqpoint{2.050907in}{1.314744in}}%
\pgfpathlineto{\pgfqpoint{2.051484in}{1.316713in}}%
\pgfpathlineto{\pgfqpoint{2.051936in}{1.314855in}}%
\pgfpathlineto{\pgfqpoint{2.052123in}{1.314573in}}%
\pgfpathlineto{\pgfqpoint{2.052218in}{1.314726in}}%
\pgfpathlineto{\pgfqpoint{2.052218in}{1.314726in}}%
\pgfpathlineto{\pgfqpoint{2.052795in}{1.316699in}}%
\pgfpathlineto{\pgfqpoint{2.053320in}{1.314623in}}%
\pgfpathlineto{\pgfqpoint{2.053413in}{1.314542in}}%
\pgfpathlineto{\pgfqpoint{2.053603in}{1.314925in}}%
\pgfpathlineto{\pgfqpoint{2.053603in}{1.314925in}}%
\pgfpathlineto{\pgfqpoint{2.054182in}{1.316672in}}%
\pgfpathlineto{\pgfqpoint{2.054580in}{1.314751in}}%
\pgfpathlineto{\pgfqpoint{2.054767in}{1.314548in}}%
\pgfpathlineto{\pgfqpoint{2.054862in}{1.314734in}}%
\pgfpathlineto{\pgfqpoint{2.054862in}{1.314734in}}%
\pgfpathlineto{\pgfqpoint{2.055442in}{1.316678in}}%
\pgfpathlineto{\pgfqpoint{2.055911in}{1.314680in}}%
\pgfpathlineto{\pgfqpoint{2.056099in}{1.314547in}}%
\pgfpathlineto{\pgfqpoint{2.056194in}{1.314759in}}%
\pgfpathlineto{\pgfqpoint{2.056194in}{1.314759in}}%
\pgfpathlineto{\pgfqpoint{2.056775in}{1.316665in}}%
\pgfpathlineto{\pgfqpoint{2.057215in}{1.314705in}}%
\pgfpathlineto{\pgfqpoint{2.057404in}{1.314510in}}%
\pgfpathlineto{\pgfqpoint{2.057499in}{1.314699in}}%
\pgfpathlineto{\pgfqpoint{2.057499in}{1.314699in}}%
\pgfpathlineto{\pgfqpoint{2.058080in}{1.316650in}}%
\pgfpathlineto{\pgfqpoint{2.058527in}{1.314716in}}%
\pgfpathlineto{\pgfqpoint{2.058715in}{1.314480in}}%
\pgfpathlineto{\pgfqpoint{2.058810in}{1.314655in}}%
\pgfpathlineto{\pgfqpoint{2.058810in}{1.314655in}}%
\pgfpathlineto{\pgfqpoint{2.059393in}{1.316632in}}%
\pgfpathlineto{\pgfqpoint{2.059915in}{1.314512in}}%
\pgfpathlineto{\pgfqpoint{2.060105in}{1.314566in}}%
\pgfpathlineto{\pgfqpoint{2.060201in}{1.314846in}}%
\pgfpathlineto{\pgfqpoint{2.060201in}{1.314846in}}%
\pgfpathlineto{\pgfqpoint{2.060785in}{1.316593in}}%
\pgfpathlineto{\pgfqpoint{2.061149in}{1.314777in}}%
\pgfpathlineto{\pgfqpoint{2.061149in}{1.314777in}}%
\pgfpathlineto{\pgfqpoint{2.061338in}{1.314421in}}%
\pgfpathlineto{\pgfqpoint{2.061529in}{1.314836in}}%
\pgfpathlineto{\pgfqpoint{2.061529in}{1.314836in}}%
\pgfpathlineto{\pgfqpoint{2.062114in}{1.316575in}}%
\pgfpathlineto{\pgfqpoint{2.062486in}{1.314711in}}%
\pgfpathlineto{\pgfqpoint{2.062676in}{1.314407in}}%
\pgfpathlineto{\pgfqpoint{2.062771in}{1.314558in}}%
\pgfpathlineto{\pgfqpoint{2.062771in}{1.314558in}}%
\pgfpathlineto{\pgfqpoint{2.063357in}{1.316580in}}%
\pgfpathlineto{\pgfqpoint{2.063806in}{1.314728in}}%
\pgfpathlineto{\pgfqpoint{2.063996in}{1.314381in}}%
\pgfpathlineto{\pgfqpoint{2.064187in}{1.314805in}}%
\pgfpathlineto{\pgfqpoint{2.064187in}{1.314805in}}%
\pgfpathlineto{\pgfqpoint{2.064774in}{1.316542in}}%
\pgfpathlineto{\pgfqpoint{2.065155in}{1.314636in}}%
\pgfpathlineto{\pgfqpoint{2.065345in}{1.314372in}}%
\pgfpathlineto{\pgfqpoint{2.065441in}{1.314540in}}%
\pgfpathlineto{\pgfqpoint{2.065441in}{1.314540in}}%
\pgfpathlineto{\pgfqpoint{2.066028in}{1.316556in}}%
\pgfpathlineto{\pgfqpoint{2.066524in}{1.314496in}}%
\pgfpathlineto{\pgfqpoint{2.066715in}{1.314395in}}%
\pgfpathlineto{\pgfqpoint{2.066811in}{1.314624in}}%
\pgfpathlineto{\pgfqpoint{2.066811in}{1.314624in}}%
\pgfpathlineto{\pgfqpoint{2.067399in}{1.316546in}}%
\pgfpathlineto{\pgfqpoint{2.067840in}{1.314536in}}%
\pgfpathlineto{\pgfqpoint{2.068031in}{1.314347in}}%
\pgfpathlineto{\pgfqpoint{2.068127in}{1.314545in}}%
\pgfpathlineto{\pgfqpoint{2.068127in}{1.314545in}}%
\pgfpathlineto{\pgfqpoint{2.068716in}{1.316532in}}%
\pgfpathlineto{\pgfqpoint{2.069169in}{1.314545in}}%
\pgfpathlineto{\pgfqpoint{2.069360in}{1.314317in}}%
\pgfpathlineto{\pgfqpoint{2.069457in}{1.314501in}}%
\pgfpathlineto{\pgfqpoint{2.069457in}{1.314501in}}%
\pgfpathlineto{\pgfqpoint{2.070047in}{1.316515in}}%
\pgfpathlineto{\pgfqpoint{2.070505in}{1.314540in}}%
\pgfpathlineto{\pgfqpoint{2.070696in}{1.314292in}}%
\pgfpathlineto{\pgfqpoint{2.070793in}{1.314469in}}%
\pgfpathlineto{\pgfqpoint{2.070793in}{1.314469in}}%
\pgfpathlineto{\pgfqpoint{2.071384in}{1.316498in}}%
\pgfpathlineto{\pgfqpoint{2.071838in}{1.314553in}}%
\pgfpathlineto{\pgfqpoint{2.072030in}{1.314263in}}%
\pgfpathlineto{\pgfqpoint{2.072126in}{1.314426in}}%
\pgfpathlineto{\pgfqpoint{2.072126in}{1.314426in}}%
\pgfpathlineto{\pgfqpoint{2.072718in}{1.316478in}}%
\pgfpathlineto{\pgfqpoint{2.073184in}{1.314527in}}%
\pgfpathlineto{\pgfqpoint{2.073376in}{1.314243in}}%
\pgfpathlineto{\pgfqpoint{2.073472in}{1.314409in}}%
\pgfpathlineto{\pgfqpoint{2.073472in}{1.314409in}}%
\pgfpathlineto{\pgfqpoint{2.074065in}{1.316464in}}%
\pgfpathlineto{\pgfqpoint{2.074583in}{1.314335in}}%
\pgfpathlineto{\pgfqpoint{2.074679in}{1.314208in}}%
\pgfpathlineto{\pgfqpoint{2.074873in}{1.314547in}}%
\pgfpathlineto{\pgfqpoint{2.074873in}{1.314547in}}%
\pgfpathlineto{\pgfqpoint{2.075468in}{1.316450in}}%
\pgfpathlineto{\pgfqpoint{2.075946in}{1.314281in}}%
\pgfpathlineto{\pgfqpoint{2.076043in}{1.314187in}}%
\pgfpathlineto{\pgfqpoint{2.076237in}{1.314576in}}%
\pgfpathlineto{\pgfqpoint{2.076237in}{1.314576in}}%
\pgfpathlineto{\pgfqpoint{2.076833in}{1.316423in}}%
\pgfpathlineto{\pgfqpoint{2.077240in}{1.314426in}}%
\pgfpathlineto{\pgfqpoint{2.077433in}{1.314189in}}%
\pgfpathlineto{\pgfqpoint{2.077530in}{1.314374in}}%
\pgfpathlineto{\pgfqpoint{2.077530in}{1.314374in}}%
\pgfpathlineto{\pgfqpoint{2.078126in}{1.316424in}}%
\pgfpathlineto{\pgfqpoint{2.078566in}{1.314513in}}%
\pgfpathlineto{\pgfqpoint{2.078760in}{1.314148in}}%
\pgfpathlineto{\pgfqpoint{2.078955in}{1.314580in}}%
\pgfpathlineto{\pgfqpoint{2.078955in}{1.314580in}}%
\pgfpathlineto{\pgfqpoint{2.079553in}{1.316379in}}%
\pgfpathlineto{\pgfqpoint{2.079945in}{1.314400in}}%
\pgfpathlineto{\pgfqpoint{2.080138in}{1.314142in}}%
\pgfpathlineto{\pgfqpoint{2.080236in}{1.314321in}}%
\pgfpathlineto{\pgfqpoint{2.080236in}{1.314321in}}%
\pgfpathlineto{\pgfqpoint{2.080834in}{1.316392in}}%
\pgfpathlineto{\pgfqpoint{2.081316in}{1.314329in}}%
\pgfpathlineto{\pgfqpoint{2.081510in}{1.314135in}}%
\pgfpathlineto{\pgfqpoint{2.081608in}{1.314339in}}%
\pgfpathlineto{\pgfqpoint{2.081608in}{1.314339in}}%
\pgfpathlineto{\pgfqpoint{2.082207in}{1.316382in}}%
\pgfpathlineto{\pgfqpoint{2.082669in}{1.314334in}}%
\pgfpathlineto{\pgfqpoint{2.082863in}{1.314105in}}%
\pgfpathlineto{\pgfqpoint{2.082961in}{1.314297in}}%
\pgfpathlineto{\pgfqpoint{2.082961in}{1.314297in}}%
\pgfpathlineto{\pgfqpoint{2.083561in}{1.316365in}}%
\pgfpathlineto{\pgfqpoint{2.084025in}{1.314335in}}%
\pgfpathlineto{\pgfqpoint{2.084220in}{1.314078in}}%
\pgfpathlineto{\pgfqpoint{2.084318in}{1.314259in}}%
\pgfpathlineto{\pgfqpoint{2.084318in}{1.314259in}}%
\pgfpathlineto{\pgfqpoint{2.084919in}{1.316347in}}%
\pgfpathlineto{\pgfqpoint{2.085405in}{1.314259in}}%
\pgfpathlineto{\pgfqpoint{2.085600in}{1.314073in}}%
\pgfpathlineto{\pgfqpoint{2.085699in}{1.314281in}}%
\pgfpathlineto{\pgfqpoint{2.085699in}{1.314281in}}%
\pgfpathlineto{\pgfqpoint{2.086301in}{1.316336in}}%
\pgfpathlineto{\pgfqpoint{2.086756in}{1.314296in}}%
\pgfpathlineto{\pgfqpoint{2.086951in}{1.314034in}}%
\pgfpathlineto{\pgfqpoint{2.087050in}{1.314215in}}%
\pgfpathlineto{\pgfqpoint{2.087050in}{1.314215in}}%
\pgfpathlineto{\pgfqpoint{2.087653in}{1.316316in}}%
\pgfpathlineto{\pgfqpoint{2.088136in}{1.314235in}}%
\pgfpathlineto{\pgfqpoint{2.088332in}{1.314023in}}%
\pgfpathlineto{\pgfqpoint{2.088431in}{1.314224in}}%
\pgfpathlineto{\pgfqpoint{2.088431in}{1.314224in}}%
\pgfpathlineto{\pgfqpoint{2.089035in}{1.316305in}}%
\pgfpathlineto{\pgfqpoint{2.089469in}{1.314373in}}%
\pgfpathlineto{\pgfqpoint{2.089664in}{1.313972in}}%
\pgfpathlineto{\pgfqpoint{2.089862in}{1.314396in}}%
\pgfpathlineto{\pgfqpoint{2.089862in}{1.314396in}}%
\pgfpathlineto{\pgfqpoint{2.090468in}{1.316263in}}%
\pgfpathlineto{\pgfqpoint{2.090868in}{1.314245in}}%
\pgfpathlineto{\pgfqpoint{2.091064in}{1.313965in}}%
\pgfpathlineto{\pgfqpoint{2.091163in}{1.314142in}}%
\pgfpathlineto{\pgfqpoint{2.091163in}{1.314142in}}%
\pgfpathlineto{\pgfqpoint{2.091769in}{1.316268in}}%
\pgfpathlineto{\pgfqpoint{2.092238in}{1.314247in}}%
\pgfpathlineto{\pgfqpoint{2.092435in}{1.313938in}}%
\pgfpathlineto{\pgfqpoint{2.092534in}{1.314105in}}%
\pgfpathlineto{\pgfqpoint{2.092534in}{1.314105in}}%
\pgfpathlineto{\pgfqpoint{2.093141in}{1.316249in}}%
\pgfpathlineto{\pgfqpoint{2.093597in}{1.314312in}}%
\pgfpathlineto{\pgfqpoint{2.093794in}{1.313906in}}%
\pgfpathlineto{\pgfqpoint{2.093992in}{1.314331in}}%
\pgfpathlineto{\pgfqpoint{2.093992in}{1.314331in}}%
\pgfpathlineto{\pgfqpoint{2.094602in}{1.316217in}}%
\pgfpathlineto{\pgfqpoint{2.095014in}{1.314142in}}%
\pgfpathlineto{\pgfqpoint{2.095211in}{1.313909in}}%
\pgfpathlineto{\pgfqpoint{2.095311in}{1.314105in}}%
\pgfpathlineto{\pgfqpoint{2.095311in}{1.314105in}}%
\pgfpathlineto{\pgfqpoint{2.095920in}{1.316226in}}%
\pgfpathlineto{\pgfqpoint{2.096353in}{1.314308in}}%
\pgfpathlineto{\pgfqpoint{2.096551in}{1.313859in}}%
\pgfpathlineto{\pgfqpoint{2.096750in}{1.314260in}}%
\pgfpathlineto{\pgfqpoint{2.096750in}{1.314260in}}%
\pgfpathlineto{\pgfqpoint{2.097361in}{1.316194in}}%
\pgfpathlineto{\pgfqpoint{2.097731in}{1.314329in}}%
\pgfpathlineto{\pgfqpoint{2.097928in}{1.313837in}}%
\pgfpathlineto{\pgfqpoint{2.098128in}{1.314208in}}%
\pgfpathlineto{\pgfqpoint{2.098128in}{1.314208in}}%
\pgfpathlineto{\pgfqpoint{2.098740in}{1.316186in}}%
\pgfpathlineto{\pgfqpoint{2.099119in}{1.314304in}}%
\pgfpathlineto{\pgfqpoint{2.099317in}{1.313814in}}%
\pgfpathlineto{\pgfqpoint{2.099517in}{1.314189in}}%
\pgfpathlineto{\pgfqpoint{2.099517in}{1.314189in}}%
\pgfpathlineto{\pgfqpoint{2.100130in}{1.316170in}}%
\pgfpathlineto{\pgfqpoint{2.100520in}{1.314233in}}%
\pgfpathlineto{\pgfqpoint{2.100718in}{1.313792in}}%
\pgfpathlineto{\pgfqpoint{2.100919in}{1.314204in}}%
\pgfpathlineto{\pgfqpoint{2.100919in}{1.314204in}}%
\pgfpathlineto{\pgfqpoint{2.101533in}{1.316145in}}%
\pgfpathlineto{\pgfqpoint{2.101959in}{1.314017in}}%
\pgfpathlineto{\pgfqpoint{2.102159in}{1.313802in}}%
\pgfpathlineto{\pgfqpoint{2.102259in}{1.314010in}}%
\pgfpathlineto{\pgfqpoint{2.102259in}{1.314010in}}%
\pgfpathlineto{\pgfqpoint{2.102874in}{1.316149in}}%
\pgfpathlineto{\pgfqpoint{2.103331in}{1.314088in}}%
\pgfpathlineto{\pgfqpoint{2.103530in}{1.313757in}}%
\pgfpathlineto{\pgfqpoint{2.103631in}{1.313922in}}%
\pgfpathlineto{\pgfqpoint{2.103631in}{1.313922in}}%
\pgfpathlineto{\pgfqpoint{2.104247in}{1.316122in}}%
\pgfpathlineto{\pgfqpoint{2.104738in}{1.314027in}}%
\pgfpathlineto{\pgfqpoint{2.104938in}{1.313742in}}%
\pgfpathlineto{\pgfqpoint{2.105039in}{1.313925in}}%
\pgfpathlineto{\pgfqpoint{2.105039in}{1.313925in}}%
\pgfpathlineto{\pgfqpoint{2.105655in}{1.316112in}}%
\pgfpathlineto{\pgfqpoint{2.106119in}{1.314090in}}%
\pgfpathlineto{\pgfqpoint{2.106318in}{1.313706in}}%
\pgfpathlineto{\pgfqpoint{2.106520in}{1.314164in}}%
\pgfpathlineto{\pgfqpoint{2.106520in}{1.314164in}}%
\pgfpathlineto{\pgfqpoint{2.107139in}{1.316067in}}%
\pgfpathlineto{\pgfqpoint{2.107537in}{1.314003in}}%
\pgfpathlineto{\pgfqpoint{2.107737in}{1.313692in}}%
\pgfpathlineto{\pgfqpoint{2.107838in}{1.313867in}}%
\pgfpathlineto{\pgfqpoint{2.107838in}{1.313867in}}%
\pgfpathlineto{\pgfqpoint{2.108457in}{1.316077in}}%
\pgfpathlineto{\pgfqpoint{2.108951in}{1.313944in}}%
\pgfpathlineto{\pgfqpoint{2.109152in}{1.313678in}}%
\pgfpathlineto{\pgfqpoint{2.109253in}{1.313870in}}%
\pgfpathlineto{\pgfqpoint{2.109253in}{1.313870in}}%
\pgfpathlineto{\pgfqpoint{2.109873in}{1.316067in}}%
\pgfpathlineto{\pgfqpoint{2.110342in}{1.313991in}}%
\pgfpathlineto{\pgfqpoint{2.110543in}{1.313641in}}%
\pgfpathlineto{\pgfqpoint{2.110644in}{1.313803in}}%
\pgfpathlineto{\pgfqpoint{2.110644in}{1.313803in}}%
\pgfpathlineto{\pgfqpoint{2.111266in}{1.316040in}}%
\pgfpathlineto{\pgfqpoint{2.111773in}{1.313883in}}%
\pgfpathlineto{\pgfqpoint{2.111975in}{1.313636in}}%
\pgfpathlineto{\pgfqpoint{2.112077in}{1.313838in}}%
\pgfpathlineto{\pgfqpoint{2.112077in}{1.313838in}}%
\pgfpathlineto{\pgfqpoint{2.112699in}{1.316037in}}%
\pgfpathlineto{\pgfqpoint{2.113148in}{1.314025in}}%
\pgfpathlineto{\pgfqpoint{2.113349in}{1.313587in}}%
\pgfpathlineto{\pgfqpoint{2.113553in}{1.314020in}}%
\pgfpathlineto{\pgfqpoint{2.113553in}{1.314020in}}%
\pgfpathlineto{\pgfqpoint{2.114177in}{1.315998in}}%
\pgfpathlineto{\pgfqpoint{2.114565in}{1.313994in}}%
\pgfpathlineto{\pgfqpoint{2.114766in}{1.313564in}}%
\pgfpathlineto{\pgfqpoint{2.114970in}{1.314005in}}%
\pgfpathlineto{\pgfqpoint{2.114970in}{1.314005in}}%
\pgfpathlineto{\pgfqpoint{2.115596in}{1.315980in}}%
\pgfpathlineto{\pgfqpoint{2.115984in}{1.313966in}}%
\pgfpathlineto{\pgfqpoint{2.116186in}{1.313541in}}%
\pgfpathlineto{\pgfqpoint{2.116390in}{1.313988in}}%
\pgfpathlineto{\pgfqpoint{2.116390in}{1.313988in}}%
\pgfpathlineto{\pgfqpoint{2.117017in}{1.315962in}}%
\pgfpathlineto{\pgfqpoint{2.117403in}{1.313950in}}%
\pgfpathlineto{\pgfqpoint{2.117605in}{1.313518in}}%
\pgfpathlineto{\pgfqpoint{2.117810in}{1.313962in}}%
\pgfpathlineto{\pgfqpoint{2.117810in}{1.313962in}}%
\pgfpathlineto{\pgfqpoint{2.118438in}{1.315948in}}%
\pgfpathlineto{\pgfqpoint{2.118846in}{1.313836in}}%
\pgfpathlineto{\pgfqpoint{2.119049in}{1.313505in}}%
\pgfpathlineto{\pgfqpoint{2.119151in}{1.313679in}}%
\pgfpathlineto{\pgfqpoint{2.119151in}{1.313679in}}%
\pgfpathlineto{\pgfqpoint{2.119779in}{1.315947in}}%
\pgfpathlineto{\pgfqpoint{2.120270in}{1.313819in}}%
\pgfpathlineto{\pgfqpoint{2.120473in}{1.313480in}}%
\pgfpathlineto{\pgfqpoint{2.120576in}{1.313653in}}%
\pgfpathlineto{\pgfqpoint{2.120576in}{1.313653in}}%
\pgfpathlineto{\pgfqpoint{2.121205in}{1.315930in}}%
\pgfpathlineto{\pgfqpoint{2.121682in}{1.313865in}}%
\pgfpathlineto{\pgfqpoint{2.121886in}{1.313448in}}%
\pgfpathlineto{\pgfqpoint{2.122092in}{1.313909in}}%
\pgfpathlineto{\pgfqpoint{2.122092in}{1.313909in}}%
\pgfpathlineto{\pgfqpoint{2.122723in}{1.315894in}}%
\pgfpathlineto{\pgfqpoint{2.123141in}{1.313722in}}%
\pgfpathlineto{\pgfqpoint{2.123345in}{1.313444in}}%
\pgfpathlineto{\pgfqpoint{2.123448in}{1.313641in}}%
\pgfpathlineto{\pgfqpoint{2.123448in}{1.313641in}}%
\pgfpathlineto{\pgfqpoint{2.124079in}{1.315906in}}%
\pgfpathlineto{\pgfqpoint{2.124566in}{1.313730in}}%
\pgfpathlineto{\pgfqpoint{2.124771in}{1.313413in}}%
\pgfpathlineto{\pgfqpoint{2.124874in}{1.313596in}}%
\pgfpathlineto{\pgfqpoint{2.124874in}{1.313596in}}%
\pgfpathlineto{\pgfqpoint{2.125506in}{1.315885in}}%
\pgfpathlineto{\pgfqpoint{2.125993in}{1.313744in}}%
\pgfpathlineto{\pgfqpoint{2.126198in}{1.313383in}}%
\pgfpathlineto{\pgfqpoint{2.126301in}{1.313551in}}%
\pgfpathlineto{\pgfqpoint{2.126301in}{1.313551in}}%
\pgfpathlineto{\pgfqpoint{2.126935in}{1.315863in}}%
\pgfpathlineto{\pgfqpoint{2.127453in}{1.313634in}}%
\pgfpathlineto{\pgfqpoint{2.127658in}{1.313378in}}%
\pgfpathlineto{\pgfqpoint{2.127762in}{1.313586in}}%
\pgfpathlineto{\pgfqpoint{2.127762in}{1.313586in}}%
\pgfpathlineto{\pgfqpoint{2.128397in}{1.315859in}}%
\pgfpathlineto{\pgfqpoint{2.128860in}{1.313754in}}%
\pgfpathlineto{\pgfqpoint{2.129065in}{1.313329in}}%
\pgfpathlineto{\pgfqpoint{2.129273in}{1.313795in}}%
\pgfpathlineto{\pgfqpoint{2.129273in}{1.313795in}}%
\pgfpathlineto{\pgfqpoint{2.129910in}{1.315813in}}%
\pgfpathlineto{\pgfqpoint{2.130320in}{1.313655in}}%
\pgfpathlineto{\pgfqpoint{2.130526in}{1.313314in}}%
\pgfpathlineto{\pgfqpoint{2.130630in}{1.313492in}}%
\pgfpathlineto{\pgfqpoint{2.130630in}{1.313492in}}%
\pgfpathlineto{\pgfqpoint{2.131267in}{1.315816in}}%
\pgfpathlineto{\pgfqpoint{2.131786in}{1.313554in}}%
\pgfpathlineto{\pgfqpoint{2.131993in}{1.313309in}}%
\pgfpathlineto{\pgfqpoint{2.132097in}{1.313524in}}%
\pgfpathlineto{\pgfqpoint{2.132097in}{1.313524in}}%
\pgfpathlineto{\pgfqpoint{2.132735in}{1.315810in}}%
\pgfpathlineto{\pgfqpoint{2.133142in}{1.313992in}}%
\pgfpathlineto{\pgfqpoint{2.133348in}{1.313270in}}%
\pgfpathlineto{\pgfqpoint{2.133662in}{1.313921in}}%
\pgfpathlineto{\pgfqpoint{2.133662in}{1.313921in}}%
\pgfpathlineto{\pgfqpoint{2.134196in}{1.315796in}}%
\pgfpathlineto{\pgfqpoint{2.134590in}{1.313991in}}%
\pgfpathlineto{\pgfqpoint{2.134590in}{1.313991in}}%
\pgfpathlineto{\pgfqpoint{2.134901in}{1.313269in}}%
\pgfpathlineto{\pgfqpoint{2.135110in}{1.313884in}}%
\pgfpathlineto{\pgfqpoint{2.135110in}{1.313884in}}%
\pgfpathlineto{\pgfqpoint{2.135646in}{1.315779in}}%
\pgfpathlineto{\pgfqpoint{2.136023in}{1.314109in}}%
\pgfpathlineto{\pgfqpoint{2.136023in}{1.314109in}}%
\pgfpathlineto{\pgfqpoint{2.136334in}{1.313222in}}%
\pgfpathlineto{\pgfqpoint{2.136544in}{1.313773in}}%
\pgfpathlineto{\pgfqpoint{2.136544in}{1.313773in}}%
\pgfpathlineto{\pgfqpoint{2.137080in}{1.315754in}}%
\pgfpathlineto{\pgfqpoint{2.137482in}{1.314068in}}%
\pgfpathlineto{\pgfqpoint{2.137482in}{1.314068in}}%
\pgfpathlineto{\pgfqpoint{2.137794in}{1.313200in}}%
\pgfpathlineto{\pgfqpoint{2.138004in}{1.313762in}}%
\pgfpathlineto{\pgfqpoint{2.138004in}{1.313762in}}%
\pgfpathlineto{\pgfqpoint{2.138542in}{1.315739in}}%
\pgfpathlineto{\pgfqpoint{2.138950in}{1.313991in}}%
\pgfpathlineto{\pgfqpoint{2.138950in}{1.313991in}}%
\pgfpathlineto{\pgfqpoint{2.139262in}{1.313184in}}%
\pgfpathlineto{\pgfqpoint{2.139473in}{1.313773in}}%
\pgfpathlineto{\pgfqpoint{2.139473in}{1.313773in}}%
\pgfpathlineto{\pgfqpoint{2.140011in}{1.315726in}}%
\pgfpathlineto{\pgfqpoint{2.140408in}{1.313997in}}%
\pgfpathlineto{\pgfqpoint{2.140408in}{1.313997in}}%
\pgfpathlineto{\pgfqpoint{2.140720in}{1.313155in}}%
\pgfpathlineto{\pgfqpoint{2.140931in}{1.313733in}}%
\pgfpathlineto{\pgfqpoint{2.140931in}{1.313733in}}%
\pgfpathlineto{\pgfqpoint{2.141471in}{1.315708in}}%
\pgfpathlineto{\pgfqpoint{2.141880in}{1.313922in}}%
\pgfpathlineto{\pgfqpoint{2.141880in}{1.313922in}}%
\pgfpathlineto{\pgfqpoint{2.142194in}{1.313139in}}%
\pgfpathlineto{\pgfqpoint{2.142405in}{1.313743in}}%
\pgfpathlineto{\pgfqpoint{2.142405in}{1.313743in}}%
\pgfpathlineto{\pgfqpoint{2.142946in}{1.315694in}}%
\pgfpathlineto{\pgfqpoint{2.143322in}{1.314077in}}%
\pgfpathlineto{\pgfqpoint{2.143322in}{1.314077in}}%
\pgfpathlineto{\pgfqpoint{2.143636in}{1.313091in}}%
\pgfpathlineto{\pgfqpoint{2.143954in}{1.314082in}}%
\pgfpathlineto{\pgfqpoint{2.143954in}{1.314082in}}%
\pgfpathlineto{\pgfqpoint{2.144390in}{1.315663in}}%
\pgfpathlineto{\pgfqpoint{2.144795in}{1.314037in}}%
\pgfpathlineto{\pgfqpoint{2.144795in}{1.314037in}}%
\pgfpathlineto{\pgfqpoint{2.145109in}{1.313068in}}%
\pgfpathlineto{\pgfqpoint{2.145428in}{1.314072in}}%
\pgfpathlineto{\pgfqpoint{2.145428in}{1.314072in}}%
\pgfpathlineto{\pgfqpoint{2.145864in}{1.315648in}}%
\pgfpathlineto{\pgfqpoint{2.146244in}{1.314192in}}%
\pgfpathlineto{\pgfqpoint{2.146244in}{1.314192in}}%
\pgfpathlineto{\pgfqpoint{2.146549in}{1.313031in}}%
\pgfpathlineto{\pgfqpoint{2.146868in}{1.313892in}}%
\pgfpathlineto{\pgfqpoint{2.146868in}{1.313892in}}%
\pgfpathlineto{\pgfqpoint{2.147412in}{1.315631in}}%
\pgfpathlineto{\pgfqpoint{2.147764in}{1.313850in}}%
\pgfpathlineto{\pgfqpoint{2.147764in}{1.313850in}}%
\pgfpathlineto{\pgfqpoint{2.148079in}{1.313036in}}%
\pgfpathlineto{\pgfqpoint{2.148293in}{1.313640in}}%
\pgfpathlineto{\pgfqpoint{2.148293in}{1.313640in}}%
\pgfpathlineto{\pgfqpoint{2.148837in}{1.315626in}}%
\pgfpathlineto{\pgfqpoint{2.149281in}{1.313589in}}%
\pgfpathlineto{\pgfqpoint{2.149491in}{1.312983in}}%
\pgfpathlineto{\pgfqpoint{2.149704in}{1.313357in}}%
\pgfpathlineto{\pgfqpoint{2.149704in}{1.313357in}}%
\pgfpathlineto{\pgfqpoint{2.150357in}{1.315606in}}%
\pgfpathlineto{\pgfqpoint{2.150805in}{1.313347in}}%
\pgfpathlineto{\pgfqpoint{2.151017in}{1.312965in}}%
\pgfpathlineto{\pgfqpoint{2.151123in}{1.313141in}}%
\pgfpathlineto{\pgfqpoint{2.151123in}{1.313141in}}%
\pgfpathlineto{\pgfqpoint{2.151778in}{1.315578in}}%
\pgfpathlineto{\pgfqpoint{2.152301in}{1.313273in}}%
\pgfpathlineto{\pgfqpoint{2.152513in}{1.312949in}}%
\pgfpathlineto{\pgfqpoint{2.152620in}{1.313148in}}%
\pgfpathlineto{\pgfqpoint{2.152620in}{1.313148in}}%
\pgfpathlineto{\pgfqpoint{2.153274in}{1.315569in}}%
\pgfpathlineto{\pgfqpoint{2.153780in}{1.313284in}}%
\pgfpathlineto{\pgfqpoint{2.153992in}{1.312917in}}%
\pgfpathlineto{\pgfqpoint{2.154099in}{1.313101in}}%
\pgfpathlineto{\pgfqpoint{2.154099in}{1.313101in}}%
\pgfpathlineto{\pgfqpoint{2.154755in}{1.315547in}}%
\pgfpathlineto{\pgfqpoint{2.155235in}{1.313427in}}%
\pgfpathlineto{\pgfqpoint{2.155447in}{1.312880in}}%
\pgfpathlineto{\pgfqpoint{2.155662in}{1.313305in}}%
\pgfpathlineto{\pgfqpoint{2.155662in}{1.313305in}}%
\pgfpathlineto{\pgfqpoint{2.156320in}{1.315529in}}%
\pgfpathlineto{\pgfqpoint{2.156671in}{1.313749in}}%
\pgfpathlineto{\pgfqpoint{2.156990in}{1.312879in}}%
\pgfpathlineto{\pgfqpoint{2.157206in}{1.313478in}}%
\pgfpathlineto{\pgfqpoint{2.157206in}{1.313478in}}%
\pgfpathlineto{\pgfqpoint{2.157756in}{1.315522in}}%
\pgfpathlineto{\pgfqpoint{2.158242in}{1.313284in}}%
\pgfpathlineto{\pgfqpoint{2.158455in}{1.312832in}}%
\pgfpathlineto{\pgfqpoint{2.158671in}{1.313328in}}%
\pgfpathlineto{\pgfqpoint{2.158671in}{1.313328in}}%
\pgfpathlineto{\pgfqpoint{2.159332in}{1.315475in}}%
\pgfpathlineto{\pgfqpoint{2.159746in}{1.313231in}}%
\pgfpathlineto{\pgfqpoint{2.159960in}{1.312808in}}%
\pgfpathlineto{\pgfqpoint{2.160176in}{1.313328in}}%
\pgfpathlineto{\pgfqpoint{2.160176in}{1.313328in}}%
\pgfpathlineto{\pgfqpoint{2.160838in}{1.315449in}}%
\pgfpathlineto{\pgfqpoint{2.161261in}{1.313145in}}%
\pgfpathlineto{\pgfqpoint{2.161475in}{1.312792in}}%
\pgfpathlineto{\pgfqpoint{2.161583in}{1.312986in}}%
\pgfpathlineto{\pgfqpoint{2.161583in}{1.312986in}}%
\pgfpathlineto{\pgfqpoint{2.162246in}{1.315463in}}%
\pgfpathlineto{\pgfqpoint{2.162718in}{1.313352in}}%
\pgfpathlineto{\pgfqpoint{2.162932in}{1.312752in}}%
\pgfpathlineto{\pgfqpoint{2.163149in}{1.313152in}}%
\pgfpathlineto{\pgfqpoint{2.163149in}{1.313152in}}%
\pgfpathlineto{\pgfqpoint{2.163813in}{1.315449in}}%
\pgfpathlineto{\pgfqpoint{2.164173in}{1.313657in}}%
\pgfpathlineto{\pgfqpoint{2.164495in}{1.312747in}}%
\pgfpathlineto{\pgfqpoint{2.164713in}{1.313345in}}%
\pgfpathlineto{\pgfqpoint{2.164713in}{1.313345in}}%
\pgfpathlineto{\pgfqpoint{2.165269in}{1.315434in}}%
\pgfpathlineto{\pgfqpoint{2.165688in}{1.313597in}}%
\pgfpathlineto{\pgfqpoint{2.166011in}{1.312726in}}%
\pgfpathlineto{\pgfqpoint{2.166229in}{1.313343in}}%
\pgfpathlineto{\pgfqpoint{2.166229in}{1.313343in}}%
\pgfpathlineto{\pgfqpoint{2.166786in}{1.315419in}}%
\pgfpathlineto{\pgfqpoint{2.167188in}{1.313658in}}%
\pgfpathlineto{\pgfqpoint{2.167188in}{1.313658in}}%
\pgfpathlineto{\pgfqpoint{2.167512in}{1.312689in}}%
\pgfpathlineto{\pgfqpoint{2.167841in}{1.313767in}}%
\pgfpathlineto{\pgfqpoint{2.167841in}{1.313767in}}%
\pgfpathlineto{\pgfqpoint{2.168288in}{1.315394in}}%
\pgfpathlineto{\pgfqpoint{2.168699in}{1.313673in}}%
\pgfpathlineto{\pgfqpoint{2.168699in}{1.313673in}}%
\pgfpathlineto{\pgfqpoint{2.169023in}{1.312658in}}%
\pgfpathlineto{\pgfqpoint{2.169352in}{1.313718in}}%
\pgfpathlineto{\pgfqpoint{2.169352in}{1.313718in}}%
\pgfpathlineto{\pgfqpoint{2.169801in}{1.315373in}}%
\pgfpathlineto{\pgfqpoint{2.170221in}{1.313622in}}%
\pgfpathlineto{\pgfqpoint{2.170221in}{1.313622in}}%
\pgfpathlineto{\pgfqpoint{2.170546in}{1.312635in}}%
\pgfpathlineto{\pgfqpoint{2.170876in}{1.313714in}}%
\pgfpathlineto{\pgfqpoint{2.170876in}{1.313714in}}%
\pgfpathlineto{\pgfqpoint{2.171325in}{1.315358in}}%
\pgfpathlineto{\pgfqpoint{2.171705in}{1.313894in}}%
\pgfpathlineto{\pgfqpoint{2.171705in}{1.313894in}}%
\pgfpathlineto{\pgfqpoint{2.172082in}{1.312621in}}%
\pgfpathlineto{\pgfqpoint{2.172413in}{1.313760in}}%
\pgfpathlineto{\pgfqpoint{2.172413in}{1.313760in}}%
\pgfpathlineto{\pgfqpoint{2.172863in}{1.315349in}}%
\pgfpathlineto{\pgfqpoint{2.173255in}{1.313665in}}%
\pgfpathlineto{\pgfqpoint{2.173255in}{1.313665in}}%
\pgfpathlineto{\pgfqpoint{2.173581in}{1.312574in}}%
\pgfpathlineto{\pgfqpoint{2.173912in}{1.313609in}}%
\pgfpathlineto{\pgfqpoint{2.173912in}{1.313609in}}%
\pgfpathlineto{\pgfqpoint{2.174364in}{1.315314in}}%
\pgfpathlineto{\pgfqpoint{2.174792in}{1.313566in}}%
\pgfpathlineto{\pgfqpoint{2.174792in}{1.313566in}}%
\pgfpathlineto{\pgfqpoint{2.175119in}{1.312554in}}%
\pgfpathlineto{\pgfqpoint{2.175451in}{1.313635in}}%
\pgfpathlineto{\pgfqpoint{2.175451in}{1.313635in}}%
\pgfpathlineto{\pgfqpoint{2.175903in}{1.315304in}}%
\pgfpathlineto{\pgfqpoint{2.176286in}{1.313830in}}%
\pgfpathlineto{\pgfqpoint{2.176286in}{1.313830in}}%
\pgfpathlineto{\pgfqpoint{2.176664in}{1.312540in}}%
\pgfpathlineto{\pgfqpoint{2.176997in}{1.313682in}}%
\pgfpathlineto{\pgfqpoint{2.176997in}{1.313682in}}%
\pgfpathlineto{\pgfqpoint{2.177450in}{1.315295in}}%
\pgfpathlineto{\pgfqpoint{2.177841in}{1.313632in}}%
\pgfpathlineto{\pgfqpoint{2.177841in}{1.313632in}}%
\pgfpathlineto{\pgfqpoint{2.178169in}{1.312491in}}%
\pgfpathlineto{\pgfqpoint{2.178502in}{1.313516in}}%
\pgfpathlineto{\pgfqpoint{2.178502in}{1.313516in}}%
\pgfpathlineto{\pgfqpoint{2.179069in}{1.315241in}}%
\pgfpathlineto{\pgfqpoint{2.179400in}{1.313444in}}%
\pgfpathlineto{\pgfqpoint{2.179728in}{1.312479in}}%
\pgfpathlineto{\pgfqpoint{2.179950in}{1.313086in}}%
\pgfpathlineto{\pgfqpoint{2.179950in}{1.313086in}}%
\pgfpathlineto{\pgfqpoint{2.180517in}{1.315256in}}%
\pgfpathlineto{\pgfqpoint{2.180928in}{1.313507in}}%
\pgfpathlineto{\pgfqpoint{2.181257in}{1.312443in}}%
\pgfpathlineto{\pgfqpoint{2.181591in}{1.313519in}}%
\pgfpathlineto{\pgfqpoint{2.181591in}{1.313519in}}%
\pgfpathlineto{\pgfqpoint{2.182048in}{1.315230in}}%
\pgfpathlineto{\pgfqpoint{2.182472in}{1.313466in}}%
\pgfpathlineto{\pgfqpoint{2.182802in}{1.312417in}}%
\pgfpathlineto{\pgfqpoint{2.183137in}{1.313506in}}%
\pgfpathlineto{\pgfqpoint{2.183137in}{1.313506in}}%
\pgfpathlineto{\pgfqpoint{2.183594in}{1.315214in}}%
\pgfpathlineto{\pgfqpoint{2.184030in}{1.313354in}}%
\pgfpathlineto{\pgfqpoint{2.184361in}{1.312400in}}%
\pgfpathlineto{\pgfqpoint{2.184584in}{1.313022in}}%
\pgfpathlineto{\pgfqpoint{2.184584in}{1.313022in}}%
\pgfpathlineto{\pgfqpoint{2.185154in}{1.315204in}}%
\pgfpathlineto{\pgfqpoint{2.185623in}{1.313040in}}%
\pgfpathlineto{\pgfqpoint{2.185843in}{1.312354in}}%
\pgfpathlineto{\pgfqpoint{2.186066in}{1.312736in}}%
\pgfpathlineto{\pgfqpoint{2.186066in}{1.312736in}}%
\pgfpathlineto{\pgfqpoint{2.186750in}{1.315190in}}%
\pgfpathlineto{\pgfqpoint{2.187208in}{1.312832in}}%
\pgfpathlineto{\pgfqpoint{2.187429in}{1.312324in}}%
\pgfpathlineto{\pgfqpoint{2.187653in}{1.312834in}}%
\pgfpathlineto{\pgfqpoint{2.187653in}{1.312834in}}%
\pgfpathlineto{\pgfqpoint{2.188339in}{1.315145in}}%
\pgfpathlineto{\pgfqpoint{2.188671in}{1.313390in}}%
\pgfpathlineto{\pgfqpoint{2.189003in}{1.312306in}}%
\pgfpathlineto{\pgfqpoint{2.189341in}{1.313399in}}%
\pgfpathlineto{\pgfqpoint{2.189341in}{1.313399in}}%
\pgfpathlineto{\pgfqpoint{2.189802in}{1.315140in}}%
\pgfpathlineto{\pgfqpoint{2.190236in}{1.313310in}}%
\pgfpathlineto{\pgfqpoint{2.190569in}{1.312283in}}%
\pgfpathlineto{\pgfqpoint{2.190908in}{1.313412in}}%
\pgfpathlineto{\pgfqpoint{2.190908in}{1.313412in}}%
\pgfpathlineto{\pgfqpoint{2.191369in}{1.315127in}}%
\pgfpathlineto{\pgfqpoint{2.191773in}{1.313473in}}%
\pgfpathlineto{\pgfqpoint{2.191773in}{1.313473in}}%
\pgfpathlineto{\pgfqpoint{2.192107in}{1.312242in}}%
\pgfpathlineto{\pgfqpoint{2.192446in}{1.313268in}}%
\pgfpathlineto{\pgfqpoint{2.192446in}{1.313268in}}%
\pgfpathlineto{\pgfqpoint{2.193023in}{1.315088in}}%
\pgfpathlineto{\pgfqpoint{2.193376in}{1.313159in}}%
\pgfpathlineto{\pgfqpoint{2.193711in}{1.312242in}}%
\pgfpathlineto{\pgfqpoint{2.193937in}{1.312899in}}%
\pgfpathlineto{\pgfqpoint{2.193937in}{1.312899in}}%
\pgfpathlineto{\pgfqpoint{2.194514in}{1.315099in}}%
\pgfpathlineto{\pgfqpoint{2.194931in}{1.313227in}}%
\pgfpathlineto{\pgfqpoint{2.195266in}{1.312201in}}%
\pgfpathlineto{\pgfqpoint{2.195492in}{1.312818in}}%
\pgfpathlineto{\pgfqpoint{2.195492in}{1.312818in}}%
\pgfpathlineto{\pgfqpoint{2.196070in}{1.315073in}}%
\pgfpathlineto{\pgfqpoint{2.196508in}{1.313151in}}%
\pgfpathlineto{\pgfqpoint{2.196844in}{1.312180in}}%
\pgfpathlineto{\pgfqpoint{2.197071in}{1.312822in}}%
\pgfpathlineto{\pgfqpoint{2.197071in}{1.312822in}}%
\pgfpathlineto{\pgfqpoint{2.197651in}{1.315059in}}%
\pgfpathlineto{\pgfqpoint{2.198084in}{1.313112in}}%
\pgfpathlineto{\pgfqpoint{2.198421in}{1.312154in}}%
\pgfpathlineto{\pgfqpoint{2.198648in}{1.312805in}}%
\pgfpathlineto{\pgfqpoint{2.198648in}{1.312805in}}%
\pgfpathlineto{\pgfqpoint{2.199229in}{1.315042in}}%
\pgfpathlineto{\pgfqpoint{2.199654in}{1.313144in}}%
\pgfpathlineto{\pgfqpoint{2.199991in}{1.312118in}}%
\pgfpathlineto{\pgfqpoint{2.200219in}{1.312745in}}%
\pgfpathlineto{\pgfqpoint{2.200219in}{1.312745in}}%
\pgfpathlineto{\pgfqpoint{2.200800in}{1.315019in}}%
\pgfpathlineto{\pgfqpoint{2.201320in}{1.312583in}}%
\pgfpathlineto{\pgfqpoint{2.201545in}{1.312073in}}%
\pgfpathlineto{\pgfqpoint{2.201773in}{1.312608in}}%
\pgfpathlineto{\pgfqpoint{2.201773in}{1.312608in}}%
\pgfpathlineto{\pgfqpoint{2.202472in}{1.314977in}}%
\pgfpathlineto{\pgfqpoint{2.202792in}{1.313291in}}%
\pgfpathlineto{\pgfqpoint{2.203131in}{1.312045in}}%
\pgfpathlineto{\pgfqpoint{2.203475in}{1.313104in}}%
\pgfpathlineto{\pgfqpoint{2.203475in}{1.313104in}}%
\pgfpathlineto{\pgfqpoint{2.204060in}{1.314956in}}%
\pgfpathlineto{\pgfqpoint{2.204375in}{1.313298in}}%
\pgfpathlineto{\pgfqpoint{2.204375in}{1.313298in}}%
\pgfpathlineto{\pgfqpoint{2.204715in}{1.312015in}}%
\pgfpathlineto{\pgfqpoint{2.205059in}{1.313060in}}%
\pgfpathlineto{\pgfqpoint{2.205059in}{1.313060in}}%
\pgfpathlineto{\pgfqpoint{2.205645in}{1.314943in}}%
\pgfpathlineto{\pgfqpoint{2.205970in}{1.313238in}}%
\pgfpathlineto{\pgfqpoint{2.206310in}{1.311988in}}%
\pgfpathlineto{\pgfqpoint{2.206655in}{1.313056in}}%
\pgfpathlineto{\pgfqpoint{2.206655in}{1.313056in}}%
\pgfpathlineto{\pgfqpoint{2.207242in}{1.314918in}}%
\pgfpathlineto{\pgfqpoint{2.207607in}{1.312881in}}%
\pgfpathlineto{\pgfqpoint{2.207948in}{1.311996in}}%
\pgfpathlineto{\pgfqpoint{2.208178in}{1.312698in}}%
\pgfpathlineto{\pgfqpoint{2.208178in}{1.312698in}}%
\pgfpathlineto{\pgfqpoint{2.208767in}{1.314936in}}%
\pgfpathlineto{\pgfqpoint{2.209159in}{1.313196in}}%
\pgfpathlineto{\pgfqpoint{2.209500in}{1.311930in}}%
\pgfpathlineto{\pgfqpoint{2.209847in}{1.313002in}}%
\pgfpathlineto{\pgfqpoint{2.209847in}{1.313002in}}%
\pgfpathlineto{\pgfqpoint{2.210437in}{1.314882in}}%
\pgfpathlineto{\pgfqpoint{2.210749in}{1.313250in}}%
\pgfpathlineto{\pgfqpoint{2.210749in}{1.313250in}}%
\pgfpathlineto{\pgfqpoint{2.211068in}{1.311900in}}%
\pgfpathlineto{\pgfqpoint{2.211415in}{1.312817in}}%
\pgfpathlineto{\pgfqpoint{2.211415in}{1.312817in}}%
\pgfpathlineto{\pgfqpoint{2.212006in}{1.314894in}}%
\pgfpathlineto{\pgfqpoint{2.212407in}{1.312804in}}%
\pgfpathlineto{\pgfqpoint{2.212751in}{1.311909in}}%
\pgfpathlineto{\pgfqpoint{2.212982in}{1.312618in}}%
\pgfpathlineto{\pgfqpoint{2.212982in}{1.312618in}}%
\pgfpathlineto{\pgfqpoint{2.213574in}{1.314880in}}%
\pgfpathlineto{\pgfqpoint{2.213975in}{1.313072in}}%
\pgfpathlineto{\pgfqpoint{2.214319in}{1.311845in}}%
\pgfpathlineto{\pgfqpoint{2.214667in}{1.312956in}}%
\pgfpathlineto{\pgfqpoint{2.214667in}{1.312956in}}%
\pgfpathlineto{\pgfqpoint{2.215261in}{1.314817in}}%
\pgfpathlineto{\pgfqpoint{2.215583in}{1.313066in}}%
\pgfpathlineto{\pgfqpoint{2.215927in}{1.311814in}}%
\pgfpathlineto{\pgfqpoint{2.216277in}{1.312919in}}%
\pgfpathlineto{\pgfqpoint{2.216277in}{1.312919in}}%
\pgfpathlineto{\pgfqpoint{2.216871in}{1.314802in}}%
\pgfpathlineto{\pgfqpoint{2.217209in}{1.312939in}}%
\pgfpathlineto{\pgfqpoint{2.217554in}{1.311792in}}%
\pgfpathlineto{\pgfqpoint{2.217905in}{1.312959in}}%
\pgfpathlineto{\pgfqpoint{2.217905in}{1.312959in}}%
\pgfpathlineto{\pgfqpoint{2.218383in}{1.314807in}}%
\pgfpathlineto{\pgfqpoint{2.218837in}{1.312828in}}%
\pgfpathlineto{\pgfqpoint{2.219183in}{1.311772in}}%
\pgfpathlineto{\pgfqpoint{2.219417in}{1.312431in}}%
\pgfpathlineto{\pgfqpoint{2.219417in}{1.312431in}}%
\pgfpathlineto{\pgfqpoint{2.220013in}{1.314796in}}%
\pgfpathlineto{\pgfqpoint{2.220440in}{1.312935in}}%
\pgfpathlineto{\pgfqpoint{2.220787in}{1.311729in}}%
\pgfpathlineto{\pgfqpoint{2.221139in}{1.312877in}}%
\pgfpathlineto{\pgfqpoint{2.221139in}{1.312877in}}%
\pgfpathlineto{\pgfqpoint{2.221620in}{1.314764in}}%
\pgfpathlineto{\pgfqpoint{2.222070in}{1.312866in}}%
\pgfpathlineto{\pgfqpoint{2.222417in}{1.311703in}}%
\pgfpathlineto{\pgfqpoint{2.222770in}{1.312879in}}%
\pgfpathlineto{\pgfqpoint{2.222770in}{1.312879in}}%
\pgfpathlineto{\pgfqpoint{2.223251in}{1.314750in}}%
\pgfpathlineto{\pgfqpoint{2.223696in}{1.312847in}}%
\pgfpathlineto{\pgfqpoint{2.224044in}{1.311672in}}%
\pgfpathlineto{\pgfqpoint{2.224397in}{1.312849in}}%
\pgfpathlineto{\pgfqpoint{2.224397in}{1.312849in}}%
\pgfpathlineto{\pgfqpoint{2.224880in}{1.314730in}}%
\pgfpathlineto{\pgfqpoint{2.225343in}{1.312695in}}%
\pgfpathlineto{\pgfqpoint{2.225692in}{1.311656in}}%
\pgfpathlineto{\pgfqpoint{2.225928in}{1.312339in}}%
\pgfpathlineto{\pgfqpoint{2.225928in}{1.312339in}}%
\pgfpathlineto{\pgfqpoint{2.226529in}{1.314722in}}%
\pgfpathlineto{\pgfqpoint{2.226978in}{1.312662in}}%
\pgfpathlineto{\pgfqpoint{2.227328in}{1.311627in}}%
\pgfpathlineto{\pgfqpoint{2.227564in}{1.312315in}}%
\pgfpathlineto{\pgfqpoint{2.227564in}{1.312315in}}%
\pgfpathlineto{\pgfqpoint{2.228167in}{1.314704in}}%
\pgfpathlineto{\pgfqpoint{2.228606in}{1.312708in}}%
\pgfpathlineto{\pgfqpoint{2.228957in}{1.311588in}}%
\pgfpathlineto{\pgfqpoint{2.229313in}{1.312812in}}%
\pgfpathlineto{\pgfqpoint{2.229313in}{1.312812in}}%
\pgfpathlineto{\pgfqpoint{2.229797in}{1.314678in}}%
\pgfpathlineto{\pgfqpoint{2.230240in}{1.312737in}}%
\pgfpathlineto{\pgfqpoint{2.230591in}{1.311552in}}%
\pgfpathlineto{\pgfqpoint{2.230948in}{1.312749in}}%
\pgfpathlineto{\pgfqpoint{2.230948in}{1.312749in}}%
\pgfpathlineto{\pgfqpoint{2.231435in}{1.314654in}}%
\pgfpathlineto{\pgfqpoint{2.231871in}{1.312828in}}%
\pgfpathlineto{\pgfqpoint{2.232223in}{1.311513in}}%
\pgfpathlineto{\pgfqpoint{2.232580in}{1.312648in}}%
\pgfpathlineto{\pgfqpoint{2.232580in}{1.312648in}}%
\pgfpathlineto{\pgfqpoint{2.233187in}{1.314612in}}%
\pgfpathlineto{\pgfqpoint{2.233543in}{1.312614in}}%
\pgfpathlineto{\pgfqpoint{2.233895in}{1.311498in}}%
\pgfpathlineto{\pgfqpoint{2.234133in}{1.312168in}}%
\pgfpathlineto{\pgfqpoint{2.234133in}{1.312168in}}%
\pgfpathlineto{\pgfqpoint{2.234741in}{1.314621in}}%
\pgfpathlineto{\pgfqpoint{2.235163in}{1.312838in}}%
\pgfpathlineto{\pgfqpoint{2.235514in}{1.311449in}}%
\pgfpathlineto{\pgfqpoint{2.235872in}{1.312543in}}%
\pgfpathlineto{\pgfqpoint{2.235872in}{1.312543in}}%
\pgfpathlineto{\pgfqpoint{2.236482in}{1.314586in}}%
\pgfpathlineto{\pgfqpoint{2.236835in}{1.312683in}}%
\pgfpathlineto{\pgfqpoint{2.237189in}{1.311425in}}%
\pgfpathlineto{\pgfqpoint{2.237548in}{1.312610in}}%
\pgfpathlineto{\pgfqpoint{2.237548in}{1.312610in}}%
\pgfpathlineto{\pgfqpoint{2.238040in}{1.314570in}}%
\pgfpathlineto{\pgfqpoint{2.238494in}{1.312656in}}%
\pgfpathlineto{\pgfqpoint{2.238849in}{1.311394in}}%
\pgfpathlineto{\pgfqpoint{2.239209in}{1.312584in}}%
\pgfpathlineto{\pgfqpoint{2.239209in}{1.312584in}}%
\pgfpathlineto{\pgfqpoint{2.239701in}{1.314551in}}%
\pgfpathlineto{\pgfqpoint{2.240179in}{1.312451in}}%
\pgfpathlineto{\pgfqpoint{2.240535in}{1.311381in}}%
\pgfpathlineto{\pgfqpoint{2.240775in}{1.312087in}}%
\pgfpathlineto{\pgfqpoint{2.240775in}{1.312087in}}%
\pgfpathlineto{\pgfqpoint{2.241389in}{1.314549in}}%
\pgfpathlineto{\pgfqpoint{2.241841in}{1.312459in}}%
\pgfpathlineto{\pgfqpoint{2.242197in}{1.311345in}}%
\pgfpathlineto{\pgfqpoint{2.242438in}{1.312037in}}%
\pgfpathlineto{\pgfqpoint{2.242438in}{1.312037in}}%
\pgfpathlineto{\pgfqpoint{2.243053in}{1.314526in}}%
\pgfpathlineto{\pgfqpoint{2.243502in}{1.312501in}}%
\pgfpathlineto{\pgfqpoint{2.243859in}{1.311306in}}%
\pgfpathlineto{\pgfqpoint{2.244222in}{1.312552in}}%
\pgfpathlineto{\pgfqpoint{2.244222in}{1.312552in}}%
\pgfpathlineto{\pgfqpoint{2.244717in}{1.314500in}}%
\pgfpathlineto{\pgfqpoint{2.245161in}{1.312597in}}%
\pgfpathlineto{\pgfqpoint{2.245519in}{1.311265in}}%
\pgfpathlineto{\pgfqpoint{2.245882in}{1.312446in}}%
\pgfpathlineto{\pgfqpoint{2.245882in}{1.312446in}}%
\pgfpathlineto{\pgfqpoint{2.246501in}{1.314452in}}%
\pgfpathlineto{\pgfqpoint{2.246827in}{1.312667in}}%
\pgfpathlineto{\pgfqpoint{2.247176in}{1.311228in}}%
\pgfpathlineto{\pgfqpoint{2.247540in}{1.312310in}}%
\pgfpathlineto{\pgfqpoint{2.247540in}{1.312310in}}%
\pgfpathlineto{\pgfqpoint{2.248159in}{1.314458in}}%
\pgfpathlineto{\pgfqpoint{2.248508in}{1.312640in}}%
\pgfpathlineto{\pgfqpoint{2.248858in}{1.311196in}}%
\pgfpathlineto{\pgfqpoint{2.249222in}{1.312282in}}%
\pgfpathlineto{\pgfqpoint{2.249222in}{1.312282in}}%
\pgfpathlineto{\pgfqpoint{2.249843in}{1.314438in}}%
\pgfpathlineto{\pgfqpoint{2.250217in}{1.312411in}}%
\pgfpathlineto{\pgfqpoint{2.250577in}{1.311178in}}%
\pgfpathlineto{\pgfqpoint{2.250944in}{1.312431in}}%
\pgfpathlineto{\pgfqpoint{2.250944in}{1.312431in}}%
\pgfpathlineto{\pgfqpoint{2.251443in}{1.314419in}}%
\pgfpathlineto{\pgfqpoint{2.251870in}{1.312686in}}%
\pgfpathlineto{\pgfqpoint{2.251870in}{1.312686in}}%
\pgfpathlineto{\pgfqpoint{2.252286in}{1.311163in}}%
\pgfpathlineto{\pgfqpoint{2.252654in}{1.312505in}}%
\pgfpathlineto{\pgfqpoint{2.252654in}{1.312505in}}%
\pgfpathlineto{\pgfqpoint{2.253153in}{1.314412in}}%
\pgfpathlineto{\pgfqpoint{2.253569in}{1.312601in}}%
\pgfpathlineto{\pgfqpoint{2.253569in}{1.312601in}}%
\pgfpathlineto{\pgfqpoint{2.253898in}{1.311108in}}%
\pgfpathlineto{\pgfqpoint{2.254388in}{1.312691in}}%
\pgfpathlineto{\pgfqpoint{2.254388in}{1.312691in}}%
\pgfpathlineto{\pgfqpoint{2.254889in}{1.314396in}}%
\pgfpathlineto{\pgfqpoint{2.255253in}{1.312687in}}%
\pgfpathlineto{\pgfqpoint{2.255253in}{1.312687in}}%
\pgfpathlineto{\pgfqpoint{2.255641in}{1.311072in}}%
\pgfpathlineto{\pgfqpoint{2.256009in}{1.312263in}}%
\pgfpathlineto{\pgfqpoint{2.256009in}{1.312263in}}%
\pgfpathlineto{\pgfqpoint{2.256636in}{1.314336in}}%
\pgfpathlineto{\pgfqpoint{2.256992in}{1.312325in}}%
\pgfpathlineto{\pgfqpoint{2.257355in}{1.311046in}}%
\pgfpathlineto{\pgfqpoint{2.257725in}{1.312305in}}%
\pgfpathlineto{\pgfqpoint{2.257725in}{1.312305in}}%
\pgfpathlineto{\pgfqpoint{2.258229in}{1.314336in}}%
\pgfpathlineto{\pgfqpoint{2.258666in}{1.312552in}}%
\pgfpathlineto{\pgfqpoint{2.258666in}{1.312552in}}%
\pgfpathlineto{\pgfqpoint{2.258979in}{1.311028in}}%
\pgfpathlineto{\pgfqpoint{2.259472in}{1.312496in}}%
\pgfpathlineto{\pgfqpoint{2.259472in}{1.312496in}}%
\pgfpathlineto{\pgfqpoint{2.259976in}{1.314338in}}%
\pgfpathlineto{\pgfqpoint{2.260368in}{1.312570in}}%
\pgfpathlineto{\pgfqpoint{2.260368in}{1.312570in}}%
\pgfpathlineto{\pgfqpoint{2.260780in}{1.310990in}}%
\pgfpathlineto{\pgfqpoint{2.261151in}{1.312307in}}%
\pgfpathlineto{\pgfqpoint{2.261151in}{1.312307in}}%
\pgfpathlineto{\pgfqpoint{2.261656in}{1.314304in}}%
\pgfpathlineto{\pgfqpoint{2.262099in}{1.312378in}}%
\pgfpathlineto{\pgfqpoint{2.262465in}{1.310940in}}%
\pgfpathlineto{\pgfqpoint{2.262837in}{1.312136in}}%
\pgfpathlineto{\pgfqpoint{2.262837in}{1.312136in}}%
\pgfpathlineto{\pgfqpoint{2.263469in}{1.314259in}}%
\pgfpathlineto{\pgfqpoint{2.263838in}{1.312157in}}%
\pgfpathlineto{\pgfqpoint{2.264205in}{1.310920in}}%
\pgfpathlineto{\pgfqpoint{2.264577in}{1.312230in}}%
\pgfpathlineto{\pgfqpoint{2.264577in}{1.312230in}}%
\pgfpathlineto{\pgfqpoint{2.265086in}{1.314261in}}%
\pgfpathlineto{\pgfqpoint{2.265541in}{1.312258in}}%
\pgfpathlineto{\pgfqpoint{2.265909in}{1.310877in}}%
\pgfpathlineto{\pgfqpoint{2.266283in}{1.312118in}}%
\pgfpathlineto{\pgfqpoint{2.266283in}{1.312118in}}%
\pgfpathlineto{\pgfqpoint{2.266918in}{1.314207in}}%
\pgfpathlineto{\pgfqpoint{2.267248in}{1.312377in}}%
\pgfpathlineto{\pgfqpoint{2.267586in}{1.310843in}}%
\pgfpathlineto{\pgfqpoint{2.267960in}{1.311848in}}%
\pgfpathlineto{\pgfqpoint{2.267960in}{1.311848in}}%
\pgfpathlineto{\pgfqpoint{2.268596in}{1.314233in}}%
\pgfpathlineto{\pgfqpoint{2.269008in}{1.312067in}}%
\pgfpathlineto{\pgfqpoint{2.269378in}{1.310821in}}%
\pgfpathlineto{\pgfqpoint{2.269754in}{1.312148in}}%
\pgfpathlineto{\pgfqpoint{2.269754in}{1.312148in}}%
\pgfpathlineto{\pgfqpoint{2.270265in}{1.314200in}}%
\pgfpathlineto{\pgfqpoint{2.270726in}{1.312152in}}%
\pgfpathlineto{\pgfqpoint{2.271096in}{1.310778in}}%
\pgfpathlineto{\pgfqpoint{2.271472in}{1.312046in}}%
\pgfpathlineto{\pgfqpoint{2.271472in}{1.312046in}}%
\pgfpathlineto{\pgfqpoint{2.271987in}{1.314167in}}%
\pgfpathlineto{\pgfqpoint{2.272443in}{1.312288in}}%
\pgfpathlineto{\pgfqpoint{2.272787in}{1.310740in}}%
\pgfpathlineto{\pgfqpoint{2.273163in}{1.311781in}}%
\pgfpathlineto{\pgfqpoint{2.273163in}{1.311781in}}%
\pgfpathlineto{\pgfqpoint{2.273804in}{1.314169in}}%
\pgfpathlineto{\pgfqpoint{2.274185in}{1.312229in}}%
\pgfpathlineto{\pgfqpoint{2.274548in}{1.310703in}}%
\pgfpathlineto{\pgfqpoint{2.274926in}{1.311858in}}%
\pgfpathlineto{\pgfqpoint{2.274926in}{1.311858in}}%
\pgfpathlineto{\pgfqpoint{2.275568in}{1.314133in}}%
\pgfpathlineto{\pgfqpoint{2.275926in}{1.312223in}}%
\pgfpathlineto{\pgfqpoint{2.276278in}{1.310670in}}%
\pgfpathlineto{\pgfqpoint{2.276656in}{1.311752in}}%
\pgfpathlineto{\pgfqpoint{2.276656in}{1.311752in}}%
\pgfpathlineto{\pgfqpoint{2.277299in}{1.314124in}}%
\pgfpathlineto{\pgfqpoint{2.277679in}{1.312147in}}%
\pgfpathlineto{\pgfqpoint{2.278053in}{1.310637in}}%
\pgfpathlineto{\pgfqpoint{2.278432in}{1.311863in}}%
\pgfpathlineto{\pgfqpoint{2.278432in}{1.311863in}}%
\pgfpathlineto{\pgfqpoint{2.279078in}{1.314077in}}%
\pgfpathlineto{\pgfqpoint{2.279433in}{1.312102in}}%
\pgfpathlineto{\pgfqpoint{2.279807in}{1.310603in}}%
\pgfpathlineto{\pgfqpoint{2.280188in}{1.311843in}}%
\pgfpathlineto{\pgfqpoint{2.280188in}{1.311843in}}%
\pgfpathlineto{\pgfqpoint{2.280835in}{1.314053in}}%
\pgfpathlineto{\pgfqpoint{2.281209in}{1.311899in}}%
\pgfpathlineto{\pgfqpoint{2.281585in}{1.310580in}}%
\pgfpathlineto{\pgfqpoint{2.281966in}{1.311923in}}%
\pgfpathlineto{\pgfqpoint{2.281966in}{1.311923in}}%
\pgfpathlineto{\pgfqpoint{2.282487in}{1.314052in}}%
\pgfpathlineto{\pgfqpoint{2.282957in}{1.311971in}}%
\pgfpathlineto{\pgfqpoint{2.283333in}{1.310537in}}%
\pgfpathlineto{\pgfqpoint{2.283716in}{1.311829in}}%
\pgfpathlineto{\pgfqpoint{2.283716in}{1.311829in}}%
\pgfpathlineto{\pgfqpoint{2.284366in}{1.313999in}}%
\pgfpathlineto{\pgfqpoint{2.284729in}{1.311873in}}%
\pgfpathlineto{\pgfqpoint{2.285106in}{1.310507in}}%
\pgfpathlineto{\pgfqpoint{2.285490in}{1.311843in}}%
\pgfpathlineto{\pgfqpoint{2.285490in}{1.311843in}}%
\pgfpathlineto{\pgfqpoint{2.286013in}{1.314006in}}%
\pgfpathlineto{\pgfqpoint{2.286468in}{1.312094in}}%
\pgfpathlineto{\pgfqpoint{2.286802in}{1.310478in}}%
\pgfpathlineto{\pgfqpoint{2.287314in}{1.312105in}}%
\pgfpathlineto{\pgfqpoint{2.287314in}{1.312105in}}%
\pgfpathlineto{\pgfqpoint{2.287837in}{1.314010in}}%
\pgfpathlineto{\pgfqpoint{2.288246in}{1.312015in}}%
\pgfpathlineto{\pgfqpoint{2.288608in}{1.310426in}}%
\pgfpathlineto{\pgfqpoint{2.288993in}{1.311569in}}%
\pgfpathlineto{\pgfqpoint{2.288993in}{1.311569in}}%
\pgfpathlineto{\pgfqpoint{2.289647in}{1.313974in}}%
\pgfpathlineto{\pgfqpoint{2.290039in}{1.311848in}}%
\pgfpathlineto{\pgfqpoint{2.290419in}{1.310398in}}%
\pgfpathlineto{\pgfqpoint{2.290805in}{1.311714in}}%
\pgfpathlineto{\pgfqpoint{2.290805in}{1.311714in}}%
\pgfpathlineto{\pgfqpoint{2.291461in}{1.313913in}}%
\pgfpathlineto{\pgfqpoint{2.291823in}{1.311789in}}%
\pgfpathlineto{\pgfqpoint{2.292204in}{1.310364in}}%
\pgfpathlineto{\pgfqpoint{2.292591in}{1.311702in}}%
\pgfpathlineto{\pgfqpoint{2.292591in}{1.311702in}}%
\pgfpathlineto{\pgfqpoint{2.293120in}{1.313918in}}%
\pgfpathlineto{\pgfqpoint{2.293609in}{1.311753in}}%
\pgfpathlineto{\pgfqpoint{2.293991in}{1.310329in}}%
\pgfpathlineto{\pgfqpoint{2.294379in}{1.311675in}}%
\pgfpathlineto{\pgfqpoint{2.294379in}{1.311675in}}%
\pgfpathlineto{\pgfqpoint{2.294909in}{1.313897in}}%
\pgfpathlineto{\pgfqpoint{2.295401in}{1.311705in}}%
\pgfpathlineto{\pgfqpoint{2.295784in}{1.310295in}}%
\pgfpathlineto{\pgfqpoint{2.296172in}{1.311657in}}%
\pgfpathlineto{\pgfqpoint{2.296172in}{1.311657in}}%
\pgfpathlineto{\pgfqpoint{2.296704in}{1.313878in}}%
\pgfpathlineto{\pgfqpoint{2.297195in}{1.311672in}}%
\pgfpathlineto{\pgfqpoint{2.297579in}{1.310259in}}%
\pgfpathlineto{\pgfqpoint{2.297968in}{1.311629in}}%
\pgfpathlineto{\pgfqpoint{2.297968in}{1.311629in}}%
\pgfpathlineto{\pgfqpoint{2.298501in}{1.313857in}}%
\pgfpathlineto{\pgfqpoint{2.298985in}{1.311714in}}%
\pgfpathlineto{\pgfqpoint{2.299369in}{1.310218in}}%
\pgfpathlineto{\pgfqpoint{2.299760in}{1.311552in}}%
\pgfpathlineto{\pgfqpoint{2.299760in}{1.311552in}}%
\pgfpathlineto{\pgfqpoint{2.300424in}{1.313809in}}%
\pgfpathlineto{\pgfqpoint{2.300784in}{1.311715in}}%
\pgfpathlineto{\pgfqpoint{2.301169in}{1.310180in}}%
\pgfpathlineto{\pgfqpoint{2.301561in}{1.311502in}}%
\pgfpathlineto{\pgfqpoint{2.301561in}{1.311502in}}%
\pgfpathlineto{\pgfqpoint{2.302227in}{1.313793in}}%
\pgfpathlineto{\pgfqpoint{2.302586in}{1.311725in}}%
\pgfpathlineto{\pgfqpoint{2.302973in}{1.310142in}}%
\pgfpathlineto{\pgfqpoint{2.303365in}{1.311447in}}%
\pgfpathlineto{\pgfqpoint{2.303365in}{1.311447in}}%
\pgfpathlineto{\pgfqpoint{2.304033in}{1.313778in}}%
\pgfpathlineto{\pgfqpoint{2.304422in}{1.311482in}}%
\pgfpathlineto{\pgfqpoint{2.304810in}{1.310120in}}%
\pgfpathlineto{\pgfqpoint{2.305203in}{1.311550in}}%
\pgfpathlineto{\pgfqpoint{2.305203in}{1.311550in}}%
\pgfpathlineto{\pgfqpoint{2.305740in}{1.313778in}}%
\pgfpathlineto{\pgfqpoint{2.306205in}{1.311737in}}%
\pgfpathlineto{\pgfqpoint{2.306567in}{1.310069in}}%
\pgfpathlineto{\pgfqpoint{2.306960in}{1.311200in}}%
\pgfpathlineto{\pgfqpoint{2.306960in}{1.311200in}}%
\pgfpathlineto{\pgfqpoint{2.307631in}{1.313768in}}%
\pgfpathlineto{\pgfqpoint{2.308061in}{1.311399in}}%
\pgfpathlineto{\pgfqpoint{2.308450in}{1.310048in}}%
\pgfpathlineto{\pgfqpoint{2.308846in}{1.311503in}}%
\pgfpathlineto{\pgfqpoint{2.308846in}{1.311503in}}%
\pgfpathlineto{\pgfqpoint{2.309385in}{1.313737in}}%
\pgfpathlineto{\pgfqpoint{2.309865in}{1.311542in}}%
\pgfpathlineto{\pgfqpoint{2.310255in}{1.309998in}}%
\pgfpathlineto{\pgfqpoint{2.310652in}{1.311359in}}%
\pgfpathlineto{\pgfqpoint{2.310652in}{1.311359in}}%
\pgfpathlineto{\pgfqpoint{2.311326in}{1.313681in}}%
\pgfpathlineto{\pgfqpoint{2.311716in}{1.311322in}}%
\pgfpathlineto{\pgfqpoint{2.312108in}{1.309976in}}%
\pgfpathlineto{\pgfqpoint{2.312372in}{1.310767in}}%
\pgfpathlineto{\pgfqpoint{2.312372in}{1.310767in}}%
\pgfpathlineto{\pgfqpoint{2.313047in}{1.313694in}}%
\pgfpathlineto{\pgfqpoint{2.313547in}{1.311321in}}%
\pgfpathlineto{\pgfqpoint{2.313939in}{1.309935in}}%
\pgfpathlineto{\pgfqpoint{2.314338in}{1.311398in}}%
\pgfpathlineto{\pgfqpoint{2.314338in}{1.311398in}}%
\pgfpathlineto{\pgfqpoint{2.314881in}{1.313670in}}%
\pgfpathlineto{\pgfqpoint{2.315374in}{1.311386in}}%
\pgfpathlineto{\pgfqpoint{2.315767in}{1.309889in}}%
\pgfpathlineto{\pgfqpoint{2.316167in}{1.311303in}}%
\pgfpathlineto{\pgfqpoint{2.316167in}{1.311303in}}%
\pgfpathlineto{\pgfqpoint{2.316712in}{1.313637in}}%
\pgfpathlineto{\pgfqpoint{2.317197in}{1.311535in}}%
\pgfpathlineto{\pgfqpoint{2.317575in}{1.309842in}}%
\pgfpathlineto{\pgfqpoint{2.317975in}{1.311075in}}%
\pgfpathlineto{\pgfqpoint{2.317975in}{1.311075in}}%
\pgfpathlineto{\pgfqpoint{2.318655in}{1.313628in}}%
\pgfpathlineto{\pgfqpoint{2.319065in}{1.311316in}}%
\pgfpathlineto{\pgfqpoint{2.319461in}{1.309814in}}%
\pgfpathlineto{\pgfqpoint{2.319862in}{1.311244in}}%
\pgfpathlineto{\pgfqpoint{2.319862in}{1.311244in}}%
\pgfpathlineto{\pgfqpoint{2.320411in}{1.313594in}}%
\pgfpathlineto{\pgfqpoint{2.320902in}{1.311422in}}%
\pgfpathlineto{\pgfqpoint{2.321298in}{1.309769in}}%
\pgfpathlineto{\pgfqpoint{2.321701in}{1.311127in}}%
\pgfpathlineto{\pgfqpoint{2.321701in}{1.311127in}}%
\pgfpathlineto{\pgfqpoint{2.322385in}{1.313559in}}%
\pgfpathlineto{\pgfqpoint{2.322760in}{1.311384in}}%
\pgfpathlineto{\pgfqpoint{2.323157in}{1.309731in}}%
\pgfpathlineto{\pgfqpoint{2.323560in}{1.311099in}}%
\pgfpathlineto{\pgfqpoint{2.323560in}{1.311099in}}%
\pgfpathlineto{\pgfqpoint{2.324247in}{1.313535in}}%
\pgfpathlineto{\pgfqpoint{2.324657in}{1.311041in}}%
\pgfpathlineto{\pgfqpoint{2.325056in}{1.309718in}}%
\pgfpathlineto{\pgfqpoint{2.325325in}{1.310558in}}%
\pgfpathlineto{\pgfqpoint{2.325325in}{1.310558in}}%
\pgfpathlineto{\pgfqpoint{2.326012in}{1.313545in}}%
\pgfpathlineto{\pgfqpoint{2.326495in}{1.311255in}}%
\pgfpathlineto{\pgfqpoint{2.326894in}{1.309657in}}%
\pgfpathlineto{\pgfqpoint{2.327300in}{1.311073in}}%
\pgfpathlineto{\pgfqpoint{2.327300in}{1.311073in}}%
\pgfpathlineto{\pgfqpoint{2.327990in}{1.313480in}}%
\pgfpathlineto{\pgfqpoint{2.328357in}{1.311308in}}%
\pgfpathlineto{\pgfqpoint{2.328757in}{1.309614in}}%
\pgfpathlineto{\pgfqpoint{2.329164in}{1.310989in}}%
\pgfpathlineto{\pgfqpoint{2.329164in}{1.310989in}}%
\pgfpathlineto{\pgfqpoint{2.329855in}{1.313471in}}%
\pgfpathlineto{\pgfqpoint{2.330231in}{1.311296in}}%
\pgfpathlineto{\pgfqpoint{2.330626in}{1.309573in}}%
\pgfpathlineto{\pgfqpoint{2.331034in}{1.310912in}}%
\pgfpathlineto{\pgfqpoint{2.331034in}{1.310912in}}%
\pgfpathlineto{\pgfqpoint{2.331727in}{1.313459in}}%
\pgfpathlineto{\pgfqpoint{2.332115in}{1.311234in}}%
\pgfpathlineto{\pgfqpoint{2.332517in}{1.309536in}}%
\pgfpathlineto{\pgfqpoint{2.332926in}{1.310929in}}%
\pgfpathlineto{\pgfqpoint{2.332926in}{1.310929in}}%
\pgfpathlineto{\pgfqpoint{2.333621in}{1.313425in}}%
\pgfpathlineto{\pgfqpoint{2.334025in}{1.310978in}}%
\pgfpathlineto{\pgfqpoint{2.334429in}{1.309511in}}%
\pgfpathlineto{\pgfqpoint{2.334839in}{1.311036in}}%
\pgfpathlineto{\pgfqpoint{2.334839in}{1.311036in}}%
\pgfpathlineto{\pgfqpoint{2.335398in}{1.313424in}}%
\pgfpathlineto{\pgfqpoint{2.335901in}{1.311080in}}%
\pgfpathlineto{\pgfqpoint{2.336305in}{1.309462in}}%
\pgfpathlineto{\pgfqpoint{2.336716in}{1.310917in}}%
\pgfpathlineto{\pgfqpoint{2.336716in}{1.310917in}}%
\pgfpathlineto{\pgfqpoint{2.337415in}{1.313365in}}%
\pgfpathlineto{\pgfqpoint{2.337800in}{1.311013in}}%
\pgfpathlineto{\pgfqpoint{2.338206in}{1.309424in}}%
\pgfpathlineto{\pgfqpoint{2.338617in}{1.310905in}}%
\pgfpathlineto{\pgfqpoint{2.338617in}{1.310905in}}%
\pgfpathlineto{\pgfqpoint{2.339180in}{1.313368in}}%
\pgfpathlineto{\pgfqpoint{2.339687in}{1.311100in}}%
\pgfpathlineto{\pgfqpoint{2.340094in}{1.309378in}}%
\pgfpathlineto{\pgfqpoint{2.340506in}{1.310798in}}%
\pgfpathlineto{\pgfqpoint{2.340506in}{1.310798in}}%
\pgfpathlineto{\pgfqpoint{2.341209in}{1.313333in}}%
\pgfpathlineto{\pgfqpoint{2.341588in}{1.311112in}}%
\pgfpathlineto{\pgfqpoint{2.341985in}{1.309334in}}%
\pgfpathlineto{\pgfqpoint{2.342399in}{1.310681in}}%
\pgfpathlineto{\pgfqpoint{2.342399in}{1.310681in}}%
\pgfpathlineto{\pgfqpoint{2.343103in}{1.313329in}}%
\pgfpathlineto{\pgfqpoint{2.343534in}{1.310753in}}%
\pgfpathlineto{\pgfqpoint{2.343943in}{1.309318in}}%
\pgfpathlineto{\pgfqpoint{2.344219in}{1.310174in}}%
\pgfpathlineto{\pgfqpoint{2.344219in}{1.310174in}}%
\pgfpathlineto{\pgfqpoint{2.344924in}{1.313316in}}%
\pgfpathlineto{\pgfqpoint{2.345430in}{1.310888in}}%
\pgfpathlineto{\pgfqpoint{2.345840in}{1.309262in}}%
\pgfpathlineto{\pgfqpoint{2.346256in}{1.310765in}}%
\pgfpathlineto{\pgfqpoint{2.346256in}{1.310765in}}%
\pgfpathlineto{\pgfqpoint{2.346825in}{1.313275in}}%
\pgfpathlineto{\pgfqpoint{2.347360in}{1.310770in}}%
\pgfpathlineto{\pgfqpoint{2.347771in}{1.309228in}}%
\pgfpathlineto{\pgfqpoint{2.348189in}{1.310785in}}%
\pgfpathlineto{\pgfqpoint{2.348189in}{1.310785in}}%
\pgfpathlineto{\pgfqpoint{2.348759in}{1.313262in}}%
\pgfpathlineto{\pgfqpoint{2.349292in}{1.310680in}}%
\pgfpathlineto{\pgfqpoint{2.349704in}{1.309193in}}%
\pgfpathlineto{\pgfqpoint{2.350123in}{1.310788in}}%
\pgfpathlineto{\pgfqpoint{2.350123in}{1.310788in}}%
\pgfpathlineto{\pgfqpoint{2.350694in}{1.313244in}}%
\pgfpathlineto{\pgfqpoint{2.351193in}{1.310914in}}%
\pgfpathlineto{\pgfqpoint{2.351606in}{1.309133in}}%
\pgfpathlineto{\pgfqpoint{2.352026in}{1.310585in}}%
\pgfpathlineto{\pgfqpoint{2.352026in}{1.310585in}}%
\pgfpathlineto{\pgfqpoint{2.352739in}{1.313196in}}%
\pgfpathlineto{\pgfqpoint{2.353111in}{1.311051in}}%
\pgfpathlineto{\pgfqpoint{2.353583in}{1.309120in}}%
\pgfpathlineto{\pgfqpoint{2.354004in}{1.310775in}}%
\pgfpathlineto{\pgfqpoint{2.354004in}{1.310775in}}%
\pgfpathlineto{\pgfqpoint{2.354577in}{1.313205in}}%
\pgfpathlineto{\pgfqpoint{2.355050in}{1.311036in}}%
\pgfpathlineto{\pgfqpoint{2.355514in}{1.309068in}}%
\pgfpathlineto{\pgfqpoint{2.355936in}{1.310680in}}%
\pgfpathlineto{\pgfqpoint{2.355936in}{1.310680in}}%
\pgfpathlineto{\pgfqpoint{2.356512in}{1.313174in}}%
\pgfpathlineto{\pgfqpoint{2.357042in}{1.310578in}}%
\pgfpathlineto{\pgfqpoint{2.357458in}{1.309023in}}%
\pgfpathlineto{\pgfqpoint{2.357881in}{1.310627in}}%
\pgfpathlineto{\pgfqpoint{2.357881in}{1.310627in}}%
\pgfpathlineto{\pgfqpoint{2.358459in}{1.313148in}}%
\pgfpathlineto{\pgfqpoint{2.358963in}{1.310819in}}%
\pgfpathlineto{\pgfqpoint{2.359367in}{1.308964in}}%
\pgfpathlineto{\pgfqpoint{2.359791in}{1.310345in}}%
\pgfpathlineto{\pgfqpoint{2.359791in}{1.310345in}}%
\pgfpathlineto{\pgfqpoint{2.360512in}{1.313125in}}%
\pgfpathlineto{\pgfqpoint{2.360944in}{1.310556in}}%
\pgfpathlineto{\pgfqpoint{2.361363in}{1.308935in}}%
\pgfpathlineto{\pgfqpoint{2.361788in}{1.310525in}}%
\pgfpathlineto{\pgfqpoint{2.361788in}{1.310525in}}%
\pgfpathlineto{\pgfqpoint{2.362369in}{1.313096in}}%
\pgfpathlineto{\pgfqpoint{2.362866in}{1.310908in}}%
\pgfpathlineto{\pgfqpoint{2.363332in}{1.308897in}}%
\pgfpathlineto{\pgfqpoint{2.363759in}{1.310524in}}%
\pgfpathlineto{\pgfqpoint{2.363759in}{1.310524in}}%
\pgfpathlineto{\pgfqpoint{2.364341in}{1.313078in}}%
\pgfpathlineto{\pgfqpoint{2.364833in}{1.310872in}}%
\pgfpathlineto{\pgfqpoint{2.365302in}{1.308856in}}%
\pgfpathlineto{\pgfqpoint{2.365730in}{1.310498in}}%
\pgfpathlineto{\pgfqpoint{2.365730in}{1.310498in}}%
\pgfpathlineto{\pgfqpoint{2.366313in}{1.313056in}}%
\pgfpathlineto{\pgfqpoint{2.366842in}{1.310493in}}%
\pgfpathlineto{\pgfqpoint{2.367264in}{1.308804in}}%
\pgfpathlineto{\pgfqpoint{2.367693in}{1.310392in}}%
\pgfpathlineto{\pgfqpoint{2.367693in}{1.310392in}}%
\pgfpathlineto{\pgfqpoint{2.368279in}{1.313021in}}%
\pgfpathlineto{\pgfqpoint{2.368783in}{1.310815in}}%
\pgfpathlineto{\pgfqpoint{2.369250in}{1.308765in}}%
\pgfpathlineto{\pgfqpoint{2.369680in}{1.310393in}}%
\pgfpathlineto{\pgfqpoint{2.369680in}{1.310393in}}%
\pgfpathlineto{\pgfqpoint{2.370267in}{1.313003in}}%
\pgfpathlineto{\pgfqpoint{2.370786in}{1.310588in}}%
\pgfpathlineto{\pgfqpoint{2.371210in}{1.308709in}}%
\pgfpathlineto{\pgfqpoint{2.371641in}{1.310216in}}%
\pgfpathlineto{\pgfqpoint{2.371641in}{1.310216in}}%
\pgfpathlineto{\pgfqpoint{2.372374in}{1.312964in}}%
\pgfpathlineto{\pgfqpoint{2.372757in}{1.310730in}}%
\pgfpathlineto{\pgfqpoint{2.373235in}{1.308687in}}%
\pgfpathlineto{\pgfqpoint{2.373667in}{1.310381in}}%
\pgfpathlineto{\pgfqpoint{2.373667in}{1.310381in}}%
\pgfpathlineto{\pgfqpoint{2.374257in}{1.312965in}}%
\pgfpathlineto{\pgfqpoint{2.374750in}{1.310705in}}%
\pgfpathlineto{\pgfqpoint{2.375225in}{1.308638in}}%
\pgfpathlineto{\pgfqpoint{2.375658in}{1.310310in}}%
\pgfpathlineto{\pgfqpoint{2.375658in}{1.310310in}}%
\pgfpathlineto{\pgfqpoint{2.376250in}{1.312936in}}%
\pgfpathlineto{\pgfqpoint{2.376751in}{1.310666in}}%
\pgfpathlineto{\pgfqpoint{2.377229in}{1.308597in}}%
\pgfpathlineto{\pgfqpoint{2.377664in}{1.310291in}}%
\pgfpathlineto{\pgfqpoint{2.377664in}{1.310291in}}%
\pgfpathlineto{\pgfqpoint{2.378257in}{1.312915in}}%
\pgfpathlineto{\pgfqpoint{2.378783in}{1.310367in}}%
\pgfpathlineto{\pgfqpoint{2.379213in}{1.308537in}}%
\pgfpathlineto{\pgfqpoint{2.379649in}{1.310120in}}%
\pgfpathlineto{\pgfqpoint{2.379649in}{1.310120in}}%
\pgfpathlineto{\pgfqpoint{2.380390in}{1.312855in}}%
\pgfpathlineto{\pgfqpoint{2.380765in}{1.310623in}}%
\pgfpathlineto{\pgfqpoint{2.381233in}{1.308496in}}%
\pgfpathlineto{\pgfqpoint{2.381670in}{1.310127in}}%
\pgfpathlineto{\pgfqpoint{2.381670in}{1.310127in}}%
\pgfpathlineto{\pgfqpoint{2.382268in}{1.312851in}}%
\pgfpathlineto{\pgfqpoint{2.382809in}{1.310321in}}%
\pgfpathlineto{\pgfqpoint{2.383241in}{1.308447in}}%
\pgfpathlineto{\pgfqpoint{2.383679in}{1.310030in}}%
\pgfpathlineto{\pgfqpoint{2.383679in}{1.310030in}}%
\pgfpathlineto{\pgfqpoint{2.384425in}{1.312811in}}%
\pgfpathlineto{\pgfqpoint{2.384819in}{1.310405in}}%
\pgfpathlineto{\pgfqpoint{2.385219in}{1.308402in}}%
\pgfpathlineto{\pgfqpoint{2.385658in}{1.309732in}}%
\pgfpathlineto{\pgfqpoint{2.385658in}{1.309732in}}%
\pgfpathlineto{\pgfqpoint{2.386405in}{1.312834in}}%
\pgfpathlineto{\pgfqpoint{2.386861in}{1.310236in}}%
\pgfpathlineto{\pgfqpoint{2.387296in}{1.308357in}}%
\pgfpathlineto{\pgfqpoint{2.387736in}{1.309963in}}%
\pgfpathlineto{\pgfqpoint{2.387736in}{1.309963in}}%
\pgfpathlineto{\pgfqpoint{2.388486in}{1.312760in}}%
\pgfpathlineto{\pgfqpoint{2.388878in}{1.310366in}}%
\pgfpathlineto{\pgfqpoint{2.389264in}{1.308326in}}%
\pgfpathlineto{\pgfqpoint{2.389854in}{1.310398in}}%
\pgfpathlineto{\pgfqpoint{2.390456in}{1.312792in}}%
\pgfpathlineto{\pgfqpoint{2.390908in}{1.310432in}}%
\pgfpathlineto{\pgfqpoint{2.391391in}{1.308277in}}%
\pgfpathlineto{\pgfqpoint{2.391834in}{1.309997in}}%
\pgfpathlineto{\pgfqpoint{2.392440in}{1.312740in}}%
\pgfpathlineto{\pgfqpoint{2.392965in}{1.310290in}}%
\pgfpathlineto{\pgfqpoint{2.393355in}{1.308232in}}%
\pgfpathlineto{\pgfqpoint{2.393948in}{1.310345in}}%
\pgfpathlineto{\pgfqpoint{2.394554in}{1.312743in}}%
\pgfpathlineto{\pgfqpoint{2.395002in}{1.310399in}}%
\pgfpathlineto{\pgfqpoint{2.395002in}{1.310399in}}%
\pgfpathlineto{\pgfqpoint{2.395469in}{1.308172in}}%
\pgfpathlineto{\pgfqpoint{2.395915in}{1.309791in}}%
\pgfpathlineto{\pgfqpoint{2.395915in}{1.309791in}}%
\pgfpathlineto{\pgfqpoint{2.396673in}{1.312667in}}%
\pgfpathlineto{\pgfqpoint{2.397069in}{1.310267in}}%
\pgfpathlineto{\pgfqpoint{2.397584in}{1.308177in}}%
\pgfpathlineto{\pgfqpoint{2.398032in}{1.310090in}}%
\pgfpathlineto{\pgfqpoint{2.398642in}{1.312691in}}%
\pgfpathlineto{\pgfqpoint{2.399123in}{1.310327in}}%
\pgfpathlineto{\pgfqpoint{2.399593in}{1.308079in}}%
\pgfpathlineto{\pgfqpoint{2.400041in}{1.309716in}}%
\pgfpathlineto{\pgfqpoint{2.400041in}{1.309716in}}%
\pgfpathlineto{\pgfqpoint{2.400804in}{1.312617in}}%
\pgfpathlineto{\pgfqpoint{2.401195in}{1.310263in}}%
\pgfpathlineto{\pgfqpoint{2.401680in}{1.308041in}}%
\pgfpathlineto{\pgfqpoint{2.402130in}{1.309772in}}%
\pgfpathlineto{\pgfqpoint{2.402745in}{1.312611in}}%
\pgfpathlineto{\pgfqpoint{2.403290in}{1.310037in}}%
\pgfpathlineto{\pgfqpoint{2.403720in}{1.307981in}}%
\pgfpathlineto{\pgfqpoint{2.404171in}{1.309516in}}%
\pgfpathlineto{\pgfqpoint{2.404171in}{1.309516in}}%
\pgfpathlineto{\pgfqpoint{2.404938in}{1.312594in}}%
\pgfpathlineto{\pgfqpoint{2.405364in}{1.310072in}}%
\pgfpathlineto{\pgfqpoint{2.405759in}{1.307952in}}%
\pgfpathlineto{\pgfqpoint{2.406363in}{1.310117in}}%
\pgfpathlineto{\pgfqpoint{2.406979in}{1.312596in}}%
\pgfpathlineto{\pgfqpoint{2.407440in}{1.310156in}}%
\pgfpathlineto{\pgfqpoint{2.407928in}{1.307898in}}%
\pgfpathlineto{\pgfqpoint{2.408382in}{1.309647in}}%
\pgfpathlineto{\pgfqpoint{2.409002in}{1.312535in}}%
\pgfpathlineto{\pgfqpoint{2.409541in}{1.310048in}}%
\pgfpathlineto{\pgfqpoint{2.410064in}{1.307892in}}%
\pgfpathlineto{\pgfqpoint{2.410520in}{1.309857in}}%
\pgfpathlineto{\pgfqpoint{2.411141in}{1.312542in}}%
\pgfpathlineto{\pgfqpoint{2.411636in}{1.310062in}}%
\pgfpathlineto{\pgfqpoint{2.412136in}{1.307810in}}%
\pgfpathlineto{\pgfqpoint{2.412593in}{1.309635in}}%
\pgfpathlineto{\pgfqpoint{2.413216in}{1.312496in}}%
\pgfpathlineto{\pgfqpoint{2.413739in}{1.310056in}}%
\pgfpathlineto{\pgfqpoint{2.414226in}{1.307749in}}%
\pgfpathlineto{\pgfqpoint{2.414684in}{1.309489in}}%
\pgfpathlineto{\pgfqpoint{2.415463in}{1.312427in}}%
\pgfpathlineto{\pgfqpoint{2.415850in}{1.310037in}}%
\pgfpathlineto{\pgfqpoint{2.416330in}{1.307696in}}%
\pgfpathlineto{\pgfqpoint{2.416789in}{1.309387in}}%
\pgfpathlineto{\pgfqpoint{2.417570in}{1.312419in}}%
\pgfpathlineto{\pgfqpoint{2.417970in}{1.309994in}}%
\pgfpathlineto{\pgfqpoint{2.418454in}{1.307648in}}%
\pgfpathlineto{\pgfqpoint{2.418914in}{1.309367in}}%
\pgfpathlineto{\pgfqpoint{2.419698in}{1.312388in}}%
\pgfpathlineto{\pgfqpoint{2.420096in}{1.309952in}}%
\pgfpathlineto{\pgfqpoint{2.420583in}{1.307600in}}%
\pgfpathlineto{\pgfqpoint{2.421045in}{1.309340in}}%
\pgfpathlineto{\pgfqpoint{2.421831in}{1.312359in}}%
\pgfpathlineto{\pgfqpoint{2.422225in}{1.309945in}}%
\pgfpathlineto{\pgfqpoint{2.422655in}{1.307557in}}%
\pgfpathlineto{\pgfqpoint{2.423273in}{1.309868in}}%
\pgfpathlineto{\pgfqpoint{2.423904in}{1.312394in}}%
\pgfpathlineto{\pgfqpoint{2.424370in}{1.309834in}}%
\pgfpathlineto{\pgfqpoint{2.424880in}{1.307516in}}%
\pgfpathlineto{\pgfqpoint{2.425345in}{1.309403in}}%
\pgfpathlineto{\pgfqpoint{2.425979in}{1.312346in}}%
\pgfpathlineto{\pgfqpoint{2.426514in}{1.309805in}}%
\pgfpathlineto{\pgfqpoint{2.427021in}{1.307462in}}%
\pgfpathlineto{\pgfqpoint{2.427488in}{1.309332in}}%
\pgfpathlineto{\pgfqpoint{2.428124in}{1.312315in}}%
\pgfpathlineto{\pgfqpoint{2.428669in}{1.309728in}}%
\pgfpathlineto{\pgfqpoint{2.429195in}{1.307433in}}%
\pgfpathlineto{\pgfqpoint{2.429664in}{1.309424in}}%
\pgfpathlineto{\pgfqpoint{2.430301in}{1.312308in}}%
\pgfpathlineto{\pgfqpoint{2.430820in}{1.309764in}}%
\pgfpathlineto{\pgfqpoint{2.431313in}{1.307347in}}%
\pgfpathlineto{\pgfqpoint{2.431782in}{1.309126in}}%
\pgfpathlineto{\pgfqpoint{2.432580in}{1.312233in}}%
\pgfpathlineto{\pgfqpoint{2.432988in}{1.309681in}}%
\pgfpathlineto{\pgfqpoint{2.433504in}{1.307314in}}%
\pgfpathlineto{\pgfqpoint{2.433975in}{1.309237in}}%
\pgfpathlineto{\pgfqpoint{2.434617in}{1.312243in}}%
\pgfpathlineto{\pgfqpoint{2.435159in}{1.309648in}}%
\pgfpathlineto{\pgfqpoint{2.435674in}{1.307259in}}%
\pgfpathlineto{\pgfqpoint{2.436146in}{1.309172in}}%
\pgfpathlineto{\pgfqpoint{2.436790in}{1.312213in}}%
\pgfpathlineto{\pgfqpoint{2.437338in}{1.309601in}}%
\pgfpathlineto{\pgfqpoint{2.437857in}{1.307211in}}%
\pgfpathlineto{\pgfqpoint{2.438331in}{1.309157in}}%
\pgfpathlineto{\pgfqpoint{2.438977in}{1.312191in}}%
\pgfpathlineto{\pgfqpoint{2.439516in}{1.309640in}}%
\pgfpathlineto{\pgfqpoint{2.439947in}{1.307157in}}%
\pgfpathlineto{\pgfqpoint{2.440580in}{1.309495in}}%
\pgfpathlineto{\pgfqpoint{2.441228in}{1.312191in}}%
\pgfpathlineto{\pgfqpoint{2.441710in}{1.309576in}}%
\pgfpathlineto{\pgfqpoint{2.442208in}{1.307087in}}%
\pgfpathlineto{\pgfqpoint{2.442684in}{1.308890in}}%
\pgfpathlineto{\pgfqpoint{2.443495in}{1.312109in}}%
\pgfpathlineto{\pgfqpoint{2.443907in}{1.309555in}}%
\pgfpathlineto{\pgfqpoint{2.444362in}{1.307037in}}%
\pgfpathlineto{\pgfqpoint{2.445000in}{1.309557in}}%
\pgfpathlineto{\pgfqpoint{2.445651in}{1.312132in}}%
\pgfpathlineto{\pgfqpoint{2.446109in}{1.309547in}}%
\pgfpathlineto{\pgfqpoint{2.446621in}{1.306984in}}%
\pgfpathlineto{\pgfqpoint{2.447100in}{1.308841in}}%
\pgfpathlineto{\pgfqpoint{2.447916in}{1.312046in}}%
\pgfpathlineto{\pgfqpoint{2.448322in}{1.309506in}}%
\pgfpathlineto{\pgfqpoint{2.448840in}{1.306934in}}%
\pgfpathlineto{\pgfqpoint{2.449321in}{1.308829in}}%
\pgfpathlineto{\pgfqpoint{2.449979in}{1.312040in}}%
\pgfpathlineto{\pgfqpoint{2.450542in}{1.309460in}}%
\pgfpathlineto{\pgfqpoint{2.451078in}{1.306893in}}%
\pgfpathlineto{\pgfqpoint{2.451561in}{1.308890in}}%
\pgfpathlineto{\pgfqpoint{2.452220in}{1.312033in}}%
\pgfpathlineto{\pgfqpoint{2.452768in}{1.309424in}}%
\pgfpathlineto{\pgfqpoint{2.453296in}{1.306830in}}%
\pgfpathlineto{\pgfqpoint{2.453780in}{1.308778in}}%
\pgfpathlineto{\pgfqpoint{2.454442in}{1.311995in}}%
\pgfpathlineto{\pgfqpoint{2.455001in}{1.309395in}}%
\pgfpathlineto{\pgfqpoint{2.455421in}{1.306812in}}%
\pgfpathlineto{\pgfqpoint{2.456068in}{1.309056in}}%
\pgfpathlineto{\pgfqpoint{2.456731in}{1.312007in}}%
\pgfpathlineto{\pgfqpoint{2.457242in}{1.309360in}}%
\pgfpathlineto{\pgfqpoint{2.457822in}{1.306775in}}%
\pgfpathlineto{\pgfqpoint{2.458310in}{1.308997in}}%
\pgfpathlineto{\pgfqpoint{2.458975in}{1.311979in}}%
\pgfpathlineto{\pgfqpoint{2.459487in}{1.309349in}}%
\pgfpathlineto{\pgfqpoint{2.460019in}{1.306663in}}%
\pgfpathlineto{\pgfqpoint{2.460507in}{1.308613in}}%
\pgfpathlineto{\pgfqpoint{2.461176in}{1.311909in}}%
\pgfpathlineto{\pgfqpoint{2.461728in}{1.309470in}}%
\pgfpathlineto{\pgfqpoint{2.462279in}{1.306609in}}%
\pgfpathlineto{\pgfqpoint{2.462769in}{1.308586in}}%
\pgfpathlineto{\pgfqpoint{2.463440in}{1.311885in}}%
\pgfpathlineto{\pgfqpoint{2.464003in}{1.309302in}}%
\pgfpathlineto{\pgfqpoint{2.464490in}{1.306544in}}%
\pgfpathlineto{\pgfqpoint{2.464980in}{1.308205in}}%
\pgfpathlineto{\pgfqpoint{2.465817in}{1.311890in}}%
\pgfpathlineto{\pgfqpoint{2.466272in}{1.309269in}}%
\pgfpathlineto{\pgfqpoint{2.466822in}{1.306502in}}%
\pgfpathlineto{\pgfqpoint{2.467316in}{1.308538in}}%
\pgfpathlineto{\pgfqpoint{2.467991in}{1.311839in}}%
\pgfpathlineto{\pgfqpoint{2.468548in}{1.309249in}}%
\pgfpathlineto{\pgfqpoint{2.469103in}{1.306447in}}%
\pgfpathlineto{\pgfqpoint{2.469599in}{1.308501in}}%
\pgfpathlineto{\pgfqpoint{2.470275in}{1.311814in}}%
\pgfpathlineto{\pgfqpoint{2.470820in}{1.309357in}}%
\pgfpathlineto{\pgfqpoint{2.471382in}{1.306384in}}%
\pgfpathlineto{\pgfqpoint{2.471879in}{1.308399in}}%
\pgfpathlineto{\pgfqpoint{2.472558in}{1.311777in}}%
\pgfpathlineto{\pgfqpoint{2.473119in}{1.309247in}}%
\pgfpathlineto{\pgfqpoint{2.473721in}{1.306373in}}%
\pgfpathlineto{\pgfqpoint{2.474220in}{1.308641in}}%
\pgfpathlineto{\pgfqpoint{2.474900in}{1.311790in}}%
\pgfpathlineto{\pgfqpoint{2.475414in}{1.309274in}}%
\pgfpathlineto{\pgfqpoint{2.475972in}{1.306265in}}%
\pgfpathlineto{\pgfqpoint{2.476472in}{1.308261in}}%
\pgfpathlineto{\pgfqpoint{2.477323in}{1.311696in}}%
\pgfpathlineto{\pgfqpoint{2.477721in}{1.309243in}}%
\pgfpathlineto{\pgfqpoint{2.478281in}{1.306207in}}%
\pgfpathlineto{\pgfqpoint{2.478782in}{1.308209in}}%
\pgfpathlineto{\pgfqpoint{2.479636in}{1.311671in}}%
\pgfpathlineto{\pgfqpoint{2.480036in}{1.309213in}}%
\pgfpathlineto{\pgfqpoint{2.480656in}{1.306207in}}%
\pgfpathlineto{\pgfqpoint{2.481161in}{1.308537in}}%
\pgfpathlineto{\pgfqpoint{2.481848in}{1.311711in}}%
\pgfpathlineto{\pgfqpoint{2.482366in}{1.309101in}}%
\pgfpathlineto{\pgfqpoint{2.482836in}{1.306117in}}%
\pgfpathlineto{\pgfqpoint{2.483510in}{1.308630in}}%
\pgfpathlineto{\pgfqpoint{2.484199in}{1.311689in}}%
\pgfpathlineto{\pgfqpoint{2.484536in}{1.310538in}}%
\pgfpathlineto{\pgfqpoint{2.485264in}{1.306034in}}%
\pgfpathlineto{\pgfqpoint{2.485771in}{1.308106in}}%
\pgfpathlineto{\pgfqpoint{2.486465in}{1.311610in}}%
\pgfpathlineto{\pgfqpoint{2.487022in}{1.309186in}}%
\pgfpathlineto{\pgfqpoint{2.487572in}{1.305964in}}%
\pgfpathlineto{\pgfqpoint{2.488080in}{1.307839in}}%
\pgfpathlineto{\pgfqpoint{2.488945in}{1.311602in}}%
\pgfpathlineto{\pgfqpoint{2.489335in}{1.309508in}}%
\pgfpathlineto{\pgfqpoint{2.490000in}{1.305960in}}%
\pgfpathlineto{\pgfqpoint{2.490511in}{1.308302in}}%
\pgfpathlineto{\pgfqpoint{2.491208in}{1.311598in}}%
\pgfpathlineto{\pgfqpoint{2.491733in}{1.309022in}}%
\pgfpathlineto{\pgfqpoint{2.492314in}{1.305855in}}%
\pgfpathlineto{\pgfqpoint{2.492826in}{1.307967in}}%
\pgfpathlineto{\pgfqpoint{2.493527in}{1.311529in}}%
\pgfpathlineto{\pgfqpoint{2.494090in}{1.309054in}}%
\pgfpathlineto{\pgfqpoint{2.494691in}{1.305802in}}%
\pgfpathlineto{\pgfqpoint{2.495205in}{1.307987in}}%
\pgfpathlineto{\pgfqpoint{2.495907in}{1.311514in}}%
\pgfpathlineto{\pgfqpoint{2.496421in}{1.309436in}}%
\pgfpathlineto{\pgfqpoint{2.497051in}{1.305732in}}%
\pgfpathlineto{\pgfqpoint{2.497566in}{1.307848in}}%
\pgfpathlineto{\pgfqpoint{2.498444in}{1.311443in}}%
\pgfpathlineto{\pgfqpoint{2.498839in}{1.309008in}}%
\pgfpathlineto{\pgfqpoint{2.499331in}{1.305708in}}%
\pgfpathlineto{\pgfqpoint{2.500021in}{1.308255in}}%
\pgfpathlineto{\pgfqpoint{2.500726in}{1.311497in}}%
\pgfpathlineto{\pgfqpoint{2.501072in}{1.310361in}}%
\pgfpathlineto{\pgfqpoint{2.501805in}{1.305601in}}%
\pgfpathlineto{\pgfqpoint{2.502324in}{1.307627in}}%
\pgfpathlineto{\pgfqpoint{2.503208in}{1.311418in}}%
\pgfpathlineto{\pgfqpoint{2.503606in}{1.309126in}}%
\pgfpathlineto{\pgfqpoint{2.504221in}{1.305545in}}%
\pgfpathlineto{\pgfqpoint{2.504742in}{1.307687in}}%
\pgfpathlineto{\pgfqpoint{2.505629in}{1.311362in}}%
\pgfpathlineto{\pgfqpoint{2.506029in}{1.308901in}}%
\pgfpathlineto{\pgfqpoint{2.506632in}{1.305484in}}%
\pgfpathlineto{\pgfqpoint{2.507155in}{1.307656in}}%
\pgfpathlineto{\pgfqpoint{2.507870in}{1.311358in}}%
\pgfpathlineto{\pgfqpoint{2.508446in}{1.308827in}}%
\pgfpathlineto{\pgfqpoint{2.509042in}{1.305417in}}%
\pgfpathlineto{\pgfqpoint{2.509566in}{1.307564in}}%
\pgfpathlineto{\pgfqpoint{2.510460in}{1.311312in}}%
\pgfpathlineto{\pgfqpoint{2.510862in}{1.308853in}}%
\pgfpathlineto{\pgfqpoint{2.511481in}{1.305363in}}%
\pgfpathlineto{\pgfqpoint{2.512008in}{1.307609in}}%
\pgfpathlineto{\pgfqpoint{2.512728in}{1.311313in}}%
\pgfpathlineto{\pgfqpoint{2.513255in}{1.309205in}}%
\pgfpathlineto{\pgfqpoint{2.513919in}{1.305303in}}%
\pgfpathlineto{\pgfqpoint{2.514448in}{1.307589in}}%
\pgfpathlineto{\pgfqpoint{2.515171in}{1.311290in}}%
\pgfpathlineto{\pgfqpoint{2.515700in}{1.309120in}}%
\pgfpathlineto{\pgfqpoint{2.516334in}{1.305222in}}%
\pgfpathlineto{\pgfqpoint{2.516865in}{1.307358in}}%
\pgfpathlineto{\pgfqpoint{2.517768in}{1.311241in}}%
\pgfpathlineto{\pgfqpoint{2.518174in}{1.308813in}}%
\pgfpathlineto{\pgfqpoint{2.518722in}{1.305172in}}%
\pgfpathlineto{\pgfqpoint{2.519253in}{1.306894in}}%
\pgfpathlineto{\pgfqpoint{2.520158in}{1.311273in}}%
\pgfpathlineto{\pgfqpoint{2.520620in}{1.308900in}}%
\pgfpathlineto{\pgfqpoint{2.521175in}{1.305113in}}%
\pgfpathlineto{\pgfqpoint{2.521708in}{1.306799in}}%
\pgfpathlineto{\pgfqpoint{2.522617in}{1.311247in}}%
\pgfpathlineto{\pgfqpoint{2.523080in}{1.308925in}}%
\pgfpathlineto{\pgfqpoint{2.523775in}{1.305079in}}%
\pgfpathlineto{\pgfqpoint{2.524312in}{1.307608in}}%
\pgfpathlineto{\pgfqpoint{2.525044in}{1.311207in}}%
\pgfpathlineto{\pgfqpoint{2.525580in}{1.308619in}}%
\pgfpathlineto{\pgfqpoint{2.526215in}{1.304972in}}%
\pgfpathlineto{\pgfqpoint{2.526754in}{1.307298in}}%
\pgfpathlineto{\pgfqpoint{2.527490in}{1.311142in}}%
\pgfpathlineto{\pgfqpoint{2.528029in}{1.308961in}}%
\pgfpathlineto{\pgfqpoint{2.528616in}{1.304908in}}%
\pgfpathlineto{\pgfqpoint{2.529154in}{1.306675in}}%
\pgfpathlineto{\pgfqpoint{2.530073in}{1.311160in}}%
\pgfpathlineto{\pgfqpoint{2.530542in}{1.308724in}}%
\pgfpathlineto{\pgfqpoint{2.531148in}{1.304823in}}%
\pgfpathlineto{\pgfqpoint{2.531689in}{1.306852in}}%
\pgfpathlineto{\pgfqpoint{2.532612in}{1.311108in}}%
\pgfpathlineto{\pgfqpoint{2.533028in}{1.308859in}}%
\pgfpathlineto{\pgfqpoint{2.533687in}{1.304760in}}%
\pgfpathlineto{\pgfqpoint{2.534231in}{1.307016in}}%
\pgfpathlineto{\pgfqpoint{2.535157in}{1.311033in}}%
\pgfpathlineto{\pgfqpoint{2.535520in}{1.309017in}}%
\pgfpathlineto{\pgfqpoint{2.536162in}{1.304688in}}%
\pgfpathlineto{\pgfqpoint{2.536708in}{1.306701in}}%
\pgfpathlineto{\pgfqpoint{2.537637in}{1.311057in}}%
\pgfpathlineto{\pgfqpoint{2.538056in}{1.308838in}}%
\pgfpathlineto{\pgfqpoint{2.538665in}{1.304630in}}%
\pgfpathlineto{\pgfqpoint{2.539212in}{1.306503in}}%
\pgfpathlineto{\pgfqpoint{2.540145in}{1.311043in}}%
\pgfpathlineto{\pgfqpoint{2.540620in}{1.308456in}}%
\pgfpathlineto{\pgfqpoint{2.541275in}{1.304562in}}%
\pgfpathlineto{\pgfqpoint{2.541826in}{1.306954in}}%
\pgfpathlineto{\pgfqpoint{2.542579in}{1.310965in}}%
\pgfpathlineto{\pgfqpoint{2.543130in}{1.308742in}}%
\pgfpathlineto{\pgfqpoint{2.543827in}{1.304498in}}%
\pgfpathlineto{\pgfqpoint{2.544380in}{1.306954in}}%
\pgfpathlineto{\pgfqpoint{2.545136in}{1.310946in}}%
\pgfpathlineto{\pgfqpoint{2.545689in}{1.308620in}}%
\pgfpathlineto{\pgfqpoint{2.546363in}{1.304416in}}%
\pgfpathlineto{\pgfqpoint{2.546917in}{1.306775in}}%
\pgfpathlineto{\pgfqpoint{2.547862in}{1.310880in}}%
\pgfpathlineto{\pgfqpoint{2.548231in}{1.308760in}}%
\pgfpathlineto{\pgfqpoint{2.548908in}{1.304338in}}%
\pgfpathlineto{\pgfqpoint{2.549464in}{1.306593in}}%
\pgfpathlineto{\pgfqpoint{2.550412in}{1.310881in}}%
\pgfpathlineto{\pgfqpoint{2.550839in}{1.308338in}}%
\pgfpathlineto{\pgfqpoint{2.551519in}{1.304288in}}%
\pgfpathlineto{\pgfqpoint{2.552078in}{1.306799in}}%
\pgfpathlineto{\pgfqpoint{2.552843in}{1.310861in}}%
\pgfpathlineto{\pgfqpoint{2.553402in}{1.308473in}}%
\pgfpathlineto{\pgfqpoint{2.554008in}{1.304212in}}%
\pgfpathlineto{\pgfqpoint{2.554568in}{1.306110in}}%
\pgfpathlineto{\pgfqpoint{2.555523in}{1.310874in}}%
\pgfpathlineto{\pgfqpoint{2.555953in}{1.308826in}}%
\pgfpathlineto{\pgfqpoint{2.556669in}{1.304126in}}%
\pgfpathlineto{\pgfqpoint{2.557232in}{1.306503in}}%
\pgfpathlineto{\pgfqpoint{2.558191in}{1.310775in}}%
\pgfpathlineto{\pgfqpoint{2.558567in}{1.308655in}}%
\pgfpathlineto{\pgfqpoint{2.559163in}{1.304117in}}%
\pgfpathlineto{\pgfqpoint{2.559727in}{1.305725in}}%
\pgfpathlineto{\pgfqpoint{2.560689in}{1.310815in}}%
\pgfpathlineto{\pgfqpoint{2.561066in}{1.309568in}}%
\pgfpathlineto{\pgfqpoint{2.561776in}{1.304044in}}%
\pgfpathlineto{\pgfqpoint{2.562342in}{1.305663in}}%
\pgfpathlineto{\pgfqpoint{2.563308in}{1.310786in}}%
\pgfpathlineto{\pgfqpoint{2.563687in}{1.309531in}}%
\pgfpathlineto{\pgfqpoint{2.564503in}{1.303903in}}%
\pgfpathlineto{\pgfqpoint{2.565073in}{1.306282in}}%
\pgfpathlineto{\pgfqpoint{2.566043in}{1.310698in}}%
\pgfpathlineto{\pgfqpoint{2.566423in}{1.308594in}}%
\pgfpathlineto{\pgfqpoint{2.567109in}{1.303827in}}%
\pgfpathlineto{\pgfqpoint{2.567680in}{1.306027in}}%
\pgfpathlineto{\pgfqpoint{2.568654in}{1.310706in}}%
\pgfpathlineto{\pgfqpoint{2.569093in}{1.308272in}}%
\pgfpathlineto{\pgfqpoint{2.569838in}{1.303796in}}%
\pgfpathlineto{\pgfqpoint{2.570221in}{1.305289in}}%
\pgfpathlineto{\pgfqpoint{2.571198in}{1.310678in}}%
\pgfpathlineto{\pgfqpoint{2.571581in}{1.309634in}}%
\pgfpathlineto{\pgfqpoint{2.572431in}{1.303675in}}%
\pgfpathlineto{\pgfqpoint{2.573008in}{1.306060in}}%
\pgfpathlineto{\pgfqpoint{2.573990in}{1.310619in}}%
\pgfpathlineto{\pgfqpoint{2.574374in}{1.308528in}}%
\pgfpathlineto{\pgfqpoint{2.575112in}{1.303603in}}%
\pgfpathlineto{\pgfqpoint{2.575691in}{1.306101in}}%
\pgfpathlineto{\pgfqpoint{2.576677in}{1.310565in}}%
\pgfpathlineto{\pgfqpoint{2.577064in}{1.308335in}}%
\pgfpathlineto{\pgfqpoint{2.577689in}{1.303572in}}%
\pgfpathlineto{\pgfqpoint{2.578268in}{1.305363in}}%
\pgfpathlineto{\pgfqpoint{2.579258in}{1.310615in}}%
\pgfpathlineto{\pgfqpoint{2.579646in}{1.309203in}}%
\pgfpathlineto{\pgfqpoint{2.580554in}{1.303533in}}%
\pgfpathlineto{\pgfqpoint{2.580943in}{1.305201in}}%
\pgfpathlineto{\pgfqpoint{2.581937in}{1.310581in}}%
\pgfpathlineto{\pgfqpoint{2.582327in}{1.309279in}}%
\pgfpathlineto{\pgfqpoint{2.583185in}{1.303375in}}%
\pgfpathlineto{\pgfqpoint{2.583771in}{1.305979in}}%
\pgfpathlineto{\pgfqpoint{2.584769in}{1.310460in}}%
\pgfpathlineto{\pgfqpoint{2.585160in}{1.308111in}}%
\pgfpathlineto{\pgfqpoint{2.585931in}{1.303328in}}%
\pgfpathlineto{\pgfqpoint{2.586322in}{1.304859in}}%
\pgfpathlineto{\pgfqpoint{2.587324in}{1.310497in}}%
\pgfpathlineto{\pgfqpoint{2.587717in}{1.309445in}}%
\pgfpathlineto{\pgfqpoint{2.588610in}{1.303213in}}%
\pgfpathlineto{\pgfqpoint{2.589201in}{1.305813in}}%
\pgfpathlineto{\pgfqpoint{2.590208in}{1.310412in}}%
\pgfpathlineto{\pgfqpoint{2.590602in}{1.308083in}}%
\pgfpathlineto{\pgfqpoint{2.591239in}{1.303176in}}%
\pgfpathlineto{\pgfqpoint{2.591831in}{1.305041in}}%
\pgfpathlineto{\pgfqpoint{2.592841in}{1.310468in}}%
\pgfpathlineto{\pgfqpoint{2.593237in}{1.308995in}}%
\pgfpathlineto{\pgfqpoint{2.594115in}{1.303072in}}%
\pgfpathlineto{\pgfqpoint{2.594512in}{1.304566in}}%
\pgfpathlineto{\pgfqpoint{2.595527in}{1.310394in}}%
\pgfpathlineto{\pgfqpoint{2.595925in}{1.309417in}}%
\pgfpathlineto{\pgfqpoint{2.596848in}{1.302972in}}%
\pgfpathlineto{\pgfqpoint{2.597447in}{1.305663in}}%
\pgfpathlineto{\pgfqpoint{2.598466in}{1.310311in}}%
\pgfpathlineto{\pgfqpoint{2.598865in}{1.307887in}}%
\pgfpathlineto{\pgfqpoint{2.599576in}{1.302880in}}%
\pgfpathlineto{\pgfqpoint{2.600176in}{1.305319in}}%
\pgfpathlineto{\pgfqpoint{2.601199in}{1.310344in}}%
\pgfpathlineto{\pgfqpoint{2.601600in}{1.308247in}}%
\pgfpathlineto{\pgfqpoint{2.602327in}{1.302807in}}%
\pgfpathlineto{\pgfqpoint{2.602930in}{1.305059in}}%
\pgfpathlineto{\pgfqpoint{2.603957in}{1.310339in}}%
\pgfpathlineto{\pgfqpoint{2.604420in}{1.307886in}}%
\pgfpathlineto{\pgfqpoint{2.605258in}{1.302810in}}%
\pgfpathlineto{\pgfqpoint{2.605662in}{1.304587in}}%
\pgfpathlineto{\pgfqpoint{2.606694in}{1.310316in}}%
\pgfpathlineto{\pgfqpoint{2.607099in}{1.308926in}}%
\pgfpathlineto{\pgfqpoint{2.607966in}{1.302630in}}%
\pgfpathlineto{\pgfqpoint{2.608575in}{1.305248in}}%
\pgfpathlineto{\pgfqpoint{2.609612in}{1.310229in}}%
\pgfpathlineto{\pgfqpoint{2.610018in}{1.307930in}}%
\pgfpathlineto{\pgfqpoint{2.610794in}{1.302549in}}%
\pgfpathlineto{\pgfqpoint{2.611201in}{1.303930in}}%
\pgfpathlineto{\pgfqpoint{2.612447in}{1.310178in}}%
\pgfpathlineto{\pgfqpoint{2.612855in}{1.307749in}}%
\pgfpathlineto{\pgfqpoint{2.613685in}{1.302524in}}%
\pgfpathlineto{\pgfqpoint{2.614094in}{1.304246in}}%
\pgfpathlineto{\pgfqpoint{2.615139in}{1.310214in}}%
\pgfpathlineto{\pgfqpoint{2.615550in}{1.308955in}}%
\pgfpathlineto{\pgfqpoint{2.616474in}{1.302383in}}%
\pgfpathlineto{\pgfqpoint{2.616884in}{1.303853in}}%
\pgfpathlineto{\pgfqpoint{2.617936in}{1.310124in}}%
\pgfpathlineto{\pgfqpoint{2.618553in}{1.307515in}}%
\pgfpathlineto{\pgfqpoint{2.619328in}{1.302296in}}%
\pgfpathlineto{\pgfqpoint{2.619740in}{1.303784in}}%
\pgfpathlineto{\pgfqpoint{2.620796in}{1.310096in}}%
\pgfpathlineto{\pgfqpoint{2.621210in}{1.309209in}}%
\pgfpathlineto{\pgfqpoint{2.622189in}{1.302206in}}%
\pgfpathlineto{\pgfqpoint{2.622812in}{1.305083in}}%
\pgfpathlineto{\pgfqpoint{2.623872in}{1.310036in}}%
\pgfpathlineto{\pgfqpoint{2.624080in}{1.309194in}}%
\pgfpathlineto{\pgfqpoint{2.625014in}{1.302108in}}%
\pgfpathlineto{\pgfqpoint{2.625638in}{1.304630in}}%
\pgfpathlineto{\pgfqpoint{2.626703in}{1.310086in}}%
\pgfpathlineto{\pgfqpoint{2.627120in}{1.307945in}}%
\pgfpathlineto{\pgfqpoint{2.627872in}{1.302038in}}%
\pgfpathlineto{\pgfqpoint{2.628498in}{1.304327in}}%
\pgfpathlineto{\pgfqpoint{2.629567in}{1.310079in}}%
\pgfpathlineto{\pgfqpoint{2.630049in}{1.307608in}}%
\pgfpathlineto{\pgfqpoint{2.630816in}{1.301925in}}%
\pgfpathlineto{\pgfqpoint{2.631234in}{1.303181in}}%
\pgfpathlineto{\pgfqpoint{2.632519in}{1.310018in}}%
\pgfpathlineto{\pgfqpoint{2.632940in}{1.307776in}}%
\pgfpathlineto{\pgfqpoint{2.633771in}{1.301841in}}%
\pgfpathlineto{\pgfqpoint{2.634192in}{1.303330in}}%
\pgfpathlineto{\pgfqpoint{2.635484in}{1.309927in}}%
\pgfpathlineto{\pgfqpoint{2.635907in}{1.307313in}}%
\pgfpathlineto{\pgfqpoint{2.636779in}{1.301837in}}%
\pgfpathlineto{\pgfqpoint{2.637203in}{1.303743in}}%
\pgfpathlineto{\pgfqpoint{2.638287in}{1.309986in}}%
\pgfpathlineto{\pgfqpoint{2.638712in}{1.308517in}}%
\pgfpathlineto{\pgfqpoint{2.639651in}{1.301659in}}%
\pgfpathlineto{\pgfqpoint{2.640076in}{1.303207in}}%
\pgfpathlineto{\pgfqpoint{2.641380in}{1.309853in}}%
\pgfpathlineto{\pgfqpoint{2.641594in}{1.308973in}}%
\pgfpathlineto{\pgfqpoint{2.642514in}{1.301582in}}%
\pgfpathlineto{\pgfqpoint{2.643155in}{1.303908in}}%
\pgfpathlineto{\pgfqpoint{2.644249in}{1.309932in}}%
\pgfpathlineto{\pgfqpoint{2.644742in}{1.307432in}}%
\pgfpathlineto{\pgfqpoint{2.645488in}{1.301485in}}%
\pgfpathlineto{\pgfqpoint{2.646132in}{1.303855in}}%
\pgfpathlineto{\pgfqpoint{2.647230in}{1.309902in}}%
\pgfpathlineto{\pgfqpoint{2.647661in}{1.307971in}}%
\pgfpathlineto{\pgfqpoint{2.648553in}{1.301369in}}%
\pgfpathlineto{\pgfqpoint{2.648984in}{1.302892in}}%
\pgfpathlineto{\pgfqpoint{2.650306in}{1.309782in}}%
\pgfpathlineto{\pgfqpoint{2.650523in}{1.308929in}}%
\pgfpathlineto{\pgfqpoint{2.651553in}{1.301273in}}%
\pgfpathlineto{\pgfqpoint{2.651986in}{1.302818in}}%
\pgfpathlineto{\pgfqpoint{2.653315in}{1.309749in}}%
\pgfpathlineto{\pgfqpoint{2.653532in}{1.308882in}}%
\pgfpathlineto{\pgfqpoint{2.654597in}{1.301198in}}%
\pgfpathlineto{\pgfqpoint{2.655033in}{1.302926in}}%
\pgfpathlineto{\pgfqpoint{2.656148in}{1.309758in}}%
\pgfpathlineto{\pgfqpoint{2.656585in}{1.308642in}}%
\pgfpathlineto{\pgfqpoint{2.657548in}{1.301072in}}%
\pgfpathlineto{\pgfqpoint{2.658205in}{1.303822in}}%
\pgfpathlineto{\pgfqpoint{2.659324in}{1.309757in}}%
\pgfpathlineto{\pgfqpoint{2.659763in}{1.307420in}}%
\pgfpathlineto{\pgfqpoint{2.660567in}{1.300983in}}%
\pgfpathlineto{\pgfqpoint{2.661227in}{1.303584in}}%
\pgfpathlineto{\pgfqpoint{2.662352in}{1.309745in}}%
\pgfpathlineto{\pgfqpoint{2.662793in}{1.307588in}}%
\pgfpathlineto{\pgfqpoint{2.663581in}{1.300928in}}%
\pgfpathlineto{\pgfqpoint{2.664243in}{1.303198in}}%
\pgfpathlineto{\pgfqpoint{2.665374in}{1.309727in}}%
\pgfpathlineto{\pgfqpoint{2.665817in}{1.307926in}}%
\pgfpathlineto{\pgfqpoint{2.666746in}{1.300769in}}%
\pgfpathlineto{\pgfqpoint{2.667190in}{1.302284in}}%
\pgfpathlineto{\pgfqpoint{2.668550in}{1.309625in}}%
\pgfpathlineto{\pgfqpoint{2.668773in}{1.308795in}}%
\pgfpathlineto{\pgfqpoint{2.669827in}{1.300666in}}%
\pgfpathlineto{\pgfqpoint{2.670497in}{1.303701in}}%
\pgfpathlineto{\pgfqpoint{2.671639in}{1.309603in}}%
\pgfpathlineto{\pgfqpoint{2.671863in}{1.308788in}}%
\pgfpathlineto{\pgfqpoint{2.672985in}{1.300596in}}%
\pgfpathlineto{\pgfqpoint{2.673433in}{1.302429in}}%
\pgfpathlineto{\pgfqpoint{2.674581in}{1.309582in}}%
\pgfpathlineto{\pgfqpoint{2.675031in}{1.308383in}}%
\pgfpathlineto{\pgfqpoint{2.676065in}{1.300464in}}%
\pgfpathlineto{\pgfqpoint{2.676515in}{1.302108in}}%
\pgfpathlineto{\pgfqpoint{2.677896in}{1.309507in}}%
\pgfpathlineto{\pgfqpoint{2.678122in}{1.308585in}}%
\pgfpathlineto{\pgfqpoint{2.679289in}{1.300471in}}%
\pgfpathlineto{\pgfqpoint{2.679742in}{1.302632in}}%
\pgfpathlineto{\pgfqpoint{2.680901in}{1.309573in}}%
\pgfpathlineto{\pgfqpoint{2.681356in}{1.307878in}}%
\pgfpathlineto{\pgfqpoint{2.682336in}{1.300249in}}%
\pgfpathlineto{\pgfqpoint{2.682791in}{1.301871in}}%
\pgfpathlineto{\pgfqpoint{2.684185in}{1.309462in}}%
\pgfpathlineto{\pgfqpoint{2.684413in}{1.308563in}}%
\pgfpathlineto{\pgfqpoint{2.685476in}{1.300137in}}%
\pgfpathlineto{\pgfqpoint{2.685933in}{1.301633in}}%
\pgfpathlineto{\pgfqpoint{2.687334in}{1.309469in}}%
\pgfpathlineto{\pgfqpoint{2.687792in}{1.306810in}}%
\pgfpathlineto{\pgfqpoint{2.688677in}{1.300031in}}%
\pgfpathlineto{\pgfqpoint{2.689136in}{1.301672in}}%
\pgfpathlineto{\pgfqpoint{2.690545in}{1.309405in}}%
\pgfpathlineto{\pgfqpoint{2.690775in}{1.308498in}}%
\pgfpathlineto{\pgfqpoint{2.691790in}{1.299952in}}%
\pgfpathlineto{\pgfqpoint{2.692483in}{1.302592in}}%
\pgfpathlineto{\pgfqpoint{2.693665in}{1.309460in}}%
\pgfpathlineto{\pgfqpoint{2.694129in}{1.307319in}}%
\pgfpathlineto{\pgfqpoint{2.695138in}{1.299851in}}%
\pgfpathlineto{\pgfqpoint{2.695602in}{1.301830in}}%
\pgfpathlineto{\pgfqpoint{2.696791in}{1.309382in}}%
\pgfpathlineto{\pgfqpoint{2.697258in}{1.308059in}}%
\pgfpathlineto{\pgfqpoint{2.698205in}{1.299762in}}%
\pgfpathlineto{\pgfqpoint{2.698904in}{1.302229in}}%
\pgfpathlineto{\pgfqpoint{2.700099in}{1.309403in}}%
\pgfpathlineto{\pgfqpoint{2.700567in}{1.307452in}}%
\pgfpathlineto{\pgfqpoint{2.701575in}{1.299592in}}%
\pgfpathlineto{\pgfqpoint{2.702043in}{1.301388in}}%
\pgfpathlineto{\pgfqpoint{2.703482in}{1.309253in}}%
\pgfpathlineto{\pgfqpoint{2.703717in}{1.308238in}}%
\pgfpathlineto{\pgfqpoint{2.704706in}{1.299542in}}%
\pgfpathlineto{\pgfqpoint{2.705413in}{1.302008in}}%
\pgfpathlineto{\pgfqpoint{2.706620in}{1.309344in}}%
\pgfpathlineto{\pgfqpoint{2.707094in}{1.307405in}}%
\pgfpathlineto{\pgfqpoint{2.708060in}{1.299353in}}%
\pgfpathlineto{\pgfqpoint{2.708534in}{1.300836in}}%
\pgfpathlineto{\pgfqpoint{2.709986in}{1.309285in}}%
\pgfpathlineto{\pgfqpoint{2.710461in}{1.306622in}}%
\pgfpathlineto{\pgfqpoint{2.711281in}{1.299311in}}%
\pgfpathlineto{\pgfqpoint{2.711995in}{1.301818in}}%
\pgfpathlineto{\pgfqpoint{2.713215in}{1.309287in}}%
\pgfpathlineto{\pgfqpoint{2.713694in}{1.307315in}}%
\pgfpathlineto{\pgfqpoint{2.714728in}{1.299126in}}%
\pgfpathlineto{\pgfqpoint{2.715207in}{1.300991in}}%
\pgfpathlineto{\pgfqpoint{2.716676in}{1.309133in}}%
\pgfpathlineto{\pgfqpoint{2.716917in}{1.308077in}}%
\pgfpathlineto{\pgfqpoint{2.718104in}{1.299047in}}%
\pgfpathlineto{\pgfqpoint{2.718586in}{1.301183in}}%
\pgfpathlineto{\pgfqpoint{2.719819in}{1.309183in}}%
\pgfpathlineto{\pgfqpoint{2.720303in}{1.307734in}}%
\pgfpathlineto{\pgfqpoint{2.721376in}{1.298873in}}%
\pgfpathlineto{\pgfqpoint{2.721859in}{1.300554in}}%
\pgfpathlineto{\pgfqpoint{2.723343in}{1.309140in}}%
\pgfpathlineto{\pgfqpoint{2.723587in}{1.308233in}}%
\pgfpathlineto{\pgfqpoint{2.724649in}{1.298823in}}%
\pgfpathlineto{\pgfqpoint{2.725379in}{1.301469in}}%
\pgfpathlineto{\pgfqpoint{2.726625in}{1.309175in}}%
\pgfpathlineto{\pgfqpoint{2.727114in}{1.307082in}}%
\pgfpathlineto{\pgfqpoint{2.728204in}{1.298673in}}%
\pgfpathlineto{\pgfqpoint{2.728694in}{1.300830in}}%
\pgfpathlineto{\pgfqpoint{2.729948in}{1.309091in}}%
\pgfpathlineto{\pgfqpoint{2.730440in}{1.307653in}}%
\pgfpathlineto{\pgfqpoint{2.731566in}{1.298509in}}%
\pgfpathlineto{\pgfqpoint{2.732058in}{1.300419in}}%
\pgfpathlineto{\pgfqpoint{2.733567in}{1.309001in}}%
\pgfpathlineto{\pgfqpoint{2.733814in}{1.307928in}}%
\pgfpathlineto{\pgfqpoint{2.734915in}{1.298390in}}%
\pgfpathlineto{\pgfqpoint{2.735409in}{1.299831in}}%
\pgfpathlineto{\pgfqpoint{2.736925in}{1.309078in}}%
\pgfpathlineto{\pgfqpoint{2.737422in}{1.306441in}}%
\pgfpathlineto{\pgfqpoint{2.738405in}{1.298245in}}%
\pgfpathlineto{\pgfqpoint{2.738902in}{1.300007in}}%
\pgfpathlineto{\pgfqpoint{2.740428in}{1.308998in}}%
\pgfpathlineto{\pgfqpoint{2.740678in}{1.308047in}}%
\pgfpathlineto{\pgfqpoint{2.741977in}{1.298216in}}%
\pgfpathlineto{\pgfqpoint{2.742478in}{1.300655in}}%
\pgfpathlineto{\pgfqpoint{2.743761in}{1.309016in}}%
\pgfpathlineto{\pgfqpoint{2.744264in}{1.307209in}}%
\pgfpathlineto{\pgfqpoint{2.745460in}{1.298083in}}%
\pgfpathlineto{\pgfqpoint{2.745964in}{1.300539in}}%
\pgfpathlineto{\pgfqpoint{2.747253in}{1.308988in}}%
\pgfpathlineto{\pgfqpoint{2.747759in}{1.307175in}}%
\pgfpathlineto{\pgfqpoint{2.748891in}{1.297865in}}%
\pgfpathlineto{\pgfqpoint{2.749397in}{1.299916in}}%
\pgfpathlineto{\pgfqpoint{2.750950in}{1.308842in}}%
\pgfpathlineto{\pgfqpoint{2.751205in}{1.307680in}}%
\pgfpathlineto{\pgfqpoint{2.752503in}{1.297848in}}%
\pgfpathlineto{\pgfqpoint{2.753013in}{1.300443in}}%
\pgfpathlineto{\pgfqpoint{2.754318in}{1.308946in}}%
\pgfpathlineto{\pgfqpoint{2.754829in}{1.306943in}}%
\pgfpathlineto{\pgfqpoint{2.755868in}{1.297601in}}%
\pgfpathlineto{\pgfqpoint{2.756379in}{1.299097in}}%
\pgfpathlineto{\pgfqpoint{2.757948in}{1.308919in}}%
\pgfpathlineto{\pgfqpoint{2.758463in}{1.306169in}}%
\pgfpathlineto{\pgfqpoint{2.759545in}{1.297475in}}%
\pgfpathlineto{\pgfqpoint{2.760060in}{1.299726in}}%
\pgfpathlineto{\pgfqpoint{2.761381in}{1.308818in}}%
\pgfpathlineto{\pgfqpoint{2.761900in}{1.307397in}}%
\pgfpathlineto{\pgfqpoint{2.763163in}{1.297376in}}%
\pgfpathlineto{\pgfqpoint{2.763681in}{1.299843in}}%
\pgfpathlineto{\pgfqpoint{2.765008in}{1.308834in}}%
\pgfpathlineto{\pgfqpoint{2.765529in}{1.307121in}}%
\pgfpathlineto{\pgfqpoint{2.766676in}{1.297162in}}%
\pgfpathlineto{\pgfqpoint{2.767197in}{1.299063in}}%
\pgfpathlineto{\pgfqpoint{2.768795in}{1.308781in}}%
\pgfpathlineto{\pgfqpoint{2.769057in}{1.307756in}}%
\pgfpathlineto{\pgfqpoint{2.770370in}{1.297056in}}%
\pgfpathlineto{\pgfqpoint{2.770894in}{1.299396in}}%
\pgfpathlineto{\pgfqpoint{2.772239in}{1.308745in}}%
\pgfpathlineto{\pgfqpoint{2.772767in}{1.307248in}}%
\pgfpathlineto{\pgfqpoint{2.773956in}{1.296874in}}%
\pgfpathlineto{\pgfqpoint{2.774483in}{1.298823in}}%
\pgfpathlineto{\pgfqpoint{2.776101in}{1.308727in}}%
\pgfpathlineto{\pgfqpoint{2.776366in}{1.307674in}}%
\pgfpathlineto{\pgfqpoint{2.777674in}{1.296747in}}%
\pgfpathlineto{\pgfqpoint{2.778205in}{1.299018in}}%
\pgfpathlineto{\pgfqpoint{2.779567in}{1.308661in}}%
\pgfpathlineto{\pgfqpoint{2.780101in}{1.307308in}}%
\pgfpathlineto{\pgfqpoint{2.781307in}{1.296579in}}%
\pgfpathlineto{\pgfqpoint{2.781841in}{1.298464in}}%
\pgfpathlineto{\pgfqpoint{2.783478in}{1.308700in}}%
\pgfpathlineto{\pgfqpoint{2.783747in}{1.307703in}}%
\pgfpathlineto{\pgfqpoint{2.785031in}{1.296428in}}%
\pgfpathlineto{\pgfqpoint{2.785568in}{1.298373in}}%
\pgfpathlineto{\pgfqpoint{2.787216in}{1.308667in}}%
\pgfpathlineto{\pgfqpoint{2.787486in}{1.307630in}}%
\pgfpathlineto{\pgfqpoint{2.788833in}{1.296299in}}%
\pgfpathlineto{\pgfqpoint{2.789373in}{1.298668in}}%
\pgfpathlineto{\pgfqpoint{2.790761in}{1.308592in}}%
\pgfpathlineto{\pgfqpoint{2.791305in}{1.307158in}}%
\pgfpathlineto{\pgfqpoint{2.792584in}{1.296136in}}%
\pgfpathlineto{\pgfqpoint{2.793128in}{1.298445in}}%
\pgfpathlineto{\pgfqpoint{2.794798in}{1.308525in}}%
\pgfpathlineto{\pgfqpoint{2.795072in}{1.307218in}}%
\pgfpathlineto{\pgfqpoint{2.796419in}{1.296026in}}%
\pgfpathlineto{\pgfqpoint{2.796967in}{1.298653in}}%
\pgfpathlineto{\pgfqpoint{2.798370in}{1.308596in}}%
\pgfpathlineto{\pgfqpoint{2.798921in}{1.306826in}}%
\pgfpathlineto{\pgfqpoint{2.800108in}{1.295815in}}%
\pgfpathlineto{\pgfqpoint{2.800658in}{1.297640in}}%
\pgfpathlineto{\pgfqpoint{2.802347in}{1.308609in}}%
\pgfpathlineto{\pgfqpoint{2.802625in}{1.307674in}}%
\pgfpathlineto{\pgfqpoint{2.804099in}{1.295755in}}%
\pgfpathlineto{\pgfqpoint{2.804654in}{1.298621in}}%
\pgfpathlineto{\pgfqpoint{2.806075in}{1.308585in}}%
\pgfpathlineto{\pgfqpoint{2.806633in}{1.306489in}}%
\pgfpathlineto{\pgfqpoint{2.807849in}{1.295483in}}%
\pgfpathlineto{\pgfqpoint{2.808406in}{1.297649in}}%
\pgfpathlineto{\pgfqpoint{2.810119in}{1.308505in}}%
\pgfpathlineto{\pgfqpoint{2.810400in}{1.307329in}}%
\pgfpathlineto{\pgfqpoint{2.811606in}{1.295435in}}%
\pgfpathlineto{\pgfqpoint{2.812167in}{1.296671in}}%
\pgfpathlineto{\pgfqpoint{2.813889in}{1.308568in}}%
\pgfpathlineto{\pgfqpoint{2.814454in}{1.306050in}}%
\pgfpathlineto{\pgfqpoint{2.815568in}{1.295191in}}%
\pgfpathlineto{\pgfqpoint{2.816133in}{1.296824in}}%
\pgfpathlineto{\pgfqpoint{2.817867in}{1.308548in}}%
\pgfpathlineto{\pgfqpoint{2.818151in}{1.307757in}}%
\pgfpathlineto{\pgfqpoint{2.819540in}{1.294991in}}%
\pgfpathlineto{\pgfqpoint{2.820108in}{1.296890in}}%
\pgfpathlineto{\pgfqpoint{2.821854in}{1.308505in}}%
\pgfpathlineto{\pgfqpoint{2.822141in}{1.307537in}}%
\pgfpathlineto{\pgfqpoint{2.823626in}{1.294858in}}%
\pgfpathlineto{\pgfqpoint{2.824199in}{1.297618in}}%
\pgfpathlineto{\pgfqpoint{2.825670in}{1.308428in}}%
\pgfpathlineto{\pgfqpoint{2.826247in}{1.306625in}}%
\pgfpathlineto{\pgfqpoint{2.827493in}{1.294646in}}%
\pgfpathlineto{\pgfqpoint{2.828069in}{1.296564in}}%
\pgfpathlineto{\pgfqpoint{2.829839in}{1.308470in}}%
\pgfpathlineto{\pgfqpoint{2.830129in}{1.307496in}}%
\pgfpathlineto{\pgfqpoint{2.831498in}{1.294480in}}%
\pgfpathlineto{\pgfqpoint{2.832078in}{1.296315in}}%
\pgfpathlineto{\pgfqpoint{2.833860in}{1.308463in}}%
\pgfpathlineto{\pgfqpoint{2.834152in}{1.307551in}}%
\pgfpathlineto{\pgfqpoint{2.835710in}{1.294362in}}%
\pgfpathlineto{\pgfqpoint{2.836296in}{1.297381in}}%
\pgfpathlineto{\pgfqpoint{2.837795in}{1.308403in}}%
\pgfpathlineto{\pgfqpoint{2.838384in}{1.306308in}}%
\pgfpathlineto{\pgfqpoint{2.839718in}{1.294104in}}%
\pgfpathlineto{\pgfqpoint{2.840308in}{1.296706in}}%
\pgfpathlineto{\pgfqpoint{2.842116in}{1.308274in}}%
\pgfpathlineto{\pgfqpoint{2.842413in}{1.306809in}}%
\pgfpathlineto{\pgfqpoint{2.843765in}{1.293907in}}%
\pgfpathlineto{\pgfqpoint{2.844358in}{1.296127in}}%
\pgfpathlineto{\pgfqpoint{2.846178in}{1.308367in}}%
\pgfpathlineto{\pgfqpoint{2.846476in}{1.307198in}}%
\pgfpathlineto{\pgfqpoint{2.848071in}{1.293902in}}%
\pgfpathlineto{\pgfqpoint{2.848369in}{1.295154in}}%
\pgfpathlineto{\pgfqpoint{2.850202in}{1.308399in}}%
\pgfpathlineto{\pgfqpoint{2.850803in}{1.305721in}}%
\pgfpathlineto{\pgfqpoint{2.851998in}{1.293573in}}%
\pgfpathlineto{\pgfqpoint{2.852599in}{1.295378in}}%
\pgfpathlineto{\pgfqpoint{2.854444in}{1.308390in}}%
\pgfpathlineto{\pgfqpoint{2.854747in}{1.307516in}}%
\pgfpathlineto{\pgfqpoint{2.856347in}{1.293379in}}%
\pgfpathlineto{\pgfqpoint{2.856953in}{1.296361in}}%
\pgfpathlineto{\pgfqpoint{2.858509in}{1.308275in}}%
\pgfpathlineto{\pgfqpoint{2.859119in}{1.306367in}}%
\pgfpathlineto{\pgfqpoint{2.860541in}{1.293153in}}%
\pgfpathlineto{\pgfqpoint{2.860845in}{1.294024in}}%
\pgfpathlineto{\pgfqpoint{2.863026in}{1.308171in}}%
\pgfpathlineto{\pgfqpoint{2.863333in}{1.306573in}}%
\pgfpathlineto{\pgfqpoint{2.864817in}{1.292974in}}%
\pgfpathlineto{\pgfqpoint{2.865124in}{1.293946in}}%
\pgfpathlineto{\pgfqpoint{2.867011in}{1.308234in}}%
\pgfpathlineto{\pgfqpoint{2.867630in}{1.306353in}}%
\pgfpathlineto{\pgfqpoint{2.868941in}{1.292790in}}%
\pgfpathlineto{\pgfqpoint{2.869560in}{1.294625in}}%
\pgfpathlineto{\pgfqpoint{2.871460in}{1.308342in}}%
\pgfpathlineto{\pgfqpoint{2.871772in}{1.307464in}}%
\pgfpathlineto{\pgfqpoint{2.873480in}{1.292657in}}%
\pgfpathlineto{\pgfqpoint{2.874105in}{1.296144in}}%
\pgfpathlineto{\pgfqpoint{2.875706in}{1.308297in}}%
\pgfpathlineto{\pgfqpoint{2.876021in}{1.307864in}}%
\pgfpathlineto{\pgfqpoint{2.877813in}{1.292415in}}%
\pgfpathlineto{\pgfqpoint{2.878443in}{1.295831in}}%
\pgfpathlineto{\pgfqpoint{2.880055in}{1.308264in}}%
\pgfpathlineto{\pgfqpoint{2.880372in}{1.307920in}}%
\pgfpathlineto{\pgfqpoint{2.881995in}{1.292153in}}%
\pgfpathlineto{\pgfqpoint{2.882947in}{1.296672in}}%
\pgfpathlineto{\pgfqpoint{2.884572in}{1.308313in}}%
\pgfpathlineto{\pgfqpoint{2.884891in}{1.307366in}}%
\pgfpathlineto{\pgfqpoint{2.886613in}{1.291985in}}%
\pgfpathlineto{\pgfqpoint{2.887253in}{1.295483in}}%
\pgfpathlineto{\pgfqpoint{2.888890in}{1.308246in}}%
\pgfpathlineto{\pgfqpoint{2.889212in}{1.307897in}}%
\pgfpathlineto{\pgfqpoint{2.891056in}{1.291752in}}%
\pgfpathlineto{\pgfqpoint{2.891701in}{1.295237in}}%
\pgfpathlineto{\pgfqpoint{2.893352in}{1.308229in}}%
\pgfpathlineto{\pgfqpoint{2.893676in}{1.307914in}}%
\pgfpathlineto{\pgfqpoint{2.895460in}{1.291442in}}%
\pgfpathlineto{\pgfqpoint{2.896436in}{1.297282in}}%
\pgfpathlineto{\pgfqpoint{2.898101in}{1.308143in}}%
\pgfpathlineto{\pgfqpoint{2.898427in}{1.306520in}}%
\pgfpathlineto{\pgfqpoint{2.900094in}{1.291354in}}%
\pgfpathlineto{\pgfqpoint{2.900421in}{1.292689in}}%
\pgfpathlineto{\pgfqpoint{2.902427in}{1.308257in}}%
\pgfpathlineto{\pgfqpoint{2.903085in}{1.305488in}}%
\pgfpathlineto{\pgfqpoint{2.904417in}{1.291041in}}%
\pgfpathlineto{\pgfqpoint{2.905075in}{1.293075in}}%
\pgfpathlineto{\pgfqpoint{2.907097in}{1.308291in}}%
\pgfpathlineto{\pgfqpoint{2.907429in}{1.307318in}}%
\pgfpathlineto{\pgfqpoint{2.909039in}{1.290771in}}%
\pgfpathlineto{\pgfqpoint{2.909703in}{1.293072in}}%
\pgfpathlineto{\pgfqpoint{2.911742in}{1.308276in}}%
\pgfpathlineto{\pgfqpoint{2.912076in}{1.307131in}}%
\pgfpathlineto{\pgfqpoint{2.913594in}{1.290651in}}%
\pgfpathlineto{\pgfqpoint{2.914263in}{1.292332in}}%
\pgfpathlineto{\pgfqpoint{2.916319in}{1.308288in}}%
\pgfpathlineto{\pgfqpoint{2.916656in}{1.307545in}}%
\pgfpathlineto{\pgfqpoint{2.918347in}{1.290279in}}%
\pgfpathlineto{\pgfqpoint{2.919022in}{1.292701in}}%
\pgfpathlineto{\pgfqpoint{2.921095in}{1.308273in}}%
\pgfpathlineto{\pgfqpoint{2.921435in}{1.307057in}}%
\pgfpathlineto{\pgfqpoint{2.923035in}{1.290050in}}%
\pgfpathlineto{\pgfqpoint{2.923716in}{1.292333in}}%
\pgfpathlineto{\pgfqpoint{2.925806in}{1.308290in}}%
\pgfpathlineto{\pgfqpoint{2.926149in}{1.307174in}}%
\pgfpathlineto{\pgfqpoint{2.927819in}{1.289763in}}%
\pgfpathlineto{\pgfqpoint{2.928505in}{1.292376in}}%
\pgfpathlineto{\pgfqpoint{2.930613in}{1.308268in}}%
\pgfpathlineto{\pgfqpoint{2.930959in}{1.306933in}}%
\pgfpathlineto{\pgfqpoint{2.932521in}{1.289622in}}%
\pgfpathlineto{\pgfqpoint{2.933212in}{1.291521in}}%
\pgfpathlineto{\pgfqpoint{2.935338in}{1.308311in}}%
\pgfpathlineto{\pgfqpoint{2.935687in}{1.307451in}}%
\pgfpathlineto{\pgfqpoint{2.937422in}{1.289253in}}%
\pgfpathlineto{\pgfqpoint{2.938120in}{1.291809in}}%
\pgfpathlineto{\pgfqpoint{2.940265in}{1.308299in}}%
\pgfpathlineto{\pgfqpoint{2.940617in}{1.307016in}}%
\pgfpathlineto{\pgfqpoint{2.942511in}{1.289122in}}%
\pgfpathlineto{\pgfqpoint{2.942864in}{1.290641in}}%
\pgfpathlineto{\pgfqpoint{2.945027in}{1.308290in}}%
\pgfpathlineto{\pgfqpoint{2.945737in}{1.305136in}}%
\pgfpathlineto{\pgfqpoint{2.947439in}{1.288873in}}%
\pgfpathlineto{\pgfqpoint{2.947794in}{1.290458in}}%
\pgfpathlineto{\pgfqpoint{2.949977in}{1.308315in}}%
\pgfpathlineto{\pgfqpoint{2.950693in}{1.304988in}}%
\pgfpathlineto{\pgfqpoint{2.952083in}{1.288565in}}%
\pgfpathlineto{\pgfqpoint{2.952799in}{1.290477in}}%
\pgfpathlineto{\pgfqpoint{2.955002in}{1.308357in}}%
\pgfpathlineto{\pgfqpoint{2.955363in}{1.307507in}}%
\pgfpathlineto{\pgfqpoint{2.957106in}{1.288240in}}%
\pgfpathlineto{\pgfqpoint{2.957829in}{1.290367in}}%
\pgfpathlineto{\pgfqpoint{2.960051in}{1.308380in}}%
\pgfpathlineto{\pgfqpoint{2.960416in}{1.307402in}}%
\pgfpathlineto{\pgfqpoint{2.962381in}{1.287884in}}%
\pgfpathlineto{\pgfqpoint{2.963112in}{1.291949in}}%
\pgfpathlineto{\pgfqpoint{2.964989in}{1.308252in}}%
\pgfpathlineto{\pgfqpoint{2.965358in}{1.308027in}}%
\pgfpathlineto{\pgfqpoint{2.966331in}{1.296596in}}%
\pgfpathlineto{\pgfqpoint{2.967430in}{1.287540in}}%
\pgfpathlineto{\pgfqpoint{2.967798in}{1.288671in}}%
\pgfpathlineto{\pgfqpoint{2.970434in}{1.308193in}}%
\pgfpathlineto{\pgfqpoint{2.970805in}{1.306088in}}%
\pgfpathlineto{\pgfqpoint{2.972387in}{1.287362in}}%
\pgfpathlineto{\pgfqpoint{2.973130in}{1.289541in}}%
\pgfpathlineto{\pgfqpoint{2.975415in}{1.308447in}}%
\pgfpathlineto{\pgfqpoint{2.975791in}{1.307453in}}%
\pgfpathlineto{\pgfqpoint{2.977890in}{1.287113in}}%
\pgfpathlineto{\pgfqpoint{2.978643in}{1.291936in}}%
\pgfpathlineto{\pgfqpoint{2.980572in}{1.308436in}}%
\pgfpathlineto{\pgfqpoint{2.980951in}{1.307789in}}%
\pgfpathlineto{\pgfqpoint{2.983062in}{1.286664in}}%
\pgfpathlineto{\pgfqpoint{2.983822in}{1.291019in}}%
\pgfpathlineto{\pgfqpoint{2.985771in}{1.308356in}}%
\pgfpathlineto{\pgfqpoint{2.986154in}{1.308098in}}%
\pgfpathlineto{\pgfqpoint{2.987341in}{1.292905in}}%
\pgfpathlineto{\pgfqpoint{2.988100in}{1.286440in}}%
\pgfpathlineto{\pgfqpoint{2.988482in}{1.286626in}}%
\pgfpathlineto{\pgfqpoint{2.990034in}{1.300629in}}%
\pgfpathlineto{\pgfqpoint{2.991217in}{1.308538in}}%
\pgfpathlineto{\pgfqpoint{2.991603in}{1.307551in}}%
\pgfpathlineto{\pgfqpoint{2.993467in}{1.286063in}}%
\pgfpathlineto{\pgfqpoint{2.994239in}{1.288512in}}%
\pgfpathlineto{\pgfqpoint{2.996614in}{1.308578in}}%
\pgfpathlineto{\pgfqpoint{2.997004in}{1.307439in}}%
\pgfpathlineto{\pgfqpoint{2.998854in}{1.285744in}}%
\pgfpathlineto{\pgfqpoint{2.999634in}{1.288153in}}%
\pgfpathlineto{\pgfqpoint{3.002033in}{1.308619in}}%
\pgfpathlineto{\pgfqpoint{3.002427in}{1.307511in}}%
\pgfpathlineto{\pgfqpoint{3.004350in}{1.285320in}}%
\pgfpathlineto{\pgfqpoint{3.005138in}{1.288206in}}%
\pgfpathlineto{\pgfqpoint{3.007561in}{1.308653in}}%
\pgfpathlineto{\pgfqpoint{3.007959in}{1.307247in}}%
\pgfpathlineto{\pgfqpoint{3.009782in}{1.285112in}}%
\pgfpathlineto{\pgfqpoint{3.010578in}{1.287333in}}%
\pgfpathlineto{\pgfqpoint{3.013026in}{1.308702in}}%
\pgfpathlineto{\pgfqpoint{3.013428in}{1.307723in}}%
\pgfpathlineto{\pgfqpoint{3.015336in}{1.284775in}}%
\pgfpathlineto{\pgfqpoint{3.016139in}{1.286949in}}%
\pgfpathlineto{\pgfqpoint{3.018613in}{1.308748in}}%
\pgfpathlineto{\pgfqpoint{3.019020in}{1.307805in}}%
\pgfpathlineto{\pgfqpoint{3.020959in}{1.284399in}}%
\pgfpathlineto{\pgfqpoint{3.021771in}{1.286648in}}%
\pgfpathlineto{\pgfqpoint{3.024271in}{1.308807in}}%
\pgfpathlineto{\pgfqpoint{3.024682in}{1.307824in}}%
\pgfpathlineto{\pgfqpoint{3.027008in}{1.284022in}}%
\pgfpathlineto{\pgfqpoint{3.027833in}{1.289638in}}%
\pgfpathlineto{\pgfqpoint{3.029946in}{1.308833in}}%
\pgfpathlineto{\pgfqpoint{3.030361in}{1.308075in}}%
\pgfpathlineto{\pgfqpoint{3.032692in}{1.283510in}}%
\pgfpathlineto{\pgfqpoint{3.033524in}{1.288738in}}%
\pgfpathlineto{\pgfqpoint{3.035661in}{1.308806in}}%
\pgfpathlineto{\pgfqpoint{3.036081in}{1.308398in}}%
\pgfpathlineto{\pgfqpoint{3.037175in}{1.294148in}}%
\pgfpathlineto{\pgfqpoint{3.038434in}{1.283027in}}%
\pgfpathlineto{\pgfqpoint{3.038854in}{1.284503in}}%
\pgfpathlineto{\pgfqpoint{3.041860in}{1.308685in}}%
\pgfpathlineto{\pgfqpoint{3.042284in}{1.305972in}}%
\pgfpathlineto{\pgfqpoint{3.044455in}{1.282909in}}%
\pgfpathlineto{\pgfqpoint{3.044880in}{1.285145in}}%
\pgfpathlineto{\pgfqpoint{3.047491in}{1.309067in}}%
\pgfpathlineto{\pgfqpoint{3.047920in}{1.308102in}}%
\pgfpathlineto{\pgfqpoint{3.050267in}{1.282233in}}%
\pgfpathlineto{\pgfqpoint{3.051127in}{1.287456in}}%
\pgfpathlineto{\pgfqpoint{3.053336in}{1.308946in}}%
\pgfpathlineto{\pgfqpoint{3.053771in}{1.308728in}}%
\pgfpathlineto{\pgfqpoint{3.054984in}{1.292598in}}%
\pgfpathlineto{\pgfqpoint{3.056283in}{1.281825in}}%
\pgfpathlineto{\pgfqpoint{3.057153in}{1.287272in}}%
\pgfpathlineto{\pgfqpoint{3.059388in}{1.309063in}}%
\pgfpathlineto{\pgfqpoint{3.059827in}{1.308752in}}%
\pgfpathlineto{\pgfqpoint{3.061017in}{1.292809in}}%
\pgfpathlineto{\pgfqpoint{3.062333in}{1.281354in}}%
\pgfpathlineto{\pgfqpoint{3.063213in}{1.286724in}}%
\pgfpathlineto{\pgfqpoint{3.065474in}{1.309100in}}%
\pgfpathlineto{\pgfqpoint{3.065918in}{1.308913in}}%
\pgfpathlineto{\pgfqpoint{3.067062in}{1.294251in}}%
\pgfpathlineto{\pgfqpoint{3.068588in}{1.281128in}}%
\pgfpathlineto{\pgfqpoint{3.069930in}{1.292852in}}%
\pgfpathlineto{\pgfqpoint{3.071766in}{1.309404in}}%
\pgfpathlineto{\pgfqpoint{3.072216in}{1.308499in}}%
\pgfpathlineto{\pgfqpoint{3.074782in}{1.280610in}}%
\pgfpathlineto{\pgfqpoint{3.075684in}{1.287003in}}%
\pgfpathlineto{\pgfqpoint{3.077997in}{1.309479in}}%
\pgfpathlineto{\pgfqpoint{3.078452in}{1.308703in}}%
\pgfpathlineto{\pgfqpoint{3.079788in}{1.288871in}}%
\pgfpathlineto{\pgfqpoint{3.080692in}{1.280137in}}%
\pgfpathlineto{\pgfqpoint{3.081146in}{1.280353in}}%
\pgfpathlineto{\pgfqpoint{3.082521in}{1.293067in}}%
\pgfpathlineto{\pgfqpoint{3.084401in}{1.309662in}}%
\pgfpathlineto{\pgfqpoint{3.085320in}{1.303861in}}%
\pgfpathlineto{\pgfqpoint{3.087116in}{1.279511in}}%
\pgfpathlineto{\pgfqpoint{3.087576in}{1.280103in}}%
\pgfpathlineto{\pgfqpoint{3.088969in}{1.293664in}}%
\pgfpathlineto{\pgfqpoint{3.090871in}{1.309783in}}%
\pgfpathlineto{\pgfqpoint{3.091801in}{1.302979in}}%
\pgfpathlineto{\pgfqpoint{3.093772in}{1.278943in}}%
\pgfpathlineto{\pgfqpoint{3.094238in}{1.280760in}}%
\pgfpathlineto{\pgfqpoint{3.097107in}{1.309602in}}%
\pgfpathlineto{\pgfqpoint{3.098050in}{1.306175in}}%
\pgfpathlineto{\pgfqpoint{3.100445in}{1.278727in}}%
\pgfpathlineto{\pgfqpoint{3.101872in}{1.291846in}}%
\pgfpathlineto{\pgfqpoint{3.103824in}{1.310020in}}%
\pgfpathlineto{\pgfqpoint{3.104302in}{1.308927in}}%
\pgfpathlineto{\pgfqpoint{3.106936in}{1.277927in}}%
\pgfpathlineto{\pgfqpoint{3.107895in}{1.284278in}}%
\pgfpathlineto{\pgfqpoint{3.110358in}{1.309960in}}%
\pgfpathlineto{\pgfqpoint{3.110842in}{1.309634in}}%
\pgfpathlineto{\pgfqpoint{3.112084in}{1.292494in}}%
\pgfpathlineto{\pgfqpoint{3.113327in}{1.277501in}}%
\pgfpathlineto{\pgfqpoint{3.113810in}{1.277851in}}%
\pgfpathlineto{\pgfqpoint{3.115275in}{1.292151in}}%
\pgfpathlineto{\pgfqpoint{3.117278in}{1.310358in}}%
\pgfpathlineto{\pgfqpoint{3.118257in}{1.303652in}}%
\pgfpathlineto{\pgfqpoint{3.120496in}{1.276980in}}%
\pgfpathlineto{\pgfqpoint{3.121978in}{1.290575in}}%
\pgfpathlineto{\pgfqpoint{3.124009in}{1.310442in}}%
\pgfpathlineto{\pgfqpoint{3.124506in}{1.309541in}}%
\pgfpathlineto{\pgfqpoint{3.125710in}{1.291744in}}%
\pgfpathlineto{\pgfqpoint{3.127046in}{1.276203in}}%
\pgfpathlineto{\pgfqpoint{3.127543in}{1.277021in}}%
\pgfpathlineto{\pgfqpoint{3.129049in}{1.292670in}}%
\pgfpathlineto{\pgfqpoint{3.131104in}{1.310686in}}%
\pgfpathlineto{\pgfqpoint{3.132110in}{1.302584in}}%
\pgfpathlineto{\pgfqpoint{3.134152in}{1.275513in}}%
\pgfpathlineto{\pgfqpoint{3.134656in}{1.277209in}}%
\pgfpathlineto{\pgfqpoint{3.138269in}{1.310627in}}%
\pgfpathlineto{\pgfqpoint{3.138779in}{1.307364in}}%
\pgfpathlineto{\pgfqpoint{3.141315in}{1.274983in}}%
\pgfpathlineto{\pgfqpoint{3.141826in}{1.277295in}}%
\pgfpathlineto{\pgfqpoint{3.144975in}{1.310827in}}%
\pgfpathlineto{\pgfqpoint{3.146010in}{1.306323in}}%
\pgfpathlineto{\pgfqpoint{3.148505in}{1.274386in}}%
\pgfpathlineto{\pgfqpoint{3.149545in}{1.281927in}}%
\pgfpathlineto{\pgfqpoint{3.152218in}{1.311107in}}%
\pgfpathlineto{\pgfqpoint{3.152743in}{1.310542in}}%
\pgfpathlineto{\pgfqpoint{3.154035in}{1.291708in}}%
\pgfpathlineto{\pgfqpoint{3.155768in}{1.273710in}}%
\pgfpathlineto{\pgfqpoint{3.156823in}{1.281410in}}%
\pgfpathlineto{\pgfqpoint{3.159535in}{1.311325in}}%
\pgfpathlineto{\pgfqpoint{3.160068in}{1.310766in}}%
\pgfpathlineto{\pgfqpoint{3.161405in}{1.290885in}}%
\pgfpathlineto{\pgfqpoint{3.162746in}{1.273219in}}%
\pgfpathlineto{\pgfqpoint{3.163279in}{1.273406in}}%
\pgfpathlineto{\pgfqpoint{3.164892in}{1.289861in}}%
\pgfpathlineto{\pgfqpoint{3.167103in}{1.311780in}}%
\pgfpathlineto{\pgfqpoint{3.167644in}{1.310200in}}%
\pgfpathlineto{\pgfqpoint{3.170664in}{1.272397in}}%
\pgfpathlineto{\pgfqpoint{3.171207in}{1.275340in}}%
\pgfpathlineto{\pgfqpoint{3.174543in}{1.311912in}}%
\pgfpathlineto{\pgfqpoint{3.175092in}{1.311031in}}%
\pgfpathlineto{\pgfqpoint{3.176430in}{1.290320in}}%
\pgfpathlineto{\pgfqpoint{3.177815in}{1.271753in}}%
\pgfpathlineto{\pgfqpoint{3.178364in}{1.271985in}}%
\pgfpathlineto{\pgfqpoint{3.180027in}{1.289343in}}%
\pgfpathlineto{\pgfqpoint{3.182306in}{1.312327in}}%
\pgfpathlineto{\pgfqpoint{3.182864in}{1.310627in}}%
\pgfpathlineto{\pgfqpoint{3.184338in}{1.284926in}}%
\pgfpathlineto{\pgfqpoint{3.186011in}{1.270994in}}%
\pgfpathlineto{\pgfqpoint{3.187133in}{1.280321in}}%
\pgfpathlineto{\pgfqpoint{3.190014in}{1.312555in}}%
\pgfpathlineto{\pgfqpoint{3.190580in}{1.311325in}}%
\pgfpathlineto{\pgfqpoint{3.192049in}{1.286638in}}%
\pgfpathlineto{\pgfqpoint{3.193754in}{1.269972in}}%
\pgfpathlineto{\pgfqpoint{3.194893in}{1.278632in}}%
\pgfpathlineto{\pgfqpoint{3.197820in}{1.312689in}}%
\pgfpathlineto{\pgfqpoint{3.198396in}{1.312111in}}%
\pgfpathlineto{\pgfqpoint{3.199811in}{1.290402in}}%
\pgfpathlineto{\pgfqpoint{3.201666in}{1.269059in}}%
\pgfpathlineto{\pgfqpoint{3.202823in}{1.277462in}}%
\pgfpathlineto{\pgfqpoint{3.205797in}{1.312884in}}%
\pgfpathlineto{\pgfqpoint{3.206382in}{1.312646in}}%
\pgfpathlineto{\pgfqpoint{3.207775in}{1.292562in}}%
\pgfpathlineto{\pgfqpoint{3.209820in}{1.268311in}}%
\pgfpathlineto{\pgfqpoint{3.210996in}{1.277630in}}%
\pgfpathlineto{\pgfqpoint{3.214020in}{1.313425in}}%
\pgfpathlineto{\pgfqpoint{3.214615in}{1.312688in}}%
\pgfpathlineto{\pgfqpoint{3.216085in}{1.289201in}}%
\pgfpathlineto{\pgfqpoint{3.217520in}{1.267921in}}%
\pgfpathlineto{\pgfqpoint{3.218115in}{1.267595in}}%
\pgfpathlineto{\pgfqpoint{3.219311in}{1.277900in}}%
\pgfpathlineto{\pgfqpoint{3.222388in}{1.313952in}}%
\pgfpathlineto{\pgfqpoint{3.222992in}{1.312652in}}%
\pgfpathlineto{\pgfqpoint{3.224410in}{1.289078in}}%
\pgfpathlineto{\pgfqpoint{3.225836in}{1.267188in}}%
\pgfpathlineto{\pgfqpoint{3.226441in}{1.266550in}}%
\pgfpathlineto{\pgfqpoint{3.227657in}{1.276735in}}%
\pgfpathlineto{\pgfqpoint{3.230786in}{1.314291in}}%
\pgfpathlineto{\pgfqpoint{3.231402in}{1.313278in}}%
\pgfpathlineto{\pgfqpoint{3.232926in}{1.287686in}}%
\pgfpathlineto{\pgfqpoint{3.234576in}{1.265414in}}%
\pgfpathlineto{\pgfqpoint{3.235192in}{1.266522in}}%
\pgfpathlineto{\pgfqpoint{3.237060in}{1.288846in}}%
\pgfpathlineto{\pgfqpoint{3.239616in}{1.314909in}}%
\pgfpathlineto{\pgfqpoint{3.240242in}{1.311919in}}%
\pgfpathlineto{\pgfqpoint{3.243510in}{1.264342in}}%
\pgfpathlineto{\pgfqpoint{3.244139in}{1.267394in}}%
\pgfpathlineto{\pgfqpoint{3.248011in}{1.314917in}}%
\pgfpathlineto{\pgfqpoint{3.248649in}{1.314671in}}%
\pgfpathlineto{\pgfqpoint{3.250110in}{1.293062in}}%
\pgfpathlineto{\pgfqpoint{3.252250in}{1.263189in}}%
\pgfpathlineto{\pgfqpoint{3.253532in}{1.272222in}}%
\pgfpathlineto{\pgfqpoint{3.257480in}{1.315484in}}%
\pgfpathlineto{\pgfqpoint{3.258775in}{1.299510in}}%
\pgfpathlineto{\pgfqpoint{3.261399in}{1.262331in}}%
\pgfpathlineto{\pgfqpoint{3.262707in}{1.273590in}}%
\pgfpathlineto{\pgfqpoint{3.266072in}{1.316221in}}%
\pgfpathlineto{\pgfqpoint{3.266734in}{1.315240in}}%
\pgfpathlineto{\pgfqpoint{3.268321in}{1.288285in}}%
\pgfpathlineto{\pgfqpoint{3.270200in}{1.260985in}}%
\pgfpathlineto{\pgfqpoint{3.270864in}{1.262459in}}%
\pgfpathlineto{\pgfqpoint{3.272876in}{1.288038in}}%
\pgfpathlineto{\pgfqpoint{3.275629in}{1.317027in}}%
\pgfpathlineto{\pgfqpoint{3.276303in}{1.313383in}}%
\pgfpathlineto{\pgfqpoint{3.279515in}{1.259777in}}%
\pgfpathlineto{\pgfqpoint{3.280192in}{1.261215in}}%
\pgfpathlineto{\pgfqpoint{3.282243in}{1.287511in}}%
\pgfpathlineto{\pgfqpoint{3.285051in}{1.317679in}}%
\pgfpathlineto{\pgfqpoint{3.285739in}{1.314013in}}%
\pgfpathlineto{\pgfqpoint{3.289348in}{1.258522in}}%
\pgfpathlineto{\pgfqpoint{3.290040in}{1.262329in}}%
\pgfpathlineto{\pgfqpoint{3.294303in}{1.317938in}}%
\pgfpathlineto{\pgfqpoint{3.295006in}{1.317371in}}%
\pgfpathlineto{\pgfqpoint{3.296614in}{1.291225in}}%
\pgfpathlineto{\pgfqpoint{3.298440in}{1.258198in}}%
\pgfpathlineto{\pgfqpoint{3.299144in}{1.257371in}}%
\pgfpathlineto{\pgfqpoint{3.300559in}{1.270471in}}%
\pgfpathlineto{\pgfqpoint{3.304204in}{1.318904in}}%
\pgfpathlineto{\pgfqpoint{3.304922in}{1.317597in}}%
\pgfpathlineto{\pgfqpoint{3.306637in}{1.286352in}}%
\pgfpathlineto{\pgfqpoint{3.308275in}{1.257217in}}%
\pgfpathlineto{\pgfqpoint{3.308994in}{1.255778in}}%
\pgfpathlineto{\pgfqpoint{3.310439in}{1.268531in}}%
\pgfpathlineto{\pgfqpoint{3.314161in}{1.319501in}}%
\pgfpathlineto{\pgfqpoint{3.314894in}{1.318751in}}%
\pgfpathlineto{\pgfqpoint{3.316571in}{1.290322in}}%
\pgfpathlineto{\pgfqpoint{3.318379in}{1.255844in}}%
\pgfpathlineto{\pgfqpoint{3.319113in}{1.254256in}}%
\pgfpathlineto{\pgfqpoint{3.320590in}{1.267339in}}%
\pgfpathlineto{\pgfqpoint{3.324393in}{1.320313in}}%
\pgfpathlineto{\pgfqpoint{3.325142in}{1.319637in}}%
\pgfpathlineto{\pgfqpoint{3.326634in}{1.296609in}}%
\pgfpathlineto{\pgfqpoint{3.329031in}{1.252810in}}%
\pgfpathlineto{\pgfqpoint{3.329783in}{1.253861in}}%
\pgfpathlineto{\pgfqpoint{3.331295in}{1.271117in}}%
\pgfpathlineto{\pgfqpoint{3.335187in}{1.321750in}}%
\pgfpathlineto{\pgfqpoint{3.335952in}{1.318134in}}%
\pgfpathlineto{\pgfqpoint{3.337744in}{1.279369in}}%
\pgfpathlineto{\pgfqpoint{3.340074in}{1.251037in}}%
\pgfpathlineto{\pgfqpoint{3.341618in}{1.265372in}}%
\pgfpathlineto{\pgfqpoint{3.345598in}{1.322284in}}%
\pgfpathlineto{\pgfqpoint{3.346382in}{1.321383in}}%
\pgfpathlineto{\pgfqpoint{3.347943in}{1.296138in}}%
\pgfpathlineto{\pgfqpoint{3.350529in}{1.249145in}}%
\pgfpathlineto{\pgfqpoint{3.351316in}{1.251073in}}%
\pgfpathlineto{\pgfqpoint{3.353702in}{1.285030in}}%
\pgfpathlineto{\pgfqpoint{3.356170in}{1.321303in}}%
\pgfpathlineto{\pgfqpoint{3.356975in}{1.323798in}}%
\pgfpathlineto{\pgfqpoint{3.357776in}{1.318983in}}%
\pgfpathlineto{\pgfqpoint{3.361967in}{1.247298in}}%
\pgfpathlineto{\pgfqpoint{3.362775in}{1.252138in}}%
\pgfpathlineto{\pgfqpoint{3.367759in}{1.324392in}}%
\pgfpathlineto{\pgfqpoint{3.368582in}{1.323788in}}%
\pgfpathlineto{\pgfqpoint{3.370219in}{1.297321in}}%
\pgfpathlineto{\pgfqpoint{3.372617in}{1.246674in}}%
\pgfpathlineto{\pgfqpoint{3.373443in}{1.245575in}}%
\pgfpathlineto{\pgfqpoint{3.375104in}{1.262644in}}%
\pgfpathlineto{\pgfqpoint{3.379387in}{1.326023in}}%
\pgfpathlineto{\pgfqpoint{3.380230in}{1.324325in}}%
\pgfpathlineto{\pgfqpoint{3.381909in}{1.294208in}}%
\pgfpathlineto{\pgfqpoint{3.384176in}{1.245135in}}%
\pgfpathlineto{\pgfqpoint{3.385023in}{1.243230in}}%
\pgfpathlineto{\pgfqpoint{3.386726in}{1.259982in}}%
\pgfpathlineto{\pgfqpoint{3.391117in}{1.327164in}}%
\pgfpathlineto{\pgfqpoint{3.391983in}{1.326198in}}%
\pgfpathlineto{\pgfqpoint{3.393706in}{1.296658in}}%
\pgfpathlineto{\pgfqpoint{3.396401in}{1.241348in}}%
\pgfpathlineto{\pgfqpoint{3.397272in}{1.242060in}}%
\pgfpathlineto{\pgfqpoint{3.399022in}{1.263265in}}%
\pgfpathlineto{\pgfqpoint{3.403538in}{1.329293in}}%
\pgfpathlineto{\pgfqpoint{3.404426in}{1.325350in}}%
\pgfpathlineto{\pgfqpoint{3.406406in}{1.280406in}}%
\pgfpathlineto{\pgfqpoint{3.408710in}{1.238791in}}%
\pgfpathlineto{\pgfqpoint{3.409605in}{1.239894in}}%
\pgfpathlineto{\pgfqpoint{3.411403in}{1.262552in}}%
\pgfpathlineto{\pgfqpoint{3.416044in}{1.330997in}}%
\pgfpathlineto{\pgfqpoint{3.416957in}{1.326476in}}%
\pgfpathlineto{\pgfqpoint{3.419105in}{1.274809in}}%
\pgfpathlineto{\pgfqpoint{3.420989in}{1.238060in}}%
\pgfpathlineto{\pgfqpoint{3.421908in}{1.235924in}}%
\pgfpathlineto{\pgfqpoint{3.423755in}{1.255178in}}%
\pgfpathlineto{\pgfqpoint{3.428523in}{1.332171in}}%
\pgfpathlineto{\pgfqpoint{3.429463in}{1.331022in}}%
\pgfpathlineto{\pgfqpoint{3.431334in}{1.297027in}}%
\pgfpathlineto{\pgfqpoint{3.434002in}{1.235025in}}%
\pgfpathlineto{\pgfqpoint{3.434949in}{1.233256in}}%
\pgfpathlineto{\pgfqpoint{3.436850in}{1.254125in}}%
\pgfpathlineto{\pgfqpoint{3.441763in}{1.334349in}}%
\pgfpathlineto{\pgfqpoint{3.442731in}{1.332629in}}%
\pgfpathlineto{\pgfqpoint{3.444658in}{1.295717in}}%
\pgfpathlineto{\pgfqpoint{3.447272in}{1.232766in}}%
\pgfpathlineto{\pgfqpoint{3.448247in}{1.230004in}}%
\pgfpathlineto{\pgfqpoint{3.450206in}{1.250678in}}%
\pgfpathlineto{\pgfqpoint{3.455263in}{1.336205in}}%
\pgfpathlineto{\pgfqpoint{3.456261in}{1.335377in}}%
\pgfpathlineto{\pgfqpoint{3.458248in}{1.298860in}}%
\pgfpathlineto{\pgfqpoint{3.460721in}{1.232902in}}%
\pgfpathlineto{\pgfqpoint{3.461727in}{1.226489in}}%
\pgfpathlineto{\pgfqpoint{3.462734in}{1.230613in}}%
\pgfpathlineto{\pgfqpoint{3.464761in}{1.261760in}}%
\pgfpathlineto{\pgfqpoint{3.468950in}{1.336700in}}%
\pgfpathlineto{\pgfqpoint{3.469983in}{1.339250in}}%
\pgfpathlineto{\pgfqpoint{3.471010in}{1.330503in}}%
\pgfpathlineto{\pgfqpoint{3.475601in}{1.223890in}}%
\pgfpathlineto{\pgfqpoint{3.476641in}{1.224676in}}%
\pgfpathlineto{\pgfqpoint{3.478730in}{1.252945in}}%
\pgfpathlineto{\pgfqpoint{3.483068in}{1.336162in}}%
\pgfpathlineto{\pgfqpoint{3.484137in}{1.342311in}}%
\pgfpathlineto{\pgfqpoint{3.485200in}{1.337153in}}%
\pgfpathlineto{\pgfqpoint{3.487482in}{1.280279in}}%
\pgfpathlineto{\pgfqpoint{3.489121in}{1.233342in}}%
\pgfpathlineto{\pgfqpoint{3.491272in}{1.220396in}}%
\pgfpathlineto{\pgfqpoint{3.493431in}{1.248934in}}%
\pgfpathlineto{\pgfqpoint{3.499027in}{1.345243in}}%
\pgfpathlineto{\pgfqpoint{3.500128in}{1.340861in}}%
\pgfpathlineto{\pgfqpoint{3.502416in}{1.286171in}}%
\pgfpathlineto{\pgfqpoint{3.504262in}{1.230263in}}%
\pgfpathlineto{\pgfqpoint{3.506489in}{1.216176in}}%
\pgfpathlineto{\pgfqpoint{3.508725in}{1.246191in}}%
\pgfpathlineto{\pgfqpoint{3.514522in}{1.348599in}}%
\pgfpathlineto{\pgfqpoint{3.515662in}{1.344130in}}%
\pgfpathlineto{\pgfqpoint{3.518032in}{1.286555in}}%
\pgfpathlineto{\pgfqpoint{3.519969in}{1.226444in}}%
\pgfpathlineto{\pgfqpoint{3.521124in}{1.211964in}}%
\pgfpathlineto{\pgfqpoint{3.522278in}{1.211689in}}%
\pgfpathlineto{\pgfqpoint{3.524596in}{1.243705in}}%
\pgfpathlineto{\pgfqpoint{3.529420in}{1.343989in}}%
\pgfpathlineto{\pgfqpoint{3.530611in}{1.352407in}}%
\pgfpathlineto{\pgfqpoint{3.531794in}{1.347530in}}%
\pgfpathlineto{\pgfqpoint{3.534148in}{1.290186in}}%
\pgfpathlineto{\pgfqpoint{3.536396in}{1.219748in}}%
\pgfpathlineto{\pgfqpoint{3.537595in}{1.206218in}}%
\pgfpathlineto{\pgfqpoint{3.538795in}{1.207557in}}%
\pgfpathlineto{\pgfqpoint{3.541203in}{1.243836in}}%
\pgfpathlineto{\pgfqpoint{3.546213in}{1.349323in}}%
\pgfpathlineto{\pgfqpoint{3.547450in}{1.356825in}}%
\pgfpathlineto{\pgfqpoint{3.548679in}{1.349862in}}%
\pgfpathlineto{\pgfqpoint{3.551269in}{1.278476in}}%
\pgfpathlineto{\pgfqpoint{3.553202in}{1.217462in}}%
\pgfpathlineto{\pgfqpoint{3.554450in}{1.201353in}}%
\pgfpathlineto{\pgfqpoint{3.555697in}{1.201365in}}%
\pgfpathlineto{\pgfqpoint{3.558201in}{1.238209in}}%
\pgfpathlineto{\pgfqpoint{3.563419in}{1.352215in}}%
\pgfpathlineto{\pgfqpoint{3.564708in}{1.361502in}}%
\pgfpathlineto{\pgfqpoint{3.565988in}{1.355563in}}%
\pgfpathlineto{\pgfqpoint{3.568533in}{1.289325in}}%
\pgfpathlineto{\pgfqpoint{3.570602in}{1.217114in}}%
\pgfpathlineto{\pgfqpoint{3.571904in}{1.196824in}}%
\pgfpathlineto{\pgfqpoint{3.573204in}{1.194220in}}%
\pgfpathlineto{\pgfqpoint{3.574506in}{1.206473in}}%
\pgfpathlineto{\pgfqpoint{3.578451in}{1.295125in}}%
\pgfpathlineto{\pgfqpoint{3.581233in}{1.353846in}}%
\pgfpathlineto{\pgfqpoint{3.582579in}{1.366296in}}%
\pgfpathlineto{\pgfqpoint{3.583915in}{1.362924in}}%
\pgfpathlineto{\pgfqpoint{3.585245in}{1.341039in}}%
\pgfpathlineto{\pgfqpoint{3.590433in}{1.189215in}}%
\pgfpathlineto{\pgfqpoint{3.591790in}{1.187932in}}%
\pgfpathlineto{\pgfqpoint{3.593150in}{1.202318in}}%
\pgfpathlineto{\pgfqpoint{3.597270in}{1.298953in}}%
\pgfpathlineto{\pgfqpoint{3.600195in}{1.360963in}}%
\pgfpathlineto{\pgfqpoint{3.601601in}{1.372869in}}%
\pgfpathlineto{\pgfqpoint{3.602996in}{1.367428in}}%
\pgfpathlineto{\pgfqpoint{3.604385in}{1.341829in}}%
\pgfpathlineto{\pgfqpoint{3.608907in}{1.190710in}}%
\pgfpathlineto{\pgfqpoint{3.610328in}{1.178723in}}%
\pgfpathlineto{\pgfqpoint{3.611749in}{1.185553in}}%
\pgfpathlineto{\pgfqpoint{3.614602in}{1.240123in}}%
\pgfpathlineto{\pgfqpoint{3.620534in}{1.374686in}}%
\pgfpathlineto{\pgfqpoint{3.622001in}{1.379853in}}%
\pgfpathlineto{\pgfqpoint{3.623459in}{1.364852in}}%
\pgfpathlineto{\pgfqpoint{3.626254in}{1.270606in}}%
\pgfpathlineto{\pgfqpoint{3.628338in}{1.196624in}}%
\pgfpathlineto{\pgfqpoint{3.629831in}{1.173204in}}%
\pgfpathlineto{\pgfqpoint{3.631321in}{1.171747in}}%
\pgfpathlineto{\pgfqpoint{3.632813in}{1.188570in}}%
\pgfpathlineto{\pgfqpoint{3.637330in}{1.301360in}}%
\pgfpathlineto{\pgfqpoint{3.640553in}{1.374125in}}%
\pgfpathlineto{\pgfqpoint{3.642100in}{1.388008in}}%
\pgfpathlineto{\pgfqpoint{3.643634in}{1.381542in}}%
\pgfpathlineto{\pgfqpoint{3.645161in}{1.351453in}}%
\pgfpathlineto{\pgfqpoint{3.650665in}{1.166755in}}%
\pgfpathlineto{\pgfqpoint{3.652232in}{1.160602in}}%
\pgfpathlineto{\pgfqpoint{3.653799in}{1.175206in}}%
\pgfpathlineto{\pgfqpoint{3.656948in}{1.247266in}}%
\pgfpathlineto{\pgfqpoint{3.661882in}{1.376017in}}%
\pgfpathlineto{\pgfqpoint{3.663514in}{1.395669in}}%
\pgfpathlineto{\pgfqpoint{3.665131in}{1.394110in}}%
\pgfpathlineto{\pgfqpoint{3.666740in}{1.367545in}}%
\pgfpathlineto{\pgfqpoint{3.672701in}{1.158805in}}%
\pgfpathlineto{\pgfqpoint{3.674354in}{1.148925in}}%
\pgfpathlineto{\pgfqpoint{3.676006in}{1.162292in}}%
\pgfpathlineto{\pgfqpoint{3.679322in}{1.237950in}}%
\pgfpathlineto{\pgfqpoint{3.684506in}{1.380514in}}%
\pgfpathlineto{\pgfqpoint{3.686234in}{1.404763in}}%
\pgfpathlineto{\pgfqpoint{3.687943in}{1.406399in}}%
\pgfpathlineto{\pgfqpoint{3.689642in}{1.381151in}}%
\pgfpathlineto{\pgfqpoint{3.692729in}{1.257636in}}%
\pgfpathlineto{\pgfqpoint{3.695957in}{1.150087in}}%
\pgfpathlineto{\pgfqpoint{3.697705in}{1.135903in}}%
\pgfpathlineto{\pgfqpoint{3.699452in}{1.147821in}}%
\pgfpathlineto{\pgfqpoint{3.702956in}{1.227525in}}%
\pgfpathlineto{\pgfqpoint{3.708433in}{1.385898in}}%
\pgfpathlineto{\pgfqpoint{3.710267in}{1.415179in}}%
\pgfpathlineto{\pgfqpoint{3.712078in}{1.420176in}}%
\pgfpathlineto{\pgfqpoint{3.713878in}{1.396089in}}%
\pgfpathlineto{\pgfqpoint{3.717392in}{1.253436in}}%
\pgfpathlineto{\pgfqpoint{3.720970in}{1.133618in}}%
\pgfpathlineto{\pgfqpoint{3.722825in}{1.121151in}}%
\pgfpathlineto{\pgfqpoint{3.724677in}{1.136929in}}%
\pgfpathlineto{\pgfqpoint{3.728394in}{1.228006in}}%
\pgfpathlineto{\pgfqpoint{3.734214in}{1.401383in}}%
\pgfpathlineto{\pgfqpoint{3.736160in}{1.431405in}}%
\pgfpathlineto{\pgfqpoint{3.738083in}{1.433970in}}%
\pgfpathlineto{\pgfqpoint{3.739994in}{1.403895in}}%
\pgfpathlineto{\pgfqpoint{3.743468in}{1.254486in}}%
\pgfpathlineto{\pgfqpoint{3.746926in}{1.125220in}}%
\pgfpathlineto{\pgfqpoint{3.748903in}{1.103975in}}%
\pgfpathlineto{\pgfqpoint{3.750876in}{1.115376in}}%
\pgfpathlineto{\pgfqpoint{3.754829in}{1.208980in}}%
\pgfpathlineto{\pgfqpoint{3.761023in}{1.405295in}}%
\pgfpathlineto{\pgfqpoint{3.763108in}{1.444171in}}%
\pgfpathlineto{\pgfqpoint{3.765162in}{1.453737in}}%
\pgfpathlineto{\pgfqpoint{3.767202in}{1.428022in}}%
\pgfpathlineto{\pgfqpoint{3.771020in}{1.269481in}}%
\pgfpathlineto{\pgfqpoint{3.774372in}{1.122768in}}%
\pgfpathlineto{\pgfqpoint{3.776491in}{1.086769in}}%
\pgfpathlineto{\pgfqpoint{3.778599in}{1.089247in}}%
\pgfpathlineto{\pgfqpoint{3.780707in}{1.123219in}}%
\pgfpathlineto{\pgfqpoint{3.784939in}{1.252053in}}%
\pgfpathlineto{\pgfqpoint{3.791724in}{1.455591in}}%
\pgfpathlineto{\pgfqpoint{3.793930in}{1.475795in}}%
\pgfpathlineto{\pgfqpoint{3.796117in}{1.458113in}}%
\pgfpathlineto{\pgfqpoint{3.798292in}{1.396996in}}%
\pgfpathlineto{\pgfqpoint{3.805270in}{1.082368in}}%
\pgfpathlineto{\pgfqpoint{3.807538in}{1.059649in}}%
\pgfpathlineto{\pgfqpoint{3.809800in}{1.077274in}}%
\pgfpathlineto{\pgfqpoint{3.814330in}{1.198847in}}%
\pgfpathlineto{\pgfqpoint{3.821442in}{1.444529in}}%
\pgfpathlineto{\pgfqpoint{3.823840in}{1.491074in}}%
\pgfpathlineto{\pgfqpoint{3.826203in}{1.499816in}}%
\pgfpathlineto{\pgfqpoint{3.828549in}{1.463367in}}%
\pgfpathlineto{\pgfqpoint{3.832821in}{1.262927in}}%
\pgfpathlineto{\pgfqpoint{3.835178in}{1.133848in}}%
\pgfpathlineto{\pgfqpoint{3.837671in}{1.055211in}}%
\pgfpathlineto{\pgfqpoint{3.840116in}{1.031326in}}%
\pgfpathlineto{\pgfqpoint{3.842553in}{1.052815in}}%
\pgfpathlineto{\pgfqpoint{3.847433in}{1.192315in}}%
\pgfpathlineto{\pgfqpoint{3.855104in}{1.470296in}}%
\pgfpathlineto{\pgfqpoint{3.857694in}{1.521984in}}%
\pgfpathlineto{\pgfqpoint{3.860245in}{1.530334in}}%
\pgfpathlineto{\pgfqpoint{3.862778in}{1.487029in}}%
\pgfpathlineto{\pgfqpoint{3.867312in}{1.260882in}}%
\pgfpathlineto{\pgfqpoint{3.869956in}{1.109849in}}%
\pgfpathlineto{\pgfqpoint{3.872654in}{1.022673in}}%
\pgfpathlineto{\pgfqpoint{3.875302in}{0.997622in}}%
\pgfpathlineto{\pgfqpoint{3.877939in}{1.023945in}}%
\pgfpathlineto{\pgfqpoint{3.883221in}{1.185238in}}%
\pgfpathlineto{\pgfqpoint{3.891533in}{1.502064in}}%
\pgfpathlineto{\pgfqpoint{3.894341in}{1.559768in}}%
\pgfpathlineto{\pgfqpoint{3.897107in}{1.567414in}}%
\pgfpathlineto{\pgfqpoint{3.899853in}{1.515564in}}%
\pgfpathlineto{\pgfqpoint{3.904646in}{1.261332in}}%
\pgfpathlineto{\pgfqpoint{3.906374in}{1.147564in}}%
\pgfpathlineto{\pgfqpoint{3.909404in}{1.014055in}}%
\pgfpathlineto{\pgfqpoint{3.912293in}{0.960501in}}%
\pgfpathlineto{\pgfqpoint{3.915163in}{0.970295in}}%
\pgfpathlineto{\pgfqpoint{3.918030in}{1.031361in}}%
\pgfpathlineto{\pgfqpoint{3.923776in}{1.250625in}}%
\pgfpathlineto{\pgfqpoint{3.929982in}{1.505395in}}%
\pgfpathlineto{\pgfqpoint{3.933073in}{1.588586in}}%
\pgfpathlineto{\pgfqpoint{3.936098in}{1.617741in}}%
\pgfpathlineto{\pgfqpoint{3.939094in}{1.581534in}}%
\pgfpathlineto{\pgfqpoint{3.942073in}{1.470951in}}%
\pgfpathlineto{\pgfqpoint{3.948279in}{1.067116in}}%
\pgfpathlineto{\pgfqpoint{3.951502in}{0.946268in}}%
\pgfpathlineto{\pgfqpoint{3.954646in}{0.909380in}}%
\pgfpathlineto{\pgfqpoint{3.957775in}{0.941366in}}%
\pgfpathlineto{\pgfqpoint{3.960903in}{1.027729in}}%
\pgfpathlineto{\pgfqpoint{3.977258in}{1.657222in}}%
\pgfpathlineto{\pgfqpoint{3.980553in}{1.671151in}}%
\pgfpathlineto{\pgfqpoint{3.983821in}{1.605860in}}%
\pgfpathlineto{\pgfqpoint{3.989008in}{1.315276in}}%
\pgfpathlineto{\pgfqpoint{3.993208in}{1.022037in}}%
\pgfpathlineto{\pgfqpoint{3.996722in}{0.889851in}}%
\pgfpathlineto{\pgfqpoint{4.000161in}{0.853418in}}%
\pgfpathlineto{\pgfqpoint{4.003584in}{0.895709in}}%
\pgfpathlineto{\pgfqpoint{4.007005in}{0.999784in}}%
\pgfpathlineto{\pgfqpoint{4.024899in}{1.726746in}}%
\pgfpathlineto{\pgfqpoint{4.028503in}{1.737444in}}%
\pgfpathlineto{\pgfqpoint{4.032078in}{1.654969in}}%
\pgfpathlineto{\pgfqpoint{4.037754in}{1.305080in}}%
\pgfpathlineto{\pgfqpoint{4.042101in}{0.985868in}}%
\pgfpathlineto{\pgfqpoint{4.045946in}{0.832231in}}%
\pgfpathlineto{\pgfqpoint{4.049709in}{0.789578in}}%
\pgfpathlineto{\pgfqpoint{4.053454in}{0.838243in}}%
\pgfpathlineto{\pgfqpoint{4.057197in}{0.958713in}}%
\pgfpathlineto{\pgfqpoint{4.076763in}{1.804432in}}%
\pgfpathlineto{\pgfqpoint{4.080701in}{1.817798in}}%
\pgfpathlineto{\pgfqpoint{4.084609in}{1.722899in}}%
\pgfpathlineto{\pgfqpoint{4.090813in}{1.317359in}}%
\pgfpathlineto{\pgfqpoint{4.096076in}{0.919330in}}%
\pgfpathlineto{\pgfqpoint{4.100235in}{0.759212in}}%
\pgfpathlineto{\pgfqpoint{4.104329in}{0.722966in}}%
\pgfpathlineto{\pgfqpoint{4.108409in}{0.789350in}}%
\pgfpathlineto{\pgfqpoint{4.112489in}{0.935830in}}%
\pgfpathlineto{\pgfqpoint{4.129459in}{1.788788in}}%
\pgfpathlineto{\pgfqpoint{4.133788in}{1.898784in}}%
\pgfpathlineto{\pgfqpoint{4.138058in}{1.901997in}}%
\pgfpathlineto{\pgfqpoint{4.142299in}{1.778653in}}%
\pgfpathlineto{\pgfqpoint{4.147778in}{1.407328in}}%
\pgfpathlineto{\pgfqpoint{4.152904in}{0.965796in}}%
\pgfpathlineto{\pgfqpoint{4.157436in}{0.737834in}}%
\pgfpathlineto{\pgfqpoint{4.161855in}{0.658023in}}%
\pgfpathlineto{\pgfqpoint{4.166255in}{0.700817in}}%
\pgfpathlineto{\pgfqpoint{4.170656in}{0.842297in}}%
\pgfpathlineto{\pgfqpoint{4.179501in}{1.306935in}}%
\pgfpathlineto{\pgfqpoint{4.188886in}{1.810243in}}%
\pgfpathlineto{\pgfqpoint{4.193541in}{1.963799in}}%
\pgfpathlineto{\pgfqpoint{4.198122in}{2.004926in}}%
\pgfpathlineto{\pgfqpoint{4.202671in}{1.909544in}}%
\pgfpathlineto{\pgfqpoint{4.208553in}{1.552290in}}%
\pgfpathlineto{\pgfqpoint{4.217671in}{0.804054in}}%
\pgfpathlineto{\pgfqpoint{4.222346in}{0.639021in}}%
\pgfpathlineto{\pgfqpoint{4.227002in}{0.618943in}}%
\pgfpathlineto{\pgfqpoint{4.231659in}{0.720867in}}%
\pgfpathlineto{\pgfqpoint{4.236327in}{0.917333in}}%
\pgfpathlineto{\pgfqpoint{4.255690in}{1.965398in}}%
\pgfpathlineto{\pgfqpoint{4.260502in}{2.076655in}}%
\pgfpathlineto{\pgfqpoint{4.265276in}{2.053519in}}%
\pgfpathlineto{\pgfqpoint{4.270031in}{1.871418in}}%
\pgfpathlineto{\pgfqpoint{4.276930in}{1.292440in}}%
\pgfpathlineto{\pgfqpoint{4.278162in}{1.174044in}}%
\pgfpathlineto{\pgfqpoint{4.283111in}{0.813508in}}%
\pgfpathlineto{\pgfqpoint{4.287887in}{0.633814in}}%
\pgfpathlineto{\pgfqpoint{4.292672in}{0.604364in}}%
\pgfpathlineto{\pgfqpoint{4.297468in}{0.703782in}}%
\pgfpathlineto{\pgfqpoint{4.302284in}{0.904549in}}%
\pgfpathlineto{\pgfqpoint{4.322086in}{1.997870in}}%
\pgfpathlineto{\pgfqpoint{4.326960in}{2.119774in}}%
\pgfpathlineto{\pgfqpoint{4.331806in}{2.105646in}}%
\pgfpathlineto{\pgfqpoint{4.336640in}{1.928675in}}%
\pgfpathlineto{\pgfqpoint{4.344453in}{1.266305in}}%
\pgfpathlineto{\pgfqpoint{4.348621in}{0.930811in}}%
\pgfpathlineto{\pgfqpoint{4.353321in}{0.705247in}}%
\pgfpathlineto{\pgfqpoint{4.358068in}{0.629816in}}%
\pgfpathlineto{\pgfqpoint{4.362841in}{0.688897in}}%
\pgfpathlineto{\pgfqpoint{4.367642in}{0.858334in}}%
\pgfpathlineto{\pgfqpoint{4.377385in}{1.402066in}}%
\pgfpathlineto{\pgfqpoint{4.387287in}{1.946533in}}%
\pgfpathlineto{\pgfqpoint{4.392067in}{2.100088in}}%
\pgfpathlineto{\pgfqpoint{4.396828in}{2.131636in}}%
\pgfpathlineto{\pgfqpoint{4.401582in}{2.012052in}}%
\pgfpathlineto{\pgfqpoint{4.407767in}{1.595433in}}%
\pgfpathlineto{\pgfqpoint{4.411368in}{1.240560in}}%
\pgfpathlineto{\pgfqpoint{4.415769in}{0.922586in}}%
\pgfpathlineto{\pgfqpoint{4.420284in}{0.743264in}}%
\pgfpathlineto{\pgfqpoint{4.424861in}{0.700708in}}%
\pgfpathlineto{\pgfqpoint{4.429475in}{0.780083in}}%
\pgfpathlineto{\pgfqpoint{4.434125in}{0.958027in}}%
\pgfpathlineto{\pgfqpoint{4.453045in}{1.969326in}}%
\pgfpathlineto{\pgfqpoint{4.457584in}{2.087328in}}%
\pgfpathlineto{\pgfqpoint{4.462118in}{2.086166in}}%
\pgfpathlineto{\pgfqpoint{4.466653in}{1.939556in}}%
\pgfpathlineto{\pgfqpoint{4.473570in}{1.398377in}}%
\pgfpathlineto{\pgfqpoint{4.477940in}{1.054790in}}%
\pgfpathlineto{\pgfqpoint{4.482148in}{0.860230in}}%
\pgfpathlineto{\pgfqpoint{4.486432in}{0.789761in}}%
\pgfpathlineto{\pgfqpoint{4.490764in}{0.833419in}}%
\pgfpathlineto{\pgfqpoint{4.495134in}{0.972434in}}%
\pgfpathlineto{\pgfqpoint{4.504040in}{1.431614in}}%
\pgfpathlineto{\pgfqpoint{4.512918in}{1.886318in}}%
\pgfpathlineto{\pgfqpoint{4.517144in}{2.013978in}}%
\pgfpathlineto{\pgfqpoint{4.521371in}{2.042749in}}%
\pgfpathlineto{\pgfqpoint{4.525601in}{1.947421in}}%
\pgfpathlineto{\pgfqpoint{4.531116in}{1.607337in}}%
\pgfpathlineto{\pgfqpoint{4.537895in}{1.077124in}}%
\pgfpathlineto{\pgfqpoint{4.541800in}{0.926705in}}%
\pgfpathlineto{\pgfqpoint{4.545768in}{0.884123in}}%
\pgfpathlineto{\pgfqpoint{4.549782in}{0.939139in}}%
\pgfpathlineto{\pgfqpoint{4.553833in}{1.074273in}}%
\pgfpathlineto{\pgfqpoint{4.570250in}{1.869479in}}%
\pgfpathlineto{\pgfqpoint{4.574136in}{1.965226in}}%
\pgfpathlineto{\pgfqpoint{4.578026in}{1.970306in}}%
\pgfpathlineto{\pgfqpoint{4.581921in}{1.863592in}}%
\pgfpathlineto{\pgfqpoint{4.588158in}{1.428324in}}%
\pgfpathlineto{\pgfqpoint{4.592813in}{1.120626in}}%
\pgfpathlineto{\pgfqpoint{4.596395in}{1.000451in}}%
\pgfpathlineto{\pgfqpoint{4.600037in}{0.972892in}}%
\pgfpathlineto{\pgfqpoint{4.603720in}{1.028483in}}%
\pgfpathlineto{\pgfqpoint{4.607439in}{1.151546in}}%
\pgfpathlineto{\pgfqpoint{4.622419in}{1.832440in}}%
\pgfpathlineto{\pgfqpoint{4.625973in}{1.907925in}}%
\pgfpathlineto{\pgfqpoint{4.629532in}{1.903355in}}%
\pgfpathlineto{\pgfqpoint{4.633096in}{1.800974in}}%
\pgfpathlineto{\pgfqpoint{4.645674in}{1.078355in}}%
\pgfpathlineto{\pgfqpoint{4.648997in}{1.047620in}}%
\pgfpathlineto{\pgfqpoint{4.652360in}{1.089496in}}%
\pgfpathlineto{\pgfqpoint{4.655755in}{1.190870in}}%
\pgfpathlineto{\pgfqpoint{4.669487in}{1.779834in}}%
\pgfpathlineto{\pgfqpoint{4.672733in}{1.849280in}}%
\pgfpathlineto{\pgfqpoint{4.675984in}{1.851295in}}%
\pgfpathlineto{\pgfqpoint{4.679241in}{1.770359in}}%
\pgfpathlineto{\pgfqpoint{4.684616in}{1.433703in}}%
\pgfpathlineto{\pgfqpoint{4.689681in}{1.170879in}}%
\pgfpathlineto{\pgfqpoint{4.692699in}{1.114567in}}%
\pgfpathlineto{\pgfqpoint{4.695759in}{1.124417in}}%
\pgfpathlineto{\pgfqpoint{4.698850in}{1.191026in}}%
\pgfpathlineto{\pgfqpoint{4.705146in}{1.437660in}}%
\pgfpathlineto{\pgfqpoint{4.711422in}{1.699879in}}%
\pgfpathlineto{\pgfqpoint{4.714392in}{1.780907in}}%
\pgfpathlineto{\pgfqpoint{4.717367in}{1.810798in}}%
\pgfpathlineto{\pgfqpoint{4.720346in}{1.774869in}}%
\pgfpathlineto{\pgfqpoint{4.725126in}{1.553331in}}%
\pgfpathlineto{\pgfqpoint{4.729588in}{1.275725in}}%
\pgfpathlineto{\pgfqpoint{4.732327in}{1.189366in}}%
\pgfpathlineto{\pgfqpoint{4.735112in}{1.163345in}}%
\pgfpathlineto{\pgfqpoint{4.737929in}{1.192045in}}%
\pgfpathlineto{\pgfqpoint{4.740774in}{1.265743in}}%
\pgfpathlineto{\pgfqpoint{4.755038in}{1.760215in}}%
\pgfpathlineto{\pgfqpoint{4.757772in}{1.764421in}}%
\pgfpathlineto{\pgfqpoint{4.760509in}{1.707233in}}%
\pgfpathlineto{\pgfqpoint{4.765009in}{1.461454in}}%
\pgfpathlineto{\pgfqpoint{4.768437in}{1.283740in}}%
\pgfpathlineto{\pgfqpoint{4.770968in}{1.219896in}}%
\pgfpathlineto{\pgfqpoint{4.773539in}{1.207960in}}%
\pgfpathlineto{\pgfqpoint{4.776138in}{1.242225in}}%
\pgfpathlineto{\pgfqpoint{4.781419in}{1.410114in}}%
\pgfpathlineto{\pgfqpoint{4.789307in}{1.691703in}}%
\pgfpathlineto{\pgfqpoint{4.791827in}{1.731366in}}%
\pgfpathlineto{\pgfqpoint{4.794350in}{1.725425in}}%
\pgfpathlineto{\pgfqpoint{4.796875in}{1.664275in}}%
\pgfpathlineto{\pgfqpoint{4.806267in}{1.250798in}}%
\pgfpathlineto{\pgfqpoint{4.808648in}{1.244266in}}%
\pgfpathlineto{\pgfqpoint{4.811053in}{1.277706in}}%
\pgfpathlineto{\pgfqpoint{4.815939in}{1.429025in}}%
\pgfpathlineto{\pgfqpoint{4.823232in}{1.672749in}}%
\pgfpathlineto{\pgfqpoint{4.825568in}{1.704350in}}%
\pgfpathlineto{\pgfqpoint{4.827905in}{1.694969in}}%
\pgfpathlineto{\pgfqpoint{4.830245in}{1.636410in}}%
\pgfpathlineto{\pgfqpoint{4.837127in}{1.309029in}}%
\pgfpathlineto{\pgfqpoint{4.839316in}{1.273735in}}%
\pgfpathlineto{\pgfqpoint{4.841533in}{1.277687in}}%
\pgfpathlineto{\pgfqpoint{4.843772in}{1.315309in}}%
\pgfpathlineto{\pgfqpoint{4.848324in}{1.458066in}}%
\pgfpathlineto{\pgfqpoint{4.855063in}{1.662530in}}%
\pgfpathlineto{\pgfqpoint{4.857238in}{1.682088in}}%
\pgfpathlineto{\pgfqpoint{4.859413in}{1.663300in}}%
\pgfpathlineto{\pgfqpoint{4.861591in}{1.599464in}}%
\pgfpathlineto{\pgfqpoint{4.867146in}{1.337128in}}%
\pgfpathlineto{\pgfqpoint{4.869183in}{1.300087in}}%
\pgfpathlineto{\pgfqpoint{4.871248in}{1.298582in}}%
\pgfpathlineto{\pgfqpoint{4.873333in}{1.327993in}}%
\pgfpathlineto{\pgfqpoint{4.877569in}{1.450615in}}%
\pgfpathlineto{\pgfqpoint{4.883878in}{1.639025in}}%
\pgfpathlineto{\pgfqpoint{4.885910in}{1.661103in}}%
\pgfpathlineto{\pgfqpoint{4.887942in}{1.650042in}}%
\pgfpathlineto{\pgfqpoint{4.889977in}{1.599609in}}%
\pgfpathlineto{\pgfqpoint{4.896183in}{1.337109in}}%
\pgfpathlineto{\pgfqpoint{4.898103in}{1.315856in}}%
\pgfpathlineto{\pgfqpoint{4.900044in}{1.325270in}}%
\pgfpathlineto{\pgfqpoint{4.903981in}{1.414372in}}%
\pgfpathlineto{\pgfqpoint{4.911887in}{1.634653in}}%
\pgfpathlineto{\pgfqpoint{4.913794in}{1.643983in}}%
\pgfpathlineto{\pgfqpoint{4.915701in}{1.621241in}}%
\pgfpathlineto{\pgfqpoint{4.919287in}{1.476727in}}%
\pgfpathlineto{\pgfqpoint{4.921562in}{1.380657in}}%
\pgfpathlineto{\pgfqpoint{4.923348in}{1.342049in}}%
\pgfpathlineto{\pgfqpoint{4.925161in}{1.332945in}}%
\pgfpathlineto{\pgfqpoint{4.926992in}{1.350170in}}%
\pgfpathlineto{\pgfqpoint{4.930705in}{1.441643in}}%
\pgfpathlineto{\pgfqpoint{4.936329in}{1.602697in}}%
\pgfpathlineto{\pgfqpoint{4.938126in}{1.626793in}}%
\pgfpathlineto{\pgfqpoint{4.939921in}{1.625554in}}%
\pgfpathlineto{\pgfqpoint{4.941717in}{1.593582in}}%
\pgfpathlineto{\pgfqpoint{4.948095in}{1.358334in}}%
\pgfpathlineto{\pgfqpoint{4.949804in}{1.347126in}}%
\pgfpathlineto{\pgfqpoint{4.951529in}{1.360260in}}%
\pgfpathlineto{\pgfqpoint{4.955026in}{1.440178in}}%
\pgfpathlineto{\pgfqpoint{4.960351in}{1.588871in}}%
\pgfpathlineto{\pgfqpoint{4.962049in}{1.612969in}}%
\pgfpathlineto{\pgfqpoint{4.963746in}{1.614638in}}%
\pgfpathlineto{\pgfqpoint{4.965442in}{1.588856in}}%
\pgfpathlineto{\pgfqpoint{4.973127in}{1.359528in}}%
\pgfpathlineto{\pgfqpoint{4.974759in}{1.369597in}}%
\pgfpathlineto{\pgfqpoint{4.978064in}{1.440085in}}%
\pgfpathlineto{\pgfqpoint{4.983107in}{1.577232in}}%
\pgfpathlineto{\pgfqpoint{4.984717in}{1.600966in}}%
\pgfpathlineto{\pgfqpoint{4.986324in}{1.604702in}}%
\pgfpathlineto{\pgfqpoint{4.987932in}{1.583759in}}%
\pgfpathlineto{\pgfqpoint{4.995553in}{1.370328in}}%
\pgfpathlineto{\pgfqpoint{4.997103in}{1.381699in}}%
\pgfpathlineto{\pgfqpoint{5.000243in}{1.448999in}}%
\pgfpathlineto{\pgfqpoint{5.005029in}{1.573242in}}%
\pgfpathlineto{\pgfqpoint{5.006559in}{1.593178in}}%
\pgfpathlineto{\pgfqpoint{5.008087in}{1.594201in}}%
\pgfpathlineto{\pgfqpoint{5.009614in}{1.572149in}}%
\pgfpathlineto{\pgfqpoint{5.016628in}{1.379570in}}%
\pgfpathlineto{\pgfqpoint{5.018103in}{1.389328in}}%
\pgfpathlineto{\pgfqpoint{5.021089in}{1.450408in}}%
\pgfpathlineto{\pgfqpoint{5.025647in}{1.565466in}}%
\pgfpathlineto{\pgfqpoint{5.027105in}{1.584505in}}%
\pgfpathlineto{\pgfqpoint{5.028561in}{1.586288in}}%
\pgfpathlineto{\pgfqpoint{5.030015in}{1.566943in}}%
\pgfpathlineto{\pgfqpoint{5.036011in}{1.389913in}}%
\pgfpathlineto{\pgfqpoint{5.037412in}{1.389842in}}%
\pgfpathlineto{\pgfqpoint{5.040246in}{1.434230in}}%
\pgfpathlineto{\pgfqpoint{5.045981in}{1.569054in}}%
\pgfpathlineto{\pgfqpoint{5.047372in}{1.580207in}}%
\pgfpathlineto{\pgfqpoint{5.048762in}{1.573713in}}%
\pgfpathlineto{\pgfqpoint{5.050152in}{1.546396in}}%
\pgfpathlineto{\pgfqpoint{5.054746in}{1.401645in}}%
\pgfpathlineto{\pgfqpoint{5.056080in}{1.394877in}}%
\pgfpathlineto{\pgfqpoint{5.057426in}{1.404147in}}%
\pgfpathlineto{\pgfqpoint{5.060150in}{1.457659in}}%
\pgfpathlineto{\pgfqpoint{5.065642in}{1.571410in}}%
\pgfpathlineto{\pgfqpoint{5.066972in}{1.571958in}}%
\pgfpathlineto{\pgfqpoint{5.068302in}{1.554172in}}%
\pgfpathlineto{\pgfqpoint{5.073449in}{1.404980in}}%
\pgfpathlineto{\pgfqpoint{5.074730in}{1.401734in}}%
\pgfpathlineto{\pgfqpoint{5.076021in}{1.412945in}}%
\pgfpathlineto{\pgfqpoint{5.079976in}{1.499649in}}%
\pgfpathlineto{\pgfqpoint{5.083867in}{1.566830in}}%
\pgfpathlineto{\pgfqpoint{5.085143in}{1.564587in}}%
\pgfpathlineto{\pgfqpoint{5.086418in}{1.544811in}}%
\pgfpathlineto{\pgfqpoint{5.091714in}{1.407030in}}%
\pgfpathlineto{\pgfqpoint{5.092948in}{1.410641in}}%
\pgfpathlineto{\pgfqpoint{5.095446in}{1.451484in}}%
\pgfpathlineto{\pgfqpoint{5.100497in}{1.556984in}}%
\pgfpathlineto{\pgfqpoint{5.101723in}{1.562464in}}%
\pgfpathlineto{\pgfqpoint{5.102948in}{1.552884in}}%
\pgfpathlineto{\pgfqpoint{5.105471in}{1.477581in}}%
\pgfpathlineto{\pgfqpoint{5.107518in}{1.422174in}}%
\pgfpathlineto{\pgfqpoint{5.108694in}{1.411935in}}%
\pgfpathlineto{\pgfqpoint{5.109882in}{1.415430in}}%
\pgfpathlineto{\pgfqpoint{5.112283in}{1.453844in}}%
\pgfpathlineto{\pgfqpoint{5.117143in}{1.552625in}}%
\pgfpathlineto{\pgfqpoint{5.118323in}{1.557656in}}%
\pgfpathlineto{\pgfqpoint{5.119502in}{1.548541in}}%
\pgfpathlineto{\pgfqpoint{5.122032in}{1.473804in}}%
\pgfpathlineto{\pgfqpoint{5.124243in}{1.421491in}}%
\pgfpathlineto{\pgfqpoint{5.125380in}{1.416060in}}%
\pgfpathlineto{\pgfqpoint{5.126527in}{1.422963in}}%
\pgfpathlineto{\pgfqpoint{5.128846in}{1.463685in}}%
\pgfpathlineto{\pgfqpoint{5.133533in}{1.551408in}}%
\pgfpathlineto{\pgfqpoint{5.134670in}{1.552107in}}%
\pgfpathlineto{\pgfqpoint{5.135807in}{1.538761in}}%
\pgfpathlineto{\pgfqpoint{5.140986in}{1.419995in}}%
\pgfpathlineto{\pgfqpoint{5.142091in}{1.425302in}}%
\pgfpathlineto{\pgfqpoint{5.144326in}{1.462158in}}%
\pgfpathlineto{\pgfqpoint{5.148849in}{1.546743in}}%
\pgfpathlineto{\pgfqpoint{5.149947in}{1.548787in}}%
\pgfpathlineto{\pgfqpoint{5.151044in}{1.537832in}}%
\pgfpathlineto{\pgfqpoint{5.156290in}{1.423605in}}%
\pgfpathlineto{\pgfqpoint{5.157358in}{1.429732in}}%
\pgfpathlineto{\pgfqpoint{5.160622in}{1.490323in}}%
\pgfpathlineto{\pgfqpoint{5.163890in}{1.544040in}}%
\pgfpathlineto{\pgfqpoint{5.164951in}{1.544695in}}%
\pgfpathlineto{\pgfqpoint{5.166011in}{1.532846in}}%
\pgfpathlineto{\pgfqpoint{5.171001in}{1.426876in}}%
\pgfpathlineto{\pgfqpoint{5.172034in}{1.432991in}}%
\pgfpathlineto{\pgfqpoint{5.175191in}{1.490744in}}%
\pgfpathlineto{\pgfqpoint{5.178353in}{1.541018in}}%
\pgfpathlineto{\pgfqpoint{5.179380in}{1.541278in}}%
\pgfpathlineto{\pgfqpoint{5.180406in}{1.529639in}}%
\pgfpathlineto{\pgfqpoint{5.185143in}{1.429747in}}%
\pgfpathlineto{\pgfqpoint{5.186144in}{1.435002in}}%
\pgfpathlineto{\pgfqpoint{5.189199in}{1.488995in}}%
\pgfpathlineto{\pgfqpoint{5.192265in}{1.537709in}}%
\pgfpathlineto{\pgfqpoint{5.193260in}{1.538592in}}%
\pgfpathlineto{\pgfqpoint{5.194254in}{1.528294in}}%
\pgfpathlineto{\pgfqpoint{5.198892in}{1.432407in}}%
\pgfpathlineto{\pgfqpoint{5.199862in}{1.437257in}}%
\pgfpathlineto{\pgfqpoint{5.202823in}{1.488329in}}%
\pgfpathlineto{\pgfqpoint{5.205797in}{1.534896in}}%
\pgfpathlineto{\pgfqpoint{5.206763in}{1.535907in}}%
\pgfpathlineto{\pgfqpoint{5.207727in}{1.526319in}}%
\pgfpathlineto{\pgfqpoint{5.212668in}{1.435893in}}%
\pgfpathlineto{\pgfqpoint{5.214564in}{1.458494in}}%
\pgfpathlineto{\pgfqpoint{5.219354in}{1.533936in}}%
\pgfpathlineto{\pgfqpoint{5.220291in}{1.530730in}}%
\pgfpathlineto{\pgfqpoint{5.222397in}{1.482790in}}%
\pgfpathlineto{\pgfqpoint{5.224371in}{1.440572in}}%
\pgfpathlineto{\pgfqpoint{5.225280in}{1.437109in}}%
\pgfpathlineto{\pgfqpoint{5.226195in}{1.442084in}}%
\pgfpathlineto{\pgfqpoint{5.228989in}{1.489193in}}%
\pgfpathlineto{\pgfqpoint{5.231793in}{1.530427in}}%
\pgfpathlineto{\pgfqpoint{5.232705in}{1.530686in}}%
\pgfpathlineto{\pgfqpoint{5.234525in}{1.500760in}}%
\pgfpathlineto{\pgfqpoint{5.237202in}{1.440894in}}%
\pgfpathlineto{\pgfqpoint{5.238088in}{1.439575in}}%
\pgfpathlineto{\pgfqpoint{5.239876in}{1.458441in}}%
\pgfpathlineto{\pgfqpoint{5.244406in}{1.529002in}}%
\pgfpathlineto{\pgfqpoint{5.245292in}{1.527398in}}%
\pgfpathlineto{\pgfqpoint{5.247062in}{1.494343in}}%
\pgfpathlineto{\pgfqpoint{5.249294in}{1.444064in}}%
\pgfpathlineto{\pgfqpoint{5.250154in}{1.441025in}}%
\pgfpathlineto{\pgfqpoint{5.251021in}{1.445666in}}%
\pgfpathlineto{\pgfqpoint{5.253665in}{1.488780in}}%
\pgfpathlineto{\pgfqpoint{5.256321in}{1.526274in}}%
\pgfpathlineto{\pgfqpoint{5.257184in}{1.526404in}}%
\pgfpathlineto{\pgfqpoint{5.258908in}{1.498793in}}%
\pgfpathlineto{\pgfqpoint{5.261434in}{1.444349in}}%
\pgfpathlineto{\pgfqpoint{5.262273in}{1.443118in}}%
\pgfpathlineto{\pgfqpoint{5.263969in}{1.460343in}}%
\pgfpathlineto{\pgfqpoint{5.268265in}{1.525002in}}%
\pgfpathlineto{\pgfqpoint{5.269106in}{1.523569in}}%
\pgfpathlineto{\pgfqpoint{5.270786in}{1.493364in}}%
\pgfpathlineto{\pgfqpoint{5.272898in}{1.447207in}}%
\pgfpathlineto{\pgfqpoint{5.273715in}{1.444318in}}%
\pgfpathlineto{\pgfqpoint{5.275366in}{1.458334in}}%
\pgfpathlineto{\pgfqpoint{5.279573in}{1.522542in}}%
\pgfpathlineto{\pgfqpoint{5.280394in}{1.522772in}}%
\pgfpathlineto{\pgfqpoint{5.282032in}{1.497669in}}%
\pgfpathlineto{\pgfqpoint{5.284296in}{1.448305in}}%
\pgfpathlineto{\pgfqpoint{5.285093in}{1.445818in}}%
\pgfpathlineto{\pgfqpoint{5.286704in}{1.459689in}}%
\pgfpathlineto{\pgfqpoint{5.290809in}{1.521013in}}%
\pgfpathlineto{\pgfqpoint{5.291609in}{1.520957in}}%
\pgfpathlineto{\pgfqpoint{5.293208in}{1.496173in}}%
\pgfpathlineto{\pgfqpoint{5.295733in}{1.447638in}}%
\pgfpathlineto{\pgfqpoint{5.296515in}{1.448260in}}%
\pgfpathlineto{\pgfqpoint{5.298092in}{1.465864in}}%
\pgfpathlineto{\pgfqpoint{5.302083in}{1.520276in}}%
\pgfpathlineto{\pgfqpoint{5.302865in}{1.517097in}}%
\pgfpathlineto{\pgfqpoint{5.304798in}{1.474861in}}%
\pgfpathlineto{\pgfqpoint{5.307050in}{1.448547in}}%
\pgfpathlineto{\pgfqpoint{5.308587in}{1.462243in}}%
\pgfpathlineto{\pgfqpoint{5.312499in}{1.518213in}}%
\pgfpathlineto{\pgfqpoint{5.313263in}{1.517590in}}%
\pgfpathlineto{\pgfqpoint{5.314788in}{1.493263in}}%
\pgfpathlineto{\pgfqpoint{5.316688in}{1.452785in}}%
\pgfpathlineto{\pgfqpoint{5.317431in}{1.449528in}}%
\pgfpathlineto{\pgfqpoint{5.318178in}{1.452415in}}%
\pgfpathlineto{\pgfqpoint{5.320457in}{1.485109in}}%
\pgfpathlineto{\pgfqpoint{5.322758in}{1.515983in}}%
\pgfpathlineto{\pgfqpoint{5.323506in}{1.516986in}}%
\pgfpathlineto{\pgfqpoint{5.324998in}{1.497452in}}%
\pgfpathlineto{\pgfqpoint{5.327819in}{1.450629in}}%
\pgfpathlineto{\pgfqpoint{5.329287in}{1.461674in}}%
\pgfpathlineto{\pgfqpoint{5.333764in}{1.515481in}}%
\pgfpathlineto{\pgfqpoint{5.335223in}{1.495566in}}%
\pgfpathlineto{\pgfqpoint{5.337986in}{1.451690in}}%
\pgfpathlineto{\pgfqpoint{5.339424in}{1.463012in}}%
\pgfpathlineto{\pgfqpoint{5.343805in}{1.514019in}}%
\pgfpathlineto{\pgfqpoint{5.345234in}{1.493699in}}%
\pgfpathlineto{\pgfqpoint{5.347238in}{1.454368in}}%
\pgfpathlineto{\pgfqpoint{5.347936in}{1.452701in}}%
\pgfpathlineto{\pgfqpoint{5.349344in}{1.464222in}}%
\pgfpathlineto{\pgfqpoint{5.352931in}{1.512950in}}%
\pgfpathlineto{\pgfqpoint{5.353632in}{1.512624in}}%
\pgfpathlineto{\pgfqpoint{5.355031in}{1.492029in}}%
\pgfpathlineto{\pgfqpoint{5.356908in}{1.455468in}}%
\pgfpathlineto{\pgfqpoint{5.357591in}{1.453528in}}%
\pgfpathlineto{\pgfqpoint{5.358970in}{1.464171in}}%
\pgfpathlineto{\pgfqpoint{5.363174in}{1.511689in}}%
\pgfpathlineto{\pgfqpoint{5.364545in}{1.492565in}}%
\pgfpathlineto{\pgfqpoint{5.366481in}{1.455855in}}%
\pgfpathlineto{\pgfqpoint{5.367151in}{1.454436in}}%
\pgfpathlineto{\pgfqpoint{5.368504in}{1.465393in}}%
\pgfpathlineto{\pgfqpoint{5.371948in}{1.510804in}}%
\pgfpathlineto{\pgfqpoint{5.372621in}{1.510377in}}%
\pgfpathlineto{\pgfqpoint{5.374167in}{1.485467in}}%
\pgfpathlineto{\pgfqpoint{5.375819in}{1.456535in}}%
\pgfpathlineto{\pgfqpoint{5.376476in}{1.455223in}}%
\pgfpathlineto{\pgfqpoint{5.377802in}{1.465909in}}%
\pgfpathlineto{\pgfqpoint{5.381179in}{1.509806in}}%
\pgfpathlineto{\pgfqpoint{5.381839in}{1.509339in}}%
\pgfpathlineto{\pgfqpoint{5.383356in}{1.485048in}}%
\pgfpathlineto{\pgfqpoint{5.384882in}{1.457785in}}%
\pgfpathlineto{\pgfqpoint{5.385526in}{1.455828in}}%
\pgfpathlineto{\pgfqpoint{5.386826in}{1.465159in}}%
\pgfpathlineto{\pgfqpoint{5.390791in}{1.508748in}}%
\pgfpathlineto{\pgfqpoint{5.392085in}{1.491875in}}%
\pgfpathlineto{\pgfqpoint{5.394577in}{1.456615in}}%
\pgfpathlineto{\pgfqpoint{5.395854in}{1.466568in}}%
\pgfpathlineto{\pgfqpoint{5.399106in}{1.507882in}}%
\pgfpathlineto{\pgfqpoint{5.399743in}{1.507497in}}%
\pgfpathlineto{\pgfqpoint{5.401261in}{1.483198in}}%
\pgfpathlineto{\pgfqpoint{5.403202in}{1.457128in}}%
\pgfpathlineto{\pgfqpoint{5.403826in}{1.459057in}}%
\pgfpathlineto{\pgfqpoint{5.405725in}{1.482875in}}%
\pgfpathlineto{\pgfqpoint{5.408274in}{1.507244in}}%
\pgfpathlineto{\pgfqpoint{5.409522in}{1.493282in}}%
\pgfpathlineto{\pgfqpoint{5.411966in}{1.457767in}}%
\pgfpathlineto{\pgfqpoint{5.413196in}{1.466404in}}%
\pgfpathlineto{\pgfqpoint{5.416950in}{1.506055in}}%
\pgfpathlineto{\pgfqpoint{5.418359in}{1.485980in}}%
\pgfpathlineto{\pgfqpoint{5.420006in}{1.459169in}}%
\pgfpathlineto{\pgfqpoint{5.420606in}{1.458579in}}%
\pgfpathlineto{\pgfqpoint{5.421816in}{1.468472in}}%
\pgfpathlineto{\pgfqpoint{5.424890in}{1.505522in}}%
\pgfpathlineto{\pgfqpoint{5.425494in}{1.504642in}}%
\pgfpathlineto{\pgfqpoint{5.426972in}{1.479820in}}%
\pgfpathlineto{\pgfqpoint{5.428424in}{1.459435in}}%
\pgfpathlineto{\pgfqpoint{5.429014in}{1.459271in}}%
\pgfpathlineto{\pgfqpoint{5.430204in}{1.469451in}}%
\pgfpathlineto{\pgfqpoint{5.433229in}{1.504860in}}%
\pgfpathlineto{\pgfqpoint{5.433822in}{1.503527in}}%
\pgfpathlineto{\pgfqpoint{5.435405in}{1.475337in}}%
\pgfpathlineto{\pgfqpoint{5.437127in}{1.459513in}}%
\pgfpathlineto{\pgfqpoint{5.438296in}{1.468423in}}%
\pgfpathlineto{\pgfqpoint{5.441272in}{1.503892in}}%
\pgfpathlineto{\pgfqpoint{5.441856in}{1.503348in}}%
\pgfpathlineto{\pgfqpoint{5.443291in}{1.480362in}}%
\pgfpathlineto{\pgfqpoint{5.445089in}{1.459834in}}%
\pgfpathlineto{\pgfqpoint{5.446238in}{1.467287in}}%
\pgfpathlineto{\pgfqpoint{5.449745in}{1.503095in}}%
\pgfpathlineto{\pgfqpoint{5.451062in}{1.485463in}}%
\pgfpathlineto{\pgfqpoint{5.453094in}{1.460336in}}%
\pgfpathlineto{\pgfqpoint{5.454224in}{1.468153in}}%
\pgfpathlineto{\pgfqpoint{5.457108in}{1.502311in}}%
\pgfpathlineto{\pgfqpoint{5.457674in}{1.502212in}}%
\pgfpathlineto{\pgfqpoint{5.459021in}{1.482573in}}%
\pgfpathlineto{\pgfqpoint{5.460732in}{1.460724in}}%
\pgfpathlineto{\pgfqpoint{5.461286in}{1.461994in}}%
\pgfpathlineto{\pgfqpoint{5.462968in}{1.480942in}}%
\pgfpathlineto{\pgfqpoint{5.465237in}{1.502050in}}%
\pgfpathlineto{\pgfqpoint{5.466347in}{1.491280in}}%
\pgfpathlineto{\pgfqpoint{5.468694in}{1.461292in}}%
\pgfpathlineto{\pgfqpoint{5.469790in}{1.469412in}}%
\pgfpathlineto{\pgfqpoint{5.472577in}{1.501212in}}%
\pgfpathlineto{\pgfqpoint{5.473125in}{1.500622in}}%
\pgfpathlineto{\pgfqpoint{5.474473in}{1.479305in}}%
\pgfpathlineto{\pgfqpoint{5.476230in}{1.461580in}}%
\pgfpathlineto{\pgfqpoint{5.477308in}{1.468995in}}%
\pgfpathlineto{\pgfqpoint{5.480057in}{1.500468in}}%
\pgfpathlineto{\pgfqpoint{5.480597in}{1.500224in}}%
\pgfpathlineto{\pgfqpoint{5.481961in}{1.479261in}}%
\pgfpathlineto{\pgfqpoint{5.483646in}{1.461868in}}%
\pgfpathlineto{\pgfqpoint{5.484706in}{1.468508in}}%
\pgfpathlineto{\pgfqpoint{5.487947in}{1.499850in}}%
\pgfpathlineto{\pgfqpoint{5.489215in}{1.482341in}}%
\pgfpathlineto{\pgfqpoint{5.490955in}{1.462180in}}%
\pgfpathlineto{\pgfqpoint{5.491998in}{1.468072in}}%
\pgfpathlineto{\pgfqpoint{5.495192in}{1.499444in}}%
\pgfpathlineto{\pgfqpoint{5.496395in}{1.484397in}}%
\pgfpathlineto{\pgfqpoint{5.497884in}{1.463072in}}%
\pgfpathlineto{\pgfqpoint{5.498395in}{1.462740in}}%
\pgfpathlineto{\pgfqpoint{5.499424in}{1.470130in}}%
\pgfpathlineto{\pgfqpoint{5.502043in}{1.498910in}}%
\pgfpathlineto{\pgfqpoint{5.502561in}{1.498215in}}%
\pgfpathlineto{\pgfqpoint{5.503851in}{1.477183in}}%
\pgfpathlineto{\pgfqpoint{5.505188in}{1.462956in}}%
\pgfpathlineto{\pgfqpoint{5.505692in}{1.463475in}}%
\pgfpathlineto{\pgfqpoint{5.507217in}{1.478293in}}%
\pgfpathlineto{\pgfqpoint{5.509306in}{1.498572in}}%
\pgfpathlineto{\pgfqpoint{5.509817in}{1.496571in}}%
\pgfpathlineto{\pgfqpoint{5.512642in}{1.463414in}}%
\pgfpathlineto{\pgfqpoint{5.513139in}{1.465893in}}%
\pgfpathlineto{\pgfqpoint{5.516187in}{1.497926in}}%
\pgfpathlineto{\pgfqpoint{5.516693in}{1.496862in}}%
\pgfpathlineto{\pgfqpoint{5.519471in}{1.463441in}}%
\pgfpathlineto{\pgfqpoint{5.519958in}{1.465103in}}%
\pgfpathlineto{\pgfqpoint{5.521926in}{1.486980in}}%
\pgfpathlineto{\pgfqpoint{5.522996in}{1.497112in}}%
\pgfpathlineto{\pgfqpoint{5.523504in}{1.496772in}}%
\pgfpathlineto{\pgfqpoint{5.526177in}{1.463646in}}%
\pgfpathlineto{\pgfqpoint{5.527126in}{1.466207in}}%
\pgfpathlineto{\pgfqpoint{5.529041in}{1.484525in}}%
\pgfpathlineto{\pgfqpoint{5.530065in}{1.496675in}}%
\pgfpathlineto{\pgfqpoint{5.531524in}{1.410995in}}%
\pgfpathlineto{\pgfqpoint{5.532296in}{1.414723in}}%
\pgfpathlineto{\pgfqpoint{5.533393in}{1.375101in}}%
\pgfpathlineto{\pgfqpoint{5.534545in}{1.307253in}}%
\pgfpathlineto{\pgfqpoint{5.534545in}{1.307253in}}%
\pgfusepath{stroke}%
\end{pgfscope}%
\begin{pgfscope}%
\pgfsetrectcap%
\pgfsetmiterjoin%
\pgfsetlinewidth{0.803000pt}%
\definecolor{currentstroke}{rgb}{0.000000,0.000000,0.000000}%
\pgfsetstrokecolor{currentstroke}%
\pgfsetdash{}{0pt}%
\pgfpathmoveto{\pgfqpoint{0.800000in}{0.528000in}}%
\pgfpathlineto{\pgfqpoint{0.800000in}{2.208000in}}%
\pgfusepath{stroke}%
\end{pgfscope}%
\begin{pgfscope}%
\pgfsetrectcap%
\pgfsetmiterjoin%
\pgfsetlinewidth{0.803000pt}%
\definecolor{currentstroke}{rgb}{0.000000,0.000000,0.000000}%
\pgfsetstrokecolor{currentstroke}%
\pgfsetdash{}{0pt}%
\pgfpathmoveto{\pgfqpoint{5.760000in}{0.528000in}}%
\pgfpathlineto{\pgfqpoint{5.760000in}{2.208000in}}%
\pgfusepath{stroke}%
\end{pgfscope}%
\begin{pgfscope}%
\pgfsetrectcap%
\pgfsetmiterjoin%
\pgfsetlinewidth{0.803000pt}%
\definecolor{currentstroke}{rgb}{0.000000,0.000000,0.000000}%
\pgfsetstrokecolor{currentstroke}%
\pgfsetdash{}{0pt}%
\pgfpathmoveto{\pgfqpoint{0.800000in}{0.528000in}}%
\pgfpathlineto{\pgfqpoint{5.760000in}{0.528000in}}%
\pgfusepath{stroke}%
\end{pgfscope}%
\begin{pgfscope}%
\pgfsetrectcap%
\pgfsetmiterjoin%
\pgfsetlinewidth{0.803000pt}%
\definecolor{currentstroke}{rgb}{0.000000,0.000000,0.000000}%
\pgfsetstrokecolor{currentstroke}%
\pgfsetdash{}{0pt}%
\pgfpathmoveto{\pgfqpoint{0.800000in}{2.208000in}}%
\pgfpathlineto{\pgfqpoint{5.760000in}{2.208000in}}%
\pgfusepath{stroke}%
\end{pgfscope}%
\end{pgfpicture}%
\makeatother%
\endgroup%

}
\caption{Difference between analytical and numerical value for the inclination in the combined $a$ and $i$ transfer orbit for the case $i_0 = 90~\text{deg}$}
\label{fig:aincdiffinc90}
\end{figure}

% Please add the following required packages to your document preamble:
% \usepackage{multirow}
\begin{table}%[b]
\centering
\begin{tabular}{|l|l|l|l|l|l|l|}
\hline
\multirow{2}{*}{} & \multicolumn{3}{l|}{\textbf{Time of flight $t_f$}}   & \multicolumn{3}{l|}{\textbf{Cost $\Delta V$}}      \\ \cline{2-7} 
                  & Expected    & Computed        & $\varepsilon$       & Expected  & Computed        & $\varepsilon$       \\ \hline
\textbf{Case 1}   & $191.26295$ & $191.262282913$ & $0.35 \cdot 10^{-5}$ & $5.78378$ & $5.7837714353$  & $0.15 \cdot 10^{-5}$ \\ \hline
\textbf{Case 2}   & $335.0$     & $335.033933749$ & $0.10 \cdot 10^{-3}$ & $10.13$   & $10.1314261566$ & $0.14 \cdot 10^{-3}$ \\ \hline
\textbf{Case 3}   & $351.0$     & $351.211665646$ & $0.60 \cdot 10^{-3}$ & $10.61$   & $10.6206407691$ & $0.10 \cdot 10^{-2}$ \\ \hline
\end{tabular}
\caption{Analytical results of the combined semimajor axis and inclination change.}
\label{tab:aincanares}
\end{table}

\subsection{Combined eccentricity and inclination change}

For each of the data points extracted from the plots in \cite{pollard2000simplified}, we were able to recover the expected values of yaw angle $\beta$ and cost $\Delta V$ with a relative error of $10^{-2}$. This case is even more challenging since the precision of the extracted data is limited by the resolution of the plots, even with the help of an automated software. Again, for improving the accuracy more numerical test cases would be needed.

The results of the integration of the third case ($e_0 = 0.4, i_f = 20.0~\text{deg}$) display similar accuracy: the expected final eccentricity and inclination are recovered with a relative error of $10^{-2}$ and $10^{-1}$ respectively. These tolerances are the worst for all the guidance laws in study, although still acceptable. The reverse change in eccentricity has been studied as well to test the algorithm against an initial circular orbit, yielding the same results.

We reproduce here the charts that can be found in \cite{pollard2000simplified} using our own formulas, where it can be seen that the general trend of the plots is respected (figure~\ref{fig:eccincnumcharts}). 

\begin{figure}%[h]
\begin{subfigure}[b]{0.5\textwidth}
\centering
\resizebox{1.0\textwidth}{!}{
%% Creator: Matplotlib, PGF backend
%%
%% To include the figure in your LaTeX document, write
%%   \input{<filename>.pgf}
%%
%% Make sure the required packages are loaded in your preamble
%%   \usepackage{pgf}
%%
%% Figures using additional raster images can only be included by \input if
%% they are in the same directory as the main LaTeX file. For loading figures
%% from other directories you can use the `import` package
%%   \usepackage{import}
%% and then include the figures with
%%   \import{<path to file>}{<filename>.pgf}
%%
%% Matplotlib used the following preamble
%%   \usepackage{fontspec}
%%
\begingroup%
\makeatletter%
\begin{pgfpicture}%
\pgfpathrectangle{\pgfpointorigin}{\pgfqpoint{6.000000in}{6.000000in}}%
\pgfusepath{use as bounding box, clip}%
\begin{pgfscope}%
\pgfsetbuttcap%
\pgfsetmiterjoin%
\definecolor{currentfill}{rgb}{1.000000,1.000000,1.000000}%
\pgfsetfillcolor{currentfill}%
\pgfsetlinewidth{0.000000pt}%
\definecolor{currentstroke}{rgb}{1.000000,1.000000,1.000000}%
\pgfsetstrokecolor{currentstroke}%
\pgfsetdash{}{0pt}%
\pgfpathmoveto{\pgfqpoint{0.000000in}{0.000000in}}%
\pgfpathlineto{\pgfqpoint{6.000000in}{0.000000in}}%
\pgfpathlineto{\pgfqpoint{6.000000in}{6.000000in}}%
\pgfpathlineto{\pgfqpoint{0.000000in}{6.000000in}}%
\pgfpathclose%
\pgfusepath{fill}%
\end{pgfscope}%
\begin{pgfscope}%
\pgfsetbuttcap%
\pgfsetmiterjoin%
\definecolor{currentfill}{rgb}{1.000000,1.000000,1.000000}%
\pgfsetfillcolor{currentfill}%
\pgfsetlinewidth{0.000000pt}%
\definecolor{currentstroke}{rgb}{0.000000,0.000000,0.000000}%
\pgfsetstrokecolor{currentstroke}%
\pgfsetstrokeopacity{0.000000}%
\pgfsetdash{}{0pt}%
\pgfpathmoveto{\pgfqpoint{0.750000in}{0.660000in}}%
\pgfpathlineto{\pgfqpoint{5.400000in}{0.660000in}}%
\pgfpathlineto{\pgfqpoint{5.400000in}{5.280000in}}%
\pgfpathlineto{\pgfqpoint{0.750000in}{5.280000in}}%
\pgfpathclose%
\pgfusepath{fill}%
\end{pgfscope}%
\begin{pgfscope}%
\pgfpathrectangle{\pgfqpoint{0.750000in}{0.660000in}}{\pgfqpoint{4.650000in}{4.620000in}} %
\pgfusepath{clip}%
\pgfsetrectcap%
\pgfsetroundjoin%
\pgfsetlinewidth{0.803000pt}%
\definecolor{currentstroke}{rgb}{0.690196,0.690196,0.690196}%
\pgfsetstrokecolor{currentstroke}%
\pgfsetdash{}{0pt}%
\pgfpathmoveto{\pgfqpoint{0.750000in}{0.660000in}}%
\pgfpathlineto{\pgfqpoint{0.750000in}{5.280000in}}%
\pgfusepath{stroke}%
\end{pgfscope}%
\begin{pgfscope}%
\pgfsetbuttcap%
\pgfsetroundjoin%
\definecolor{currentfill}{rgb}{0.000000,0.000000,0.000000}%
\pgfsetfillcolor{currentfill}%
\pgfsetlinewidth{0.803000pt}%
\definecolor{currentstroke}{rgb}{0.000000,0.000000,0.000000}%
\pgfsetstrokecolor{currentstroke}%
\pgfsetdash{}{0pt}%
\pgfsys@defobject{currentmarker}{\pgfqpoint{0.000000in}{-0.048611in}}{\pgfqpoint{0.000000in}{0.000000in}}{%
\pgfpathmoveto{\pgfqpoint{0.000000in}{0.000000in}}%
\pgfpathlineto{\pgfqpoint{0.000000in}{-0.048611in}}%
\pgfusepath{stroke,fill}%
}%
\begin{pgfscope}%
\pgfsys@transformshift{0.750000in}{0.660000in}%
\pgfsys@useobject{currentmarker}{}%
\end{pgfscope}%
\end{pgfscope}%
\begin{pgfscope}%
\pgftext[x=0.750000in,y=0.562778in,,top]{\sffamily\fontsize{10.000000}{12.000000}\selectfont \(\displaystyle 0\)}%
\end{pgfscope}%
\begin{pgfscope}%
\pgfpathrectangle{\pgfqpoint{0.750000in}{0.660000in}}{\pgfqpoint{4.650000in}{4.620000in}} %
\pgfusepath{clip}%
\pgfsetrectcap%
\pgfsetroundjoin%
\pgfsetlinewidth{0.803000pt}%
\definecolor{currentstroke}{rgb}{0.690196,0.690196,0.690196}%
\pgfsetstrokecolor{currentstroke}%
\pgfsetdash{}{0pt}%
\pgfpathmoveto{\pgfqpoint{1.525000in}{0.660000in}}%
\pgfpathlineto{\pgfqpoint{1.525000in}{5.280000in}}%
\pgfusepath{stroke}%
\end{pgfscope}%
\begin{pgfscope}%
\pgfsetbuttcap%
\pgfsetroundjoin%
\definecolor{currentfill}{rgb}{0.000000,0.000000,0.000000}%
\pgfsetfillcolor{currentfill}%
\pgfsetlinewidth{0.803000pt}%
\definecolor{currentstroke}{rgb}{0.000000,0.000000,0.000000}%
\pgfsetstrokecolor{currentstroke}%
\pgfsetdash{}{0pt}%
\pgfsys@defobject{currentmarker}{\pgfqpoint{0.000000in}{-0.048611in}}{\pgfqpoint{0.000000in}{0.000000in}}{%
\pgfpathmoveto{\pgfqpoint{0.000000in}{0.000000in}}%
\pgfpathlineto{\pgfqpoint{0.000000in}{-0.048611in}}%
\pgfusepath{stroke,fill}%
}%
\begin{pgfscope}%
\pgfsys@transformshift{1.525000in}{0.660000in}%
\pgfsys@useobject{currentmarker}{}%
\end{pgfscope}%
\end{pgfscope}%
\begin{pgfscope}%
\pgftext[x=1.525000in,y=0.562778in,,top]{\sffamily\fontsize{10.000000}{12.000000}\selectfont \(\displaystyle 5\)}%
\end{pgfscope}%
\begin{pgfscope}%
\pgfpathrectangle{\pgfqpoint{0.750000in}{0.660000in}}{\pgfqpoint{4.650000in}{4.620000in}} %
\pgfusepath{clip}%
\pgfsetrectcap%
\pgfsetroundjoin%
\pgfsetlinewidth{0.803000pt}%
\definecolor{currentstroke}{rgb}{0.690196,0.690196,0.690196}%
\pgfsetstrokecolor{currentstroke}%
\pgfsetdash{}{0pt}%
\pgfpathmoveto{\pgfqpoint{2.300000in}{0.660000in}}%
\pgfpathlineto{\pgfqpoint{2.300000in}{5.280000in}}%
\pgfusepath{stroke}%
\end{pgfscope}%
\begin{pgfscope}%
\pgfsetbuttcap%
\pgfsetroundjoin%
\definecolor{currentfill}{rgb}{0.000000,0.000000,0.000000}%
\pgfsetfillcolor{currentfill}%
\pgfsetlinewidth{0.803000pt}%
\definecolor{currentstroke}{rgb}{0.000000,0.000000,0.000000}%
\pgfsetstrokecolor{currentstroke}%
\pgfsetdash{}{0pt}%
\pgfsys@defobject{currentmarker}{\pgfqpoint{0.000000in}{-0.048611in}}{\pgfqpoint{0.000000in}{0.000000in}}{%
\pgfpathmoveto{\pgfqpoint{0.000000in}{0.000000in}}%
\pgfpathlineto{\pgfqpoint{0.000000in}{-0.048611in}}%
\pgfusepath{stroke,fill}%
}%
\begin{pgfscope}%
\pgfsys@transformshift{2.300000in}{0.660000in}%
\pgfsys@useobject{currentmarker}{}%
\end{pgfscope}%
\end{pgfscope}%
\begin{pgfscope}%
\pgftext[x=2.300000in,y=0.562778in,,top]{\sffamily\fontsize{10.000000}{12.000000}\selectfont \(\displaystyle 10\)}%
\end{pgfscope}%
\begin{pgfscope}%
\pgfpathrectangle{\pgfqpoint{0.750000in}{0.660000in}}{\pgfqpoint{4.650000in}{4.620000in}} %
\pgfusepath{clip}%
\pgfsetrectcap%
\pgfsetroundjoin%
\pgfsetlinewidth{0.803000pt}%
\definecolor{currentstroke}{rgb}{0.690196,0.690196,0.690196}%
\pgfsetstrokecolor{currentstroke}%
\pgfsetdash{}{0pt}%
\pgfpathmoveto{\pgfqpoint{3.075000in}{0.660000in}}%
\pgfpathlineto{\pgfqpoint{3.075000in}{5.280000in}}%
\pgfusepath{stroke}%
\end{pgfscope}%
\begin{pgfscope}%
\pgfsetbuttcap%
\pgfsetroundjoin%
\definecolor{currentfill}{rgb}{0.000000,0.000000,0.000000}%
\pgfsetfillcolor{currentfill}%
\pgfsetlinewidth{0.803000pt}%
\definecolor{currentstroke}{rgb}{0.000000,0.000000,0.000000}%
\pgfsetstrokecolor{currentstroke}%
\pgfsetdash{}{0pt}%
\pgfsys@defobject{currentmarker}{\pgfqpoint{0.000000in}{-0.048611in}}{\pgfqpoint{0.000000in}{0.000000in}}{%
\pgfpathmoveto{\pgfqpoint{0.000000in}{0.000000in}}%
\pgfpathlineto{\pgfqpoint{0.000000in}{-0.048611in}}%
\pgfusepath{stroke,fill}%
}%
\begin{pgfscope}%
\pgfsys@transformshift{3.075000in}{0.660000in}%
\pgfsys@useobject{currentmarker}{}%
\end{pgfscope}%
\end{pgfscope}%
\begin{pgfscope}%
\pgftext[x=3.075000in,y=0.562778in,,top]{\sffamily\fontsize{10.000000}{12.000000}\selectfont \(\displaystyle 15\)}%
\end{pgfscope}%
\begin{pgfscope}%
\pgfpathrectangle{\pgfqpoint{0.750000in}{0.660000in}}{\pgfqpoint{4.650000in}{4.620000in}} %
\pgfusepath{clip}%
\pgfsetrectcap%
\pgfsetroundjoin%
\pgfsetlinewidth{0.803000pt}%
\definecolor{currentstroke}{rgb}{0.690196,0.690196,0.690196}%
\pgfsetstrokecolor{currentstroke}%
\pgfsetdash{}{0pt}%
\pgfpathmoveto{\pgfqpoint{3.850000in}{0.660000in}}%
\pgfpathlineto{\pgfqpoint{3.850000in}{5.280000in}}%
\pgfusepath{stroke}%
\end{pgfscope}%
\begin{pgfscope}%
\pgfsetbuttcap%
\pgfsetroundjoin%
\definecolor{currentfill}{rgb}{0.000000,0.000000,0.000000}%
\pgfsetfillcolor{currentfill}%
\pgfsetlinewidth{0.803000pt}%
\definecolor{currentstroke}{rgb}{0.000000,0.000000,0.000000}%
\pgfsetstrokecolor{currentstroke}%
\pgfsetdash{}{0pt}%
\pgfsys@defobject{currentmarker}{\pgfqpoint{0.000000in}{-0.048611in}}{\pgfqpoint{0.000000in}{0.000000in}}{%
\pgfpathmoveto{\pgfqpoint{0.000000in}{0.000000in}}%
\pgfpathlineto{\pgfqpoint{0.000000in}{-0.048611in}}%
\pgfusepath{stroke,fill}%
}%
\begin{pgfscope}%
\pgfsys@transformshift{3.850000in}{0.660000in}%
\pgfsys@useobject{currentmarker}{}%
\end{pgfscope}%
\end{pgfscope}%
\begin{pgfscope}%
\pgftext[x=3.850000in,y=0.562778in,,top]{\sffamily\fontsize{10.000000}{12.000000}\selectfont \(\displaystyle 20\)}%
\end{pgfscope}%
\begin{pgfscope}%
\pgfpathrectangle{\pgfqpoint{0.750000in}{0.660000in}}{\pgfqpoint{4.650000in}{4.620000in}} %
\pgfusepath{clip}%
\pgfsetrectcap%
\pgfsetroundjoin%
\pgfsetlinewidth{0.803000pt}%
\definecolor{currentstroke}{rgb}{0.690196,0.690196,0.690196}%
\pgfsetstrokecolor{currentstroke}%
\pgfsetdash{}{0pt}%
\pgfpathmoveto{\pgfqpoint{4.625000in}{0.660000in}}%
\pgfpathlineto{\pgfqpoint{4.625000in}{5.280000in}}%
\pgfusepath{stroke}%
\end{pgfscope}%
\begin{pgfscope}%
\pgfsetbuttcap%
\pgfsetroundjoin%
\definecolor{currentfill}{rgb}{0.000000,0.000000,0.000000}%
\pgfsetfillcolor{currentfill}%
\pgfsetlinewidth{0.803000pt}%
\definecolor{currentstroke}{rgb}{0.000000,0.000000,0.000000}%
\pgfsetstrokecolor{currentstroke}%
\pgfsetdash{}{0pt}%
\pgfsys@defobject{currentmarker}{\pgfqpoint{0.000000in}{-0.048611in}}{\pgfqpoint{0.000000in}{0.000000in}}{%
\pgfpathmoveto{\pgfqpoint{0.000000in}{0.000000in}}%
\pgfpathlineto{\pgfqpoint{0.000000in}{-0.048611in}}%
\pgfusepath{stroke,fill}%
}%
\begin{pgfscope}%
\pgfsys@transformshift{4.625000in}{0.660000in}%
\pgfsys@useobject{currentmarker}{}%
\end{pgfscope}%
\end{pgfscope}%
\begin{pgfscope}%
\pgftext[x=4.625000in,y=0.562778in,,top]{\sffamily\fontsize{10.000000}{12.000000}\selectfont \(\displaystyle 25\)}%
\end{pgfscope}%
\begin{pgfscope}%
\pgfpathrectangle{\pgfqpoint{0.750000in}{0.660000in}}{\pgfqpoint{4.650000in}{4.620000in}} %
\pgfusepath{clip}%
\pgfsetrectcap%
\pgfsetroundjoin%
\pgfsetlinewidth{0.803000pt}%
\definecolor{currentstroke}{rgb}{0.690196,0.690196,0.690196}%
\pgfsetstrokecolor{currentstroke}%
\pgfsetdash{}{0pt}%
\pgfpathmoveto{\pgfqpoint{5.400000in}{0.660000in}}%
\pgfpathlineto{\pgfqpoint{5.400000in}{5.280000in}}%
\pgfusepath{stroke}%
\end{pgfscope}%
\begin{pgfscope}%
\pgfsetbuttcap%
\pgfsetroundjoin%
\definecolor{currentfill}{rgb}{0.000000,0.000000,0.000000}%
\pgfsetfillcolor{currentfill}%
\pgfsetlinewidth{0.803000pt}%
\definecolor{currentstroke}{rgb}{0.000000,0.000000,0.000000}%
\pgfsetstrokecolor{currentstroke}%
\pgfsetdash{}{0pt}%
\pgfsys@defobject{currentmarker}{\pgfqpoint{0.000000in}{-0.048611in}}{\pgfqpoint{0.000000in}{0.000000in}}{%
\pgfpathmoveto{\pgfqpoint{0.000000in}{0.000000in}}%
\pgfpathlineto{\pgfqpoint{0.000000in}{-0.048611in}}%
\pgfusepath{stroke,fill}%
}%
\begin{pgfscope}%
\pgfsys@transformshift{5.400000in}{0.660000in}%
\pgfsys@useobject{currentmarker}{}%
\end{pgfscope}%
\end{pgfscope}%
\begin{pgfscope}%
\pgftext[x=5.400000in,y=0.562778in,,top]{\sffamily\fontsize{10.000000}{12.000000}\selectfont \(\displaystyle 30\)}%
\end{pgfscope}%
\begin{pgfscope}%
\pgftext[x=3.075000in,y=0.383889in,,top]{\sffamily\fontsize{10.000000}{12.000000}\selectfont Inclination change (deg)}%
\end{pgfscope}%
\begin{pgfscope}%
\pgfpathrectangle{\pgfqpoint{0.750000in}{0.660000in}}{\pgfqpoint{4.650000in}{4.620000in}} %
\pgfusepath{clip}%
\pgfsetrectcap%
\pgfsetroundjoin%
\pgfsetlinewidth{0.803000pt}%
\definecolor{currentstroke}{rgb}{0.690196,0.690196,0.690196}%
\pgfsetstrokecolor{currentstroke}%
\pgfsetdash{}{0pt}%
\pgfpathmoveto{\pgfqpoint{0.750000in}{0.660000in}}%
\pgfpathlineto{\pgfqpoint{5.400000in}{0.660000in}}%
\pgfusepath{stroke}%
\end{pgfscope}%
\begin{pgfscope}%
\pgfsetbuttcap%
\pgfsetroundjoin%
\definecolor{currentfill}{rgb}{0.000000,0.000000,0.000000}%
\pgfsetfillcolor{currentfill}%
\pgfsetlinewidth{0.803000pt}%
\definecolor{currentstroke}{rgb}{0.000000,0.000000,0.000000}%
\pgfsetstrokecolor{currentstroke}%
\pgfsetdash{}{0pt}%
\pgfsys@defobject{currentmarker}{\pgfqpoint{-0.048611in}{0.000000in}}{\pgfqpoint{0.000000in}{0.000000in}}{%
\pgfpathmoveto{\pgfqpoint{0.000000in}{0.000000in}}%
\pgfpathlineto{\pgfqpoint{-0.048611in}{0.000000in}}%
\pgfusepath{stroke,fill}%
}%
\begin{pgfscope}%
\pgfsys@transformshift{0.750000in}{0.660000in}%
\pgfsys@useobject{currentmarker}{}%
\end{pgfscope}%
\end{pgfscope}%
\begin{pgfscope}%
\pgftext[x=0.583333in,y=0.611806in,left,base]{\sffamily\fontsize{10.000000}{12.000000}\selectfont \(\displaystyle 0\)}%
\end{pgfscope}%
\begin{pgfscope}%
\pgfpathrectangle{\pgfqpoint{0.750000in}{0.660000in}}{\pgfqpoint{4.650000in}{4.620000in}} %
\pgfusepath{clip}%
\pgfsetrectcap%
\pgfsetroundjoin%
\pgfsetlinewidth{0.803000pt}%
\definecolor{currentstroke}{rgb}{0.690196,0.690196,0.690196}%
\pgfsetstrokecolor{currentstroke}%
\pgfsetdash{}{0pt}%
\pgfpathmoveto{\pgfqpoint{0.750000in}{1.173333in}}%
\pgfpathlineto{\pgfqpoint{5.400000in}{1.173333in}}%
\pgfusepath{stroke}%
\end{pgfscope}%
\begin{pgfscope}%
\pgfsetbuttcap%
\pgfsetroundjoin%
\definecolor{currentfill}{rgb}{0.000000,0.000000,0.000000}%
\pgfsetfillcolor{currentfill}%
\pgfsetlinewidth{0.803000pt}%
\definecolor{currentstroke}{rgb}{0.000000,0.000000,0.000000}%
\pgfsetstrokecolor{currentstroke}%
\pgfsetdash{}{0pt}%
\pgfsys@defobject{currentmarker}{\pgfqpoint{-0.048611in}{0.000000in}}{\pgfqpoint{0.000000in}{0.000000in}}{%
\pgfpathmoveto{\pgfqpoint{0.000000in}{0.000000in}}%
\pgfpathlineto{\pgfqpoint{-0.048611in}{0.000000in}}%
\pgfusepath{stroke,fill}%
}%
\begin{pgfscope}%
\pgfsys@transformshift{0.750000in}{1.173333in}%
\pgfsys@useobject{currentmarker}{}%
\end{pgfscope}%
\end{pgfscope}%
\begin{pgfscope}%
\pgftext[x=0.513888in,y=1.125139in,left,base]{\sffamily\fontsize{10.000000}{12.000000}\selectfont \(\displaystyle 10\)}%
\end{pgfscope}%
\begin{pgfscope}%
\pgfpathrectangle{\pgfqpoint{0.750000in}{0.660000in}}{\pgfqpoint{4.650000in}{4.620000in}} %
\pgfusepath{clip}%
\pgfsetrectcap%
\pgfsetroundjoin%
\pgfsetlinewidth{0.803000pt}%
\definecolor{currentstroke}{rgb}{0.690196,0.690196,0.690196}%
\pgfsetstrokecolor{currentstroke}%
\pgfsetdash{}{0pt}%
\pgfpathmoveto{\pgfqpoint{0.750000in}{1.686667in}}%
\pgfpathlineto{\pgfqpoint{5.400000in}{1.686667in}}%
\pgfusepath{stroke}%
\end{pgfscope}%
\begin{pgfscope}%
\pgfsetbuttcap%
\pgfsetroundjoin%
\definecolor{currentfill}{rgb}{0.000000,0.000000,0.000000}%
\pgfsetfillcolor{currentfill}%
\pgfsetlinewidth{0.803000pt}%
\definecolor{currentstroke}{rgb}{0.000000,0.000000,0.000000}%
\pgfsetstrokecolor{currentstroke}%
\pgfsetdash{}{0pt}%
\pgfsys@defobject{currentmarker}{\pgfqpoint{-0.048611in}{0.000000in}}{\pgfqpoint{0.000000in}{0.000000in}}{%
\pgfpathmoveto{\pgfqpoint{0.000000in}{0.000000in}}%
\pgfpathlineto{\pgfqpoint{-0.048611in}{0.000000in}}%
\pgfusepath{stroke,fill}%
}%
\begin{pgfscope}%
\pgfsys@transformshift{0.750000in}{1.686667in}%
\pgfsys@useobject{currentmarker}{}%
\end{pgfscope}%
\end{pgfscope}%
\begin{pgfscope}%
\pgftext[x=0.513888in,y=1.638472in,left,base]{\sffamily\fontsize{10.000000}{12.000000}\selectfont \(\displaystyle 20\)}%
\end{pgfscope}%
\begin{pgfscope}%
\pgfpathrectangle{\pgfqpoint{0.750000in}{0.660000in}}{\pgfqpoint{4.650000in}{4.620000in}} %
\pgfusepath{clip}%
\pgfsetrectcap%
\pgfsetroundjoin%
\pgfsetlinewidth{0.803000pt}%
\definecolor{currentstroke}{rgb}{0.690196,0.690196,0.690196}%
\pgfsetstrokecolor{currentstroke}%
\pgfsetdash{}{0pt}%
\pgfpathmoveto{\pgfqpoint{0.750000in}{2.200000in}}%
\pgfpathlineto{\pgfqpoint{5.400000in}{2.200000in}}%
\pgfusepath{stroke}%
\end{pgfscope}%
\begin{pgfscope}%
\pgfsetbuttcap%
\pgfsetroundjoin%
\definecolor{currentfill}{rgb}{0.000000,0.000000,0.000000}%
\pgfsetfillcolor{currentfill}%
\pgfsetlinewidth{0.803000pt}%
\definecolor{currentstroke}{rgb}{0.000000,0.000000,0.000000}%
\pgfsetstrokecolor{currentstroke}%
\pgfsetdash{}{0pt}%
\pgfsys@defobject{currentmarker}{\pgfqpoint{-0.048611in}{0.000000in}}{\pgfqpoint{0.000000in}{0.000000in}}{%
\pgfpathmoveto{\pgfqpoint{0.000000in}{0.000000in}}%
\pgfpathlineto{\pgfqpoint{-0.048611in}{0.000000in}}%
\pgfusepath{stroke,fill}%
}%
\begin{pgfscope}%
\pgfsys@transformshift{0.750000in}{2.200000in}%
\pgfsys@useobject{currentmarker}{}%
\end{pgfscope}%
\end{pgfscope}%
\begin{pgfscope}%
\pgftext[x=0.513888in,y=2.151806in,left,base]{\sffamily\fontsize{10.000000}{12.000000}\selectfont \(\displaystyle 30\)}%
\end{pgfscope}%
\begin{pgfscope}%
\pgfpathrectangle{\pgfqpoint{0.750000in}{0.660000in}}{\pgfqpoint{4.650000in}{4.620000in}} %
\pgfusepath{clip}%
\pgfsetrectcap%
\pgfsetroundjoin%
\pgfsetlinewidth{0.803000pt}%
\definecolor{currentstroke}{rgb}{0.690196,0.690196,0.690196}%
\pgfsetstrokecolor{currentstroke}%
\pgfsetdash{}{0pt}%
\pgfpathmoveto{\pgfqpoint{0.750000in}{2.713333in}}%
\pgfpathlineto{\pgfqpoint{5.400000in}{2.713333in}}%
\pgfusepath{stroke}%
\end{pgfscope}%
\begin{pgfscope}%
\pgfsetbuttcap%
\pgfsetroundjoin%
\definecolor{currentfill}{rgb}{0.000000,0.000000,0.000000}%
\pgfsetfillcolor{currentfill}%
\pgfsetlinewidth{0.803000pt}%
\definecolor{currentstroke}{rgb}{0.000000,0.000000,0.000000}%
\pgfsetstrokecolor{currentstroke}%
\pgfsetdash{}{0pt}%
\pgfsys@defobject{currentmarker}{\pgfqpoint{-0.048611in}{0.000000in}}{\pgfqpoint{0.000000in}{0.000000in}}{%
\pgfpathmoveto{\pgfqpoint{0.000000in}{0.000000in}}%
\pgfpathlineto{\pgfqpoint{-0.048611in}{0.000000in}}%
\pgfusepath{stroke,fill}%
}%
\begin{pgfscope}%
\pgfsys@transformshift{0.750000in}{2.713333in}%
\pgfsys@useobject{currentmarker}{}%
\end{pgfscope}%
\end{pgfscope}%
\begin{pgfscope}%
\pgftext[x=0.513888in,y=2.665139in,left,base]{\sffamily\fontsize{10.000000}{12.000000}\selectfont \(\displaystyle 40\)}%
\end{pgfscope}%
\begin{pgfscope}%
\pgfpathrectangle{\pgfqpoint{0.750000in}{0.660000in}}{\pgfqpoint{4.650000in}{4.620000in}} %
\pgfusepath{clip}%
\pgfsetrectcap%
\pgfsetroundjoin%
\pgfsetlinewidth{0.803000pt}%
\definecolor{currentstroke}{rgb}{0.690196,0.690196,0.690196}%
\pgfsetstrokecolor{currentstroke}%
\pgfsetdash{}{0pt}%
\pgfpathmoveto{\pgfqpoint{0.750000in}{3.226667in}}%
\pgfpathlineto{\pgfqpoint{5.400000in}{3.226667in}}%
\pgfusepath{stroke}%
\end{pgfscope}%
\begin{pgfscope}%
\pgfsetbuttcap%
\pgfsetroundjoin%
\definecolor{currentfill}{rgb}{0.000000,0.000000,0.000000}%
\pgfsetfillcolor{currentfill}%
\pgfsetlinewidth{0.803000pt}%
\definecolor{currentstroke}{rgb}{0.000000,0.000000,0.000000}%
\pgfsetstrokecolor{currentstroke}%
\pgfsetdash{}{0pt}%
\pgfsys@defobject{currentmarker}{\pgfqpoint{-0.048611in}{0.000000in}}{\pgfqpoint{0.000000in}{0.000000in}}{%
\pgfpathmoveto{\pgfqpoint{0.000000in}{0.000000in}}%
\pgfpathlineto{\pgfqpoint{-0.048611in}{0.000000in}}%
\pgfusepath{stroke,fill}%
}%
\begin{pgfscope}%
\pgfsys@transformshift{0.750000in}{3.226667in}%
\pgfsys@useobject{currentmarker}{}%
\end{pgfscope}%
\end{pgfscope}%
\begin{pgfscope}%
\pgftext[x=0.513888in,y=3.178472in,left,base]{\sffamily\fontsize{10.000000}{12.000000}\selectfont \(\displaystyle 50\)}%
\end{pgfscope}%
\begin{pgfscope}%
\pgfpathrectangle{\pgfqpoint{0.750000in}{0.660000in}}{\pgfqpoint{4.650000in}{4.620000in}} %
\pgfusepath{clip}%
\pgfsetrectcap%
\pgfsetroundjoin%
\pgfsetlinewidth{0.803000pt}%
\definecolor{currentstroke}{rgb}{0.690196,0.690196,0.690196}%
\pgfsetstrokecolor{currentstroke}%
\pgfsetdash{}{0pt}%
\pgfpathmoveto{\pgfqpoint{0.750000in}{3.740000in}}%
\pgfpathlineto{\pgfqpoint{5.400000in}{3.740000in}}%
\pgfusepath{stroke}%
\end{pgfscope}%
\begin{pgfscope}%
\pgfsetbuttcap%
\pgfsetroundjoin%
\definecolor{currentfill}{rgb}{0.000000,0.000000,0.000000}%
\pgfsetfillcolor{currentfill}%
\pgfsetlinewidth{0.803000pt}%
\definecolor{currentstroke}{rgb}{0.000000,0.000000,0.000000}%
\pgfsetstrokecolor{currentstroke}%
\pgfsetdash{}{0pt}%
\pgfsys@defobject{currentmarker}{\pgfqpoint{-0.048611in}{0.000000in}}{\pgfqpoint{0.000000in}{0.000000in}}{%
\pgfpathmoveto{\pgfqpoint{0.000000in}{0.000000in}}%
\pgfpathlineto{\pgfqpoint{-0.048611in}{0.000000in}}%
\pgfusepath{stroke,fill}%
}%
\begin{pgfscope}%
\pgfsys@transformshift{0.750000in}{3.740000in}%
\pgfsys@useobject{currentmarker}{}%
\end{pgfscope}%
\end{pgfscope}%
\begin{pgfscope}%
\pgftext[x=0.513888in,y=3.691806in,left,base]{\sffamily\fontsize{10.000000}{12.000000}\selectfont \(\displaystyle 60\)}%
\end{pgfscope}%
\begin{pgfscope}%
\pgfpathrectangle{\pgfqpoint{0.750000in}{0.660000in}}{\pgfqpoint{4.650000in}{4.620000in}} %
\pgfusepath{clip}%
\pgfsetrectcap%
\pgfsetroundjoin%
\pgfsetlinewidth{0.803000pt}%
\definecolor{currentstroke}{rgb}{0.690196,0.690196,0.690196}%
\pgfsetstrokecolor{currentstroke}%
\pgfsetdash{}{0pt}%
\pgfpathmoveto{\pgfqpoint{0.750000in}{4.253333in}}%
\pgfpathlineto{\pgfqpoint{5.400000in}{4.253333in}}%
\pgfusepath{stroke}%
\end{pgfscope}%
\begin{pgfscope}%
\pgfsetbuttcap%
\pgfsetroundjoin%
\definecolor{currentfill}{rgb}{0.000000,0.000000,0.000000}%
\pgfsetfillcolor{currentfill}%
\pgfsetlinewidth{0.803000pt}%
\definecolor{currentstroke}{rgb}{0.000000,0.000000,0.000000}%
\pgfsetstrokecolor{currentstroke}%
\pgfsetdash{}{0pt}%
\pgfsys@defobject{currentmarker}{\pgfqpoint{-0.048611in}{0.000000in}}{\pgfqpoint{0.000000in}{0.000000in}}{%
\pgfpathmoveto{\pgfqpoint{0.000000in}{0.000000in}}%
\pgfpathlineto{\pgfqpoint{-0.048611in}{0.000000in}}%
\pgfusepath{stroke,fill}%
}%
\begin{pgfscope}%
\pgfsys@transformshift{0.750000in}{4.253333in}%
\pgfsys@useobject{currentmarker}{}%
\end{pgfscope}%
\end{pgfscope}%
\begin{pgfscope}%
\pgftext[x=0.513888in,y=4.205139in,left,base]{\sffamily\fontsize{10.000000}{12.000000}\selectfont \(\displaystyle 70\)}%
\end{pgfscope}%
\begin{pgfscope}%
\pgfpathrectangle{\pgfqpoint{0.750000in}{0.660000in}}{\pgfqpoint{4.650000in}{4.620000in}} %
\pgfusepath{clip}%
\pgfsetrectcap%
\pgfsetroundjoin%
\pgfsetlinewidth{0.803000pt}%
\definecolor{currentstroke}{rgb}{0.690196,0.690196,0.690196}%
\pgfsetstrokecolor{currentstroke}%
\pgfsetdash{}{0pt}%
\pgfpathmoveto{\pgfqpoint{0.750000in}{4.766667in}}%
\pgfpathlineto{\pgfqpoint{5.400000in}{4.766667in}}%
\pgfusepath{stroke}%
\end{pgfscope}%
\begin{pgfscope}%
\pgfsetbuttcap%
\pgfsetroundjoin%
\definecolor{currentfill}{rgb}{0.000000,0.000000,0.000000}%
\pgfsetfillcolor{currentfill}%
\pgfsetlinewidth{0.803000pt}%
\definecolor{currentstroke}{rgb}{0.000000,0.000000,0.000000}%
\pgfsetstrokecolor{currentstroke}%
\pgfsetdash{}{0pt}%
\pgfsys@defobject{currentmarker}{\pgfqpoint{-0.048611in}{0.000000in}}{\pgfqpoint{0.000000in}{0.000000in}}{%
\pgfpathmoveto{\pgfqpoint{0.000000in}{0.000000in}}%
\pgfpathlineto{\pgfqpoint{-0.048611in}{0.000000in}}%
\pgfusepath{stroke,fill}%
}%
\begin{pgfscope}%
\pgfsys@transformshift{0.750000in}{4.766667in}%
\pgfsys@useobject{currentmarker}{}%
\end{pgfscope}%
\end{pgfscope}%
\begin{pgfscope}%
\pgftext[x=0.513888in,y=4.718472in,left,base]{\sffamily\fontsize{10.000000}{12.000000}\selectfont \(\displaystyle 80\)}%
\end{pgfscope}%
\begin{pgfscope}%
\pgfpathrectangle{\pgfqpoint{0.750000in}{0.660000in}}{\pgfqpoint{4.650000in}{4.620000in}} %
\pgfusepath{clip}%
\pgfsetrectcap%
\pgfsetroundjoin%
\pgfsetlinewidth{0.803000pt}%
\definecolor{currentstroke}{rgb}{0.690196,0.690196,0.690196}%
\pgfsetstrokecolor{currentstroke}%
\pgfsetdash{}{0pt}%
\pgfpathmoveto{\pgfqpoint{0.750000in}{5.280000in}}%
\pgfpathlineto{\pgfqpoint{5.400000in}{5.280000in}}%
\pgfusepath{stroke}%
\end{pgfscope}%
\begin{pgfscope}%
\pgfsetbuttcap%
\pgfsetroundjoin%
\definecolor{currentfill}{rgb}{0.000000,0.000000,0.000000}%
\pgfsetfillcolor{currentfill}%
\pgfsetlinewidth{0.803000pt}%
\definecolor{currentstroke}{rgb}{0.000000,0.000000,0.000000}%
\pgfsetstrokecolor{currentstroke}%
\pgfsetdash{}{0pt}%
\pgfsys@defobject{currentmarker}{\pgfqpoint{-0.048611in}{0.000000in}}{\pgfqpoint{0.000000in}{0.000000in}}{%
\pgfpathmoveto{\pgfqpoint{0.000000in}{0.000000in}}%
\pgfpathlineto{\pgfqpoint{-0.048611in}{0.000000in}}%
\pgfusepath{stroke,fill}%
}%
\begin{pgfscope}%
\pgfsys@transformshift{0.750000in}{5.280000in}%
\pgfsys@useobject{currentmarker}{}%
\end{pgfscope}%
\end{pgfscope}%
\begin{pgfscope}%
\pgftext[x=0.513888in,y=5.231806in,left,base]{\sffamily\fontsize{10.000000}{12.000000}\selectfont \(\displaystyle 90\)}%
\end{pgfscope}%
\begin{pgfscope}%
\pgftext[x=0.458333in,y=2.970000in,,bottom,rotate=90.000000]{\sffamily\fontsize{10.000000}{12.000000}\selectfont Yaw angle \(\displaystyle |\beta| (deg)\)}%
\end{pgfscope}%
\begin{pgfscope}%
\pgfpathrectangle{\pgfqpoint{0.750000in}{0.660000in}}{\pgfqpoint{4.650000in}{4.620000in}} %
\pgfusepath{clip}%
\pgfsetrectcap%
\pgfsetroundjoin%
\pgfsetlinewidth{1.505625pt}%
\definecolor{currentstroke}{rgb}{0.121569,0.466667,0.705882}%
\pgfsetstrokecolor{currentstroke}%
\pgfsetdash{}{0pt}%
\pgfpathmoveto{\pgfqpoint{0.750000in}{0.660000in}}%
\pgfpathlineto{\pgfqpoint{0.844898in}{1.380799in}}%
\pgfpathlineto{\pgfqpoint{0.939796in}{2.024135in}}%
\pgfpathlineto{\pgfqpoint{1.034694in}{2.553209in}}%
\pgfpathlineto{\pgfqpoint{1.129592in}{2.970578in}}%
\pgfpathlineto{\pgfqpoint{1.224490in}{3.296027in}}%
\pgfpathlineto{\pgfqpoint{1.319388in}{3.551110in}}%
\pgfpathlineto{\pgfqpoint{1.414286in}{3.753594in}}%
\pgfpathlineto{\pgfqpoint{1.509184in}{3.916789in}}%
\pgfpathlineto{\pgfqpoint{1.604082in}{4.050354in}}%
\pgfpathlineto{\pgfqpoint{1.698980in}{4.161259in}}%
\pgfpathlineto{\pgfqpoint{1.793878in}{4.254572in}}%
\pgfpathlineto{\pgfqpoint{1.888776in}{4.334019in}}%
\pgfpathlineto{\pgfqpoint{1.983673in}{4.402385in}}%
\pgfpathlineto{\pgfqpoint{2.078571in}{4.461775in}}%
\pgfpathlineto{\pgfqpoint{2.173469in}{4.513808in}}%
\pgfpathlineto{\pgfqpoint{2.268367in}{4.559745in}}%
\pgfpathlineto{\pgfqpoint{2.363265in}{4.600578in}}%
\pgfpathlineto{\pgfqpoint{2.458163in}{4.637099in}}%
\pgfpathlineto{\pgfqpoint{2.553061in}{4.669948in}}%
\pgfpathlineto{\pgfqpoint{2.647959in}{4.699646in}}%
\pgfpathlineto{\pgfqpoint{2.742857in}{4.726618in}}%
\pgfpathlineto{\pgfqpoint{2.837755in}{4.751221in}}%
\pgfpathlineto{\pgfqpoint{2.932653in}{4.773750in}}%
\pgfpathlineto{\pgfqpoint{3.027551in}{4.794455in}}%
\pgfpathlineto{\pgfqpoint{3.122449in}{4.813546in}}%
\pgfpathlineto{\pgfqpoint{3.217347in}{4.831204in}}%
\pgfpathlineto{\pgfqpoint{3.312245in}{4.847583in}}%
\pgfpathlineto{\pgfqpoint{3.407143in}{4.862816in}}%
\pgfpathlineto{\pgfqpoint{3.502041in}{4.877020in}}%
\pgfpathlineto{\pgfqpoint{3.596939in}{4.890293in}}%
\pgfpathlineto{\pgfqpoint{3.691837in}{4.902724in}}%
\pgfpathlineto{\pgfqpoint{3.786735in}{4.914391in}}%
\pgfpathlineto{\pgfqpoint{3.881633in}{4.925362in}}%
\pgfpathlineto{\pgfqpoint{3.976531in}{4.935696in}}%
\pgfpathlineto{\pgfqpoint{4.071429in}{4.945447in}}%
\pgfpathlineto{\pgfqpoint{4.166327in}{4.954664in}}%
\pgfpathlineto{\pgfqpoint{4.261224in}{4.963388in}}%
\pgfpathlineto{\pgfqpoint{4.356122in}{4.971658in}}%
\pgfpathlineto{\pgfqpoint{4.451020in}{4.979509in}}%
\pgfpathlineto{\pgfqpoint{4.545918in}{4.986971in}}%
\pgfpathlineto{\pgfqpoint{4.640816in}{4.994072in}}%
\pgfpathlineto{\pgfqpoint{4.735714in}{5.000839in}}%
\pgfpathlineto{\pgfqpoint{4.830612in}{5.007293in}}%
\pgfpathlineto{\pgfqpoint{4.925510in}{5.013457in}}%
\pgfpathlineto{\pgfqpoint{5.020408in}{5.019349in}}%
\pgfpathlineto{\pgfqpoint{5.115306in}{5.024986in}}%
\pgfpathlineto{\pgfqpoint{5.210204in}{5.030386in}}%
\pgfpathlineto{\pgfqpoint{5.305102in}{5.035562in}}%
\pgfpathlineto{\pgfqpoint{5.400000in}{5.040528in}}%
\pgfusepath{stroke}%
\end{pgfscope}%
\begin{pgfscope}%
\pgfpathrectangle{\pgfqpoint{0.750000in}{0.660000in}}{\pgfqpoint{4.650000in}{4.620000in}} %
\pgfusepath{clip}%
\pgfsetrectcap%
\pgfsetroundjoin%
\pgfsetlinewidth{1.505625pt}%
\definecolor{currentstroke}{rgb}{1.000000,0.498039,0.054902}%
\pgfsetstrokecolor{currentstroke}%
\pgfsetdash{}{0pt}%
\pgfpathmoveto{\pgfqpoint{0.750000in}{0.660000in}}%
\pgfpathlineto{\pgfqpoint{0.844898in}{1.018623in}}%
\pgfpathlineto{\pgfqpoint{0.939796in}{1.366889in}}%
\pgfpathlineto{\pgfqpoint{1.034694in}{1.696129in}}%
\pgfpathlineto{\pgfqpoint{1.129592in}{2.000423in}}%
\pgfpathlineto{\pgfqpoint{1.224490in}{2.276778in}}%
\pgfpathlineto{\pgfqpoint{1.319388in}{2.524666in}}%
\pgfpathlineto{\pgfqpoint{1.414286in}{2.745272in}}%
\pgfpathlineto{\pgfqpoint{1.509184in}{2.940764in}}%
\pgfpathlineto{\pgfqpoint{1.604082in}{3.113744in}}%
\pgfpathlineto{\pgfqpoint{1.698980in}{3.266878in}}%
\pgfpathlineto{\pgfqpoint{1.793878in}{3.402696in}}%
\pgfpathlineto{\pgfqpoint{1.888776in}{3.523486in}}%
\pgfpathlineto{\pgfqpoint{1.983673in}{3.631265in}}%
\pgfpathlineto{\pgfqpoint{2.078571in}{3.727781in}}%
\pgfpathlineto{\pgfqpoint{2.173469in}{3.814532in}}%
\pgfpathlineto{\pgfqpoint{2.268367in}{3.892798in}}%
\pgfpathlineto{\pgfqpoint{2.363265in}{3.963671in}}%
\pgfpathlineto{\pgfqpoint{2.458163in}{4.028079in}}%
\pgfpathlineto{\pgfqpoint{2.553061in}{4.086814in}}%
\pgfpathlineto{\pgfqpoint{2.647959in}{4.140552in}}%
\pgfpathlineto{\pgfqpoint{2.742857in}{4.189874in}}%
\pgfpathlineto{\pgfqpoint{2.837755in}{4.235277in}}%
\pgfpathlineto{\pgfqpoint{2.932653in}{4.277192in}}%
\pgfpathlineto{\pgfqpoint{3.027551in}{4.315990in}}%
\pgfpathlineto{\pgfqpoint{3.122449in}{4.351995in}}%
\pgfpathlineto{\pgfqpoint{3.217347in}{4.385488in}}%
\pgfpathlineto{\pgfqpoint{3.312245in}{4.416715in}}%
\pgfpathlineto{\pgfqpoint{3.407143in}{4.445893in}}%
\pgfpathlineto{\pgfqpoint{3.502041in}{4.473210in}}%
\pgfpathlineto{\pgfqpoint{3.596939in}{4.498836in}}%
\pgfpathlineto{\pgfqpoint{3.691837in}{4.522920in}}%
\pgfpathlineto{\pgfqpoint{3.786735in}{4.545593in}}%
\pgfpathlineto{\pgfqpoint{3.881633in}{4.566973in}}%
\pgfpathlineto{\pgfqpoint{3.976531in}{4.587166in}}%
\pgfpathlineto{\pgfqpoint{4.071429in}{4.606267in}}%
\pgfpathlineto{\pgfqpoint{4.166327in}{4.624360in}}%
\pgfpathlineto{\pgfqpoint{4.261224in}{4.641521in}}%
\pgfpathlineto{\pgfqpoint{4.356122in}{4.657820in}}%
\pgfpathlineto{\pgfqpoint{4.451020in}{4.673319in}}%
\pgfpathlineto{\pgfqpoint{4.545918in}{4.688075in}}%
\pgfpathlineto{\pgfqpoint{4.640816in}{4.702139in}}%
\pgfpathlineto{\pgfqpoint{4.735714in}{4.715558in}}%
\pgfpathlineto{\pgfqpoint{4.830612in}{4.728375in}}%
\pgfpathlineto{\pgfqpoint{4.925510in}{4.740630in}}%
\pgfpathlineto{\pgfqpoint{5.020408in}{4.752358in}}%
\pgfpathlineto{\pgfqpoint{5.115306in}{4.763591in}}%
\pgfpathlineto{\pgfqpoint{5.210204in}{4.774361in}}%
\pgfpathlineto{\pgfqpoint{5.305102in}{4.784695in}}%
\pgfpathlineto{\pgfqpoint{5.400000in}{4.794619in}}%
\pgfusepath{stroke}%
\end{pgfscope}%
\begin{pgfscope}%
\pgfpathrectangle{\pgfqpoint{0.750000in}{0.660000in}}{\pgfqpoint{4.650000in}{4.620000in}} %
\pgfusepath{clip}%
\pgfsetrectcap%
\pgfsetroundjoin%
\pgfsetlinewidth{1.505625pt}%
\definecolor{currentstroke}{rgb}{0.172549,0.627451,0.172549}%
\pgfsetstrokecolor{currentstroke}%
\pgfsetdash{}{0pt}%
\pgfpathmoveto{\pgfqpoint{0.750000in}{0.660000in}}%
\pgfpathlineto{\pgfqpoint{0.844898in}{0.825379in}}%
\pgfpathlineto{\pgfqpoint{0.939796in}{0.989719in}}%
\pgfpathlineto{\pgfqpoint{1.034694in}{1.152020in}}%
\pgfpathlineto{\pgfqpoint{1.129592in}{1.311353in}}%
\pgfpathlineto{\pgfqpoint{1.224490in}{1.466896in}}%
\pgfpathlineto{\pgfqpoint{1.319388in}{1.617947in}}%
\pgfpathlineto{\pgfqpoint{1.414286in}{1.763938in}}%
\pgfpathlineto{\pgfqpoint{1.509184in}{1.904437in}}%
\pgfpathlineto{\pgfqpoint{1.604082in}{2.039145in}}%
\pgfpathlineto{\pgfqpoint{1.698980in}{2.167883in}}%
\pgfpathlineto{\pgfqpoint{1.793878in}{2.290579in}}%
\pgfpathlineto{\pgfqpoint{1.888776in}{2.407251in}}%
\pgfpathlineto{\pgfqpoint{1.983673in}{2.517992in}}%
\pgfpathlineto{\pgfqpoint{2.078571in}{2.622951in}}%
\pgfpathlineto{\pgfqpoint{2.173469in}{2.722324in}}%
\pgfpathlineto{\pgfqpoint{2.268367in}{2.816335in}}%
\pgfpathlineto{\pgfqpoint{2.363265in}{2.905229in}}%
\pgfpathlineto{\pgfqpoint{2.458163in}{2.989265in}}%
\pgfpathlineto{\pgfqpoint{2.553061in}{3.068702in}}%
\pgfpathlineto{\pgfqpoint{2.647959in}{3.143801in}}%
\pgfpathlineto{\pgfqpoint{2.742857in}{3.214819in}}%
\pgfpathlineto{\pgfqpoint{2.837755in}{3.282002in}}%
\pgfpathlineto{\pgfqpoint{2.932653in}{3.345589in}}%
\pgfpathlineto{\pgfqpoint{3.027551in}{3.405805in}}%
\pgfpathlineto{\pgfqpoint{3.122449in}{3.462866in}}%
\pgfpathlineto{\pgfqpoint{3.217347in}{3.516972in}}%
\pgfpathlineto{\pgfqpoint{3.312245in}{3.568315in}}%
\pgfpathlineto{\pgfqpoint{3.407143in}{3.617070in}}%
\pgfpathlineto{\pgfqpoint{3.502041in}{3.663405in}}%
\pgfpathlineto{\pgfqpoint{3.596939in}{3.707473in}}%
\pgfpathlineto{\pgfqpoint{3.691837in}{3.749419in}}%
\pgfpathlineto{\pgfqpoint{3.786735in}{3.789376in}}%
\pgfpathlineto{\pgfqpoint{3.881633in}{3.827470in}}%
\pgfpathlineto{\pgfqpoint{3.976531in}{3.863814in}}%
\pgfpathlineto{\pgfqpoint{4.071429in}{3.898517in}}%
\pgfpathlineto{\pgfqpoint{4.166327in}{3.931679in}}%
\pgfpathlineto{\pgfqpoint{4.261224in}{3.963391in}}%
\pgfpathlineto{\pgfqpoint{4.356122in}{3.993741in}}%
\pgfpathlineto{\pgfqpoint{4.451020in}{4.022807in}}%
\pgfpathlineto{\pgfqpoint{4.545918in}{4.050665in}}%
\pgfpathlineto{\pgfqpoint{4.640816in}{4.077383in}}%
\pgfpathlineto{\pgfqpoint{4.735714in}{4.103027in}}%
\pgfpathlineto{\pgfqpoint{4.830612in}{4.127654in}}%
\pgfpathlineto{\pgfqpoint{4.925510in}{4.151323in}}%
\pgfpathlineto{\pgfqpoint{5.020408in}{4.174084in}}%
\pgfpathlineto{\pgfqpoint{5.115306in}{4.195987in}}%
\pgfpathlineto{\pgfqpoint{5.210204in}{4.217076in}}%
\pgfpathlineto{\pgfqpoint{5.305102in}{4.237395in}}%
\pgfpathlineto{\pgfqpoint{5.400000in}{4.256982in}}%
\pgfusepath{stroke}%
\end{pgfscope}%
\begin{pgfscope}%
\pgfpathrectangle{\pgfqpoint{0.750000in}{0.660000in}}{\pgfqpoint{4.650000in}{4.620000in}} %
\pgfusepath{clip}%
\pgfsetrectcap%
\pgfsetroundjoin%
\pgfsetlinewidth{1.505625pt}%
\definecolor{currentstroke}{rgb}{0.839216,0.152941,0.156863}%
\pgfsetstrokecolor{currentstroke}%
\pgfsetdash{}{0pt}%
\pgfpathmoveto{\pgfqpoint{0.750000in}{0.660000in}}%
\pgfpathlineto{\pgfqpoint{0.844898in}{0.754146in}}%
\pgfpathlineto{\pgfqpoint{0.939796in}{0.848100in}}%
\pgfpathlineto{\pgfqpoint{1.034694in}{0.941670in}}%
\pgfpathlineto{\pgfqpoint{1.129592in}{1.034673in}}%
\pgfpathlineto{\pgfqpoint{1.224490in}{1.126929in}}%
\pgfpathlineto{\pgfqpoint{1.319388in}{1.218267in}}%
\pgfpathlineto{\pgfqpoint{1.414286in}{1.308529in}}%
\pgfpathlineto{\pgfqpoint{1.509184in}{1.397566in}}%
\pgfpathlineto{\pgfqpoint{1.604082in}{1.485244in}}%
\pgfpathlineto{\pgfqpoint{1.698980in}{1.571441in}}%
\pgfpathlineto{\pgfqpoint{1.793878in}{1.656050in}}%
\pgfpathlineto{\pgfqpoint{1.888776in}{1.738980in}}%
\pgfpathlineto{\pgfqpoint{1.983673in}{1.820151in}}%
\pgfpathlineto{\pgfqpoint{2.078571in}{1.899499in}}%
\pgfpathlineto{\pgfqpoint{2.173469in}{1.976974in}}%
\pgfpathlineto{\pgfqpoint{2.268367in}{2.052538in}}%
\pgfpathlineto{\pgfqpoint{2.363265in}{2.126163in}}%
\pgfpathlineto{\pgfqpoint{2.458163in}{2.197836in}}%
\pgfpathlineto{\pgfqpoint{2.553061in}{2.267550in}}%
\pgfpathlineto{\pgfqpoint{2.647959in}{2.335311in}}%
\pgfpathlineto{\pgfqpoint{2.742857in}{2.401130in}}%
\pgfpathlineto{\pgfqpoint{2.837755in}{2.465026in}}%
\pgfpathlineto{\pgfqpoint{2.932653in}{2.527025in}}%
\pgfpathlineto{\pgfqpoint{3.027551in}{2.587157in}}%
\pgfpathlineto{\pgfqpoint{3.122449in}{2.645457in}}%
\pgfpathlineto{\pgfqpoint{3.217347in}{2.701964in}}%
\pgfpathlineto{\pgfqpoint{3.312245in}{2.756720in}}%
\pgfpathlineto{\pgfqpoint{3.407143in}{2.809768in}}%
\pgfpathlineto{\pgfqpoint{3.502041in}{2.861155in}}%
\pgfpathlineto{\pgfqpoint{3.596939in}{2.910927in}}%
\pgfpathlineto{\pgfqpoint{3.691837in}{2.959131in}}%
\pgfpathlineto{\pgfqpoint{3.786735in}{3.005817in}}%
\pgfpathlineto{\pgfqpoint{3.881633in}{3.051032in}}%
\pgfpathlineto{\pgfqpoint{3.976531in}{3.094824in}}%
\pgfpathlineto{\pgfqpoint{4.071429in}{3.137242in}}%
\pgfpathlineto{\pgfqpoint{4.166327in}{3.178331in}}%
\pgfpathlineto{\pgfqpoint{4.261224in}{3.218138in}}%
\pgfpathlineto{\pgfqpoint{4.356122in}{3.256708in}}%
\pgfpathlineto{\pgfqpoint{4.451020in}{3.294086in}}%
\pgfpathlineto{\pgfqpoint{4.545918in}{3.330313in}}%
\pgfpathlineto{\pgfqpoint{4.640816in}{3.365433in}}%
\pgfpathlineto{\pgfqpoint{4.735714in}{3.399485in}}%
\pgfpathlineto{\pgfqpoint{4.830612in}{3.432509in}}%
\pgfpathlineto{\pgfqpoint{4.925510in}{3.464542in}}%
\pgfpathlineto{\pgfqpoint{5.020408in}{3.495622in}}%
\pgfpathlineto{\pgfqpoint{5.115306in}{3.525783in}}%
\pgfpathlineto{\pgfqpoint{5.210204in}{3.555059in}}%
\pgfpathlineto{\pgfqpoint{5.305102in}{3.583484in}}%
\pgfpathlineto{\pgfqpoint{5.400000in}{3.611089in}}%
\pgfusepath{stroke}%
\end{pgfscope}%
\begin{pgfscope}%
\pgfpathrectangle{\pgfqpoint{0.750000in}{0.660000in}}{\pgfqpoint{4.650000in}{4.620000in}} %
\pgfusepath{clip}%
\pgfsetrectcap%
\pgfsetroundjoin%
\pgfsetlinewidth{1.505625pt}%
\definecolor{currentstroke}{rgb}{0.580392,0.403922,0.741176}%
\pgfsetstrokecolor{currentstroke}%
\pgfsetdash{}{0pt}%
\pgfpathmoveto{\pgfqpoint{0.750000in}{0.660000in}}%
\pgfpathlineto{\pgfqpoint{0.844898in}{0.712993in}}%
\pgfpathlineto{\pgfqpoint{0.939796in}{0.765953in}}%
\pgfpathlineto{\pgfqpoint{1.034694in}{0.818843in}}%
\pgfpathlineto{\pgfqpoint{1.129592in}{0.871631in}}%
\pgfpathlineto{\pgfqpoint{1.224490in}{0.924283in}}%
\pgfpathlineto{\pgfqpoint{1.319388in}{0.976765in}}%
\pgfpathlineto{\pgfqpoint{1.414286in}{1.029046in}}%
\pgfpathlineto{\pgfqpoint{1.509184in}{1.081093in}}%
\pgfpathlineto{\pgfqpoint{1.604082in}{1.132876in}}%
\pgfpathlineto{\pgfqpoint{1.698980in}{1.184365in}}%
\pgfpathlineto{\pgfqpoint{1.793878in}{1.235531in}}%
\pgfpathlineto{\pgfqpoint{1.888776in}{1.286347in}}%
\pgfpathlineto{\pgfqpoint{1.983673in}{1.336786in}}%
\pgfpathlineto{\pgfqpoint{2.078571in}{1.386823in}}%
\pgfpathlineto{\pgfqpoint{2.173469in}{1.436435in}}%
\pgfpathlineto{\pgfqpoint{2.268367in}{1.485598in}}%
\pgfpathlineto{\pgfqpoint{2.363265in}{1.534291in}}%
\pgfpathlineto{\pgfqpoint{2.458163in}{1.582496in}}%
\pgfpathlineto{\pgfqpoint{2.553061in}{1.630194in}}%
\pgfpathlineto{\pgfqpoint{2.647959in}{1.677368in}}%
\pgfpathlineto{\pgfqpoint{2.742857in}{1.724003in}}%
\pgfpathlineto{\pgfqpoint{2.837755in}{1.770085in}}%
\pgfpathlineto{\pgfqpoint{2.932653in}{1.815601in}}%
\pgfpathlineto{\pgfqpoint{3.027551in}{1.860541in}}%
\pgfpathlineto{\pgfqpoint{3.122449in}{1.904895in}}%
\pgfpathlineto{\pgfqpoint{3.217347in}{1.948654in}}%
\pgfpathlineto{\pgfqpoint{3.312245in}{1.991812in}}%
\pgfpathlineto{\pgfqpoint{3.407143in}{2.034363in}}%
\pgfpathlineto{\pgfqpoint{3.502041in}{2.076301in}}%
\pgfpathlineto{\pgfqpoint{3.596939in}{2.117623in}}%
\pgfpathlineto{\pgfqpoint{3.691837in}{2.158327in}}%
\pgfpathlineto{\pgfqpoint{3.786735in}{2.198411in}}%
\pgfpathlineto{\pgfqpoint{3.881633in}{2.237875in}}%
\pgfpathlineto{\pgfqpoint{3.976531in}{2.276720in}}%
\pgfpathlineto{\pgfqpoint{4.071429in}{2.314945in}}%
\pgfpathlineto{\pgfqpoint{4.166327in}{2.352554in}}%
\pgfpathlineto{\pgfqpoint{4.261224in}{2.389550in}}%
\pgfpathlineto{\pgfqpoint{4.356122in}{2.425935in}}%
\pgfpathlineto{\pgfqpoint{4.451020in}{2.461714in}}%
\pgfpathlineto{\pgfqpoint{4.545918in}{2.496892in}}%
\pgfpathlineto{\pgfqpoint{4.640816in}{2.531473in}}%
\pgfpathlineto{\pgfqpoint{4.735714in}{2.565464in}}%
\pgfpathlineto{\pgfqpoint{4.830612in}{2.598871in}}%
\pgfpathlineto{\pgfqpoint{4.925510in}{2.631699in}}%
\pgfpathlineto{\pgfqpoint{5.020408in}{2.663957in}}%
\pgfpathlineto{\pgfqpoint{5.115306in}{2.695651in}}%
\pgfpathlineto{\pgfqpoint{5.210204in}{2.726789in}}%
\pgfpathlineto{\pgfqpoint{5.305102in}{2.757378in}}%
\pgfpathlineto{\pgfqpoint{5.400000in}{2.787427in}}%
\pgfusepath{stroke}%
\end{pgfscope}%
\begin{pgfscope}%
\pgfsetrectcap%
\pgfsetmiterjoin%
\pgfsetlinewidth{0.803000pt}%
\definecolor{currentstroke}{rgb}{0.000000,0.000000,0.000000}%
\pgfsetstrokecolor{currentstroke}%
\pgfsetdash{}{0pt}%
\pgfpathmoveto{\pgfqpoint{0.750000in}{0.660000in}}%
\pgfpathlineto{\pgfqpoint{0.750000in}{5.280000in}}%
\pgfusepath{stroke}%
\end{pgfscope}%
\begin{pgfscope}%
\pgfsetrectcap%
\pgfsetmiterjoin%
\pgfsetlinewidth{0.803000pt}%
\definecolor{currentstroke}{rgb}{0.000000,0.000000,0.000000}%
\pgfsetstrokecolor{currentstroke}%
\pgfsetdash{}{0pt}%
\pgfpathmoveto{\pgfqpoint{5.400000in}{0.660000in}}%
\pgfpathlineto{\pgfqpoint{5.400000in}{5.280000in}}%
\pgfusepath{stroke}%
\end{pgfscope}%
\begin{pgfscope}%
\pgfsetrectcap%
\pgfsetmiterjoin%
\pgfsetlinewidth{0.803000pt}%
\definecolor{currentstroke}{rgb}{0.000000,0.000000,0.000000}%
\pgfsetstrokecolor{currentstroke}%
\pgfsetdash{}{0pt}%
\pgfpathmoveto{\pgfqpoint{0.750000in}{0.660000in}}%
\pgfpathlineto{\pgfqpoint{5.400000in}{0.660000in}}%
\pgfusepath{stroke}%
\end{pgfscope}%
\begin{pgfscope}%
\pgfsetrectcap%
\pgfsetmiterjoin%
\pgfsetlinewidth{0.803000pt}%
\definecolor{currentstroke}{rgb}{0.000000,0.000000,0.000000}%
\pgfsetstrokecolor{currentstroke}%
\pgfsetdash{}{0pt}%
\pgfpathmoveto{\pgfqpoint{0.750000in}{5.280000in}}%
\pgfpathlineto{\pgfqpoint{5.400000in}{5.280000in}}%
\pgfusepath{stroke}%
\end{pgfscope}%
\begin{pgfscope}%
\pgfsetbuttcap%
\pgfsetmiterjoin%
\definecolor{currentfill}{rgb}{1.000000,1.000000,1.000000}%
\pgfsetfillcolor{currentfill}%
\pgfsetfillopacity{0.800000}%
\pgfsetlinewidth{1.003750pt}%
\definecolor{currentstroke}{rgb}{0.800000,0.800000,0.800000}%
\pgfsetstrokecolor{currentstroke}%
\pgfsetstrokeopacity{0.800000}%
\pgfsetdash{}{0pt}%
\pgfpathmoveto{\pgfqpoint{4.431010in}{0.729444in}}%
\pgfpathlineto{\pgfqpoint{5.302778in}{0.729444in}}%
\pgfpathquadraticcurveto{\pgfqpoint{5.330556in}{0.729444in}}{\pgfqpoint{5.330556in}{0.757222in}}%
\pgfpathlineto{\pgfqpoint{5.330556in}{1.711388in}}%
\pgfpathquadraticcurveto{\pgfqpoint{5.330556in}{1.739166in}}{\pgfqpoint{5.302778in}{1.739166in}}%
\pgfpathlineto{\pgfqpoint{4.431010in}{1.739166in}}%
\pgfpathquadraticcurveto{\pgfqpoint{4.403232in}{1.739166in}}{\pgfqpoint{4.403232in}{1.711388in}}%
\pgfpathlineto{\pgfqpoint{4.403232in}{0.757222in}}%
\pgfpathquadraticcurveto{\pgfqpoint{4.403232in}{0.729444in}}{\pgfqpoint{4.431010in}{0.729444in}}%
\pgfpathclose%
\pgfusepath{stroke,fill}%
\end{pgfscope}%
\begin{pgfscope}%
\pgfsetrectcap%
\pgfsetroundjoin%
\pgfsetlinewidth{1.505625pt}%
\definecolor{currentstroke}{rgb}{0.121569,0.466667,0.705882}%
\pgfsetstrokecolor{currentstroke}%
\pgfsetdash{}{0pt}%
\pgfpathmoveto{\pgfqpoint{4.458787in}{1.634999in}}%
\pgfpathlineto{\pgfqpoint{4.736565in}{1.634999in}}%
\pgfusepath{stroke}%
\end{pgfscope}%
\begin{pgfscope}%
\pgftext[x=4.847676in,y=1.586388in,left,base]{\sffamily\fontsize{10.000000}{12.000000}\selectfont \(\displaystyle e = 0.1\)}%
\end{pgfscope}%
\begin{pgfscope}%
\pgfsetrectcap%
\pgfsetroundjoin%
\pgfsetlinewidth{1.505625pt}%
\definecolor{currentstroke}{rgb}{1.000000,0.498039,0.054902}%
\pgfsetstrokecolor{currentstroke}%
\pgfsetdash{}{0pt}%
\pgfpathmoveto{\pgfqpoint{4.458787in}{1.441388in}}%
\pgfpathlineto{\pgfqpoint{4.736565in}{1.441388in}}%
\pgfusepath{stroke}%
\end{pgfscope}%
\begin{pgfscope}%
\pgftext[x=4.847676in,y=1.392777in,left,base]{\sffamily\fontsize{10.000000}{12.000000}\selectfont \(\displaystyle e = 0.2\)}%
\end{pgfscope}%
\begin{pgfscope}%
\pgfsetrectcap%
\pgfsetroundjoin%
\pgfsetlinewidth{1.505625pt}%
\definecolor{currentstroke}{rgb}{0.172549,0.627451,0.172549}%
\pgfsetstrokecolor{currentstroke}%
\pgfsetdash{}{0pt}%
\pgfpathmoveto{\pgfqpoint{4.458787in}{1.247777in}}%
\pgfpathlineto{\pgfqpoint{4.736565in}{1.247777in}}%
\pgfusepath{stroke}%
\end{pgfscope}%
\begin{pgfscope}%
\pgftext[x=4.847676in,y=1.199166in,left,base]{\sffamily\fontsize{10.000000}{12.000000}\selectfont \(\displaystyle e = 0.4\)}%
\end{pgfscope}%
\begin{pgfscope}%
\pgfsetrectcap%
\pgfsetroundjoin%
\pgfsetlinewidth{1.505625pt}%
\definecolor{currentstroke}{rgb}{0.839216,0.152941,0.156863}%
\pgfsetstrokecolor{currentstroke}%
\pgfsetdash{}{0pt}%
\pgfpathmoveto{\pgfqpoint{4.458787in}{1.054166in}}%
\pgfpathlineto{\pgfqpoint{4.736565in}{1.054166in}}%
\pgfusepath{stroke}%
\end{pgfscope}%
\begin{pgfscope}%
\pgftext[x=4.847676in,y=1.005555in,left,base]{\sffamily\fontsize{10.000000}{12.000000}\selectfont \(\displaystyle e = 0.6\)}%
\end{pgfscope}%
\begin{pgfscope}%
\pgfsetrectcap%
\pgfsetroundjoin%
\pgfsetlinewidth{1.505625pt}%
\definecolor{currentstroke}{rgb}{0.580392,0.403922,0.741176}%
\pgfsetstrokecolor{currentstroke}%
\pgfsetdash{}{0pt}%
\pgfpathmoveto{\pgfqpoint{4.458787in}{0.860555in}}%
\pgfpathlineto{\pgfqpoint{4.736565in}{0.860555in}}%
\pgfusepath{stroke}%
\end{pgfscope}%
\begin{pgfscope}%
\pgftext[x=4.847676in,y=0.811944in,left,base]{\sffamily\fontsize{10.000000}{12.000000}\selectfont \(\displaystyle e = 0.8\)}%
\end{pgfscope}%
\end{pgfpicture}%
\makeatother%
\endgroup%

}
\end{subfigure}
\begin{subfigure}[b]{0.5\textwidth}
\centering
\resizebox{1.0\textwidth}{!}{
%% Creator: Matplotlib, PGF backend
%%
%% To include the figure in your LaTeX document, write
%%   \input{<filename>.pgf}
%%
%% Make sure the required packages are loaded in your preamble
%%   \usepackage{pgf}
%%
%% Figures using additional raster images can only be included by \input if
%% they are in the same directory as the main LaTeX file. For loading figures
%% from other directories you can use the `import` package
%%   \usepackage{import}
%% and then include the figures with
%%   \import{<path to file>}{<filename>.pgf}
%%
%% Matplotlib used the following preamble
%%   \usepackage{fontspec}
%%
\begingroup%
\makeatletter%
\begin{pgfpicture}%
\pgfpathrectangle{\pgfpointorigin}{\pgfqpoint{6.000000in}{6.000000in}}%
\pgfusepath{use as bounding box, clip}%
\begin{pgfscope}%
\pgfsetbuttcap%
\pgfsetmiterjoin%
\definecolor{currentfill}{rgb}{1.000000,1.000000,1.000000}%
\pgfsetfillcolor{currentfill}%
\pgfsetlinewidth{0.000000pt}%
\definecolor{currentstroke}{rgb}{1.000000,1.000000,1.000000}%
\pgfsetstrokecolor{currentstroke}%
\pgfsetdash{}{0pt}%
\pgfpathmoveto{\pgfqpoint{0.000000in}{0.000000in}}%
\pgfpathlineto{\pgfqpoint{6.000000in}{0.000000in}}%
\pgfpathlineto{\pgfqpoint{6.000000in}{6.000000in}}%
\pgfpathlineto{\pgfqpoint{0.000000in}{6.000000in}}%
\pgfpathclose%
\pgfusepath{fill}%
\end{pgfscope}%
\begin{pgfscope}%
\pgfsetbuttcap%
\pgfsetmiterjoin%
\definecolor{currentfill}{rgb}{1.000000,1.000000,1.000000}%
\pgfsetfillcolor{currentfill}%
\pgfsetlinewidth{0.000000pt}%
\definecolor{currentstroke}{rgb}{0.000000,0.000000,0.000000}%
\pgfsetstrokecolor{currentstroke}%
\pgfsetstrokeopacity{0.000000}%
\pgfsetdash{}{0pt}%
\pgfpathmoveto{\pgfqpoint{0.750000in}{0.660000in}}%
\pgfpathlineto{\pgfqpoint{5.400000in}{0.660000in}}%
\pgfpathlineto{\pgfqpoint{5.400000in}{5.280000in}}%
\pgfpathlineto{\pgfqpoint{0.750000in}{5.280000in}}%
\pgfpathclose%
\pgfusepath{fill}%
\end{pgfscope}%
\begin{pgfscope}%
\pgfpathrectangle{\pgfqpoint{0.750000in}{0.660000in}}{\pgfqpoint{4.650000in}{4.620000in}} %
\pgfusepath{clip}%
\pgfsetrectcap%
\pgfsetroundjoin%
\pgfsetlinewidth{0.803000pt}%
\definecolor{currentstroke}{rgb}{0.690196,0.690196,0.690196}%
\pgfsetstrokecolor{currentstroke}%
\pgfsetdash{}{0pt}%
\pgfpathmoveto{\pgfqpoint{0.750000in}{0.660000in}}%
\pgfpathlineto{\pgfqpoint{0.750000in}{5.280000in}}%
\pgfusepath{stroke}%
\end{pgfscope}%
\begin{pgfscope}%
\pgfsetbuttcap%
\pgfsetroundjoin%
\definecolor{currentfill}{rgb}{0.000000,0.000000,0.000000}%
\pgfsetfillcolor{currentfill}%
\pgfsetlinewidth{0.803000pt}%
\definecolor{currentstroke}{rgb}{0.000000,0.000000,0.000000}%
\pgfsetstrokecolor{currentstroke}%
\pgfsetdash{}{0pt}%
\pgfsys@defobject{currentmarker}{\pgfqpoint{0.000000in}{-0.048611in}}{\pgfqpoint{0.000000in}{0.000000in}}{%
\pgfpathmoveto{\pgfqpoint{0.000000in}{0.000000in}}%
\pgfpathlineto{\pgfqpoint{0.000000in}{-0.048611in}}%
\pgfusepath{stroke,fill}%
}%
\begin{pgfscope}%
\pgfsys@transformshift{0.750000in}{0.660000in}%
\pgfsys@useobject{currentmarker}{}%
\end{pgfscope}%
\end{pgfscope}%
\begin{pgfscope}%
\pgftext[x=0.750000in,y=0.562778in,,top]{\sffamily\fontsize{10.000000}{12.000000}\selectfont \(\displaystyle 0\)}%
\end{pgfscope}%
\begin{pgfscope}%
\pgfpathrectangle{\pgfqpoint{0.750000in}{0.660000in}}{\pgfqpoint{4.650000in}{4.620000in}} %
\pgfusepath{clip}%
\pgfsetrectcap%
\pgfsetroundjoin%
\pgfsetlinewidth{0.803000pt}%
\definecolor{currentstroke}{rgb}{0.690196,0.690196,0.690196}%
\pgfsetstrokecolor{currentstroke}%
\pgfsetdash{}{0pt}%
\pgfpathmoveto{\pgfqpoint{1.525000in}{0.660000in}}%
\pgfpathlineto{\pgfqpoint{1.525000in}{5.280000in}}%
\pgfusepath{stroke}%
\end{pgfscope}%
\begin{pgfscope}%
\pgfsetbuttcap%
\pgfsetroundjoin%
\definecolor{currentfill}{rgb}{0.000000,0.000000,0.000000}%
\pgfsetfillcolor{currentfill}%
\pgfsetlinewidth{0.803000pt}%
\definecolor{currentstroke}{rgb}{0.000000,0.000000,0.000000}%
\pgfsetstrokecolor{currentstroke}%
\pgfsetdash{}{0pt}%
\pgfsys@defobject{currentmarker}{\pgfqpoint{0.000000in}{-0.048611in}}{\pgfqpoint{0.000000in}{0.000000in}}{%
\pgfpathmoveto{\pgfqpoint{0.000000in}{0.000000in}}%
\pgfpathlineto{\pgfqpoint{0.000000in}{-0.048611in}}%
\pgfusepath{stroke,fill}%
}%
\begin{pgfscope}%
\pgfsys@transformshift{1.525000in}{0.660000in}%
\pgfsys@useobject{currentmarker}{}%
\end{pgfscope}%
\end{pgfscope}%
\begin{pgfscope}%
\pgftext[x=1.525000in,y=0.562778in,,top]{\sffamily\fontsize{10.000000}{12.000000}\selectfont \(\displaystyle 5\)}%
\end{pgfscope}%
\begin{pgfscope}%
\pgfpathrectangle{\pgfqpoint{0.750000in}{0.660000in}}{\pgfqpoint{4.650000in}{4.620000in}} %
\pgfusepath{clip}%
\pgfsetrectcap%
\pgfsetroundjoin%
\pgfsetlinewidth{0.803000pt}%
\definecolor{currentstroke}{rgb}{0.690196,0.690196,0.690196}%
\pgfsetstrokecolor{currentstroke}%
\pgfsetdash{}{0pt}%
\pgfpathmoveto{\pgfqpoint{2.300000in}{0.660000in}}%
\pgfpathlineto{\pgfqpoint{2.300000in}{5.280000in}}%
\pgfusepath{stroke}%
\end{pgfscope}%
\begin{pgfscope}%
\pgfsetbuttcap%
\pgfsetroundjoin%
\definecolor{currentfill}{rgb}{0.000000,0.000000,0.000000}%
\pgfsetfillcolor{currentfill}%
\pgfsetlinewidth{0.803000pt}%
\definecolor{currentstroke}{rgb}{0.000000,0.000000,0.000000}%
\pgfsetstrokecolor{currentstroke}%
\pgfsetdash{}{0pt}%
\pgfsys@defobject{currentmarker}{\pgfqpoint{0.000000in}{-0.048611in}}{\pgfqpoint{0.000000in}{0.000000in}}{%
\pgfpathmoveto{\pgfqpoint{0.000000in}{0.000000in}}%
\pgfpathlineto{\pgfqpoint{0.000000in}{-0.048611in}}%
\pgfusepath{stroke,fill}%
}%
\begin{pgfscope}%
\pgfsys@transformshift{2.300000in}{0.660000in}%
\pgfsys@useobject{currentmarker}{}%
\end{pgfscope}%
\end{pgfscope}%
\begin{pgfscope}%
\pgftext[x=2.300000in,y=0.562778in,,top]{\sffamily\fontsize{10.000000}{12.000000}\selectfont \(\displaystyle 10\)}%
\end{pgfscope}%
\begin{pgfscope}%
\pgfpathrectangle{\pgfqpoint{0.750000in}{0.660000in}}{\pgfqpoint{4.650000in}{4.620000in}} %
\pgfusepath{clip}%
\pgfsetrectcap%
\pgfsetroundjoin%
\pgfsetlinewidth{0.803000pt}%
\definecolor{currentstroke}{rgb}{0.690196,0.690196,0.690196}%
\pgfsetstrokecolor{currentstroke}%
\pgfsetdash{}{0pt}%
\pgfpathmoveto{\pgfqpoint{3.075000in}{0.660000in}}%
\pgfpathlineto{\pgfqpoint{3.075000in}{5.280000in}}%
\pgfusepath{stroke}%
\end{pgfscope}%
\begin{pgfscope}%
\pgfsetbuttcap%
\pgfsetroundjoin%
\definecolor{currentfill}{rgb}{0.000000,0.000000,0.000000}%
\pgfsetfillcolor{currentfill}%
\pgfsetlinewidth{0.803000pt}%
\definecolor{currentstroke}{rgb}{0.000000,0.000000,0.000000}%
\pgfsetstrokecolor{currentstroke}%
\pgfsetdash{}{0pt}%
\pgfsys@defobject{currentmarker}{\pgfqpoint{0.000000in}{-0.048611in}}{\pgfqpoint{0.000000in}{0.000000in}}{%
\pgfpathmoveto{\pgfqpoint{0.000000in}{0.000000in}}%
\pgfpathlineto{\pgfqpoint{0.000000in}{-0.048611in}}%
\pgfusepath{stroke,fill}%
}%
\begin{pgfscope}%
\pgfsys@transformshift{3.075000in}{0.660000in}%
\pgfsys@useobject{currentmarker}{}%
\end{pgfscope}%
\end{pgfscope}%
\begin{pgfscope}%
\pgftext[x=3.075000in,y=0.562778in,,top]{\sffamily\fontsize{10.000000}{12.000000}\selectfont \(\displaystyle 15\)}%
\end{pgfscope}%
\begin{pgfscope}%
\pgfpathrectangle{\pgfqpoint{0.750000in}{0.660000in}}{\pgfqpoint{4.650000in}{4.620000in}} %
\pgfusepath{clip}%
\pgfsetrectcap%
\pgfsetroundjoin%
\pgfsetlinewidth{0.803000pt}%
\definecolor{currentstroke}{rgb}{0.690196,0.690196,0.690196}%
\pgfsetstrokecolor{currentstroke}%
\pgfsetdash{}{0pt}%
\pgfpathmoveto{\pgfqpoint{3.850000in}{0.660000in}}%
\pgfpathlineto{\pgfqpoint{3.850000in}{5.280000in}}%
\pgfusepath{stroke}%
\end{pgfscope}%
\begin{pgfscope}%
\pgfsetbuttcap%
\pgfsetroundjoin%
\definecolor{currentfill}{rgb}{0.000000,0.000000,0.000000}%
\pgfsetfillcolor{currentfill}%
\pgfsetlinewidth{0.803000pt}%
\definecolor{currentstroke}{rgb}{0.000000,0.000000,0.000000}%
\pgfsetstrokecolor{currentstroke}%
\pgfsetdash{}{0pt}%
\pgfsys@defobject{currentmarker}{\pgfqpoint{0.000000in}{-0.048611in}}{\pgfqpoint{0.000000in}{0.000000in}}{%
\pgfpathmoveto{\pgfqpoint{0.000000in}{0.000000in}}%
\pgfpathlineto{\pgfqpoint{0.000000in}{-0.048611in}}%
\pgfusepath{stroke,fill}%
}%
\begin{pgfscope}%
\pgfsys@transformshift{3.850000in}{0.660000in}%
\pgfsys@useobject{currentmarker}{}%
\end{pgfscope}%
\end{pgfscope}%
\begin{pgfscope}%
\pgftext[x=3.850000in,y=0.562778in,,top]{\sffamily\fontsize{10.000000}{12.000000}\selectfont \(\displaystyle 20\)}%
\end{pgfscope}%
\begin{pgfscope}%
\pgfpathrectangle{\pgfqpoint{0.750000in}{0.660000in}}{\pgfqpoint{4.650000in}{4.620000in}} %
\pgfusepath{clip}%
\pgfsetrectcap%
\pgfsetroundjoin%
\pgfsetlinewidth{0.803000pt}%
\definecolor{currentstroke}{rgb}{0.690196,0.690196,0.690196}%
\pgfsetstrokecolor{currentstroke}%
\pgfsetdash{}{0pt}%
\pgfpathmoveto{\pgfqpoint{4.625000in}{0.660000in}}%
\pgfpathlineto{\pgfqpoint{4.625000in}{5.280000in}}%
\pgfusepath{stroke}%
\end{pgfscope}%
\begin{pgfscope}%
\pgfsetbuttcap%
\pgfsetroundjoin%
\definecolor{currentfill}{rgb}{0.000000,0.000000,0.000000}%
\pgfsetfillcolor{currentfill}%
\pgfsetlinewidth{0.803000pt}%
\definecolor{currentstroke}{rgb}{0.000000,0.000000,0.000000}%
\pgfsetstrokecolor{currentstroke}%
\pgfsetdash{}{0pt}%
\pgfsys@defobject{currentmarker}{\pgfqpoint{0.000000in}{-0.048611in}}{\pgfqpoint{0.000000in}{0.000000in}}{%
\pgfpathmoveto{\pgfqpoint{0.000000in}{0.000000in}}%
\pgfpathlineto{\pgfqpoint{0.000000in}{-0.048611in}}%
\pgfusepath{stroke,fill}%
}%
\begin{pgfscope}%
\pgfsys@transformshift{4.625000in}{0.660000in}%
\pgfsys@useobject{currentmarker}{}%
\end{pgfscope}%
\end{pgfscope}%
\begin{pgfscope}%
\pgftext[x=4.625000in,y=0.562778in,,top]{\sffamily\fontsize{10.000000}{12.000000}\selectfont \(\displaystyle 25\)}%
\end{pgfscope}%
\begin{pgfscope}%
\pgfpathrectangle{\pgfqpoint{0.750000in}{0.660000in}}{\pgfqpoint{4.650000in}{4.620000in}} %
\pgfusepath{clip}%
\pgfsetrectcap%
\pgfsetroundjoin%
\pgfsetlinewidth{0.803000pt}%
\definecolor{currentstroke}{rgb}{0.690196,0.690196,0.690196}%
\pgfsetstrokecolor{currentstroke}%
\pgfsetdash{}{0pt}%
\pgfpathmoveto{\pgfqpoint{5.400000in}{0.660000in}}%
\pgfpathlineto{\pgfqpoint{5.400000in}{5.280000in}}%
\pgfusepath{stroke}%
\end{pgfscope}%
\begin{pgfscope}%
\pgfsetbuttcap%
\pgfsetroundjoin%
\definecolor{currentfill}{rgb}{0.000000,0.000000,0.000000}%
\pgfsetfillcolor{currentfill}%
\pgfsetlinewidth{0.803000pt}%
\definecolor{currentstroke}{rgb}{0.000000,0.000000,0.000000}%
\pgfsetstrokecolor{currentstroke}%
\pgfsetdash{}{0pt}%
\pgfsys@defobject{currentmarker}{\pgfqpoint{0.000000in}{-0.048611in}}{\pgfqpoint{0.000000in}{0.000000in}}{%
\pgfpathmoveto{\pgfqpoint{0.000000in}{0.000000in}}%
\pgfpathlineto{\pgfqpoint{0.000000in}{-0.048611in}}%
\pgfusepath{stroke,fill}%
}%
\begin{pgfscope}%
\pgfsys@transformshift{5.400000in}{0.660000in}%
\pgfsys@useobject{currentmarker}{}%
\end{pgfscope}%
\end{pgfscope}%
\begin{pgfscope}%
\pgftext[x=5.400000in,y=0.562778in,,top]{\sffamily\fontsize{10.000000}{12.000000}\selectfont \(\displaystyle 30\)}%
\end{pgfscope}%
\begin{pgfscope}%
\pgftext[x=3.075000in,y=0.383889in,,top]{\sffamily\fontsize{10.000000}{12.000000}\selectfont Inclination change (deg)}%
\end{pgfscope}%
\begin{pgfscope}%
\pgfpathrectangle{\pgfqpoint{0.750000in}{0.660000in}}{\pgfqpoint{4.650000in}{4.620000in}} %
\pgfusepath{clip}%
\pgfsetrectcap%
\pgfsetroundjoin%
\pgfsetlinewidth{0.803000pt}%
\definecolor{currentstroke}{rgb}{0.690196,0.690196,0.690196}%
\pgfsetstrokecolor{currentstroke}%
\pgfsetdash{}{0pt}%
\pgfpathmoveto{\pgfqpoint{0.750000in}{0.660000in}}%
\pgfpathlineto{\pgfqpoint{5.400000in}{0.660000in}}%
\pgfusepath{stroke}%
\end{pgfscope}%
\begin{pgfscope}%
\pgfsetbuttcap%
\pgfsetroundjoin%
\definecolor{currentfill}{rgb}{0.000000,0.000000,0.000000}%
\pgfsetfillcolor{currentfill}%
\pgfsetlinewidth{0.803000pt}%
\definecolor{currentstroke}{rgb}{0.000000,0.000000,0.000000}%
\pgfsetstrokecolor{currentstroke}%
\pgfsetdash{}{0pt}%
\pgfsys@defobject{currentmarker}{\pgfqpoint{-0.048611in}{0.000000in}}{\pgfqpoint{0.000000in}{0.000000in}}{%
\pgfpathmoveto{\pgfqpoint{0.000000in}{0.000000in}}%
\pgfpathlineto{\pgfqpoint{-0.048611in}{0.000000in}}%
\pgfusepath{stroke,fill}%
}%
\begin{pgfscope}%
\pgfsys@transformshift{0.750000in}{0.660000in}%
\pgfsys@useobject{currentmarker}{}%
\end{pgfscope}%
\end{pgfscope}%
\begin{pgfscope}%
\pgftext[x=0.475308in,y=0.611806in,left,base]{\sffamily\fontsize{10.000000}{12.000000}\selectfont \(\displaystyle 0.0\)}%
\end{pgfscope}%
\begin{pgfscope}%
\pgfpathrectangle{\pgfqpoint{0.750000in}{0.660000in}}{\pgfqpoint{4.650000in}{4.620000in}} %
\pgfusepath{clip}%
\pgfsetrectcap%
\pgfsetroundjoin%
\pgfsetlinewidth{0.803000pt}%
\definecolor{currentstroke}{rgb}{0.690196,0.690196,0.690196}%
\pgfsetstrokecolor{currentstroke}%
\pgfsetdash{}{0pt}%
\pgfpathmoveto{\pgfqpoint{0.750000in}{1.584000in}}%
\pgfpathlineto{\pgfqpoint{5.400000in}{1.584000in}}%
\pgfusepath{stroke}%
\end{pgfscope}%
\begin{pgfscope}%
\pgfsetbuttcap%
\pgfsetroundjoin%
\definecolor{currentfill}{rgb}{0.000000,0.000000,0.000000}%
\pgfsetfillcolor{currentfill}%
\pgfsetlinewidth{0.803000pt}%
\definecolor{currentstroke}{rgb}{0.000000,0.000000,0.000000}%
\pgfsetstrokecolor{currentstroke}%
\pgfsetdash{}{0pt}%
\pgfsys@defobject{currentmarker}{\pgfqpoint{-0.048611in}{0.000000in}}{\pgfqpoint{0.000000in}{0.000000in}}{%
\pgfpathmoveto{\pgfqpoint{0.000000in}{0.000000in}}%
\pgfpathlineto{\pgfqpoint{-0.048611in}{0.000000in}}%
\pgfusepath{stroke,fill}%
}%
\begin{pgfscope}%
\pgfsys@transformshift{0.750000in}{1.584000in}%
\pgfsys@useobject{currentmarker}{}%
\end{pgfscope}%
\end{pgfscope}%
\begin{pgfscope}%
\pgftext[x=0.475308in,y=1.535806in,left,base]{\sffamily\fontsize{10.000000}{12.000000}\selectfont \(\displaystyle 0.5\)}%
\end{pgfscope}%
\begin{pgfscope}%
\pgfpathrectangle{\pgfqpoint{0.750000in}{0.660000in}}{\pgfqpoint{4.650000in}{4.620000in}} %
\pgfusepath{clip}%
\pgfsetrectcap%
\pgfsetroundjoin%
\pgfsetlinewidth{0.803000pt}%
\definecolor{currentstroke}{rgb}{0.690196,0.690196,0.690196}%
\pgfsetstrokecolor{currentstroke}%
\pgfsetdash{}{0pt}%
\pgfpathmoveto{\pgfqpoint{0.750000in}{2.508000in}}%
\pgfpathlineto{\pgfqpoint{5.400000in}{2.508000in}}%
\pgfusepath{stroke}%
\end{pgfscope}%
\begin{pgfscope}%
\pgfsetbuttcap%
\pgfsetroundjoin%
\definecolor{currentfill}{rgb}{0.000000,0.000000,0.000000}%
\pgfsetfillcolor{currentfill}%
\pgfsetlinewidth{0.803000pt}%
\definecolor{currentstroke}{rgb}{0.000000,0.000000,0.000000}%
\pgfsetstrokecolor{currentstroke}%
\pgfsetdash{}{0pt}%
\pgfsys@defobject{currentmarker}{\pgfqpoint{-0.048611in}{0.000000in}}{\pgfqpoint{0.000000in}{0.000000in}}{%
\pgfpathmoveto{\pgfqpoint{0.000000in}{0.000000in}}%
\pgfpathlineto{\pgfqpoint{-0.048611in}{0.000000in}}%
\pgfusepath{stroke,fill}%
}%
\begin{pgfscope}%
\pgfsys@transformshift{0.750000in}{2.508000in}%
\pgfsys@useobject{currentmarker}{}%
\end{pgfscope}%
\end{pgfscope}%
\begin{pgfscope}%
\pgftext[x=0.475308in,y=2.459806in,left,base]{\sffamily\fontsize{10.000000}{12.000000}\selectfont \(\displaystyle 1.0\)}%
\end{pgfscope}%
\begin{pgfscope}%
\pgfpathrectangle{\pgfqpoint{0.750000in}{0.660000in}}{\pgfqpoint{4.650000in}{4.620000in}} %
\pgfusepath{clip}%
\pgfsetrectcap%
\pgfsetroundjoin%
\pgfsetlinewidth{0.803000pt}%
\definecolor{currentstroke}{rgb}{0.690196,0.690196,0.690196}%
\pgfsetstrokecolor{currentstroke}%
\pgfsetdash{}{0pt}%
\pgfpathmoveto{\pgfqpoint{0.750000in}{3.432000in}}%
\pgfpathlineto{\pgfqpoint{5.400000in}{3.432000in}}%
\pgfusepath{stroke}%
\end{pgfscope}%
\begin{pgfscope}%
\pgfsetbuttcap%
\pgfsetroundjoin%
\definecolor{currentfill}{rgb}{0.000000,0.000000,0.000000}%
\pgfsetfillcolor{currentfill}%
\pgfsetlinewidth{0.803000pt}%
\definecolor{currentstroke}{rgb}{0.000000,0.000000,0.000000}%
\pgfsetstrokecolor{currentstroke}%
\pgfsetdash{}{0pt}%
\pgfsys@defobject{currentmarker}{\pgfqpoint{-0.048611in}{0.000000in}}{\pgfqpoint{0.000000in}{0.000000in}}{%
\pgfpathmoveto{\pgfqpoint{0.000000in}{0.000000in}}%
\pgfpathlineto{\pgfqpoint{-0.048611in}{0.000000in}}%
\pgfusepath{stroke,fill}%
}%
\begin{pgfscope}%
\pgfsys@transformshift{0.750000in}{3.432000in}%
\pgfsys@useobject{currentmarker}{}%
\end{pgfscope}%
\end{pgfscope}%
\begin{pgfscope}%
\pgftext[x=0.475308in,y=3.383806in,left,base]{\sffamily\fontsize{10.000000}{12.000000}\selectfont \(\displaystyle 1.5\)}%
\end{pgfscope}%
\begin{pgfscope}%
\pgfpathrectangle{\pgfqpoint{0.750000in}{0.660000in}}{\pgfqpoint{4.650000in}{4.620000in}} %
\pgfusepath{clip}%
\pgfsetrectcap%
\pgfsetroundjoin%
\pgfsetlinewidth{0.803000pt}%
\definecolor{currentstroke}{rgb}{0.690196,0.690196,0.690196}%
\pgfsetstrokecolor{currentstroke}%
\pgfsetdash{}{0pt}%
\pgfpathmoveto{\pgfqpoint{0.750000in}{4.356000in}}%
\pgfpathlineto{\pgfqpoint{5.400000in}{4.356000in}}%
\pgfusepath{stroke}%
\end{pgfscope}%
\begin{pgfscope}%
\pgfsetbuttcap%
\pgfsetroundjoin%
\definecolor{currentfill}{rgb}{0.000000,0.000000,0.000000}%
\pgfsetfillcolor{currentfill}%
\pgfsetlinewidth{0.803000pt}%
\definecolor{currentstroke}{rgb}{0.000000,0.000000,0.000000}%
\pgfsetstrokecolor{currentstroke}%
\pgfsetdash{}{0pt}%
\pgfsys@defobject{currentmarker}{\pgfqpoint{-0.048611in}{0.000000in}}{\pgfqpoint{0.000000in}{0.000000in}}{%
\pgfpathmoveto{\pgfqpoint{0.000000in}{0.000000in}}%
\pgfpathlineto{\pgfqpoint{-0.048611in}{0.000000in}}%
\pgfusepath{stroke,fill}%
}%
\begin{pgfscope}%
\pgfsys@transformshift{0.750000in}{4.356000in}%
\pgfsys@useobject{currentmarker}{}%
\end{pgfscope}%
\end{pgfscope}%
\begin{pgfscope}%
\pgftext[x=0.475308in,y=4.307806in,left,base]{\sffamily\fontsize{10.000000}{12.000000}\selectfont \(\displaystyle 2.0\)}%
\end{pgfscope}%
\begin{pgfscope}%
\pgfpathrectangle{\pgfqpoint{0.750000in}{0.660000in}}{\pgfqpoint{4.650000in}{4.620000in}} %
\pgfusepath{clip}%
\pgfsetrectcap%
\pgfsetroundjoin%
\pgfsetlinewidth{0.803000pt}%
\definecolor{currentstroke}{rgb}{0.690196,0.690196,0.690196}%
\pgfsetstrokecolor{currentstroke}%
\pgfsetdash{}{0pt}%
\pgfpathmoveto{\pgfqpoint{0.750000in}{5.280000in}}%
\pgfpathlineto{\pgfqpoint{5.400000in}{5.280000in}}%
\pgfusepath{stroke}%
\end{pgfscope}%
\begin{pgfscope}%
\pgfsetbuttcap%
\pgfsetroundjoin%
\definecolor{currentfill}{rgb}{0.000000,0.000000,0.000000}%
\pgfsetfillcolor{currentfill}%
\pgfsetlinewidth{0.803000pt}%
\definecolor{currentstroke}{rgb}{0.000000,0.000000,0.000000}%
\pgfsetstrokecolor{currentstroke}%
\pgfsetdash{}{0pt}%
\pgfsys@defobject{currentmarker}{\pgfqpoint{-0.048611in}{0.000000in}}{\pgfqpoint{0.000000in}{0.000000in}}{%
\pgfpathmoveto{\pgfqpoint{0.000000in}{0.000000in}}%
\pgfpathlineto{\pgfqpoint{-0.048611in}{0.000000in}}%
\pgfusepath{stroke,fill}%
}%
\begin{pgfscope}%
\pgfsys@transformshift{0.750000in}{5.280000in}%
\pgfsys@useobject{currentmarker}{}%
\end{pgfscope}%
\end{pgfscope}%
\begin{pgfscope}%
\pgftext[x=0.475308in,y=5.231806in,left,base]{\sffamily\fontsize{10.000000}{12.000000}\selectfont \(\displaystyle 2.5\)}%
\end{pgfscope}%
\begin{pgfscope}%
\pgftext[x=0.419752in,y=2.970000in,,bottom,rotate=90.000000]{\sffamily\fontsize{10.000000}{12.000000}\selectfont \(\displaystyle \Delta V\) (km/s)}%
\end{pgfscope}%
\begin{pgfscope}%
\pgfpathrectangle{\pgfqpoint{0.750000in}{0.660000in}}{\pgfqpoint{4.650000in}{4.620000in}} %
\pgfusepath{clip}%
\pgfsetrectcap%
\pgfsetroundjoin%
\pgfsetlinewidth{1.505625pt}%
\definecolor{currentstroke}{rgb}{0.121569,0.466667,0.705882}%
\pgfsetstrokecolor{currentstroke}%
\pgfsetdash{}{0pt}%
\pgfpathmoveto{\pgfqpoint{0.750000in}{1.039433in}}%
\pgfpathlineto{\pgfqpoint{0.844898in}{1.051119in}}%
\pgfpathlineto{\pgfqpoint{0.939796in}{1.084252in}}%
\pgfpathlineto{\pgfqpoint{1.034694in}{1.134358in}}%
\pgfpathlineto{\pgfqpoint{1.129592in}{1.196705in}}%
\pgfpathlineto{\pgfqpoint{1.224490in}{1.267534in}}%
\pgfpathlineto{\pgfqpoint{1.319388in}{1.344218in}}%
\pgfpathlineto{\pgfqpoint{1.414286in}{1.424998in}}%
\pgfpathlineto{\pgfqpoint{1.509184in}{1.508704in}}%
\pgfpathlineto{\pgfqpoint{1.604082in}{1.594552in}}%
\pgfpathlineto{\pgfqpoint{1.698980in}{1.682000in}}%
\pgfpathlineto{\pgfqpoint{1.793878in}{1.770672in}}%
\pgfpathlineto{\pgfqpoint{1.888776in}{1.860296in}}%
\pgfpathlineto{\pgfqpoint{1.983673in}{1.950674in}}%
\pgfpathlineto{\pgfqpoint{2.078571in}{2.041658in}}%
\pgfpathlineto{\pgfqpoint{2.173469in}{2.133136in}}%
\pgfpathlineto{\pgfqpoint{2.268367in}{2.225020in}}%
\pgfpathlineto{\pgfqpoint{2.363265in}{2.317244in}}%
\pgfpathlineto{\pgfqpoint{2.458163in}{2.409754in}}%
\pgfpathlineto{\pgfqpoint{2.553061in}{2.502506in}}%
\pgfpathlineto{\pgfqpoint{2.647959in}{2.595467in}}%
\pgfpathlineto{\pgfqpoint{2.742857in}{2.688606in}}%
\pgfpathlineto{\pgfqpoint{2.837755in}{2.781901in}}%
\pgfpathlineto{\pgfqpoint{2.932653in}{2.875332in}}%
\pgfpathlineto{\pgfqpoint{3.027551in}{2.968882in}}%
\pgfpathlineto{\pgfqpoint{3.122449in}{3.062538in}}%
\pgfpathlineto{\pgfqpoint{3.217347in}{3.156288in}}%
\pgfpathlineto{\pgfqpoint{3.312245in}{3.250121in}}%
\pgfpathlineto{\pgfqpoint{3.407143in}{3.344029in}}%
\pgfpathlineto{\pgfqpoint{3.502041in}{3.438003in}}%
\pgfpathlineto{\pgfqpoint{3.596939in}{3.532039in}}%
\pgfpathlineto{\pgfqpoint{3.691837in}{3.626129in}}%
\pgfpathlineto{\pgfqpoint{3.786735in}{3.720269in}}%
\pgfpathlineto{\pgfqpoint{3.881633in}{3.814454in}}%
\pgfpathlineto{\pgfqpoint{3.976531in}{3.908681in}}%
\pgfpathlineto{\pgfqpoint{4.071429in}{4.002945in}}%
\pgfpathlineto{\pgfqpoint{4.166327in}{4.097244in}}%
\pgfpathlineto{\pgfqpoint{4.261224in}{4.191575in}}%
\pgfpathlineto{\pgfqpoint{4.356122in}{4.285936in}}%
\pgfpathlineto{\pgfqpoint{4.451020in}{4.380324in}}%
\pgfpathlineto{\pgfqpoint{4.545918in}{4.474737in}}%
\pgfpathlineto{\pgfqpoint{4.640816in}{4.569173in}}%
\pgfpathlineto{\pgfqpoint{4.735714in}{4.663631in}}%
\pgfpathlineto{\pgfqpoint{4.830612in}{4.758109in}}%
\pgfpathlineto{\pgfqpoint{4.925510in}{4.852606in}}%
\pgfpathlineto{\pgfqpoint{5.020408in}{4.947121in}}%
\pgfpathlineto{\pgfqpoint{5.115306in}{5.041652in}}%
\pgfpathlineto{\pgfqpoint{5.210204in}{5.136198in}}%
\pgfpathlineto{\pgfqpoint{5.305102in}{5.230760in}}%
\pgfpathlineto{\pgfqpoint{5.364545in}{5.290000in}}%
\pgfusepath{stroke}%
\end{pgfscope}%
\begin{pgfscope}%
\pgfpathrectangle{\pgfqpoint{0.750000in}{0.660000in}}{\pgfqpoint{4.650000in}{4.620000in}} %
\pgfusepath{clip}%
\pgfsetrectcap%
\pgfsetroundjoin%
\pgfsetlinewidth{1.505625pt}%
\definecolor{currentstroke}{rgb}{1.000000,0.498039,0.054902}%
\pgfsetstrokecolor{currentstroke}%
\pgfsetdash{}{0pt}%
\pgfpathmoveto{\pgfqpoint{0.750000in}{1.422741in}}%
\pgfpathlineto{\pgfqpoint{0.844898in}{1.428446in}}%
\pgfpathlineto{\pgfqpoint{0.939796in}{1.445314in}}%
\pgfpathlineto{\pgfqpoint{1.034694in}{1.472648in}}%
\pgfpathlineto{\pgfqpoint{1.129592in}{1.509440in}}%
\pgfpathlineto{\pgfqpoint{1.224490in}{1.554522in}}%
\pgfpathlineto{\pgfqpoint{1.319388in}{1.606713in}}%
\pgfpathlineto{\pgfqpoint{1.414286in}{1.664904in}}%
\pgfpathlineto{\pgfqpoint{1.509184in}{1.728115in}}%
\pgfpathlineto{\pgfqpoint{1.604082in}{1.795508in}}%
\pgfpathlineto{\pgfqpoint{1.698980in}{1.866383in}}%
\pgfpathlineto{\pgfqpoint{1.793878in}{1.940161in}}%
\pgfpathlineto{\pgfqpoint{1.888776in}{2.016369in}}%
\pgfpathlineto{\pgfqpoint{1.983673in}{2.094620in}}%
\pgfpathlineto{\pgfqpoint{2.078571in}{2.174596in}}%
\pgfpathlineto{\pgfqpoint{2.173469in}{2.256039in}}%
\pgfpathlineto{\pgfqpoint{2.268367in}{2.338736in}}%
\pgfpathlineto{\pgfqpoint{2.363265in}{2.422509in}}%
\pgfpathlineto{\pgfqpoint{2.458163in}{2.507212in}}%
\pgfpathlineto{\pgfqpoint{2.553061in}{2.592723in}}%
\pgfpathlineto{\pgfqpoint{2.647959in}{2.678940in}}%
\pgfpathlineto{\pgfqpoint{2.742857in}{2.765775in}}%
\pgfpathlineto{\pgfqpoint{2.837755in}{2.853155in}}%
\pgfpathlineto{\pgfqpoint{2.932653in}{2.941018in}}%
\pgfpathlineto{\pgfqpoint{3.027551in}{3.029310in}}%
\pgfpathlineto{\pgfqpoint{3.122449in}{3.117984in}}%
\pgfpathlineto{\pgfqpoint{3.217347in}{3.207001in}}%
\pgfpathlineto{\pgfqpoint{3.312245in}{3.296326in}}%
\pgfpathlineto{\pgfqpoint{3.407143in}{3.385929in}}%
\pgfpathlineto{\pgfqpoint{3.502041in}{3.475782in}}%
\pgfpathlineto{\pgfqpoint{3.596939in}{3.565864in}}%
\pgfpathlineto{\pgfqpoint{3.691837in}{3.656153in}}%
\pgfpathlineto{\pgfqpoint{3.786735in}{3.746631in}}%
\pgfpathlineto{\pgfqpoint{3.881633in}{3.837282in}}%
\pgfpathlineto{\pgfqpoint{3.976531in}{3.928092in}}%
\pgfpathlineto{\pgfqpoint{4.071429in}{4.019047in}}%
\pgfpathlineto{\pgfqpoint{4.166327in}{4.110137in}}%
\pgfpathlineto{\pgfqpoint{4.261224in}{4.201350in}}%
\pgfpathlineto{\pgfqpoint{4.356122in}{4.292678in}}%
\pgfpathlineto{\pgfqpoint{4.451020in}{4.384112in}}%
\pgfpathlineto{\pgfqpoint{4.545918in}{4.475645in}}%
\pgfpathlineto{\pgfqpoint{4.640816in}{4.567269in}}%
\pgfpathlineto{\pgfqpoint{4.735714in}{4.658978in}}%
\pgfpathlineto{\pgfqpoint{4.830612in}{4.750767in}}%
\pgfpathlineto{\pgfqpoint{4.925510in}{4.842630in}}%
\pgfpathlineto{\pgfqpoint{5.020408in}{4.934563in}}%
\pgfpathlineto{\pgfqpoint{5.115306in}{5.026560in}}%
\pgfpathlineto{\pgfqpoint{5.210204in}{5.118619in}}%
\pgfpathlineto{\pgfqpoint{5.305102in}{5.210735in}}%
\pgfpathlineto{\pgfqpoint{5.386713in}{5.290000in}}%
\pgfusepath{stroke}%
\end{pgfscope}%
\begin{pgfscope}%
\pgfpathrectangle{\pgfqpoint{0.750000in}{0.660000in}}{\pgfqpoint{4.650000in}{4.620000in}} %
\pgfusepath{clip}%
\pgfsetrectcap%
\pgfsetroundjoin%
\pgfsetlinewidth{1.505625pt}%
\definecolor{currentstroke}{rgb}{0.172549,0.627451,0.172549}%
\pgfsetstrokecolor{currentstroke}%
\pgfsetdash{}{0pt}%
\pgfpathmoveto{\pgfqpoint{0.750000in}{2.218820in}}%
\pgfpathlineto{\pgfqpoint{0.844898in}{2.221288in}}%
\pgfpathlineto{\pgfqpoint{0.939796in}{2.228667in}}%
\pgfpathlineto{\pgfqpoint{1.034694in}{2.240889in}}%
\pgfpathlineto{\pgfqpoint{1.129592in}{2.257843in}}%
\pgfpathlineto{\pgfqpoint{1.224490in}{2.279380in}}%
\pgfpathlineto{\pgfqpoint{1.319388in}{2.305320in}}%
\pgfpathlineto{\pgfqpoint{1.414286in}{2.335460in}}%
\pgfpathlineto{\pgfqpoint{1.509184in}{2.369575in}}%
\pgfpathlineto{\pgfqpoint{1.604082in}{2.407435in}}%
\pgfpathlineto{\pgfqpoint{1.698980in}{2.448800in}}%
\pgfpathlineto{\pgfqpoint{1.793878in}{2.493434in}}%
\pgfpathlineto{\pgfqpoint{1.888776in}{2.541105in}}%
\pgfpathlineto{\pgfqpoint{1.983673in}{2.591587in}}%
\pgfpathlineto{\pgfqpoint{2.078571in}{2.644665in}}%
\pgfpathlineto{\pgfqpoint{2.173469in}{2.700138in}}%
\pgfpathlineto{\pgfqpoint{2.268367in}{2.757815in}}%
\pgfpathlineto{\pgfqpoint{2.363265in}{2.817519in}}%
\pgfpathlineto{\pgfqpoint{2.458163in}{2.879087in}}%
\pgfpathlineto{\pgfqpoint{2.553061in}{2.942368in}}%
\pgfpathlineto{\pgfqpoint{2.647959in}{3.007224in}}%
\pgfpathlineto{\pgfqpoint{2.742857in}{3.073527in}}%
\pgfpathlineto{\pgfqpoint{2.837755in}{3.141162in}}%
\pgfpathlineto{\pgfqpoint{2.932653in}{3.210022in}}%
\pgfpathlineto{\pgfqpoint{3.027551in}{3.280011in}}%
\pgfpathlineto{\pgfqpoint{3.122449in}{3.351041in}}%
\pgfpathlineto{\pgfqpoint{3.217347in}{3.423032in}}%
\pgfpathlineto{\pgfqpoint{3.312245in}{3.495909in}}%
\pgfpathlineto{\pgfqpoint{3.407143in}{3.569608in}}%
\pgfpathlineto{\pgfqpoint{3.502041in}{3.644067in}}%
\pgfpathlineto{\pgfqpoint{3.596939in}{3.719230in}}%
\pgfpathlineto{\pgfqpoint{3.691837in}{3.795046in}}%
\pgfpathlineto{\pgfqpoint{3.786735in}{3.871471in}}%
\pgfpathlineto{\pgfqpoint{3.881633in}{3.948460in}}%
\pgfpathlineto{\pgfqpoint{3.976531in}{4.025976in}}%
\pgfpathlineto{\pgfqpoint{4.071429in}{4.103982in}}%
\pgfpathlineto{\pgfqpoint{4.166327in}{4.182447in}}%
\pgfpathlineto{\pgfqpoint{4.261224in}{4.261340in}}%
\pgfpathlineto{\pgfqpoint{4.356122in}{4.340633in}}%
\pgfpathlineto{\pgfqpoint{4.451020in}{4.420302in}}%
\pgfpathlineto{\pgfqpoint{4.545918in}{4.500323in}}%
\pgfpathlineto{\pgfqpoint{4.640816in}{4.580674in}}%
\pgfpathlineto{\pgfqpoint{4.735714in}{4.661336in}}%
\pgfpathlineto{\pgfqpoint{4.830612in}{4.742290in}}%
\pgfpathlineto{\pgfqpoint{4.925510in}{4.823519in}}%
\pgfpathlineto{\pgfqpoint{5.020408in}{4.905008in}}%
\pgfpathlineto{\pgfqpoint{5.115306in}{4.986741in}}%
\pgfpathlineto{\pgfqpoint{5.210204in}{5.068705in}}%
\pgfpathlineto{\pgfqpoint{5.305102in}{5.150888in}}%
\pgfpathlineto{\pgfqpoint{5.400000in}{5.233277in}}%
\pgfusepath{stroke}%
\end{pgfscope}%
\begin{pgfscope}%
\pgfpathrectangle{\pgfqpoint{0.750000in}{0.660000in}}{\pgfqpoint{4.650000in}{4.620000in}} %
\pgfusepath{clip}%
\pgfsetrectcap%
\pgfsetroundjoin%
\pgfsetlinewidth{1.505625pt}%
\definecolor{currentstroke}{rgb}{0.839216,0.152941,0.156863}%
\pgfsetstrokecolor{currentstroke}%
\pgfsetdash{}{0pt}%
\pgfpathmoveto{\pgfqpoint{0.750000in}{3.097574in}}%
\pgfpathlineto{\pgfqpoint{0.844898in}{3.098823in}}%
\pgfpathlineto{\pgfqpoint{0.939796in}{3.102567in}}%
\pgfpathlineto{\pgfqpoint{1.034694in}{3.108795in}}%
\pgfpathlineto{\pgfqpoint{1.129592in}{3.117487in}}%
\pgfpathlineto{\pgfqpoint{1.224490in}{3.128617in}}%
\pgfpathlineto{\pgfqpoint{1.319388in}{3.142153in}}%
\pgfpathlineto{\pgfqpoint{1.414286in}{3.158056in}}%
\pgfpathlineto{\pgfqpoint{1.509184in}{3.176280in}}%
\pgfpathlineto{\pgfqpoint{1.604082in}{3.196776in}}%
\pgfpathlineto{\pgfqpoint{1.698980in}{3.219489in}}%
\pgfpathlineto{\pgfqpoint{1.793878in}{3.244360in}}%
\pgfpathlineto{\pgfqpoint{1.888776in}{3.271329in}}%
\pgfpathlineto{\pgfqpoint{1.983673in}{3.300330in}}%
\pgfpathlineto{\pgfqpoint{2.078571in}{3.331298in}}%
\pgfpathlineto{\pgfqpoint{2.173469in}{3.364164in}}%
\pgfpathlineto{\pgfqpoint{2.268367in}{3.398862in}}%
\pgfpathlineto{\pgfqpoint{2.363265in}{3.435321in}}%
\pgfpathlineto{\pgfqpoint{2.458163in}{3.473473in}}%
\pgfpathlineto{\pgfqpoint{2.553061in}{3.513251in}}%
\pgfpathlineto{\pgfqpoint{2.647959in}{3.554587in}}%
\pgfpathlineto{\pgfqpoint{2.742857in}{3.597416in}}%
\pgfpathlineto{\pgfqpoint{2.837755in}{3.641674in}}%
\pgfpathlineto{\pgfqpoint{2.932653in}{3.687297in}}%
\pgfpathlineto{\pgfqpoint{3.027551in}{3.734225in}}%
\pgfpathlineto{\pgfqpoint{3.122449in}{3.782399in}}%
\pgfpathlineto{\pgfqpoint{3.217347in}{3.831762in}}%
\pgfpathlineto{\pgfqpoint{3.312245in}{3.882261in}}%
\pgfpathlineto{\pgfqpoint{3.407143in}{3.933841in}}%
\pgfpathlineto{\pgfqpoint{3.502041in}{3.986453in}}%
\pgfpathlineto{\pgfqpoint{3.596939in}{4.040048in}}%
\pgfpathlineto{\pgfqpoint{3.691837in}{4.094581in}}%
\pgfpathlineto{\pgfqpoint{3.786735in}{4.150008in}}%
\pgfpathlineto{\pgfqpoint{3.881633in}{4.206286in}}%
\pgfpathlineto{\pgfqpoint{3.976531in}{4.263377in}}%
\pgfpathlineto{\pgfqpoint{4.071429in}{4.321241in}}%
\pgfpathlineto{\pgfqpoint{4.166327in}{4.379842in}}%
\pgfpathlineto{\pgfqpoint{4.261224in}{4.439148in}}%
\pgfpathlineto{\pgfqpoint{4.356122in}{4.499124in}}%
\pgfpathlineto{\pgfqpoint{4.451020in}{4.559739in}}%
\pgfpathlineto{\pgfqpoint{4.545918in}{4.620966in}}%
\pgfpathlineto{\pgfqpoint{4.640816in}{4.682775in}}%
\pgfpathlineto{\pgfqpoint{4.735714in}{4.745140in}}%
\pgfpathlineto{\pgfqpoint{4.830612in}{4.808036in}}%
\pgfpathlineto{\pgfqpoint{4.925510in}{4.871439in}}%
\pgfpathlineto{\pgfqpoint{5.020408in}{4.935327in}}%
\pgfpathlineto{\pgfqpoint{5.115306in}{4.999679in}}%
\pgfpathlineto{\pgfqpoint{5.210204in}{5.064473in}}%
\pgfpathlineto{\pgfqpoint{5.305102in}{5.129692in}}%
\pgfpathlineto{\pgfqpoint{5.400000in}{5.195315in}}%
\pgfusepath{stroke}%
\end{pgfscope}%
\begin{pgfscope}%
\pgfpathrectangle{\pgfqpoint{0.750000in}{0.660000in}}{\pgfqpoint{4.650000in}{4.620000in}} %
\pgfusepath{clip}%
\pgfsetrectcap%
\pgfsetroundjoin%
\pgfsetlinewidth{1.505625pt}%
\definecolor{currentstroke}{rgb}{0.580392,0.403922,0.741176}%
\pgfsetstrokecolor{currentstroke}%
\pgfsetdash{}{0pt}%
\pgfpathmoveto{\pgfqpoint{0.750000in}{4.172582in}}%
\pgfpathlineto{\pgfqpoint{0.844898in}{4.173152in}}%
\pgfpathlineto{\pgfqpoint{0.939796in}{4.174862in}}%
\pgfpathlineto{\pgfqpoint{1.034694in}{4.177711in}}%
\pgfpathlineto{\pgfqpoint{1.129592in}{4.181695in}}%
\pgfpathlineto{\pgfqpoint{1.224490in}{4.186810in}}%
\pgfpathlineto{\pgfqpoint{1.319388in}{4.193053in}}%
\pgfpathlineto{\pgfqpoint{1.414286in}{4.200416in}}%
\pgfpathlineto{\pgfqpoint{1.509184in}{4.208893in}}%
\pgfpathlineto{\pgfqpoint{1.604082in}{4.218475in}}%
\pgfpathlineto{\pgfqpoint{1.698980in}{4.229155in}}%
\pgfpathlineto{\pgfqpoint{1.793878in}{4.240922in}}%
\pgfpathlineto{\pgfqpoint{1.888776in}{4.253765in}}%
\pgfpathlineto{\pgfqpoint{1.983673in}{4.267673in}}%
\pgfpathlineto{\pgfqpoint{2.078571in}{4.282634in}}%
\pgfpathlineto{\pgfqpoint{2.173469in}{4.298634in}}%
\pgfpathlineto{\pgfqpoint{2.268367in}{4.315661in}}%
\pgfpathlineto{\pgfqpoint{2.363265in}{4.333699in}}%
\pgfpathlineto{\pgfqpoint{2.458163in}{4.352734in}}%
\pgfpathlineto{\pgfqpoint{2.553061in}{4.372751in}}%
\pgfpathlineto{\pgfqpoint{2.647959in}{4.393734in}}%
\pgfpathlineto{\pgfqpoint{2.742857in}{4.415666in}}%
\pgfpathlineto{\pgfqpoint{2.837755in}{4.438532in}}%
\pgfpathlineto{\pgfqpoint{2.932653in}{4.462313in}}%
\pgfpathlineto{\pgfqpoint{3.027551in}{4.486994in}}%
\pgfpathlineto{\pgfqpoint{3.122449in}{4.512557in}}%
\pgfpathlineto{\pgfqpoint{3.217347in}{4.538984in}}%
\pgfpathlineto{\pgfqpoint{3.312245in}{4.566258in}}%
\pgfpathlineto{\pgfqpoint{3.407143in}{4.594362in}}%
\pgfpathlineto{\pgfqpoint{3.502041in}{4.623277in}}%
\pgfpathlineto{\pgfqpoint{3.596939in}{4.652986in}}%
\pgfpathlineto{\pgfqpoint{3.691837in}{4.683472in}}%
\pgfpathlineto{\pgfqpoint{3.786735in}{4.714716in}}%
\pgfpathlineto{\pgfqpoint{3.881633in}{4.746702in}}%
\pgfpathlineto{\pgfqpoint{3.976531in}{4.779413in}}%
\pgfpathlineto{\pgfqpoint{4.071429in}{4.812830in}}%
\pgfpathlineto{\pgfqpoint{4.166327in}{4.846938in}}%
\pgfpathlineto{\pgfqpoint{4.261224in}{4.881719in}}%
\pgfpathlineto{\pgfqpoint{4.356122in}{4.917157in}}%
\pgfpathlineto{\pgfqpoint{4.451020in}{4.953236in}}%
\pgfpathlineto{\pgfqpoint{4.545918in}{4.989940in}}%
\pgfpathlineto{\pgfqpoint{4.640816in}{5.027252in}}%
\pgfpathlineto{\pgfqpoint{4.735714in}{5.065158in}}%
\pgfpathlineto{\pgfqpoint{4.830612in}{5.103642in}}%
\pgfpathlineto{\pgfqpoint{4.925510in}{5.142690in}}%
\pgfpathlineto{\pgfqpoint{5.020408in}{5.182286in}}%
\pgfpathlineto{\pgfqpoint{5.115306in}{5.222417in}}%
\pgfpathlineto{\pgfqpoint{5.210204in}{5.263068in}}%
\pgfpathlineto{\pgfqpoint{5.272300in}{5.290000in}}%
\pgfusepath{stroke}%
\end{pgfscope}%
\begin{pgfscope}%
\pgfsetrectcap%
\pgfsetmiterjoin%
\pgfsetlinewidth{0.803000pt}%
\definecolor{currentstroke}{rgb}{0.000000,0.000000,0.000000}%
\pgfsetstrokecolor{currentstroke}%
\pgfsetdash{}{0pt}%
\pgfpathmoveto{\pgfqpoint{0.750000in}{0.660000in}}%
\pgfpathlineto{\pgfqpoint{0.750000in}{5.280000in}}%
\pgfusepath{stroke}%
\end{pgfscope}%
\begin{pgfscope}%
\pgfsetrectcap%
\pgfsetmiterjoin%
\pgfsetlinewidth{0.803000pt}%
\definecolor{currentstroke}{rgb}{0.000000,0.000000,0.000000}%
\pgfsetstrokecolor{currentstroke}%
\pgfsetdash{}{0pt}%
\pgfpathmoveto{\pgfqpoint{5.400000in}{0.660000in}}%
\pgfpathlineto{\pgfqpoint{5.400000in}{5.280000in}}%
\pgfusepath{stroke}%
\end{pgfscope}%
\begin{pgfscope}%
\pgfsetrectcap%
\pgfsetmiterjoin%
\pgfsetlinewidth{0.803000pt}%
\definecolor{currentstroke}{rgb}{0.000000,0.000000,0.000000}%
\pgfsetstrokecolor{currentstroke}%
\pgfsetdash{}{0pt}%
\pgfpathmoveto{\pgfqpoint{0.750000in}{0.660000in}}%
\pgfpathlineto{\pgfqpoint{5.400000in}{0.660000in}}%
\pgfusepath{stroke}%
\end{pgfscope}%
\begin{pgfscope}%
\pgfsetrectcap%
\pgfsetmiterjoin%
\pgfsetlinewidth{0.803000pt}%
\definecolor{currentstroke}{rgb}{0.000000,0.000000,0.000000}%
\pgfsetstrokecolor{currentstroke}%
\pgfsetdash{}{0pt}%
\pgfpathmoveto{\pgfqpoint{0.750000in}{5.280000in}}%
\pgfpathlineto{\pgfqpoint{5.400000in}{5.280000in}}%
\pgfusepath{stroke}%
\end{pgfscope}%
\begin{pgfscope}%
\pgfsetbuttcap%
\pgfsetmiterjoin%
\definecolor{currentfill}{rgb}{1.000000,1.000000,1.000000}%
\pgfsetfillcolor{currentfill}%
\pgfsetfillopacity{0.800000}%
\pgfsetlinewidth{1.003750pt}%
\definecolor{currentstroke}{rgb}{0.800000,0.800000,0.800000}%
\pgfsetstrokecolor{currentstroke}%
\pgfsetstrokeopacity{0.800000}%
\pgfsetdash{}{0pt}%
\pgfpathmoveto{\pgfqpoint{4.431010in}{0.729444in}}%
\pgfpathlineto{\pgfqpoint{5.302778in}{0.729444in}}%
\pgfpathquadraticcurveto{\pgfqpoint{5.330556in}{0.729444in}}{\pgfqpoint{5.330556in}{0.757222in}}%
\pgfpathlineto{\pgfqpoint{5.330556in}{1.711388in}}%
\pgfpathquadraticcurveto{\pgfqpoint{5.330556in}{1.739166in}}{\pgfqpoint{5.302778in}{1.739166in}}%
\pgfpathlineto{\pgfqpoint{4.431010in}{1.739166in}}%
\pgfpathquadraticcurveto{\pgfqpoint{4.403232in}{1.739166in}}{\pgfqpoint{4.403232in}{1.711388in}}%
\pgfpathlineto{\pgfqpoint{4.403232in}{0.757222in}}%
\pgfpathquadraticcurveto{\pgfqpoint{4.403232in}{0.729444in}}{\pgfqpoint{4.431010in}{0.729444in}}%
\pgfpathclose%
\pgfusepath{stroke,fill}%
\end{pgfscope}%
\begin{pgfscope}%
\pgfsetrectcap%
\pgfsetroundjoin%
\pgfsetlinewidth{1.505625pt}%
\definecolor{currentstroke}{rgb}{0.121569,0.466667,0.705882}%
\pgfsetstrokecolor{currentstroke}%
\pgfsetdash{}{0pt}%
\pgfpathmoveto{\pgfqpoint{4.458787in}{1.634999in}}%
\pgfpathlineto{\pgfqpoint{4.736565in}{1.634999in}}%
\pgfusepath{stroke}%
\end{pgfscope}%
\begin{pgfscope}%
\pgftext[x=4.847676in,y=1.586388in,left,base]{\sffamily\fontsize{10.000000}{12.000000}\selectfont \(\displaystyle e = 0.1\)}%
\end{pgfscope}%
\begin{pgfscope}%
\pgfsetrectcap%
\pgfsetroundjoin%
\pgfsetlinewidth{1.505625pt}%
\definecolor{currentstroke}{rgb}{1.000000,0.498039,0.054902}%
\pgfsetstrokecolor{currentstroke}%
\pgfsetdash{}{0pt}%
\pgfpathmoveto{\pgfqpoint{4.458787in}{1.441388in}}%
\pgfpathlineto{\pgfqpoint{4.736565in}{1.441388in}}%
\pgfusepath{stroke}%
\end{pgfscope}%
\begin{pgfscope}%
\pgftext[x=4.847676in,y=1.392777in,left,base]{\sffamily\fontsize{10.000000}{12.000000}\selectfont \(\displaystyle e = 0.2\)}%
\end{pgfscope}%
\begin{pgfscope}%
\pgfsetrectcap%
\pgfsetroundjoin%
\pgfsetlinewidth{1.505625pt}%
\definecolor{currentstroke}{rgb}{0.172549,0.627451,0.172549}%
\pgfsetstrokecolor{currentstroke}%
\pgfsetdash{}{0pt}%
\pgfpathmoveto{\pgfqpoint{4.458787in}{1.247777in}}%
\pgfpathlineto{\pgfqpoint{4.736565in}{1.247777in}}%
\pgfusepath{stroke}%
\end{pgfscope}%
\begin{pgfscope}%
\pgftext[x=4.847676in,y=1.199166in,left,base]{\sffamily\fontsize{10.000000}{12.000000}\selectfont \(\displaystyle e = 0.4\)}%
\end{pgfscope}%
\begin{pgfscope}%
\pgfsetrectcap%
\pgfsetroundjoin%
\pgfsetlinewidth{1.505625pt}%
\definecolor{currentstroke}{rgb}{0.839216,0.152941,0.156863}%
\pgfsetstrokecolor{currentstroke}%
\pgfsetdash{}{0pt}%
\pgfpathmoveto{\pgfqpoint{4.458787in}{1.054166in}}%
\pgfpathlineto{\pgfqpoint{4.736565in}{1.054166in}}%
\pgfusepath{stroke}%
\end{pgfscope}%
\begin{pgfscope}%
\pgftext[x=4.847676in,y=1.005555in,left,base]{\sffamily\fontsize{10.000000}{12.000000}\selectfont \(\displaystyle e = 0.6\)}%
\end{pgfscope}%
\begin{pgfscope}%
\pgfsetrectcap%
\pgfsetroundjoin%
\pgfsetlinewidth{1.505625pt}%
\definecolor{currentstroke}{rgb}{0.580392,0.403922,0.741176}%
\pgfsetstrokecolor{currentstroke}%
\pgfsetdash{}{0pt}%
\pgfpathmoveto{\pgfqpoint{4.458787in}{0.860555in}}%
\pgfpathlineto{\pgfqpoint{4.736565in}{0.860555in}}%
\pgfusepath{stroke}%
\end{pgfscope}%
\begin{pgfscope}%
\pgftext[x=4.847676in,y=0.811944in,left,base]{\sffamily\fontsize{10.000000}{12.000000}\selectfont \(\displaystyle e = 0.8\)}%
\end{pgfscope}%
\end{pgfpicture}%
\makeatother%
\endgroup%

}
\end{subfigure}
\caption{Charts of yaw angle and velocity requirements in terms of inclination and eccentricity as found in  \cite{pollard2000simplified}.}
\label{fig:eccincnumcharts}
\end{figure}

On the other hand, we also represent the evolution of the eccentricity during the integration of the equations of motion versus the theoretical trend, given by:

\[
e = \sin{\left( \pm \frac{3}{2} f \cos{|\beta|} \sqrt{\frac{a}{\mu}} t + \arcsin{e_0} \right)}
\]

\begin{figure}[b]
\centering
\resizebox{0.8\textwidth}{!}{
%% Creator: Matplotlib, PGF backend
%%
%% To include the figure in your LaTeX document, write
%%   \input{<filename>.pgf}
%%
%% Make sure the required packages are loaded in your preamble
%%   \usepackage{pgf}
%%
%% Figures using additional raster images can only be included by \input if
%% they are in the same directory as the main LaTeX file. For loading figures
%% from other directories you can use the `import` package
%%   \usepackage{import}
%% and then include the figures with
%%   \import{<path to file>}{<filename>.pgf}
%%
%% Matplotlib used the following preamble
%%   \usepackage{fontspec}
%%
\begingroup%
\makeatletter%
\begin{pgfpicture}%
\pgfpathrectangle{\pgfpointorigin}{\pgfqpoint{6.400000in}{3.610000in}}%
\pgfusepath{use as bounding box, clip}%
\begin{pgfscope}%
\pgfsetbuttcap%
\pgfsetmiterjoin%
\definecolor{currentfill}{rgb}{1.000000,1.000000,1.000000}%
\pgfsetfillcolor{currentfill}%
\pgfsetlinewidth{0.000000pt}%
\definecolor{currentstroke}{rgb}{1.000000,1.000000,1.000000}%
\pgfsetstrokecolor{currentstroke}%
\pgfsetdash{}{0pt}%
\pgfpathmoveto{\pgfqpoint{0.000000in}{0.000000in}}%
\pgfpathlineto{\pgfqpoint{6.400000in}{0.000000in}}%
\pgfpathlineto{\pgfqpoint{6.400000in}{3.610000in}}%
\pgfpathlineto{\pgfqpoint{0.000000in}{3.610000in}}%
\pgfpathclose%
\pgfusepath{fill}%
\end{pgfscope}%
\begin{pgfscope}%
\pgfsetbuttcap%
\pgfsetmiterjoin%
\definecolor{currentfill}{rgb}{1.000000,1.000000,1.000000}%
\pgfsetfillcolor{currentfill}%
\pgfsetlinewidth{0.000000pt}%
\definecolor{currentstroke}{rgb}{0.000000,0.000000,0.000000}%
\pgfsetstrokecolor{currentstroke}%
\pgfsetstrokeopacity{0.000000}%
\pgfsetdash{}{0pt}%
\pgfpathmoveto{\pgfqpoint{0.800000in}{1.913300in}}%
\pgfpathlineto{\pgfqpoint{5.760000in}{1.913300in}}%
\pgfpathlineto{\pgfqpoint{5.760000in}{3.176800in}}%
\pgfpathlineto{\pgfqpoint{0.800000in}{3.176800in}}%
\pgfpathclose%
\pgfusepath{fill}%
\end{pgfscope}%
\begin{pgfscope}%
\pgfsetbuttcap%
\pgfsetroundjoin%
\definecolor{currentfill}{rgb}{0.000000,0.000000,0.000000}%
\pgfsetfillcolor{currentfill}%
\pgfsetlinewidth{0.803000pt}%
\definecolor{currentstroke}{rgb}{0.000000,0.000000,0.000000}%
\pgfsetstrokecolor{currentstroke}%
\pgfsetdash{}{0pt}%
\pgfsys@defobject{currentmarker}{\pgfqpoint{0.000000in}{-0.048611in}}{\pgfqpoint{0.000000in}{0.000000in}}{%
\pgfpathmoveto{\pgfqpoint{0.000000in}{0.000000in}}%
\pgfpathlineto{\pgfqpoint{0.000000in}{-0.048611in}}%
\pgfusepath{stroke,fill}%
}%
\begin{pgfscope}%
\pgfsys@transformshift{1.025455in}{1.913300in}%
\pgfsys@useobject{currentmarker}{}%
\end{pgfscope}%
\end{pgfscope}%
\begin{pgfscope}%
\pgfsetbuttcap%
\pgfsetroundjoin%
\definecolor{currentfill}{rgb}{0.000000,0.000000,0.000000}%
\pgfsetfillcolor{currentfill}%
\pgfsetlinewidth{0.803000pt}%
\definecolor{currentstroke}{rgb}{0.000000,0.000000,0.000000}%
\pgfsetstrokecolor{currentstroke}%
\pgfsetdash{}{0pt}%
\pgfsys@defobject{currentmarker}{\pgfqpoint{0.000000in}{-0.048611in}}{\pgfqpoint{0.000000in}{0.000000in}}{%
\pgfpathmoveto{\pgfqpoint{0.000000in}{0.000000in}}%
\pgfpathlineto{\pgfqpoint{0.000000in}{-0.048611in}}%
\pgfusepath{stroke,fill}%
}%
\begin{pgfscope}%
\pgfsys@transformshift{1.638398in}{1.913300in}%
\pgfsys@useobject{currentmarker}{}%
\end{pgfscope}%
\end{pgfscope}%
\begin{pgfscope}%
\pgfsetbuttcap%
\pgfsetroundjoin%
\definecolor{currentfill}{rgb}{0.000000,0.000000,0.000000}%
\pgfsetfillcolor{currentfill}%
\pgfsetlinewidth{0.803000pt}%
\definecolor{currentstroke}{rgb}{0.000000,0.000000,0.000000}%
\pgfsetstrokecolor{currentstroke}%
\pgfsetdash{}{0pt}%
\pgfsys@defobject{currentmarker}{\pgfqpoint{0.000000in}{-0.048611in}}{\pgfqpoint{0.000000in}{0.000000in}}{%
\pgfpathmoveto{\pgfqpoint{0.000000in}{0.000000in}}%
\pgfpathlineto{\pgfqpoint{0.000000in}{-0.048611in}}%
\pgfusepath{stroke,fill}%
}%
\begin{pgfscope}%
\pgfsys@transformshift{2.251341in}{1.913300in}%
\pgfsys@useobject{currentmarker}{}%
\end{pgfscope}%
\end{pgfscope}%
\begin{pgfscope}%
\pgfsetbuttcap%
\pgfsetroundjoin%
\definecolor{currentfill}{rgb}{0.000000,0.000000,0.000000}%
\pgfsetfillcolor{currentfill}%
\pgfsetlinewidth{0.803000pt}%
\definecolor{currentstroke}{rgb}{0.000000,0.000000,0.000000}%
\pgfsetstrokecolor{currentstroke}%
\pgfsetdash{}{0pt}%
\pgfsys@defobject{currentmarker}{\pgfqpoint{0.000000in}{-0.048611in}}{\pgfqpoint{0.000000in}{0.000000in}}{%
\pgfpathmoveto{\pgfqpoint{0.000000in}{0.000000in}}%
\pgfpathlineto{\pgfqpoint{0.000000in}{-0.048611in}}%
\pgfusepath{stroke,fill}%
}%
\begin{pgfscope}%
\pgfsys@transformshift{2.864284in}{1.913300in}%
\pgfsys@useobject{currentmarker}{}%
\end{pgfscope}%
\end{pgfscope}%
\begin{pgfscope}%
\pgfsetbuttcap%
\pgfsetroundjoin%
\definecolor{currentfill}{rgb}{0.000000,0.000000,0.000000}%
\pgfsetfillcolor{currentfill}%
\pgfsetlinewidth{0.803000pt}%
\definecolor{currentstroke}{rgb}{0.000000,0.000000,0.000000}%
\pgfsetstrokecolor{currentstroke}%
\pgfsetdash{}{0pt}%
\pgfsys@defobject{currentmarker}{\pgfqpoint{0.000000in}{-0.048611in}}{\pgfqpoint{0.000000in}{0.000000in}}{%
\pgfpathmoveto{\pgfqpoint{0.000000in}{0.000000in}}%
\pgfpathlineto{\pgfqpoint{0.000000in}{-0.048611in}}%
\pgfusepath{stroke,fill}%
}%
\begin{pgfscope}%
\pgfsys@transformshift{3.477227in}{1.913300in}%
\pgfsys@useobject{currentmarker}{}%
\end{pgfscope}%
\end{pgfscope}%
\begin{pgfscope}%
\pgfsetbuttcap%
\pgfsetroundjoin%
\definecolor{currentfill}{rgb}{0.000000,0.000000,0.000000}%
\pgfsetfillcolor{currentfill}%
\pgfsetlinewidth{0.803000pt}%
\definecolor{currentstroke}{rgb}{0.000000,0.000000,0.000000}%
\pgfsetstrokecolor{currentstroke}%
\pgfsetdash{}{0pt}%
\pgfsys@defobject{currentmarker}{\pgfqpoint{0.000000in}{-0.048611in}}{\pgfqpoint{0.000000in}{0.000000in}}{%
\pgfpathmoveto{\pgfqpoint{0.000000in}{0.000000in}}%
\pgfpathlineto{\pgfqpoint{0.000000in}{-0.048611in}}%
\pgfusepath{stroke,fill}%
}%
\begin{pgfscope}%
\pgfsys@transformshift{4.090170in}{1.913300in}%
\pgfsys@useobject{currentmarker}{}%
\end{pgfscope}%
\end{pgfscope}%
\begin{pgfscope}%
\pgfsetbuttcap%
\pgfsetroundjoin%
\definecolor{currentfill}{rgb}{0.000000,0.000000,0.000000}%
\pgfsetfillcolor{currentfill}%
\pgfsetlinewidth{0.803000pt}%
\definecolor{currentstroke}{rgb}{0.000000,0.000000,0.000000}%
\pgfsetstrokecolor{currentstroke}%
\pgfsetdash{}{0pt}%
\pgfsys@defobject{currentmarker}{\pgfqpoint{0.000000in}{-0.048611in}}{\pgfqpoint{0.000000in}{0.000000in}}{%
\pgfpathmoveto{\pgfqpoint{0.000000in}{0.000000in}}%
\pgfpathlineto{\pgfqpoint{0.000000in}{-0.048611in}}%
\pgfusepath{stroke,fill}%
}%
\begin{pgfscope}%
\pgfsys@transformshift{4.703113in}{1.913300in}%
\pgfsys@useobject{currentmarker}{}%
\end{pgfscope}%
\end{pgfscope}%
\begin{pgfscope}%
\pgfsetbuttcap%
\pgfsetroundjoin%
\definecolor{currentfill}{rgb}{0.000000,0.000000,0.000000}%
\pgfsetfillcolor{currentfill}%
\pgfsetlinewidth{0.803000pt}%
\definecolor{currentstroke}{rgb}{0.000000,0.000000,0.000000}%
\pgfsetstrokecolor{currentstroke}%
\pgfsetdash{}{0pt}%
\pgfsys@defobject{currentmarker}{\pgfqpoint{0.000000in}{-0.048611in}}{\pgfqpoint{0.000000in}{0.000000in}}{%
\pgfpathmoveto{\pgfqpoint{0.000000in}{0.000000in}}%
\pgfpathlineto{\pgfqpoint{0.000000in}{-0.048611in}}%
\pgfusepath{stroke,fill}%
}%
\begin{pgfscope}%
\pgfsys@transformshift{5.316057in}{1.913300in}%
\pgfsys@useobject{currentmarker}{}%
\end{pgfscope}%
\end{pgfscope}%
\begin{pgfscope}%
\pgfsetbuttcap%
\pgfsetroundjoin%
\definecolor{currentfill}{rgb}{0.000000,0.000000,0.000000}%
\pgfsetfillcolor{currentfill}%
\pgfsetlinewidth{0.803000pt}%
\definecolor{currentstroke}{rgb}{0.000000,0.000000,0.000000}%
\pgfsetstrokecolor{currentstroke}%
\pgfsetdash{}{0pt}%
\pgfsys@defobject{currentmarker}{\pgfqpoint{-0.048611in}{0.000000in}}{\pgfqpoint{0.000000in}{0.000000in}}{%
\pgfpathmoveto{\pgfqpoint{0.000000in}{0.000000in}}%
\pgfpathlineto{\pgfqpoint{-0.048611in}{0.000000in}}%
\pgfusepath{stroke,fill}%
}%
\begin{pgfscope}%
\pgfsys@transformshift{0.800000in}{1.970732in}%
\pgfsys@useobject{currentmarker}{}%
\end{pgfscope}%
\end{pgfscope}%
\begin{pgfscope}%
\pgftext[x=0.525308in,y=1.922537in,left,base]{\sffamily\fontsize{10.000000}{12.000000}\selectfont \(\displaystyle 0.0\)}%
\end{pgfscope}%
\begin{pgfscope}%
\pgfsetbuttcap%
\pgfsetroundjoin%
\definecolor{currentfill}{rgb}{0.000000,0.000000,0.000000}%
\pgfsetfillcolor{currentfill}%
\pgfsetlinewidth{0.803000pt}%
\definecolor{currentstroke}{rgb}{0.000000,0.000000,0.000000}%
\pgfsetstrokecolor{currentstroke}%
\pgfsetdash{}{0pt}%
\pgfsys@defobject{currentmarker}{\pgfqpoint{-0.048611in}{0.000000in}}{\pgfqpoint{0.000000in}{0.000000in}}{%
\pgfpathmoveto{\pgfqpoint{0.000000in}{0.000000in}}%
\pgfpathlineto{\pgfqpoint{-0.048611in}{0.000000in}}%
\pgfusepath{stroke,fill}%
}%
\begin{pgfscope}%
\pgfsys@transformshift{0.800000in}{2.545050in}%
\pgfsys@useobject{currentmarker}{}%
\end{pgfscope}%
\end{pgfscope}%
\begin{pgfscope}%
\pgftext[x=0.525308in,y=2.496856in,left,base]{\sffamily\fontsize{10.000000}{12.000000}\selectfont \(\displaystyle 0.2\)}%
\end{pgfscope}%
\begin{pgfscope}%
\pgfsetbuttcap%
\pgfsetroundjoin%
\definecolor{currentfill}{rgb}{0.000000,0.000000,0.000000}%
\pgfsetfillcolor{currentfill}%
\pgfsetlinewidth{0.803000pt}%
\definecolor{currentstroke}{rgb}{0.000000,0.000000,0.000000}%
\pgfsetstrokecolor{currentstroke}%
\pgfsetdash{}{0pt}%
\pgfsys@defobject{currentmarker}{\pgfqpoint{-0.048611in}{0.000000in}}{\pgfqpoint{0.000000in}{0.000000in}}{%
\pgfpathmoveto{\pgfqpoint{0.000000in}{0.000000in}}%
\pgfpathlineto{\pgfqpoint{-0.048611in}{0.000000in}}%
\pgfusepath{stroke,fill}%
}%
\begin{pgfscope}%
\pgfsys@transformshift{0.800000in}{3.119368in}%
\pgfsys@useobject{currentmarker}{}%
\end{pgfscope}%
\end{pgfscope}%
\begin{pgfscope}%
\pgftext[x=0.525308in,y=3.071174in,left,base]{\sffamily\fontsize{10.000000}{12.000000}\selectfont \(\displaystyle 0.4\)}%
\end{pgfscope}%
\begin{pgfscope}%
\pgfpathrectangle{\pgfqpoint{0.800000in}{1.913300in}}{\pgfqpoint{4.960000in}{1.263500in}} %
\pgfusepath{clip}%
\pgfsetrectcap%
\pgfsetroundjoin%
\pgfsetlinewidth{1.505625pt}%
\definecolor{currentstroke}{rgb}{0.121569,0.466667,0.705882}%
\pgfsetstrokecolor{currentstroke}%
\pgfsetdash{}{0pt}%
\pgfpathmoveto{\pgfqpoint{1.025455in}{3.119368in}}%
\pgfpathlineto{\pgfqpoint{1.522579in}{2.998821in}}%
\pgfpathlineto{\pgfqpoint{2.050565in}{2.868479in}}%
\pgfpathlineto{\pgfqpoint{2.619612in}{2.725677in}}%
\pgfpathlineto{\pgfqpoint{3.245948in}{2.566158in}}%
\pgfpathlineto{\pgfqpoint{3.963160in}{2.381138in}}%
\pgfpathlineto{\pgfqpoint{4.875949in}{2.143228in}}%
\pgfpathlineto{\pgfqpoint{5.534545in}{1.970732in}}%
\pgfpathlineto{\pgfqpoint{5.534545in}{1.970732in}}%
\pgfusepath{stroke}%
\end{pgfscope}%
\begin{pgfscope}%
\pgfpathrectangle{\pgfqpoint{0.800000in}{1.913300in}}{\pgfqpoint{4.960000in}{1.263500in}} %
\pgfusepath{clip}%
\pgfsetrectcap%
\pgfsetroundjoin%
\pgfsetlinewidth{1.505625pt}%
\definecolor{currentstroke}{rgb}{1.000000,0.498039,0.054902}%
\pgfsetstrokecolor{currentstroke}%
\pgfsetdash{}{0pt}%
\pgfpathmoveto{\pgfqpoint{1.025455in}{3.119368in}}%
\pgfpathlineto{\pgfqpoint{1.035177in}{3.117293in}}%
\pgfpathlineto{\pgfqpoint{1.046602in}{3.114668in}}%
\pgfpathlineto{\pgfqpoint{1.069785in}{3.108562in}}%
\pgfpathlineto{\pgfqpoint{1.082675in}{3.105466in}}%
\pgfpathlineto{\pgfqpoint{1.105545in}{3.100030in}}%
\pgfpathlineto{\pgfqpoint{1.121632in}{3.095998in}}%
\pgfpathlineto{\pgfqpoint{1.135046in}{3.092830in}}%
\pgfpathlineto{\pgfqpoint{1.159268in}{3.087000in}}%
\pgfpathlineto{\pgfqpoint{1.175513in}{3.083092in}}%
\pgfpathlineto{\pgfqpoint{1.187997in}{3.080044in}}%
\pgfpathlineto{\pgfqpoint{1.212294in}{3.074169in}}%
\pgfpathlineto{\pgfqpoint{1.228337in}{3.070325in}}%
\pgfpathlineto{\pgfqpoint{1.240034in}{3.067444in}}%
\pgfpathlineto{\pgfqpoint{1.317846in}{3.048592in}}%
\pgfpathlineto{\pgfqpoint{1.333729in}{3.044759in}}%
\pgfpathlineto{\pgfqpoint{1.345983in}{3.041737in}}%
\pgfpathlineto{\pgfqpoint{1.396217in}{3.029608in}}%
\pgfpathlineto{\pgfqpoint{1.406985in}{3.027449in}}%
\pgfpathlineto{\pgfqpoint{1.418843in}{3.024394in}}%
\pgfpathlineto{\pgfqpoint{1.437889in}{3.019296in}}%
\pgfpathlineto{\pgfqpoint{1.451500in}{3.016042in}}%
\pgfpathlineto{\pgfqpoint{1.525404in}{2.998316in}}%
\pgfpathlineto{\pgfqpoint{1.543589in}{2.993502in}}%
\pgfpathlineto{\pgfqpoint{1.556451in}{2.990407in}}%
\pgfpathlineto{\pgfqpoint{1.670421in}{2.962956in}}%
\pgfpathlineto{\pgfqpoint{1.682720in}{2.959855in}}%
\pgfpathlineto{\pgfqpoint{1.701476in}{2.954765in}}%
\pgfpathlineto{\pgfqpoint{1.715258in}{2.951445in}}%
\pgfpathlineto{\pgfqpoint{1.902288in}{2.905146in}}%
\pgfpathlineto{\pgfqpoint{1.917673in}{2.901893in}}%
\pgfpathlineto{\pgfqpoint{1.948760in}{2.894129in}}%
\pgfpathlineto{\pgfqpoint{1.965894in}{2.889622in}}%
\pgfpathlineto{\pgfqpoint{1.978319in}{2.886584in}}%
\pgfpathlineto{\pgfqpoint{1.991767in}{2.883914in}}%
\pgfpathlineto{\pgfqpoint{2.058531in}{2.866618in}}%
\pgfpathlineto{\pgfqpoint{2.073776in}{2.863103in}}%
\pgfpathlineto{\pgfqpoint{2.085305in}{2.860031in}}%
\pgfpathlineto{\pgfqpoint{2.215380in}{2.827786in}}%
\pgfpathlineto{\pgfqpoint{2.230730in}{2.824004in}}%
\pgfpathlineto{\pgfqpoint{2.242140in}{2.821115in}}%
\pgfpathlineto{\pgfqpoint{2.253392in}{2.818849in}}%
\pgfpathlineto{\pgfqpoint{2.264720in}{2.815791in}}%
\pgfpathlineto{\pgfqpoint{2.282007in}{2.811115in}}%
\pgfpathlineto{\pgfqpoint{2.294822in}{2.808004in}}%
\pgfpathlineto{\pgfqpoint{2.306805in}{2.805574in}}%
\pgfpathlineto{\pgfqpoint{2.318704in}{2.802248in}}%
\pgfpathlineto{\pgfqpoint{2.335342in}{2.797874in}}%
\pgfpathlineto{\pgfqpoint{2.347655in}{2.794834in}}%
\pgfpathlineto{\pgfqpoint{2.358888in}{2.792545in}}%
\pgfpathlineto{\pgfqpoint{2.372447in}{2.788770in}}%
\pgfpathlineto{\pgfqpoint{2.387326in}{2.784845in}}%
\pgfpathlineto{\pgfqpoint{2.401179in}{2.781442in}}%
\pgfpathlineto{\pgfqpoint{2.440273in}{2.771644in}}%
\pgfpathlineto{\pgfqpoint{2.453320in}{2.768458in}}%
\pgfpathlineto{\pgfqpoint{2.464906in}{2.766091in}}%
\pgfpathlineto{\pgfqpoint{2.478695in}{2.762188in}}%
\pgfpathlineto{\pgfqpoint{2.493090in}{2.758453in}}%
\pgfpathlineto{\pgfqpoint{2.506489in}{2.755148in}}%
\pgfpathlineto{\pgfqpoint{2.518747in}{2.752680in}}%
\pgfpathlineto{\pgfqpoint{2.533400in}{2.748361in}}%
\pgfpathlineto{\pgfqpoint{2.548400in}{2.744840in}}%
\pgfpathlineto{\pgfqpoint{2.559726in}{2.741811in}}%
\pgfpathlineto{\pgfqpoint{2.598194in}{2.732129in}}%
\pgfpathlineto{\pgfqpoint{2.611941in}{2.728771in}}%
\pgfpathlineto{\pgfqpoint{2.623225in}{2.726463in}}%
\pgfpathlineto{\pgfqpoint{2.636607in}{2.722720in}}%
\pgfpathlineto{\pgfqpoint{2.651180in}{2.718878in}}%
\pgfpathlineto{\pgfqpoint{2.665071in}{2.715443in}}%
\pgfpathlineto{\pgfqpoint{2.677980in}{2.712832in}}%
\pgfpathlineto{\pgfqpoint{2.692913in}{2.708340in}}%
\pgfpathlineto{\pgfqpoint{2.706774in}{2.705185in}}%
\pgfpathlineto{\pgfqpoint{2.717795in}{2.702230in}}%
\pgfpathlineto{\pgfqpoint{2.730729in}{2.699601in}}%
\pgfpathlineto{\pgfqpoint{2.745392in}{2.695203in}}%
\pgfpathlineto{\pgfqpoint{2.760092in}{2.691858in}}%
\pgfpathlineto{\pgfqpoint{2.771518in}{2.688712in}}%
\pgfpathlineto{\pgfqpoint{2.815129in}{2.678184in}}%
\pgfpathlineto{\pgfqpoint{2.849659in}{2.669131in}}%
\pgfpathlineto{\pgfqpoint{2.864496in}{2.665561in}}%
\pgfpathlineto{\pgfqpoint{2.876055in}{2.662516in}}%
\pgfpathlineto{\pgfqpoint{2.887951in}{2.660063in}}%
\pgfpathlineto{\pgfqpoint{2.901548in}{2.656170in}}%
\pgfpathlineto{\pgfqpoint{2.915569in}{2.652586in}}%
\pgfpathlineto{\pgfqpoint{2.927857in}{2.649569in}}%
\pgfpathlineto{\pgfqpoint{2.938949in}{2.647119in}}%
\pgfpathlineto{\pgfqpoint{2.950823in}{2.644075in}}%
\pgfpathlineto{\pgfqpoint{2.967946in}{2.639389in}}%
\pgfpathlineto{\pgfqpoint{2.980739in}{2.636268in}}%
\pgfpathlineto{\pgfqpoint{2.991257in}{2.633920in}}%
\pgfpathlineto{\pgfqpoint{3.002642in}{2.631118in}}%
\pgfpathlineto{\pgfqpoint{3.021262in}{2.626028in}}%
\pgfpathlineto{\pgfqpoint{3.033943in}{2.622857in}}%
\pgfpathlineto{\pgfqpoint{3.045051in}{2.620458in}}%
\pgfpathlineto{\pgfqpoint{3.057354in}{2.617208in}}%
\pgfpathlineto{\pgfqpoint{3.073774in}{2.612797in}}%
\pgfpathlineto{\pgfqpoint{3.086082in}{2.609788in}}%
\pgfpathlineto{\pgfqpoint{3.096316in}{2.607427in}}%
\pgfpathlineto{\pgfqpoint{3.109685in}{2.604077in}}%
\pgfpathlineto{\pgfqpoint{3.126483in}{2.599527in}}%
\pgfpathlineto{\pgfqpoint{3.139154in}{2.596407in}}%
\pgfpathlineto{\pgfqpoint{3.150560in}{2.593880in}}%
\pgfpathlineto{\pgfqpoint{3.162714in}{2.590708in}}%
\pgfpathlineto{\pgfqpoint{3.179353in}{2.586225in}}%
\pgfpathlineto{\pgfqpoint{3.191794in}{2.583165in}}%
\pgfpathlineto{\pgfqpoint{3.202108in}{2.580789in}}%
\pgfpathlineto{\pgfqpoint{3.215634in}{2.577370in}}%
\pgfpathlineto{\pgfqpoint{3.232170in}{2.572927in}}%
\pgfpathlineto{\pgfqpoint{3.244823in}{2.569789in}}%
\pgfpathlineto{\pgfqpoint{3.255440in}{2.567390in}}%
\pgfpathlineto{\pgfqpoint{3.267051in}{2.564524in}}%
\pgfpathlineto{\pgfqpoint{3.285756in}{2.559498in}}%
\pgfpathlineto{\pgfqpoint{3.297793in}{2.556430in}}%
\pgfpathlineto{\pgfqpoint{3.309140in}{2.553922in}}%
\pgfpathlineto{\pgfqpoint{3.321382in}{2.550715in}}%
\pgfpathlineto{\pgfqpoint{3.337766in}{2.546328in}}%
\pgfpathlineto{\pgfqpoint{3.350967in}{2.543001in}}%
\pgfpathlineto{\pgfqpoint{3.362181in}{2.540564in}}%
\pgfpathlineto{\pgfqpoint{3.374593in}{2.537269in}}%
\pgfpathlineto{\pgfqpoint{3.389529in}{2.533200in}}%
\pgfpathlineto{\pgfqpoint{3.403766in}{2.529692in}}%
\pgfpathlineto{\pgfqpoint{3.414919in}{2.527257in}}%
\pgfpathlineto{\pgfqpoint{3.427268in}{2.524001in}}%
\pgfpathlineto{\pgfqpoint{3.443593in}{2.519679in}}%
\pgfpathlineto{\pgfqpoint{3.456167in}{2.516510in}}%
\pgfpathlineto{\pgfqpoint{3.467528in}{2.513970in}}%
\pgfpathlineto{\pgfqpoint{3.479721in}{2.510804in}}%
\pgfpathlineto{\pgfqpoint{3.496190in}{2.506404in}}%
\pgfpathlineto{\pgfqpoint{3.509085in}{2.503158in}}%
\pgfpathlineto{\pgfqpoint{3.520887in}{2.500550in}}%
\pgfpathlineto{\pgfqpoint{3.533465in}{2.497174in}}%
\pgfpathlineto{\pgfqpoint{3.548219in}{2.493224in}}%
\pgfpathlineto{\pgfqpoint{3.561825in}{2.489863in}}%
\pgfpathlineto{\pgfqpoint{3.573427in}{2.487272in}}%
\pgfpathlineto{\pgfqpoint{3.585825in}{2.484007in}}%
\pgfpathlineto{\pgfqpoint{3.600688in}{2.479970in}}%
\pgfpathlineto{\pgfqpoint{3.614765in}{2.476501in}}%
\pgfpathlineto{\pgfqpoint{3.626333in}{2.473930in}}%
\pgfpathlineto{\pgfqpoint{3.638796in}{2.470634in}}%
\pgfpathlineto{\pgfqpoint{3.653598in}{2.466639in}}%
\pgfpathlineto{\pgfqpoint{3.667575in}{2.463182in}}%
\pgfpathlineto{\pgfqpoint{3.678195in}{2.460767in}}%
\pgfpathlineto{\pgfqpoint{3.690209in}{2.457779in}}%
\pgfpathlineto{\pgfqpoint{3.707518in}{2.453139in}}%
\pgfpathlineto{\pgfqpoint{3.720068in}{2.449967in}}%
\pgfpathlineto{\pgfqpoint{3.731986in}{2.447272in}}%
\pgfpathlineto{\pgfqpoint{3.744482in}{2.443974in}}%
\pgfpathlineto{\pgfqpoint{3.759145in}{2.440026in}}%
\pgfpathlineto{\pgfqpoint{3.772915in}{2.436637in}}%
\pgfpathlineto{\pgfqpoint{3.784253in}{2.434042in}}%
\pgfpathlineto{\pgfqpoint{3.796460in}{2.430933in}}%
\pgfpathlineto{\pgfqpoint{3.812974in}{2.426542in}}%
\pgfpathlineto{\pgfqpoint{3.824842in}{2.423614in}}%
\pgfpathlineto{\pgfqpoint{3.835634in}{2.420958in}}%
\pgfpathlineto{\pgfqpoint{3.849375in}{2.417582in}}%
\pgfpathlineto{\pgfqpoint{3.865791in}{2.413231in}}%
\pgfpathlineto{\pgfqpoint{3.878070in}{2.410159in}}%
\pgfpathlineto{\pgfqpoint{3.889498in}{2.407459in}}%
\pgfpathlineto{\pgfqpoint{3.901686in}{2.404433in}}%
\pgfpathlineto{\pgfqpoint{3.918680in}{2.399910in}}%
\pgfpathlineto{\pgfqpoint{3.931220in}{2.396732in}}%
\pgfpathlineto{\pgfqpoint{3.942335in}{2.394135in}}%
\pgfpathlineto{\pgfqpoint{3.954576in}{2.391094in}}%
\pgfpathlineto{\pgfqpoint{3.971490in}{2.386604in}}%
\pgfpathlineto{\pgfqpoint{3.983604in}{2.383563in}}%
\pgfpathlineto{\pgfqpoint{3.994882in}{2.380861in}}%
\pgfpathlineto{\pgfqpoint{4.007079in}{2.377885in}}%
\pgfpathlineto{\pgfqpoint{4.023985in}{2.373354in}}%
\pgfpathlineto{\pgfqpoint{4.036094in}{2.370363in}}%
\pgfpathlineto{\pgfqpoint{4.048357in}{2.367435in}}%
\pgfpathlineto{\pgfqpoint{4.060800in}{2.364278in}}%
\pgfpathlineto{\pgfqpoint{4.076932in}{2.360032in}}%
\pgfpathlineto{\pgfqpoint{4.088596in}{2.357162in}}%
\pgfpathlineto{\pgfqpoint{4.100201in}{2.354287in}}%
\pgfpathlineto{\pgfqpoint{4.112390in}{2.351377in}}%
\pgfpathlineto{\pgfqpoint{4.128627in}{2.346925in}}%
\pgfpathlineto{\pgfqpoint{4.141859in}{2.343710in}}%
\pgfpathlineto{\pgfqpoint{4.153872in}{2.340836in}}%
\pgfpathlineto{\pgfqpoint{4.166376in}{2.337692in}}%
\pgfpathlineto{\pgfqpoint{4.182423in}{2.333468in}}%
\pgfpathlineto{\pgfqpoint{4.195147in}{2.330254in}}%
\pgfpathlineto{\pgfqpoint{4.206799in}{2.327515in}}%
\pgfpathlineto{\pgfqpoint{4.219377in}{2.324335in}}%
\pgfpathlineto{\pgfqpoint{4.233476in}{2.320481in}}%
\pgfpathlineto{\pgfqpoint{4.248173in}{2.316892in}}%
\pgfpathlineto{\pgfqpoint{4.260221in}{2.314114in}}%
\pgfpathlineto{\pgfqpoint{4.273007in}{2.310773in}}%
\pgfpathlineto{\pgfqpoint{4.287666in}{2.306964in}}%
\pgfpathlineto{\pgfqpoint{4.300032in}{2.303924in}}%
\pgfpathlineto{\pgfqpoint{4.312420in}{2.300927in}}%
\pgfpathlineto{\pgfqpoint{4.325098in}{2.297730in}}%
\pgfpathlineto{\pgfqpoint{4.339082in}{2.293927in}}%
\pgfpathlineto{\pgfqpoint{4.353476in}{2.290433in}}%
\pgfpathlineto{\pgfqpoint{4.364489in}{2.287766in}}%
\pgfpathlineto{\pgfqpoint{4.377046in}{2.284739in}}%
\pgfpathlineto{\pgfqpoint{4.393727in}{2.280347in}}%
\pgfpathlineto{\pgfqpoint{4.405177in}{2.277531in}}%
\pgfpathlineto{\pgfqpoint{4.417552in}{2.274445in}}%
\pgfpathlineto{\pgfqpoint{4.430249in}{2.271339in}}%
\pgfpathlineto{\pgfqpoint{4.446139in}{2.267152in}}%
\pgfpathlineto{\pgfqpoint{4.458622in}{2.264052in}}%
\pgfpathlineto{\pgfqpoint{4.471081in}{2.261053in}}%
\pgfpathlineto{\pgfqpoint{4.483951in}{2.257781in}}%
\pgfpathlineto{\pgfqpoint{4.499043in}{2.253882in}}%
\pgfpathlineto{\pgfqpoint{4.511412in}{2.250800in}}%
\pgfpathlineto{\pgfqpoint{4.523490in}{2.247856in}}%
\pgfpathlineto{\pgfqpoint{4.536372in}{2.244655in}}%
\pgfpathlineto{\pgfqpoint{4.550240in}{2.240908in}}%
\pgfpathlineto{\pgfqpoint{4.564187in}{2.237562in}}%
\pgfpathlineto{\pgfqpoint{4.576289in}{2.234602in}}%
\pgfpathlineto{\pgfqpoint{4.589232in}{2.231392in}}%
\pgfpathlineto{\pgfqpoint{4.603086in}{2.227663in}}%
\pgfpathlineto{\pgfqpoint{4.617607in}{2.224116in}}%
\pgfpathlineto{\pgfqpoint{4.629764in}{2.221243in}}%
\pgfpathlineto{\pgfqpoint{4.642822in}{2.217893in}}%
\pgfpathlineto{\pgfqpoint{4.657522in}{2.214159in}}%
\pgfpathlineto{\pgfqpoint{4.669080in}{2.211328in}}%
\pgfpathlineto{\pgfqpoint{4.681585in}{2.208170in}}%
\pgfpathlineto{\pgfqpoint{4.693806in}{2.205268in}}%
\pgfpathlineto{\pgfqpoint{4.710530in}{2.200905in}}%
\pgfpathlineto{\pgfqpoint{4.722345in}{2.197952in}}%
\pgfpathlineto{\pgfqpoint{4.735312in}{2.194787in}}%
\pgfpathlineto{\pgfqpoint{4.748484in}{2.191430in}}%
\pgfpathlineto{\pgfqpoint{4.763009in}{2.187749in}}%
\pgfpathlineto{\pgfqpoint{4.775111in}{2.184752in}}%
\pgfpathlineto{\pgfqpoint{4.788037in}{2.181582in}}%
\pgfpathlineto{\pgfqpoint{4.801266in}{2.178230in}}%
\pgfpathlineto{\pgfqpoint{4.815796in}{2.174552in}}%
\pgfpathlineto{\pgfqpoint{4.827767in}{2.171599in}}%
\pgfpathlineto{\pgfqpoint{4.840865in}{2.168370in}}%
\pgfpathlineto{\pgfqpoint{4.854153in}{2.165005in}}%
\pgfpathlineto{\pgfqpoint{4.868741in}{2.161339in}}%
\pgfpathlineto{\pgfqpoint{4.881464in}{2.158110in}}%
\pgfpathlineto{\pgfqpoint{4.893905in}{2.155134in}}%
\pgfpathlineto{\pgfqpoint{4.907255in}{2.151720in}}%
\pgfpathlineto{\pgfqpoint{4.921515in}{2.148166in}}%
\pgfpathlineto{\pgfqpoint{4.933988in}{2.145022in}}%
\pgfpathlineto{\pgfqpoint{4.946762in}{2.141939in}}%
\pgfpathlineto{\pgfqpoint{4.958263in}{2.139139in}}%
\pgfpathlineto{\pgfqpoint{4.974310in}{2.135000in}}%
\pgfpathlineto{\pgfqpoint{4.985616in}{2.132244in}}%
\pgfpathlineto{\pgfqpoint{4.999344in}{2.128802in}}%
\pgfpathlineto{\pgfqpoint{5.012786in}{2.125408in}}%
\pgfpathlineto{\pgfqpoint{5.026985in}{2.121867in}}%
\pgfpathlineto{\pgfqpoint{5.039464in}{2.118741in}}%
\pgfpathlineto{\pgfqpoint{5.051996in}{2.115666in}}%
\pgfpathlineto{\pgfqpoint{5.065473in}{2.112297in}}%
\pgfpathlineto{\pgfqpoint{5.079791in}{2.108724in}}%
\pgfpathlineto{\pgfqpoint{5.091217in}{2.105945in}}%
\pgfpathlineto{\pgfqpoint{5.105007in}{2.102483in}}%
\pgfpathlineto{\pgfqpoint{5.118523in}{2.099078in}}%
\pgfpathlineto{\pgfqpoint{5.132730in}{2.095573in}}%
\pgfpathlineto{\pgfqpoint{5.145323in}{2.092383in}}%
\pgfpathlineto{\pgfqpoint{5.158028in}{2.089311in}}%
\pgfpathlineto{\pgfqpoint{5.169641in}{2.086508in}}%
\pgfpathlineto{\pgfqpoint{5.185227in}{2.082508in}}%
\pgfpathlineto{\pgfqpoint{5.196493in}{2.079805in}}%
\pgfpathlineto{\pgfqpoint{5.210684in}{2.076213in}}%
\pgfpathlineto{\pgfqpoint{5.224236in}{2.072808in}}%
\pgfpathlineto{\pgfqpoint{5.237984in}{2.069413in}}%
\pgfpathlineto{\pgfqpoint{5.249423in}{2.066660in}}%
\pgfpathlineto{\pgfqpoint{5.263393in}{2.063121in}}%
\pgfpathlineto{\pgfqpoint{5.276360in}{2.059937in}}%
\pgfpathlineto{\pgfqpoint{5.290864in}{2.056313in}}%
\pgfpathlineto{\pgfqpoint{5.303147in}{2.053268in}}%
\pgfpathlineto{\pgfqpoint{5.316388in}{2.049997in}}%
\pgfpathlineto{\pgfqpoint{5.329849in}{2.046629in}}%
\pgfpathlineto{\pgfqpoint{5.343524in}{2.043264in}}%
\pgfpathlineto{\pgfqpoint{5.355429in}{2.040367in}}%
\pgfpathlineto{\pgfqpoint{5.369475in}{2.036872in}}%
\pgfpathlineto{\pgfqpoint{5.381173in}{2.034050in}}%
\pgfpathlineto{\pgfqpoint{5.396496in}{2.030179in}}%
\pgfpathlineto{\pgfqpoint{5.409451in}{2.026907in}}%
\pgfpathlineto{\pgfqpoint{5.422290in}{2.023808in}}%
\pgfpathlineto{\pgfqpoint{5.433969in}{2.021002in}}%
\pgfpathlineto{\pgfqpoint{5.449136in}{2.017165in}}%
\pgfpathlineto{\pgfqpoint{5.461932in}{2.013972in}}%
\pgfpathlineto{\pgfqpoint{5.475674in}{2.010666in}}%
\pgfpathlineto{\pgfqpoint{5.487414in}{2.007759in}}%
\pgfpathlineto{\pgfqpoint{5.501772in}{2.004168in}}%
\pgfpathlineto{\pgfqpoint{5.514117in}{2.001151in}}%
\pgfpathlineto{\pgfqpoint{5.527763in}{1.997755in}}%
\pgfpathlineto{\pgfqpoint{5.534545in}{1.996422in}}%
\pgfpathlineto{\pgfqpoint{5.534545in}{1.996422in}}%
\pgfusepath{stroke}%
\end{pgfscope}%
\begin{pgfscope}%
\pgfsetrectcap%
\pgfsetmiterjoin%
\pgfsetlinewidth{0.803000pt}%
\definecolor{currentstroke}{rgb}{0.000000,0.000000,0.000000}%
\pgfsetstrokecolor{currentstroke}%
\pgfsetdash{}{0pt}%
\pgfpathmoveto{\pgfqpoint{0.800000in}{1.913300in}}%
\pgfpathlineto{\pgfqpoint{0.800000in}{3.176800in}}%
\pgfusepath{stroke}%
\end{pgfscope}%
\begin{pgfscope}%
\pgfsetrectcap%
\pgfsetmiterjoin%
\pgfsetlinewidth{0.803000pt}%
\definecolor{currentstroke}{rgb}{0.000000,0.000000,0.000000}%
\pgfsetstrokecolor{currentstroke}%
\pgfsetdash{}{0pt}%
\pgfpathmoveto{\pgfqpoint{5.760000in}{1.913300in}}%
\pgfpathlineto{\pgfqpoint{5.760000in}{3.176800in}}%
\pgfusepath{stroke}%
\end{pgfscope}%
\begin{pgfscope}%
\pgfsetrectcap%
\pgfsetmiterjoin%
\pgfsetlinewidth{0.803000pt}%
\definecolor{currentstroke}{rgb}{0.000000,0.000000,0.000000}%
\pgfsetstrokecolor{currentstroke}%
\pgfsetdash{}{0pt}%
\pgfpathmoveto{\pgfqpoint{0.800000in}{1.913300in}}%
\pgfpathlineto{\pgfqpoint{5.760000in}{1.913300in}}%
\pgfusepath{stroke}%
\end{pgfscope}%
\begin{pgfscope}%
\pgfsetrectcap%
\pgfsetmiterjoin%
\pgfsetlinewidth{0.803000pt}%
\definecolor{currentstroke}{rgb}{0.000000,0.000000,0.000000}%
\pgfsetstrokecolor{currentstroke}%
\pgfsetdash{}{0pt}%
\pgfpathmoveto{\pgfqpoint{0.800000in}{3.176800in}}%
\pgfpathlineto{\pgfqpoint{5.760000in}{3.176800in}}%
\pgfusepath{stroke}%
\end{pgfscope}%
\begin{pgfscope}%
\pgfsetbuttcap%
\pgfsetmiterjoin%
\definecolor{currentfill}{rgb}{1.000000,1.000000,1.000000}%
\pgfsetfillcolor{currentfill}%
\pgfsetlinewidth{0.000000pt}%
\definecolor{currentstroke}{rgb}{0.000000,0.000000,0.000000}%
\pgfsetstrokecolor{currentstroke}%
\pgfsetstrokeopacity{0.000000}%
\pgfsetdash{}{0pt}%
\pgfpathmoveto{\pgfqpoint{0.800000in}{0.397100in}}%
\pgfpathlineto{\pgfqpoint{5.760000in}{0.397100in}}%
\pgfpathlineto{\pgfqpoint{5.760000in}{1.660600in}}%
\pgfpathlineto{\pgfqpoint{0.800000in}{1.660600in}}%
\pgfpathclose%
\pgfusepath{fill}%
\end{pgfscope}%
\begin{pgfscope}%
\pgfsetbuttcap%
\pgfsetroundjoin%
\definecolor{currentfill}{rgb}{0.000000,0.000000,0.000000}%
\pgfsetfillcolor{currentfill}%
\pgfsetlinewidth{0.803000pt}%
\definecolor{currentstroke}{rgb}{0.000000,0.000000,0.000000}%
\pgfsetstrokecolor{currentstroke}%
\pgfsetdash{}{0pt}%
\pgfsys@defobject{currentmarker}{\pgfqpoint{0.000000in}{-0.048611in}}{\pgfqpoint{0.000000in}{0.000000in}}{%
\pgfpathmoveto{\pgfqpoint{0.000000in}{0.000000in}}%
\pgfpathlineto{\pgfqpoint{0.000000in}{-0.048611in}}%
\pgfusepath{stroke,fill}%
}%
\begin{pgfscope}%
\pgfsys@transformshift{1.025455in}{0.397100in}%
\pgfsys@useobject{currentmarker}{}%
\end{pgfscope}%
\end{pgfscope}%
\begin{pgfscope}%
\pgftext[x=1.025455in,y=0.299878in,,top]{\sffamily\fontsize{10.000000}{12.000000}\selectfont \(\displaystyle 0\)}%
\end{pgfscope}%
\begin{pgfscope}%
\pgfsetbuttcap%
\pgfsetroundjoin%
\definecolor{currentfill}{rgb}{0.000000,0.000000,0.000000}%
\pgfsetfillcolor{currentfill}%
\pgfsetlinewidth{0.803000pt}%
\definecolor{currentstroke}{rgb}{0.000000,0.000000,0.000000}%
\pgfsetstrokecolor{currentstroke}%
\pgfsetdash{}{0pt}%
\pgfsys@defobject{currentmarker}{\pgfqpoint{0.000000in}{-0.048611in}}{\pgfqpoint{0.000000in}{0.000000in}}{%
\pgfpathmoveto{\pgfqpoint{0.000000in}{0.000000in}}%
\pgfpathlineto{\pgfqpoint{0.000000in}{-0.048611in}}%
\pgfusepath{stroke,fill}%
}%
\begin{pgfscope}%
\pgfsys@transformshift{1.638398in}{0.397100in}%
\pgfsys@useobject{currentmarker}{}%
\end{pgfscope}%
\end{pgfscope}%
\begin{pgfscope}%
\pgftext[x=1.638398in,y=0.299878in,,top]{\sffamily\fontsize{10.000000}{12.000000}\selectfont \(\displaystyle 1000000\)}%
\end{pgfscope}%
\begin{pgfscope}%
\pgfsetbuttcap%
\pgfsetroundjoin%
\definecolor{currentfill}{rgb}{0.000000,0.000000,0.000000}%
\pgfsetfillcolor{currentfill}%
\pgfsetlinewidth{0.803000pt}%
\definecolor{currentstroke}{rgb}{0.000000,0.000000,0.000000}%
\pgfsetstrokecolor{currentstroke}%
\pgfsetdash{}{0pt}%
\pgfsys@defobject{currentmarker}{\pgfqpoint{0.000000in}{-0.048611in}}{\pgfqpoint{0.000000in}{0.000000in}}{%
\pgfpathmoveto{\pgfqpoint{0.000000in}{0.000000in}}%
\pgfpathlineto{\pgfqpoint{0.000000in}{-0.048611in}}%
\pgfusepath{stroke,fill}%
}%
\begin{pgfscope}%
\pgfsys@transformshift{2.251341in}{0.397100in}%
\pgfsys@useobject{currentmarker}{}%
\end{pgfscope}%
\end{pgfscope}%
\begin{pgfscope}%
\pgftext[x=2.251341in,y=0.299878in,,top]{\sffamily\fontsize{10.000000}{12.000000}\selectfont \(\displaystyle 2000000\)}%
\end{pgfscope}%
\begin{pgfscope}%
\pgfsetbuttcap%
\pgfsetroundjoin%
\definecolor{currentfill}{rgb}{0.000000,0.000000,0.000000}%
\pgfsetfillcolor{currentfill}%
\pgfsetlinewidth{0.803000pt}%
\definecolor{currentstroke}{rgb}{0.000000,0.000000,0.000000}%
\pgfsetstrokecolor{currentstroke}%
\pgfsetdash{}{0pt}%
\pgfsys@defobject{currentmarker}{\pgfqpoint{0.000000in}{-0.048611in}}{\pgfqpoint{0.000000in}{0.000000in}}{%
\pgfpathmoveto{\pgfqpoint{0.000000in}{0.000000in}}%
\pgfpathlineto{\pgfqpoint{0.000000in}{-0.048611in}}%
\pgfusepath{stroke,fill}%
}%
\begin{pgfscope}%
\pgfsys@transformshift{2.864284in}{0.397100in}%
\pgfsys@useobject{currentmarker}{}%
\end{pgfscope}%
\end{pgfscope}%
\begin{pgfscope}%
\pgftext[x=2.864284in,y=0.299878in,,top]{\sffamily\fontsize{10.000000}{12.000000}\selectfont \(\displaystyle 3000000\)}%
\end{pgfscope}%
\begin{pgfscope}%
\pgfsetbuttcap%
\pgfsetroundjoin%
\definecolor{currentfill}{rgb}{0.000000,0.000000,0.000000}%
\pgfsetfillcolor{currentfill}%
\pgfsetlinewidth{0.803000pt}%
\definecolor{currentstroke}{rgb}{0.000000,0.000000,0.000000}%
\pgfsetstrokecolor{currentstroke}%
\pgfsetdash{}{0pt}%
\pgfsys@defobject{currentmarker}{\pgfqpoint{0.000000in}{-0.048611in}}{\pgfqpoint{0.000000in}{0.000000in}}{%
\pgfpathmoveto{\pgfqpoint{0.000000in}{0.000000in}}%
\pgfpathlineto{\pgfqpoint{0.000000in}{-0.048611in}}%
\pgfusepath{stroke,fill}%
}%
\begin{pgfscope}%
\pgfsys@transformshift{3.477227in}{0.397100in}%
\pgfsys@useobject{currentmarker}{}%
\end{pgfscope}%
\end{pgfscope}%
\begin{pgfscope}%
\pgftext[x=3.477227in,y=0.299878in,,top]{\sffamily\fontsize{10.000000}{12.000000}\selectfont \(\displaystyle 4000000\)}%
\end{pgfscope}%
\begin{pgfscope}%
\pgfsetbuttcap%
\pgfsetroundjoin%
\definecolor{currentfill}{rgb}{0.000000,0.000000,0.000000}%
\pgfsetfillcolor{currentfill}%
\pgfsetlinewidth{0.803000pt}%
\definecolor{currentstroke}{rgb}{0.000000,0.000000,0.000000}%
\pgfsetstrokecolor{currentstroke}%
\pgfsetdash{}{0pt}%
\pgfsys@defobject{currentmarker}{\pgfqpoint{0.000000in}{-0.048611in}}{\pgfqpoint{0.000000in}{0.000000in}}{%
\pgfpathmoveto{\pgfqpoint{0.000000in}{0.000000in}}%
\pgfpathlineto{\pgfqpoint{0.000000in}{-0.048611in}}%
\pgfusepath{stroke,fill}%
}%
\begin{pgfscope}%
\pgfsys@transformshift{4.090170in}{0.397100in}%
\pgfsys@useobject{currentmarker}{}%
\end{pgfscope}%
\end{pgfscope}%
\begin{pgfscope}%
\pgftext[x=4.090170in,y=0.299878in,,top]{\sffamily\fontsize{10.000000}{12.000000}\selectfont \(\displaystyle 5000000\)}%
\end{pgfscope}%
\begin{pgfscope}%
\pgfsetbuttcap%
\pgfsetroundjoin%
\definecolor{currentfill}{rgb}{0.000000,0.000000,0.000000}%
\pgfsetfillcolor{currentfill}%
\pgfsetlinewidth{0.803000pt}%
\definecolor{currentstroke}{rgb}{0.000000,0.000000,0.000000}%
\pgfsetstrokecolor{currentstroke}%
\pgfsetdash{}{0pt}%
\pgfsys@defobject{currentmarker}{\pgfqpoint{0.000000in}{-0.048611in}}{\pgfqpoint{0.000000in}{0.000000in}}{%
\pgfpathmoveto{\pgfqpoint{0.000000in}{0.000000in}}%
\pgfpathlineto{\pgfqpoint{0.000000in}{-0.048611in}}%
\pgfusepath{stroke,fill}%
}%
\begin{pgfscope}%
\pgfsys@transformshift{4.703113in}{0.397100in}%
\pgfsys@useobject{currentmarker}{}%
\end{pgfscope}%
\end{pgfscope}%
\begin{pgfscope}%
\pgftext[x=4.703113in,y=0.299878in,,top]{\sffamily\fontsize{10.000000}{12.000000}\selectfont \(\displaystyle 6000000\)}%
\end{pgfscope}%
\begin{pgfscope}%
\pgfsetbuttcap%
\pgfsetroundjoin%
\definecolor{currentfill}{rgb}{0.000000,0.000000,0.000000}%
\pgfsetfillcolor{currentfill}%
\pgfsetlinewidth{0.803000pt}%
\definecolor{currentstroke}{rgb}{0.000000,0.000000,0.000000}%
\pgfsetstrokecolor{currentstroke}%
\pgfsetdash{}{0pt}%
\pgfsys@defobject{currentmarker}{\pgfqpoint{0.000000in}{-0.048611in}}{\pgfqpoint{0.000000in}{0.000000in}}{%
\pgfpathmoveto{\pgfqpoint{0.000000in}{0.000000in}}%
\pgfpathlineto{\pgfqpoint{0.000000in}{-0.048611in}}%
\pgfusepath{stroke,fill}%
}%
\begin{pgfscope}%
\pgfsys@transformshift{5.316057in}{0.397100in}%
\pgfsys@useobject{currentmarker}{}%
\end{pgfscope}%
\end{pgfscope}%
\begin{pgfscope}%
\pgftext[x=5.316057in,y=0.299878in,,top]{\sffamily\fontsize{10.000000}{12.000000}\selectfont \(\displaystyle 7000000\)}%
\end{pgfscope}%
\begin{pgfscope}%
\pgfsetbuttcap%
\pgfsetroundjoin%
\definecolor{currentfill}{rgb}{0.000000,0.000000,0.000000}%
\pgfsetfillcolor{currentfill}%
\pgfsetlinewidth{0.803000pt}%
\definecolor{currentstroke}{rgb}{0.000000,0.000000,0.000000}%
\pgfsetstrokecolor{currentstroke}%
\pgfsetdash{}{0pt}%
\pgfsys@defobject{currentmarker}{\pgfqpoint{-0.048611in}{0.000000in}}{\pgfqpoint{0.000000in}{0.000000in}}{%
\pgfpathmoveto{\pgfqpoint{0.000000in}{0.000000in}}%
\pgfpathlineto{\pgfqpoint{-0.048611in}{0.000000in}}%
\pgfusepath{stroke,fill}%
}%
\begin{pgfscope}%
\pgfsys@transformshift{0.800000in}{0.478049in}%
\pgfsys@useobject{currentmarker}{}%
\end{pgfscope}%
\end{pgfscope}%
\begin{pgfscope}%
\pgftext[x=0.386419in,y=0.429854in,left,base]{\sffamily\fontsize{10.000000}{12.000000}\selectfont \(\displaystyle 0.000\)}%
\end{pgfscope}%
\begin{pgfscope}%
\pgfsetbuttcap%
\pgfsetroundjoin%
\definecolor{currentfill}{rgb}{0.000000,0.000000,0.000000}%
\pgfsetfillcolor{currentfill}%
\pgfsetlinewidth{0.803000pt}%
\definecolor{currentstroke}{rgb}{0.000000,0.000000,0.000000}%
\pgfsetstrokecolor{currentstroke}%
\pgfsetdash{}{0pt}%
\pgfsys@defobject{currentmarker}{\pgfqpoint{-0.048611in}{0.000000in}}{\pgfqpoint{0.000000in}{0.000000in}}{%
\pgfpathmoveto{\pgfqpoint{0.000000in}{0.000000in}}%
\pgfpathlineto{\pgfqpoint{-0.048611in}{0.000000in}}%
\pgfusepath{stroke,fill}%
}%
\begin{pgfscope}%
\pgfsys@transformshift{0.800000in}{1.106857in}%
\pgfsys@useobject{currentmarker}{}%
\end{pgfscope}%
\end{pgfscope}%
\begin{pgfscope}%
\pgftext[x=0.386419in,y=1.058662in,left,base]{\sffamily\fontsize{10.000000}{12.000000}\selectfont \(\displaystyle 0.005\)}%
\end{pgfscope}%
\begin{pgfscope}%
\pgfpathrectangle{\pgfqpoint{0.800000in}{0.397100in}}{\pgfqpoint{4.960000in}{1.263500in}} %
\pgfusepath{clip}%
\pgfsetrectcap%
\pgfsetroundjoin%
\pgfsetlinewidth{1.505625pt}%
\definecolor{currentstroke}{rgb}{0.121569,0.466667,0.705882}%
\pgfsetstrokecolor{currentstroke}%
\pgfsetdash{}{0pt}%
\pgfpathmoveto{\pgfqpoint{1.025455in}{0.478049in}}%
\pgfpathlineto{\pgfqpoint{1.027863in}{0.471515in}}%
\pgfpathlineto{\pgfqpoint{1.028793in}{0.470972in}}%
\pgfpathlineto{\pgfqpoint{1.030665in}{0.473605in}}%
\pgfpathlineto{\pgfqpoint{1.039976in}{0.500884in}}%
\pgfpathlineto{\pgfqpoint{1.041923in}{0.501612in}}%
\pgfpathlineto{\pgfqpoint{1.044111in}{0.499759in}}%
\pgfpathlineto{\pgfqpoint{1.046602in}{0.494737in}}%
\pgfpathlineto{\pgfqpoint{1.059189in}{0.456981in}}%
\pgfpathlineto{\pgfqpoint{1.062211in}{0.454532in}}%
\pgfpathlineto{\pgfqpoint{1.064796in}{0.456773in}}%
\pgfpathlineto{\pgfqpoint{1.068894in}{0.468204in}}%
\pgfpathlineto{\pgfqpoint{1.074383in}{0.484974in}}%
\pgfpathlineto{\pgfqpoint{1.075361in}{0.485122in}}%
\pgfpathlineto{\pgfqpoint{1.077251in}{0.481470in}}%
\pgfpathlineto{\pgfqpoint{1.080711in}{0.471422in}}%
\pgfpathlineto{\pgfqpoint{1.081671in}{0.470888in}}%
\pgfpathlineto{\pgfqpoint{1.083589in}{0.473695in}}%
\pgfpathlineto{\pgfqpoint{1.092547in}{0.500526in}}%
\pgfpathlineto{\pgfqpoint{1.094466in}{0.501602in}}%
\pgfpathlineto{\pgfqpoint{1.096780in}{0.499961in}}%
\pgfpathlineto{\pgfqpoint{1.099456in}{0.494658in}}%
\pgfpathlineto{\pgfqpoint{1.112031in}{0.456945in}}%
\pgfpathlineto{\pgfqpoint{1.115067in}{0.454605in}}%
\pgfpathlineto{\pgfqpoint{1.117614in}{0.456927in}}%
\pgfpathlineto{\pgfqpoint{1.121632in}{0.468188in}}%
\pgfpathlineto{\pgfqpoint{1.126857in}{0.484754in}}%
\pgfpathlineto{\pgfqpoint{1.127887in}{0.485327in}}%
\pgfpathlineto{\pgfqpoint{1.129770in}{0.482363in}}%
\pgfpathlineto{\pgfqpoint{1.134099in}{0.470874in}}%
\pgfpathlineto{\pgfqpoint{1.135046in}{0.471109in}}%
\pgfpathlineto{\pgfqpoint{1.137508in}{0.476806in}}%
\pgfpathlineto{\pgfqpoint{1.144317in}{0.498847in}}%
\pgfpathlineto{\pgfqpoint{1.145957in}{0.501038in}}%
\pgfpathlineto{\pgfqpoint{1.147971in}{0.501420in}}%
\pgfpathlineto{\pgfqpoint{1.150379in}{0.498773in}}%
\pgfpathlineto{\pgfqpoint{1.156149in}{0.482720in}}%
\pgfpathlineto{\pgfqpoint{1.162320in}{0.462412in}}%
\pgfpathlineto{\pgfqpoint{1.165680in}{0.455837in}}%
\pgfpathlineto{\pgfqpoint{1.168614in}{0.454974in}}%
\pgfpathlineto{\pgfqpoint{1.170911in}{0.458016in}}%
\pgfpathlineto{\pgfqpoint{1.175513in}{0.472485in}}%
\pgfpathlineto{\pgfqpoint{1.179691in}{0.484957in}}%
\pgfpathlineto{\pgfqpoint{1.180728in}{0.485446in}}%
\pgfpathlineto{\pgfqpoint{1.182644in}{0.482251in}}%
\pgfpathlineto{\pgfqpoint{1.187039in}{0.470793in}}%
\pgfpathlineto{\pgfqpoint{1.187997in}{0.471165in}}%
\pgfpathlineto{\pgfqpoint{1.190589in}{0.477528in}}%
\pgfpathlineto{\pgfqpoint{1.197325in}{0.499109in}}%
\pgfpathlineto{\pgfqpoint{1.198980in}{0.501162in}}%
\pgfpathlineto{\pgfqpoint{1.201006in}{0.501332in}}%
\pgfpathlineto{\pgfqpoint{1.203423in}{0.498423in}}%
\pgfpathlineto{\pgfqpoint{1.209192in}{0.481993in}}%
\pgfpathlineto{\pgfqpoint{1.215466in}{0.461600in}}%
\pgfpathlineto{\pgfqpoint{1.218781in}{0.455621in}}%
\pgfpathlineto{\pgfqpoint{1.221660in}{0.455296in}}%
\pgfpathlineto{\pgfqpoint{1.225394in}{0.462501in}}%
\pgfpathlineto{\pgfqpoint{1.233570in}{0.485600in}}%
\pgfpathlineto{\pgfqpoint{1.235520in}{0.482167in}}%
\pgfpathlineto{\pgfqpoint{1.240034in}{0.470760in}}%
\pgfpathlineto{\pgfqpoint{1.241966in}{0.473124in}}%
\pgfpathlineto{\pgfqpoint{1.246311in}{0.488207in}}%
\pgfpathlineto{\pgfqpoint{1.250305in}{0.499353in}}%
\pgfpathlineto{\pgfqpoint{1.251970in}{0.501292in}}%
\pgfpathlineto{\pgfqpoint{1.254001in}{0.501294in}}%
\pgfpathlineto{\pgfqpoint{1.256419in}{0.498190in}}%
\pgfpathlineto{\pgfqpoint{1.262173in}{0.481525in}}%
\pgfpathlineto{\pgfqpoint{1.268501in}{0.461152in}}%
\pgfpathlineto{\pgfqpoint{1.271784in}{0.455597in}}%
\pgfpathlineto{\pgfqpoint{1.274606in}{0.455629in}}%
\pgfpathlineto{\pgfqpoint{1.278214in}{0.462863in}}%
\pgfpathlineto{\pgfqpoint{1.286144in}{0.485876in}}%
\pgfpathlineto{\pgfqpoint{1.288105in}{0.482974in}}%
\pgfpathlineto{\pgfqpoint{1.292682in}{0.470862in}}%
\pgfpathlineto{\pgfqpoint{1.293673in}{0.471191in}}%
\pgfpathlineto{\pgfqpoint{1.296336in}{0.477703in}}%
\pgfpathlineto{\pgfqpoint{1.303104in}{0.499344in}}%
\pgfpathlineto{\pgfqpoint{1.304738in}{0.501331in}}%
\pgfpathlineto{\pgfqpoint{1.306731in}{0.501468in}}%
\pgfpathlineto{\pgfqpoint{1.309105in}{0.498597in}}%
\pgfpathlineto{\pgfqpoint{1.314783in}{0.482418in}}%
\pgfpathlineto{\pgfqpoint{1.320949in}{0.462236in}}%
\pgfpathlineto{\pgfqpoint{1.324269in}{0.456112in}}%
\pgfpathlineto{\pgfqpoint{1.327124in}{0.455675in}}%
\pgfpathlineto{\pgfqpoint{1.330815in}{0.462635in}}%
\pgfpathlineto{\pgfqpoint{1.339234in}{0.486081in}}%
\pgfpathlineto{\pgfqpoint{1.341253in}{0.482219in}}%
\pgfpathlineto{\pgfqpoint{1.344955in}{0.471449in}}%
\pgfpathlineto{\pgfqpoint{1.345983in}{0.470960in}}%
\pgfpathlineto{\pgfqpoint{1.348050in}{0.474260in}}%
\pgfpathlineto{\pgfqpoint{1.358020in}{0.501740in}}%
\pgfpathlineto{\pgfqpoint{1.360080in}{0.501299in}}%
\pgfpathlineto{\pgfqpoint{1.362514in}{0.497670in}}%
\pgfpathlineto{\pgfqpoint{1.368245in}{0.480389in}}%
\pgfpathlineto{\pgfqpoint{1.374704in}{0.460235in}}%
\pgfpathlineto{\pgfqpoint{1.377892in}{0.455791in}}%
\pgfpathlineto{\pgfqpoint{1.380541in}{0.456621in}}%
\pgfpathlineto{\pgfqpoint{1.383896in}{0.463934in}}%
\pgfpathlineto{\pgfqpoint{1.390927in}{0.486178in}}%
\pgfpathlineto{\pgfqpoint{1.392003in}{0.486466in}}%
\pgfpathlineto{\pgfqpoint{1.394049in}{0.482572in}}%
\pgfpathlineto{\pgfqpoint{1.397802in}{0.471652in}}%
\pgfpathlineto{\pgfqpoint{1.398841in}{0.471169in}}%
\pgfpathlineto{\pgfqpoint{1.400938in}{0.474561in}}%
\pgfpathlineto{\pgfqpoint{1.409617in}{0.500943in}}%
\pgfpathlineto{\pgfqpoint{1.411396in}{0.502132in}}%
\pgfpathlineto{\pgfqpoint{1.413538in}{0.500947in}}%
\pgfpathlineto{\pgfqpoint{1.416043in}{0.496398in}}%
\pgfpathlineto{\pgfqpoint{1.428440in}{0.458763in}}%
\pgfpathlineto{\pgfqpoint{1.431482in}{0.455996in}}%
\pgfpathlineto{\pgfqpoint{1.433893in}{0.457832in}}%
\pgfpathlineto{\pgfqpoint{1.437889in}{0.468506in}}%
\pgfpathlineto{\pgfqpoint{1.443566in}{0.486498in}}%
\pgfpathlineto{\pgfqpoint{1.444669in}{0.486987in}}%
\pgfpathlineto{\pgfqpoint{1.446730in}{0.483370in}}%
\pgfpathlineto{\pgfqpoint{1.451500in}{0.471440in}}%
\pgfpathlineto{\pgfqpoint{1.453542in}{0.474207in}}%
\pgfpathlineto{\pgfqpoint{1.462621in}{0.501465in}}%
\pgfpathlineto{\pgfqpoint{1.464415in}{0.502493in}}%
\pgfpathlineto{\pgfqpoint{1.466569in}{0.501083in}}%
\pgfpathlineto{\pgfqpoint{1.469078in}{0.496296in}}%
\pgfpathlineto{\pgfqpoint{1.481410in}{0.458955in}}%
\pgfpathlineto{\pgfqpoint{1.484402in}{0.456527in}}%
\pgfpathlineto{\pgfqpoint{1.486743in}{0.458493in}}%
\pgfpathlineto{\pgfqpoint{1.490680in}{0.469150in}}%
\pgfpathlineto{\pgfqpoint{1.496687in}{0.487408in}}%
\pgfpathlineto{\pgfqpoint{1.497767in}{0.487387in}}%
\pgfpathlineto{\pgfqpoint{1.499907in}{0.482731in}}%
\pgfpathlineto{\pgfqpoint{1.503848in}{0.472018in}}%
\pgfpathlineto{\pgfqpoint{1.505008in}{0.472167in}}%
\pgfpathlineto{\pgfqpoint{1.507420in}{0.477604in}}%
\pgfpathlineto{\pgfqpoint{1.515915in}{0.502365in}}%
\pgfpathlineto{\pgfqpoint{1.517785in}{0.502874in}}%
\pgfpathlineto{\pgfqpoint{1.520013in}{0.500698in}}%
\pgfpathlineto{\pgfqpoint{1.525404in}{0.486212in}}%
\pgfpathlineto{\pgfqpoint{1.531167in}{0.466646in}}%
\pgfpathlineto{\pgfqpoint{1.534788in}{0.458695in}}%
\pgfpathlineto{\pgfqpoint{1.537651in}{0.457285in}}%
\pgfpathlineto{\pgfqpoint{1.539838in}{0.459699in}}%
\pgfpathlineto{\pgfqpoint{1.543589in}{0.470339in}}%
\pgfpathlineto{\pgfqpoint{1.549259in}{0.487927in}}%
\pgfpathlineto{\pgfqpoint{1.550373in}{0.488201in}}%
\pgfpathlineto{\pgfqpoint{1.552513in}{0.484016in}}%
\pgfpathlineto{\pgfqpoint{1.556451in}{0.472757in}}%
\pgfpathlineto{\pgfqpoint{1.557562in}{0.472479in}}%
\pgfpathlineto{\pgfqpoint{1.559726in}{0.476438in}}%
\pgfpathlineto{\pgfqpoint{1.567974in}{0.502037in}}%
\pgfpathlineto{\pgfqpoint{1.569663in}{0.503471in}}%
\pgfpathlineto{\pgfqpoint{1.571694in}{0.502787in}}%
\pgfpathlineto{\pgfqpoint{1.574075in}{0.498983in}}%
\pgfpathlineto{\pgfqpoint{1.579644in}{0.481887in}}%
\pgfpathlineto{\pgfqpoint{1.585974in}{0.462241in}}%
\pgfpathlineto{\pgfqpoint{1.589067in}{0.458102in}}%
\pgfpathlineto{\pgfqpoint{1.591575in}{0.458952in}}%
\pgfpathlineto{\pgfqpoint{1.594845in}{0.466022in}}%
\pgfpathlineto{\pgfqpoint{1.602103in}{0.488820in}}%
\pgfpathlineto{\pgfqpoint{1.603224in}{0.488990in}}%
\pgfpathlineto{\pgfqpoint{1.605405in}{0.484518in}}%
\pgfpathlineto{\pgfqpoint{1.609430in}{0.473292in}}%
\pgfpathlineto{\pgfqpoint{1.610589in}{0.473257in}}%
\pgfpathlineto{\pgfqpoint{1.612762in}{0.477638in}}%
\pgfpathlineto{\pgfqpoint{1.620953in}{0.502913in}}%
\pgfpathlineto{\pgfqpoint{1.622655in}{0.504225in}}%
\pgfpathlineto{\pgfqpoint{1.624694in}{0.503363in}}%
\pgfpathlineto{\pgfqpoint{1.627078in}{0.499365in}}%
\pgfpathlineto{\pgfqpoint{1.632624in}{0.482112in}}%
\pgfpathlineto{\pgfqpoint{1.638958in}{0.462715in}}%
\pgfpathlineto{\pgfqpoint{1.641997in}{0.458948in}}%
\pgfpathlineto{\pgfqpoint{1.644438in}{0.459983in}}%
\pgfpathlineto{\pgfqpoint{1.647645in}{0.467064in}}%
\pgfpathlineto{\pgfqpoint{1.655194in}{0.490014in}}%
\pgfpathlineto{\pgfqpoint{1.656296in}{0.489719in}}%
\pgfpathlineto{\pgfqpoint{1.658399in}{0.484823in}}%
\pgfpathlineto{\pgfqpoint{1.662635in}{0.473885in}}%
\pgfpathlineto{\pgfqpoint{1.663732in}{0.474345in}}%
\pgfpathlineto{\pgfqpoint{1.666538in}{0.481490in}}%
\pgfpathlineto{\pgfqpoint{1.673916in}{0.503907in}}%
\pgfpathlineto{\pgfqpoint{1.675626in}{0.505111in}}%
\pgfpathlineto{\pgfqpoint{1.677671in}{0.504101in}}%
\pgfpathlineto{\pgfqpoint{1.680053in}{0.499948in}}%
\pgfpathlineto{\pgfqpoint{1.685572in}{0.482600in}}%
\pgfpathlineto{\pgfqpoint{1.691889in}{0.463466in}}%
\pgfpathlineto{\pgfqpoint{1.694878in}{0.459977in}}%
\pgfpathlineto{\pgfqpoint{1.697266in}{0.461130in}}%
\pgfpathlineto{\pgfqpoint{1.700427in}{0.468205in}}%
\pgfpathlineto{\pgfqpoint{1.707705in}{0.490960in}}%
\pgfpathlineto{\pgfqpoint{1.708847in}{0.491050in}}%
\pgfpathlineto{\pgfqpoint{1.711100in}{0.486241in}}%
\pgfpathlineto{\pgfqpoint{1.715258in}{0.474957in}}%
\pgfpathlineto{\pgfqpoint{1.716473in}{0.475191in}}%
\pgfpathlineto{\pgfqpoint{1.719012in}{0.481139in}}%
\pgfpathlineto{\pgfqpoint{1.726849in}{0.505016in}}%
\pgfpathlineto{\pgfqpoint{1.728562in}{0.506145in}}%
\pgfpathlineto{\pgfqpoint{1.730605in}{0.505038in}}%
\pgfpathlineto{\pgfqpoint{1.732980in}{0.500796in}}%
\pgfpathlineto{\pgfqpoint{1.738464in}{0.483444in}}%
\pgfpathlineto{\pgfqpoint{1.744747in}{0.464532in}}%
\pgfpathlineto{\pgfqpoint{1.747695in}{0.461194in}}%
\pgfpathlineto{\pgfqpoint{1.750047in}{0.462384in}}%
\pgfpathlineto{\pgfqpoint{1.753185in}{0.469434in}}%
\pgfpathlineto{\pgfqpoint{1.760774in}{0.492445in}}%
\pgfpathlineto{\pgfqpoint{1.761900in}{0.492100in}}%
\pgfpathlineto{\pgfqpoint{1.764053in}{0.486994in}}%
\pgfpathlineto{\pgfqpoint{1.767350in}{0.476966in}}%
\pgfpathlineto{\pgfqpoint{1.768375in}{0.475964in}}%
\pgfpathlineto{\pgfqpoint{1.769500in}{0.476520in}}%
\pgfpathlineto{\pgfqpoint{1.772486in}{0.484434in}}%
\pgfpathlineto{\pgfqpoint{1.778833in}{0.504881in}}%
\pgfpathlineto{\pgfqpoint{1.780357in}{0.506862in}}%
\pgfpathlineto{\pgfqpoint{1.782186in}{0.507228in}}%
\pgfpathlineto{\pgfqpoint{1.784347in}{0.504988in}}%
\pgfpathlineto{\pgfqpoint{1.789534in}{0.490902in}}%
\pgfpathlineto{\pgfqpoint{1.795085in}{0.471970in}}%
\pgfpathlineto{\pgfqpoint{1.798537in}{0.464229in}}%
\pgfpathlineto{\pgfqpoint{1.801281in}{0.462606in}}%
\pgfpathlineto{\pgfqpoint{1.803414in}{0.464685in}}%
\pgfpathlineto{\pgfqpoint{1.807314in}{0.475512in}}%
\pgfpathlineto{\pgfqpoint{1.812891in}{0.493326in}}%
\pgfpathlineto{\pgfqpoint{1.814095in}{0.493970in}}%
\pgfpathlineto{\pgfqpoint{1.815198in}{0.492838in}}%
\pgfpathlineto{\pgfqpoint{1.821640in}{0.477283in}}%
\pgfpathlineto{\pgfqpoint{1.823992in}{0.481300in}}%
\pgfpathlineto{\pgfqpoint{1.833049in}{0.508091in}}%
\pgfpathlineto{\pgfqpoint{1.834834in}{0.508673in}}%
\pgfpathlineto{\pgfqpoint{1.836944in}{0.506773in}}%
\pgfpathlineto{\pgfqpoint{1.842036in}{0.493453in}}%
\pgfpathlineto{\pgfqpoint{1.847586in}{0.474393in}}%
\pgfpathlineto{\pgfqpoint{1.851020in}{0.466252in}}%
\pgfpathlineto{\pgfqpoint{1.853790in}{0.464152in}}%
\pgfpathlineto{\pgfqpoint{1.855965in}{0.465943in}}%
\pgfpathlineto{\pgfqpoint{1.859975in}{0.476762in}}%
\pgfpathlineto{\pgfqpoint{1.865662in}{0.494979in}}%
\pgfpathlineto{\pgfqpoint{1.866879in}{0.495631in}}%
\pgfpathlineto{\pgfqpoint{1.867996in}{0.494476in}}%
\pgfpathlineto{\pgfqpoint{1.874543in}{0.478804in}}%
\pgfpathlineto{\pgfqpoint{1.876918in}{0.483012in}}%
\pgfpathlineto{\pgfqpoint{1.885200in}{0.508837in}}%
\pgfpathlineto{\pgfqpoint{1.886835in}{0.510223in}}%
\pgfpathlineto{\pgfqpoint{1.888781in}{0.509594in}}%
\pgfpathlineto{\pgfqpoint{1.891048in}{0.506045in}}%
\pgfpathlineto{\pgfqpoint{1.896330in}{0.489963in}}%
\pgfpathlineto{\pgfqpoint{1.902288in}{0.470988in}}%
\pgfpathlineto{\pgfqpoint{1.905266in}{0.466344in}}%
\pgfpathlineto{\pgfqpoint{1.907714in}{0.466512in}}%
\pgfpathlineto{\pgfqpoint{1.911062in}{0.473035in}}%
\pgfpathlineto{\pgfqpoint{1.918953in}{0.497351in}}%
\pgfpathlineto{\pgfqpoint{1.920134in}{0.497210in}}%
\pgfpathlineto{\pgfqpoint{1.922387in}{0.492115in}}%
\pgfpathlineto{\pgfqpoint{1.925862in}{0.481508in}}%
\pgfpathlineto{\pgfqpoint{1.926922in}{0.480478in}}%
\pgfpathlineto{\pgfqpoint{1.928220in}{0.481262in}}%
\pgfpathlineto{\pgfqpoint{1.931331in}{0.489976in}}%
\pgfpathlineto{\pgfqpoint{1.936560in}{0.507781in}}%
\pgfpathlineto{\pgfqpoint{1.939470in}{0.511909in}}%
\pgfpathlineto{\pgfqpoint{1.941363in}{0.511602in}}%
\pgfpathlineto{\pgfqpoint{1.943572in}{0.508503in}}%
\pgfpathlineto{\pgfqpoint{1.948760in}{0.493211in}}%
\pgfpathlineto{\pgfqpoint{1.954469in}{0.474442in}}%
\pgfpathlineto{\pgfqpoint{1.957773in}{0.468555in}}%
\pgfpathlineto{\pgfqpoint{1.960252in}{0.468335in}}%
\pgfpathlineto{\pgfqpoint{1.963675in}{0.474620in}}%
\pgfpathlineto{\pgfqpoint{1.972544in}{0.499531in}}%
\pgfpathlineto{\pgfqpoint{1.973687in}{0.498191in}}%
\pgfpathlineto{\pgfqpoint{1.980547in}{0.482603in}}%
\pgfpathlineto{\pgfqpoint{1.983283in}{0.488696in}}%
\pgfpathlineto{\pgfqpoint{1.990277in}{0.511543in}}%
\pgfpathlineto{\pgfqpoint{1.991767in}{0.513530in}}%
\pgfpathlineto{\pgfqpoint{1.993545in}{0.513966in}}%
\pgfpathlineto{\pgfqpoint{1.995634in}{0.511918in}}%
\pgfpathlineto{\pgfqpoint{2.000635in}{0.498587in}}%
\pgfpathlineto{\pgfqpoint{2.006043in}{0.480032in}}%
\pgfpathlineto{\pgfqpoint{2.009374in}{0.472204in}}%
\pgfpathlineto{\pgfqpoint{2.012056in}{0.470169in}}%
\pgfpathlineto{\pgfqpoint{2.014185in}{0.471870in}}%
\pgfpathlineto{\pgfqpoint{2.017145in}{0.479017in}}%
\pgfpathlineto{\pgfqpoint{2.024888in}{0.501878in}}%
\pgfpathlineto{\pgfqpoint{2.026065in}{0.501158in}}%
\pgfpathlineto{\pgfqpoint{2.029095in}{0.492547in}}%
\pgfpathlineto{\pgfqpoint{2.031645in}{0.485408in}}%
\pgfpathlineto{\pgfqpoint{2.032753in}{0.484515in}}%
\pgfpathlineto{\pgfqpoint{2.033947in}{0.485358in}}%
\pgfpathlineto{\pgfqpoint{2.037012in}{0.493929in}}%
\pgfpathlineto{\pgfqpoint{2.042997in}{0.513481in}}%
\pgfpathlineto{\pgfqpoint{2.046188in}{0.516239in}}%
\pgfpathlineto{\pgfqpoint{2.048229in}{0.514531in}}%
\pgfpathlineto{\pgfqpoint{2.053139in}{0.501972in}}%
\pgfpathlineto{\pgfqpoint{2.061837in}{0.475164in}}%
\pgfpathlineto{\pgfqpoint{2.064546in}{0.472625in}}%
\pgfpathlineto{\pgfqpoint{2.066719in}{0.474000in}}%
\pgfpathlineto{\pgfqpoint{2.069760in}{0.481021in}}%
\pgfpathlineto{\pgfqpoint{2.077020in}{0.504024in}}%
\pgfpathlineto{\pgfqpoint{2.078267in}{0.504274in}}%
\pgfpathlineto{\pgfqpoint{2.080731in}{0.499106in}}%
\pgfpathlineto{\pgfqpoint{2.085305in}{0.486968in}}%
\pgfpathlineto{\pgfqpoint{2.086625in}{0.487504in}}%
\pgfpathlineto{\pgfqpoint{2.089734in}{0.495816in}}%
\pgfpathlineto{\pgfqpoint{2.095896in}{0.515989in}}%
\pgfpathlineto{\pgfqpoint{2.099086in}{0.518681in}}%
\pgfpathlineto{\pgfqpoint{2.101120in}{0.516933in}}%
\pgfpathlineto{\pgfqpoint{2.106002in}{0.504377in}}%
\pgfpathlineto{\pgfqpoint{2.114623in}{0.477832in}}%
\pgfpathlineto{\pgfqpoint{2.117304in}{0.475322in}}%
\pgfpathlineto{\pgfqpoint{2.119462in}{0.476673in}}%
\pgfpathlineto{\pgfqpoint{2.122506in}{0.483689in}}%
\pgfpathlineto{\pgfqpoint{2.129582in}{0.506553in}}%
\pgfpathlineto{\pgfqpoint{2.130862in}{0.507138in}}%
\pgfpathlineto{\pgfqpoint{2.132051in}{0.505775in}}%
\pgfpathlineto{\pgfqpoint{2.139194in}{0.489827in}}%
\pgfpathlineto{\pgfqpoint{2.142024in}{0.496436in}}%
\pgfpathlineto{\pgfqpoint{2.149289in}{0.519598in}}%
\pgfpathlineto{\pgfqpoint{2.150842in}{0.521217in}}%
\pgfpathlineto{\pgfqpoint{2.152679in}{0.521044in}}%
\pgfpathlineto{\pgfqpoint{2.154814in}{0.518209in}}%
\pgfpathlineto{\pgfqpoint{2.159819in}{0.503736in}}%
\pgfpathlineto{\pgfqpoint{2.165210in}{0.485708in}}%
\pgfpathlineto{\pgfqpoint{2.168397in}{0.479330in}}%
\pgfpathlineto{\pgfqpoint{2.170876in}{0.478410in}}%
\pgfpathlineto{\pgfqpoint{2.172848in}{0.480600in}}%
\pgfpathlineto{\pgfqpoint{2.176802in}{0.491977in}}%
\pgfpathlineto{\pgfqpoint{2.181543in}{0.508187in}}%
\pgfpathlineto{\pgfqpoint{2.182890in}{0.510072in}}%
\pgfpathlineto{\pgfqpoint{2.184135in}{0.509823in}}%
\pgfpathlineto{\pgfqpoint{2.186519in}{0.504218in}}%
\pgfpathlineto{\pgfqpoint{2.190154in}{0.493310in}}%
\pgfpathlineto{\pgfqpoint{2.191295in}{0.492389in}}%
\pgfpathlineto{\pgfqpoint{2.192664in}{0.493488in}}%
\pgfpathlineto{\pgfqpoint{2.195710in}{0.502299in}}%
\pgfpathlineto{\pgfqpoint{2.202208in}{0.522631in}}%
\pgfpathlineto{\pgfqpoint{2.203765in}{0.524200in}}%
\pgfpathlineto{\pgfqpoint{2.205603in}{0.523953in}}%
\pgfpathlineto{\pgfqpoint{2.207734in}{0.521041in}}%
\pgfpathlineto{\pgfqpoint{2.212714in}{0.506530in}}%
\pgfpathlineto{\pgfqpoint{2.218109in}{0.488578in}}%
\pgfpathlineto{\pgfqpoint{2.221261in}{0.482432in}}%
\pgfpathlineto{\pgfqpoint{2.223700in}{0.481649in}}%
\pgfpathlineto{\pgfqpoint{2.225649in}{0.483885in}}%
\pgfpathlineto{\pgfqpoint{2.229607in}{0.495379in}}%
\pgfpathlineto{\pgfqpoint{2.234712in}{0.512235in}}%
\pgfpathlineto{\pgfqpoint{2.236052in}{0.513499in}}%
\pgfpathlineto{\pgfqpoint{2.237285in}{0.512657in}}%
\pgfpathlineto{\pgfqpoint{2.240446in}{0.503519in}}%
\pgfpathlineto{\pgfqpoint{2.243146in}{0.496236in}}%
\pgfpathlineto{\pgfqpoint{2.244335in}{0.495544in}}%
\pgfpathlineto{\pgfqpoint{2.245713in}{0.496972in}}%
\pgfpathlineto{\pgfqpoint{2.248964in}{0.506951in}}%
\pgfpathlineto{\pgfqpoint{2.254652in}{0.525142in}}%
\pgfpathlineto{\pgfqpoint{2.256127in}{0.527094in}}%
\pgfpathlineto{\pgfqpoint{2.257869in}{0.527499in}}%
\pgfpathlineto{\pgfqpoint{2.259900in}{0.525494in}}%
\pgfpathlineto{\pgfqpoint{2.264720in}{0.512680in}}%
\pgfpathlineto{\pgfqpoint{2.273096in}{0.487195in}}%
\pgfpathlineto{\pgfqpoint{2.275678in}{0.484958in}}%
\pgfpathlineto{\pgfqpoint{2.277776in}{0.486363in}}%
\pgfpathlineto{\pgfqpoint{2.280812in}{0.493458in}}%
\pgfpathlineto{\pgfqpoint{2.288301in}{0.516822in}}%
\pgfpathlineto{\pgfqpoint{2.289586in}{0.516784in}}%
\pgfpathlineto{\pgfqpoint{2.292042in}{0.511273in}}%
\pgfpathlineto{\pgfqpoint{2.295826in}{0.499889in}}%
\pgfpathlineto{\pgfqpoint{2.296990in}{0.498950in}}%
\pgfpathlineto{\pgfqpoint{2.298384in}{0.500081in}}%
\pgfpathlineto{\pgfqpoint{2.301489in}{0.509124in}}%
\pgfpathlineto{\pgfqpoint{2.306805in}{0.527175in}}%
\pgfpathlineto{\pgfqpoint{2.309741in}{0.531019in}}%
\pgfpathlineto{\pgfqpoint{2.311603in}{0.530447in}}%
\pgfpathlineto{\pgfqpoint{2.313748in}{0.527150in}}%
\pgfpathlineto{\pgfqpoint{2.318704in}{0.512227in}}%
\pgfpathlineto{\pgfqpoint{2.324211in}{0.494386in}}%
\pgfpathlineto{\pgfqpoint{2.327281in}{0.489216in}}%
\pgfpathlineto{\pgfqpoint{2.329591in}{0.489123in}}%
\pgfpathlineto{\pgfqpoint{2.332942in}{0.495324in}}%
\pgfpathlineto{\pgfqpoint{2.341040in}{0.520636in}}%
\pgfpathlineto{\pgfqpoint{2.342342in}{0.520653in}}%
\pgfpathlineto{\pgfqpoint{2.344831in}{0.515130in}}%
\pgfpathlineto{\pgfqpoint{2.348672in}{0.503586in}}%
\pgfpathlineto{\pgfqpoint{2.349850in}{0.502653in}}%
\pgfpathlineto{\pgfqpoint{2.351257in}{0.503822in}}%
\pgfpathlineto{\pgfqpoint{2.354438in}{0.513183in}}%
\pgfpathlineto{\pgfqpoint{2.360135in}{0.531941in}}%
\pgfpathlineto{\pgfqpoint{2.363240in}{0.534868in}}%
\pgfpathlineto{\pgfqpoint{2.365195in}{0.533404in}}%
\pgfpathlineto{\pgfqpoint{2.369867in}{0.521835in}}%
\pgfpathlineto{\pgfqpoint{2.378128in}{0.495915in}}%
\pgfpathlineto{\pgfqpoint{2.380745in}{0.492869in}}%
\pgfpathlineto{\pgfqpoint{2.382907in}{0.493698in}}%
\pgfpathlineto{\pgfqpoint{2.386074in}{0.500502in}}%
\pgfpathlineto{\pgfqpoint{2.393405in}{0.524332in}}%
\pgfpathlineto{\pgfqpoint{2.394756in}{0.525008in}}%
\pgfpathlineto{\pgfqpoint{2.396023in}{0.523564in}}%
\pgfpathlineto{\pgfqpoint{2.402306in}{0.506765in}}%
\pgfpathlineto{\pgfqpoint{2.403674in}{0.507212in}}%
\pgfpathlineto{\pgfqpoint{2.406651in}{0.514826in}}%
\pgfpathlineto{\pgfqpoint{2.413255in}{0.536504in}}%
\pgfpathlineto{\pgfqpoint{2.414726in}{0.538463in}}%
\pgfpathlineto{\pgfqpoint{2.416449in}{0.538871in}}%
\pgfpathlineto{\pgfqpoint{2.418447in}{0.536911in}}%
\pgfpathlineto{\pgfqpoint{2.423166in}{0.524423in}}%
\pgfpathlineto{\pgfqpoint{2.431330in}{0.499490in}}%
\pgfpathlineto{\pgfqpoint{2.433860in}{0.497146in}}%
\pgfpathlineto{\pgfqpoint{2.435942in}{0.498391in}}%
\pgfpathlineto{\pgfqpoint{2.439029in}{0.505488in}}%
\pgfpathlineto{\pgfqpoint{2.446410in}{0.529043in}}%
\pgfpathlineto{\pgfqpoint{2.447756in}{0.529346in}}%
\pgfpathlineto{\pgfqpoint{2.449040in}{0.527540in}}%
\pgfpathlineto{\pgfqpoint{2.455569in}{0.510957in}}%
\pgfpathlineto{\pgfqpoint{2.457000in}{0.512198in}}%
\pgfpathlineto{\pgfqpoint{2.460300in}{0.522070in}}%
\pgfpathlineto{\pgfqpoint{2.466206in}{0.541040in}}%
\pgfpathlineto{\pgfqpoint{2.467691in}{0.542930in}}%
\pgfpathlineto{\pgfqpoint{2.469425in}{0.543219in}}%
\pgfpathlineto{\pgfqpoint{2.471429in}{0.541103in}}%
\pgfpathlineto{\pgfqpoint{2.476139in}{0.528400in}}%
\pgfpathlineto{\pgfqpoint{2.484211in}{0.503971in}}%
\pgfpathlineto{\pgfqpoint{2.486703in}{0.501807in}}%
\pgfpathlineto{\pgfqpoint{2.488758in}{0.503138in}}%
\pgfpathlineto{\pgfqpoint{2.491837in}{0.510329in}}%
\pgfpathlineto{\pgfqpoint{2.498588in}{0.532968in}}%
\pgfpathlineto{\pgfqpoint{2.499995in}{0.534251in}}%
\pgfpathlineto{\pgfqpoint{2.501300in}{0.533273in}}%
\pgfpathlineto{\pgfqpoint{2.504668in}{0.523377in}}%
\pgfpathlineto{\pgfqpoint{2.507586in}{0.516011in}}%
\pgfpathlineto{\pgfqpoint{2.508885in}{0.515712in}}%
\pgfpathlineto{\pgfqpoint{2.511621in}{0.521249in}}%
\pgfpathlineto{\pgfqpoint{2.520183in}{0.547302in}}%
\pgfpathlineto{\pgfqpoint{2.521851in}{0.548045in}}%
\pgfpathlineto{\pgfqpoint{2.523782in}{0.546576in}}%
\pgfpathlineto{\pgfqpoint{2.528367in}{0.535212in}}%
\pgfpathlineto{\pgfqpoint{2.536418in}{0.509961in}}%
\pgfpathlineto{\pgfqpoint{2.538973in}{0.506908in}}%
\pgfpathlineto{\pgfqpoint{2.541108in}{0.507629in}}%
\pgfpathlineto{\pgfqpoint{2.544312in}{0.514440in}}%
\pgfpathlineto{\pgfqpoint{2.551661in}{0.538502in}}%
\pgfpathlineto{\pgfqpoint{2.553058in}{0.539297in}}%
\pgfpathlineto{\pgfqpoint{2.554372in}{0.537860in}}%
\pgfpathlineto{\pgfqpoint{2.560898in}{0.520602in}}%
\pgfpathlineto{\pgfqpoint{2.562306in}{0.521182in}}%
\pgfpathlineto{\pgfqpoint{2.565379in}{0.529289in}}%
\pgfpathlineto{\pgfqpoint{2.571420in}{0.549751in}}%
\pgfpathlineto{\pgfqpoint{2.574462in}{0.553088in}}%
\pgfpathlineto{\pgfqpoint{2.576348in}{0.551986in}}%
\pgfpathlineto{\pgfqpoint{2.578490in}{0.548132in}}%
\pgfpathlineto{\pgfqpoint{2.591376in}{0.512411in}}%
\pgfpathlineto{\pgfqpoint{2.593563in}{0.512652in}}%
\pgfpathlineto{\pgfqpoint{2.596861in}{0.519137in}}%
\pgfpathlineto{\pgfqpoint{2.605206in}{0.544637in}}%
\pgfpathlineto{\pgfqpoint{2.606559in}{0.544181in}}%
\pgfpathlineto{\pgfqpoint{2.609160in}{0.537725in}}%
\pgfpathlineto{\pgfqpoint{2.613038in}{0.526505in}}%
\pgfpathlineto{\pgfqpoint{2.614311in}{0.525821in}}%
\pgfpathlineto{\pgfqpoint{2.615780in}{0.527450in}}%
\pgfpathlineto{\pgfqpoint{2.619612in}{0.539808in}}%
\pgfpathlineto{\pgfqpoint{2.624529in}{0.555607in}}%
\pgfpathlineto{\pgfqpoint{2.627652in}{0.558433in}}%
\pgfpathlineto{\pgfqpoint{2.629578in}{0.556872in}}%
\pgfpathlineto{\pgfqpoint{2.634121in}{0.545459in}}%
\pgfpathlineto{\pgfqpoint{2.642035in}{0.520780in}}%
\pgfpathlineto{\pgfqpoint{2.644538in}{0.517868in}}%
\pgfpathlineto{\pgfqpoint{2.646643in}{0.518623in}}%
\pgfpathlineto{\pgfqpoint{2.649858in}{0.525533in}}%
\pgfpathlineto{\pgfqpoint{2.658319in}{0.550482in}}%
\pgfpathlineto{\pgfqpoint{2.659671in}{0.549486in}}%
\pgfpathlineto{\pgfqpoint{2.663163in}{0.539227in}}%
\pgfpathlineto{\pgfqpoint{2.666220in}{0.531747in}}%
\pgfpathlineto{\pgfqpoint{2.667574in}{0.531648in}}%
\pgfpathlineto{\pgfqpoint{2.670382in}{0.537740in}}%
\pgfpathlineto{\pgfqpoint{2.677980in}{0.562353in}}%
\pgfpathlineto{\pgfqpoint{2.679518in}{0.563975in}}%
\pgfpathlineto{\pgfqpoint{2.681289in}{0.563805in}}%
\pgfpathlineto{\pgfqpoint{2.683312in}{0.561124in}}%
\pgfpathlineto{\pgfqpoint{2.687974in}{0.547761in}}%
\pgfpathlineto{\pgfqpoint{2.692913in}{0.531261in}}%
\pgfpathlineto{\pgfqpoint{2.695801in}{0.525161in}}%
\pgfpathlineto{\pgfqpoint{2.698141in}{0.523805in}}%
\pgfpathlineto{\pgfqpoint{2.700091in}{0.525547in}}%
\pgfpathlineto{\pgfqpoint{2.703145in}{0.533259in}}%
\pgfpathlineto{\pgfqpoint{2.709493in}{0.554929in}}%
\pgfpathlineto{\pgfqpoint{2.710962in}{0.556555in}}%
\pgfpathlineto{\pgfqpoint{2.712334in}{0.555764in}}%
\pgfpathlineto{\pgfqpoint{2.715865in}{0.545654in}}%
\pgfpathlineto{\pgfqpoint{2.718943in}{0.537833in}}%
\pgfpathlineto{\pgfqpoint{2.720284in}{0.537526in}}%
\pgfpathlineto{\pgfqpoint{2.723136in}{0.543389in}}%
\pgfpathlineto{\pgfqpoint{2.730729in}{0.568219in}}%
\pgfpathlineto{\pgfqpoint{2.732253in}{0.569964in}}%
\pgfpathlineto{\pgfqpoint{2.734001in}{0.569966in}}%
\pgfpathlineto{\pgfqpoint{2.735996in}{0.567517in}}%
\pgfpathlineto{\pgfqpoint{2.740601in}{0.554615in}}%
\pgfpathlineto{\pgfqpoint{2.745392in}{0.538449in}}%
\pgfpathlineto{\pgfqpoint{2.748262in}{0.531956in}}%
\pgfpathlineto{\pgfqpoint{2.750631in}{0.530135in}}%
\pgfpathlineto{\pgfqpoint{2.752620in}{0.531567in}}%
\pgfpathlineto{\pgfqpoint{2.755744in}{0.539110in}}%
\pgfpathlineto{\pgfqpoint{2.763006in}{0.562500in}}%
\pgfpathlineto{\pgfqpoint{2.764438in}{0.562896in}}%
\pgfpathlineto{\pgfqpoint{2.765819in}{0.560972in}}%
\pgfpathlineto{\pgfqpoint{2.772818in}{0.543733in}}%
\pgfpathlineto{\pgfqpoint{2.774319in}{0.545347in}}%
\pgfpathlineto{\pgfqpoint{2.777597in}{0.555499in}}%
\pgfpathlineto{\pgfqpoint{2.782569in}{0.572563in}}%
\pgfpathlineto{\pgfqpoint{2.785559in}{0.576589in}}%
\pgfpathlineto{\pgfqpoint{2.787367in}{0.576008in}}%
\pgfpathlineto{\pgfqpoint{2.789416in}{0.572836in}}%
\pgfpathlineto{\pgfqpoint{2.794071in}{0.558895in}}%
\pgfpathlineto{\pgfqpoint{2.799184in}{0.542416in}}%
\pgfpathlineto{\pgfqpoint{2.801992in}{0.537500in}}%
\pgfpathlineto{\pgfqpoint{2.804200in}{0.537108in}}%
\pgfpathlineto{\pgfqpoint{2.806050in}{0.539401in}}%
\pgfpathlineto{\pgfqpoint{2.810295in}{0.552336in}}%
\pgfpathlineto{\pgfqpoint{2.815129in}{0.568356in}}%
\pgfpathlineto{\pgfqpoint{2.816619in}{0.569850in}}%
\pgfpathlineto{\pgfqpoint{2.818018in}{0.568879in}}%
\pgfpathlineto{\pgfqpoint{2.821635in}{0.558311in}}%
\pgfpathlineto{\pgfqpoint{2.824824in}{0.550682in}}%
\pgfpathlineto{\pgfqpoint{2.826224in}{0.550735in}}%
\pgfpathlineto{\pgfqpoint{2.829106in}{0.557293in}}%
\pgfpathlineto{\pgfqpoint{2.837058in}{0.582308in}}%
\pgfpathlineto{\pgfqpoint{2.838669in}{0.583437in}}%
\pgfpathlineto{\pgfqpoint{2.840507in}{0.582521in}}%
\pgfpathlineto{\pgfqpoint{2.842579in}{0.578963in}}%
\pgfpathlineto{\pgfqpoint{2.849659in}{0.556229in}}%
\pgfpathlineto{\pgfqpoint{2.852429in}{0.548381in}}%
\pgfpathlineto{\pgfqpoint{2.855188in}{0.544261in}}%
\pgfpathlineto{\pgfqpoint{2.857317in}{0.544464in}}%
\pgfpathlineto{\pgfqpoint{2.860664in}{0.551085in}}%
\pgfpathlineto{\pgfqpoint{2.868490in}{0.576431in}}%
\pgfpathlineto{\pgfqpoint{2.869953in}{0.576943in}}%
\pgfpathlineto{\pgfqpoint{2.871365in}{0.575059in}}%
\pgfpathlineto{\pgfqpoint{2.878550in}{0.557518in}}%
\pgfpathlineto{\pgfqpoint{2.880070in}{0.559222in}}%
\pgfpathlineto{\pgfqpoint{2.884187in}{0.572721in}}%
\pgfpathlineto{\pgfqpoint{2.889350in}{0.588631in}}%
\pgfpathlineto{\pgfqpoint{2.890892in}{0.590380in}}%
\pgfpathlineto{\pgfqpoint{2.892637in}{0.590334in}}%
\pgfpathlineto{\pgfqpoint{2.894614in}{0.587846in}}%
\pgfpathlineto{\pgfqpoint{2.899140in}{0.575109in}}%
\pgfpathlineto{\pgfqpoint{2.903847in}{0.559340in}}%
\pgfpathlineto{\pgfqpoint{2.906628in}{0.553159in}}%
\pgfpathlineto{\pgfqpoint{2.908940in}{0.551443in}}%
\pgfpathlineto{\pgfqpoint{2.910906in}{0.552894in}}%
\pgfpathlineto{\pgfqpoint{2.914078in}{0.560667in}}%
\pgfpathlineto{\pgfqpoint{2.921248in}{0.583986in}}%
\pgfpathlineto{\pgfqpoint{2.922725in}{0.584529in}}%
\pgfpathlineto{\pgfqpoint{2.924152in}{0.582642in}}%
\pgfpathlineto{\pgfqpoint{2.931424in}{0.564989in}}%
\pgfpathlineto{\pgfqpoint{2.932954in}{0.566752in}}%
\pgfpathlineto{\pgfqpoint{2.936324in}{0.577387in}}%
\pgfpathlineto{\pgfqpoint{2.941510in}{0.594829in}}%
\pgfpathlineto{\pgfqpoint{2.944625in}{0.598185in}}%
\pgfpathlineto{\pgfqpoint{2.946484in}{0.596960in}}%
\pgfpathlineto{\pgfqpoint{2.950823in}{0.586605in}}%
\pgfpathlineto{\pgfqpoint{2.958332in}{0.562933in}}%
\pgfpathlineto{\pgfqpoint{2.961017in}{0.559451in}}%
\pgfpathlineto{\pgfqpoint{2.963081in}{0.560049in}}%
\pgfpathlineto{\pgfqpoint{2.966394in}{0.567129in}}%
\pgfpathlineto{\pgfqpoint{2.974265in}{0.592215in}}%
\pgfpathlineto{\pgfqpoint{2.975740in}{0.592354in}}%
\pgfpathlineto{\pgfqpoint{2.978713in}{0.585631in}}%
\pgfpathlineto{\pgfqpoint{2.983022in}{0.573404in}}%
\pgfpathlineto{\pgfqpoint{2.984396in}{0.572893in}}%
\pgfpathlineto{\pgfqpoint{2.985938in}{0.574881in}}%
\pgfpathlineto{\pgfqpoint{2.990078in}{0.588777in}}%
\pgfpathlineto{\pgfqpoint{2.995170in}{0.604281in}}%
\pgfpathlineto{\pgfqpoint{2.996735in}{0.605960in}}%
\pgfpathlineto{\pgfqpoint{2.998491in}{0.605771in}}%
\pgfpathlineto{\pgfqpoint{3.000468in}{0.603117in}}%
\pgfpathlineto{\pgfqpoint{3.004956in}{0.590274in}}%
\pgfpathlineto{\pgfqpoint{3.009695in}{0.574616in}}%
\pgfpathlineto{\pgfqpoint{3.012412in}{0.568943in}}%
\pgfpathlineto{\pgfqpoint{3.014663in}{0.567610in}}%
\pgfpathlineto{\pgfqpoint{3.016589in}{0.569291in}}%
\pgfpathlineto{\pgfqpoint{3.019929in}{0.577994in}}%
\pgfpathlineto{\pgfqpoint{3.026186in}{0.599381in}}%
\pgfpathlineto{\pgfqpoint{3.027728in}{0.600969in}}%
\pgfpathlineto{\pgfqpoint{3.029186in}{0.599970in}}%
\pgfpathlineto{\pgfqpoint{3.032972in}{0.588930in}}%
\pgfpathlineto{\pgfqpoint{3.036364in}{0.581222in}}%
\pgfpathlineto{\pgfqpoint{3.037832in}{0.581609in}}%
\pgfpathlineto{\pgfqpoint{3.040803in}{0.588935in}}%
\pgfpathlineto{\pgfqpoint{3.047754in}{0.612150in}}%
\pgfpathlineto{\pgfqpoint{3.049298in}{0.614132in}}%
\pgfpathlineto{\pgfqpoint{3.051015in}{0.614338in}}%
\pgfpathlineto{\pgfqpoint{3.052946in}{0.612180in}}%
\pgfpathlineto{\pgfqpoint{3.057354in}{0.600213in}}%
\pgfpathlineto{\pgfqpoint{3.064664in}{0.578387in}}%
\pgfpathlineto{\pgfqpoint{3.066966in}{0.576279in}}%
\pgfpathlineto{\pgfqpoint{3.068955in}{0.577414in}}%
\pgfpathlineto{\pgfqpoint{3.072227in}{0.585133in}}%
\pgfpathlineto{\pgfqpoint{3.079266in}{0.608674in}}%
\pgfpathlineto{\pgfqpoint{3.080802in}{0.609749in}}%
\pgfpathlineto{\pgfqpoint{3.082271in}{0.608255in}}%
\pgfpathlineto{\pgfqpoint{3.089672in}{0.589746in}}%
\pgfpathlineto{\pgfqpoint{3.091207in}{0.591010in}}%
\pgfpathlineto{\pgfqpoint{3.094959in}{0.602479in}}%
\pgfpathlineto{\pgfqpoint{3.100289in}{0.620272in}}%
\pgfpathlineto{\pgfqpoint{3.103488in}{0.623251in}}%
\pgfpathlineto{\pgfqpoint{3.105367in}{0.621655in}}%
\pgfpathlineto{\pgfqpoint{3.109685in}{0.610757in}}%
\pgfpathlineto{\pgfqpoint{3.117001in}{0.588240in}}%
\pgfpathlineto{\pgfqpoint{3.119583in}{0.585401in}}%
\pgfpathlineto{\pgfqpoint{3.121587in}{0.586353in}}%
\pgfpathlineto{\pgfqpoint{3.125073in}{0.594539in}}%
\pgfpathlineto{\pgfqpoint{3.132223in}{0.618130in}}%
\pgfpathlineto{\pgfqpoint{3.133759in}{0.618899in}}%
\pgfpathlineto{\pgfqpoint{3.135244in}{0.617101in}}%
\pgfpathlineto{\pgfqpoint{3.142893in}{0.598902in}}%
\pgfpathlineto{\pgfqpoint{3.144460in}{0.600850in}}%
\pgfpathlineto{\pgfqpoint{3.147875in}{0.611820in}}%
\pgfpathlineto{\pgfqpoint{3.153266in}{0.629681in}}%
\pgfpathlineto{\pgfqpoint{3.154808in}{0.631919in}}%
\pgfpathlineto{\pgfqpoint{3.156504in}{0.632408in}}%
\pgfpathlineto{\pgfqpoint{3.158395in}{0.630603in}}%
\pgfpathlineto{\pgfqpoint{3.162714in}{0.619386in}}%
\pgfpathlineto{\pgfqpoint{3.169945in}{0.597428in}}%
\pgfpathlineto{\pgfqpoint{3.172489in}{0.594891in}}%
\pgfpathlineto{\pgfqpoint{3.174469in}{0.596026in}}%
\pgfpathlineto{\pgfqpoint{3.177940in}{0.604445in}}%
\pgfpathlineto{\pgfqpoint{3.184854in}{0.627542in}}%
\pgfpathlineto{\pgfqpoint{3.186411in}{0.628563in}}%
\pgfpathlineto{\pgfqpoint{3.187909in}{0.626966in}}%
\pgfpathlineto{\pgfqpoint{3.195506in}{0.608363in}}%
\pgfpathlineto{\pgfqpoint{3.197069in}{0.609907in}}%
\pgfpathlineto{\pgfqpoint{3.200739in}{0.621386in}}%
\pgfpathlineto{\pgfqpoint{3.206173in}{0.639377in}}%
\pgfpathlineto{\pgfqpoint{3.207729in}{0.641591in}}%
\pgfpathlineto{\pgfqpoint{3.209435in}{0.642012in}}%
\pgfpathlineto{\pgfqpoint{3.211330in}{0.640115in}}%
\pgfpathlineto{\pgfqpoint{3.215634in}{0.628806in}}%
\pgfpathlineto{\pgfqpoint{3.222788in}{0.607239in}}%
\pgfpathlineto{\pgfqpoint{3.225302in}{0.604835in}}%
\pgfpathlineto{\pgfqpoint{3.227271in}{0.606035in}}%
\pgfpathlineto{\pgfqpoint{3.230748in}{0.614586in}}%
\pgfpathlineto{\pgfqpoint{3.237772in}{0.637796in}}%
\pgfpathlineto{\pgfqpoint{3.239332in}{0.638578in}}%
\pgfpathlineto{\pgfqpoint{3.240846in}{0.636737in}}%
\pgfpathlineto{\pgfqpoint{3.248662in}{0.618410in}}%
\pgfpathlineto{\pgfqpoint{3.250246in}{0.620508in}}%
\pgfpathlineto{\pgfqpoint{3.254057in}{0.633179in}}%
\pgfpathlineto{\pgfqpoint{3.259579in}{0.650394in}}%
\pgfpathlineto{\pgfqpoint{3.261192in}{0.652039in}}%
\pgfpathlineto{\pgfqpoint{3.262964in}{0.651681in}}%
\pgfpathlineto{\pgfqpoint{3.264923in}{0.648848in}}%
\pgfpathlineto{\pgfqpoint{3.269293in}{0.636143in}}%
\pgfpathlineto{\pgfqpoint{3.273969in}{0.621089in}}%
\pgfpathlineto{\pgfqpoint{3.276541in}{0.616188in}}%
\pgfpathlineto{\pgfqpoint{3.278693in}{0.615303in}}%
\pgfpathlineto{\pgfqpoint{3.280584in}{0.617271in}}%
\pgfpathlineto{\pgfqpoint{3.283974in}{0.626647in}}%
\pgfpathlineto{\pgfqpoint{3.289465in}{0.646246in}}%
\pgfpathlineto{\pgfqpoint{3.291085in}{0.648864in}}%
\pgfpathlineto{\pgfqpoint{3.292629in}{0.648734in}}%
\pgfpathlineto{\pgfqpoint{3.295747in}{0.641302in}}%
\pgfpathlineto{\pgfqpoint{3.300298in}{0.629066in}}%
\pgfpathlineto{\pgfqpoint{3.301776in}{0.628977in}}%
\pgfpathlineto{\pgfqpoint{3.304893in}{0.635983in}}%
\pgfpathlineto{\pgfqpoint{3.311927in}{0.659967in}}%
\pgfpathlineto{\pgfqpoint{3.313507in}{0.662195in}}%
\pgfpathlineto{\pgfqpoint{3.315225in}{0.662561in}}%
\pgfpathlineto{\pgfqpoint{3.317118in}{0.660584in}}%
\pgfpathlineto{\pgfqpoint{3.321382in}{0.649275in}}%
\pgfpathlineto{\pgfqpoint{3.328394in}{0.628332in}}%
\pgfpathlineto{\pgfqpoint{3.330855in}{0.626057in}}%
\pgfpathlineto{\pgfqpoint{3.332815in}{0.627302in}}%
\pgfpathlineto{\pgfqpoint{3.336319in}{0.636047in}}%
\pgfpathlineto{\pgfqpoint{3.342716in}{0.658221in}}%
\pgfpathlineto{\pgfqpoint{3.344331in}{0.660053in}}%
\pgfpathlineto{\pgfqpoint{3.345872in}{0.659133in}}%
\pgfpathlineto{\pgfqpoint{3.348828in}{0.651286in}}%
\pgfpathlineto{\pgfqpoint{3.352216in}{0.641196in}}%
\pgfpathlineto{\pgfqpoint{3.353587in}{0.639629in}}%
\pgfpathlineto{\pgfqpoint{3.355131in}{0.640358in}}%
\pgfpathlineto{\pgfqpoint{3.358678in}{0.650210in}}%
\pgfpathlineto{\pgfqpoint{3.365036in}{0.671376in}}%
\pgfpathlineto{\pgfqpoint{3.366646in}{0.673356in}}%
\pgfpathlineto{\pgfqpoint{3.368394in}{0.673366in}}%
\pgfpathlineto{\pgfqpoint{3.370314in}{0.670963in}}%
\pgfpathlineto{\pgfqpoint{3.374593in}{0.659047in}}%
\pgfpathlineto{\pgfqpoint{3.381539in}{0.639013in}}%
\pgfpathlineto{\pgfqpoint{3.383948in}{0.637363in}}%
\pgfpathlineto{\pgfqpoint{3.385870in}{0.639029in}}%
\pgfpathlineto{\pgfqpoint{3.389529in}{0.648971in}}%
\pgfpathlineto{\pgfqpoint{3.395441in}{0.669517in}}%
\pgfpathlineto{\pgfqpoint{3.397068in}{0.671442in}}%
\pgfpathlineto{\pgfqpoint{3.398623in}{0.670589in}}%
\pgfpathlineto{\pgfqpoint{3.401611in}{0.662746in}}%
\pgfpathlineto{\pgfqpoint{3.405029in}{0.652551in}}%
\pgfpathlineto{\pgfqpoint{3.406413in}{0.650946in}}%
\pgfpathlineto{\pgfqpoint{3.407963in}{0.651649in}}%
\pgfpathlineto{\pgfqpoint{3.411094in}{0.659899in}}%
\pgfpathlineto{\pgfqpoint{3.417774in}{0.682550in}}%
\pgfpathlineto{\pgfqpoint{3.419386in}{0.684688in}}%
\pgfpathlineto{\pgfqpoint{3.421127in}{0.684864in}}%
\pgfpathlineto{\pgfqpoint{3.423030in}{0.682649in}}%
\pgfpathlineto{\pgfqpoint{3.427268in}{0.671097in}}%
\pgfpathlineto{\pgfqpoint{3.434107in}{0.651070in}}%
\pgfpathlineto{\pgfqpoint{3.436508in}{0.649099in}}%
\pgfpathlineto{\pgfqpoint{3.438451in}{0.650520in}}%
\pgfpathlineto{\pgfqpoint{3.442156in}{0.660277in}}%
\pgfpathlineto{\pgfqpoint{3.448889in}{0.682506in}}%
\pgfpathlineto{\pgfqpoint{3.450492in}{0.683282in}}%
\pgfpathlineto{\pgfqpoint{3.452059in}{0.681346in}}%
\pgfpathlineto{\pgfqpoint{3.458712in}{0.663106in}}%
\pgfpathlineto{\pgfqpoint{3.460210in}{0.662841in}}%
\pgfpathlineto{\pgfqpoint{3.461828in}{0.665249in}}%
\pgfpathlineto{\pgfqpoint{3.466134in}{0.680190in}}%
\pgfpathlineto{\pgfqpoint{3.470351in}{0.693848in}}%
\pgfpathlineto{\pgfqpoint{3.471954in}{0.696322in}}%
\pgfpathlineto{\pgfqpoint{3.473676in}{0.696886in}}%
\pgfpathlineto{\pgfqpoint{3.475548in}{0.695114in}}%
\pgfpathlineto{\pgfqpoint{3.479721in}{0.684346in}}%
\pgfpathlineto{\pgfqpoint{3.486571in}{0.663805in}}%
\pgfpathlineto{\pgfqpoint{3.488970in}{0.661351in}}%
\pgfpathlineto{\pgfqpoint{3.490943in}{0.662387in}}%
\pgfpathlineto{\pgfqpoint{3.494708in}{0.671793in}}%
\pgfpathlineto{\pgfqpoint{3.502079in}{0.695307in}}%
\pgfpathlineto{\pgfqpoint{3.503674in}{0.695377in}}%
\pgfpathlineto{\pgfqpoint{3.506877in}{0.688068in}}%
\pgfpathlineto{\pgfqpoint{3.511711in}{0.675212in}}%
\pgfpathlineto{\pgfqpoint{3.513239in}{0.675231in}}%
\pgfpathlineto{\pgfqpoint{3.516436in}{0.682644in}}%
\pgfpathlineto{\pgfqpoint{3.523852in}{0.707405in}}%
\pgfpathlineto{\pgfqpoint{3.525512in}{0.709176in}}%
\pgfpathlineto{\pgfqpoint{3.527296in}{0.708824in}}%
\pgfpathlineto{\pgfqpoint{3.529229in}{0.706005in}}%
\pgfpathlineto{\pgfqpoint{3.533465in}{0.693738in}}%
\pgfpathlineto{\pgfqpoint{3.537981in}{0.679440in}}%
\pgfpathlineto{\pgfqpoint{3.540420in}{0.674922in}}%
\pgfpathlineto{\pgfqpoint{3.542722in}{0.674216in}}%
\pgfpathlineto{\pgfqpoint{3.544603in}{0.676521in}}%
\pgfpathlineto{\pgfqpoint{3.548219in}{0.687288in}}%
\pgfpathlineto{\pgfqpoint{3.554400in}{0.707595in}}%
\pgfpathlineto{\pgfqpoint{3.556025in}{0.708453in}}%
\pgfpathlineto{\pgfqpoint{3.557615in}{0.706554in}}%
\pgfpathlineto{\pgfqpoint{3.564435in}{0.688050in}}%
\pgfpathlineto{\pgfqpoint{3.565961in}{0.687868in}}%
\pgfpathlineto{\pgfqpoint{3.567594in}{0.690390in}}%
\pgfpathlineto{\pgfqpoint{3.572019in}{0.705914in}}%
\pgfpathlineto{\pgfqpoint{3.576353in}{0.719584in}}%
\pgfpathlineto{\pgfqpoint{3.577999in}{0.721798in}}%
\pgfpathlineto{\pgfqpoint{3.579759in}{0.721953in}}%
\pgfpathlineto{\pgfqpoint{3.581657in}{0.719690in}}%
\pgfpathlineto{\pgfqpoint{3.585825in}{0.708287in}}%
\pgfpathlineto{\pgfqpoint{3.592495in}{0.689031in}}%
\pgfpathlineto{\pgfqpoint{3.594822in}{0.687215in}}%
\pgfpathlineto{\pgfqpoint{3.596971in}{0.689024in}}%
\pgfpathlineto{\pgfqpoint{3.600688in}{0.699422in}}%
\pgfpathlineto{\pgfqpoint{3.606012in}{0.718635in}}%
\pgfpathlineto{\pgfqpoint{3.607692in}{0.721473in}}%
\pgfpathlineto{\pgfqpoint{3.609305in}{0.721450in}}%
\pgfpathlineto{\pgfqpoint{3.612550in}{0.713946in}}%
\pgfpathlineto{\pgfqpoint{3.617467in}{0.701111in}}%
\pgfpathlineto{\pgfqpoint{3.619024in}{0.701273in}}%
\pgfpathlineto{\pgfqpoint{3.622256in}{0.709016in}}%
\pgfpathlineto{\pgfqpoint{3.629299in}{0.733067in}}%
\pgfpathlineto{\pgfqpoint{3.630963in}{0.735206in}}%
\pgfpathlineto{\pgfqpoint{3.632737in}{0.735228in}}%
\pgfpathlineto{\pgfqpoint{3.634640in}{0.732810in}}%
\pgfpathlineto{\pgfqpoint{3.638796in}{0.721259in}}%
\pgfpathlineto{\pgfqpoint{3.645485in}{0.702374in}}%
\pgfpathlineto{\pgfqpoint{3.647777in}{0.700919in}}%
\pgfpathlineto{\pgfqpoint{3.649895in}{0.702990in}}%
\pgfpathlineto{\pgfqpoint{3.653598in}{0.713703in}}%
\pgfpathlineto{\pgfqpoint{3.658577in}{0.731892in}}%
\pgfpathlineto{\pgfqpoint{3.660269in}{0.735075in}}%
\pgfpathlineto{\pgfqpoint{3.661899in}{0.735400in}}%
\pgfpathlineto{\pgfqpoint{3.663517in}{0.732984in}}%
\pgfpathlineto{\pgfqpoint{3.670293in}{0.714843in}}%
\pgfpathlineto{\pgfqpoint{3.671860in}{0.714985in}}%
\pgfpathlineto{\pgfqpoint{3.675111in}{0.722749in}}%
\pgfpathlineto{\pgfqpoint{3.682686in}{0.747786in}}%
\pgfpathlineto{\pgfqpoint{3.684394in}{0.749305in}}%
\pgfpathlineto{\pgfqpoint{3.686213in}{0.748552in}}%
\pgfpathlineto{\pgfqpoint{3.690209in}{0.739790in}}%
\pgfpathlineto{\pgfqpoint{3.696842in}{0.719148in}}%
\pgfpathlineto{\pgfqpoint{3.699187in}{0.715512in}}%
\pgfpathlineto{\pgfqpoint{3.701397in}{0.715541in}}%
\pgfpathlineto{\pgfqpoint{3.703431in}{0.718759in}}%
\pgfpathlineto{\pgfqpoint{3.708651in}{0.736969in}}%
\pgfpathlineto{\pgfqpoint{3.712471in}{0.748590in}}%
\pgfpathlineto{\pgfqpoint{3.714134in}{0.749910in}}%
\pgfpathlineto{\pgfqpoint{3.715755in}{0.748394in}}%
\pgfpathlineto{\pgfqpoint{3.724266in}{0.728905in}}%
\pgfpathlineto{\pgfqpoint{3.725915in}{0.731064in}}%
\pgfpathlineto{\pgfqpoint{3.730544in}{0.747002in}}%
\pgfpathlineto{\pgfqpoint{3.735001in}{0.761170in}}%
\pgfpathlineto{\pgfqpoint{3.736688in}{0.763383in}}%
\pgfpathlineto{\pgfqpoint{3.738476in}{0.763410in}}%
\pgfpathlineto{\pgfqpoint{3.740375in}{0.760980in}}%
\pgfpathlineto{\pgfqpoint{3.744482in}{0.749571in}}%
\pgfpathlineto{\pgfqpoint{3.751075in}{0.731153in}}%
\pgfpathlineto{\pgfqpoint{3.753321in}{0.729793in}}%
\pgfpathlineto{\pgfqpoint{3.755414in}{0.731875in}}%
\pgfpathlineto{\pgfqpoint{3.759145in}{0.742752in}}%
\pgfpathlineto{\pgfqpoint{3.765259in}{0.763412in}}%
\pgfpathlineto{\pgfqpoint{3.766929in}{0.764719in}}%
\pgfpathlineto{\pgfqpoint{3.768560in}{0.763179in}}%
\pgfpathlineto{\pgfqpoint{3.777169in}{0.743673in}}%
\pgfpathlineto{\pgfqpoint{3.778827in}{0.745939in}}%
\pgfpathlineto{\pgfqpoint{3.782781in}{0.759284in}}%
\pgfpathlineto{\pgfqpoint{3.787178in}{0.774522in}}%
\pgfpathlineto{\pgfqpoint{3.790592in}{0.778546in}}%
\pgfpathlineto{\pgfqpoint{3.792447in}{0.777140in}}%
\pgfpathlineto{\pgfqpoint{3.796460in}{0.767335in}}%
\pgfpathlineto{\pgfqpoint{3.802979in}{0.747775in}}%
\pgfpathlineto{\pgfqpoint{3.805250in}{0.745080in}}%
\pgfpathlineto{\pgfqpoint{3.807390in}{0.745920in}}%
\pgfpathlineto{\pgfqpoint{3.811235in}{0.755518in}}%
\pgfpathlineto{\pgfqpoint{3.818621in}{0.779517in}}%
\pgfpathlineto{\pgfqpoint{3.820276in}{0.779825in}}%
\pgfpathlineto{\pgfqpoint{3.821924in}{0.777359in}}%
\pgfpathlineto{\pgfqpoint{3.828918in}{0.758992in}}%
\pgfpathlineto{\pgfqpoint{3.830525in}{0.759335in}}%
\pgfpathlineto{\pgfqpoint{3.833825in}{0.767550in}}%
\pgfpathlineto{\pgfqpoint{3.841746in}{0.792980in}}%
\pgfpathlineto{\pgfqpoint{3.843514in}{0.793920in}}%
\pgfpathlineto{\pgfqpoint{3.845375in}{0.792431in}}%
\pgfpathlineto{\pgfqpoint{3.849375in}{0.782547in}}%
\pgfpathlineto{\pgfqpoint{3.855826in}{0.763343in}}%
\pgfpathlineto{\pgfqpoint{3.858071in}{0.760782in}}%
\pgfpathlineto{\pgfqpoint{3.860192in}{0.761701in}}%
\pgfpathlineto{\pgfqpoint{3.864039in}{0.771438in}}%
\pgfpathlineto{\pgfqpoint{3.870137in}{0.793264in}}%
\pgfpathlineto{\pgfqpoint{3.871848in}{0.795731in}}%
\pgfpathlineto{\pgfqpoint{3.873500in}{0.795278in}}%
\pgfpathlineto{\pgfqpoint{3.876840in}{0.787079in}}%
\pgfpathlineto{\pgfqpoint{3.880683in}{0.776039in}}%
\pgfpathlineto{\pgfqpoint{3.882213in}{0.774596in}}%
\pgfpathlineto{\pgfqpoint{3.883859in}{0.775754in}}%
\pgfpathlineto{\pgfqpoint{3.887641in}{0.786979in}}%
\pgfpathlineto{\pgfqpoint{3.894149in}{0.808244in}}%
\pgfpathlineto{\pgfqpoint{3.895905in}{0.809808in}}%
\pgfpathlineto{\pgfqpoint{3.897746in}{0.808969in}}%
\pgfpathlineto{\pgfqpoint{3.901686in}{0.800189in}}%
\pgfpathlineto{\pgfqpoint{3.908106in}{0.780568in}}%
\pgfpathlineto{\pgfqpoint{3.910349in}{0.777296in}}%
\pgfpathlineto{\pgfqpoint{3.912483in}{0.777462in}}%
\pgfpathlineto{\pgfqpoint{3.914493in}{0.780739in}}%
\pgfpathlineto{\pgfqpoint{3.919946in}{0.800069in}}%
\pgfpathlineto{\pgfqpoint{3.923200in}{0.810241in}}%
\pgfpathlineto{\pgfqpoint{3.924906in}{0.812233in}}%
\pgfpathlineto{\pgfqpoint{3.926563in}{0.811314in}}%
\pgfpathlineto{\pgfqpoint{3.929922in}{0.802530in}}%
\pgfpathlineto{\pgfqpoint{3.933974in}{0.791685in}}%
\pgfpathlineto{\pgfqpoint{3.935555in}{0.790968in}}%
\pgfpathlineto{\pgfqpoint{3.937230in}{0.793006in}}%
\pgfpathlineto{\pgfqpoint{3.941152in}{0.805929in}}%
\pgfpathlineto{\pgfqpoint{3.945352in}{0.821109in}}%
\pgfpathlineto{\pgfqpoint{3.948800in}{0.826257in}}%
\pgfpathlineto{\pgfqpoint{3.950648in}{0.825372in}}%
\pgfpathlineto{\pgfqpoint{3.954576in}{0.816548in}}%
\pgfpathlineto{\pgfqpoint{3.960942in}{0.797207in}}%
\pgfpathlineto{\pgfqpoint{3.963160in}{0.794052in}}%
\pgfpathlineto{\pgfqpoint{3.965275in}{0.794285in}}%
\pgfpathlineto{\pgfqpoint{3.967279in}{0.797608in}}%
\pgfpathlineto{\pgfqpoint{3.972494in}{0.816085in}}%
\pgfpathlineto{\pgfqpoint{3.975588in}{0.826314in}}%
\pgfpathlineto{\pgfqpoint{3.977314in}{0.828962in}}%
\pgfpathlineto{\pgfqpoint{3.978983in}{0.828678in}}%
\pgfpathlineto{\pgfqpoint{3.982359in}{0.820663in}}%
\pgfpathlineto{\pgfqpoint{3.986280in}{0.809342in}}%
\pgfpathlineto{\pgfqpoint{3.987834in}{0.807784in}}%
\pgfpathlineto{\pgfqpoint{3.989494in}{0.808846in}}%
\pgfpathlineto{\pgfqpoint{3.992832in}{0.818195in}}%
\pgfpathlineto{\pgfqpoint{3.999578in}{0.841149in}}%
\pgfpathlineto{\pgfqpoint{4.001347in}{0.843137in}}%
\pgfpathlineto{\pgfqpoint{4.003188in}{0.842692in}}%
\pgfpathlineto{\pgfqpoint{4.005099in}{0.839745in}}%
\pgfpathlineto{\pgfqpoint{4.015575in}{0.811664in}}%
\pgfpathlineto{\pgfqpoint{4.017692in}{0.811353in}}%
\pgfpathlineto{\pgfqpoint{4.019708in}{0.814207in}}%
\pgfpathlineto{\pgfqpoint{4.024501in}{0.830455in}}%
\pgfpathlineto{\pgfqpoint{4.027938in}{0.842706in}}%
\pgfpathlineto{\pgfqpoint{4.029680in}{0.846049in}}%
\pgfpathlineto{\pgfqpoint{4.031364in}{0.846473in}}%
\pgfpathlineto{\pgfqpoint{4.033046in}{0.844087in}}%
\pgfpathlineto{\pgfqpoint{4.040405in}{0.825288in}}%
\pgfpathlineto{\pgfqpoint{4.042060in}{0.825879in}}%
\pgfpathlineto{\pgfqpoint{4.045422in}{0.834632in}}%
\pgfpathlineto{\pgfqpoint{4.053183in}{0.859797in}}%
\pgfpathlineto{\pgfqpoint{4.054996in}{0.860793in}}%
\pgfpathlineto{\pgfqpoint{4.056871in}{0.859251in}}%
\pgfpathlineto{\pgfqpoint{4.060800in}{0.849500in}}%
\pgfpathlineto{\pgfqpoint{4.067020in}{0.831352in}}%
\pgfpathlineto{\pgfqpoint{4.069173in}{0.829026in}}%
\pgfpathlineto{\pgfqpoint{4.071238in}{0.829993in}}%
\pgfpathlineto{\pgfqpoint{4.075120in}{0.839994in}}%
\pgfpathlineto{\pgfqpoint{4.081788in}{0.863096in}}%
\pgfpathlineto{\pgfqpoint{4.083500in}{0.864669in}}%
\pgfpathlineto{\pgfqpoint{4.085181in}{0.863343in}}%
\pgfpathlineto{\pgfqpoint{4.088596in}{0.854055in}}%
\pgfpathlineto{\pgfqpoint{4.092923in}{0.843461in}}%
\pgfpathlineto{\pgfqpoint{4.094571in}{0.843498in}}%
\pgfpathlineto{\pgfqpoint{4.096275in}{0.846357in}}%
\pgfpathlineto{\pgfqpoint{4.101725in}{0.866248in}}%
\pgfpathlineto{\pgfqpoint{4.104916in}{0.875988in}}%
\pgfpathlineto{\pgfqpoint{4.106695in}{0.878530in}}%
\pgfpathlineto{\pgfqpoint{4.108541in}{0.878617in}}%
\pgfpathlineto{\pgfqpoint{4.110439in}{0.876153in}}%
\pgfpathlineto{\pgfqpoint{4.114386in}{0.865243in}}%
\pgfpathlineto{\pgfqpoint{4.118495in}{0.852569in}}%
\pgfpathlineto{\pgfqpoint{4.120670in}{0.848452in}}%
\pgfpathlineto{\pgfqpoint{4.122770in}{0.847468in}}%
\pgfpathlineto{\pgfqpoint{4.124793in}{0.849684in}}%
\pgfpathlineto{\pgfqpoint{4.128627in}{0.861303in}}%
\pgfpathlineto{\pgfqpoint{4.133702in}{0.879824in}}%
\pgfpathlineto{\pgfqpoint{4.135448in}{0.882839in}}%
\pgfpathlineto{\pgfqpoint{4.137140in}{0.882920in}}%
\pgfpathlineto{\pgfqpoint{4.138842in}{0.880217in}}%
\pgfpathlineto{\pgfqpoint{4.146213in}{0.861720in}}%
\pgfpathlineto{\pgfqpoint{4.147894in}{0.862550in}}%
\pgfpathlineto{\pgfqpoint{4.151283in}{0.871720in}}%
\pgfpathlineto{\pgfqpoint{4.158771in}{0.896344in}}%
\pgfpathlineto{\pgfqpoint{4.160603in}{0.897554in}}%
\pgfpathlineto{\pgfqpoint{4.162484in}{0.896167in}}%
\pgfpathlineto{\pgfqpoint{4.166376in}{0.886726in}}%
\pgfpathlineto{\pgfqpoint{4.172511in}{0.868866in}}%
\pgfpathlineto{\pgfqpoint{4.174626in}{0.866499in}}%
\pgfpathlineto{\pgfqpoint{4.176671in}{0.867346in}}%
\pgfpathlineto{\pgfqpoint{4.180576in}{0.877287in}}%
\pgfpathlineto{\pgfqpoint{4.186910in}{0.899951in}}%
\pgfpathlineto{\pgfqpoint{4.188645in}{0.902265in}}%
\pgfpathlineto{\pgfqpoint{4.190340in}{0.901649in}}%
\pgfpathlineto{\pgfqpoint{4.193790in}{0.893147in}}%
\pgfpathlineto{\pgfqpoint{4.198021in}{0.881709in}}%
\pgfpathlineto{\pgfqpoint{4.199659in}{0.880783in}}%
\pgfpathlineto{\pgfqpoint{4.201369in}{0.882670in}}%
\pgfpathlineto{\pgfqpoint{4.205607in}{0.896631in}}%
\pgfpathlineto{\pgfqpoint{4.209977in}{0.912273in}}%
\pgfpathlineto{\pgfqpoint{4.211756in}{0.915722in}}%
\pgfpathlineto{\pgfqpoint{4.213604in}{0.916768in}}%
\pgfpathlineto{\pgfqpoint{4.215493in}{0.915186in}}%
\pgfpathlineto{\pgfqpoint{4.219377in}{0.905520in}}%
\pgfpathlineto{\pgfqpoint{4.225448in}{0.888137in}}%
\pgfpathlineto{\pgfqpoint{4.227540in}{0.886030in}}%
\pgfpathlineto{\pgfqpoint{4.229569in}{0.887103in}}%
\pgfpathlineto{\pgfqpoint{4.233476in}{0.897416in}}%
\pgfpathlineto{\pgfqpoint{4.239862in}{0.920016in}}%
\pgfpathlineto{\pgfqpoint{4.241594in}{0.922043in}}%
\pgfpathlineto{\pgfqpoint{4.243296in}{0.921152in}}%
\pgfpathlineto{\pgfqpoint{4.246766in}{0.912300in}}%
\pgfpathlineto{\pgfqpoint{4.251143in}{0.901030in}}%
\pgfpathlineto{\pgfqpoint{4.252810in}{0.900581in}}%
\pgfpathlineto{\pgfqpoint{4.254532in}{0.902992in}}%
\pgfpathlineto{\pgfqpoint{4.258681in}{0.917223in}}%
\pgfpathlineto{\pgfqpoint{4.263510in}{0.933621in}}%
\pgfpathlineto{\pgfqpoint{4.265335in}{0.936241in}}%
\pgfpathlineto{\pgfqpoint{4.267211in}{0.936259in}}%
\pgfpathlineto{\pgfqpoint{4.269115in}{0.933681in}}%
\pgfpathlineto{\pgfqpoint{4.274977in}{0.916457in}}%
\pgfpathlineto{\pgfqpoint{4.279201in}{0.906914in}}%
\pgfpathlineto{\pgfqpoint{4.281248in}{0.906293in}}%
\pgfpathlineto{\pgfqpoint{4.283248in}{0.908798in}}%
\pgfpathlineto{\pgfqpoint{4.287666in}{0.923214in}}%
\pgfpathlineto{\pgfqpoint{4.291554in}{0.937666in}}%
\pgfpathlineto{\pgfqpoint{4.293322in}{0.941442in}}%
\pgfpathlineto{\pgfqpoint{4.295038in}{0.942330in}}%
\pgfpathlineto{\pgfqpoint{4.296755in}{0.940358in}}%
\pgfpathlineto{\pgfqpoint{4.304688in}{0.920756in}}%
\pgfpathlineto{\pgfqpoint{4.306397in}{0.921530in}}%
\pgfpathlineto{\pgfqpoint{4.309829in}{0.930735in}}%
\pgfpathlineto{\pgfqpoint{4.317469in}{0.955877in}}%
\pgfpathlineto{\pgfqpoint{4.319340in}{0.956963in}}%
\pgfpathlineto{\pgfqpoint{4.321239in}{0.955357in}}%
\pgfpathlineto{\pgfqpoint{4.325098in}{0.945738in}}%
\pgfpathlineto{\pgfqpoint{4.331093in}{0.928835in}}%
\pgfpathlineto{\pgfqpoint{4.333149in}{0.926910in}}%
\pgfpathlineto{\pgfqpoint{4.335159in}{0.928109in}}%
\pgfpathlineto{\pgfqpoint{4.339082in}{0.938714in}}%
\pgfpathlineto{\pgfqpoint{4.345123in}{0.960552in}}%
\pgfpathlineto{\pgfqpoint{4.346870in}{0.963075in}}%
\pgfpathlineto{\pgfqpoint{4.348584in}{0.962678in}}%
\pgfpathlineto{\pgfqpoint{4.352085in}{0.954426in}}%
\pgfpathlineto{\pgfqpoint{4.356418in}{0.942630in}}%
\pgfpathlineto{\pgfqpoint{4.358090in}{0.941534in}}%
\pgfpathlineto{\pgfqpoint{4.359822in}{0.943284in}}%
\pgfpathlineto{\pgfqpoint{4.364053in}{0.957001in}}%
\pgfpathlineto{\pgfqpoint{4.369452in}{0.975394in}}%
\pgfpathlineto{\pgfqpoint{4.371315in}{0.977766in}}%
\pgfpathlineto{\pgfqpoint{4.373212in}{0.977416in}}%
\pgfpathlineto{\pgfqpoint{4.377046in}{0.969531in}}%
\pgfpathlineto{\pgfqpoint{4.383059in}{0.951671in}}%
\pgfpathlineto{\pgfqpoint{4.385123in}{0.948625in}}%
\pgfpathlineto{\pgfqpoint{4.387138in}{0.948605in}}%
\pgfpathlineto{\pgfqpoint{4.389126in}{0.951668in}}%
\pgfpathlineto{\pgfqpoint{4.394217in}{0.969611in}}%
\pgfpathlineto{\pgfqpoint{4.398191in}{0.982627in}}%
\pgfpathlineto{\pgfqpoint{4.399930in}{0.984678in}}%
\pgfpathlineto{\pgfqpoint{4.401652in}{0.983823in}}%
\pgfpathlineto{\pgfqpoint{4.405177in}{0.974979in}}%
\pgfpathlineto{\pgfqpoint{4.409718in}{0.963501in}}%
\pgfpathlineto{\pgfqpoint{4.411424in}{0.963176in}}%
\pgfpathlineto{\pgfqpoint{4.413168in}{0.965751in}}%
\pgfpathlineto{\pgfqpoint{4.417552in}{0.981113in}}%
\pgfpathlineto{\pgfqpoint{4.422624in}{0.997535in}}%
\pgfpathlineto{\pgfqpoint{4.424509in}{0.999483in}}%
\pgfpathlineto{\pgfqpoint{4.426416in}{0.998653in}}%
\pgfpathlineto{\pgfqpoint{4.430249in}{0.990094in}}%
\pgfpathlineto{\pgfqpoint{4.436226in}{0.972797in}}%
\pgfpathlineto{\pgfqpoint{4.438265in}{0.970317in}}%
\pgfpathlineto{\pgfqpoint{4.440265in}{0.970881in}}%
\pgfpathlineto{\pgfqpoint{4.442248in}{0.974491in}}%
\pgfpathlineto{\pgfqpoint{4.448093in}{0.996022in}}%
\pgfpathlineto{\pgfqpoint{4.451603in}{1.005770in}}%
\pgfpathlineto{\pgfqpoint{4.453331in}{1.006775in}}%
\pgfpathlineto{\pgfqpoint{4.455069in}{1.004917in}}%
\pgfpathlineto{\pgfqpoint{4.463489in}{0.985016in}}%
\pgfpathlineto{\pgfqpoint{4.465233in}{0.986299in}}%
\pgfpathlineto{\pgfqpoint{4.468700in}{0.996305in}}%
\pgfpathlineto{\pgfqpoint{4.474422in}{1.017404in}}%
\pgfpathlineto{\pgfqpoint{4.476295in}{1.020887in}}%
\pgfpathlineto{\pgfqpoint{4.478205in}{1.021646in}}%
\pgfpathlineto{\pgfqpoint{4.480118in}{1.019652in}}%
\pgfpathlineto{\pgfqpoint{4.483951in}{1.009665in}}%
\pgfpathlineto{\pgfqpoint{4.489970in}{0.993786in}}%
\pgfpathlineto{\pgfqpoint{4.491977in}{0.992735in}}%
\pgfpathlineto{\pgfqpoint{4.493963in}{0.994774in}}%
\pgfpathlineto{\pgfqpoint{4.497902in}{1.006735in}}%
\pgfpathlineto{\pgfqpoint{4.504363in}{1.028416in}}%
\pgfpathlineto{\pgfqpoint{4.506095in}{1.029462in}}%
\pgfpathlineto{\pgfqpoint{4.507840in}{1.027643in}}%
\pgfpathlineto{\pgfqpoint{4.516336in}{1.007673in}}%
\pgfpathlineto{\pgfqpoint{4.518089in}{1.008977in}}%
\pgfpathlineto{\pgfqpoint{4.521569in}{1.019045in}}%
\pgfpathlineto{\pgfqpoint{4.528711in}{1.043080in}}%
\pgfpathlineto{\pgfqpoint{4.530628in}{1.044524in}}%
\pgfpathlineto{\pgfqpoint{4.532547in}{1.043129in}}%
\pgfpathlineto{\pgfqpoint{4.536372in}{1.033867in}}%
\pgfpathlineto{\pgfqpoint{4.542298in}{1.017550in}}%
\pgfpathlineto{\pgfqpoint{4.544302in}{1.015857in}}%
\pgfpathlineto{\pgfqpoint{4.546286in}{1.017220in}}%
\pgfpathlineto{\pgfqpoint{4.550240in}{1.028289in}}%
\pgfpathlineto{\pgfqpoint{4.557108in}{1.051655in}}%
\pgfpathlineto{\pgfqpoint{4.558844in}{1.052765in}}%
\pgfpathlineto{\pgfqpoint{4.560596in}{1.051009in}}%
\pgfpathlineto{\pgfqpoint{4.569169in}{1.030950in}}%
\pgfpathlineto{\pgfqpoint{4.570930in}{1.032249in}}%
\pgfpathlineto{\pgfqpoint{4.574424in}{1.042347in}}%
\pgfpathlineto{\pgfqpoint{4.581557in}{1.066460in}}%
\pgfpathlineto{\pgfqpoint{4.583485in}{1.067942in}}%
\pgfpathlineto{\pgfqpoint{4.585410in}{1.066552in}}%
\pgfpathlineto{\pgfqpoint{4.589232in}{1.057305in}}%
\pgfpathlineto{\pgfqpoint{4.595152in}{1.041183in}}%
\pgfpathlineto{\pgfqpoint{4.597146in}{1.039624in}}%
\pgfpathlineto{\pgfqpoint{4.599125in}{1.041119in}}%
\pgfpathlineto{\pgfqpoint{4.603086in}{1.052434in}}%
\pgfpathlineto{\pgfqpoint{4.609030in}{1.073984in}}%
\pgfpathlineto{\pgfqpoint{4.610783in}{1.076500in}}%
\pgfpathlineto{\pgfqpoint{4.612527in}{1.076134in}}%
\pgfpathlineto{\pgfqpoint{4.616113in}{1.067891in}}%
\pgfpathlineto{\pgfqpoint{4.620747in}{1.055735in}}%
\pgfpathlineto{\pgfqpoint{4.622491in}{1.054909in}}%
\pgfpathlineto{\pgfqpoint{4.624265in}{1.057027in}}%
\pgfpathlineto{\pgfqpoint{4.628705in}{1.072036in}}%
\pgfpathlineto{\pgfqpoint{4.633214in}{1.088088in}}%
\pgfpathlineto{\pgfqpoint{4.635144in}{1.091399in}}%
\pgfpathlineto{\pgfqpoint{4.637082in}{1.091784in}}%
\pgfpathlineto{\pgfqpoint{4.639005in}{1.089394in}}%
\pgfpathlineto{\pgfqpoint{4.644750in}{1.073017in}}%
\pgfpathlineto{\pgfqpoint{4.648858in}{1.064479in}}%
\pgfpathlineto{\pgfqpoint{4.650836in}{1.064341in}}%
\pgfpathlineto{\pgfqpoint{4.652814in}{1.067286in}}%
\pgfpathlineto{\pgfqpoint{4.657904in}{1.085247in}}%
\pgfpathlineto{\pgfqpoint{4.661968in}{1.098861in}}%
\pgfpathlineto{\pgfqpoint{4.663718in}{1.101125in}}%
\pgfpathlineto{\pgfqpoint{4.665471in}{1.100514in}}%
\pgfpathlineto{\pgfqpoint{4.669080in}{1.091941in}}%
\pgfpathlineto{\pgfqpoint{4.673854in}{1.079954in}}%
\pgfpathlineto{\pgfqpoint{4.675616in}{1.079583in}}%
\pgfpathlineto{\pgfqpoint{4.677399in}{1.082173in}}%
\pgfpathlineto{\pgfqpoint{4.681585in}{1.096711in}}%
\pgfpathlineto{\pgfqpoint{4.686081in}{1.112749in}}%
\pgfpathlineto{\pgfqpoint{4.688027in}{1.116089in}}%
\pgfpathlineto{\pgfqpoint{4.689972in}{1.116450in}}%
\pgfpathlineto{\pgfqpoint{4.691897in}{1.114020in}}%
\pgfpathlineto{\pgfqpoint{4.697644in}{1.097662in}}%
\pgfpathlineto{\pgfqpoint{4.701738in}{1.089418in}}%
\pgfpathlineto{\pgfqpoint{4.703711in}{1.089463in}}%
\pgfpathlineto{\pgfqpoint{4.705687in}{1.092591in}}%
\pgfpathlineto{\pgfqpoint{4.710712in}{1.110539in}}%
\pgfpathlineto{\pgfqpoint{4.715189in}{1.124865in}}%
\pgfpathlineto{\pgfqpoint{4.716936in}{1.126421in}}%
\pgfpathlineto{\pgfqpoint{4.718704in}{1.125106in}}%
\pgfpathlineto{\pgfqpoint{4.722345in}{1.115684in}}%
\pgfpathlineto{\pgfqpoint{4.725748in}{1.106462in}}%
\pgfpathlineto{\pgfqpoint{4.727497in}{1.104592in}}%
\pgfpathlineto{\pgfqpoint{4.729284in}{1.105594in}}%
\pgfpathlineto{\pgfqpoint{4.732822in}{1.115397in}}%
\pgfpathlineto{\pgfqpoint{4.738826in}{1.137747in}}%
\pgfpathlineto{\pgfqpoint{4.740786in}{1.141296in}}%
\pgfpathlineto{\pgfqpoint{4.742740in}{1.141822in}}%
\pgfpathlineto{\pgfqpoint{4.744669in}{1.139514in}}%
\pgfpathlineto{\pgfqpoint{4.750416in}{1.123307in}}%
\pgfpathlineto{\pgfqpoint{4.754507in}{1.115086in}}%
\pgfpathlineto{\pgfqpoint{4.756476in}{1.115142in}}%
\pgfpathlineto{\pgfqpoint{4.758450in}{1.118286in}}%
\pgfpathlineto{\pgfqpoint{4.763448in}{1.136174in}}%
\pgfpathlineto{\pgfqpoint{4.767929in}{1.150609in}}%
\pgfpathlineto{\pgfqpoint{4.769680in}{1.152238in}}%
\pgfpathlineto{\pgfqpoint{4.771454in}{1.150996in}}%
\pgfpathlineto{\pgfqpoint{4.775111in}{1.141654in}}%
\pgfpathlineto{\pgfqpoint{4.778553in}{1.132325in}}%
\pgfpathlineto{\pgfqpoint{4.780315in}{1.130415in}}%
\pgfpathlineto{\pgfqpoint{4.782110in}{1.131389in}}%
\pgfpathlineto{\pgfqpoint{4.785662in}{1.141188in}}%
\pgfpathlineto{\pgfqpoint{4.791581in}{1.163406in}}%
\pgfpathlineto{\pgfqpoint{4.793555in}{1.167150in}}%
\pgfpathlineto{\pgfqpoint{4.795516in}{1.167830in}}%
\pgfpathlineto{\pgfqpoint{4.797449in}{1.165635in}}%
\pgfpathlineto{\pgfqpoint{4.801266in}{1.155540in}}%
\pgfpathlineto{\pgfqpoint{4.805297in}{1.144221in}}%
\pgfpathlineto{\pgfqpoint{4.807289in}{1.141391in}}%
\pgfpathlineto{\pgfqpoint{4.809254in}{1.141480in}}%
\pgfpathlineto{\pgfqpoint{4.811226in}{1.144659in}}%
\pgfpathlineto{\pgfqpoint{4.816167in}{1.162388in}}%
\pgfpathlineto{\pgfqpoint{4.820566in}{1.176814in}}%
\pgfpathlineto{\pgfqpoint{4.822320in}{1.178680in}}%
\pgfpathlineto{\pgfqpoint{4.824099in}{1.177679in}}%
\pgfpathlineto{\pgfqpoint{4.827767in}{1.168616in}}%
\pgfpathlineto{\pgfqpoint{4.831243in}{1.159053in}}%
\pgfpathlineto{\pgfqpoint{4.833014in}{1.156935in}}%
\pgfpathlineto{\pgfqpoint{4.834817in}{1.157682in}}%
\pgfpathlineto{\pgfqpoint{4.838387in}{1.167197in}}%
\pgfpathlineto{\pgfqpoint{4.844445in}{1.189997in}}%
\pgfpathlineto{\pgfqpoint{4.846433in}{1.193776in}}%
\pgfpathlineto{\pgfqpoint{4.848398in}{1.194444in}}%
\pgfpathlineto{\pgfqpoint{4.850333in}{1.192221in}}%
\pgfpathlineto{\pgfqpoint{4.856093in}{1.176157in}}%
\pgfpathlineto{\pgfqpoint{4.860169in}{1.168283in}}%
\pgfpathlineto{\pgfqpoint{4.862131in}{1.168561in}}%
\pgfpathlineto{\pgfqpoint{4.864102in}{1.171928in}}%
\pgfpathlineto{\pgfqpoint{4.869045in}{1.189913in}}%
\pgfpathlineto{\pgfqpoint{4.872416in}{1.201856in}}%
\pgfpathlineto{\pgfqpoint{4.874181in}{1.205171in}}%
\pgfpathlineto{\pgfqpoint{4.875949in}{1.205689in}}%
\pgfpathlineto{\pgfqpoint{4.877754in}{1.203369in}}%
\pgfpathlineto{\pgfqpoint{4.884899in}{1.184870in}}%
\pgfpathlineto{\pgfqpoint{4.886700in}{1.184036in}}%
\pgfpathlineto{\pgfqpoint{4.888514in}{1.186208in}}%
\pgfpathlineto{\pgfqpoint{4.892971in}{1.201313in}}%
\pgfpathlineto{\pgfqpoint{4.897536in}{1.217833in}}%
\pgfpathlineto{\pgfqpoint{4.899533in}{1.221296in}}%
\pgfpathlineto{\pgfqpoint{4.901498in}{1.221600in}}%
\pgfpathlineto{\pgfqpoint{4.903431in}{1.219065in}}%
\pgfpathlineto{\pgfqpoint{4.913251in}{1.195754in}}%
\pgfpathlineto{\pgfqpoint{4.915210in}{1.196539in}}%
\pgfpathlineto{\pgfqpoint{4.917182in}{1.200401in}}%
\pgfpathlineto{\pgfqpoint{4.928461in}{1.233586in}}%
\pgfpathlineto{\pgfqpoint{4.930268in}{1.231694in}}%
\pgfpathlineto{\pgfqpoint{4.939328in}{1.211776in}}%
\pgfpathlineto{\pgfqpoint{4.941150in}{1.213608in}}%
\pgfpathlineto{\pgfqpoint{4.945268in}{1.226831in}}%
\pgfpathlineto{\pgfqpoint{4.950445in}{1.245868in}}%
\pgfpathlineto{\pgfqpoint{4.952450in}{1.249287in}}%
\pgfpathlineto{\pgfqpoint{4.954416in}{1.249516in}}%
\pgfpathlineto{\pgfqpoint{4.956350in}{1.246910in}}%
\pgfpathlineto{\pgfqpoint{4.966170in}{1.224005in}}%
\pgfpathlineto{\pgfqpoint{4.968127in}{1.225044in}}%
\pgfpathlineto{\pgfqpoint{4.972091in}{1.235851in}}%
\pgfpathlineto{\pgfqpoint{4.978368in}{1.259039in}}%
\pgfpathlineto{\pgfqpoint{4.980132in}{1.261765in}}%
\pgfpathlineto{\pgfqpoint{4.981921in}{1.261666in}}%
\pgfpathlineto{\pgfqpoint{4.985616in}{1.253794in}}%
\pgfpathlineto{\pgfqpoint{4.990669in}{1.241005in}}%
\pgfpathlineto{\pgfqpoint{4.992493in}{1.240349in}}%
\pgfpathlineto{\pgfqpoint{4.994322in}{1.242722in}}%
\pgfpathlineto{\pgfqpoint{4.998827in}{1.258275in}}%
\pgfpathlineto{\pgfqpoint{5.003026in}{1.273737in}}%
\pgfpathlineto{\pgfqpoint{5.005042in}{1.277623in}}%
\pgfpathlineto{\pgfqpoint{5.007014in}{1.278290in}}%
\pgfpathlineto{\pgfqpoint{5.008952in}{1.276033in}}%
\pgfpathlineto{\pgfqpoint{5.018787in}{1.253028in}}%
\pgfpathlineto{\pgfqpoint{5.020742in}{1.253840in}}%
\pgfpathlineto{\pgfqpoint{5.022708in}{1.257734in}}%
\pgfpathlineto{\pgfqpoint{5.033898in}{1.291203in}}%
\pgfpathlineto{\pgfqpoint{5.035717in}{1.289539in}}%
\pgfpathlineto{\pgfqpoint{5.044915in}{1.269383in}}%
\pgfpathlineto{\pgfqpoint{5.046753in}{1.271081in}}%
\pgfpathlineto{\pgfqpoint{5.050904in}{1.284224in}}%
\pgfpathlineto{\pgfqpoint{5.055693in}{1.302474in}}%
\pgfpathlineto{\pgfqpoint{5.057717in}{1.306690in}}%
\pgfpathlineto{\pgfqpoint{5.059694in}{1.307667in}}%
\pgfpathlineto{\pgfqpoint{5.061634in}{1.305662in}}%
\pgfpathlineto{\pgfqpoint{5.065473in}{1.295761in}}%
\pgfpathlineto{\pgfqpoint{5.069511in}{1.284993in}}%
\pgfpathlineto{\pgfqpoint{5.071482in}{1.282713in}}%
\pgfpathlineto{\pgfqpoint{5.073435in}{1.283423in}}%
\pgfpathlineto{\pgfqpoint{5.075397in}{1.287221in}}%
\pgfpathlineto{\pgfqpoint{5.085687in}{1.320694in}}%
\pgfpathlineto{\pgfqpoint{5.087493in}{1.320630in}}%
\pgfpathlineto{\pgfqpoint{5.089337in}{1.317777in}}%
\pgfpathlineto{\pgfqpoint{5.096402in}{1.299912in}}%
\pgfpathlineto{\pgfqpoint{5.098246in}{1.299372in}}%
\pgfpathlineto{\pgfqpoint{5.100089in}{1.301880in}}%
\pgfpathlineto{\pgfqpoint{5.104587in}{1.317560in}}%
\pgfpathlineto{\pgfqpoint{5.108750in}{1.332969in}}%
\pgfpathlineto{\pgfqpoint{5.110770in}{1.336906in}}%
\pgfpathlineto{\pgfqpoint{5.112741in}{1.337593in}}%
\pgfpathlineto{\pgfqpoint{5.114678in}{1.335343in}}%
\pgfpathlineto{\pgfqpoint{5.124517in}{1.312972in}}%
\pgfpathlineto{\pgfqpoint{5.126467in}{1.314122in}}%
\pgfpathlineto{\pgfqpoint{5.130412in}{1.325115in}}%
\pgfpathlineto{\pgfqpoint{5.137898in}{1.350601in}}%
\pgfpathlineto{\pgfqpoint{5.139700in}{1.351474in}}%
\pgfpathlineto{\pgfqpoint{5.141540in}{1.349479in}}%
\pgfpathlineto{\pgfqpoint{5.150850in}{1.329803in}}%
\pgfpathlineto{\pgfqpoint{5.152702in}{1.331938in}}%
\pgfpathlineto{\pgfqpoint{5.157042in}{1.346454in}}%
\pgfpathlineto{\pgfqpoint{5.161810in}{1.364147in}}%
\pgfpathlineto{\pgfqpoint{5.163822in}{1.367790in}}%
\pgfpathlineto{\pgfqpoint{5.165786in}{1.368191in}}%
\pgfpathlineto{\pgfqpoint{5.167720in}{1.365704in}}%
\pgfpathlineto{\pgfqpoint{5.177561in}{1.343988in}}%
\pgfpathlineto{\pgfqpoint{5.179509in}{1.345594in}}%
\pgfpathlineto{\pgfqpoint{5.183452in}{1.357297in}}%
\pgfpathlineto{\pgfqpoint{5.189157in}{1.378849in}}%
\pgfpathlineto{\pgfqpoint{5.190937in}{1.382097in}}%
\pgfpathlineto{\pgfqpoint{5.192753in}{1.382563in}}%
\pgfpathlineto{\pgfqpoint{5.194606in}{1.380186in}}%
\pgfpathlineto{\pgfqpoint{5.202077in}{1.361661in}}%
\pgfpathlineto{\pgfqpoint{5.203939in}{1.361144in}}%
\pgfpathlineto{\pgfqpoint{5.205796in}{1.363690in}}%
\pgfpathlineto{\pgfqpoint{5.210349in}{1.379633in}}%
\pgfpathlineto{\pgfqpoint{5.214472in}{1.394976in}}%
\pgfpathlineto{\pgfqpoint{5.216484in}{1.398952in}}%
\pgfpathlineto{\pgfqpoint{5.218447in}{1.399688in}}%
\pgfpathlineto{\pgfqpoint{5.220382in}{1.397478in}}%
\pgfpathlineto{\pgfqpoint{5.230222in}{1.375734in}}%
\pgfpathlineto{\pgfqpoint{5.232167in}{1.377183in}}%
\pgfpathlineto{\pgfqpoint{5.236097in}{1.388628in}}%
\pgfpathlineto{\pgfqpoint{5.242048in}{1.411019in}}%
\pgfpathlineto{\pgfqpoint{5.243838in}{1.414114in}}%
\pgfpathlineto{\pgfqpoint{5.245665in}{1.414400in}}%
\pgfpathlineto{\pgfqpoint{5.247528in}{1.411859in}}%
\pgfpathlineto{\pgfqpoint{5.255023in}{1.393482in}}%
\pgfpathlineto{\pgfqpoint{5.256894in}{1.393153in}}%
\pgfpathlineto{\pgfqpoint{5.258758in}{1.395892in}}%
\pgfpathlineto{\pgfqpoint{5.263393in}{1.412426in}}%
\pgfpathlineto{\pgfqpoint{5.268596in}{1.429918in}}%
\pgfpathlineto{\pgfqpoint{5.270567in}{1.431882in}}%
\pgfpathlineto{\pgfqpoint{5.272507in}{1.430741in}}%
\pgfpathlineto{\pgfqpoint{5.276360in}{1.421747in}}%
\pgfpathlineto{\pgfqpoint{5.280413in}{1.410863in}}%
\pgfpathlineto{\pgfqpoint{5.282383in}{1.408369in}}%
\pgfpathlineto{\pgfqpoint{5.284325in}{1.408855in}}%
\pgfpathlineto{\pgfqpoint{5.286271in}{1.412438in}}%
\pgfpathlineto{\pgfqpoint{5.292100in}{1.434576in}}%
\pgfpathlineto{\pgfqpoint{5.295676in}{1.445421in}}%
\pgfpathlineto{\pgfqpoint{5.297494in}{1.447232in}}%
\pgfpathlineto{\pgfqpoint{5.299347in}{1.446139in}}%
\pgfpathlineto{\pgfqpoint{5.303147in}{1.437073in}}%
\pgfpathlineto{\pgfqpoint{5.306938in}{1.427353in}}%
\pgfpathlineto{\pgfqpoint{5.308816in}{1.425586in}}%
\pgfpathlineto{\pgfqpoint{5.310691in}{1.426840in}}%
\pgfpathlineto{\pgfqpoint{5.314395in}{1.437453in}}%
\pgfpathlineto{\pgfqpoint{5.320114in}{1.459597in}}%
\pgfpathlineto{\pgfqpoint{5.322106in}{1.463722in}}%
\pgfpathlineto{\pgfqpoint{5.324058in}{1.464656in}}%
\pgfpathlineto{\pgfqpoint{5.325989in}{1.462625in}}%
\pgfpathlineto{\pgfqpoint{5.335822in}{1.441355in}}%
\pgfpathlineto{\pgfqpoint{5.337761in}{1.442933in}}%
\pgfpathlineto{\pgfqpoint{5.341667in}{1.454537in}}%
\pgfpathlineto{\pgfqpoint{5.347963in}{1.477762in}}%
\pgfpathlineto{\pgfqpoint{5.349781in}{1.480353in}}%
\pgfpathlineto{\pgfqpoint{5.351634in}{1.480058in}}%
\pgfpathlineto{\pgfqpoint{5.355429in}{1.472000in}}%
\pgfpathlineto{\pgfqpoint{5.359216in}{1.461742in}}%
\pgfpathlineto{\pgfqpoint{5.361102in}{1.459182in}}%
\pgfpathlineto{\pgfqpoint{5.362987in}{1.459532in}}%
\pgfpathlineto{\pgfqpoint{5.364861in}{1.462932in}}%
\pgfpathlineto{\pgfqpoint{5.369475in}{1.480142in}}%
\pgfpathlineto{\pgfqpoint{5.373415in}{1.494273in}}%
\pgfpathlineto{\pgfqpoint{5.375383in}{1.497718in}}%
\pgfpathlineto{\pgfqpoint{5.377322in}{1.498010in}}%
\pgfpathlineto{\pgfqpoint{5.379248in}{1.495450in}}%
\pgfpathlineto{\pgfqpoint{5.387155in}{1.476064in}}%
\pgfpathlineto{\pgfqpoint{5.389097in}{1.475361in}}%
\pgfpathlineto{\pgfqpoint{5.391031in}{1.477765in}}%
\pgfpathlineto{\pgfqpoint{5.395526in}{1.493019in}}%
\pgfpathlineto{\pgfqpoint{5.400062in}{1.510142in}}%
\pgfpathlineto{\pgfqpoint{5.401878in}{1.513756in}}%
\pgfpathlineto{\pgfqpoint{5.403730in}{1.514574in}}%
\pgfpathlineto{\pgfqpoint{5.405609in}{1.512515in}}%
\pgfpathlineto{\pgfqpoint{5.415187in}{1.493172in}}%
\pgfpathlineto{\pgfqpoint{5.417072in}{1.495572in}}%
\pgfpathlineto{\pgfqpoint{5.421677in}{1.511523in}}%
\pgfpathlineto{\pgfqpoint{5.426236in}{1.528409in}}%
\pgfpathlineto{\pgfqpoint{5.428190in}{1.531927in}}%
\pgfpathlineto{\pgfqpoint{5.430121in}{1.532329in}}%
\pgfpathlineto{\pgfqpoint{5.432044in}{1.529871in}}%
\pgfpathlineto{\pgfqpoint{5.439941in}{1.510667in}}%
\pgfpathlineto{\pgfqpoint{5.441879in}{1.509995in}}%
\pgfpathlineto{\pgfqpoint{5.443810in}{1.512434in}}%
\pgfpathlineto{\pgfqpoint{5.448462in}{1.528441in}}%
\pgfpathlineto{\pgfqpoint{5.452528in}{1.544034in}}%
\pgfpathlineto{\pgfqpoint{5.454352in}{1.548110in}}%
\pgfpathlineto{\pgfqpoint{5.456209in}{1.549437in}}%
\pgfpathlineto{\pgfqpoint{5.458092in}{1.547853in}}%
\pgfpathlineto{\pgfqpoint{5.461932in}{1.538361in}}%
\pgfpathlineto{\pgfqpoint{5.465812in}{1.529103in}}%
\pgfpathlineto{\pgfqpoint{5.467714in}{1.527920in}}%
\pgfpathlineto{\pgfqpoint{5.469607in}{1.529818in}}%
\pgfpathlineto{\pgfqpoint{5.473911in}{1.543786in}}%
\pgfpathlineto{\pgfqpoint{5.479713in}{1.564824in}}%
\pgfpathlineto{\pgfqpoint{5.481644in}{1.567388in}}%
\pgfpathlineto{\pgfqpoint{5.483565in}{1.566892in}}%
\pgfpathlineto{\pgfqpoint{5.487414in}{1.558709in}}%
\pgfpathlineto{\pgfqpoint{5.491429in}{1.548025in}}%
\pgfpathlineto{\pgfqpoint{5.493384in}{1.545458in}}%
\pgfpathlineto{\pgfqpoint{5.495314in}{1.545856in}}%
\pgfpathlineto{\pgfqpoint{5.497240in}{1.549354in}}%
\pgfpathlineto{\pgfqpoint{5.502926in}{1.571047in}}%
\pgfpathlineto{\pgfqpoint{5.506541in}{1.582584in}}%
\pgfpathlineto{\pgfqpoint{5.508402in}{1.584866in}}%
\pgfpathlineto{\pgfqpoint{5.510285in}{1.584214in}}%
\pgfpathlineto{\pgfqpoint{5.514117in}{1.575806in}}%
\pgfpathlineto{\pgfqpoint{5.518009in}{1.565720in}}%
\pgfpathlineto{\pgfqpoint{5.519923in}{1.563579in}}%
\pgfpathlineto{\pgfqpoint{5.521825in}{1.564439in}}%
\pgfpathlineto{\pgfqpoint{5.525599in}{1.574693in}}%
\pgfpathlineto{\pgfqpoint{5.532614in}{1.600637in}}%
\pgfpathlineto{\pgfqpoint{5.534545in}{1.603168in}}%
\pgfpathlineto{\pgfqpoint{5.534545in}{1.603168in}}%
\pgfusepath{stroke}%
\end{pgfscope}%
\begin{pgfscope}%
\pgfsetrectcap%
\pgfsetmiterjoin%
\pgfsetlinewidth{0.803000pt}%
\definecolor{currentstroke}{rgb}{0.000000,0.000000,0.000000}%
\pgfsetstrokecolor{currentstroke}%
\pgfsetdash{}{0pt}%
\pgfpathmoveto{\pgfqpoint{0.800000in}{0.397100in}}%
\pgfpathlineto{\pgfqpoint{0.800000in}{1.660600in}}%
\pgfusepath{stroke}%
\end{pgfscope}%
\begin{pgfscope}%
\pgfsetrectcap%
\pgfsetmiterjoin%
\pgfsetlinewidth{0.803000pt}%
\definecolor{currentstroke}{rgb}{0.000000,0.000000,0.000000}%
\pgfsetstrokecolor{currentstroke}%
\pgfsetdash{}{0pt}%
\pgfpathmoveto{\pgfqpoint{5.760000in}{0.397100in}}%
\pgfpathlineto{\pgfqpoint{5.760000in}{1.660600in}}%
\pgfusepath{stroke}%
\end{pgfscope}%
\begin{pgfscope}%
\pgfsetrectcap%
\pgfsetmiterjoin%
\pgfsetlinewidth{0.803000pt}%
\definecolor{currentstroke}{rgb}{0.000000,0.000000,0.000000}%
\pgfsetstrokecolor{currentstroke}%
\pgfsetdash{}{0pt}%
\pgfpathmoveto{\pgfqpoint{0.800000in}{0.397100in}}%
\pgfpathlineto{\pgfqpoint{5.760000in}{0.397100in}}%
\pgfusepath{stroke}%
\end{pgfscope}%
\begin{pgfscope}%
\pgfsetrectcap%
\pgfsetmiterjoin%
\pgfsetlinewidth{0.803000pt}%
\definecolor{currentstroke}{rgb}{0.000000,0.000000,0.000000}%
\pgfsetstrokecolor{currentstroke}%
\pgfsetdash{}{0pt}%
\pgfpathmoveto{\pgfqpoint{0.800000in}{1.660600in}}%
\pgfpathlineto{\pgfqpoint{5.760000in}{1.660600in}}%
\pgfusepath{stroke}%
\end{pgfscope}%
\end{pgfpicture}%
\makeatother%
\endgroup%

}
\caption{Evolution of the eccentricity in the combined $e$ and $i$ transfer orbit.}
\label{fig:eccincecc}
\end{figure}

The discrepancy in the final part of the transfer corresponds with the numerical tolerance achieved for the integration.
