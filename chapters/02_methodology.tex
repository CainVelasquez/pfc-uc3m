\chapter{Methodology} \label{methodology}

\section{Cases in study} \label{cases}

For this study we have considered a variety of low-thrust control
laws that have been developed in the last 60 years. In particular,
we have focused our attention in fixed-thrust maneuvers \todo{we have
made the assumption that...?}: as the most common case for
low-thrust systems is constant acceleration \todo{[citation needed]}, this 
effectively takes the variation of mass of the spacecraft out of 
consideration.

On the other hand, to keep the implementations as simple 
as possible \todo{in fact, we focused on the simplest equations} we 
have selected those references that make no assumptions about the 
natural orbital perturbations and do not take into account the time 
spent in eclipse by the spacecraft \todo{rewrite, take from Kechichian introduction}.

Lastly, for the implementation and discussion of 
results we have taken out the guidance laws that do not have 
accompanying validation cases available in the literature \todo{or rather: that we haven't found them}. These criteria left us with the following selection of guidance laws for low-thrust trajectories:

\todo{Replace name of authors by proper references}

\begin{itemize}
\item Optimal transfer between circular coplanar orbits $a_0 \rightarrow a_f$ \cite{edelbaum1961propulsion,burt1967space}
\item Optimal transfer between circular inclined orbits $(a_0, i_0) \rightarrow (a_f, i_f), e = 0$ \cite{edelbaum1961propulsion,kechichian1997reformulation}
\item Quasi-optimal eccentricity-only change $e_0 \rightarrow e_f$ \cite{pollard1997simplified}
\item Simultaneous eccentricity and inclination change $(e_0, i_0) \rightarrow (e_f, i_f), a_0 = a_f$ \cite{pollard2000simplified}
\item Argument of periapsis adjustment $\omega_0 \rightarrow \omega_f$ \cite{pollard1998evaluation}
\end{itemize}

\section{Planar maneuvers} \label{metplanar}

Let us first focus on the planar maneuvers, those that only affect the semimajor axis $a$, eccentricity $e$ and argument of periapsis $\omega$ of the orbit. While some of them can be obtained as particular cases of more complicated guidance laws studied in \ref{metnonplanar}, for the sake of simplicity and clarity of exposition we decided to consider them separately.

\subsection{Semimajor axis change} \label{metsma}

In this section we discuss several strategies to perform a transfer between two coplanar, near-circular orbits, and we select the tangential one among the other solutions for practical considerations.

Burt studied the secular changes in the classical orbital elements originated by several thrust programs assuming that the perturbing forces are "sensibly constant over one orbital period" and also "small enough to produce a negligible variation" of each element during that period \cite{burt1967space}. For the semimajor axis in particular, and using results previously obtained by King-Hele, he considers two guidance laws: applying thrust in a direction that is within the orbital plane and perpendicular to the radius vector ($f_2$) \todo{use some clear convention for reference frames somewhere} and along the direction of the velocity ($f_T$).

\begin{figure}%[h]
  \centering
  \resizebox{1.0\textwidth}{!}
  {
  %% Creator: Matplotlib, PGF backend
%%
%% To include the figure in your LaTeX document, write
%%   \input{<filename>.pgf}
%%
%% Make sure the required packages are loaded in your preamble
%%   \usepackage{pgf}
%%
%% Figures using additional raster images can only be included by \input if
%% they are in the same directory as the main LaTeX file. For loading figures
%% from other directories you can use the `import` package
%%   \usepackage{import}
%% and then include the figures with
%%   \import{<path to file>}{<filename>.pgf}
%%
%% Matplotlib used the following preamble
%%   \usepackage{fontspec}
%%
\begingroup%
\makeatletter%
\begin{pgfpicture}%
\pgfpathrectangle{\pgfpointorigin}{\pgfqpoint{6.400000in}{4.800000in}}%
\pgfusepath{use as bounding box, clip}%
\begin{pgfscope}%
\pgfsetbuttcap%
\pgfsetmiterjoin%
\definecolor{currentfill}{rgb}{1.000000,1.000000,1.000000}%
\pgfsetfillcolor{currentfill}%
\pgfsetlinewidth{0.000000pt}%
\definecolor{currentstroke}{rgb}{1.000000,1.000000,1.000000}%
\pgfsetstrokecolor{currentstroke}%
\pgfsetdash{}{0pt}%
\pgfpathmoveto{\pgfqpoint{0.000000in}{0.000000in}}%
\pgfpathlineto{\pgfqpoint{6.400000in}{0.000000in}}%
\pgfpathlineto{\pgfqpoint{6.400000in}{4.800000in}}%
\pgfpathlineto{\pgfqpoint{0.000000in}{4.800000in}}%
\pgfpathclose%
\pgfusepath{fill}%
\end{pgfscope}%
\begin{pgfscope}%
\pgfsetbuttcap%
\pgfsetmiterjoin%
\definecolor{currentfill}{rgb}{1.000000,1.000000,1.000000}%
\pgfsetfillcolor{currentfill}%
\pgfsetlinewidth{0.000000pt}%
\definecolor{currentstroke}{rgb}{0.000000,0.000000,0.000000}%
\pgfsetstrokecolor{currentstroke}%
\pgfsetstrokeopacity{0.000000}%
\pgfsetdash{}{0pt}%
\pgfpathmoveto{\pgfqpoint{0.800000in}{0.528000in}}%
\pgfpathlineto{\pgfqpoint{5.760000in}{0.528000in}}%
\pgfpathlineto{\pgfqpoint{5.760000in}{4.224000in}}%
\pgfpathlineto{\pgfqpoint{0.800000in}{4.224000in}}%
\pgfpathclose%
\pgfusepath{fill}%
\end{pgfscope}%
\begin{pgfscope}%
\pgfsetbuttcap%
\pgfsetroundjoin%
\definecolor{currentfill}{rgb}{0.000000,0.000000,0.000000}%
\pgfsetfillcolor{currentfill}%
\pgfsetlinewidth{0.803000pt}%
\definecolor{currentstroke}{rgb}{0.000000,0.000000,0.000000}%
\pgfsetstrokecolor{currentstroke}%
\pgfsetdash{}{0pt}%
\pgfsys@defobject{currentmarker}{\pgfqpoint{0.000000in}{-0.048611in}}{\pgfqpoint{0.000000in}{0.000000in}}{%
\pgfpathmoveto{\pgfqpoint{0.000000in}{0.000000in}}%
\pgfpathlineto{\pgfqpoint{0.000000in}{-0.048611in}}%
\pgfusepath{stroke,fill}%
}%
\begin{pgfscope}%
\pgfsys@transformshift{1.025455in}{0.528000in}%
\pgfsys@useobject{currentmarker}{}%
\end{pgfscope}%
\end{pgfscope}%
\begin{pgfscope}%
\pgftext[x=1.025455in,y=0.430778in,,top]{\sffamily\fontsize{10.000000}{12.000000}\selectfont \(\displaystyle 10^{-1}\)}%
\end{pgfscope}%
\begin{pgfscope}%
\pgfsetbuttcap%
\pgfsetroundjoin%
\definecolor{currentfill}{rgb}{0.000000,0.000000,0.000000}%
\pgfsetfillcolor{currentfill}%
\pgfsetlinewidth{0.803000pt}%
\definecolor{currentstroke}{rgb}{0.000000,0.000000,0.000000}%
\pgfsetstrokecolor{currentstroke}%
\pgfsetdash{}{0pt}%
\pgfsys@defobject{currentmarker}{\pgfqpoint{0.000000in}{-0.048611in}}{\pgfqpoint{0.000000in}{0.000000in}}{%
\pgfpathmoveto{\pgfqpoint{0.000000in}{0.000000in}}%
\pgfpathlineto{\pgfqpoint{0.000000in}{-0.048611in}}%
\pgfusepath{stroke,fill}%
}%
\begin{pgfscope}%
\pgfsys@transformshift{3.280000in}{0.528000in}%
\pgfsys@useobject{currentmarker}{}%
\end{pgfscope}%
\end{pgfscope}%
\begin{pgfscope}%
\pgftext[x=3.280000in,y=0.430778in,,top]{\sffamily\fontsize{10.000000}{12.000000}\selectfont \(\displaystyle 10^{0}\)}%
\end{pgfscope}%
\begin{pgfscope}%
\pgfsetbuttcap%
\pgfsetroundjoin%
\definecolor{currentfill}{rgb}{0.000000,0.000000,0.000000}%
\pgfsetfillcolor{currentfill}%
\pgfsetlinewidth{0.803000pt}%
\definecolor{currentstroke}{rgb}{0.000000,0.000000,0.000000}%
\pgfsetstrokecolor{currentstroke}%
\pgfsetdash{}{0pt}%
\pgfsys@defobject{currentmarker}{\pgfqpoint{0.000000in}{-0.048611in}}{\pgfqpoint{0.000000in}{0.000000in}}{%
\pgfpathmoveto{\pgfqpoint{0.000000in}{0.000000in}}%
\pgfpathlineto{\pgfqpoint{0.000000in}{-0.048611in}}%
\pgfusepath{stroke,fill}%
}%
\begin{pgfscope}%
\pgfsys@transformshift{5.534545in}{0.528000in}%
\pgfsys@useobject{currentmarker}{}%
\end{pgfscope}%
\end{pgfscope}%
\begin{pgfscope}%
\pgftext[x=5.534545in,y=0.430778in,,top]{\sffamily\fontsize{10.000000}{12.000000}\selectfont \(\displaystyle 10^{1}\)}%
\end{pgfscope}%
\begin{pgfscope}%
\pgfsetbuttcap%
\pgfsetroundjoin%
\definecolor{currentfill}{rgb}{0.000000,0.000000,0.000000}%
\pgfsetfillcolor{currentfill}%
\pgfsetlinewidth{0.602250pt}%
\definecolor{currentstroke}{rgb}{0.000000,0.000000,0.000000}%
\pgfsetstrokecolor{currentstroke}%
\pgfsetdash{}{0pt}%
\pgfsys@defobject{currentmarker}{\pgfqpoint{0.000000in}{-0.027778in}}{\pgfqpoint{0.000000in}{0.000000in}}{%
\pgfpathmoveto{\pgfqpoint{0.000000in}{0.000000in}}%
\pgfpathlineto{\pgfqpoint{0.000000in}{-0.027778in}}%
\pgfusepath{stroke,fill}%
}%
\begin{pgfscope}%
\pgfsys@transformshift{0.806967in}{0.528000in}%
\pgfsys@useobject{currentmarker}{}%
\end{pgfscope}%
\end{pgfscope}%
\begin{pgfscope}%
\pgfsetbuttcap%
\pgfsetroundjoin%
\definecolor{currentfill}{rgb}{0.000000,0.000000,0.000000}%
\pgfsetfillcolor{currentfill}%
\pgfsetlinewidth{0.602250pt}%
\definecolor{currentstroke}{rgb}{0.000000,0.000000,0.000000}%
\pgfsetstrokecolor{currentstroke}%
\pgfsetdash{}{0pt}%
\pgfsys@defobject{currentmarker}{\pgfqpoint{0.000000in}{-0.027778in}}{\pgfqpoint{0.000000in}{0.000000in}}{%
\pgfpathmoveto{\pgfqpoint{0.000000in}{0.000000in}}%
\pgfpathlineto{\pgfqpoint{0.000000in}{-0.027778in}}%
\pgfusepath{stroke,fill}%
}%
\begin{pgfscope}%
\pgfsys@transformshift{0.922292in}{0.528000in}%
\pgfsys@useobject{currentmarker}{}%
\end{pgfscope}%
\end{pgfscope}%
\begin{pgfscope}%
\pgfsetbuttcap%
\pgfsetroundjoin%
\definecolor{currentfill}{rgb}{0.000000,0.000000,0.000000}%
\pgfsetfillcolor{currentfill}%
\pgfsetlinewidth{0.602250pt}%
\definecolor{currentstroke}{rgb}{0.000000,0.000000,0.000000}%
\pgfsetstrokecolor{currentstroke}%
\pgfsetdash{}{0pt}%
\pgfsys@defobject{currentmarker}{\pgfqpoint{0.000000in}{-0.027778in}}{\pgfqpoint{0.000000in}{0.000000in}}{%
\pgfpathmoveto{\pgfqpoint{0.000000in}{0.000000in}}%
\pgfpathlineto{\pgfqpoint{0.000000in}{-0.027778in}}%
\pgfusepath{stroke,fill}%
}%
\begin{pgfscope}%
\pgfsys@transformshift{1.704140in}{0.528000in}%
\pgfsys@useobject{currentmarker}{}%
\end{pgfscope}%
\end{pgfscope}%
\begin{pgfscope}%
\pgfsetbuttcap%
\pgfsetroundjoin%
\definecolor{currentfill}{rgb}{0.000000,0.000000,0.000000}%
\pgfsetfillcolor{currentfill}%
\pgfsetlinewidth{0.602250pt}%
\definecolor{currentstroke}{rgb}{0.000000,0.000000,0.000000}%
\pgfsetstrokecolor{currentstroke}%
\pgfsetdash{}{0pt}%
\pgfsys@defobject{currentmarker}{\pgfqpoint{0.000000in}{-0.027778in}}{\pgfqpoint{0.000000in}{0.000000in}}{%
\pgfpathmoveto{\pgfqpoint{0.000000in}{0.000000in}}%
\pgfpathlineto{\pgfqpoint{0.000000in}{-0.027778in}}%
\pgfusepath{stroke,fill}%
}%
\begin{pgfscope}%
\pgfsys@transformshift{2.101146in}{0.528000in}%
\pgfsys@useobject{currentmarker}{}%
\end{pgfscope}%
\end{pgfscope}%
\begin{pgfscope}%
\pgfsetbuttcap%
\pgfsetroundjoin%
\definecolor{currentfill}{rgb}{0.000000,0.000000,0.000000}%
\pgfsetfillcolor{currentfill}%
\pgfsetlinewidth{0.602250pt}%
\definecolor{currentstroke}{rgb}{0.000000,0.000000,0.000000}%
\pgfsetstrokecolor{currentstroke}%
\pgfsetdash{}{0pt}%
\pgfsys@defobject{currentmarker}{\pgfqpoint{0.000000in}{-0.027778in}}{\pgfqpoint{0.000000in}{0.000000in}}{%
\pgfpathmoveto{\pgfqpoint{0.000000in}{0.000000in}}%
\pgfpathlineto{\pgfqpoint{0.000000in}{-0.027778in}}%
\pgfusepath{stroke,fill}%
}%
\begin{pgfscope}%
\pgfsys@transformshift{2.382826in}{0.528000in}%
\pgfsys@useobject{currentmarker}{}%
\end{pgfscope}%
\end{pgfscope}%
\begin{pgfscope}%
\pgfsetbuttcap%
\pgfsetroundjoin%
\definecolor{currentfill}{rgb}{0.000000,0.000000,0.000000}%
\pgfsetfillcolor{currentfill}%
\pgfsetlinewidth{0.602250pt}%
\definecolor{currentstroke}{rgb}{0.000000,0.000000,0.000000}%
\pgfsetstrokecolor{currentstroke}%
\pgfsetdash{}{0pt}%
\pgfsys@defobject{currentmarker}{\pgfqpoint{0.000000in}{-0.027778in}}{\pgfqpoint{0.000000in}{0.000000in}}{%
\pgfpathmoveto{\pgfqpoint{0.000000in}{0.000000in}}%
\pgfpathlineto{\pgfqpoint{0.000000in}{-0.027778in}}%
\pgfusepath{stroke,fill}%
}%
\begin{pgfscope}%
\pgfsys@transformshift{2.601314in}{0.528000in}%
\pgfsys@useobject{currentmarker}{}%
\end{pgfscope}%
\end{pgfscope}%
\begin{pgfscope}%
\pgfsetbuttcap%
\pgfsetroundjoin%
\definecolor{currentfill}{rgb}{0.000000,0.000000,0.000000}%
\pgfsetfillcolor{currentfill}%
\pgfsetlinewidth{0.602250pt}%
\definecolor{currentstroke}{rgb}{0.000000,0.000000,0.000000}%
\pgfsetstrokecolor{currentstroke}%
\pgfsetdash{}{0pt}%
\pgfsys@defobject{currentmarker}{\pgfqpoint{0.000000in}{-0.027778in}}{\pgfqpoint{0.000000in}{0.000000in}}{%
\pgfpathmoveto{\pgfqpoint{0.000000in}{0.000000in}}%
\pgfpathlineto{\pgfqpoint{0.000000in}{-0.027778in}}%
\pgfusepath{stroke,fill}%
}%
\begin{pgfscope}%
\pgfsys@transformshift{2.779832in}{0.528000in}%
\pgfsys@useobject{currentmarker}{}%
\end{pgfscope}%
\end{pgfscope}%
\begin{pgfscope}%
\pgfsetbuttcap%
\pgfsetroundjoin%
\definecolor{currentfill}{rgb}{0.000000,0.000000,0.000000}%
\pgfsetfillcolor{currentfill}%
\pgfsetlinewidth{0.602250pt}%
\definecolor{currentstroke}{rgb}{0.000000,0.000000,0.000000}%
\pgfsetstrokecolor{currentstroke}%
\pgfsetdash{}{0pt}%
\pgfsys@defobject{currentmarker}{\pgfqpoint{0.000000in}{-0.027778in}}{\pgfqpoint{0.000000in}{0.000000in}}{%
\pgfpathmoveto{\pgfqpoint{0.000000in}{0.000000in}}%
\pgfpathlineto{\pgfqpoint{0.000000in}{-0.027778in}}%
\pgfusepath{stroke,fill}%
}%
\begin{pgfscope}%
\pgfsys@transformshift{2.930766in}{0.528000in}%
\pgfsys@useobject{currentmarker}{}%
\end{pgfscope}%
\end{pgfscope}%
\begin{pgfscope}%
\pgfsetbuttcap%
\pgfsetroundjoin%
\definecolor{currentfill}{rgb}{0.000000,0.000000,0.000000}%
\pgfsetfillcolor{currentfill}%
\pgfsetlinewidth{0.602250pt}%
\definecolor{currentstroke}{rgb}{0.000000,0.000000,0.000000}%
\pgfsetstrokecolor{currentstroke}%
\pgfsetdash{}{0pt}%
\pgfsys@defobject{currentmarker}{\pgfqpoint{0.000000in}{-0.027778in}}{\pgfqpoint{0.000000in}{0.000000in}}{%
\pgfpathmoveto{\pgfqpoint{0.000000in}{0.000000in}}%
\pgfpathlineto{\pgfqpoint{0.000000in}{-0.027778in}}%
\pgfusepath{stroke,fill}%
}%
\begin{pgfscope}%
\pgfsys@transformshift{3.061512in}{0.528000in}%
\pgfsys@useobject{currentmarker}{}%
\end{pgfscope}%
\end{pgfscope}%
\begin{pgfscope}%
\pgfsetbuttcap%
\pgfsetroundjoin%
\definecolor{currentfill}{rgb}{0.000000,0.000000,0.000000}%
\pgfsetfillcolor{currentfill}%
\pgfsetlinewidth{0.602250pt}%
\definecolor{currentstroke}{rgb}{0.000000,0.000000,0.000000}%
\pgfsetstrokecolor{currentstroke}%
\pgfsetdash{}{0pt}%
\pgfsys@defobject{currentmarker}{\pgfqpoint{0.000000in}{-0.027778in}}{\pgfqpoint{0.000000in}{0.000000in}}{%
\pgfpathmoveto{\pgfqpoint{0.000000in}{0.000000in}}%
\pgfpathlineto{\pgfqpoint{0.000000in}{-0.027778in}}%
\pgfusepath{stroke,fill}%
}%
\begin{pgfscope}%
\pgfsys@transformshift{3.176838in}{0.528000in}%
\pgfsys@useobject{currentmarker}{}%
\end{pgfscope}%
\end{pgfscope}%
\begin{pgfscope}%
\pgfsetbuttcap%
\pgfsetroundjoin%
\definecolor{currentfill}{rgb}{0.000000,0.000000,0.000000}%
\pgfsetfillcolor{currentfill}%
\pgfsetlinewidth{0.602250pt}%
\definecolor{currentstroke}{rgb}{0.000000,0.000000,0.000000}%
\pgfsetstrokecolor{currentstroke}%
\pgfsetdash{}{0pt}%
\pgfsys@defobject{currentmarker}{\pgfqpoint{0.000000in}{-0.027778in}}{\pgfqpoint{0.000000in}{0.000000in}}{%
\pgfpathmoveto{\pgfqpoint{0.000000in}{0.000000in}}%
\pgfpathlineto{\pgfqpoint{0.000000in}{-0.027778in}}%
\pgfusepath{stroke,fill}%
}%
\begin{pgfscope}%
\pgfsys@transformshift{3.958686in}{0.528000in}%
\pgfsys@useobject{currentmarker}{}%
\end{pgfscope}%
\end{pgfscope}%
\begin{pgfscope}%
\pgfsetbuttcap%
\pgfsetroundjoin%
\definecolor{currentfill}{rgb}{0.000000,0.000000,0.000000}%
\pgfsetfillcolor{currentfill}%
\pgfsetlinewidth{0.602250pt}%
\definecolor{currentstroke}{rgb}{0.000000,0.000000,0.000000}%
\pgfsetstrokecolor{currentstroke}%
\pgfsetdash{}{0pt}%
\pgfsys@defobject{currentmarker}{\pgfqpoint{0.000000in}{-0.027778in}}{\pgfqpoint{0.000000in}{0.000000in}}{%
\pgfpathmoveto{\pgfqpoint{0.000000in}{0.000000in}}%
\pgfpathlineto{\pgfqpoint{0.000000in}{-0.027778in}}%
\pgfusepath{stroke,fill}%
}%
\begin{pgfscope}%
\pgfsys@transformshift{4.355692in}{0.528000in}%
\pgfsys@useobject{currentmarker}{}%
\end{pgfscope}%
\end{pgfscope}%
\begin{pgfscope}%
\pgfsetbuttcap%
\pgfsetroundjoin%
\definecolor{currentfill}{rgb}{0.000000,0.000000,0.000000}%
\pgfsetfillcolor{currentfill}%
\pgfsetlinewidth{0.602250pt}%
\definecolor{currentstroke}{rgb}{0.000000,0.000000,0.000000}%
\pgfsetstrokecolor{currentstroke}%
\pgfsetdash{}{0pt}%
\pgfsys@defobject{currentmarker}{\pgfqpoint{0.000000in}{-0.027778in}}{\pgfqpoint{0.000000in}{0.000000in}}{%
\pgfpathmoveto{\pgfqpoint{0.000000in}{0.000000in}}%
\pgfpathlineto{\pgfqpoint{0.000000in}{-0.027778in}}%
\pgfusepath{stroke,fill}%
}%
\begin{pgfscope}%
\pgfsys@transformshift{4.637372in}{0.528000in}%
\pgfsys@useobject{currentmarker}{}%
\end{pgfscope}%
\end{pgfscope}%
\begin{pgfscope}%
\pgfsetbuttcap%
\pgfsetroundjoin%
\definecolor{currentfill}{rgb}{0.000000,0.000000,0.000000}%
\pgfsetfillcolor{currentfill}%
\pgfsetlinewidth{0.602250pt}%
\definecolor{currentstroke}{rgb}{0.000000,0.000000,0.000000}%
\pgfsetstrokecolor{currentstroke}%
\pgfsetdash{}{0pt}%
\pgfsys@defobject{currentmarker}{\pgfqpoint{0.000000in}{-0.027778in}}{\pgfqpoint{0.000000in}{0.000000in}}{%
\pgfpathmoveto{\pgfqpoint{0.000000in}{0.000000in}}%
\pgfpathlineto{\pgfqpoint{0.000000in}{-0.027778in}}%
\pgfusepath{stroke,fill}%
}%
\begin{pgfscope}%
\pgfsys@transformshift{4.855860in}{0.528000in}%
\pgfsys@useobject{currentmarker}{}%
\end{pgfscope}%
\end{pgfscope}%
\begin{pgfscope}%
\pgfsetbuttcap%
\pgfsetroundjoin%
\definecolor{currentfill}{rgb}{0.000000,0.000000,0.000000}%
\pgfsetfillcolor{currentfill}%
\pgfsetlinewidth{0.602250pt}%
\definecolor{currentstroke}{rgb}{0.000000,0.000000,0.000000}%
\pgfsetstrokecolor{currentstroke}%
\pgfsetdash{}{0pt}%
\pgfsys@defobject{currentmarker}{\pgfqpoint{0.000000in}{-0.027778in}}{\pgfqpoint{0.000000in}{0.000000in}}{%
\pgfpathmoveto{\pgfqpoint{0.000000in}{0.000000in}}%
\pgfpathlineto{\pgfqpoint{0.000000in}{-0.027778in}}%
\pgfusepath{stroke,fill}%
}%
\begin{pgfscope}%
\pgfsys@transformshift{5.034377in}{0.528000in}%
\pgfsys@useobject{currentmarker}{}%
\end{pgfscope}%
\end{pgfscope}%
\begin{pgfscope}%
\pgfsetbuttcap%
\pgfsetroundjoin%
\definecolor{currentfill}{rgb}{0.000000,0.000000,0.000000}%
\pgfsetfillcolor{currentfill}%
\pgfsetlinewidth{0.602250pt}%
\definecolor{currentstroke}{rgb}{0.000000,0.000000,0.000000}%
\pgfsetstrokecolor{currentstroke}%
\pgfsetdash{}{0pt}%
\pgfsys@defobject{currentmarker}{\pgfqpoint{0.000000in}{-0.027778in}}{\pgfqpoint{0.000000in}{0.000000in}}{%
\pgfpathmoveto{\pgfqpoint{0.000000in}{0.000000in}}%
\pgfpathlineto{\pgfqpoint{0.000000in}{-0.027778in}}%
\pgfusepath{stroke,fill}%
}%
\begin{pgfscope}%
\pgfsys@transformshift{5.185312in}{0.528000in}%
\pgfsys@useobject{currentmarker}{}%
\end{pgfscope}%
\end{pgfscope}%
\begin{pgfscope}%
\pgfsetbuttcap%
\pgfsetroundjoin%
\definecolor{currentfill}{rgb}{0.000000,0.000000,0.000000}%
\pgfsetfillcolor{currentfill}%
\pgfsetlinewidth{0.602250pt}%
\definecolor{currentstroke}{rgb}{0.000000,0.000000,0.000000}%
\pgfsetstrokecolor{currentstroke}%
\pgfsetdash{}{0pt}%
\pgfsys@defobject{currentmarker}{\pgfqpoint{0.000000in}{-0.027778in}}{\pgfqpoint{0.000000in}{0.000000in}}{%
\pgfpathmoveto{\pgfqpoint{0.000000in}{0.000000in}}%
\pgfpathlineto{\pgfqpoint{0.000000in}{-0.027778in}}%
\pgfusepath{stroke,fill}%
}%
\begin{pgfscope}%
\pgfsys@transformshift{5.316057in}{0.528000in}%
\pgfsys@useobject{currentmarker}{}%
\end{pgfscope}%
\end{pgfscope}%
\begin{pgfscope}%
\pgfsetbuttcap%
\pgfsetroundjoin%
\definecolor{currentfill}{rgb}{0.000000,0.000000,0.000000}%
\pgfsetfillcolor{currentfill}%
\pgfsetlinewidth{0.602250pt}%
\definecolor{currentstroke}{rgb}{0.000000,0.000000,0.000000}%
\pgfsetstrokecolor{currentstroke}%
\pgfsetdash{}{0pt}%
\pgfsys@defobject{currentmarker}{\pgfqpoint{0.000000in}{-0.027778in}}{\pgfqpoint{0.000000in}{0.000000in}}{%
\pgfpathmoveto{\pgfqpoint{0.000000in}{0.000000in}}%
\pgfpathlineto{\pgfqpoint{0.000000in}{-0.027778in}}%
\pgfusepath{stroke,fill}%
}%
\begin{pgfscope}%
\pgfsys@transformshift{5.431383in}{0.528000in}%
\pgfsys@useobject{currentmarker}{}%
\end{pgfscope}%
\end{pgfscope}%
\begin{pgfscope}%
\pgftext[x=3.280000in,y=0.251889in,,top]{\sffamily\fontsize{10.000000}{12.000000}\selectfont \(\displaystyle a / a_0\)}%
\end{pgfscope}%
\begin{pgfscope}%
\pgfsetbuttcap%
\pgfsetroundjoin%
\definecolor{currentfill}{rgb}{0.000000,0.000000,0.000000}%
\pgfsetfillcolor{currentfill}%
\pgfsetlinewidth{0.803000pt}%
\definecolor{currentstroke}{rgb}{0.000000,0.000000,0.000000}%
\pgfsetstrokecolor{currentstroke}%
\pgfsetdash{}{0pt}%
\pgfsys@defobject{currentmarker}{\pgfqpoint{-0.048611in}{0.000000in}}{\pgfqpoint{0.000000in}{0.000000in}}{%
\pgfpathmoveto{\pgfqpoint{0.000000in}{0.000000in}}%
\pgfpathlineto{\pgfqpoint{-0.048611in}{0.000000in}}%
\pgfusepath{stroke,fill}%
}%
\begin{pgfscope}%
\pgfsys@transformshift{0.800000in}{0.624571in}%
\pgfsys@useobject{currentmarker}{}%
\end{pgfscope}%
\end{pgfscope}%
\begin{pgfscope}%
\pgftext[x=0.278394in,y=0.576377in,left,base]{\sffamily\fontsize{10.000000}{12.000000}\selectfont \(\displaystyle -0.003\)}%
\end{pgfscope}%
\begin{pgfscope}%
\pgfsetbuttcap%
\pgfsetroundjoin%
\definecolor{currentfill}{rgb}{0.000000,0.000000,0.000000}%
\pgfsetfillcolor{currentfill}%
\pgfsetlinewidth{0.803000pt}%
\definecolor{currentstroke}{rgb}{0.000000,0.000000,0.000000}%
\pgfsetstrokecolor{currentstroke}%
\pgfsetdash{}{0pt}%
\pgfsys@defobject{currentmarker}{\pgfqpoint{-0.048611in}{0.000000in}}{\pgfqpoint{0.000000in}{0.000000in}}{%
\pgfpathmoveto{\pgfqpoint{0.000000in}{0.000000in}}%
\pgfpathlineto{\pgfqpoint{-0.048611in}{0.000000in}}%
\pgfusepath{stroke,fill}%
}%
\begin{pgfscope}%
\pgfsys@transformshift{0.800000in}{1.092619in}%
\pgfsys@useobject{currentmarker}{}%
\end{pgfscope}%
\end{pgfscope}%
\begin{pgfscope}%
\pgftext[x=0.278394in,y=1.044425in,left,base]{\sffamily\fontsize{10.000000}{12.000000}\selectfont \(\displaystyle -0.002\)}%
\end{pgfscope}%
\begin{pgfscope}%
\pgfsetbuttcap%
\pgfsetroundjoin%
\definecolor{currentfill}{rgb}{0.000000,0.000000,0.000000}%
\pgfsetfillcolor{currentfill}%
\pgfsetlinewidth{0.803000pt}%
\definecolor{currentstroke}{rgb}{0.000000,0.000000,0.000000}%
\pgfsetstrokecolor{currentstroke}%
\pgfsetdash{}{0pt}%
\pgfsys@defobject{currentmarker}{\pgfqpoint{-0.048611in}{0.000000in}}{\pgfqpoint{0.000000in}{0.000000in}}{%
\pgfpathmoveto{\pgfqpoint{0.000000in}{0.000000in}}%
\pgfpathlineto{\pgfqpoint{-0.048611in}{0.000000in}}%
\pgfusepath{stroke,fill}%
}%
\begin{pgfscope}%
\pgfsys@transformshift{0.800000in}{1.560667in}%
\pgfsys@useobject{currentmarker}{}%
\end{pgfscope}%
\end{pgfscope}%
\begin{pgfscope}%
\pgftext[x=0.278394in,y=1.512473in,left,base]{\sffamily\fontsize{10.000000}{12.000000}\selectfont \(\displaystyle -0.001\)}%
\end{pgfscope}%
\begin{pgfscope}%
\pgfsetbuttcap%
\pgfsetroundjoin%
\definecolor{currentfill}{rgb}{0.000000,0.000000,0.000000}%
\pgfsetfillcolor{currentfill}%
\pgfsetlinewidth{0.803000pt}%
\definecolor{currentstroke}{rgb}{0.000000,0.000000,0.000000}%
\pgfsetstrokecolor{currentstroke}%
\pgfsetdash{}{0pt}%
\pgfsys@defobject{currentmarker}{\pgfqpoint{-0.048611in}{0.000000in}}{\pgfqpoint{0.000000in}{0.000000in}}{%
\pgfpathmoveto{\pgfqpoint{0.000000in}{0.000000in}}%
\pgfpathlineto{\pgfqpoint{-0.048611in}{0.000000in}}%
\pgfusepath{stroke,fill}%
}%
\begin{pgfscope}%
\pgfsys@transformshift{0.800000in}{2.028715in}%
\pgfsys@useobject{currentmarker}{}%
\end{pgfscope}%
\end{pgfscope}%
\begin{pgfscope}%
\pgftext[x=0.386419in,y=1.980520in,left,base]{\sffamily\fontsize{10.000000}{12.000000}\selectfont \(\displaystyle 0.000\)}%
\end{pgfscope}%
\begin{pgfscope}%
\pgfsetbuttcap%
\pgfsetroundjoin%
\definecolor{currentfill}{rgb}{0.000000,0.000000,0.000000}%
\pgfsetfillcolor{currentfill}%
\pgfsetlinewidth{0.803000pt}%
\definecolor{currentstroke}{rgb}{0.000000,0.000000,0.000000}%
\pgfsetstrokecolor{currentstroke}%
\pgfsetdash{}{0pt}%
\pgfsys@defobject{currentmarker}{\pgfqpoint{-0.048611in}{0.000000in}}{\pgfqpoint{0.000000in}{0.000000in}}{%
\pgfpathmoveto{\pgfqpoint{0.000000in}{0.000000in}}%
\pgfpathlineto{\pgfqpoint{-0.048611in}{0.000000in}}%
\pgfusepath{stroke,fill}%
}%
\begin{pgfscope}%
\pgfsys@transformshift{0.800000in}{2.496763in}%
\pgfsys@useobject{currentmarker}{}%
\end{pgfscope}%
\end{pgfscope}%
\begin{pgfscope}%
\pgftext[x=0.386419in,y=2.448568in,left,base]{\sffamily\fontsize{10.000000}{12.000000}\selectfont \(\displaystyle 0.001\)}%
\end{pgfscope}%
\begin{pgfscope}%
\pgfsetbuttcap%
\pgfsetroundjoin%
\definecolor{currentfill}{rgb}{0.000000,0.000000,0.000000}%
\pgfsetfillcolor{currentfill}%
\pgfsetlinewidth{0.803000pt}%
\definecolor{currentstroke}{rgb}{0.000000,0.000000,0.000000}%
\pgfsetstrokecolor{currentstroke}%
\pgfsetdash{}{0pt}%
\pgfsys@defobject{currentmarker}{\pgfqpoint{-0.048611in}{0.000000in}}{\pgfqpoint{0.000000in}{0.000000in}}{%
\pgfpathmoveto{\pgfqpoint{0.000000in}{0.000000in}}%
\pgfpathlineto{\pgfqpoint{-0.048611in}{0.000000in}}%
\pgfusepath{stroke,fill}%
}%
\begin{pgfscope}%
\pgfsys@transformshift{0.800000in}{2.964811in}%
\pgfsys@useobject{currentmarker}{}%
\end{pgfscope}%
\end{pgfscope}%
\begin{pgfscope}%
\pgftext[x=0.386419in,y=2.916616in,left,base]{\sffamily\fontsize{10.000000}{12.000000}\selectfont \(\displaystyle 0.002\)}%
\end{pgfscope}%
\begin{pgfscope}%
\pgfsetbuttcap%
\pgfsetroundjoin%
\definecolor{currentfill}{rgb}{0.000000,0.000000,0.000000}%
\pgfsetfillcolor{currentfill}%
\pgfsetlinewidth{0.803000pt}%
\definecolor{currentstroke}{rgb}{0.000000,0.000000,0.000000}%
\pgfsetstrokecolor{currentstroke}%
\pgfsetdash{}{0pt}%
\pgfsys@defobject{currentmarker}{\pgfqpoint{-0.048611in}{0.000000in}}{\pgfqpoint{0.000000in}{0.000000in}}{%
\pgfpathmoveto{\pgfqpoint{0.000000in}{0.000000in}}%
\pgfpathlineto{\pgfqpoint{-0.048611in}{0.000000in}}%
\pgfusepath{stroke,fill}%
}%
\begin{pgfscope}%
\pgfsys@transformshift{0.800000in}{3.432858in}%
\pgfsys@useobject{currentmarker}{}%
\end{pgfscope}%
\end{pgfscope}%
\begin{pgfscope}%
\pgftext[x=0.386419in,y=3.384664in,left,base]{\sffamily\fontsize{10.000000}{12.000000}\selectfont \(\displaystyle 0.003\)}%
\end{pgfscope}%
\begin{pgfscope}%
\pgfsetbuttcap%
\pgfsetroundjoin%
\definecolor{currentfill}{rgb}{0.000000,0.000000,0.000000}%
\pgfsetfillcolor{currentfill}%
\pgfsetlinewidth{0.803000pt}%
\definecolor{currentstroke}{rgb}{0.000000,0.000000,0.000000}%
\pgfsetstrokecolor{currentstroke}%
\pgfsetdash{}{0pt}%
\pgfsys@defobject{currentmarker}{\pgfqpoint{-0.048611in}{0.000000in}}{\pgfqpoint{0.000000in}{0.000000in}}{%
\pgfpathmoveto{\pgfqpoint{0.000000in}{0.000000in}}%
\pgfpathlineto{\pgfqpoint{-0.048611in}{0.000000in}}%
\pgfusepath{stroke,fill}%
}%
\begin{pgfscope}%
\pgfsys@transformshift{0.800000in}{3.900906in}%
\pgfsys@useobject{currentmarker}{}%
\end{pgfscope}%
\end{pgfscope}%
\begin{pgfscope}%
\pgftext[x=0.386419in,y=3.852712in,left,base]{\sffamily\fontsize{10.000000}{12.000000}\selectfont \(\displaystyle 0.004\)}%
\end{pgfscope}%
\begin{pgfscope}%
\pgfpathrectangle{\pgfqpoint{0.800000in}{0.528000in}}{\pgfqpoint{4.960000in}{3.696000in}} %
\pgfusepath{clip}%
\pgfsetrectcap%
\pgfsetroundjoin%
\pgfsetlinewidth{1.505625pt}%
\definecolor{currentstroke}{rgb}{0.000000,0.000000,0.000000}%
\pgfsetstrokecolor{currentstroke}%
\pgfsetdash{}{0pt}%
\pgfpathmoveto{\pgfqpoint{1.025455in}{2.186928in}}%
\pgfpathlineto{\pgfqpoint{1.117477in}{2.185425in}}%
\pgfpathlineto{\pgfqpoint{1.209499in}{2.183864in}}%
\pgfpathlineto{\pgfqpoint{1.301521in}{2.182240in}}%
\pgfpathlineto{\pgfqpoint{1.393544in}{2.180549in}}%
\pgfpathlineto{\pgfqpoint{1.485566in}{2.178785in}}%
\pgfpathlineto{\pgfqpoint{1.577588in}{2.176945in}}%
\pgfpathlineto{\pgfqpoint{1.669610in}{2.175024in}}%
\pgfpathlineto{\pgfqpoint{1.761633in}{2.173015in}}%
\pgfpathlineto{\pgfqpoint{1.853655in}{2.170916in}}%
\pgfpathlineto{\pgfqpoint{1.945677in}{2.168720in}}%
\pgfpathlineto{\pgfqpoint{2.037699in}{2.166422in}}%
\pgfpathlineto{\pgfqpoint{2.129722in}{2.164018in}}%
\pgfpathlineto{\pgfqpoint{2.221744in}{2.161500in}}%
\pgfpathlineto{\pgfqpoint{2.313766in}{2.158864in}}%
\pgfpathlineto{\pgfqpoint{2.405788in}{2.156103in}}%
\pgfpathlineto{\pgfqpoint{2.497811in}{2.153212in}}%
\pgfpathlineto{\pgfqpoint{2.589833in}{2.150183in}}%
\pgfpathlineto{\pgfqpoint{2.681855in}{2.147010in}}%
\pgfpathlineto{\pgfqpoint{2.773878in}{2.143685in}}%
\pgfpathlineto{\pgfqpoint{2.865900in}{2.140202in}}%
\pgfpathlineto{\pgfqpoint{2.957922in}{2.136552in}}%
\pgfpathlineto{\pgfqpoint{3.049944in}{2.132727in}}%
\pgfpathlineto{\pgfqpoint{3.141967in}{2.128719in}}%
\pgfpathlineto{\pgfqpoint{3.233989in}{2.124518in}}%
\pgfpathlineto{\pgfqpoint{3.326011in}{2.120116in}}%
\pgfpathlineto{\pgfqpoint{3.418033in}{2.115503in}}%
\pgfpathlineto{\pgfqpoint{3.510056in}{2.110669in}}%
\pgfpathlineto{\pgfqpoint{3.602078in}{2.105601in}}%
\pgfpathlineto{\pgfqpoint{3.694100in}{2.100291in}}%
\pgfpathlineto{\pgfqpoint{3.786122in}{2.094725in}}%
\pgfpathlineto{\pgfqpoint{3.878145in}{2.088892in}}%
\pgfpathlineto{\pgfqpoint{3.970167in}{2.082778in}}%
\pgfpathlineto{\pgfqpoint{4.062189in}{2.076371in}}%
\pgfpathlineto{\pgfqpoint{4.154212in}{2.069655in}}%
\pgfpathlineto{\pgfqpoint{4.246234in}{2.062616in}}%
\pgfpathlineto{\pgfqpoint{4.338256in}{2.055238in}}%
\pgfpathlineto{\pgfqpoint{4.430278in}{2.047506in}}%
\pgfpathlineto{\pgfqpoint{4.522301in}{2.039402in}}%
\pgfpathlineto{\pgfqpoint{4.614323in}{2.030908in}}%
\pgfpathlineto{\pgfqpoint{4.706345in}{2.022005in}}%
\pgfpathlineto{\pgfqpoint{4.798367in}{2.012674in}}%
\pgfpathlineto{\pgfqpoint{4.890390in}{2.002895in}}%
\pgfpathlineto{\pgfqpoint{4.982412in}{1.992644in}}%
\pgfpathlineto{\pgfqpoint{5.074434in}{1.981901in}}%
\pgfpathlineto{\pgfqpoint{5.166456in}{1.970641in}}%
\pgfpathlineto{\pgfqpoint{5.258479in}{1.958839in}}%
\pgfpathlineto{\pgfqpoint{5.350501in}{1.946469in}}%
\pgfpathlineto{\pgfqpoint{5.442523in}{1.933504in}}%
\pgfpathlineto{\pgfqpoint{5.534545in}{1.919915in}}%
\pgfusepath{stroke}%
\end{pgfscope}%
\begin{pgfscope}%
\pgfpathrectangle{\pgfqpoint{0.800000in}{0.528000in}}{\pgfqpoint{4.960000in}{3.696000in}} %
\pgfusepath{clip}%
\pgfsetrectcap%
\pgfsetroundjoin%
\pgfsetlinewidth{1.505625pt}%
\definecolor{currentstroke}{rgb}{0.000000,0.000000,0.000000}%
\pgfsetstrokecolor{currentstroke}%
\pgfsetdash{}{0pt}%
\pgfpathmoveto{\pgfqpoint{1.025455in}{3.037911in}}%
\pgfpathlineto{\pgfqpoint{1.117477in}{3.025727in}}%
\pgfpathlineto{\pgfqpoint{1.209499in}{3.013576in}}%
\pgfpathlineto{\pgfqpoint{1.301521in}{3.001369in}}%
\pgfpathlineto{\pgfqpoint{1.393544in}{2.989026in}}%
\pgfpathlineto{\pgfqpoint{1.485566in}{2.976478in}}%
\pgfpathlineto{\pgfqpoint{1.577588in}{2.963660in}}%
\pgfpathlineto{\pgfqpoint{1.669610in}{2.950511in}}%
\pgfpathlineto{\pgfqpoint{1.761633in}{2.936977in}}%
\pgfpathlineto{\pgfqpoint{1.853655in}{2.923005in}}%
\pgfpathlineto{\pgfqpoint{1.945677in}{2.908544in}}%
\pgfpathlineto{\pgfqpoint{2.037699in}{2.893547in}}%
\pgfpathlineto{\pgfqpoint{2.129722in}{2.877965in}}%
\pgfpathlineto{\pgfqpoint{2.221744in}{2.861753in}}%
\pgfpathlineto{\pgfqpoint{2.313766in}{2.844863in}}%
\pgfpathlineto{\pgfqpoint{2.405788in}{2.827250in}}%
\pgfpathlineto{\pgfqpoint{2.497811in}{2.808867in}}%
\pgfpathlineto{\pgfqpoint{2.589833in}{2.789666in}}%
\pgfpathlineto{\pgfqpoint{2.681855in}{2.769599in}}%
\pgfpathlineto{\pgfqpoint{2.773878in}{2.748617in}}%
\pgfpathlineto{\pgfqpoint{2.865900in}{2.726669in}}%
\pgfpathlineto{\pgfqpoint{2.957922in}{2.703703in}}%
\pgfpathlineto{\pgfqpoint{3.049944in}{2.679665in}}%
\pgfpathlineto{\pgfqpoint{3.141967in}{2.654499in}}%
\pgfpathlineto{\pgfqpoint{3.233989in}{2.628147in}}%
\pgfpathlineto{\pgfqpoint{3.326011in}{2.600548in}}%
\pgfpathlineto{\pgfqpoint{3.418033in}{2.571640in}}%
\pgfpathlineto{\pgfqpoint{3.510056in}{2.541357in}}%
\pgfpathlineto{\pgfqpoint{3.602078in}{2.509631in}}%
\pgfpathlineto{\pgfqpoint{3.694100in}{2.476391in}}%
\pgfpathlineto{\pgfqpoint{3.786122in}{2.441562in}}%
\pgfpathlineto{\pgfqpoint{3.878145in}{2.405067in}}%
\pgfpathlineto{\pgfqpoint{3.970167in}{2.366824in}}%
\pgfpathlineto{\pgfqpoint{4.062189in}{2.326748in}}%
\pgfpathlineto{\pgfqpoint{4.154212in}{2.284749in}}%
\pgfpathlineto{\pgfqpoint{4.246234in}{2.240735in}}%
\pgfpathlineto{\pgfqpoint{4.338256in}{2.194608in}}%
\pgfpathlineto{\pgfqpoint{4.430278in}{2.146266in}}%
\pgfpathlineto{\pgfqpoint{4.522301in}{2.095601in}}%
\pgfpathlineto{\pgfqpoint{4.614323in}{2.042501in}}%
\pgfpathlineto{\pgfqpoint{4.706345in}{1.986849in}}%
\pgfpathlineto{\pgfqpoint{4.798367in}{1.928522in}}%
\pgfpathlineto{\pgfqpoint{4.890390in}{1.867391in}}%
\pgfpathlineto{\pgfqpoint{4.982412in}{1.803320in}}%
\pgfpathlineto{\pgfqpoint{5.074434in}{1.736167in}}%
\pgfpathlineto{\pgfqpoint{5.166456in}{1.665785in}}%
\pgfpathlineto{\pgfqpoint{5.258479in}{1.592018in}}%
\pgfpathlineto{\pgfqpoint{5.350501in}{1.514703in}}%
\pgfpathlineto{\pgfqpoint{5.442523in}{1.433669in}}%
\pgfpathlineto{\pgfqpoint{5.534545in}{1.348736in}}%
\pgfusepath{stroke}%
\end{pgfscope}%
\begin{pgfscope}%
\pgfpathrectangle{\pgfqpoint{0.800000in}{0.528000in}}{\pgfqpoint{4.960000in}{3.696000in}} %
\pgfusepath{clip}%
\pgfsetrectcap%
\pgfsetroundjoin%
\pgfsetlinewidth{1.505625pt}%
\definecolor{currentstroke}{rgb}{0.000000,0.000000,0.000000}%
\pgfsetstrokecolor{currentstroke}%
\pgfsetdash{}{0pt}%
\pgfpathmoveto{\pgfqpoint{1.025455in}{4.056000in}}%
\pgfpathlineto{\pgfqpoint{1.117477in}{4.024914in}}%
\pgfpathlineto{\pgfqpoint{1.209499in}{3.995036in}}%
\pgfpathlineto{\pgfqpoint{1.301521in}{3.965985in}}%
\pgfpathlineto{\pgfqpoint{1.393544in}{3.937447in}}%
\pgfpathlineto{\pgfqpoint{1.485566in}{3.909152in}}%
\pgfpathlineto{\pgfqpoint{1.577588in}{3.880869in}}%
\pgfpathlineto{\pgfqpoint{1.669610in}{3.852395in}}%
\pgfpathlineto{\pgfqpoint{1.761633in}{3.823550in}}%
\pgfpathlineto{\pgfqpoint{1.853655in}{3.794174in}}%
\pgfpathlineto{\pgfqpoint{1.945677in}{3.764119in}}%
\pgfpathlineto{\pgfqpoint{2.037699in}{3.733249in}}%
\pgfpathlineto{\pgfqpoint{2.129722in}{3.701437in}}%
\pgfpathlineto{\pgfqpoint{2.221744in}{3.668563in}}%
\pgfpathlineto{\pgfqpoint{2.313766in}{3.634512in}}%
\pgfpathlineto{\pgfqpoint{2.405788in}{3.599170in}}%
\pgfpathlineto{\pgfqpoint{2.497811in}{3.562430in}}%
\pgfpathlineto{\pgfqpoint{2.589833in}{3.524183in}}%
\pgfpathlineto{\pgfqpoint{2.681855in}{3.484321in}}%
\pgfpathlineto{\pgfqpoint{2.773878in}{3.442736in}}%
\pgfpathlineto{\pgfqpoint{2.865900in}{3.399319in}}%
\pgfpathlineto{\pgfqpoint{2.957922in}{3.353960in}}%
\pgfpathlineto{\pgfqpoint{3.049944in}{3.306546in}}%
\pgfpathlineto{\pgfqpoint{3.141967in}{3.256960in}}%
\pgfpathlineto{\pgfqpoint{3.233989in}{3.205084in}}%
\pgfpathlineto{\pgfqpoint{3.326011in}{3.150794in}}%
\pgfpathlineto{\pgfqpoint{3.418033in}{3.093965in}}%
\pgfpathlineto{\pgfqpoint{3.510056in}{3.034463in}}%
\pgfpathlineto{\pgfqpoint{3.602078in}{2.972153in}}%
\pgfpathlineto{\pgfqpoint{3.694100in}{2.906892in}}%
\pgfpathlineto{\pgfqpoint{3.786122in}{2.838531in}}%
\pgfpathlineto{\pgfqpoint{3.878145in}{2.766917in}}%
\pgfpathlineto{\pgfqpoint{3.970167in}{2.691887in}}%
\pgfpathlineto{\pgfqpoint{4.062189in}{2.613275in}}%
\pgfpathlineto{\pgfqpoint{4.154212in}{2.530903in}}%
\pgfpathlineto{\pgfqpoint{4.246234in}{2.444588in}}%
\pgfpathlineto{\pgfqpoint{4.338256in}{2.354138in}}%
\pgfpathlineto{\pgfqpoint{4.430278in}{2.259351in}}%
\pgfpathlineto{\pgfqpoint{4.522301in}{2.160017in}}%
\pgfpathlineto{\pgfqpoint{4.614323in}{2.055914in}}%
\pgfpathlineto{\pgfqpoint{4.706345in}{1.946813in}}%
\pgfpathlineto{\pgfqpoint{4.798367in}{1.832472in}}%
\pgfpathlineto{\pgfqpoint{4.890390in}{1.712636in}}%
\pgfpathlineto{\pgfqpoint{4.982412in}{1.587041in}}%
\pgfpathlineto{\pgfqpoint{5.074434in}{1.455410in}}%
\pgfpathlineto{\pgfqpoint{5.166456in}{1.317449in}}%
\pgfpathlineto{\pgfqpoint{5.258479in}{1.172856in}}%
\pgfpathlineto{\pgfqpoint{5.350501in}{1.021309in}}%
\pgfpathlineto{\pgfqpoint{5.442523in}{0.862474in}}%
\pgfpathlineto{\pgfqpoint{5.534545in}{0.696000in}}%
\pgfusepath{stroke}%
\end{pgfscope}%
\begin{pgfscope}%
\pgfpathrectangle{\pgfqpoint{0.800000in}{0.528000in}}{\pgfqpoint{4.960000in}{3.696000in}} %
\pgfusepath{clip}%
\pgfsetbuttcap%
\pgfsetroundjoin%
\pgfsetlinewidth{1.003750pt}%
\definecolor{currentstroke}{rgb}{0.000000,0.000000,0.000000}%
\pgfsetstrokecolor{currentstroke}%
\pgfsetdash{{5.600000pt}{2.400000pt}}{0.000000pt}%
\pgfpathmoveto{\pgfqpoint{4.637372in}{0.528000in}}%
\pgfpathlineto{\pgfqpoint{4.637372in}{4.224000in}}%
\pgfusepath{stroke}%
\end{pgfscope}%
\begin{pgfscope}%
\pgfsetrectcap%
\pgfsetmiterjoin%
\pgfsetlinewidth{0.803000pt}%
\definecolor{currentstroke}{rgb}{0.000000,0.000000,0.000000}%
\pgfsetstrokecolor{currentstroke}%
\pgfsetdash{}{0pt}%
\pgfpathmoveto{\pgfqpoint{0.800000in}{0.528000in}}%
\pgfpathlineto{\pgfqpoint{0.800000in}{4.224000in}}%
\pgfusepath{stroke}%
\end{pgfscope}%
\begin{pgfscope}%
\pgfsetrectcap%
\pgfsetmiterjoin%
\pgfsetlinewidth{0.803000pt}%
\definecolor{currentstroke}{rgb}{0.000000,0.000000,0.000000}%
\pgfsetstrokecolor{currentstroke}%
\pgfsetdash{}{0pt}%
\pgfpathmoveto{\pgfqpoint{5.760000in}{0.528000in}}%
\pgfpathlineto{\pgfqpoint{5.760000in}{4.224000in}}%
\pgfusepath{stroke}%
\end{pgfscope}%
\begin{pgfscope}%
\pgfsetrectcap%
\pgfsetmiterjoin%
\pgfsetlinewidth{0.803000pt}%
\definecolor{currentstroke}{rgb}{0.000000,0.000000,0.000000}%
\pgfsetstrokecolor{currentstroke}%
\pgfsetdash{}{0pt}%
\pgfpathmoveto{\pgfqpoint{0.800000in}{0.528000in}}%
\pgfpathlineto{\pgfqpoint{5.760000in}{0.528000in}}%
\pgfusepath{stroke}%
\end{pgfscope}%
\begin{pgfscope}%
\pgfsetrectcap%
\pgfsetmiterjoin%
\pgfsetlinewidth{0.803000pt}%
\definecolor{currentstroke}{rgb}{0.000000,0.000000,0.000000}%
\pgfsetstrokecolor{currentstroke}%
\pgfsetdash{}{0pt}%
\pgfpathmoveto{\pgfqpoint{0.800000in}{4.224000in}}%
\pgfpathlineto{\pgfqpoint{5.760000in}{4.224000in}}%
\pgfusepath{stroke}%
\end{pgfscope}%
\begin{pgfscope}%
\pgftext[x=2.101616in,y=2.239351in,left,base,rotate=359.000000]{\sffamily\fontsize{10.000000}{12.000000}\selectfont \(\displaystyle e_0 = 0.02\)}%
\end{pgfscope}%
\begin{pgfscope}%
\pgftext[x=2.388887in,y=2.920476in,left,base,rotate=347.000000]{\sffamily\fontsize{10.000000}{12.000000}\selectfont \(\displaystyle e_0 = 0.05\)}%
\end{pgfscope}%
\begin{pgfscope}%
\pgftext[x=2.793304in,y=3.539382in,left,base,rotate=330.000000]{\sffamily\fontsize{10.000000}{12.000000}\selectfont \(\displaystyle e_0 = 0.07\)}%
\end{pgfscope}%
\end{pgfpicture}%
\makeatother%
\endgroup%

  }
  \caption{Difference of $\frac{da}{dt}$ between tangential ($f_T$) and perpendicular ($f_2$) thrust.}
  \label{fig:burtdiff}
\end{figure}

Kechichian directly assumes only tangential and out-of-plane thrust acceleration to reformulate Edelbaum's theory, therefore setting the normal component to zero. However, in his original analysis Edelbaum already noticed that the tangential program is not necessarily optimal\footnote{It would have been extremely interesting to study in detail how this result relates to the cases studied by Burt and King-Hele. However, we have been unable to obtain Lawden's paper in any physical or electronic form, so we have been deprived of his valuable derivations.}:

\begin{displayquote}
It is demonstrated in (17) [Lawden 1958, AN] that the optimum thrust direction for escape with low thrust is halfway between the tangent to the orbit and the normal to the radius vector. This result is also largely applicable to transfer between circular orbits. For the purposes of this report, the difference between this program and tangential thrust can be considered negligible.
\end{displayquote}

\section{Non planar maneuvers} \label{metnonplanar}

\clearpage